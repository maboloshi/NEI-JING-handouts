% -*- coding: utf-8 -*-
%!TEX program = xelatex
\ifx \allfiles \undefined
\documentclass[draft,12pt]{ctexbook}
%\usepackage{xeCJK}
%\usepackage[14pt]{extsizes} %支持8,9,10,11,12,14,17,20pt

%===================文档页面设置====================
%---------------------印刷版尺寸--------------------
%\usepackage[a4paper,hmargin={2.3cm,1.7cm},vmargin=2.3cm,driver=xetex]{geometry}
%--------------------电子版------------------------
\usepackage[a4paper,margin=2cm,driver=xetex]{geometry}
%\usepackage[paperwidth=9.2cm, paperheight=12.4cm, width=9cm, height=12cm,top=0.2cm,
%            bottom=0.4cm,left=0.2cm,right=0.2cm,foot=0cm, nohead,nofoot,driver=xetex]{geometry}

%===================自定义颜色=====================
\usepackage{xcolor}
  \definecolor{mybackgroundcolor}{cmyk}{0.03,0.03,0.18,0}
  \definecolor{myblue}{rgb}{0,0.2,0.6}

%====================字体设置======================
%--------------------中文字体----------------------
%-----------------------xeCJK下设置中文字体------------------------------%
\setCJKfamilyfont{song}{SimSun}                             %宋体 song
\newcommand{\song}{\CJKfamily{song}}                        % 宋体   (Windows自带simsun.ttf)
\setCJKfamilyfont{xs}{NSimSun}                              %新宋体 xs
\newcommand{\xs}{\CJKfamily{xs}}
\setCJKfamilyfont{fs}{FangSong_GB2312}                      %仿宋2312 fs
\newcommand{\fs}{\CJKfamily{fs}}                            %仿宋体 (Windows自带simfs.ttf)
\setCJKfamilyfont{kai}{KaiTi_GB2312}                        %楷体2312  kai
\newcommand{\kai}{\CJKfamily{kai}}
\setCJKfamilyfont{yh}{Microsoft YaHei}                    %微软雅黑 yh
\newcommand{\yh}{\CJKfamily{yh}}
\setCJKfamilyfont{hei}{SimHei}                                    %黑体  hei
\newcommand{\hei}{\CJKfamily{hei}}                          % 黑体   (Windows自带simhei.ttf)
\setCJKfamilyfont{msunicode}{Arial Unicode MS}            %Arial Unicode MS: msunicode
\newcommand{\msunicode}{\CJKfamily{msunicode}}
\setCJKfamilyfont{li}{LiSu}                                            %隶书  li
\newcommand{\li}{\CJKfamily{li}}
\setCJKfamilyfont{yy}{YouYuan}                             %幼圆  yy
\newcommand{\yy}{\CJKfamily{yy}}
\setCJKfamilyfont{xm}{MingLiU}                                        %细明体  xm
\newcommand{\xm}{\CJKfamily{xm}}
\setCJKfamilyfont{xxm}{PMingLiU}                             %新细明体  xxm
\newcommand{\xxm}{\CJKfamily{xxm}}

\setCJKfamilyfont{hwsong}{STSong}                            %华文宋体  hwsong
\newcommand{\hwsong}{\CJKfamily{hwsong}}
\setCJKfamilyfont{hwzs}{STZhongsong}                        %华文中宋  hwzs
\newcommand{\hwzs}{\CJKfamily{hwzs}}
\setCJKfamilyfont{hwfs}{STFangsong}                            %华文仿宋  hwfs
\newcommand{\hwfs}{\CJKfamily{hwfs}}
\setCJKfamilyfont{hwxh}{STXihei}                                %华文细黑  hwxh
\newcommand{\hwxh}{\CJKfamily{hwxh}}
\setCJKfamilyfont{hwl}{STLiti}                                        %华文隶书  hwl
\newcommand{\hwl}{\CJKfamily{hwl}}
\setCJKfamilyfont{hwxw}{STXinwei}                                %华文新魏  hwxw
\newcommand{\hwxw}{\CJKfamily{hwxw}}
\setCJKfamilyfont{hwk}{STKaiti}                                    %华文楷体  hwk
\newcommand{\hwk}{\CJKfamily{hwk}}
\setCJKfamilyfont{hwxk}{STXingkai}                            %华文行楷  hwxk
\newcommand{\hwxk}{\CJKfamily{hwxk}}
\setCJKfamilyfont{hwcy}{STCaiyun}                                 %华文彩云 hwcy
\newcommand{\hwcy}{\CJKfamily{hwcy}}
\setCJKfamilyfont{hwhp}{STHupo}                                 %华文琥珀   hwhp
\newcommand{\hwhp}{\CJKfamily{hwhp}}

\setCJKfamilyfont{fzsong}{Simsun (Founder Extended)}     %方正宋体超大字符集   fzsong
\newcommand{\fzsong}{\CJKfamily{fzsong}}
\setCJKfamilyfont{fzyao}{FZYaoTi}                                    %方正姚体  fzy
\newcommand{\fzyao}{\CJKfamily{fzyao}}
\setCJKfamilyfont{fzshu}{FZShuTi}                                    %方正舒体 fzshu
\newcommand{\fzshu}{\CJKfamily{fzshu}}

\setCJKfamilyfont{asong}{Adobe Song Std}                        %Adobe 宋体  asong
\newcommand{\asong}{\CJKfamily{asong}}
\setCJKfamilyfont{ahei}{Adobe Heiti Std}                            %Adobe 黑体  ahei
\newcommand{\ahei}{\CJKfamily{ahei}}
\setCJKfamilyfont{akai}{Adobe Kaiti Std}                            %Adobe 楷体  akai
\newcommand{\akai}{\CJKfamily{akai}}

%------------------------------设置字体大小------------------------%
\newcommand{\chuhao}{\fontsize{42pt}{\baselineskip}\selectfont}     %初号
\newcommand{\xiaochuhao}{\fontsize{36pt}{\baselineskip}\selectfont} %小初号
\newcommand{\yihao}{\fontsize{28pt}{\baselineskip}\selectfont}      %一号
\newcommand{\xiaoyihao}{\fontsize{24pt}{\baselineskip}\selectfont}
\newcommand{\erhao}{\fontsize{21pt}{\baselineskip}\selectfont}      %二号
\newcommand{\xiaoerhao}{\fontsize{18pt}{\baselineskip}\selectfont}  %小二号
\newcommand{\sanhao}{\fontsize{15.75pt}{\baselineskip}\selectfont}  %三号
\newcommand{\sihao}{\fontsize{14pt}{\baselineskip}\selectfont}%     四号
\newcommand{\xiaosihao}{\fontsize{12pt}{\baselineskip}\selectfont}  %小四号
\newcommand{\wuhao}{\fontsize{10.5pt}{\baselineskip}\selectfont}    %五号
\newcommand{\xiaowuhao}{\fontsize{9pt}{\baselineskip}\selectfont}   %小五号
\newcommand{\liuhao}{\fontsize{7.875pt}{\baselineskip}\selectfont}  %六号
\newcommand{\qihao}{\fontsize{5.25pt}{\baselineskip}\selectfont}    %七号   %中文字体及字号设置
\xeCJKDeclareSubCJKBlock{SIP}{
  "20000 -> "2A6DF,   % CJK Unified Ideographs Extension B
  "2A700 -> "2B73F,   % CJK Unified Ideographs Extension C
  "2B740 -> "2B81F    % CJK Unified Ideographs Extension D
}
%\setCJKmainfont[SIP={[AutoFakeBold=1.8,Color=red]Sun-ExtB},BoldFont=黑体]{宋体}    % 衬线字体 缺省中文字体

\setCJKmainfont{simsun.ttc}[
  Path=fonts/,
  SIP={[Path=fonts/,AutoFakeBold=1.8,Color=red]simsunb.ttf},
  BoldFont=simhei.ttf
]

%SimSun-ExtB
%Sun-ExtB
%AutoFakeBold:自动伪粗,即正文使用\bfseries时生僻字使用伪粗体;
%FakeBold:强制伪粗,即正文中生僻字均使用伪粗体
%\setCJKmainfont[BoldFont=STHeiti,ItalicFont=STKaiti]{STSong}
%\setCJKsansfont{微软雅黑}黑体
%\setCJKsansfont[BoldFont=STHeiti]{STXihei} %serif是有衬线字体sans serif 无衬线字体
%\setCJKmonofont{STFangsong}    %中文等宽字体

%--------------------英文字体----------------------
\setmainfont{simsun.ttc}[
  Path=fonts/,
  BoldFont=simhei.ttf
]
%\setmainfont[BoldFont=黑体]{宋体}  %缺省英文字体
%\setsansfont
%\setmonofont

%===================目录分栏设置====================
\usepackage[toc,lof,lot]{multitoc}    % 目录(含目录、表格目录、插图目录)分栏设置
  %\renewcommand*{\multicolumntoc}{3} % toc分栏数设置,默认为两栏(\multicolumnlof,\multicolumnlot)
  %\setlength{\columnsep}{1.5cm}      % 调整分栏间距
  \setlength{\columnseprule}{0.2pt}   % 调整分栏竖线的宽度

%==================章节格式设置====================
\setcounter{secnumdepth}{3} % 章节等编号深度 3:子子节\subsubsection
\setcounter{tocdepth}{2}    % 目录显示等度 2:子节

\xeCJKsetup{%
  CJKecglue=\hspace{0.15em},      % 调整中英(含数字)间的字间距
  %CJKmath=true,                  % 在数学环境中直接输出汉字(不需要\text{})
  AllowBreakBetweenPuncts=true,   % 允许标点中间断行,减少文字行溢出
}

\ctexset{%
  part={
    name={,篇},
    number=\SZX{part},
    format={\chuhao\bfseries\centering},
    nameformat={},titleformat={}
  },
  section={
    number={\chinese{section}},
    name={第,节}
  },
  subsection={
    number={\chinese{subsection}、},
    aftername={\hspace{-0.01em}}
  },
  subsubsection={
    number={(\chinese{subsubsection})},
    aftername={\hspace {-0.01em}},
    beforeskip={1.3ex minus .8ex},
    afterskip={1ex minus .6ex},
    indent={\parindent}
  },
  paragraph={
    beforeskip=.1\baselineskip,
    indent={\parindent}
  }
}

\newcommand*\SZX[1]{%
  \ifcase\value{#1}%
    \or 上%
    \or 中%
    \or 下%
  \fi
}

%====================页眉设置======================
\usepackage{titleps}%或者\usepackage{titlesec},titlesec包含titleps
\newpagestyle{special}[\small\sffamily]{
  %\setheadrule{.1pt}
  \headrule
  \sethead[\usepage][][\chaptertitle]
  {\chaptertitle}{}{\usepage}
}

\newpagestyle{main}[\small\sffamily]{
  \headrule
  %\sethead[\usepage][][第\thechapter 章\quad\chaptertitle]
%  {\thesection\quad\sectiontitle}{}{\usepage}}
  \sethead[\usepage][][第\chinese{chapter}章\quad\chaptertitle]
  {第\chinese{section}节\quad\sectiontitle}{}{\usepage}
}

\newpagestyle{main2}[\small\sffamily]{
  \headrule
  \sethead[\usepage][][第\chinese{chapter}章\quad\chaptertitle]
  {第\chinese{section}節\quad\sectiontitle}{}{\usepage}
}

%================ PDF 书签设置=====================
\usepackage{bookmark}[
  depth=2,        % 书签深度 2:子节
  open,           % 默认展开书签
  openlevel=2,    % 展开书签深度 2:子节
  numbered,       % 显示编号
  atend,
]
  % 相比hyperref,bookmark宏包大多数时候只需要编译一次,
  % 而且书签的颜色和字体也可以定制。
  % 比hyperref 更专业 (自动加载hyperref)

%\bookmarksetup{italic,bold,color=blue} % 书签字体斜体/粗体/颜色设置

%------------重置每篇章计数器,必须在hyperref/bookmark之后------------
\makeatletter
  \@addtoreset{chapter}{part}
\makeatother

%------------hyperref 超链接设置------------------------
\hypersetup{%
  pdfencoding=auto,   % 解决新版ctex,引起hyperref UTF-16预警
  colorlinks=true,    % 注释掉此项则交叉引用为彩色边框true/false
  pdfborder=001,      % 注释掉此项则交叉引用为彩色边框
  citecolor=teal,
  linkcolor=myblue,
  urlcolor=black,
  %psdextra,          % 配合使用bookmark宏包,可以直接在pdf 书签中显示数学公式
}

%------------PDF 属性设置------------------------------
\hypersetup{%
  pdfkeywords={黄帝内经,内经,内经讲义,21世纪课程教材},    % 关键词
  %pdfsubject={latex},        % 主题
  pdfauthor={主编:王洪图},   % 作者
  pdftitle={内经讲义},        % 标题
  %pdfcreator={texlive2011}   % pdf创建器
}

%------------PDF 加密----------------------------------
%仅适用于xelatex引擎 基于xdvipdfmx
%\special{pdf:encrypt ownerpw (abc) userpw (xyz) length 128 perm 2052}

%仅适用于pdflatex引擎
%\usepackage[owner=Donald,user=Knuth,print=false]{pdfcrypt}

%其他可使用第三方工具 如:pdftk
%pdftk inputfile.pdf output outputfile.pdf encrypt_128bit owner_pw yourownerpw user_pw youruserpw

%=============自定义环境、列表及列表设置================
% 标题
\def\biaoti#1{\vspace{1.7ex plus 3ex minus .2ex}{\bfseries #1}}%\noindent\hei
% 小标题
\def\xiaobt#1{{\bfseries #1}}
% 小结
\def\xiaojie {\vspace{1.8ex plus .3ex minus .3ex}\centerline{\large\bfseries 小\ \ 结}\vspace{.1\baselineskip}}
% 作者
\def\zuozhe#1{\rightline{\bfseries #1}}

\newcounter{yuanwen}    % 新计数器 yuanwen
\newcounter{jiaozhu}    % 新计数器 jiaozhu

\newenvironment{yuanwen}[2][【原文】]{%
  %\biaoti{#1}\par
  \stepcounter{yuanwen}   % 计数器 yuanwen+1
  \bfseries #2}
  {}

\usepackage{enumitem}
\newenvironment{jiaozhu}[1][【校注】]{%
  %\biaoti{#1}\par
  \stepcounter{jiaozhu}   % 计数器 jiaozhu+1
  \begin{enumerate}[%
    label=\mylabel{\arabic*}{\circledctr*},before=\small,fullwidth,%
    itemindent=\parindent,listparindent=\parindent,%labelsep=-1pt,%labelwidth=0em,
    itemsep=0pt,topsep=0pt,partopsep=0pt,parsep=0pt
  ]}
  {\end{enumerate}}

%===================注解与原文相互跳转====================
%----------------第1部分 设置相互跳转锚点-----------------
\makeatletter
  \protected\def\mylabel#1#2{% 注解-->原文
    \hyperlink{back:\theyuanwen:#1}{\Hy@raisedlink{\hypertarget{\thejiaozhu:#1}{}}#2}}

  \protected\def\myref#1#2{% 原文-->注解
    \hyperlink{\theyuanwen:#1}{\Hy@raisedlink{\hypertarget{back:\theyuanwen:#1}{}}#2}}
  %此处\theyuanwen:#1实际指thejiaozhu:#1,只是\thejiaozhu计数器还没更新,故使用\theyuanwen计数器代替
\makeatother

\protected\def\myjzref#1{% 脚注中的引用(引用到原文)
  \hyperlink{\theyuanwen:#1}{\circlednum{#1}}}

\def\sb#1{\myref{#1}{\textsuperscript{\circlednum{#1}}}}    % 带圈数字上标

%----------------第2部分 调整锚点垂直距离-----------------
\def\HyperRaiseLinkDefault{.8\baselineskip} %调整锚点垂直距离
%\let\oldhypertarget\hypertarget
%\makeatletter
%  \def\hypertarget#1#2{\Hy@raisedlink{\oldhypertarget{#1}{#2}}}
%\makeatother

%====================带圈数字列表标头====================
\newfontfamily\circledfont[Path = fonts/]{meiryo.ttc}  % 日文字体,明瞭体
%\newfontfamily\circledfont{Meiryo}  % 日文字体,明瞭体

\protected\def\circlednum#1{{\makexeCJKinactive\circledfont\textcircled{#1}}}

\newcommand*\circledctr[1]{%
  \expandafter\circlednum\expandafter{\number\value{#1}}}
\AddEnumerateCounter*\circledctr\circlednum{1}

% 参考自:http://bbs.ctex.org/forum.php?mod=redirect&goto=findpost&ptid=78709&pid=460496&fromuid=40353

%======================插图/tikz图========================
\usepackage{graphicx,subcaption,wrapfig}    % 图,subcaption含子图功能代替subfig,图文混排
  \graphicspath{{img/}}                     % 设置图片文件路径

\def\pgfsysdriver{pgfsys-xetex.def}         % 设置tikz的驱动引擎
\usepackage{tikz}
  \usetikzlibrary{calc,decorations.text,arrows,positioning}

%---------设置tikz图片默认格式(字号、行间距、单元格高度)-------
\let\oldtikzpicture\tikzpicture
\renewcommand{\tikzpicture}{%
  \small
  \renewcommand{\baselinestretch}{0.2}
  \linespread{0.2}
  \oldtikzpicture
}

%=========================表格相关===============================
\usepackage{%
  multirow,                   % 单元格纵向合并
  array,makecell,longtable,   % 表格功能加强,tabu的依赖
  tabu-last-fix,              % "强大的表格工具" 本地修复版
  diagbox,                    % 表头斜线
  threeparttable,             % 表格内脚注(需打补丁支持tabu,longtabu)
}

%----------给threeparttable打补丁用于tabu,longtabu--------------
%解决方案来自:http://bbs.ctex.org/forum.php?mod=redirect&goto=findpost&ptid=80318&pid=467217&fromuid=40353
\usepackage{xpatch}

\makeatletter
  \chardef\TPT@@@asteriskcatcode=\catcode`*
  \catcode`*=11
  \xpatchcmd{\threeparttable}
    {\TPT@hookin{tabular}}
    {\TPT@hookin{tabular}\TPT@hookin{tabu}}
    {}{}
  \catcode`*=\TPT@@@asteriskcatcode
\makeatother

%------------设置表格默认格式(字号、行间距、单元格高度)------------
\let\oldtabular\tabular
\renewcommand{\tabular}{%
  \renewcommand\baselinestretch{0.9}\small    % 设置行间距和字号
  \renewcommand\arraystretch{1.5}             % 调整单元格高度
  %\renewcommand\multirowsetup{\centering}
  \oldtabular
}
%设置行间距,且必须放在字号设置前 否则无效
%或者使用\fontsize{<size>}{<baseline>}\selectfont 同时设置字号和行间距

\let\oldtabu\tabu
\renewcommand{\tabu}{%
  \renewcommand\baselinestretch{0.9}\small    % 设置行间距和字号
  \renewcommand\arraystretch{1.8}             % 调整单元格高度
  %\renewcommand\multirowsetup{\centering}
  \oldtabu
}

%------------模仿booktabs宏包的三线宽度设置---------------
\def\toprule   {\Xhline{.08em}}
\def\midrule   {\Xhline{.05em}}
\def\bottomrule{\Xhline{.08em}}
%-------------------------------------
%\setlength{\arrayrulewidth}{2pt} 设定表格中所有边框的线宽为同样的值
%\Xhline{} \Xcline{}分别设定表格中水平线的宽度 makecell包提供

%表格中垂直线的宽度可以通过在表格导言区(preamble),利用命令 !{\vrule width1.2pt} 替换 | 即可

%=================图表设置===============================
%---------------图表标号设置-----------------------------
\renewcommand\thefigure{\arabic{section}-\arabic{figure}}
\renewcommand\thetable {\arabic{section}-\arabic{table}}

\usepackage{caption}
  \captionsetup{font=small,}
  \captionsetup[table] {labelfont=bf,textfont=bf,belowskip=3pt,aboveskip=0pt} %仅表格 top
  \captionsetup[figure]{belowskip=0pt,aboveskip=3pt}  %仅图片 below

%\setlength{\abovecaptionskip}{3pt}
%\setlength{\belowcaptionskip}{3pt} %图、表题目上下的间距
\setlength{\intextsep}   {5pt}  %浮动体和正文间的距离
\setlength{\textfloatsep}{5pt}

%====================全文水印==========================
%解决方案来自:
%http://bbs.ctex.org/forum.php?mod=redirect&goto=findpost&ptid=79190&pid=462496&fromuid=40353
%https://zhuanlan.zhihu.com/p/19734756?columnSlug=LaTeX
\usepackage{eso-pic}

%eso-pic中\AtPageCenter有点水平偏右
\renewcommand\AtPageCenter[1]{\parbox[b][\paperheight]{\paperwidth}{\vfill\centering#1\vfill}}

\newcommand{\watermark}[3]{%
  \AddToShipoutPictureBG{%
    \AtPageCenter{%
      \tikz\node[%
        overlay,
        text=red!50,
        %font=\sffamily\bfseries,
        rotate=#1,
        scale=#2
      ]{#3};
    }
  }
}

\newcommand{\watermarkoff}{\ClearShipoutPictureBG}

\watermark{45}{15}{草\ 稿}    %启用全文水印

%=============花括号分支结构图=========================
\usepackage{schemata}

\xpatchcmd{\schema}
  {1.44265ex}{-1ex}
  {}{}

\newcommand\SC[2] {\schema{\schemabox{#1}}{\schemabox{#2}}}
\newcommand\SCh[4]{\Schema{#1}{#2}{\schemabox{#3}}{\schemabox{#4}}}

%=======================================================

\begin{document}
\pagestyle{main}
\fi
\chapter{《黄帝内经》的学术体系} %第二章

《黄帝内经》是中医学的奠基之作,它整理先人们积累的丰富的医疗经验,升华为理性认识,形成系统的医学理论,并且进一步驾驭医疗实践,建立了中医学临床规范,使中医学基本上跳出了经验医学的窠臼,成为中国传统科学中探讨生命规律及其医学应用的系统学问,也就是我们所说的学术体系。它为中医学理论体系的建立打好了结构框架,奠定了中医学发展的基础;及至现代,与西医学相比较,中医学仍具有自己鲜明的特色与优势,是其存在和发展的理由和依据。

对于《内经》学术体系的研究,大致可分为三个时期:自隋至清,主要是分析《内经》理论的基本架构及其学术思想,这是第一个时期。这个时期的研究,仅限于本学科理论的阐释、医学知识的分类,缺乏比较,没有参比物。1922年恽树珏《论医集》提出中医理论体系的概念,继之杨则民在1935年《浙江中医专门学校校刊》“内经之哲学的检讨”中指出,《内经》理论的方法论是辩证法,明确提出了以西医学为参比物研究中医理论的课题。但是时代与条件所限,半个世纪以来,这种研究并没有深入,这是第二个时期。近20多年来,通过现代多学科知识和方法的探讨与论证,明确了以《内经》为基础的中医学理论具有不同于西医学的独特内涵、科学价值和临床意义,基本上改变了中医学术的从属地位,但也提出了众多课题留待人们去探讨,标志着《内经》学术体系的研究进入第三个时期。

\section{《内经》学术体系的结构} %第一节

《内经》学术体系究竟由哪些相互联系、相互制约的内容或部分所组成?其形式如何?过去,中医界普遍认为,古代医家对《内经》的类分研究即体现了学术体系的框架结构。延至近代,有的学者将阴阳五行学说归入医学方法学或学术思想类。据有关概念,一个学科的学术体系,应当包括关于研究对象的理论及形成这些理论的知识基础和方法学基础。毫无疑问,《内经》学术体系主要围绕人的健康、疾病展开研究,形成有关人的生命规律及其医学应用的知识和理论,而这些知识和理论的形成,必有古代自然科学、社会科学等方面的知识和方法的渗透与影响,它既是医学理论形成的基础,又是《内经》学术体系的有机组成部分。为此,我们将《内经》学术体系结构划分为医学理论和医学基础两部分。

\subsection{医学理论} %一、

《内经》学术体系医学理论的具体内容,基本可以由历代医家对《内经》理论的分类来概括。历代医家的分类,从唐代杨上善《黄帝内经太素》19类、明代张介宾《类经》11类、李中梓《内经知要》8类,到清代沈又彭《医经读》4类,虽繁简精粗不同、类别间有出入,但大体可以概括《内经》医学理论的内容,经过繁简修合、纲目条贯的整理,分类如下:

藏象(脏腑;经络;精气神)

疾病(病机;病证)

诊法(诊病方法;断病法则)

论治(治则;治法;疗法)

养生(摄生;康复)

\paragraph{藏象}	是研究脏腑经脉形体官窍的形态结构、生理活动规律及其相互关系的理论。是《内经》学术体系的核心。“象”,指外在的生命现象,既包括有形可见的躯体肢节官窍、脏腑血脉骨肉组织的形象及其动态变化,又包括各种无形的生理、精神现象,“藏”,其本义是藏于内不知其所以,但却是外象变化本质的机制所在,因而它是指藏于体内有形、无形的脏腑、经络、精气神及其永不休止的活动。藏象学说的内容主要包括脏腑、经络、精气神。同时,人的活动与天地自然、社会人事密切相关,因而人们在探索人体生命活动时,又将它们联系在一起进行研究,因而也成为藏象内容的有机组成部分。

\paragraph{疾病}	论异常的生命活动,有两部分内容:一是病因病机。病因即生病的原因,病机是阐述疾病发生、发展变化及转归的机理和规律,其内容包括发病、病理、传变等。其中的病理,主要论述病变的机理,如表里出入、寒热进退、邪正虚实、阴阳盛衰等基本病变机理和脏腑、经络、精气神的具体病变机理。二是论疾病的概念、分类及其临床表现,即病证。《内经》论疾病多以“疾”“病”“候”表述,个别之处提到“证”,与候同义。《内经》论疾病重其整体机能异常和阶段变化性质,后世概括为“证候”,体现了中医学疾病学理论的特色。

\paragraph{诊法}	论疾病的诊断方法,包括疾病的诊断原理、诊察方法与判断法则。《内经》除深刻地阐述了中医的诊病原理,如以表知里,以我知彼,先别阴阳以及太过与不及之理等,主要是创造性提出望闻问切与四诊合参的直观察验的疾病诊察方法。此外,在疾病判断法则方面,《内经》的特点是以“审察病机”为中心的审机论病,并体现在疾病的脏腑分证、经络分证、病因分证等,实为后世“证候诊断”之源。

\paragraph{论治}	《内经》除了阐述天人合一、心身一体以及治未病等治疗思想外,主要阐述治疗原则和治疗方法。治疗原则,是指导治法、疗法的准绳和法则,如治病求本、标本先后、调节阴阳、虚实补泻、因势利导等。治疗方法又有无形的方略技巧与有形的处理措施之别,前者称治法,如解表清里、理气活血、利湿化痰等;后者称疗法,如药物疗法、针灸疗法、饮食疗法、精神疗法等。《内经》记载了多种疗法,并论述了它们的治病原理、使用方法与宜忌等。《内经》论治内容丰富,形成了完整的论治体系。

\paragraph{养生}	即颐养生命,包括两个方面内容:一是无病之摄养,目的是健身缓老与防病。在对疾病和衰老认识基础上,《内经》确立了“治未病”养生思想,并提出外以避邪、内以养正的原则和多种养生方法,建立了中医学独特的养生学说。二是残邪未尽与因病致残的康复。

以上是从学术内涵对《内经》医学理论内容进行的分类,即其医学理论的系统结构,但考虑到各部分内容的多寡与内在分合独立性,结合中医界习俗成见,上述分类又可以变通为:

藏象(脏腑,精气神)

经络

病机(病因,发病,病理,传变)

病证

诊法(诊病方法,断病法则)

论治(治则,治法)

疗法

养生(摄生,康复)

以上分类,未将阴阳五行与运气学说包括于内。阴阳五行,是战国秦汉占主流的哲学,是中国传统科学的主要方法学工具。在《内经》学术体系之中,它虽然含有具体的医学内容,但其主要作用也是论证和规范医学知识与理论的方法学工具,因而本书将其划归医学基础的哲学部分,而未列入医学理论。运气学说,不仅讨论气候变化规律同生物生存、人的生理病理关系,也探索论治、方药原理与原则,还富含藏象、病机、病证、诊法与摄生等内容,属于综合性理论,宜作专题研究,非医学理论固有组成部分。

\subsection{医学基础} %二、

《内经》学术体系的医学基础,主要包括哲学、天文历法、地理学、气象学、数学以及社会学等。

\subsubsection{哲学}%(一)

任何学科都有其知识、理论之认识来源和依据。中医学之所以形成独特的医学概念、理论和别具一格的疾病诊治方法,与民族文化背景及其思维方式、医疗实践的特定条件有密切关系。在战国秦汉,中国的生产力水平较世界其它地区高,与生产力发展水平相同步、代表中华民族思维能力、方式与特点的自然哲学——精气论、阴阳五行论基本形成,但科学技术仍然处于萌芽和幼稚时期。在医学领域,对于动物和人体器官、组织,即使剖开也无法了解其机能,医学研究只能在感觉直观、生命体验和医疗验证中进行,因而医学理论也只能在医疗实践基础上,借助自然哲学的直接参与,总结、概括而产生,这从《内经》推崇阴阳五行是分析生命体生杀变化、指导疾病治疗的根本法则即可得到证明。古代哲学对于中医学形成和发展的影响是全面的、深刻的,它不仅表现在《内经》绝大部分篇章均贯穿精气、阴阳五行学说,深入到中医理论的各部分,而且成为中医方法学基础,形成了中医学术体系的民族特色。具体而言,哲学的作用,一是引导医疗活动的指向,赋予医学观察和医疗实践以特定内容;二是约定医学概念内涵和独特表述方式;三是建立推理体系、理论模式和学术框架。

特别值得提出的是,由于《内经》成书时代人文、自然科学各学科与哲学之间尚未分化,因而中医学借助了某些哲学术语表述医学概念,这在中国传统科学各学科中也是常见的现象,但它们一旦成为医学术语,便赋予特定的医学含义,与纯哲学概念大不相同,目前仍不具备运用中医学自己的表述方式给予置换的条件。

此外,《内经》的哲学除具有医学哲学的特点外,在中国古代哲学史上也占有重要地位。它不仅赋予哲学理念以生命体之“物质基础”,客现存在,生动鲜活,而且在阴阳太少基础上,还创造性地提出厥阴、阳明,从而发展了阴阳哲学。

\subsubsection{天文历法、地理学、气象学}%(二)

我国古代有关自然科学的内容非常丰富,门类众多,涉及天文历法、地理学、气象学、生物学以及数学、物理、化学等。人生天地之间,气交之中,在广泛的时空条件下受着自然力量的制约,因而人类必须了解自然、把握自然,进而顺应自然、利用自然,为自身的生存和发展服务;同时,中医学是一门应用学科,它的建立、发展和成熟,也有赖于其它学科,成为医学理论形成和建立的基础。根据中国古代传统科学的分类,结合中医理论形成过程中多学科知识和方法渗透、影响的情况,自然科学中的天文历法、地理学、气象学以及数学等学科,具有这种基础作用。现仅就数学方面作简要说明,其它学科的情况可参见有关章节。

\subsubsection{数学}%(三)

数学是研究物质世界空间形式和数量关系的科学,在生物学中,由于生物特性常以随机变量出现,具有多方面特征,同时生物系统中还大量存在无法用数值表示的特性,即非实数性,所以只有近百年来数学的高速发展,如概率论和统计数学的出现,集合论、微分方程、对策论和网络理论等系统数学的发展,以及数量分类学中的特性编码技术和分类分析法的建立,才使之在生物学以及医学领域展示其基础科学的地位,在中国古代,《周易》不仅用象征符号及其变化表示世界秩序及事物变化法则,而且认为象中有数;一切事物既有可感知的性质,又有数量的规定性,只有认识信息(象)和度量,(数)两个方面,才能了解和控制事物变化过程。《内经》接受了这一思想、也用数表示天地、万物及人体生理、病理及诊治的度量,不过在其理论表述过程中,符号数学运用较少,主要是:①运用数学语言和思维论述中医理论,如《素问·金匮真言论》以五方——五时——五行——五脏之数,演示方位、时序阴阳生化五行之理,并合于人,说明五脏时空与机能特性,是《内经》据“象为主,数为用”原则论证藏象学说的方法之一。②以数学模型建构中医学理论框架,如用河图数字排列之时空含义,推论脏腑气化过程中的生克制化规律,诸如肝肺升降、心肾相交、脾主中宫以及五行生成数之阴阳互根、气化太过不及等。③以活的动态数学关系量化诊治标准,一是活的相对定量标准,如以呼吸与脉搏比例为参数,定脉象速率指标;同身寸法量度脉、骨、肠胃长度宽狭、定俞穴位置等。二是证候与治疗量化的模糊性,如《素问·玉机真脏论》脉证太过不及和死证死脉的量度,《灵枢·五色》病色浮沉夭泽的判断。

\subsubsection{社会学}%(四)

人类及其个体的生理、精神活动,是宇宙自然发展、演化的最髙产物,它除具有生物的一般特性,即自然属性,还使自己的本质进入社会历史领域,具有社会属性,并表现出复杂的心理活动,因此,古代医学家们在探索生命规律时,除了运用自然科学知识和方法外,还广泛涉及社会人文科学方面的知识和方法,而成为《内经》学术体系重要的医学基础,故《素问·气交变大论》说:“夫道者,上知天文,下知地理,中知人事,可以长久。”《内经》社会人文科学方面的内容也十分丰富,如社会学、教育学、军事学、民俗学、语言学、文学等,但主要体现在社会学中。

社会学是研究社会和社会问题的学科,而社会因素同人类群体或个体健康、疾病发生发展及其防治有着密切关系。《内经》中许多篇章描述了古代社会与医学起源、发展情况。如远古穴居野处,私有观念尚未萌生而称“恬谈之世”,疾病以外感外伤为主,医药学初建,“祝由”盛行;随着社会发展,私有制的建立,经济、政治、文化交往增多,疾病也渐致复杂,单纯的汤液醪醴不能适应临床需要,于是发展成为多种剂型、多种治法。此外,《内经》还有大量社会经济状况、风土习俗、人情心理以及社会地位变迁等及其与疾病关系的记载,并贯穿于病因、发病、诊断、治疗、养生诸学说之中。它将人与社会生存环境的失调作为重要致病因素,丰富了中医病因学理论;它重视在疾病防治过程中纠正社会性致病因素,并以此作为疾病防治的重要原则,完善了中医学疾病防治理论,为中医学社会心理医学模式奠定了理论基础。

《内经》学术体系基本框架结构,可如下表所示:

%{
%\label{fig:《内经》学术体系基本框架结构图}
%\renewcommand{\baselinestretch}{0.9}
%\small
%\hspace{.1\textwidth}
%\SCh{9.9ex}{11.2ex}{《内经》学术体系}{
%        \SCh{0.1ex}{13.7ex}{医学理论}{
%            \SC{藏象}{脏腑\\经络\\精气神}\\
%            \SCh{2.8ex}{2.9ex}{疾病}{
%                \SC{病机}{病因\\发病\\病理\\传变}\\
%                病症}\\
%            \SC{诊法}{诊病方法\\断病法则}\\
%            \SC{论治}{治则\\治法\\疗法}\\
%            \SC{养生}{摄生\\康复}
%        }\\
%        \SC{医学基础}{
%            哲学\\
%            天文历法\\
%            地理学\\
%            气象学\\
%            数学\\
%            社会学}
%}\hfill }

{%
	\label{fig:《内经》学术体系基本框架结构图}
	\renewcommand{\baselinestretch}{1}
	\small
	\hspace{.1\textwidth}%左侧距离
	\SCh{9.5ex}{11.3ex}{《内经》学术体系}{
		\SCh{0.1ex}{13.7ex}{医学理论}{\smallskip
			\SC{藏象}{脏腑\\经络\\精气神}\\\smallskip
			\SCh{2.8ex}{2.9ex}{疾病}{
				\SC{病机}{病因\\发病\\病理\\传变}\\
				病症}\\\smallskip
			\SC{诊法}{\smallskip 诊病方法\\断病法则}\\\smallskip
			\SC{论治}{治则\\治法\\疗法\\测试}\\\smallskip
			\SC{养生}{\smallskip 摄生\\康复}
		}\\
		\SC{医学基础}{
			哲学\\
			天文历法\\
			地理学\\
			气象学\\
			数学\\
			社会学}
	}
	\hfill
}

\section{《内经》学术体系的形成} %第二节

《内经》学术体系形成,既有长期医疗实践的基础,又与古代人文、自然科学知识的渗透,特别是哲学思想的影响分不开。

\subsection{医疗实践的观察与验证} %一、

\subsubsection{解剖学基础}%(一)

在我国古代,早就通过生活观察、战争、刑罚观察尸体和施行解剖的方法了解人体形态结构。根据甲骨文、金文有关字形结构分析,夏、商、周三代对人体躯体官窍、骨骼、内脏已有基本正确认识。《内经》也记载了古代解剖方法,并详细录有脏腑之大小、坚脆、容量,血脉之长短、清浊等。其中消化道与食管长度之比为55.8:1.6≈35:1(《灵枢·肠胃》),同现代解剖学(上海科技出版社1995年《正常人体解剖学》为850:25=34:1)基本相等。同时,《内经》还论述了针刺误中重要脏腑、器官招致医疗事故的症状。这些文献都以无可辩驳的事实,说明《内经》学术体系的形成有坚实的解剖学基础。中国古人对于人体及其生命活动研究的方法,其初始阶段以解剖学观察为主,所以中医学对内脏器官、组织的命名,多基于形态结构;内脏器官、组织的机能及其与外在生命现象的宏观联系,凡显而易见的,均与近现代解剖生理的认识相同,如肺司呼吸、心合血脉等。但人体及其生命活动是非常复杂的,凡属微观领域的种种现象,这种直观解剖方法便无能为力,只好求助于理性思辨,从而使中医学走上“精于气化,略于形质”的独特发展道路,这也是自《内经》之后,中医解剖学发展缓慢的根本原因。

\subsubsection{人体生命现象的观察}%(二)

对人体生命现象的长期观察,包括生理的、病理的,有目的无目的治疗反应等等,即成为医疗经验积累;通过对丰富医疗经验的反复比较、联系起来的思考,就会发现众多生理、病理现象并非杂乱无章,在它们之间存在着自然、有序的联系,确定这种联系,进而推测其内在生理机制,于是形成片断的医学理论,为建立系统理论和学术体系奠定了基础。如天暑衣厚则多汗少尿,天寒衣薄则多尿少汗,联系到体内气血津液受气候寒暑变化的影响;发怒时气满胸中、胁胀、头晕目眩,甚至昏厥、吐血,联系到体内气的运行受情绪的影响。又如,当人受外界气候变化影响而发生身体不适时,恶寒发热的皮毛症状、鼻塞流涕的鼻腔症状、咳嗽胸痛的肺部症状三者常相伴而至,便建立了皮毛、鼻、肺的联系,这就是肺主呼吸、外合皮毛、开窍于鼻的理论原型。基于相关生命现象系统、有序联系的观察,再经过理性思维,进而整合为藏象学说中的各种功能模型,这是中医理论形成的重要方法学模式之一。

\subsubsection{医疗实践的反复验证}%(三)

通过观察、推论获取的医学知识和从医疗经验上升的理论,经过反复的临床实践验证,去粗取精、去伪存真,是中医学理论形成的基本过程。这一过程由于是自发进行而付出过沉重代价,但也同时造就中医学经验、知识与理论的客观真实性。如《素问·玉机真脏论》“浆粥入胃、泄注止,则虚者活;身汗、得后利,则实者活。”验证实证邪有出路,虚证能进食则预后良好的判断,并引出相应的治疗原则。有人指责中医治病经不起重复,这是误解。中医按自己的理法方药治病,千百年重复不忒。不能重复之说,主要是在研究思路、方法和价值观上有差异,疗效评价体系也不相同的缘故。

\subsection{古代哲学思想的影响} %二、

战国秦汉哲学,以精气、阴阳、五行之说为代表,其论说始载于诸子,但对中医学都产生过影响。尤其是《周易》,其价值与贡献主要在于哲学,它阐发对自然、社会普遍规律的认识,除儒家的政治观和伦理观外,还溶进道家和阴阳家的天道观。分析《内经》学术体系的哲学基础,《周易》具有一定的代表性。

\subsubsection{观象明理和思维模式化}%(一)

《周易》的多种思维方式,都离不开象。观象是思维过程的起点。人们运用感官直接感受或体验事物之象,最初直接比照,随着思维能力的发展,人们不满足于具体范例的约束,提出“观象玩辞”“观象蕴意”,引出道理和原则,并发展到“观象明理”,这个理就是事物的功能、作用和运动形式,并从中引出功能性原则,成为传统科学的特点。

1.《内经》观象明理,创建藏象学说

藏象学说是《内经》理论的核心内容,它是怎样形成的?人们看到生命活动的外在之象及相关的自然之象,但其内在的变化本质是什么,恰似黑箱中物,不得而知,浑之曰“藏”。用什么方法搞清楚“藏”的内容?首先结合以往的医疗经验、医学知识,参比哲学分类思想,对象进行医学的类属性整理、归纳,每类象具有共性,不同类的象相互之间存在有机联系,犹如《周易》爻与爻、爻与卦、卦与卦的离合、相互关系(涉及辩证思维内容),从而形成关于人体生理活动机制与规律,也就是外在“象”与内在“藏”有机联系的理论。经过医疗实践的反复验证、修正、完善,遂成定论。显然,它已不是生命体原型的描纂,而是生理活动方式的概括。其中的“藏”字,也不宜用解剖的“脏”代替,以免误解。

2.关于思维模式化

思维模型是人们按某种特定目的,对认识对象所做的简化描述,是对原型进行模拟所形成的特定样态。《周易》逻辑思维并不发达,但思维模式化倾向很明显,是后世创制多种思维模型的源头,如阴阳、三才、四象、河图、洛书、八卦等模型。《内经》在整理医疗经验、医学知识,使之上升为理论的过程中,受《周易》思维模式化的影响,也建立多种理论模型。在藏象方面,有阴阳模型论脏腑、气血、营卫,三阴三阳模型论六经,五行模型论五脏等。在病机方面,有表里出入、寒热进退、邪正虚实、阴阳盛衰模型等。模式思维是中医进行理论思维、临床方法规范化的重要过程和环节。

\subsubsection{辩证思维}%(二)

辩证思维是《周易》最为突出、最为系统、最为丰富、最为珍贵的一种思维方式,它的形成受道家、阴阳家的影响,再由儒家加以政治化、伦理化,归纳、发挥而成,它对《内经》理论形成,主要有以下三个方面:

1.整体思维

整体思维以普遍联系、相互制约的观念看待世界及一切事物,自然万物是一个连续的、不可割裂的有机整体;部分作为整体的构成要素,其本身也是一个连续、不可割裂的整体,因而认为万物同源、同构、同律。如经卦六画同时具有下中上、初中末、天地人之义,反映《周易》天人时空整体观。这就使医学家们在面对大量的临床经验和知识中人体生理、病理与天时气候、地土方宜、社会人事联系资料,用整体思维原理进行理论阐释和概括,建构《内经》三才合一的整体医学模式,如《素问·阴阳应象大论》说:“其在天为玄,在人为道,在地为化。化生五味,道生智,玄生神。”并以三才为经,五行为纬,详为论述天、地、人诸事物的类属及其相互关系。此外,《周易》还启发医学家们运用整体思维分析躯体与生理、心理相关现象。如每卦六爻彼此联系,爻性、爻位一变即生卦变。而《内经》则视脏腑、精气神是一个整体,它们在相互联系、相互调控中存在,因而有形神、心身一体的理论。

《内经》全息医学思想可能也受《周易》整体思维的启发。《系辞》说:“极天下之赜者存乎卦,鼓天下之动者存乎辞。”认为六十四卦贮藏宇宙全部信息。后世医易学家提出“宇宙大天地”“人身小天地”,《内经》则有脉诊、目诊、面诊等,被今世称为全息诊法。

2.变易思维

《周易》强调事物变易的属性,《系辞》说:“知变化之道者,其知神之所为乎?”卦象之间的关系通过爻象及其位置变化来实现,反映自然、人事变化乃不易规律,凶吉祸福随事应时而变,在观念上指导《内经》作者,从运动变化角度研究人的生命活动,并使之理论化。如《素问·玉机真脏论》的“神转不回”之论及其在审察病机与诊治方面的应用,就是《内经》对生命运动不息的认识。

3.相成思维

整体联系、运动变化,都要依赖其内部相互对待两方面相互作用而成就,即相反相成。《系辞》说:“阴阳和德而刚柔有体”“刚柔相推而生变化”,并概括为“一阴一阳之谓道”,这就是相成思维。在人的生理活动和疾病过程中存在大量相反相成的医学现象,相成思维正是把握其变化规律的哲学工具。

相成的前提是相反,而相反之双方是相互依存而不离的,《周易》列举大量相反方面和事物,如乾坤、刚柔、动静等,然而必须将它们约定在一个统一整体之中相互关联,相互对立才有意义。《内经》的阴阳即分多层次,如天地阴阳,天之阴阳,地之阴阳;身形阴阳,形气阴阳,脏腑阴阳,五脏阴阳等。层次不同,各以对立面为前提,目的是将人的功能活动依性质不同分为不同的对立面,以便从相反功能相互作用的方式上,分析其相成机制和规律,如气血营卫、脏腑藏泻。

相反双方相互作用的结果是相成。《周易·乾卦》彖曰“保合太和乃利贞”,和谐是相成的稳态表现,太和是事物高度和谐的境界。这种和谐观为中医学所接受,《素问·生气通天论》以“阴平阳秘”作为健康标准;《素问·五脏别论》强调脏腑藏泻之和;《素问·至真要大论》以“谨察阴阳所在而调之,以平为期”“令其调达而致和平”作为治疗追求的目标。因而,守中贵和成为中医学分析生理病理、确定诊断治法、制订养生方案的基本原则。

\subsection{古代科学技术知识的渗透} %三、

\subsubsection{天文历法}%(一)

1.天文学

天文学是研究天体的位置、分布、运动、形态、结构、化学组成、物理状态和演化的学科。我国是天文学发达最早的国家之一,从春秋至秦汉是中国古代天文学体系形成时期,其知识和方法不仅影响《内经》学术体系的形成,成为“天人合一”内容之一,而且还渗透到中医学的基本概念和理论之中。《内经》天文学的记载及对其学术体系的影响主要体现在三个方面:

一是宇宙演化、宇宙结构观。春秋战国即有天地形成的论述,至《淮南子》明确表述了由混沌无形生有形、生天地阴阳、生万物的宇宙起源与演化观。宇宙结构则有盖天、宣夜、浑天三说。古代的宇宙形成与结构观,一是引导人们从宇宙整体角度探索生命规律:生命体是宇宙演化的产物,因而受养、受制于自然,人必须顺应自然。二是为探索生命规律提供方法学借鉴;以天喻人,将天文学研究方法移植过来,变为医学的研究方法,甚至借用天文学术语表述医学内容。如《素问·宝命全形论》以四时法则研究人的生命活动规律,《素问·阴阳应象大论》以天不足西北、地不满东南之天地阴阳盛衰为喻,解释人的右耳目不如左明、左手足不如右强的生理现象;《灵枢·邪客》“天圆地方,人头圆足方以应之。”则是古代盖天说的遗痕。

二是天象变化。其一,运用北斗星斗柄所指确定地平方位与四时十二月,推知气候变化规律及对人体影响,如《灵枢·九宫八风》的八方之风,其中“虚风”,成为中医病因学说的内容之一。其二,以二十八宿节度太阳运行,把握卫气运行规律,如《灵枢·卫气行》说:“是故房至毕为阳,昴至心为阴,阳主昼,阴主夜。故卫气之行,一日一夜五十周于身,昼日行子阳二十五周,夜行于阴二十五周,周于五脏。”此即《灵枢·营卫生会》所说卫气“与天地同纪”的天文学依据。其三,以黄道标度日月运行节律,将黄道划分为不同的节点系统,这些节点是太阳在黄道上特征位置,用以司天地之气的分、至、启、闭,由此定出四时、八正、二十四节气,推测人体脏腑气血盛衰变化规律。

三是天地日月星辰系统,《内经》将古代天文学关于地球与宇宙天体相互作用的理论,概括为天地日月星辰系统,其中有天地关系、日地关系、地月关系以及五星与地球的关系,并探讨这些天文因素对地球生物、人的影响。如《素问·六微旨大论》说:“天枢之上,天气主之;天枢之下,地气主之;气交之分,人气从之,万物由之。”天枢,指天气地气升降之枢机,在中之位,亦即气交,是人与万物生存的大气圈。大气圈上受天体运动“天气”的影响,如太阳对地球周期性热辐射变化,是形成四季的基础;下受地球地质结构、地面形物及其各种物理效应、散逸气体“地气”的影响,制约生物繁殖生长。天地之气升降相因、形气相感,化生万物,从而构成人类生存环境,制约着人的生命活动,成为《内经》“生气通天”论的天文学基础。

2.历法

历法标度日月星辰运行,把握太阳对地面辐射的周期及其它天体对地球的影响,反映天地阴阳之气消长气数和生命活动的节律,因而也是《内经》学术体系形成基础之一。中国古代历法,主要实行“四分历”,它以一回归年等于365.25日,岁余四分之一日而得名。四分历又用朔望月来定月,用闰月的办法使年的平均长度接近回归年,兼有阴历月和回归年双重性质,属于阴阳合历。《内经》实行的也是四分历,其中的太阳历又有24节气历,与气候、物候变化相符,以表示一年之中生物的生化节律,预测疾病的生死。此外,《内经》还独创了“五运六气历”,它也属于阴阳合历,以天干地支作为运算符号进行推演,阐明六十甲子年中天度、气数、气候、物候、疾病变化与防治规律,从时空角度反映天地人的统一。详细内容可参见“五运六气”章节。

\subsubsection{地理学}%(二)

我国古代地理区划土要有九州说与五方说。九州最早由《尚书·禹贡》提出,《内经》也有九州之名。《素问·五常政大论》说“天不足西北,左寒而右凉;地不满东南,右热而左温。”即按九区划分中国地土方域,它以“阴阳之气,高下之理,太少之异”解说我国西北、东南山川地势、季节气候、气象物候的差异,并与疾病、治则联系起来。此外,该篇还说:“崇高则阴气治之,污下则阳气治之。阳胜者先天,阴胜者后天。此地理之常,生化之道也。”“高者其气寿,·下者其气夭。”“高下之理,地势使然也。”提出地势高低也影响阴阳盛衰,制约生物的生化。五方说最早见于殷人甲骨文中,并将五方与四时气候联系起来观察。《素问·异法方宜论》按五方自然区划,述说各方地势气候、水土物产、衣食起居习惯不同,造就各方人群体质、生理的不同特点,因而发病各异,并发明了不同治法。

以上地理九州说和五方说,《内经》都是将地理因素,通过医学阴阳五行形式,纳入天人一体的方法论轨道,成为学术体系的有机组成部分,也是论治学说中因地制宜的理论根据。

\subsubsection{气象学}%(三)

气象及其周期性、灾害性变化,同人类的生活、生产活动密切相关,也影响人的生存与生命活动,故《素问·六微旨大论》说:“言天者,求之本;言地者,求之位;言人者,求之气交。”气交之分属地球大气层,是与人关系密切的气象变化的所在,因而为古代医学家所关注。古代气象知识对于《内经》学术体系形成的影响,主要有两个方面:

一是充实天人一体整体观,将人与气象相关的思想纳入《内经》学术体系,确立人与气象关系的基本格调。《素问·五运行大论》说:“燥以干之,暑以蒸之,风以动之,湿以润之,寒以坚之,火以温之,故风寒在下,燥热在上,湿气在中,火遊行其间,寒暑六人,故令虚而生化也。”认识到地球气象的周期变化,形成了四季气候,造就了动植物生生化化,因而《素问·五常政大论》建构了谷、果、菜、畜、虫五类生物气象常变繁育、衰耗系统。气象周期性变化即季节气候,也是人类生存、演化的基本条件,《内经》把它们与人体适应这种周期性变化的功能结构联系起来,形成五脏功能活动系统,如《素问·阴阳应象大论》说:“天有四时五行,以生长收藏,以生寒暑燥湿风;人有五脏化五气,以生喜怒悲忧恐。”因而气象的太过、不及和灾害性变化则是疾病发生与变化的重要因素,并贯穿于诊法、防治理论之中。除此而外,《内经》还借气象学名词术语及其变化机理,表述医学内容,如外邪六淫、内生六气的命名;以天地云雨形成之理阐述阳阴互根、升降、转化的机制等。

二是创建中医气象医学——五运六气学说,用以推算气象变化规律及其对人体影响,判定疾病流行情况,审察病机,确立处方、用药法则。

\section{《内经》学术体系的特点} %第三节

\subsection{独特的医学理论} %一、

《内经》按照自己的思维方式确立研究角度和研究方法,形成了独特的人体观、疾病观及疾病防治观。

\subsubsection{人体观}%(一)

在“人与自然相参”思路的指引下,《内经》把人放在宇宙自然中来考察,认为人是自然界的产物和有机组成部分,提出“生气通天”的著名论断,形成了天人相互联系、相互制约的生命整体观,较之割裂人与自然有机联系的医学观念更符合生命活动的客观过程;在古代哲学“道器观”“精气论”的影响下,《内经》将人视为精气聚合、离散之器,生命现象是精气升降出入运动的过程和结果,主要不是研究其形质结构之器,而是从整体机能活动的方式、方法及其相互联系的“道”的方面,研究生命过程及其机制与规律,提出人“以四时之法成”的生命机能结构学说,“阴平阳秘”与五行生克制化的生命机能稳态学说,“奇恒”“回转”的动态生命过程学说,集中体现在藏象、经络、精气神等理论中,较之从形态结构机能活动、从局部到整体的解剖生理分析方法,具有辨证综合的鲜明特点。

\subsubsection{疾病观}%(二)

在人体观基础上,结合医疗实践,《内经》根据“奇恒常变”的观念,确立了自己的疾病理论。《素问·玉机真脏论》说:“天下至数,五色脉变,揆度奇恒,道在于一。神转不回,回则不转,乃失其机。”一就是有序、和谐、统一,在于神气正常运转,而这种有序、和谐的破坏,即神回失机,就是疾病,后世医家从阴阳角度概括为“一阴一阳之谓道,偏阴偏阳之谓疾。”诸凡行卧饮食、情志思维等一切身心活动反生理之常者,均属阴阳失调而为病。它不以形质结构及其物量变化的超标作为衡量疾病与健康的单一标准,而是更强调整体机能的紊乱与失常。关于疾病的发生,《内经》以“邪正相争”阐明其机理,提出六淫疫邪侵袭、七情饮食失调与劳伤概括其致病方式,从致病因素与机体抗病能力相互作用的结果审求其病理意义的病因学、发病学理论,称作“审证求因”,与以理、化、生物性因素致病的因果决定论还原模式不同。关于病理变化的机制,《内经》着眼于动态分析整体机能失调的方式、状态和过程,提出了以脏腑、经络、气血津液病变为基础的表里出入、寒热进退、邪正虚实、气血运行紊乱和疾病传变等理论,与从组织器官的形质异常论病理,其特点自明。

\subsubsection{疾病防治观}%(三)

在疾病观基础上,《内经》提出“审机论治”的诊治原则,后世演化为辨证论治。“证”作为诊断概念和治疗对象,是对疾病过程中致病因素与机体相互作用所产生的整体机能失调病候本质的概栝,因时而异、因人而别,因而中医治疗学的基本特点是整体机能的动态、综合之协调,即《素问·至真要大论》所说的“谨察其阴阳所在而调之,以平为期。”它将治疗个体化,强调治患病之人;提倡各种方法配合应用,强调综合疗法;它的逆从求本、标本缓急、病治异同以及虚实补泻、寒热温清、因势利导等治则,颇似系统调控方法,与针对理化生物性病因、局部病灶治以特效药,革除有余、填充不足的替代疗法以及重视群体共性病变的治疗观念相比,两者的诊治思路,有重人与重病、重个性与重共性、重整体与重局部、重机能与重形体的不同。对于疾病的预防,《内经》根据自己的健康观和发病观,提出以增强体质为核心的健身防病思想,制定了外以适应自然变化、内以促进机体抗病能力、机体协调能力的养生原则,并有效指导了各种自我健身法的实施,在世界保健医学上独树一帜。

\subsection{方法论特点} %二、

方法论是关于认识世界和改造世界一般方法的理论。从层次而言,有哲学方法论、一般科学方法论和具体科学方法论。受民族文化的深刻影响,中国传统科学具有自己鲜明的方法论特点,中医学是中华民族文化的重要组成部分,与西医学相比,在研究人的生命规律及医学活动中,也有明显的方法论特点。这方面的研究中西医汇通学派的有些学者已经有所触及,如恽树珏说“西医之生理以解剖,《内经》之生理以气化。”(《群经见智录》)“中西医学基础不同……此则中西文化不同之故。”(《论医集·对于统一病名建议书之商榷》)经过一个世纪研究,特别是近20多年的探索、论证,《内经》学术体系的方法论特点,主要是侧重于从功能角度、整体角度、变化角度把握生律。

\subsubsection{从功能角度把握生命规律}%(一)

人们认识生命奥秘,首先从生命现象入手。在医学理论形成初期,东西方都以解剖作为研究手段,如《灵枢·经水》就有“其死可解剖而视之”的记载。但如何把生命现象与解剖内脏器官相联系,没有先进仪器和精密测量方法,是不可能做到的。观察的结果,也不能有效指导临床实践。也就是说,当时解剖并不能直接导致医学理论的产生。由于这种原因,西方的医学直至近代,由于哲学的变革,观察手段的改进,如使用显微镜等,才使医学研究思路和方法来了一场大革命,将解剖形态结构与生命现象直接联系起来,形成西医学基础理论。但在中国古代并非如此。当意识到解剖并不能直接解释生命现象与指导医疗活动后,转而采用当时盛行的自然哲学方法。首先对生命现象及与其相眹系的各方面进行观察,然后把观察内容中的“共相”提取出来,按其形态、功能、格局、演化方式进行分类,并将具有代表性的、具有共相的“类”,用象征性符号、图象或有代表性的具体事物表达,进而以类相推,探讨生命现象的机理,这就是古代的意象思维方式。明清之际的哲学家王夫之把这种思维过程叫作“观象明理”“观象体义”。用这种思维方法进行研究,只能引出功能性概念,而非解剖实体概念。如《素问·五脏生成论》说:“五脏之象,可以类推”,王冰解释说:“象,谓气象也,言五脏虽隐而不见,然其气象性用,犹可以物推之,何者?肝象木而曲直,心象火而炎上,脾象土而安静,肺象金而刚决,肾象水而润下。”这里的木火土金水只是象征性符号,它所表征的五脏“气象性用”即其功能特性,是类推而来的,其本质是基于外在相关生命现象而存在于体内的生理功能分类整合。

有人担心中医学概念与实体脏器不符,违背结构与机能统一的原则,没有物质基础。这是一种狭隘观念。王夫之说:“天下之用,皆其有者也。吾从其用而知其体之有,岂疑待哉?”(《周易外传》卷二)。生命活动机制是复杂的,生命活动规律也应从多角度探索。中医理论所反映的生命活动机制及规律,既经千余年医疗实践得以证实,必定有其相应的物质结构存在,因而可换一种思路,从多系统、多层次、多维向地研究,而非简单的组织解剖学物质基础“认同”。

从功能角度把握生命规律是《内经》学术体系的一个基本特征,其他特征以此为前提而成立。如讲整体应是功能上的相互联系与调控,即黄元御《四圣心源》所说五行生克“以气而不以质”,“成质则不能生克矣。”因而《内经》的基本医学概念和理论规范,是生命活动中各种功能相互联系的方式、机制与过程的概括。如中医所论气与血均系相互联系中存在的概念,其实质应从气血关系中探索,割裂气血,单独研究气或血的思路是没有意义的。又如五脏的功能活动是在整体联系中存在的,应从五脏的整体联系纽带上探索,不宜绝然割裂式的单独研究。这就为中医学与中国古代的其他学科,包括自然科学、人文科学中的众多门类,在理论和方法上的沟通架起桥梁,如将治病与治国、打仗相提并论等。更重要的是,它赋予中医诊治理论与方法以功能化的内涵。所谓辨证即辨别人体病理性综合功能状态,治疗也是对病理性功能状态进行综合调节。这虽然从方法论上袒露出中医学形态研究不足,并隐含着中医作为应用科学技术的缺憾,但也有其优势。它从功能上进行宏观而综合调节的论治思路与方法,对于多系统、多脏器、多组织的复杂病变,功能失调性、原因不明的病变以及与自我功能失调关系密切的病毒性感染疾病等,均显示出不凡的疗效,不但具有使用价值,在医学模式转变的今天更有深刻的学术意义。

\subsubsection{从整体角度把握生命规律}%(二)

整体观是指用普遍联系的有机整体观念来看待一切事物,承认事物与事物之间、事物内部的各部分、各层次之间的相互联系、相互影响。与近代科学在知性分析指导下割裂联系进行实验观察,然后还原其联系的研究思路与方法不同,中医学的整体观念源于把生命现象放在其生存环境,即自然、社会中进行总体的直观察验,并接受中国古代自然哲学的指导,将这种观察引向理性认识的层次。首先,古人观察到人的生命活动与其生存环境有着密切关系,于是确立了人与自然及社会的有机联系,形成“天人一体”观念;观察到人生命活动中,生命能力与躯体形骸之间、精神心理与躯体生理之间有着密切关系,于是确立了人的生理、心理、躯体三者的有机联系,形成“形神一体”和“心身一体”观念。其次,适应于医学研究与应用的需要,古人还将这三种联系融入中医学的基本概念与理论模式之中,成为中医理论的基本学术内涵和临床诊治的指导原则及价值取向。体现人与自然有机联系,《内经》有“生气通天”的著名论断,因而中医五脏不仅维持体内生理环境的协调,同时还有时空的内涵,主司人体适应自然界季节昼夜、方域水土的调节功能,故《素问·宝命全形论》有人以“四时之法成”、《素问·刺禁论》有“肝生于左,肺藏于右”之说;体现人体生理、心理、躯体的有机联系,《内经》则有五脏主五体、藏精、舍神的理论,所以五脏即“五藏”,又称“五神脏”。于是五脏就成为人体联系内外、协调心身的生命活动中枢,是中医整体观在基本概念的集中体现。中医学的其它基本概念,如经络、气血等,其内涵亦均类于此。这就造就了中医学从自然环境与社会环境、生物属性与心理作用上研究人的生命活动及其医学应用的医学模式。

《内经》不仅重视普遍体联系,还认为这种联系的本然秩序是整体和谐,并以精气、阴阳、五行学说作为思维框架,推演其机制。如《灵枢·天年》以“气之盛衰”论生命过程,并以“好走”、“好坐”、“好卧”等作为气盛衰的象征。这里的气即精气,亦即生命力,是包括形体状态、生理功能、精神心理的综合性整体概念。又如以藏泻阴阳论脏腑,藏是生命物质化生、贮藏、发挥作用,并固护之不得无故丧失等功能的总和;泻则是生命物质宣泄消耗、代谢产物排除等功能的概括;而藏泻间互相依存、互相制约,概括生命活动中两类相反功能和谐相成的机制。再如《素问·金匮真言论》以五脏为中心,外应五方、五时、五味等,内系五腑、五官、五体、五志等构建五大功能活动系统,融天人联系、躯体生理心理联系于一体,即“四时五藏阴阳”,是整体观念在中医理论中的重要体现。

当然,也应看到,在整体观念指导下形成的中医学概念,其内涵包容性太大,外延过于宽泛。如神有自然规律、生命能力、精神意识情志等多种概念;五脏概念既有藏精,又舍神,还含应四时等内涵。这是由于中医强调整体联系并欲在概念中反映出来,而联系的对象、范围以及时空条件变化不定的缘故。这就要在学习和研究中加以辨识,区别概念的层次,明确其中心内涵、概念泛化以及条件限定等。

\subsubsection{从变化角度把握生命规律}%(三)

运动变化是事物存在的本质属性,也是生命存在的固有特征。古人早已观察到生命随着时间连续流转而变化的事实。但由于生命参数非常复杂,运动所产生的变量更是难以把握。对此,近代科学釆取“定格”的知性分析方法,即将连续的时间割断,进行静态的研究。其结果精密准确,但也有失于自然、真实之弊。中医学在形成初期,没有精密仪器,不可能对人生命活动中的多变量进行分别度量、“定格”分析,只能整体观察、综合研究;而观察对象则是变动不息的生理、病理现象,从而形成从运动变化角度把握生命规律的方法学特点,并具有动态化的理论表述。主要本现在三个方面:

一是在医学理论中,明确表述了生命运动变化原理,如《素问·玉版论要》说:“道之至数……神转不回,回则不转,乃失其机”,认为有序的运动变化是生命存在的基本形式。尽管它在表达形式上的传统文字、术语需要诠释,但其表达的生长壮老已生命过程、脏腑经络气血升降出入运动机制与规律的含义是显而易见的。

二是医学概念具有时间内涵。时间是事物运动及其状态变化的度量,凡概念中标示出时间含义,便说明这一概念具有动态的内涵。如《素问·金匮真言论》说“五脏应四时”,是指五脏应时而旺,乃人体精气随季节递迁而流转消长的四个阶段之功能整合,举肝为例,它不仅是藏血主疏泄之脏,还有主春之内涵。

三是辨证论治体现中医诊治动态观。证是疾病过程中阶段性病机模式,它虽然具有一定稳定性,但随病变而变;同时证本身的形成与内外环境的时序流转也有密切关系,如外感邪气形成、致病特点及病症种类时效性很强;内伤病症与患者年龄变化、与体内脏腑经络气血营卫运动节律,无不相关,从而为中医诊断所关注,并成为治疗中重视时间因素的依据。所谓“毋逆天时”、“无失气宜”论即基于此。而一病前后证异,用药施治随时变换,则是中医理论动态化特征的明显表现。

《内经》学术体系的这一特点,赋予中医理论两个方面的先天素质:第一,忽略生命体物质的规定性和测量性,从功能象变角度对生命的动态轨迹进行模糊地整体表述。在中医理论肇始之初,对形体结构精密度量既不可能,又劳而无功,特别是它在表述生命的动变特性时无能力力,所以古人只能寻求从生命之象及其内在功能变化角度描述其运动轨迹,而对生命体形态结构采取模糊化处理。如脉证太过不及和死证死脉的度量,色泽浮沉夭泽的判断,阴阳表里寒热虚实八纲的表述方法,都具有模糊的性质。这种表述方法用宏观整体、边界不清、随时变化的文章数学语言,而不用符号数学计量,虽然失于粗疏,但更接近于生命的自然动态演化机栝与过程。与之相应,在疾病治疗的探索中,中医也摸索到使用天然药物等进行模糊调控的临床处理方法,至今仍有其科学意义和实用价值。第二,中国古人把时间流转和空间变化结合起来,认为时间流转具有周期性,同时还发生着相应空间状态的周期演变,在《内经》则形成有关生命节律的思想,并用于指导疾病的诊治,显示出它的科学意义和实用价值。

\subsection{别具一格的诊治方法} %三、

受文化背景、思维方式与研究方法的影响,在长期的医疗实践中,中医学形成了自己的诊治方法。在收集疾病信息方面,没有仪器可资利用,便充分发挥眼耳鼻舌身等人体感官的自然机能,通过视听嗅味触等感觉以及详尽询问患者,最大限度地获取对辨别病情有意义的体征与症状。这些信息源于患者个体的自然流露,未受干扰,保持着真实性;它是患病个体机能失调的动态、综合反应,容易避免珍断上的片面性。尽管这些方法、手段在获取病变微观信息方面有天然不足,但在全面、系统、特异性地获取宏观信息以及与治疗相配合方面,却是不可取代的。在治疗方法上,中医使用天然药物组成方剂,又有针灸、推拿、气功、饮食、精神诸疗法单独或相配合实施,与中医学的疾病观、治疗观相契合,疗效卓著,千百年来历验不忒。其方药的天人合一作用原理、性味归经方法与药效实践,能发挥宏观、综合地协调机能的效应;而针灸发挥信息综合调控效应,气功与精神疗法则发挥身心相互调控效应等。这些均系中医学所特有,与使用化学合成药物、手术切除病灶疗法相比,虽各有长短优劣,但其发挥的综合协调作用。特别是顾及致病因素与机体反应状态两个方面,强调促进机体自身抗病能力,易为人体所接受,在治疗方法评价上,具有鲜明特色和学术价值。

\section{《内经》学术体系的价值} %第四节

\subsection{《内经》揭示的生命活动规律,对于生命科学具有重要价值} %一、

历史发展到21世纪,人类科学技术的步伐正跨步进入辩证综合阶段。在这种大背景下,19世纪80$\sim$90年代人们以科学方法论,即系统方法来研究、论征中医学科学内涵与特点,确认中、西医学属于两种不同的学术体系,并各有自己的医学发现。与中华民族的文化社会观、哲学义理观、科学技术观同源共辙,中医学注重研究生命体的机能结构,强调机能的整体、动态和谐,并概括为藏象学说、经络学说、病机学说等。这些理论,由于它不同于近现代科学的表述形式,长期以来不为世界所认识和理解。经过长期研究、整理,人们才逐渐认识到其部分科学内涵,其中蕴涵着中医学特有的医学发现,如人与自然对立统一的关系,生命活动的系统法则、调控法则与节律现象,心身相互作用现象,经络现象,生命全息现象等。它们虽然在科学形态上有待进一步提高,但其临床应用与重复已有两千余年历史,客观实在性是不容置疑的。特别应当提出的是,中医学理论及其临床验证与疗效的真实,提示人们:研究生命规律,除了近代科学及西医学从生命体的解剖形态及其生理物质变化的途径探索外,还可以从生命过程中各种生理机能之间及其与生存环境相互作用的关系与途径来探索;人类个体的生命活动,除了已知的解剖生理系统及其机能调控规律外,还可能存在基于相关生命现象、整合为若干机能系统自我调控的规律,这在生命科学研究的思路、方法上有重要价值。

\subsection{运用多学科研究医学,建构了天地人“三才”医学模式} %二、

《内经》在建立自己的学术体系时,吸收了天文、历法、地理、气象、生物、社会、心理、哲学等中国古代传统的人文、自然多学科的研究方法与成果,建构天地人“三才”医学模式,使《内经》能够真实反映人体生命活动的客观过程,经得起永久考验,并成为多学科研究医学的典范。正如《素问·著至教论》说“道上知天文,下知地理,中知人事,可以长久。以教众庶,亦不疑殆,医道论篇,可传后世,可以为宝。”

运用多学科知识和方法研究医学,出自中医学自身发展的需要。《内经》认为,人生天地之间,并非孑然独立,而与其生存环境有着密不可分的联系。人类的生存环境,一是自然环境,即三才中的天、地。中国古代传统科学中的天文、历法、地理、气象、生物等学科,就是研究天地变化规律的学问;天地变化对人生命活动的影响,就是医学研究的内容。《内经》将天地学问有机地融入医学内容,形成理论,这就是藏象学说中“生气通天”的理论,病机学说中外感病因、发病、病理及其传变的理论,诊治学说中外感病辨证、治法的理论。《内经》的后期作品还有“五运六气学说”,论气候变化各种周期及其对人体生理、病理影响和诊治原则与方法,也是涉及天地的医学内容。二是社会环境,即三才中的人,也就是人事,泛指社会人际之事,大而至于社会政治背景、经济水平、文化环境、民风习俗等,小而至于病人的政治、经济地位,文化教养、家境遭遇、个人经历等,都与人的心身健康有着密切关系,并在藏象、病机、诊治学说中得到反映。除此之外,人的精神、心理,也属于人事的一个重要组成部分,与自然、社会环境,特别是社会遭遇、人际交往关系更大,《内经》专有“五神脏”理论,贯穿于藏象、病机、诊治学说之中。

《内经》天地人三才医学模式重视人与自然、社会的协调,将人与生存环境的和谐、人体心身的和谐视为健康的基本标准,并贯穿于疾病防治和抗衰老理论与实践之中,这是《内经》对于世界医学的贡献。它与近年医学界提出的“社会——心理——生物医学模式”的基本观点是相通的,但其可贵之处是,它已经完全融入于自己的理论,并且作为临床的基本原则和方法实施于医疗活动之中;它历史悠久,经过千百年临床验证,无论在学术研究,还是医疗实践,对于医学科学都有重要价值。

\subsection{独特的医学发明,创建了世界特有的疾病诊疗体系} %三、

在疾病观和疾病诊治观的基础上,《内经》创建了自己的疾病诊疗体系。在收集疾病信息方面,没有仪器可资利用,古人便创造性地提出望、闻、问、切“四诊”之法,利用人体感官本能,通过视觉、听觉、嗅觉、味觉、触觉以及详尽询问患者,详尽收集疾病的体征与症状,并通过“四诊合参”,去粗取精、去伪存真,最大限度地为疾病的诊断提供全面、系统资料,并且排除干扰,保证资料的真实性。同时,这些信息源于病体的自然流露,未受人为干扰,保持着原始性;是病体机能失调动态、综合的反应,与治疗中的综合调控相呼应。尽管这些方法、手段在获取病变微观信息方而有天然不足,但在全面、系统、特异性地获取疾病综合信息方面,却是现代实验方法仅取单一理化指标所不能取代的。在治疗方法上,如前所述,中医使用天然药物组成方剂,又有针灸、推拿、气功、饮食、精神多种疗法相配合,其方药天人合一作用原理、性味归经独特理论与药效实践,能发挥宏观、综合地协调机能效应;而针灸发挥信息综合调控效应,气功与精神疗法则发挥身心相互调控效应,与中医独特的诊法相呼应,形成了中医诊疗的自己评价、自我调节和自我完善机制。在这一沴疗体系中,孕育了《内经》的医学发明,如面诊、目诊、耳诊、脉诊等所体现的全息诊察法,涉及自然、社会、心理等诊察范围和内容的系统诊察法,天然药物的识别、分类、采集、炮制、配伍和辨证应用法以及针灸、推拿、气功、精神、饮食等无创伤疗法,丰富了世界医学宝库。这一诊疗体系,在世界传统医学中具有代表性,对于临床疾病的诊疗也有自己的特色和优势,在人类面临新的医学课题和征服世界性重大疾病的机遇和挑战中,可以有大的作为。

\zuozhe{(烟建华)}
\ifx \allfiles \undefined
\end{document}
\fi