% -*- coding: utf-8 -*-
%!TEX program = xelatex
\ifx \allfiles \undefined
\documentclass[12pt]{ctexbook}
%\usepackage{xeCJK}
%\usepackage[14pt]{extsizes} %支持8,9,10,11,12,14,17,20pt

%===================文档页面设置====================
%---------------------印刷版尺寸--------------------
%\usepackage[a4paper,hmargin={2.3cm,1.7cm},vmargin=2.3cm,driver=xetex]{geometry}
%--------------------电子版------------------------
\usepackage[a4paper,margin=2cm,driver=xetex]{geometry}
%\usepackage[paperwidth=9.2cm, paperheight=12.4cm, width=9cm, height=12cm,top=0.2cm,
%            bottom=0.4cm,left=0.2cm,right=0.2cm,foot=0cm, nohead,nofoot,driver=xetex]{geometry}

%===================自定义颜色=====================
\usepackage{xcolor}
  \definecolor{mybackgroundcolor}{cmyk}{0.03,0.03,0.18,0}
  \definecolor{myblue}{rgb}{0,0.2,0.6}

%====================字体设置======================
%--------------------中文字体----------------------
%-----------------------xeCJK下设置中文字体------------------------------%
\setCJKfamilyfont{song}{SimSun}                             %宋体 song
\newcommand{\song}{\CJKfamily{song}}                        % 宋体   (Windows自带simsun.ttf)
\setCJKfamilyfont{xs}{NSimSun}                              %新宋体 xs
\newcommand{\xs}{\CJKfamily{xs}}
\setCJKfamilyfont{fs}{FangSong_GB2312}                      %仿宋2312 fs
\newcommand{\fs}{\CJKfamily{fs}}                            %仿宋体 (Windows自带simfs.ttf)
\setCJKfamilyfont{kai}{KaiTi_GB2312}                        %楷体2312  kai
\newcommand{\kai}{\CJKfamily{kai}}
\setCJKfamilyfont{yh}{Microsoft YaHei}                    %微软雅黑 yh
\newcommand{\yh}{\CJKfamily{yh}}
\setCJKfamilyfont{hei}{SimHei}                                    %黑体  hei
\newcommand{\hei}{\CJKfamily{hei}}                          % 黑体   (Windows自带simhei.ttf)
\setCJKfamilyfont{msunicode}{Arial Unicode MS}            %Arial Unicode MS: msunicode
\newcommand{\msunicode}{\CJKfamily{msunicode}}
\setCJKfamilyfont{li}{LiSu}                                            %隶书  li
\newcommand{\li}{\CJKfamily{li}}
\setCJKfamilyfont{yy}{YouYuan}                             %幼圆  yy
\newcommand{\yy}{\CJKfamily{yy}}
\setCJKfamilyfont{xm}{MingLiU}                                        %细明体  xm
\newcommand{\xm}{\CJKfamily{xm}}
\setCJKfamilyfont{xxm}{PMingLiU}                             %新细明体  xxm
\newcommand{\xxm}{\CJKfamily{xxm}}

\setCJKfamilyfont{hwsong}{STSong}                            %华文宋体  hwsong
\newcommand{\hwsong}{\CJKfamily{hwsong}}
\setCJKfamilyfont{hwzs}{STZhongsong}                        %华文中宋  hwzs
\newcommand{\hwzs}{\CJKfamily{hwzs}}
\setCJKfamilyfont{hwfs}{STFangsong}                            %华文仿宋  hwfs
\newcommand{\hwfs}{\CJKfamily{hwfs}}
\setCJKfamilyfont{hwxh}{STXihei}                                %华文细黑  hwxh
\newcommand{\hwxh}{\CJKfamily{hwxh}}
\setCJKfamilyfont{hwl}{STLiti}                                        %华文隶书  hwl
\newcommand{\hwl}{\CJKfamily{hwl}}
\setCJKfamilyfont{hwxw}{STXinwei}                                %华文新魏  hwxw
\newcommand{\hwxw}{\CJKfamily{hwxw}}
\setCJKfamilyfont{hwk}{STKaiti}                                    %华文楷体  hwk
\newcommand{\hwk}{\CJKfamily{hwk}}
\setCJKfamilyfont{hwxk}{STXingkai}                            %华文行楷  hwxk
\newcommand{\hwxk}{\CJKfamily{hwxk}}
\setCJKfamilyfont{hwcy}{STCaiyun}                                 %华文彩云 hwcy
\newcommand{\hwcy}{\CJKfamily{hwcy}}
\setCJKfamilyfont{hwhp}{STHupo}                                 %华文琥珀   hwhp
\newcommand{\hwhp}{\CJKfamily{hwhp}}

\setCJKfamilyfont{fzsong}{Simsun (Founder Extended)}     %方正宋体超大字符集   fzsong
\newcommand{\fzsong}{\CJKfamily{fzsong}}
\setCJKfamilyfont{fzyao}{FZYaoTi}                                    %方正姚体  fzy
\newcommand{\fzyao}{\CJKfamily{fzyao}}
\setCJKfamilyfont{fzshu}{FZShuTi}                                    %方正舒体 fzshu
\newcommand{\fzshu}{\CJKfamily{fzshu}}

\setCJKfamilyfont{asong}{Adobe Song Std}                        %Adobe 宋体  asong
\newcommand{\asong}{\CJKfamily{asong}}
\setCJKfamilyfont{ahei}{Adobe Heiti Std}                            %Adobe 黑体  ahei
\newcommand{\ahei}{\CJKfamily{ahei}}
\setCJKfamilyfont{akai}{Adobe Kaiti Std}                            %Adobe 楷体  akai
\newcommand{\akai}{\CJKfamily{akai}}

%------------------------------设置字体大小------------------------%
\newcommand{\chuhao}{\fontsize{42pt}{\baselineskip}\selectfont}     %初号
\newcommand{\xiaochuhao}{\fontsize{36pt}{\baselineskip}\selectfont} %小初号
\newcommand{\yihao}{\fontsize{28pt}{\baselineskip}\selectfont}      %一号
\newcommand{\xiaoyihao}{\fontsize{24pt}{\baselineskip}\selectfont}
\newcommand{\erhao}{\fontsize{21pt}{\baselineskip}\selectfont}      %二号
\newcommand{\xiaoerhao}{\fontsize{18pt}{\baselineskip}\selectfont}  %小二号
\newcommand{\sanhao}{\fontsize{15.75pt}{\baselineskip}\selectfont}  %三号
\newcommand{\sihao}{\fontsize{14pt}{\baselineskip}\selectfont}%     四号
\newcommand{\xiaosihao}{\fontsize{12pt}{\baselineskip}\selectfont}  %小四号
\newcommand{\wuhao}{\fontsize{10.5pt}{\baselineskip}\selectfont}    %五号
\newcommand{\xiaowuhao}{\fontsize{9pt}{\baselineskip}\selectfont}   %小五号
\newcommand{\liuhao}{\fontsize{7.875pt}{\baselineskip}\selectfont}  %六号
\newcommand{\qihao}{\fontsize{5.25pt}{\baselineskip}\selectfont}    %七号   %中文字体及字号设置
\xeCJKDeclareSubCJKBlock{SIP}{
  "20000 -> "2A6DF,   % CJK Unified Ideographs Extension B
  "2A700 -> "2B73F,   % CJK Unified Ideographs Extension C
  "2B740 -> "2B81F    % CJK Unified Ideographs Extension D
}
%\setCJKmainfont[SIP={[AutoFakeBold=1.8,Color=red]Sun-ExtB},BoldFont=黑体]{宋体}    % 衬线字体 缺省中文字体

\setCJKmainfont{simsun.ttc}[
  Path=fonts/,
  SIP={[Path=fonts/,AutoFakeBold=1.8,Color=red]simsunb.ttf},
  BoldFont=simhei.ttf
]

%SimSun-ExtB
%Sun-ExtB
%AutoFakeBold:自动伪粗,即正文使用\bfseries时生僻字使用伪粗体;
%FakeBold:强制伪粗,即正文中生僻字均使用伪粗体
%\setCJKmainfont[BoldFont=STHeiti,ItalicFont=STKaiti]{STSong}
%\setCJKsansfont{微软雅黑}黑体
%\setCJKsansfont[BoldFont=STHeiti]{STXihei} %serif是有衬线字体sans serif 无衬线字体
%\setCJKmonofont{STFangsong}    %中文等宽字体

%--------------------英文字体----------------------
\setmainfont{simsun.ttc}[
  Path=fonts/,
  BoldFont=simhei.ttf
]
%\setmainfont[BoldFont=黑体]{宋体}  %缺省英文字体
%\setsansfont
%\setmonofont

%===================目录分栏设置====================
\usepackage[toc,lof,lot]{multitoc}    % 目录(含目录、表格目录、插图目录)分栏设置
  %\renewcommand*{\multicolumntoc}{3} % toc分栏数设置,默认为两栏(\multicolumnlof,\multicolumnlot)
  %\setlength{\columnsep}{1.5cm}      % 调整分栏间距
  \setlength{\columnseprule}{0.2pt}   % 调整分栏竖线的宽度

%==================章节格式设置====================
\setcounter{secnumdepth}{3} % 章节等编号深度 3:子子节\subsubsection
\setcounter{tocdepth}{2}    % 目录显示等度 2:子节

\xeCJKsetup{%
  CJKecglue=\hspace{0.15em},      % 调整中英(含数字)间的字间距
  %CJKmath=true,                  % 在数学环境中直接输出汉字(不需要\text{})
  AllowBreakBetweenPuncts=true,   % 允许标点中间断行,减少文字行溢出
}

\ctexset{%
  part={
    name={,篇},
    number=\SZX{part},
    format={\chuhao\bfseries\centering},
    nameformat={},titleformat={}
  },
  section={
    number={\chinese{section}},
    name={第,节}
  },
  subsection={
    number={\chinese{subsection}、},
    aftername={\hspace{-0.01em}}
  },
  subsubsection={
    number={(\chinese{subsubsection})},
    aftername={\hspace {-0.01em}},
    beforeskip={1.3ex minus .8ex},
    afterskip={1ex minus .6ex},
    indent={\parindent}
  },
  paragraph={
    beforeskip=.1\baselineskip,
    indent={\parindent}
  }
}

\newcommand*\SZX[1]{%
  \ifcase\value{#1}%
    \or 上%
    \or 中%
    \or 下%
  \fi
}

%====================页眉设置======================
\usepackage{titleps}%或者\usepackage{titlesec},titlesec包含titleps
\newpagestyle{special}[\small\sffamily]{
  %\setheadrule{.1pt}
  \headrule
  \sethead[\usepage][][\chaptertitle]
  {\chaptertitle}{}{\usepage}
}

\newpagestyle{main}[\small\sffamily]{
  \headrule
  %\sethead[\usepage][][第\thechapter 章\quad\chaptertitle]
%  {\thesection\quad\sectiontitle}{}{\usepage}}
  \sethead[\usepage][][第\chinese{chapter}章\quad\chaptertitle]
  {第\chinese{section}节\quad\sectiontitle}{}{\usepage}
}

\newpagestyle{main2}[\small\sffamily]{
  \headrule
  \sethead[\usepage][][第\chinese{chapter}章\quad\chaptertitle]
  {第\chinese{section}節\quad\sectiontitle}{}{\usepage}
}

%================ PDF 书签设置=====================
\usepackage{bookmark}[
  depth=2,        % 书签深度 2:子节
  open,           % 默认展开书签
  openlevel=2,    % 展开书签深度 2:子节
  numbered,       % 显示编号
  atend,
]
  % 相比hyperref,bookmark宏包大多数时候只需要编译一次,
  % 而且书签的颜色和字体也可以定制。
  % 比hyperref 更专业 (自动加载hyperref)

%\bookmarksetup{italic,bold,color=blue} % 书签字体斜体/粗体/颜色设置

%------------重置每篇章计数器,必须在hyperref/bookmark之后------------
\makeatletter
  \@addtoreset{chapter}{part}
\makeatother

%------------hyperref 超链接设置------------------------
\hypersetup{%
  pdfencoding=auto,   % 解决新版ctex,引起hyperref UTF-16预警
  colorlinks=true,    % 注释掉此项则交叉引用为彩色边框true/false
  pdfborder=001,      % 注释掉此项则交叉引用为彩色边框
  citecolor=teal,
  linkcolor=myblue,
  urlcolor=black,
  %psdextra,          % 配合使用bookmark宏包,可以直接在pdf 书签中显示数学公式
}

%------------PDF 属性设置------------------------------
\hypersetup{%
  pdfkeywords={黄帝内经,内经,内经讲义,21世纪课程教材},    % 关键词
  %pdfsubject={latex},        % 主题
  pdfauthor={主编:王洪图},   % 作者
  pdftitle={内经讲义},        % 标题
  %pdfcreator={texlive2011}   % pdf创建器
}

%------------PDF 加密----------------------------------
%仅适用于xelatex引擎 基于xdvipdfmx
%\special{pdf:encrypt ownerpw (abc) userpw (xyz) length 128 perm 2052}

%仅适用于pdflatex引擎
%\usepackage[owner=Donald,user=Knuth,print=false]{pdfcrypt}

%其他可使用第三方工具 如:pdftk
%pdftk inputfile.pdf output outputfile.pdf encrypt_128bit owner_pw yourownerpw user_pw youruserpw

%=============自定义环境、列表及列表设置================
% 标题
\def\biaoti#1{\vspace{1.7ex plus 3ex minus .2ex}{\bfseries #1}}%\noindent\hei
% 小标题
\def\xiaobt#1{{\bfseries #1}}
% 小结
\def\xiaojie {\vspace{1.8ex plus .3ex minus .3ex}\centerline{\large\bfseries 小\ \ 结}\vspace{.1\baselineskip}}
% 作者
\def\zuozhe#1{\rightline{\bfseries #1}}

\newcounter{yuanwen}    % 新计数器 yuanwen
\newcounter{jiaozhu}    % 新计数器 jiaozhu

\newenvironment{yuanwen}[2][【原文】]{%
  %\biaoti{#1}\par
  \stepcounter{yuanwen}   % 计数器 yuanwen+1
  \bfseries #2}
  {}

\usepackage{enumitem}
\newenvironment{jiaozhu}[1][【校注】]{%
  %\biaoti{#1}\par
  \stepcounter{jiaozhu}   % 计数器 jiaozhu+1
  \begin{enumerate}[%
    label=\mylabel{\arabic*}{\circledctr*},before=\small,fullwidth,%
    itemindent=\parindent,listparindent=\parindent,%labelsep=-1pt,%labelwidth=0em,
    itemsep=0pt,topsep=0pt,partopsep=0pt,parsep=0pt
  ]}
  {\end{enumerate}}

%===================注解与原文相互跳转====================
%----------------第1部分 设置相互跳转锚点-----------------
\makeatletter
  \protected\def\mylabel#1#2{% 注解-->原文
    \hyperlink{back:\theyuanwen:#1}{\Hy@raisedlink{\hypertarget{\thejiaozhu:#1}{}}#2}}

  \protected\def\myref#1#2{% 原文-->注解
    \hyperlink{\theyuanwen:#1}{\Hy@raisedlink{\hypertarget{back:\theyuanwen:#1}{}}#2}}
  %此处\theyuanwen:#1实际指thejiaozhu:#1,只是\thejiaozhu计数器还没更新,故使用\theyuanwen计数器代替
\makeatother

\protected\def\myjzref#1{% 脚注中的引用(引用到原文)
  \hyperlink{\theyuanwen:#1}{\circlednum{#1}}}

\def\sb#1{\myref{#1}{\textsuperscript{\circlednum{#1}}}}    % 带圈数字上标

%----------------第2部分 调整锚点垂直距离-----------------
\def\HyperRaiseLinkDefault{.8\baselineskip} %调整锚点垂直距离
%\let\oldhypertarget\hypertarget
%\makeatletter
%  \def\hypertarget#1#2{\Hy@raisedlink{\oldhypertarget{#1}{#2}}}
%\makeatother

%====================带圈数字列表标头====================
\newfontfamily\circledfont[Path = fonts/]{meiryo.ttc}  % 日文字体,明瞭体
%\newfontfamily\circledfont{Meiryo}  % 日文字体,明瞭体

\protected\def\circlednum#1{{\makexeCJKinactive\circledfont\textcircled{#1}}}

\newcommand*\circledctr[1]{%
  \expandafter\circlednum\expandafter{\number\value{#1}}}
\AddEnumerateCounter*\circledctr\circlednum{1}

% 参考自:http://bbs.ctex.org/forum.php?mod=redirect&goto=findpost&ptid=78709&pid=460496&fromuid=40353

%======================插图/tikz图========================
\usepackage{graphicx,subcaption,wrapfig}    % 图,subcaption含子图功能代替subfig,图文混排
  \graphicspath{{img/}}                     % 设置图片文件路径

\def\pgfsysdriver{pgfsys-xetex.def}         % 设置tikz的驱动引擎
\usepackage{tikz}
  \usetikzlibrary{calc,decorations.text,arrows,positioning}

%---------设置tikz图片默认格式(字号、行间距、单元格高度)-------
\let\oldtikzpicture\tikzpicture
\renewcommand{\tikzpicture}{%
  \small
  \renewcommand{\baselinestretch}{0.2}
  \linespread{0.2}
  \oldtikzpicture
}

%=========================表格相关===============================
\usepackage{%
  multirow,                   % 单元格纵向合并
  array,makecell,longtable,   % 表格功能加强,tabu的依赖
  tabu-last-fix,              % "强大的表格工具" 本地修复版
  diagbox,                    % 表头斜线
  threeparttable,             % 表格内脚注(需打补丁支持tabu,longtabu)
}

%----------给threeparttable打补丁用于tabu,longtabu--------------
%解决方案来自:http://bbs.ctex.org/forum.php?mod=redirect&goto=findpost&ptid=80318&pid=467217&fromuid=40353
\usepackage{xpatch}

\makeatletter
  \chardef\TPT@@@asteriskcatcode=\catcode`*
  \catcode`*=11
  \xpatchcmd{\threeparttable}
    {\TPT@hookin{tabular}}
    {\TPT@hookin{tabular}\TPT@hookin{tabu}}
    {}{}
  \catcode`*=\TPT@@@asteriskcatcode
\makeatother

%------------设置表格默认格式(字号、行间距、单元格高度)------------
\let\oldtabular\tabular
\renewcommand{\tabular}{%
  \renewcommand\baselinestretch{0.9}\small    % 设置行间距和字号
  \renewcommand\arraystretch{1.5}             % 调整单元格高度
  %\renewcommand\multirowsetup{\centering}
  \oldtabular
}
%设置行间距,且必须放在字号设置前 否则无效
%或者使用\fontsize{<size>}{<baseline>}\selectfont 同时设置字号和行间距

\let\oldtabu\tabu
\renewcommand{\tabu}{%
  \renewcommand\baselinestretch{0.9}\small    % 设置行间距和字号
  \renewcommand\arraystretch{1.8}             % 调整单元格高度
  %\renewcommand\multirowsetup{\centering}
  \oldtabu
}

%------------模仿booktabs宏包的三线宽度设置---------------
\def\toprule   {\Xhline{.08em}}
\def\midrule   {\Xhline{.05em}}
\def\bottomrule{\Xhline{.08em}}
%-------------------------------------
%\setlength{\arrayrulewidth}{2pt} 设定表格中所有边框的线宽为同样的值
%\Xhline{} \Xcline{}分别设定表格中水平线的宽度 makecell包提供

%表格中垂直线的宽度可以通过在表格导言区(preamble),利用命令 !{\vrule width1.2pt} 替换 | 即可

%=================图表设置===============================
%---------------图表标号设置-----------------------------
\renewcommand\thefigure{\arabic{section}-\arabic{figure}}
\renewcommand\thetable {\arabic{section}-\arabic{table}}

\usepackage{caption}
  \captionsetup{font=small,}
  \captionsetup[table] {labelfont=bf,textfont=bf,belowskip=3pt,aboveskip=0pt} %仅表格 top
  \captionsetup[figure]{belowskip=0pt,aboveskip=3pt}  %仅图片 below

%\setlength{\abovecaptionskip}{3pt}
%\setlength{\belowcaptionskip}{3pt} %图、表题目上下的间距
\setlength{\intextsep}   {5pt}  %浮动体和正文间的距离
\setlength{\textfloatsep}{5pt}

%====================全文水印==========================
%解决方案来自:
%http://bbs.ctex.org/forum.php?mod=redirect&goto=findpost&ptid=79190&pid=462496&fromuid=40353
%https://zhuanlan.zhihu.com/p/19734756?columnSlug=LaTeX
\usepackage{eso-pic}

%eso-pic中\AtPageCenter有点水平偏右
\renewcommand\AtPageCenter[1]{\parbox[b][\paperheight]{\paperwidth}{\vfill\centering#1\vfill}}

\newcommand{\watermark}[3]{%
  \AddToShipoutPictureBG{%
    \AtPageCenter{%
      \tikz\node[%
        overlay,
        text=red!50,
        %font=\sffamily\bfseries,
        rotate=#1,
        scale=#2
      ]{#3};
    }
  }
}

\newcommand{\watermarkoff}{\ClearShipoutPictureBG}

\watermark{45}{15}{草\ 稿}    %启用全文水印

%=============花括号分支结构图=========================
\usepackage{schemata}

\xpatchcmd{\schema}
  {1.44265ex}{-1ex}
  {}{}

\newcommand\SC[2] {\schema{\schemabox{#1}}{\schemabox{#2}}}
\newcommand\SCh[4]{\Schema{#1}{#2}{\schemabox{#3}}{\schemabox{#4}}}

%=======================================================

\begin{document}
\pagestyle{main2}
\fi
%中篇经文选读
\chapter{阴阳五行}%第一章

阴阳五行学说是古代哲学的重要内容,古代医家将其引用到医学领域,并与医学内容结合起来,成为《内经》理论体系的有机组成部分并贯穿于藏象经络、病因病机、诊治法则诸理论之中,既是中医研究和阐释人体规律的说理方法,又作为医学理论有效地指导临床诊断、治疗和防病保健。

《内经》关于阴阳五行学说的内容甚为丰富,既立《阴阳应象大论》、《金匮真言论》、《阴阳离合论》等专篇论述,又与其它理论密切结合而散见于各具体篇章之中,本章仅摘要介绍《阴阳应象大论》、《脏气法时论》、《六微旨大论》中的有关内容。

\section{素問·陰陽應象大論(節選)}%第一節

\biaoti{【原文】}

\begin{yuanwen}%[【原文】]
黃帝曰:陰陽者,天地之道\sb{1}也,萬物之綱紀\sb{2},變化之父母,生殺之本始\sb{3},神明之府\sb{4}也。治病必求於本。

故積陽爲天,積陰爲地。陰靜陽躁,陽生陰長,陽殺陰藏\sb{5},陽化氣,陰成形。寒極生熱,熱極生寒。寒氣生濁,熱氣生清。清氣在下,則生飧泄\sb{6};濁氣在上,則生䐜脹\sb{7}。此陰陽反作,病之逆從也。

故清陽爲天,濁陰爲地。地氣上爲雲,天氣下爲雨;雨出地氣,雲出天氣\sb{8}。故清陽出上竅,濁陰出下竅;清陽發腠理,濁陰走五藏;清陽實四肢,濁陰歸六府。
\end{yuanwen}

\biaoti{【校注】}

\begin{jiaozhu}
	\item 道:法则、规律。
	\item 纲纪:纲领。张介宾注:“大曰纲,小曰纪,总之为纲,周之为纪。”
	\item 生杀之本始:生,发生;杀,消亡。本始,即本原、由来。
	\item 神明之府:神明,指自然界生长变化万物的内在力量。《淮南子·泰族训》:“其生物也,莫见其所养而物长;其杀物也,莫见其所丧而物亡,此之请神明。”府,所在之处。
	\item 阳生阴长,阳杀阴藏:二句宜作互文理解。总而言之:阴阳既为万物生长之本,也为藏杀之本;分而言之:阳既主生发也主肃杀,阴既主长养也主藏敛。
	\item 飧泄:飧,水浇饭。《礼记正义·玉藻》孔疏:“飧,谓用饮浇饭于器中也。”飧泄,指腹泻而大便中挟有未消化食物。
	\item 䐜胀:䐜,胀满。䐜胀即指胸膈胀满。
	\item 雨出地气,云出天气:雨虽自天降,实由地阴之气吸引所致,故谓其“出地气”;云虽由地气上而成,实由天阳之气蒸腾而成,故谓其“出天气”。
\end{jiaozhu}

\biaoti{【理论阐释】}

1.治病必求于本

“治病必求于本”的“本”,在文中指阴阳而言。因为阴阳是“天地之道,万物之纲纪,变化之父母”,因此疾病发生和发展变化的根本原因也就在于阴阳的失调。如何“治病求本”?必须做到《素问·至真要大论》所言的“谨察阴阳所在而调之,以平为期”——诊断上要诊察阴阳的失调状况,而治疗则要重视纠正阴阳的盛衰偏颇,恢复和促进其平衡协调。

2.阳生阴长,阳杀阴藏

文中以天地、静躁、生长杀藏、化气成形、寒热、清浊等,说明阴阳是对具有对立统一关系的事物性质和功用的归纳。关于“阳生阴长,阳杀阴藏”,历代医家从不同角度加以阐释发挥:

(1)张志聪《黄帝内经素问集注》:“春夏者,天之阴阳也,故主阳生阴长;秋冬者,地之阴阳也,故主阳杀阴藏。”系从天地四时阴阳的生长收藏作用加以说明。

(2)张介宾《类经·阴阳类》:“此即四象之义。阳生阴长,言阳中之阴阳也;阳杀阴藏,言阴中之阴阳也。盖阳不独立,必得阴而后成,如发生赖于阳和,而长养由乎雨露,是阳生阴长也;阴不自专,必因阳而后行,如闭藏因于寒冽,而肃杀由乎风霜,是阳杀阴藏也。此于对待之中而复有互藏之道,所谓独阳不生,独阴不成也。”系根据《周易》“阴阳太少”理论,说明阴阳之间的相反相成、对立统一关系。

(3)李中梓云:“阳之和者为发育,阴之和者为成实,故曰阳生阴长,此阴阳之治也;阳之亢者为焦枯,阴之凝者为封闭,故曰阳杀阴藏,此阴阳之乱也。”系从阴阳之“治乱”,即正常与失常的角度加以说明。

以上各家的不同发挥,说明经文的要旨在于既指出阴阳的可分性——阴阳之中更有阴阳,又阐明了阴阳之间相辅相成的对立统一关系。

3.清阳浊阴在人体中的不同分布和走向

文中在论述天地阴阳升降与云雨化生的关系之后,说明清阳和浊阴在人体中的不同分布和走向。其中三对“清阳”、“浊阴”所指不同:

“清阳出上窍,浊阴出下窍”:指排泄、分泌物而言。出于上部头面的眼泪、痰涎、鼻涕以及呼出的气体等,较之出于下部二阴的大、小便等轻清,故为“清阳”,反之为“浊阴”。

“清阳发腠理,浊阴走五脏”:指阳气、阴精而言。阳气轻清,故发于体表腠理;阴精重浊,故归藏于体内五脏。

“清阳实四肢,浊阴归六腑”:指水谷及其代谢产物而言。饮食水谷所化生的精气充养四肢,以其性质轻清,故称清阳;而由六腑所受纳的饮食水谷及其消化吸收后所余糟粕废液,为重浊有形之物,故称浊阴。

由此可见阴阳为一相对概念,在不同范畴中所指不同。因此要分别事物的阴阳属性,必须规定其具体范畴,否则将无法作出正确划分。例如心,在“五脏”范畴中属阳,但在“脏腑”范畴中,则又属阴,若不规定其所处范畴,则不可能确定心为阴还是为阳。

\biaoti{【临证指要】}

“清气在下,则生飧泄;浊气在上,则生䐜胀”

此论对因阴阳升降失常而致的泄泻、痞胀等病证具有重要的临床指导意义。飧泄指完谷不化的一类泄泻,系由中气虚陷,清阳不升所致(后文“春伤于风,夏生飧泄”,也是由于春天阳气生发之际,为风邪所伤,至夏天不能盛长而虚陷不升)。清·尤怡《金匮翼》泄泻门谓:“飧泄,完谷不化也。脾胃气衰,不能腐熟水谷,而食物完出。经所谓脾病者,虚则腹满肠鸣,飧泄食不化是也。又清气在下,则生飧泄者,谓阳气虚则下陷也。”历代医家论治本证,多宗东垣升阳益气之法,用补中益气汤、升阳除湿汤之类,如清·王九峰治飧泄一案,即用此法:“清气在下,则生飧泄;浊气在上,则生䐜胀。肝脉循于两胁,肝实胁胀;脾虚腹满,本乘土位;食少运迟,营卫不和。补中益气,是其法程,更兼以涩固胃关之品,冀效。洋参、茯苓、冬术、炙草、川连、升麻、柴胡、归身、木香、陈皮、山药、补骨脂、肉豆蔻。”(《王九峰医案·泄泻》)

关于䐜胀,则包括痞证、鼓胀等一类病证,其病机关键在于浊阴不降,但浊阴之不降也每与清阳不升有关,故《金匮翼》胀满门谓:“䐜胀,即气胀,胸膈胀满也。经云:浊气在上,则生䐜胀是也,宜升清降浊,盖清不升则浊不降也。……东垣云:浊阴本归六腑而出下窍,今在上,是浊气反行清道,气乱于中,则胀作矣。”《名医类案·痞满》载:“东垣治一贵妇,八月中,先因劳役饮食失节,加之忧思,病痞结,心腹胀满,旦食不能暮食,两胁刺痛,诊其脉弦而细,至夜,浊阴之气当降而不降,䐜胀尤甚。大抵阳主运化,饮食劳倦损伤脾胃,阳气不能运化精微,聚而不散,故为胀满,先灸中脘,乃胃之募穴,引胃中生发之气上行阳道,又以木香顺气助之,使浊阴之气自此而降矣。”

\biaoti{【原文】}

\begin{yuanwen}
水爲陰,火爲陽。陽爲氣,陰爲味。味歸形,形歸氣,氣歸精,精歸化\sb{1};精食氣,形食味\sb{2},化生精,氣生形\sb{3}。昧傷形,氣傷精\sb{4};精化爲氣,氣傷於味\sb{5}。陰味出下竅,陽氣出上竅。味厚者爲陰,薄爲陰之陽;氣厚者爲陽,薄爲陽之陰。味厚則泄,薄則通;氣薄則發泄,厚則發熟\sb{6}。壯火之氣衰,少火之氣壯\sb{7};壯火食氣,氣食少火\sb{8};壯火散氣,少火生氣。氣味辛甘發散爲陽,酸苦涌泄爲陰。
\end{yuanwen}

\biaoti{【校注】}

\begin{jiaozhu}
	\item 味归形,形归气,气归精,精归化:归,有二义,一为生成、充养;一为被生成、被充养。气,也有二义,一为人体之气,一为饮食物之气(食气)。“味归形,形归气”,谓饮食物之五味充养形体,而形体也仰赖人身元气的充养。“气归精,精归化”,谓饮食物之五气充养人体精气,而饮食五气又须经过化生作用才能转变为人体之精气。
	\item 精食气,形食味:食,读为“饲”,即“赖……所滋养”。该二句系对“味归形”、“气归精”的进一步说明。
	\item 化生精,气生形:系对前“形归气”、“精归化”的进一步说明。
	\item 味伤形,气伤精:饮食五味虽能充养形身,但五味太过反能伤害形身;饮食五气虽能化生精气,但五气太过也能伤害精气。
	\item 精化为气,气伤于味:气,指人身之气。人体之精能够化生气,气也可因饮食五味偏嗜而受伤。
	\item 味厚则泄,薄则通;气薄则发泄,厚则发热:系对饮食药物性味功能的概括。吴昆注:“阴气润下,故味厚则泄利,薄则通利;阳气炎上,故气薄则发散,厚则发热。”
	\item 壮火之气衰,少火之气壮:壮火、少火,历代有两种不同解释:一谓壮火指饮食药物之气味辛热纯阳者,少火指饮食药物之气味辛甘温和者;另一种解释认为火指阳气,壮火即过亢之阳气,少火则指温和而不亢旺之阳气。气,指人身之正气。气衰、气壮,作使动用法理解。
	\item 壮火食气,气食少火:食,一读若“蚀”,销蚀、耗伤,“壮火食气”谓亢旺之火能耗蚀正气;一读若“饲”,即饲养,“气食少火”谓正气赖阳和之火所温养。
\end{jiaozhu}


\biaoti{【理论阐释】}

1.气、味、形、精的阴阳属性及其化生关系

“味归形……气伤于味”一节,说明了气、味、形、精之间的化生关系,其中气有两义;一指人身之正气,如“形归气”、“气生形”、“精化为气,气伤于味”等。另一指食物和药物中作用于人体的成分,如“气归精”、“精食气”、“气伤精”等。必须分清其不同所指,才能正确理解本段经文的意义。

把食物和药物中作用于人体的成分分成“气”和“味”两类,“阳为气,阴为味”。人身的形、精、气三者,精、气相对而言,精为阴,气为阳;但精相对于形而言,则精为阳而形为阴。按同气相求的原则,故有“味归形”、“气归精”、“精食气,形食味”之说。精虽由药食之气所生,但药食之气在体内需经过一定的化生过程才能转化为人体之精,故又谓“精归化”、“化生精”。这是从阴阳角度说明药食气味与人体形精同气相求的滋养关系。另一方面形体虽赖阴味以生长,也靠人体之气所充养,即所谓“形归气”、“气生形”,而气则由精所化生(“精化为气”),因此人身中形、精、气三者也因其阴阳属性的不同,存在着精化生气,气充养形的生理关系。

药食气味既能充养人体的形体精气,但过度摄入则反能伤害形体精气,故经文又有“味伤形,气伤精”、“气伤于味”之说,这与《素问·至真要大论》“五味入胃……久而增气,物化之常也,气增而久,夭之由也”同样说明药食气味对人体健康的正反二重作用,也是中医“生病起于过用”这一观点的反映。

2.关于“壮火”、“少火”的含义

历代对“壮火”、“少火”,有两种不同解释:一种根据该节经文包含于讨论药食气味阴阳属性的内容之中,认为壮火是指药食中气厚性烈之品的性能作用,少火则指药食气味温和者的性能作用,如马莳《黄帝内经素问注证发微》谓:“气味太厚者,火之壮也。用壮火之品,则吾人之气不能当之而反衰矣,如用乌附之类,而吾人之气不能胜之,故发热。气味之温者,火之少也。用少火之品,则吾人之气渐尔生旺,而益壮矣,如用参归之类,而气血渐旺者是也。”另一种解释认为少火为正常生理之火,指人体阳气的阳和、温煦作用,壮火则指亢烈为害之火,为阳气亢旺而化的火邪。阳和之火能够温养阳气,亢旺之邪火则不仅耗伤阴精,且也损蚀阳气。如张介宾《类经·阴阳类》谓:“火,天地之阳气也。天非此火,不能生物;人非此火,不能有生。故万物之生,皆由阳气。但阳和之火则生物,亢烈之火反害物,故火太过则气反衰,火和平则气乃壮。壮火散气,故云食气,犹言火食此气也;少火生气,故云食火,犹言气食此火也。此虽承气味而言,然造化之道,少则壮,壮则衰,自是如此,不特专言气味者。”后一种解释从生理、病理角度阐释气、火关系,对中医学术理论做出进一步发挥。李东垣所言“相火元气之贼”,其“相火”即指壮火而言,而朱丹溪的“气有余便是火”,也是对《内经》这一气火关系理论的发挥。

\biaoti{【临证指要】}

\xiaobt{药食气味性能的指导意义}

经文从阴阳角度说明药食气味厚薄及其功效性能,对临床选药组方具有重要指导意义。金元医家张元素在《医学启源》中列有“用药法象”一节,将药物分为风升生、热浮长、湿化成、燥降收、寒沉藏五类,指出:“气之厚者,阳中之阳,气厚则发热,辛甘温热是也。”“味之厚者,阴中之阴,味厚则泄,酸苦咸寒是也。”系对《内经》这一理论的进一步发挥,对后世有颇大影响,张仲景在《伤寒杂病论》中创制的经方,如桂枝汤类之辛甘发散、承气汤类之味厚则泄、乌头汤之气厚则发热、猪苓汤之味薄则通等,也是秉这一理论立法,而为历代所师承。叶天士治胸中清阳不运,痰气凝阻之胸痹证,每宗仲景栝萎薤白半夏汤、枳实蕹白桂枝汤方意,重用桂枝、薤白、生姜(或干姜)等辛甘发散之品以温通阳气。如:“浦,中阳困顿,浊阴凝沍,胃痛彻背,午后为甚,即不嗜食,亦是阳伤,温通阳气,在所必施。薤白三钱、半夏三钱、茯苓五钱、干姜一钱、桂枝五分。”“王,胸前附骨板痛,甚至呼吸不通,必捶背稍缓,病来迅速,莫晓其因,议从仲景胸痹症,乃清阳失展,主以辛滑。薤白、川桂枝尖、半夏、生姜,加白酒一杯同煎。”《临证指南医案·胸痹》)可谓是临床上运用“辛甘发散为阳”这一理论以立法制方的范例。

\biaoti{【原文】}

\begin{yuanwen}
陰勝則陽病,陽勝則陰病。陽勝則熱,陰勝則寒。重寒則熱,重熱則寒。

寒傷形,熱傷氣,氣傷痛,形傷腫\sb{1}。故先痛而後腫者,氣傷形也;先腫而後痛者,形傷氣也。

風勝則動,熱勝則腫,燥勝則乾,寒勝則浮\sb{2},濕勝則濡瀉\sb{3}。

天有四時五行,以生長收藏,以生寒暑燥濕風。人有五藏化五氣,以生喜怒悲\sb{4}憂恐。故喜怒傷氣,寒暑傷形\sb{5};暴怒傷陰,暴喜傷陽\sb{6}。厥氣上行,滿脈去形\sb{7}。喜怒不節,寒暑過度,生乃不固。故重陰必陽,重陽必陰。故曰:冬傷於寒,春必溫病\sb{8};春傷于風,夏生飧泄;夏傷於暑,秋必痎瘧\sb{9};秋傷於濕,冬生欬嗽。
\end{yuanwen}

\biaoti{【校注】}

\begin{jiaozhu}
	\item 寒伤形,热伤气,气伤痛,形伤肿:形,指形体;气,指气机。寒为阴邪,故伤人形体;热为阳邪,故伤人气分,扰乱气机。李中梓《内经知要》:“气喜宣通,气伤则壅闭而不通,故痛;形为质象,形伤则稽留而不化,故肿。”
	\item 寒胜则浮;浮,浮肿。张介宾注:“寒胜者阳气不行,为胀满虚浮之病。”
	\item 湿胜则濡泻;濡泻,指泄泻而大便溏薄,由于湿胜伤脾而致。
	\item 悲:《新校正》:“按《天元纪大论》,‘悲’作‘思’。”
	\item 喜怒伤气,寒暑伤形:喜怒,概指七情,七情过激,伤五脏气机,故云“伤气”;寒暑,概指六淫,六淫袭人,先伤肌表形身,故云“伤形”。
	\item 暴怒伤阴,暴喜伤阳:张介宾注:“气为阳,血为阴;肝藏血,心藏神。暴怒则肝气逆而血乱,故伤阴;暴喜则心气缓而神逸,故伤阳。”
	\item 厥气上行,满脉去形:王冰注:“厥,气逆也,逆气上行,满于经洛,则神气浮越,去离形骸矣。”
	\item 春必温病:明·熊宗立种德堂刊本、道藏本均作“春必病温”。胡澍《内经素问校义》:“春必温病,于文不顺,写者误倒也。……《金匮真言论》曰‘故藏于精者,春不病温’、《玉版论要》曰‘病温虚甚死’、《平人气象论》曰‘尺热曰病温’、《热论》曰‘先夏至日者为病温’、《评热病论》曰‘有病温者,汗出辄复热’,皆作‘病温’。”
	\item 痎疟:痎疟,泛指各种疟疾。
\end{jiaozhu}

\biaoti{【理论阐释】}

1.“阴胜则阳病,阳胜则阴病”与“重阴必阳,重阳必阴”

关于“阴胜则阳病,阳胜则阴病”,可以从三个层次理解其意义:①联系上段药食气味阴阳性能之由,此句系指过用酸苦涌泄等阴性药食,可能损害人体阳气;反之,过用辛甘发散等阳性药食,则可能损伤人体阴精。即马莳《黄帝内经素问注证发微》所言:“用酸苦涌泄之品至于太过,则阴胜矣,阴胜则吾人之阳分不能敌阴品,而阳分斯病也;用辛甘发散之品至于太过,则阳胜矣,阳胜则吾人之阴分不能敌阳品,而阴分斯病也。”②联系下文寒热病机之论,则此句指阴阳偏胜偏衰的病机。张介宾《类经》注:“此下言阴阳偏胜之为病也。阴阳不和,则有胜有亏,故皆能为病。”李中梓《内经知要》也谓:“阴阳和则得其平,一至有偏胜,病斯作矣。”均从病机角度说明阴阳之间一方偏胜则致另一方受损,从而失去平衡协调而致病。③推而广之,从哲学角度来说,本句经文又说明了阴阳之间的对立斗争和消长关系:阴阳作为统一体的两个对立面,阴长则阳消,阳长则阴消,两者互相斗争而互为消长。上述三种解释从药食性味、疾病机理以至阴阳法则等不同角度阐释了《内经》这一理论,是对中医阴阳学说的进一步发挥和完善。

“重阴必阳,重阳必阴”既从哲学角度说明阴阳之间在一定条件下可以向对立而转化,也从病机角度说明阴阳寒热的转化病机。但应注意,与“寒极生热,热极生寒”、“重寒则热,重热则寒”同样,阴阳寒热只有在一定条件下才能向对立面转化,“重”、“极”就是转化的条件。如果没有转化的条件或条件尚不成熟,则阴阳之间不能向对立面转化。

2.“五气”偏胜致病

“风胜则动,热胜则肿,燥胜则干,寒胜则浮,湿胜则濡泻”一节,说明人体在致病因素作用下出现的病变情况。其中风、热、燥、寒、湿是致病因素作用于人体后出现的病理变化,即后世所称为“化风”、“化热”、“化燥”、“化寒”、“化湿”者,其病因虽然与相同名称的风、寒、热等六淫邪气有关,但没有必然、对应的直接关系,如“风胜则动”不一定与感受风邪有关,更多的是肝阳化风、血虚生风、热极生风等所致。同样,“寒胜则浮”也不一定因于感受寒邪,更多的是由阳虚里寒所致,即使是感受寒邪,也必须经过寒邪伤人阳气,阳虚引致里寒,才出现“寒胜则浮”的病变。只有把本节所言的“五气”,视为人体内部产生的病理变化或病变类型,而不简单理解为六淫邪气,才能准确把握其病变机理。

\biaoti{【临证指要】}

“冬伤于寒,春必温病”之说,是后世“伏气温病”学说的理论根据,但也引起了两种不同观点的对立和争议。部分医家根据一些温病(如春温)发病即见邪热炽盛于营血分的里热证候,不同于风温等之发病始于卫分表证,故以《内经》之论说明其发病机理,认为是冬伤于寒而不即病,寒邪内藏,郁而化热,至春阳气发动而热发于外。至于感寒之后,邪气藏匿之处,王叔和《伤寒例》谓:“中而即病者,名曰伤寒,不即病者,寒毒藏于肌肤,至春变为温病,至夏变为暑病。”一些《内经》注家,如王冰、吴昆等,也宗叔和之说。另一些医家,如叶天士,则认为寒邪藏于少阴,如《临证指南医案·幼科要略》:“春温一证,由冬令收藏未固,昔人以冬寒内伏,藏于少阴,人春发于少阳。”但另一派医家,则否定“寒邪内藏,至春而发”之说。如吴又可《温疫论·伤寒例正误》即针对王叔和、成无己之说指出:“风寒暑湿之邪,与吾身之营卫,势不两立,一有所中,疾苦作矣,苟或不除,不危即毙。……今冬时严寒所伤,非细事也,反能伏藏过时而发耶?”陈平伯《外感温病篇》也说:“即春必温病之语,亦是就近指点,总见里虚者表不固,一切时邪皆易感受。”而张琦更认为“冬伤于寒,春必温病”是指伤于寒之后,肾精失藏,相火妄动,内热郁积而致,其于《素问释义·生气通天论篇》注曰:“冬主蛰藏,气应乎肾。《金匮真言论》:‘藏于精者,春不病温。’此伤于寒者,即冬不藏精之义也。严寒封蛰之时,相火燔腾,反行炎赫之令,内热郁积,一交春气,木火司权,又遇风露闭其皮毛,内热莫宣,遂成温病,以其火盛精枯,故内外皆热。……非如王叔和冬时感寒不病,至春变为温病之说也。”上述两种不同观点的争鸣,促进了温病学说的学术发展及辨证论治法则的完善。

关于“冬伤于寒,春必温病”,本篇与《生气通天论》凡两见,究其原意,本篇系承“重阴必阳,重阳必阴”而说明之,吴昆于《素问吴注》谓:“秋冬,时之阴也,寒湿,气之阴也。冬伤寒,秋伤湿,谓之重阴。冬伤寒而病温,秋伤湿而咳嗽,重阴而变阳证也。”《生气通天论》中则承接“阴平阳秘,精神乃治;阴阳离决,精气乃绝,因于露风,乃生寒热”而言,故其要旨均在于说明疾病的发生与阴阳失调密切相关,提示我们治病必须“谨察阴阳所在而调之”,即审察疾病的阴阳失调病机,并采取协调阴阳的措施以治疗疾病,而养生则需顺应自然界四时阴阳变化,外避虚邪贼风,内养正气以避免阴阳失调而致病。

\biaoti{【原文】}

\begin{yuanwen}
帝曰:余聞上古聖人,論理人形,列別\sb{1}藏府,端絡\sb{2}經脈,會通六合\sb{3},各從其經;氣穴所發,各有處名;谿谷屬骨\sb{4},皆有所起;分部逆從\sb{5},各有條理;四時陰陽,盡有經紀\sb{6}。外内之應,皆有表裏,其信然乎\sb{7}?

岐伯對曰:東方生風\sb{8},風生木\sb{9},木生酸\sb{10},酸生肝,肝生筋,筋生心,肝主目。其在天爲玄,在人爲道,在地爲化;化生五味,道生智,玄生神\sb{11}。神在天爲風,在地爲木,在體爲筋,在藏爲肝,在色爲蒼\sb{12},在音爲角\sb{13},在聲爲呼\sb{14},在變動爲握\sb{15},在竅爲目,在味爲酸,在志爲怒。怒傷肝,悲勝怒;風傷筋,燥勝風;酸傷筋,辛勝酸。

南方生熱,熱生火,火生苦,苦生心,心生血,血生脾,心主舌。其在天爲熱,在地爲火,在體爲脈,在藏爲心,在色爲赤,在音爲徵;在聲爲笑,在變動爲憂\sb{16},在竅爲舌,在味爲苦,在志爲喜。喜傷心,恐勝喜;熱傷氣,寒勝熱;苦傷氣,鹹勝苦。

中央生濕,濕生土,土生甘,甘生脾,脾生肉,肉生肺,脾主口。其在天爲濕,在地爲土,在體爲肉,在藏爲脾,在色爲黃,在音爲宫,在聲爲歌,在變動爲噦,在竅爲口,在味爲甘,在志爲思。思傷脾,怒勝思;濕傷肉,風勝濕;甘傷肉,酸勝甘。

西方生燥,燥生金,金生辛,辛生肺,肺生皮毛,皮毛生腎,肺主鼻。其在天爲燥,在地爲金,在體爲皮毛,在藏爲肺,在色爲白,在音爲商,在聲爲哭,在變動爲欬,在竅爲鼻,在味爲辛,在志爲憂。憂傷肺,喜勝憂;熱傷皮毛,寒勝熱;辛傷皮毛,苦勝辛。

北方生寒,寒生水,水生鹹,醎生腎,腎生骨髓,髓生肝,腎主耳。其在天爲寒,在地爲水,在體爲骨,在藏爲腎,在色爲黑,在音爲羽,在聲爲呻,在變動爲慄,在竅爲耳,在味爲鹹,在志爲恐。恐傷腎,思勝恐;寒傷血,燥勝寒\sb{17};咸傷血,甘勝鹹。

故曰:天地者,萬物之上下也;陰陽者,血氣之男女\sb{18}也;左右者,陰陽之道路也\sb{19};水火者,陰陽之徵兆也;陰陽者,萬物之能始\sb{20}也。故曰:陰在內,陽之守也;陽在外,陰之使也\sb{21}。
\end{yuanwen}

\biaoti{【校注】}

\begin{jiaozhu}
	\item 列别:罗列辨别,即比较、分辨的意思。
	\item 端络:端,开头;络,分支,网络。端络,此处联用作动词,即推求经脉的起始及循行分布。
	\item 六合;指十二经脉中阴阳表里两经相配成为六对组合。
	\item 谿谷属骨:《素问·气穴论》:“肉之大会为谷,肉之小会为谿。”谿谷,指大小分肉。属骨:指与谿谷栢连属的骨节。
	\item 分部逆从:张志聪《黄帝内经素问集注》:“分部者,皮之分部也。皮部中之浮络,分三阴三阳,有顺有逆,各有条理也。”
	\item 经纪:经,经纬;纪,纲纪。指四时阴阳变化的规律。
	\item 其信然乎:从“帝曰”开始至此所提出的问题,下面岐伯答文仅论及四时五行等个别方面,问答不符。张琦《素问释义》认为:“问辞与下岐伯对文义不合,他经错简也。”
	\item 东方生风:东、西、南、北、中,称“五方”,风、热、湿、燥、寒,为在天之“五气”,两者均配属五行并与自然界春、夏、长夏、秋、冬相应,以我国地理气候环境而言,东方滨海,春季多风,且东风化雨,生发万物,故谓“东方生风”。下文“南方生火”、“中央生湿”、“西方生燥”、“北方生寒”也分别指南方和夏季气候多热、中央(黄河中游平原地区)和长夏气候潮湿、西方和秋季气候干燥、北方和冬季气候寒冷。
	\item 风生木:风热湿燥寒为在天之五气,木火土金水为在地之五行,在天之五气化生在地之五行,即《素问·天元纪大论》所言之“在天为气,在地成形”。以自然现象而言,风动则木荣,热极则火生,湿润则土气旺而化生万物,干燥则物具金属刚劲之性,气为寒凝则化为木,故有风生木及下文热生火、湿生土、燥生金、寒生水之说。
	\item 木生酸:按五行学说,酸、苦、甘、辛、咸五味也为五行所化生。《尚书·洪范》:“水曰润下,火曰炎上,木曰曲直,金曰从革,土爰稼穑。润下作咸,炎上作苦,曲直作酸,从革作辛,稼穑作甘。”这是古代根据实物滋味总结出来的五行与五味化生关系,孔颖达疏谓:“水性本甘,久浸其地,变而为卤,卤味乃咸。”“火性炎上,焚物则焦,焦是苦气。”“木生子实,其味多酸,五果之味虽殊,其为酸一也,是木质之性然也。”“金之在火,别有腥气,非苦非酸,其味近辛,故辛为金之气味。”“甘味生于百谷,谷是土之所生,故甘为土之味也。”
	\item 其在天为玄,……玄生神:其,指阴阳变化;玄,指自然界幽微深远的生化力量;道,规律,此处指人的生命活动规律;化,指大地化生万物的作用;神,指自然界阴阳不测的神妙变化。又,此数句列于“东方生风”之下,与其下“南方”、“中央”、“西方”、“北方”文例不同,对此,前人有两种不同见解:一种认为系概括五方而言,非独指东方,如张介宾谓:“此盖通举五行六气之大法,非独指东方也。”一种认为属衍文,如丹波元简谓:“据下文例,‘在天’以下二十三字,系衍文,且与肝脏不相属,宜删之。”
	\item 在色为苍:苍,即青,为木之色。下文赤、黄、白、黑分别为火、土、金、水之色。
	\item 在音为角:角、徵、宫、商、羽为古代五音,也配属五行:角音应木气而展放,徵音应火气而高亢,宫音应土气而平稳,商音应金气而内收,羽音应水气而下降。
	\item 在声为呼:呼、笑、歌、哭、呻为五声,五声发自五脏,为五脏情志活动的外在表现:肝在志为怒,怒则呼;心在志为喜,喜则笑;脾在志为思,思而有得则歌;肺在志为悲,悲则哭;肾在志为恐,恐则气下,声欲呻而出之。
	\item 在变动为握:握、憂(即“嚘”,见注\myjzref{16})哕、咳、慄称为“五变”,为五脏病变所表现的五种病证。握,搐搦握拳,为肝所主的筋的病变表现;嚘,气逆声嘶,为心火上炎的病变表现;哕,干呕,为与脾相表里的胃的病变表现;咳,为肺气上逆的病变表现;慄,战栗,为肾阳虚衰,失子温煦的病变表现。
	\item 在变动为憂:于鬯《香草续校书》:“此憂字盖当读为嚘。心之变动为嚘,与下文言肺之志为忧者不同。忧既为肺志,自不应复为心之变动也。”嚘,《玉篇·口部》引《老子》“终日号而不嚘”句,训为“气逆”,为气逆声嘶哑之义。又释:按《说文解字·口部》:“嚘,语未定貌,从口憂声。”则嚘为言语吞吐反复不定,盖“心主言”,心神不宁则言语反复不清。
	\item 寒伤血,燥胜寒:丹波元筒注:“据《太素》,血作骨,燥作湿,为是,张云:‘若以五行正序,当云湿胜寒,但寒湿同类,不能相胜,故曰燥胜寒也。”
	\item 血气之男女:之,作连词用,义同“与”、“和”。
	\item 左右者,阴阳之道路也:天为阳,左行;地为阴,右行。阳从左升,阴从右降,故谓左右为阴阳之道路。
	\item 万物之能始:能,“胎”之通假字。能始,即胎始、本始。孙诒让《礼迻》:“‘能’者,‘胎’之借字。《尔雅·释诂》云:‘胎,始也。’《释文》:‘胎,本或作台。’《史记·天官书》‘三能’即‘三台’,是胎、台、能古字并通用。”
	\item 阴在内,阳之守也;阳在外,阴之使也:守,镇守子内;使,役使于外。吴昆《素问吴注》:“阴静故为阳之镇守,阳动故为阴之役使,见阴阳相为内外,不可相离也。”说明阴阳之间的相反相成、互根互用关系。
\end{jiaozhu}

\biaoti{【理论阐释】}

1.五行的取象比类

五行学说在医学上的重要运用之一,就是以五行为中介,根据其物象特征,将人体与自然界进行归类联系,从而构建了以五脏为中心的内外相应的系统结构。按本篇所述,这一系统结构的内容如下表。(表\ref{tab:人体内外相应的系统结构表})

\begin{table}[htb]%人体内外相应的系统结构表
	\centering
	\caption{人体内外相应的系统结构表}\label{tab:人体内外相应的系统结构表}
	\begin{tabu}to.87\textwidth{*{11}{X[c]|}X[c]}
		\toprule
			 & \multicolumn{5}{c|}{自然界}      & \multicolumn{6}{c}{人体}                \\ \hline
		五行 & 方位 & 气候 & 五味 & 五色 & 五音 & 五脏 & 七窍 & 五体 & 五声 & 五志 & 病变 \\
		\midrule
		木   & 东   & 风   & 酸   & 青   & 角   & 肝   & 目   & 筋   & 呼   & 怒   & 提   \\
		火   & 南   & 热   & 苦   & 赤   & 徵   & 心   & 舌   & 脉   & 笑   & 喜   & 嚘   \\
		土   & 中   & 湿   & 甘   & 黄   & 宫   & 脾   & 口   & 肉   & 歌   & 思   & 哕   \\
		金   & 西   & 燥   & 辛   & 白   & 商   & 肺   & 鼻   & 皮毛 & 哭   & 忧   & 咳   \\
		水   & 北   & 寒   & 咸   & 黑   & 羽   & 肾   & 耳   & 骨   & 呻   & 恐   & 慄   \\
		\bottomrule
	\end{tabu}
\end{table}
该系统结构既是中医藏象学说的核心内容,又体现了人与天地相参应、人体表里相通应的整体观念。

2.五行的相生相胜

五行之间具有互相资生、互相制约的关系,这种关系本篇称为相生、相胜,后世则称为相生、相克。

相生,在本篇指两种情况:一指五行之间的相生,如筋(肝、木)生心(火),血(心、火)生脾(土),肉(脾、土)生肺(金),皮毛(肺、金)生肾(水),髓(肾、水)生肝(木)等;另一则指同行之间的相生,实际指同行之间的衍生、归类关系,如东方生风,风生木,木生酸,酸生肝,肝生筋,以及在天为风,在地为木,在体为筋,在脏为肝,在色为苍,在音为角,在声为呼,在变动为握,在窍为目,在味为酸,在志为怒等。

相克,又称“相胜”,指五行之间的互相制约关系,如悲胜怒、燥胜风、辛胜酸,恐胜喜、寒胜热、咸胜苦等。

五行相生相克是事物之间的正常资生制约关系,包括人体在内的任何事物,都必须既互相资生、互相促进,又互相克制、互相约束,才能保持正常的稳定状态,如果五行之间失去正常的相生相克关系,则其稳定状态将受到破坏而出现相乘、相侮的失常状态。中医运用五行学说,除了用其归纳、类比人体生理、病理及疾病诊断之外,也常运用其生克乘侮法则说明人体以五脏为中心的正常生理活动和异常病理变化,并提出防治疾病的法则。

3.阴阳的互根互用

原文在以天地、上下、血气、男女、左右、水火等征象说明阴阳的对立统一关系之后,更进一步指出:“阴在内,阳之守也;阳在外,阴之使也。”强调了阴阳之间的互根互用关系。阴阳代表事物中互相对立的两方面,它们之间既互相斗争、互为消长,又互相依存、互相为用。对于人体来说,阴精和阳气的互相依存、互相为用是正常生命活动的保证,故《素问·生气通天论》有“阴者藏精而起亟也,阳者卫外而为固也”之说,
一旦阴阳之间失去互根互用的协调关系,则“阴阳离决,精气乃绝”,生命也将解体。

后世医家对阴阳之间互根互用关系的论述颇多,如赵献可《医贯·阴阳论》:“阴阳又各为其根,阳根于阴,阴根于阳,无阳则阴无以生,无阴则阳无以化。”张介宾《类经附翼·真阴论》亦指出:“阴不可以无阳,非气无以生形也;阳不可以无阴,非形无以载气也。故物之生也生于阳,物之成也成于阴。”均强调说明阴阳各以对方为自己存在的基础,并以对方为自己发挥功能作用的前提。而张介宾《景岳全书·新方八阵》中提出的“善补阳者,必于阴中求阳,则阳得阴助而生化无穷;善补阴者,必于阳中求阴、则阴得阳升而泉源不竭”,则是对这一理论临床运用的深刻阐发。

\biaoti{【临证指要】}

1.五志过激致病

怒喜思忧恐称“五志”,是人体对外界刺激的反应。正常情志既是人体与外界交流的情神活动,也能够调节脏腑气机,调畅气血运行。但情志过激,超出正常承受能力,则能导致五脏气机紊乱,所藏精气神受伤,而产生疾病,因此把七情内伤与外感六淫等同列为致病的主要原因。经文所言“怒伤肝”、“喜伤心”、“思伤脾”、“忧伤肺”、“恐伤肾”,既是临床认识情志过激致病病机的理论基础,也为从调理五脏精气入手治疗这些疾病提供指导。下述病例即根椐“恐伤肾”这一理论辨证立法而取效:某41岁男子,因见同事中有数人患尿毒症,遂恐惧自己有肾炎,虽经检査证明无病,但恐惧心理不除,大便每日3$\sim$4次,小便每昼7$\sim$8次,夜3$\sim$4次,腰酸痛,睡眠不安,乏力,舌红苔薄白腻,脉弦细数。治以益肾清热祛湿之法,方用女贞子l0g、玄参12g、生地黄10g、生苡仁15g、猪苓15g、生龙骨30g、木通6g、枳壳10g、琥珀粉lg(冲),两剂。并告之以病极轻微,调摄自愈,以解除其精神负担。二日后再诊,病情大减,再用上方加减:女贞子10g、旱莲草10g、玄参12g、川断15g、菟丝子10g、生地黄12g、生龙骨20g、萆薢12g、丹参12g,三剂。病愈。(王洪图、詹海洪《黄帝医术临征切要》)

2.以情胜情治疗方法的运用

经文中“悲胜怒”、“恐胜喜”、“怒胜思”、“喜胜优”、“思胜恐”这一情志相胜的理论为情志过激疾病的治疗提供重要启示。据《吕氏春秋·至忠》篇记载,早在战国时期,文挚就已采用故意激怒病人的方法治愈齐湣王的病。《三国志·华佗传》也载华佗曾用激怒法治愈一郡守之病。金元以降,医家运用情志相胜法治疗一些因情志过激而致的疑难病证,更具丰富经验。张从正不仅在《儒门事亲·九气感疾更相为治衍二十六》指出:“悲可以治怒,以枪恻苦楚之言感之;喜可以治悲,以谑浪亵狎之言娱之;恐可以治喜,以恐惧死亡之言怖之;怒可以治思,以污辱欺罔之言触之;思可以治恐,以虑彼志此之言夺之。凡此五者,必诡诈谲怪,无所不至,然后可以动人耳目,易人视听。若胸中无材器之人,亦不能用此五法也。”同时也留下众多精彩验案,兹举二例于下:

例一:一富家妇,伤思虑过甚,二年不寐,无药可疗,其夫求戴人治之。戴人曰:“两手脉俱缓,此脾受之也,脾主思故也。”乃与其夫以怒而激之,多取其财,饮酒数日,不处一法而去,其人大怒汗出,是夜困眠,如此者八九日不寤,自是而食进,脉得其平。(《懦门事亲·不寐一百二》)

例二:(有)庄先生者,治以喜乐之极而病者。庄切其脉为之失声,佯曰:“吾取药去。”数日更不来。病者悲泣,辞其亲友曰:“吾不久矣。”庄知其将愈,慰之。诘其故,庄引《素问》曰:“惧胜喜。”(《儒门事亲·九气感疾更相为治衍二十六》)

3.阴阳互根互用理论的临床运用

经文“阴在内,阳之守也;阳在外,阴之使也”,说明了阴阳之间的互根互用关系。临床上除了运用其互相化生关系而“善补阳者,必于阴中求阳”、“善补阴者,必于阳中求阴”之外,对于阳虚而阴盛、阴虚而阳亢之证尚有益火消阴、滋阴潜阳之法,即王冰所谓“壮水之主,以制阳光:益火之源,以消阴翳”者,也是下文“阳病治阴,阴病治阳”具体方法之一,运用得当,收效甚良。兹举薛己治韩州同一案为例:“韩州同年四十六,仲夏色欲过度,烦热作渴,饮水不绝,小便淋沥,大便秘结,唾痰如涌,而目俱赤,满舌生刺,两唇燥裂,遍身发热,或时身如芒刺而无定处,两足心如火烙,以冰折之作痛,脉洪而无伦。此肾阴虚,阳无所附而发于外,非火也。盖大热而甚,寒之不寒,是无水也,当唆补其阴。遂以加减八味丸料一斤,内肉桂一两,以水顿煎六碗,冰水浸冷与饮,半晌,已用大半,睡觉而食温粥一碗,复睡至晚。乃与前药温饮一碗,食热粥二碗,乃睡至晓,诸症悉退。翌日,畏寒足冷至膝诸症仍至,或以为伤寒,薛曰:非也,大寒而甚,热之不热,是无火也,阳气也虚矣。急以八味一剂服之,稍缓。四剂,诸症复退。大便至十三日不通,以猪胆导之,诸症复作,急用十全大补,方应。”(《名医类案?火热门》)本证阴虚不能内守,致虚阳外越而现假热之象,薛氏以加减八味丸壮水之主,并加肉桂引火归原,更用热药冷服之反佐法以防格拒,故初战告捷,继而阴虽复而阳虚仍甚,故又用益火消阴之法以收功,前后治法虽异,但均以阴阳互根互用理论为指导,故效果卓著。

\biaoti{【原文】}

\begin{yuanwen}
帝曰:法陰陽奈何?岐伯曰:陽勝則身熱,腠理閉,喘麤爲之俛仰\sb{1},汗不出而熱,齒乾以煩冤\sb{2},腹滿,死,能冬不能夏\sb{3};陰勝則身寒,汗出身常清,數慄而寒,寒則厥,厥則腹滿,死,能夏不能冬。此陰陽更勝之變,病之形能\sb{4}也。

帝曰:調此二者奈何?岐伯曰:能知七損八益\sb{5},則二者可調;不知用此,則早衰之節\sb{6}也。年四十,而陰氣自半也,起居衰矣;年五十,體重,耳目不聦明矣;年六十,陰痿\sb{7},氣大衰,九竅不利,下虛上實,涕泣倶出矣。故曰:知之則強,不知則老,故同出而名異\sb{8}耳。智者察同,愚者察異\sb{9}。愚者不足,智者有餘。有餘則耳目聰明,身體輕強,老者復壯,壯者益治。是以聖人爲無爲之事,樂恬儋之能,從欲快志於虛無之守\sb{10},故壽命無窮,與天地終,此聖人之治身也。
\end{yuanwen}

\biaoti{【校注】}

\begin{jiaozhu}
	\item 喘麤为之俛仰:麤,同“粗”,俛;同“俯”。谓气息喘急粗促而前俯后仰。
	\item 烦冤:冤同“悗”。烦冤,即心胸烦乱、满闷之意。
	\item 能冬不能复:能,“耐”的通假字。张介宾注:“阴竭者,得冬之助,犹可支持,遇夏之热,不能耐受矣。”
	\item 病之形能:形,指病之形证。胡澍《内经素问校义》:“能,读为‘态’。病之形能也者,病之形态也。”
	\item 七损八益:一般认为是古代房中术的术语,“七损”指房事中损伤人体精气的七种情况,“八益,指房事中有益人体精气的八种方法。
	\item 早衰之节:节,征信。
	\item 阴痿:痿,与“萎”通。指阴器萎弱不用,后世称为“阳痿”。
	\item 同出而名异:同出,指人同禀天地阴阳之气而生;名异,指善于养生与不善于养生者有强壮与早衰的不同结果。马莳《黄帝内经素问注证发微》:“阴阳之要,人所同然,而或强或老,其名各异。”
	\item 智者察同,愚者察异:同,指人所共同禀受的天地阴阳精气,为生命健康之本原。异,指强壮与早衰的身体差异。察,懂得。
	\item 虚无之守:虚无,恬淡清静。守,当作“宇”,胡澍《内经素问校义》守字义不相属,守当为宇。……宇,居也。虚无之宇,谓虚无之居也。”即淡泊宁静的心态。
\end{jiaozhu}

\biaoti{【理论阐释】}

1.证候的阴阳划分和“腹满”的病机

经文认为“阴阳更胜”各有其相应的病变特点,并指出“阳胜”的证候特点是身热,腠理闭汗不出,喘粗俯仰,齿乾而烦冤,腹满,耐冬不耐夏;“阴胜”的证候特点是身寒,汗出身常清冷,寒栗肢厥,厥则腹满,耐夏不耐冬。提示区别证候必须以阴阳为纲领,归纳诊候,分析病机,才能抓住疾病本质。故后世作为各种疾病基本辨证纲领的八纲辨证,也以阴阳为总纲。

“阴胜”(阴证)和“阳胜”(阳证)是两类性质相反的证候,但文中指出阴胜和阳胜之甚均可出现“腹满”病候,且提示病情危重(死)。究其原因,这不是一般单纯因脾胃消化不良,饮食积滞所致的腹满,而是由于阴阳偏胜至极,脏腑气机阻绝不通所致的腹满重证。阳胜之极,里热炽盛,脾胃阴液耗竭,脏腑气机阻绝,可出现腹满;阴胜之极,脾胃阳气衰败,脏腑气机阻绝亦可出现腹满。因此,对于危重病证出现腹满病候,既应该准确审辨其阴阳偏胜病机,也要认识其病情的危重性,及时、果断地采取治疗措施以挽救危亡。

2.关于“七损八益”

关于“七损八益”的涵义,历代注家见解不同,综其大意,约有五说;一、杨上善《黄帝内经太素·阴阳》认为“阳胜八益为实,阴盛七损为虚”,并把上文“阳胜”的证候分为身热、腠理闭等八个症状,称为八益,把“阴胜”的证候分为身寒、汗出等七个症状,称为七损。二、王冰《黄帝内经素问注》:“女子以七七为天癸之终,丈夫以八八为天癸之极。……然阴七可损,则海满而血自下;阳八宜益,交会而泄精。由此则七损八益,理可知矣。”认为七损指女子月事贵乎时下,八益指男子精气贵乎充满。三、张介宾《类经·阴阳类》认为:“七为少阳之数,八为少阴之数。七损者言阳消之渐,八益者言阴长之由也,夫阴阳者生杀之本始也。生从乎阳,阳不宜消也;死从乎阴,阴不宜长也。使能知七损八益之道,而得其消长之几,则阴阳之柄,把握在我,故二者可调,否则未央而衰矣。”李中梓《内经知要》也同此论,均认为“七损八益”系说明阳消阴长是衰老的机理。四、张志聪《黄帝内经素问集注》与上说相反,认为:“七损益者,言阳常有余而阴常不足也。然阳气生于阴精,知阴之不足,而无使其亏损,则二者可调,不知阴阳相生之道而用此调养之法,则年未半百而早衰矣。”五、丹波元简《素问识》:“《天真论》云女子五七,阳明脉衰,六七三阳脉衰于上,七七任脉衰。此女子有三损也。丈夫五八肾气衰,六八阴气衰于上,七八肝气衰,八八肾气衰齿落。此丈夫有四损也。三四合为七损矣。女子七岁肾气盛,二七天癸至,三七肾气平均,四七筋骨坚。此女子有四益也。丈夫八岁肾气实,二八肾气盛,三八肾气平均,四八筋骨隆盛。此丈夫有四益也。四四合为八益矣。”则以男女生长、发育、衰老过程解释“七损八益”。

自马王堆汉墓帛书出土以后,近世学者多认为“七损八益”即《天下至道谈》“气有八益,有(又)有七孙(损),不能用八益去七孙,则行年册(四十)而阴气自半也”所言的古代房中养生术。“七损”指房事中七种有损人体精气的情况,“八益”则指八种有益人体精气的房中术法。另据《医心方》卷廿八所引《玉房秘诀》(宋代以前的房中类古籍),也载有“七损八益”之论,与《天下至道谈》内容相近。

上述各说虽因学术观点不同而见解各异,但均说明阴精阳气是生命健康之本,养生防衰老的要旨在于协调阴阳,这也体现阴阳学说对养生防病的纲领性指导意义。

\biaoti{【临证指要】}

\xiaobt{“能冬不能夏”、“能夏不能冬”理论的临床运用}

文中指出阳胜之病“能冬不能夏”、阴胜之病“能夏不能冬”,说明了时令季节与病情的关系,既提示我们在诊治疾病的过程中,应该根据时令气候的阴阳特点推测病情的发展变化,把握其预后转归,又与下文所强调的“治不法天之纪,不用地之理,则灾害至矣”相呼应,指导我们在防治疾病时必须因时制宜,顺时调摄。

叶天士在《临证指南医案》中每有结合时令季节特点辨析病机、指导处方遣药的精彩验案,如该书“痉厥”门载:“王,四一。经云:‘烦劳则张,精绝,辟积于夏,令人煎厥。’夫劳动阳气弛张,则阴精不司留恋其阳,虽有若无,故曰绝。积之既久,逢夏季阳正开泄,五志火动风生,若煎厥者然,斯为晕厥耳。治法以清心益肾,使肝胆相火、内风不为暴起,然必薄味静养为稳。连翘心、元参心、竹叶心、知母、细生地、生白芍。”案中不仅指出病人素有阴虚阳亢体质,故“能冬不能夏”,当夏季阳盛之时,肝阳亢越、相火升腾而发为煎厥,并据此提出清心火益肾阴的泻南补北治法和“薄味静养”以防复发的调摄措施。

\biaoti{【原文】}

\begin{yuanwen}
天不足西北,故西北方陰也,而人右耳目不如左明也。地不滿東南,故東南方陽也,而人左手足不如右強也。帝曰:何以然?岐伯曰:東方陽也,陽者其精并於上,并於上則上明而下虛,故使耳目聰明,而手足不便也。西方陰也,陰者其精并於下,并於下則下盛而上虛,故其耳目不聰明,而手足便也。故俱感於邪,其在上則右甚,在下則左甚,此天地陰陽所不能全\sb{1}也,故邪居之。

故天有精,地有形;天有八紀,地有五里\sb{2},故能爲萬物之父母。清陽上天,濁陰歸地,是故天地之動靜,神明\sb{3}爲之綱紀,故能以生長收藏,終而復始。惟賢人上配天以養頭,下象地以養足,中傍人事以養五藏\sb{4}。天氣通於肺,地氣通於嗌\sb{5},風氣通于肝,雷氣通於心\sb{6},谷氣通於脾,雨氣通於腎。六經爲川,腸胃爲海,九竅爲水注之氣\sb{7}。以天地爲之陰陽,陽之汗,以天地之雨名之;陽之氣,以天地之疾風名之\sb{8}。暴氣象雷,逆氣象陽\sb{9}。故治不法天之紀,不用地之理,則災害至矣。
\end{yuanwen}

\biaoti{【校注】}

\begin{jiaozhu}
	\item 天地阴阳所不能全:自然界(包括人体)的阴阳不可能绝对平衡,系指上文东方(左)阳精并于上,西方(右)阴精并于下而言。
	\item 天有八纪,地有五里:二至(冬至、夏至)、二分(春分、秋分)和立春、立夏、立秋、立冬八个节气,为太阳在天球视运动轨迹中的八个等分点,也是自然界四时阴阳消长转化的关键节气,故为天之八纪。里,通“理”,东西南北中为地面五方配合五行的道理,故称地之五里(理)。
	\item 神明:此处指阴阳而言,因阴阳为“神明之府”。
	\item 中傍人事以养五脏:傍,依附,此处有按照、效法之意。吴昆注:“中傍人事以养五脏,法人事之和也,和则阴之五宫(即五脏)无伤矣。”
	\item 地气通于嗌:嗌,咽也。地所产之五谷,通过咽嗌进入胃中。
	\item 雷气通于心:雷气,火气也。心为火脏,同气相求,故雷火之气通于心。
	\item 九窍为水注之气:张介宾注:“上七窍,下二窍,是为九窍。水注之气,言水气之注也。如目之泪,鼻之涕,口之津,二阴之尿秽皆是也。虽耳若无水,而耳中津气湿而成垢,是即水气所致。”
	\item 阳之气,以天地之疾风名之:据《太素·阴阳》原文及杨上善注,“天地之疾风”作“天地之风”,与上文“天地之雨”为对文,文理、义理均胜,可参。
	\item 暴气象雷,逆气象阳:人之暴躁怒气和上逆之气比类于天之雷霆和亢阳。张介宾注:“天有雷霆,火郁之发也;人有刚暴,怒气之逆也,故语曰雷霆之怒。”“天地之气升降和则不逆矣。天不降,地不升,则阳亢于上,人之逆气亦犹此也。”
\end{jiaozhu}

\biaoti{【理论阐释】}

1.天不足西北,地不满东南

“天不足西北,地不满东南”之说也见于《淮南子·天文训》:“昔者共工与颛顼争为帝,怒而触不周之山,天柱折,地维绝。天倾西北,故日月星辰移焉;地不满东南,故水潦尘埃归焉。”这是古代基于天文地理知识而产生的民间传说。从地理来说,我国地势西北高而东南低;从天体运行来说,日月从东方升起而从西方下沉;从季节与方位关系来说,西北对应秋冬而东南对应春夏;从气候来说,西北为高寒地带,日照时间短而气温较低,东南为湿热地带,日照时间长而气温较高。这些都是“天不足西北,地不满东南”的认识基础,也反映了处于北半球并具有西高东低地理特点的中华大地的阴阳盛衰消长情况。

至于以“天(阳)不足西北,地(阴)不满东南”类比说明人的“右耳目不如左明,左手足不如右强”,则是事理上的巧合。古代无法了解到多数人的大脑以左半球为优势半球,故其所主管的左耳目和右手足功能相对较强这一生理机制,只通过类比方法以解释所观察到的生理现象。而这一结论和解释对于不是以大脑左半球为优势半球的人(如左撇子)来说,就不合适了。

2.“取象比类”研究方法

本段经文把人与天地自然联系起来,以自然现象来说明人体生理病理和防治疾病的法则,这种研究说理方法称为取象比类。格物致知,取象比类是古代经常采用的研究方法,它能够克服科学技术条件的限制,发挥思维智慧以创建新理论,或解释原有理论无法解释的现象,因此对中医学术理论的产生和发展曾经发挥了巨大作用,而且,该方法的广泛运用是形成中医“人与天地相参应”这一基本学术观念的重要促进因素。但也应该认识到,由于取象比类建立在事物表面征象的联系上,因此其结论可能带有偶然性,必须通过实践的验证并加以去芜存精的筛选取舍,进一步揭示其内在本质,才能上升为理论。中医理论体系也正是在对各种假说进行长期临床验证的过程中逐步完善起来。

\biaoti{【原文】}

\begin{yuanwen}
故邪風之至,疾如風雨,故善治者治皮毛,其次治肌膚,其次治筋脈,其次治六府,其次治五藏。治五藏者,半死半生也。故天之邪氣,感則害人五藏;水穀之寒熱,感則害於六府;地之濕氣,感則害皮肉筋脈。

故善用鍼者,從陰引陽,從陽引陰\sb{1};以右治左,以左治右\sb{2};以我知彼,以表知裏,以觀過與不及之理,見微得過,用之不殆。

善診者,察色按脈,先別陰陽。審清濁而知部分;視喘息、聽音聲而知所苦;觀權衡規矩\sb{3}而知病所主;按尺寸,觀浮沈滑澀\sb{4}而知病所生。以治無過,以診則不失矣。
\end{yuanwen}

\biaoti{【校注】}

\begin{jiaozhu}
	\item 从阴引阳,从阳引阴:引,引导经洛之气以调节虚实,疏散邪气。张志聪《黄帝内经素问集注》:“阴阳气血,外内左右,交相贯通。故善用针者,从阴而引阳分之邪,从阳而引阴分之气。”
	\item 以右治左,以左治右:三阴三阳经脉左右互相贯通,故针刺左侧经脉的俞穴可治右侧病变,针刺右侧经脉的俞穴可治左侧病变,刺法上的缪刺、巨刺皆用此法。又,“左右者,阴阳之道路也”,故本句也是对上句“从阴引阳,从阳引明”的进一步说明。
	\item 观权衡规矩:权为秤锤,有沉实之象;衡为秤杆,有平衡之象;规为作圆之器,有轻灵圆滑之象;矩为作方之器,有平盛之象。四者比喻四时的脉象特征。即《脉要精微论》所说的“春应中规,夏应中矩,秋应中衡,冬应中权”。
	\item 按尺寸,观浮沉滑涩:尺,指尺肤,古代有触按尺肤以诊病的方法,称“尺肤诊”;寸,指寸口脉。浮沉,指寸口脉的浮沉;滑涩,指尺肤的滑涩。
\end{jiaozhu}

\biaoti{【理论阐释】}

\xiaobt{善治者治皮毛}

“善治者治皮毛,……治五脏者,半死半生矣”一节,强调了早期治疗的重要性。疾病,特别是外感疾病都有一个由表入里,由阳入阴的过程。病在阳分,病情尚轻浅,及时诊断,早期治疗,易于康复;若深入阴分,则病情深重而难以救治,故经云“治五脏者半死半生”。这一早期治疗的原则,《内经》多处着重强调,《素问·宝命全形论》提出“上工救其萌牙,……下工救其已成,救其已败”,《灵枢·官能》也谓:“上工之取气,乃救其萌芽;下工守其已成,因败其形。”《素问·刺热篇》则谓:“病虽未发,见赤色而刺之者,名曰治未病。”可见《内经》的“治未病”有二义;一为末病先防,一是早期治疗,已病防变。而这一早期治疗的思想得到了《难经》和《伤寒论》进一步阐发,如《难经·七十七难》在引申《内经》“治未病”理论时说:“所谓治未病者,见肝之病,则知肝当传之与脾,故先实其脾气,无令得受肝之邪,故曰治未病也。”此后,“治未病”理论更为历代医家所继承和发扬,从而成为防治疾病基本原则之一。

\section{素問·藏氣法時論(節選)}%第二節

\biaoti{【原文】}

\begin{yuanwen}
黃帝問曰:合人形以法四時五行而治,何如而從,何如而逆,得失之意,願聞其事。岐伯對曰:五行者,金、木、水、火、土也,更貴更賤\sb{1},以知死生,以決成敗,而定五藏之氣,間甚\sb{2}之時,死生之期也,……。

病在肝,愈于夏,夏不愈,甚於秋,秋不死,持於冬,起於春,禁當風。肝病者,愈在丙丁,丙丁不愈,加于庚辛,庚辛不死,持於壬癸,起於甲乙。肝病者,平旦慧,下晡\sb{3}甚,夜半靜。肝欲散,急食辛以散之,用辛補之,酸寫之。

病在心,愈在長夏,長夏不愈,甚於冬,冬不死,持於春,起於夏,禁溫食、熱衣。心病者,愈在戊己,戊己不愈,加於壬癸,壬癸不死,持於甲乙,起於丙丁。心病者,日中慧,夜半甚,平旦靜。心欲耎,急食鹹以耎之,用鹹補之,甘寫之。

病在脾,愈在秋,秋不愈,甚於春,春不死,持於夏,起於長夏,禁溫食飽食、濕地濡衣。脾病者,愈在庚辛,庚辛不愈,加於甲乙,甲乙不死,持于丙丁,起於戊己。脾病者,日昳\sb{4}慧,日出甚\sb{5},下晡靜\sb{6}。脾欲缓,急食甘以緩之,用苦寫之,甘補之。

病在肺,愈在冬,冬不愈,甚於夏,夏不死,持於長夏,起於秋,禁寒飲食、寒衣。肺病者,愈在壬癸,壬癸不愈,加於丙丁,丙丁不死,持於戊己,起于庚辛。肺病者,下晡慧,日中甚,夜半靜\sb{7}。肺欲收,急食酸以收之,用酸補之,辛寫之。

病在腎,愈在春,春不愈,甚於長夏,長夏不死,持於秋,起於冬,禁犯焠㶼熱食\sb{8}、溫炙衣。腎病者,愈在甲乙,甲乙不愈,甚於戊己,戊己不死,持於庚辛,起於壬癸。腎病者,夜半慧,四季\sb{9}甚,下晡靜。腎欲堅,急食苦以堅之,用苦補之,鹹寫之。

夫邪氣之客於身也,以勝相加\sb{10},至其所生\sb{11}而愈,至其所不勝\sb{12}而甚,至于所生\sb{13}而持,自得其位而起\sb{14};必先定五藏之脈,乃可言間甚之時,死生之期也。
\end{yuanwen}

\biaoti{【校注】}

\begin{jiaozhu}
	\item 更贵更贱:高士宗注云:“四时之气,不外五行,五行者,金木水火土也。贵者,木旺于春,火旺于夏;贱者,木败于秋,火灭于冬。更贵更贱者,生化迭乘,寒暑往来也。”指四时五行之气互相更替,互为衰旺。
	\item 间甚:间,轻缓;甚,加甚、加重。
	\item 下晡:指申时,即下午三至五时。古人一日两餐,早餐称朝食、早(蚤)食,在辰时(早上七至九时),晚餐称晡食,在申时。
	\item 日昳:昳,日偏斜,又称“日昃”,指未时,即午后一至三时。
	\item 日出甚:《脉经》、《甲乙经》、《千金方》均作“平旦甚”。林亿《新校正》:“虽日出与平旦时等,按前文言木王之时,皆云平旦而不云日出,盖日出于冬夏之期有早晚,不若平旦之为得也。”平旦为天亮(寅时),日出则稍后于平旦(卯时)。
	\item 下晡静;丹波元简注:“据前后文例,当是云‘日中静’。”
	\item 夜半静:丹波元简注:“据前后文例,当是云‘日昳静,。”
	\item 焠㶼热食:焠,音翠,烧灼;㶼,音哀,热甚。张介宾《类经·疾病类》:“焠㶼,烧爆之物也,肾恶燥烈,故当禁此。”
	\item 四季:指一日中辰、戌、丑、未四个时辰,相当于一年四季中各季的最后一个月,按脾不专主一时,寄旺四季的理论,为脾(土)旺的时辰。
	\item 以胜相加:加,加临,即侵犯。邪气常于主令之时侵犯所胜(所克)之脏而发病,如风邪于春季伤脾而致病。
	\item 其所生:张介宾注:“我所生也,以时而言。”如肝病之于属火之时。
	\item 其所不胜:指受病之脏被克之时,如肝病之于属金之时。
	\item 所生:指生我之脏主令之时。张志聪注:“得所生之母气,而能支持也。”如肝病之于属水之时。
	\item 自得其位而起:起,指病情好转。指受病之脏当令之时(如肝病之于春季),病情可能好转。
\end{jiaozhu}

\biaoti{【理论阐释】}

1.合人形以法四时五行而治

本篇运用五行学说,把作为生命活动中心的五脏,与自然界四时联系起来,构建了以五脏为中心,以五行为中介的天人相应系统,成为中医藏象学说和病机学说的重要内容之一。关于本文所言的四时,既指一年之四(五)季、一月之干支(十天干)时日、也指一日之时辰,其与五脏的关联关系如下表。(表\ref{tab:时脏五行归类表})

\begin{table}[htb]%时脏五行归类表
	\centering
	\caption{时脏五行归类表}\label{tab:时脏五行归类表}
	\begin{tabu}to.87\textwidth{*3{X[c]|}X[-2.5,c]*2{|X[c]}}
		\toprule
		五行 & 木       & 火      & 土              & 金      & 水      \\
		\midrule
		五脏 & 肝       & 心      & 脾              & 肺      & 肾      \\
		四季 & 春       & 夏      & 长夏            & 秋      & 冬      \\
		时日 & 甲乙     & 丙丁    & 戊己            & 庚辛    & 壬癸    \\
		时段 & 平旦     & 日中    & 日昳            & 下晡    & 夜半    \\
			 & (寅卯) & (巳午)& 四季(辰戌丑未)& (申酉)& (亥子)\\
		\bottomrule
	\end{tabu}
\end{table}
上述五脏与四时的通应关系,实际上反映了《内经》对以五脏为中心的人体系统的生命节律的朴素认识,可以说是中医有关生物钟理论和时间医学之滥觞,对认识人体生命活动,指导临床辨证论治,均有重要意义,而且从中也体现了五行学说的取象比类及其生克制化规律在医学领域的运用。

2.五脏病愈、甚、持、起时间的预测

疾病发生以后,其病情的发展变化,决定于邪正之间的力量对比,正气强盛,战胜邪气,则病情好转、向愈(起、愈);正气衰弱,邪气胜正,则病情加甚(甚);正邪相持,则病情胶着(持),对于五脏病来说,在致病因素(邪气)相对固定的情况下,病情与受病之脏的脏气盛衰密切相关。由于脏气随自然界四时变化而消长盛衰,因此病情在一定的周期(一年、一月、一日)内可以表现出相应的变化节律,本篇即据此指出五脏病“以胜相加”发病之后,具有“至其所生(我所生)而愈,至其所不胜而甚,至于所生(生我者)而持,自得其位而起”的发展变化规律。

这一五脏病愈、甚、持、起时间节律,既是以“五脏应四时,各有收受”(《素问·金匮真言论》)的脏气消长变化规律为基础,也是古代医家运用五行生克关系对长期观察到的临床资料进行总结而得出的认识。掌握这一理论,对把握病情发展变化趋势,推测疾病预后转归,及时采取“已病防变”措施以防止病情蔓延恶化,颇具实际意义,故为历代医家所继承和进一步发挥。张仲景《伤寒论》“六经病欲解时”、《金匮要略》的“见肝之病,知肝传脾”等说,均是对这一理论的引申和发挥,而后世医家对于内伤杂病,特别是中风、虚劳、吐血等危重或慢性而复杂的病证,也每灵活运用《内经》这一理论以指导辨证论治、推测病情转归。当然,由于疾病是一种复杂的异常生命活动,其病理变化往往受多种因素的共同影响,并不单纯决定于脏气的消长盛衰,因此不是所有五脏病变都必定呈现上述发展变化规律。本篇提供给我们的,只是一种可能的模式或趋势,而非必然的结果,因此临床上必须综合多种因素加以客观分析,不可刻舟求剑、胶柱鼓瑟而机械套用。

另外,应该注意的是,本篇所言的病情发展变化规律主要针对五脏病,如后世所称的内科杂病等而言。对于外感病,由于病情常决定于外感邪气与人体正气(阳气)抗邪机能的盛衰消长,故每随人体阳气的昼夜消长而有《灵枢·顺气一日分为四时》篇所言的“旦慧、昼安、夕加、夜甚”变化规律。两者所适应的病证不同,需加区别。

\biaoti{【临证指要】}

\xiaobt{“五脏苦欲补泻”理论的指导意义}

“五脏苦欲补泻”是根据五脏的特性以指导处方用药的理论。五脏各有其性,如肝苦急欲散,心苦缓欲软,脾苦湿欲缓等。而药物的五味则有辛散、酸收、甘缓、苦坚(燥)和咸软的作用,故可用其性味以纠正脏气之偏,如肝苦急欲散,即恶(苦)酸喜(欲)辛,故以酸泻之,以辛补之,即李念莪《医宗必读·苦欲补泻论》所言的“违其性则苦,遂其性则欲。本脏所恶,即名为泻;本脏所喜,即名为补。”金元以降,不少医家均把“五脏苦欲补泻”理论作为临床上辨证用药的指导原则,如张元素《医学启源·用药备旨》即引用本篇原文加以具体药物举例:“肝苦急,急食甘以缓之,甘草;心苦缓,急食酸以收之,五味子;脾苦湿,急食苦以燥之,白术;肺苦气上逆,急食苦以泄之,黄芩;肾苦燥,急食辛以润之,黄柏、知母。……肝欲散,急食辛以散之,川芎;以辛补之,细辛;以酸泻之,白芍药。”其后,王好古的《汤液本草》、缪希雍的《神农本草经疏》、李念莪的《医宗必读》等,都立专篇对张元素之说作了更深刻和详细的发挥,缪、李二氏并强调:“夫五脏之苦欲补泻,乃用药第一义也,不明乎此,不足以言医。”(《医宗必读·苦欲补泻论》)可见这一理论对临床遣药处方的指导意义,兹举叶天士《临证指南医案·吐血门》中医案三则,以见其运用:

“沈,味进辛辣,助热之用,致肺伤嗽甚,其血震动不息,阳少潜伏,而夜分为甚,清气热而不妨胃口,甘寒是投,与《内经》‘肝苦急,急食甘以缓之’恰符。生甘草、玉竹、麦冬、生白芍、茯神。”

“陆,食酸助木,胃土受侮,脘中阳逆,络血上溢。《内经》辛酸太过,都从甘缓立法,沉苦勿进。生扁豆、北沙参、炒麦冬、茯苓、川斛、甘蔗浆。”

“万,脉数左坚,当夏四月,阳气方张,陡然嗔怒,肝阳勃升,络血上涌,虽血得止而咳逆欲呕,眠不得欹左,此肝阳左升太过,木失水涵,阴亏则生热,是皆本体阴阳迭偏,非客邪实火可清可降之比,……木犯胃土贯膈,即至冲咽人肺,肺衰木反刑金,从《内经》甘缓以制其急。米炒麦冬、糯稻根须、女贞子、茯神、生甘草、南枣肉。”

肝藏血,大抵吐血证多与肝气亢急,肝阳上逆克土或刑金有关,叶氏在三案中皆遵《内经》“肝苦急,急食甘以缓之”之旨,从正本澄源入手,治血证而不用一味止血药,堪称妙用《内经》理论以指导临床之典范。

\section{素問·六微旨大論(節選)}%第三節

\biaoti{【原文】}

\begin{yuanwen}
願聞地理之應六節氣位\sb{1}何如?岐伯曰:顯明之右\sb{2},君火\sb{3}之位也;君火之右,退行一步\sb{4},相火\sb{3}治之;復行一步\sb{5},土氣治之;復行一步,金氣治之;復行一步,水氣治之;復行一步,木气治之;復行一步,君火治之。相火之下,水氣承\sb{6}之;水位之下,土氣承之;土位之下,風氣承之;風位之下,金氣承之;金位之下,火氣承之;君火之下,陰精承之\sb{7}。帝曰:何也?岐伯曰:亢則害,承乃制\sb{8},制則生化,外列盛衰\sb{9},害則敗亂,生化大病。
\end{yuanwen}

\biaoti{【校注】}

\begin{jiaozhu}
	\item 地理之应六节气位:地理,此处指大地的气候变化规律。六节,指天体运行的六个时间段节,每一段节包括历法上的四个节气。气位,指主气六步的方位和时间。张介宾《类经·运气》:“即主气之静而守位者,故曰六气,亦曰六步,乃六气所主之位也。”如初之气为厥阴风木,位于丑至卯位(十二月中至二月中,包栝大寒、立春、雨水、惊蛰四个节气),二之气为少阴君火,位于卯至巳位(二月中至四月中,包括春分、清明、谷雨、立夏四个节气)。(图\ref{fig:六气六步主治、承制图})
	\item 显明之右:王冰《黄帝内经素问注》:“日出谓之显明,则卯地气春分也。”日出于东方卯位、卯时,卯为二月中春分节气。显明之右,即从卯至巳的方位和时间。
	\item 君火、相火:六气中火有二,分别称为少阴君火和少阳相火。少阴君火,主气为热,主令为二之气(二月中春分至四月中立夏)。少阳相火,主气为火(暑),主令为三之气(四月中小满至六月中小暑)。
	\item 退行一步:《类经·运气》:“退行一步,谓退于君火之右一步也。”古代天文学把自东(左)而西(右)称为“进”,自西(右)而东(左)称为“退”,六气的运转方向是自右而左,故称“退行”。
	\item 复行一步;指从二之气再向右行一步。
	\item 承:承袭。六气中的每一气主令,其气较盛,有亢极为害的可能;按照自然界正常调节机制,则有其所不胜之气以制约之,即所谓“承”。吴昆《素问吴注》:“六气各专主一令,专令者常太过,故各有所承,不欲其亢甚为害也。”
	\item 君火之下,阴精承之:六气中火有君火、相火二气,君火也为阳热之气,阴能制阳,故以阴精承制之。
	\item 亢则害,承乃制:《类经·运气》:“亢者,盛之极也。制者,因其极而抑之也。”六气之某气过盛,则对其所胜者过度克制而造成危害,从而使自然界的阴阳五行失去正常平衡协调关系。在此情况下,能够克制该过盛之气者则起而承袭之,使其受到制约而不过亢为害。
	\item 外列盛衰:外列,谓六气盛衰更替所表现出来的征象分别呈现于外。清·高士栻《黄帝素问直解》:“外列盛衰者,盛已而衰,衰已而盛,四时之气可征也。”
\end{jiaozhu}

%\begin{figure}[htb]
%	\centering
%	% Requires \usepackage{graphicx}
%	\includegraphics[width=0.50\textwidth]{六气六步主治、承制图.png}\\
%	\caption{六气六步主治、承制图}\label{fig:六气六步主治、承制图}
%\end{figure}
\begin{figure}[htb]%六气六步主治、承制图
	\centering
	\begin{tikzpicture}
	\def\nj{0.2}
	\def\bc{0.7}

	\foreach \r in {1,...,4,5.5} { \draw (0,0) circle(\nj+\r*\bc); } %画同心圆
	\foreach \firstlayer/\secondlayer/\thirdlayer [count=\i] in {
		风气/太阴湿土/气之四,火气/金燥明阳/气之五,土气/太阳寒水/气之终,
		金气/厥阴风木/气之初,阴精/少阴君火/气之二,水气/少阴相火/气之三,
	}{
		\pgfmathsetmacro{\angle}{90-360/6*\i}
		\draw (\angle-360/6/2:\nj+\bc) -- (\angle-360/6/2:\nj+\bc*5.5);%画等分线
		%\node[rotate=90+\angle] at (\angle:\nj+\bc*1.5-0.05) {\firstlayer};
		\draw[decorate, decoration={text align=center, text along path,
			text={\firstlayer}}] (\angle-360/6:\nj+\bc*1.5+0.1) arc (\angle-360/6:\angle+360/6:\nj+\bc*1.5+0.1);
		\draw[decorate, decoration={text align=center, text along path,
			text={\secondlayer}}] (\angle-360/6:\nj+\bc*2.5+0.1) arc (\angle-360/6:\angle+360/6:\nj+\bc*2.5+0.1);
		%\node[rotate=90+\angle] at (\angle:\nj+\bc*2.5-0.05) {\secondlayer};
		\draw[decorate, decoration={text align=center, text along path,
			text={\thirdlayer}}] (\angle-360/6:\nj+\bc*3.5+0.1) arc (\angle-360/6:\angle+360/6:\nj+\bc*3.5+0.1);
		%\node[rotate=90+\angle] at (\angle:\nj+\bc*3.5-0.05) {\thirdlayer};
	}
	\foreach \fourthlayer [count=\i] in {
		大暑,立秋,处暑,白露,
		秋分,寒露,霜降,立冬,
		小雪,大雪,冬至,小寒,
		大寒,立春,雨水,惊蛰,
		春分,清明,谷雨,立夏,
		小满,芒种,夏至,小暑,
	}{
		\pgfmathsetmacro{\angle}{90-360/24*(\i+1.5)}
		\node[rotate=90+\angle,text width =1em] at (\angle:\nj+\bc*4.8-0.05) {\fourthlayer};
	}
	\end{tikzpicture}
	\caption{六气六步主治、承制图}\label{fig:六气六步主治、承制图}
\end{figure}


\biaoti{【理论阐释】}

\xiaobt{“亢则害,承乃制”的理解及意义}

正常情况下,包括人体在内的自然界处于阴阳五行的动态平衡之中,但由于阴阳的互相对立消长,故一方偏盛必致一方偏衰,而五行的相克互制也会出现“气有余则制己所胜而侮所不胜”。这种阴阳五行的失常必然导致自然界以至人体等复杂系统的动态平衡遭受破坏,这就是“亢则害”,“害则败乱,生化大病”。然而,对于自然界等具有内稳定能力的复杂系统来说,其内在的自然力能够作出自我调节以克服失衡状态,经文所言的“相火之下,水气承之;水位之下,土气承之;土位之下,风气承之;风位之下,金气承之;金位之下,火气承之;君火之下,阴精承之”,正是指自然界的这种调节功能。关于“承”,王履在《医经溯洄集·亢则害承乃制论》中有相当精辟的解释:“承,犹随也。然不言随而曰承者,以下言之,则有上奉之象,故曰承。虽谓之承,而有防之之义存焉。亢者,过极也;害者,害物也;制者,克己之胜也,然所承也,其不亢则随之而已,故虽承而不见;既亢,则克胜以平之,承斯见矣。”王氏并认为:“盖造化之常,不能以无亢,亦不能以无制焉。”“造化之道,苟变至于极,则亦终必自反,而复其常矣。”张介宾《类经·运气》也指出:“盖阴阳五行之道,亢极则乖,而强弱相残矣。故凡有偏盛,则必有偏衰,使强无所制,则强者愈强,弱者愈弱,而乖乱日甚。所以亢而过甚,则害乎所胜,而承其下者,必从而制之。此天地自然之妙,真有莫之使然而不得不然者。天下无常胜之理,亦无常屈之理。”

可见,“亢则害,承乃制”符合自然事物的阴阳五行法则:事物内部的阴阳五行关系处于互相制约、互相促进的正常动态平衡状态,从而维持事物的相对稳定。不论阴阳或五行,其中的一方偏胜,则会破坏事物的相对稳定,从而出现“充则害”的失常状态。但要克服失常状态而达到新的动态平衡和稳定,这种调节机制就离不开“承乃制”。对于自然界来说,“承乃制”是“天地自然之妙,真有莫之使然而不得不然者”,故《墨子》有“五行无常胜”之说,而《素问·阴阳应象大论》也有“重阴必阳,重阳必阴”之论。

\biaoti{【临证指要】}

\xiaobt{“亢害承制”理论的临床意义}

人体与自然界同样是一个复杂的、自我调控的内稳定系统,因此也存在“亢害承制”的调节机制。但人体与自然界不同之处在于它是生命有机体,不仅对“亢则害”的承受能力脆弱,而且自我调控能力也甚为有限,超出其调控能力的过亢,会导致人体系统的失控,甚至造成无可挽回的系统崩溃。中医治疗疾病,就是借用医药的力量,帮助人体增强“承乃制”的力度以克服“亢则害”所造成的损伤,避免生命系统的崩溃而挽救危亡,所谓“医具回天之力”,殆即指此而言。后世医家运用这一理论于临床,一是用以对“亢害”证候的辨识,一是用以指导“亢害”病证的论治。

金元医家刘完素对“亢害承制”理论有深刻而独到的阐发,认为人体化风、化热、化火、化湿、化燥、化寒等病变机理,也与自然界六气变化一样,可能出现“亢极反兼胜己之化”的情况,其在《素问玄机原病式·序》中说:“所谓木极似金,金极似火,火极似水,水极似土,土极似木者也。故《经》曰:亢则害,承乃制。谓己亢过极反兼胜己之化也。俗之未知,认似作是,以阳为阴,失其意也。”书中并举例说明:“故病湿过极则为痙(痉),反兼风化制之也;病风过极则反燥,筋脉劲急,反兼金化制之也;病燥过极则烦渴,反兼火化制之也;病热过极而反出五液(指热甚汗出,汗为五液之一),或为战慄恶寒,反兼水化制之也。”(《素问玄机原病式·寒类》)特别对于火热病证,刘氏再三强调必须注意“火极似水”病候的辨识:“(下)利色黑,亦言为热者,由火热过极,则反兼水化制之,故色黑也”。“亢则害,承乃制,谓亢过极,则反兼胜己之化,制其甚也。如火炼金,热极则反为水。……故下部任脉湿热甚者,津液涌溢而为带下也。”“或谓呕吐紫凝血为寒者,误也。此作冷凝,由热甚销烁以为稠独,而热甚则水化制之,故赤兼黑而为紫也。”“此由心火热甚,亢极而战,反兼水化制之,故寒慄也,然寒慄者,由火甚似水,实非兼有寒气也。”(《素问玄机原病式·热类》)刘氏之论,对临床上危重病证的证候真假,特别是真热假寒证的辨析,颇具意义。

在治疗方面,对于六气过亢致病,治疗上必须“承”以制之,如火热亢甚者,宜寒凉以清热,甚则苦寒泻下以直折亢阳,故张仲景《伤寒论》中不仅以大承气汤荡涤阳明病腑实热结之燥屎,对于少阴病亢热为害而阴津欲绝之三急下证,也以之苦寒泻火,釜底抽薪以急下亢热的大承气汤挽救危亡。刘完素对于热证善用寒凉泻火之法,也是以“承乃制”为理论依据,其不仅对心火热甚之战慄证主张“以大承气汤下之”,对于“中外怫热郁结,燥而无汗”,“表热服石膏、知母、甘草、葱、豉之类寒药,汗出而解者;及热病半在表,半在里,服小柴胡汤寒药,和令汗出而解者;热甚服大柴胡汤下之,更甚者小承气汤、调胃承气汤、大承气汤下之;发黄者茵陈蒿汤下之;结胸者陷胸汤、丸下之,此皆大寒之利药也,反能中病,以令汗出而愈”(《素问玄机原病式·热类》)。又如戴思恭治诸仲文火极似水验案:“松江诸仲文,盛夏畏寒,常御重绩,饮食必令极热始下咽,微温即吐。他医投以胡椒煮伏雌之法,日啖鸡者三,病更剧。戴曰:‘脉数而大且不弱,刘守真云火极似水,此之谓也。椒发三阴之火,鸡能助痰,只益其病耳。’乃以大承气汤下之,昼夜行二十余度,顿减纩之半,后以黄连导痰汤加竹沥饮之,竟瘳”(《续名医类案·恶寒》)。这些都是医家运用“亢害承制”理论以治疗急重疑难疾病的范例,掌握这一理论,对临床辨证论治甚有裨益。

\xiaojie

本章所选经文,主要讨论了以下几个方面的内容。

一、举例说明阴阳、五行的基本属性及取象比类。凡天地、上下、高低、水火、气血、男女、左右、东南西北、清浊、寒热、辛酸甘苦、气味厚薄、喜怒、寒暑等互有联系的对立双方均可根据其属性分为阴阳并加以类比。

凡五方、五季、五脏、五体、五官、五音、五声、五色、五志、五气、五味、五谷、五菜、五果、五畜等,均可因其特性而分为五行并加以归类。

二、阐述了阴阳学说和五行学说的基本内容,尤其是在医学中的应用。提出阴阳是世间万事万物的“纲纪”,在此甚础上以“治病必求于本”一语,道出其在医学中的地位和作用。指出天地阴阳与人体脏腑清浊相关、阳为使阴为守、精气形味转化、寒暑喜怒病因特点、阴阳寒热病证及其转化、察色按脉闻声诊断、药食气味及其在治疗养生中的运用、阳病治阴及从阳引阴等治法,贯穿于医学理论的各个方面。

“天有四时五行……人有五脏”,五脏特性不同,其病各异,疾病发展变化及转归也均与五行生克制化有关,因此可以预知某些疾病的“间甚之时,死生之期”。掌握疾病发展变化规律,使之为正确施治提供依据,无论用药用膳,或苦或咸、或软或坚,可以做到“四时五脏,病随五味所宜”。

三、亢害承制是在阴阳五行学说基础上提出来的具体理论,用以说明事物内部的阴阳五行关系处于相互制约、相互促进的正常动态平衡状态,从而维持事物的相对稳定。这一理论对分析病机、指导治疗都有重要意义。

四、关于治疗原则,本章经文除根据阴阳五行理论所确立的治疗原则外,还提出“善治者治皮毛……治五脏者,半死半生也”的早期治疗原则,为中医治疗学发展打下坚实的理论基础。

\zuozhe{(吴弥漫)}
\ifx \allfiles \undefined
\end{document}
\fi