% -*- coding: utf-8 -*-
%!TEX program = xelatex
\ifx \allfiles \undefined
\documentclass[draft,12pt]{ctexbook}
%\usepackage{xeCJK}
%\usepackage[14pt]{extsizes} %支持8,9,10,11,12,14,17,20pt

%===================文档页面设置====================
%---------------------印刷版尺寸--------------------
%\usepackage[a4paper,hmargin={2.3cm,1.7cm},vmargin=2.3cm,driver=xetex]{geometry}
%--------------------电子版------------------------
\usepackage[a4paper,margin=2cm,driver=xetex]{geometry}
%\usepackage[paperwidth=9.2cm, paperheight=12.4cm, width=9cm, height=12cm,top=0.2cm,
%            bottom=0.4cm,left=0.2cm,right=0.2cm,foot=0cm, nohead,nofoot,driver=xetex]{geometry}

%===================自定义颜色=====================
\usepackage{xcolor}
  \definecolor{mybackgroundcolor}{cmyk}{0.03,0.03,0.18,0}
  \definecolor{myblue}{rgb}{0,0.2,0.6}

%====================字体设置======================
%--------------------中文字体----------------------
%-----------------------xeCJK下设置中文字体------------------------------%
\setCJKfamilyfont{song}{SimSun}                             %宋体 song
\newcommand{\song}{\CJKfamily{song}}                        % 宋体   (Windows自带simsun.ttf)
\setCJKfamilyfont{xs}{NSimSun}                              %新宋体 xs
\newcommand{\xs}{\CJKfamily{xs}}
\setCJKfamilyfont{fs}{FangSong_GB2312}                      %仿宋2312 fs
\newcommand{\fs}{\CJKfamily{fs}}                            %仿宋体 (Windows自带simfs.ttf)
\setCJKfamilyfont{kai}{KaiTi_GB2312}                        %楷体2312  kai
\newcommand{\kai}{\CJKfamily{kai}}
\setCJKfamilyfont{yh}{Microsoft YaHei}                    %微软雅黑 yh
\newcommand{\yh}{\CJKfamily{yh}}
\setCJKfamilyfont{hei}{SimHei}                                    %黑体  hei
\newcommand{\hei}{\CJKfamily{hei}}                          % 黑体   (Windows自带simhei.ttf)
\setCJKfamilyfont{msunicode}{Arial Unicode MS}            %Arial Unicode MS: msunicode
\newcommand{\msunicode}{\CJKfamily{msunicode}}
\setCJKfamilyfont{li}{LiSu}                                            %隶书  li
\newcommand{\li}{\CJKfamily{li}}
\setCJKfamilyfont{yy}{YouYuan}                             %幼圆  yy
\newcommand{\yy}{\CJKfamily{yy}}
\setCJKfamilyfont{xm}{MingLiU}                                        %细明体  xm
\newcommand{\xm}{\CJKfamily{xm}}
\setCJKfamilyfont{xxm}{PMingLiU}                             %新细明体  xxm
\newcommand{\xxm}{\CJKfamily{xxm}}

\setCJKfamilyfont{hwsong}{STSong}                            %华文宋体  hwsong
\newcommand{\hwsong}{\CJKfamily{hwsong}}
\setCJKfamilyfont{hwzs}{STZhongsong}                        %华文中宋  hwzs
\newcommand{\hwzs}{\CJKfamily{hwzs}}
\setCJKfamilyfont{hwfs}{STFangsong}                            %华文仿宋  hwfs
\newcommand{\hwfs}{\CJKfamily{hwfs}}
\setCJKfamilyfont{hwxh}{STXihei}                                %华文细黑  hwxh
\newcommand{\hwxh}{\CJKfamily{hwxh}}
\setCJKfamilyfont{hwl}{STLiti}                                        %华文隶书  hwl
\newcommand{\hwl}{\CJKfamily{hwl}}
\setCJKfamilyfont{hwxw}{STXinwei}                                %华文新魏  hwxw
\newcommand{\hwxw}{\CJKfamily{hwxw}}
\setCJKfamilyfont{hwk}{STKaiti}                                    %华文楷体  hwk
\newcommand{\hwk}{\CJKfamily{hwk}}
\setCJKfamilyfont{hwxk}{STXingkai}                            %华文行楷  hwxk
\newcommand{\hwxk}{\CJKfamily{hwxk}}
\setCJKfamilyfont{hwcy}{STCaiyun}                                 %华文彩云 hwcy
\newcommand{\hwcy}{\CJKfamily{hwcy}}
\setCJKfamilyfont{hwhp}{STHupo}                                 %华文琥珀   hwhp
\newcommand{\hwhp}{\CJKfamily{hwhp}}

\setCJKfamilyfont{fzsong}{Simsun (Founder Extended)}     %方正宋体超大字符集   fzsong
\newcommand{\fzsong}{\CJKfamily{fzsong}}
\setCJKfamilyfont{fzyao}{FZYaoTi}                                    %方正姚体  fzy
\newcommand{\fzyao}{\CJKfamily{fzyao}}
\setCJKfamilyfont{fzshu}{FZShuTi}                                    %方正舒体 fzshu
\newcommand{\fzshu}{\CJKfamily{fzshu}}

\setCJKfamilyfont{asong}{Adobe Song Std}                        %Adobe 宋体  asong
\newcommand{\asong}{\CJKfamily{asong}}
\setCJKfamilyfont{ahei}{Adobe Heiti Std}                            %Adobe 黑体  ahei
\newcommand{\ahei}{\CJKfamily{ahei}}
\setCJKfamilyfont{akai}{Adobe Kaiti Std}                            %Adobe 楷体  akai
\newcommand{\akai}{\CJKfamily{akai}}

%------------------------------设置字体大小------------------------%
\newcommand{\chuhao}{\fontsize{42pt}{\baselineskip}\selectfont}     %初号
\newcommand{\xiaochuhao}{\fontsize{36pt}{\baselineskip}\selectfont} %小初号
\newcommand{\yihao}{\fontsize{28pt}{\baselineskip}\selectfont}      %一号
\newcommand{\xiaoyihao}{\fontsize{24pt}{\baselineskip}\selectfont}
\newcommand{\erhao}{\fontsize{21pt}{\baselineskip}\selectfont}      %二号
\newcommand{\xiaoerhao}{\fontsize{18pt}{\baselineskip}\selectfont}  %小二号
\newcommand{\sanhao}{\fontsize{15.75pt}{\baselineskip}\selectfont}  %三号
\newcommand{\sihao}{\fontsize{14pt}{\baselineskip}\selectfont}%     四号
\newcommand{\xiaosihao}{\fontsize{12pt}{\baselineskip}\selectfont}  %小四号
\newcommand{\wuhao}{\fontsize{10.5pt}{\baselineskip}\selectfont}    %五号
\newcommand{\xiaowuhao}{\fontsize{9pt}{\baselineskip}\selectfont}   %小五号
\newcommand{\liuhao}{\fontsize{7.875pt}{\baselineskip}\selectfont}  %六号
\newcommand{\qihao}{\fontsize{5.25pt}{\baselineskip}\selectfont}    %七号   %中文字体及字号设置
\xeCJKDeclareSubCJKBlock{SIP}{
  "20000 -> "2A6DF,   % CJK Unified Ideographs Extension B
  "2A700 -> "2B73F,   % CJK Unified Ideographs Extension C
  "2B740 -> "2B81F    % CJK Unified Ideographs Extension D
}
%\setCJKmainfont[SIP={[AutoFakeBold=1.8,Color=red]Sun-ExtB},BoldFont=黑体]{宋体}    % 衬线字体 缺省中文字体

\setCJKmainfont{simsun.ttc}[
  Path=fonts/,
  SIP={[Path=fonts/,AutoFakeBold=1.8,Color=red]simsunb.ttf},
  BoldFont=simhei.ttf
]

%SimSun-ExtB
%Sun-ExtB
%AutoFakeBold:自动伪粗,即正文使用\bfseries时生僻字使用伪粗体;
%FakeBold:强制伪粗,即正文中生僻字均使用伪粗体
%\setCJKmainfont[BoldFont=STHeiti,ItalicFont=STKaiti]{STSong}
%\setCJKsansfont{微软雅黑}黑体
%\setCJKsansfont[BoldFont=STHeiti]{STXihei} %serif是有衬线字体sans serif 无衬线字体
%\setCJKmonofont{STFangsong}    %中文等宽字体

%--------------------英文字体----------------------
\setmainfont{simsun.ttc}[
  Path=fonts/,
  BoldFont=simhei.ttf
]
%\setmainfont[BoldFont=黑体]{宋体}  %缺省英文字体
%\setsansfont
%\setmonofont

%===================目录分栏设置====================
\usepackage[toc,lof,lot]{multitoc}    % 目录(含目录、表格目录、插图目录)分栏设置
  %\renewcommand*{\multicolumntoc}{3} % toc分栏数设置,默认为两栏(\multicolumnlof,\multicolumnlot)
  %\setlength{\columnsep}{1.5cm}      % 调整分栏间距
  \setlength{\columnseprule}{0.2pt}   % 调整分栏竖线的宽度

%==================章节格式设置====================
\setcounter{secnumdepth}{3} % 章节等编号深度 3:子子节\subsubsection
\setcounter{tocdepth}{2}    % 目录显示等度 2:子节

\xeCJKsetup{%
  CJKecglue=\hspace{0.15em},      % 调整中英(含数字)间的字间距
  %CJKmath=true,                  % 在数学环境中直接输出汉字(不需要\text{})
  AllowBreakBetweenPuncts=true,   % 允许标点中间断行,减少文字行溢出
}

\ctexset{%
  part={
    name={,篇},
    number=\SZX{part},
    format={\chuhao\bfseries\centering},
    nameformat={},titleformat={}
  },
  section={
    number={\chinese{section}},
    name={第,节}
  },
  subsection={
    number={\chinese{subsection}、},
    aftername={\hspace{-0.01em}}
  },
  subsubsection={
    number={(\chinese{subsubsection})},
    aftername={\hspace {-0.01em}},
    beforeskip={1.3ex minus .8ex},
    afterskip={1ex minus .6ex},
    indent={\parindent}
  },
  paragraph={
    beforeskip=.1\baselineskip,
    indent={\parindent}
  }
}

\newcommand*\SZX[1]{%
  \ifcase\value{#1}%
    \or 上%
    \or 中%
    \or 下%
  \fi
}

%====================页眉设置======================
\usepackage{titleps}%或者\usepackage{titlesec},titlesec包含titleps
\newpagestyle{special}[\small\sffamily]{
  %\setheadrule{.1pt}
  \headrule
  \sethead[\usepage][][\chaptertitle]
  {\chaptertitle}{}{\usepage}
}

\newpagestyle{main}[\small\sffamily]{
  \headrule
  %\sethead[\usepage][][第\thechapter 章\quad\chaptertitle]
%  {\thesection\quad\sectiontitle}{}{\usepage}}
  \sethead[\usepage][][第\chinese{chapter}章\quad\chaptertitle]
  {第\chinese{section}节\quad\sectiontitle}{}{\usepage}
}

\newpagestyle{main2}[\small\sffamily]{
  \headrule
  \sethead[\usepage][][第\chinese{chapter}章\quad\chaptertitle]
  {第\chinese{section}節\quad\sectiontitle}{}{\usepage}
}

%================ PDF 书签设置=====================
\usepackage{bookmark}[
  depth=2,        % 书签深度 2:子节
  open,           % 默认展开书签
  openlevel=2,    % 展开书签深度 2:子节
  numbered,       % 显示编号
  atend,
]
  % 相比hyperref,bookmark宏包大多数时候只需要编译一次,
  % 而且书签的颜色和字体也可以定制。
  % 比hyperref 更专业 (自动加载hyperref)

%\bookmarksetup{italic,bold,color=blue} % 书签字体斜体/粗体/颜色设置

%------------重置每篇章计数器,必须在hyperref/bookmark之后------------
\makeatletter
  \@addtoreset{chapter}{part}
\makeatother

%------------hyperref 超链接设置------------------------
\hypersetup{%
  pdfencoding=auto,   % 解决新版ctex,引起hyperref UTF-16预警
  colorlinks=true,    % 注释掉此项则交叉引用为彩色边框true/false
  pdfborder=001,      % 注释掉此项则交叉引用为彩色边框
  citecolor=teal,
  linkcolor=myblue,
  urlcolor=black,
  %psdextra,          % 配合使用bookmark宏包,可以直接在pdf 书签中显示数学公式
}

%------------PDF 属性设置------------------------------
\hypersetup{%
  pdfkeywords={黄帝内经,内经,内经讲义,21世纪课程教材},    % 关键词
  %pdfsubject={latex},        % 主题
  pdfauthor={主编:王洪图},   % 作者
  pdftitle={内经讲义},        % 标题
  %pdfcreator={texlive2011}   % pdf创建器
}

%------------PDF 加密----------------------------------
%仅适用于xelatex引擎 基于xdvipdfmx
%\special{pdf:encrypt ownerpw (abc) userpw (xyz) length 128 perm 2052}

%仅适用于pdflatex引擎
%\usepackage[owner=Donald,user=Knuth,print=false]{pdfcrypt}

%其他可使用第三方工具 如:pdftk
%pdftk inputfile.pdf output outputfile.pdf encrypt_128bit owner_pw yourownerpw user_pw youruserpw

%=============自定义环境、列表及列表设置================
% 标题
\def\biaoti#1{\vspace{1.7ex plus 3ex minus .2ex}{\bfseries #1}}%\noindent\hei
% 小标题
\def\xiaobt#1{{\bfseries #1}}
% 小结
\def\xiaojie {\vspace{1.8ex plus .3ex minus .3ex}\centerline{\large\bfseries 小\ \ 结}\vspace{.1\baselineskip}}
% 作者
\def\zuozhe#1{\rightline{\bfseries #1}}

\newcounter{yuanwen}    % 新计数器 yuanwen
\newcounter{jiaozhu}    % 新计数器 jiaozhu

\newenvironment{yuanwen}[2][【原文】]{%
  %\biaoti{#1}\par
  \stepcounter{yuanwen}   % 计数器 yuanwen+1
  \bfseries #2}
  {}

\usepackage{enumitem}
\newenvironment{jiaozhu}[1][【校注】]{%
  %\biaoti{#1}\par
  \stepcounter{jiaozhu}   % 计数器 jiaozhu+1
  \begin{enumerate}[%
    label=\mylabel{\arabic*}{\circledctr*},before=\small,fullwidth,%
    itemindent=\parindent,listparindent=\parindent,%labelsep=-1pt,%labelwidth=0em,
    itemsep=0pt,topsep=0pt,partopsep=0pt,parsep=0pt
  ]}
  {\end{enumerate}}

%===================注解与原文相互跳转====================
%----------------第1部分 设置相互跳转锚点-----------------
\makeatletter
  \protected\def\mylabel#1#2{% 注解-->原文
    \hyperlink{back:\theyuanwen:#1}{\Hy@raisedlink{\hypertarget{\thejiaozhu:#1}{}}#2}}

  \protected\def\myref#1#2{% 原文-->注解
    \hyperlink{\theyuanwen:#1}{\Hy@raisedlink{\hypertarget{back:\theyuanwen:#1}{}}#2}}
  %此处\theyuanwen:#1实际指thejiaozhu:#1,只是\thejiaozhu计数器还没更新,故使用\theyuanwen计数器代替
\makeatother

\protected\def\myjzref#1{% 脚注中的引用(引用到原文)
  \hyperlink{\theyuanwen:#1}{\circlednum{#1}}}

\def\sb#1{\myref{#1}{\textsuperscript{\circlednum{#1}}}}    % 带圈数字上标

%----------------第2部分 调整锚点垂直距离-----------------
\def\HyperRaiseLinkDefault{.8\baselineskip} %调整锚点垂直距离
%\let\oldhypertarget\hypertarget
%\makeatletter
%  \def\hypertarget#1#2{\Hy@raisedlink{\oldhypertarget{#1}{#2}}}
%\makeatother

%====================带圈数字列表标头====================
\newfontfamily\circledfont[Path = fonts/]{meiryo.ttc}  % 日文字体,明瞭体
%\newfontfamily\circledfont{Meiryo}  % 日文字体,明瞭体

\protected\def\circlednum#1{{\makexeCJKinactive\circledfont\textcircled{#1}}}

\newcommand*\circledctr[1]{%
  \expandafter\circlednum\expandafter{\number\value{#1}}}
\AddEnumerateCounter*\circledctr\circlednum{1}

% 参考自:http://bbs.ctex.org/forum.php?mod=redirect&goto=findpost&ptid=78709&pid=460496&fromuid=40353

%======================插图/tikz图========================
\usepackage{graphicx,subcaption,wrapfig}    % 图,subcaption含子图功能代替subfig,图文混排
  \graphicspath{{img/}}                     % 设置图片文件路径

\def\pgfsysdriver{pgfsys-xetex.def}         % 设置tikz的驱动引擎
\usepackage{tikz}
  \usetikzlibrary{calc,decorations.text,arrows,positioning}

%---------设置tikz图片默认格式(字号、行间距、单元格高度)-------
\let\oldtikzpicture\tikzpicture
\renewcommand{\tikzpicture}{%
  \small
  \renewcommand{\baselinestretch}{0.2}
  \linespread{0.2}
  \oldtikzpicture
}

%=========================表格相关===============================
\usepackage{%
  multirow,                   % 单元格纵向合并
  array,makecell,longtable,   % 表格功能加强,tabu的依赖
  tabu-last-fix,              % "强大的表格工具" 本地修复版
  diagbox,                    % 表头斜线
  threeparttable,             % 表格内脚注(需打补丁支持tabu,longtabu)
}

%----------给threeparttable打补丁用于tabu,longtabu--------------
%解决方案来自:http://bbs.ctex.org/forum.php?mod=redirect&goto=findpost&ptid=80318&pid=467217&fromuid=40353
\usepackage{xpatch}

\makeatletter
  \chardef\TPT@@@asteriskcatcode=\catcode`*
  \catcode`*=11
  \xpatchcmd{\threeparttable}
    {\TPT@hookin{tabular}}
    {\TPT@hookin{tabular}\TPT@hookin{tabu}}
    {}{}
  \catcode`*=\TPT@@@asteriskcatcode
\makeatother

%------------设置表格默认格式(字号、行间距、单元格高度)------------
\let\oldtabular\tabular
\renewcommand{\tabular}{%
  \renewcommand\baselinestretch{0.9}\small    % 设置行间距和字号
  \renewcommand\arraystretch{1.5}             % 调整单元格高度
  %\renewcommand\multirowsetup{\centering}
  \oldtabular
}
%设置行间距,且必须放在字号设置前 否则无效
%或者使用\fontsize{<size>}{<baseline>}\selectfont 同时设置字号和行间距

\let\oldtabu\tabu
\renewcommand{\tabu}{%
  \renewcommand\baselinestretch{0.9}\small    % 设置行间距和字号
  \renewcommand\arraystretch{1.8}             % 调整单元格高度
  %\renewcommand\multirowsetup{\centering}
  \oldtabu
}

%------------模仿booktabs宏包的三线宽度设置---------------
\def\toprule   {\Xhline{.08em}}
\def\midrule   {\Xhline{.05em}}
\def\bottomrule{\Xhline{.08em}}
%-------------------------------------
%\setlength{\arrayrulewidth}{2pt} 设定表格中所有边框的线宽为同样的值
%\Xhline{} \Xcline{}分别设定表格中水平线的宽度 makecell包提供

%表格中垂直线的宽度可以通过在表格导言区(preamble),利用命令 !{\vrule width1.2pt} 替换 | 即可

%=================图表设置===============================
%---------------图表标号设置-----------------------------
\renewcommand\thefigure{\arabic{section}-\arabic{figure}}
\renewcommand\thetable {\arabic{section}-\arabic{table}}

\usepackage{caption}
  \captionsetup{font=small,}
  \captionsetup[table] {labelfont=bf,textfont=bf,belowskip=3pt,aboveskip=0pt} %仅表格 top
  \captionsetup[figure]{belowskip=0pt,aboveskip=3pt}  %仅图片 below

%\setlength{\abovecaptionskip}{3pt}
%\setlength{\belowcaptionskip}{3pt} %图、表题目上下的间距
\setlength{\intextsep}   {5pt}  %浮动体和正文间的距离
\setlength{\textfloatsep}{5pt}

%====================全文水印==========================
%解决方案来自:
%http://bbs.ctex.org/forum.php?mod=redirect&goto=findpost&ptid=79190&pid=462496&fromuid=40353
%https://zhuanlan.zhihu.com/p/19734756?columnSlug=LaTeX
\usepackage{eso-pic}

%eso-pic中\AtPageCenter有点水平偏右
\renewcommand\AtPageCenter[1]{\parbox[b][\paperheight]{\paperwidth}{\vfill\centering#1\vfill}}

\newcommand{\watermark}[3]{%
  \AddToShipoutPictureBG{%
    \AtPageCenter{%
      \tikz\node[%
        overlay,
        text=red!50,
        %font=\sffamily\bfseries,
        rotate=#1,
        scale=#2
      ]{#3};
    }
  }
}

\newcommand{\watermarkoff}{\ClearShipoutPictureBG}

\watermark{45}{15}{草\ 稿}    %启用全文水印

%=============花括号分支结构图=========================
\usepackage{schemata}

\xpatchcmd{\schema}
  {1.44265ex}{-1ex}
  {}{}

\newcommand\SC[2] {\schema{\schemabox{#1}}{\schemabox{#2}}}
\newcommand\SCh[4]{\Schema{#1}{#2}{\schemabox{#3}}{\schemabox{#4}}}

%=======================================================

\begin{document}
\pagestyle{main}
\fi
\chapter{《黄帝内经》的地理医学与气象医学思想}%第三章

人与自然息息相关,地理环境、自然气候时刻都在影响人体,因此只有“上知天文,下知地理,中知人事”,医学理论才“可以长久”(《素问·气交变大论》)。故《黄帝内经》在讨论医学理论的同时,也结合讨论地理、气象等相关学科的内容。

\section{《内经》的地理医学思想}%第一节

\subsection{《内经》地理医学思想的基本内容}%一、

地理医学主要是研究地理区域内的各种自然因素、社会经济条件以及地区生活习惯与人类健康关系的科学。《内经》中的地理医学思想,主要介绍自然地理环境的区划,讨论地理气候、物候物产、生活环境等对人的体质及生理病理影响,并探求地理生活环境与健康长寿的关系及因地制宜防治疾病的原则和方法。

\subsubsection{《内经》对自然地理环境的区划}%(一)

《内经》对自然地理环境的区划,主要有阴阳、五方、九野的不同,最常用的仍是五方区划法。

1.阴阳:《内经》对自然地理的区划应用了阴阳分类方法,如《素问·五常政大论》云:“天不足西北,故西北方阴也。……地不满东南,故东南方阳也。”原文还以天地阴阳与人体阴阳相应的观点,进一步将天地东南西北分阴阳,人之上下左右亦分阴阳,同时明确指出南北高下之地,有寒热温凉的气候差异,主要是由于“阴阳之气,高下之理,太少之异也”的缘故。因为“东南方,阳也,阳者其精降于下,故右热而左温;西北方,阴也,阴者其精奉于上,故左寒而右凉。是以地有高下,气有温凉,高者气寒,下者气热。”运用阴阳辩证观点,科学地解释了地域不同,气候亦异的自然现象。

2.五方:五方区划最早见于殷商时期的甲骨文中,那时已经开始用五方观念来确定空间方位,而且发现不同方位的风雨与不同的气候有密切联系,进而把春夏秋冬四时风雨气候的变化与五个空间方位联系起来,通过整体观察,逐步认识了五个方位的重要意义。如春秋战国时期的地理学著作《山海经》中之“山经”,即根据山脉的分布,把中国大地划分为中、南、西、北、东五大区域,详细介绍了不同区域的河流、物产、物候。而《素问·异法方宜论》亦采用五方区划法,分别论述了东西北南中五方各自的地理位置、地形地貌、水文、气候、物候、物产以及人的生活习俗、体质特点与发病、治疗情况,初步概括了自然地理环境及人文地理环境与医疗的关系。

3.九野:九野又称九州,早在《尚书·禹贡》就有论述,书中以山脉、河川、大海等自然环境为依据,将全国划分为冀、兖、青、徐、扬、荆、豫、梁、雍九州,侧重介绍九个州的水文、土壤、植被、湖泽、物产、贡赋和交通情况,是古代重要的地理学文献。《内经》之《生气通天论》、《六节藏象论》、《邪客》、《九针》等篇均载有人体脏腑身形应九野的内容,如《素问·三部九候论》中诊脉法的建立,亦取法自然界“天地人”以应“九野”的思路,将诊脉与自然环境结合起来。此外,《灵枢·九宫八风》还将九宫分野作为模式,立九宫而后知八风,突出反映了人与地理环境,自然气象相应的观点。《素问·六节藏象论》进一步提出了“地以九九制会”的八十一州区划设制,这种大小九州说,是古人对自然地理区域的理想化划分。

\subsubsection{自然地理环境对人体的影响}%(二)

1.自然地理环境对人体生理的影响:自然地理环境对人体生理的影响,主要通过地理环境及其形成的自然气候、生活条件等因素影响人体,使之产生相应的变化,形成带有地域特点的体质类型。

(1)地理与气候:自然地理环境与气候的形成变化关系极为密切,不同的自然地理环境,可以产生不同的气候特点;反之,若气候发生异常变化,也会影响地理环境,使之发生异常改变。如《素问·阴阳应象大论》记述我国五方气候的基本特点,即东方生风,南方生热,西方生燥,北方生寒,中央生湿。又如《素问·五运行大论》列举出异常气候可致地理环境改变的实例:“燥胜则地干,暑胜则地热,风胜则地动,湿胜则地泥,寒胜则地裂,火胜则地固”。九宫图是《灵枢·九宫八风》据九宫分野制作的古代模式图,包括节气、方位、星宿、时辰等方面内容,是以中央和四正、四隅九个地理方位为基础,测定八节循序交换的日期,察看八方气候正常和异常变化,了解其对人体的不同影响,反映了人与天地自然相应的观点。由于是立九宫而后知八风的虚实邪正,故名称“九宫八风”。

(2)地理与物候、物产:《素问·异法方宜论》云:“东方之域,天地之所始生也,鱼盐之地,海滨傍水,……西方者,金玉之域,沙石之处,天地之所收引也,其民陵居而多风,水土刚强,……北方者,天地所闭藏之域也,其地高陵居,风寒冰冽,……南方者,天地之所长养,阳之所盛处也,其地下,水土弱,雾露之所聚也,……中央者,其地平以湿,天地所以生万物也众。”指出东、西、北、南、中五方地域的地理位置、地形地貌、气候特点及其丰富的物产,这些记载,大体符合我国五方地域的特征及其气象物候特点。自然界生物体中的物质成分大都与土壤、空气和水密切相关,不同的地形地貌,形成不同的土质,受不同的气候因素影响,动植物的生长繁殖亦出现一定的差别。由此分析五方地域各自盛产的动植物,其生态及内含的物质成分亦具有明显的差异性,如《素问·金匮真言论》指出:“东方色青,……其畜鸡,其谷麦”,“南方色赤,……其畜羊,其谷黍”,“中央色黄,……其畜牛,其谷稷”,“西方色白,……其畜马,其谷稻”,“北方色黑,……其畜彘,其谷豆”。说明在五方自然要素的长期作用影响下,其所生成的动物或植物各具有不同的特质,显示出代表本区域的特征和形象。《素问·至真要大论》提出“司岁备物”观点,认为采备主岁所化所生之药物,则因得天地精专之化而气全力厚。同样道理,采备适宜种植之地域环境生长的药物,亦可得到气全力厚之效用,如今我们习用的道地药材,如川杜仲、藏红花、杭白菊等皆打上自然环境的烙印。与其他产地相同药材比较,因其有“天地精专”之异同,故能产生质同异等的效应。总之,《内经》认为天复地载,万物悉备,人类正是依赖自然界中各种有利的物质条件而生存。

(3)地理环境与体质:我国幅员辽阔,各地域人体体质差异相当明显,《素问·异法方宜论》载:“东方之域,……其民食鱼而嗜咸,……皆黑色疏理”;“西方者,……水土刚强,其民华食而脂肥”等。说明五方之人的生活习惯及体质特点的形成,直接受着地理、环境、气候等因素的影响,充分反映了“因地异质”思想。《灵枢·阴阳二十五人》根据人体各方面的特征进行系统分类,指出木形人象东方地区之人,火形人象南方地区之人,土形人象中央地区之人,金形人象西方地区之人,水形人象北方地区之人,认为长期生活在不同的地区,其禀赋显示出地域的差异性。

2.自然地理环境对人体病理的影响:自然地理环境不同,其疾病的影响有着相应的差异。

(1)地理环境与时令病:自然地理高下不同,阴阳之气盛衰各异,其影响人体发病,亦表现出地域性疾病多发的倾向。《素问·五常正大论》云:“地有高下,气有温凉,高者气寒,下者气热,故适寒凉者胀,之温热者疮”。指出寒凉之地及寒凉之时节多犯胀病,温热之地及温热季节多患疮疡。

(2)地理环境与地方病:不少疾病发生是因地而异的,《素问·异法方宜论》说:东方之人易患痈疡,西方之人其病生于内,北方之人脏寒生满病,南方之人易病挛痹,中央之人易病痿厥寒热。说明了地域不同,易于发生某些地区性疾病。《内经》同时指出地方性疾病的发生,与地势地质、生活环境及其形成的体质类型等因素亦有较为密切的联系,如《素问·五脏生成篇》云:“多食咸,则脉凝泣而变色”,结合《素问·异法方宜论》旨意演择,东方之域,海滨傍水,盛产鱼盐,其民多食咸,表现出黑色疏理的特质,之所以出现黑色疏理,是因为“盐者胜血”之故。若食咸太过,导致血脉凝泣,逆于肉理而产生痈疡疾患。同样道理,“多食苦,则皮槁而毛拨;多食辛,则筋急而爪枯;多食酸,则肉胝皱而唇揭;多食甘,则骨痛而发落。”这些病虽为五味所伤导致,但其“多食”,仍是受生活环境、地域物产的影响。又如《素问·瘘论》云:“有渐于湿,以水为事,若有所留,居处相湿,肌肉濡渍,痹而不仁,发为肉痿”,“肉痿者,得之湿地也”;《灵枢·百病始生》亦云:“清湿袭虚,则病起于下”。均明确指出湿病产生的原因,是久居湿地或感清湿之气,与地域环境影响密切相关。

(3)地理环境与疫病:现代所指的生物性大气环境污染,古人称之浊气、杂气、毒气、瘴气,是引发疫病的重要因素。“天地迭移,三年化疫”(《素问·刺法论》),主要是由于“气交失易位,气交乃变,变易非常,即四时失序,万化不安,变民病也”,强调在气交失易位的情况下,气候反常,四时失序是疫疠发生的主要原因,但亦不排除地质环境不良,而蕴酿毒气、瘴气的重要作用,如《淮南子·地形训》载“嶂气多喑,风气多聋,林气多隆,木气多伛,岸下气多肿”等等,指出特异的地质环境不仅会引起地域性常见病、多发病,还会导致地域性疫病的发生与流行。

\subsubsection{探索健康长寿的环境原因}%(三)

1.地域环境与健康长寿:《素问·五常政大论》论述地势髙下与气候、物候变化以及人体健康之间的关系时指出:“高者气寒”,“下者气热”,因高者节气来迟,下者节气来早,故物候的变化有迟早之异。人之寿命亦因地而异,“高者其气寿,下者其气夭”。原因是西北地高气寒,阴寒之气用事,致使生化较慢而万物晚成晚衰,元气不易耗散,所以长寿;东南地低气热,阳热之气用事,致使生化较快而万物早成早衰,元气容易耗泄,所以多夭。结合我国第三次人口普査的统计资料,证明这一理论是正确的,如人群中百岁老人的比例及老人寿龄的调查结杲显示,新疆、西藏、青海等离寒地区明显高于国内其它地区。因此,可以肯定地理环境对人的生命过程是有一定影响的。

2.生活环境与健康长寿:《素问·汤液醪醴论》认为“稻米者完,稻薪者坚”,其稻之完、坚主要得益于自然地理环境的培育,故原文进一步解释曰:“此得天地之和,高下之宜,故能至完,伐取得时,故能至坚。”植物的生长,尚需要适宜的生长环境。人生活在大自然中,除地理环境适宜外,更需要不断地从自然环境中获得维持生命活动的物质。“天食人以五气,地食人以五味”,五气五味为人所用,才能“气和而生,津液相成,神乃自生”(《素问·六节藏象论》)。同时机体对营养的需求是多方面的,不同方域的自然环境培育出种类繁多的动植物,这些动植物大都具有生长环境所禀赋的特征,营养成分也不完全相同,《素问·脏气法时论》将与人类生活关系密切的动植物按各自的禀赋特征作了归类,即“肝色青,宜食甘,粳米、牛肉、枣、葵皆甘;心色赤,宜食酸,小豆、犬肉、李、韭皆酸;肺色白,宜食苦,麦、羊肉、杏、薤皆苦;脾色黄,宜食咸,大豆、豕肉、粟、藿皆咸;肾色黑,宜食辛,黄黍、鸡肉、排、葱皆辛。”并总结各类动植物对人体的作用分别是“五谷为养,五果为助。五畜为益,五莱为充,气味合而服之,以补精益气”。旨在运用禀天地之气而生成的各种动植物,据人体所需选择调配,以其所宜,补养五脏,颐养人体,却病延年。

\subsection{《内经》地理医学思想在治疗中的应用}%二、

\subsubsection{因地制宜的治则}%(一)

《素问·五常政大论》说:“西北之气散而寒之,东南之气收而温之,所谓同病异治也”。指出虽是同一种病,对不同地区的患者,应采取不同的治疗方法。而《素问·异法方宜论》则云:“一病,治各不同,皆愈”,说明同一种病,由于患者米自不同的地区,因而采取各自不同的治法,均可治愈疾病。二文从不同角度论证同一主题,突出阐明因地制宜法则的重要性。临床上根据地理环境的不同,对疾病施以因地制宜的治疗,往往可以收到很好的疗效,正如《素问·宝命全形论》说:“若夫法天则地,随应而动,和之者若响,随之者若影”。

\subsubsection{因地制宜的治法}%(二)

《素问·异法方宜论》所载的来自东方的砭石,西方的药物,北方的灸焫,南方的九针,中央的导引按蹻等,都是我国古代劳动人民在同疾病作斗争的过程中,根据各地人们的体质及其地域性多发病的特点,创造出来的适宜于各种不同病证的治疗方法。又《素问·五常政大论》根据“高者气寒”,“适寒凉者胀”的情况,总结出“下之则胀已”之结论;根据“下者气热”,“之温热者疮”的情况,归纳了“汗之则疮已”之论断。本篇还结合东南西北的地域气候特点,针对常见多发的病证,制定适宜的治疗方法,如西北之地“气寒气凉”,人们多因寒邪外束而热郁于内,故治宜“散而寒之”,“治以寒凉,行水渍之”;东南之地“气温气热”,人们多因阳气外泄而内生虚寒,治宜“收而温之”,“治以温热,强其内守”。这些治疗方法都是因地制宜的具体体现。

\section{《内经》的气象医学思想}%第二节

气象医学主要是研究气象因素对人体生理、病理的影响,并为诊断和防治疾病服务的一门边缘学科。《黄帝内经》有关气象医学的内容颇为丰富,是研究气象医学的宝贵资料。

\subsection{《内经》气象医学思想内容}%一、

\subsubsection{《内经》气象医学思想研究的内容和方法}%(一)

《内经》作者将自然界中发生的诸多气象活动与医学理论紧密地结合在一起,使之成为医学理论中一个不可忽视的内容。《内经》主要是以大气环境中常见的云、雨、风、寒、暑、湿、燥、火等气象因素及其对自然界的影响为研究对象,通过整体动态及全面系统的观察方法进行研究,由于历史条件的限制,故其研究具有直观、经验的特点。

\subsubsection{《内经》气象医学思想的主要内容}%(二)

《内经》中涉及的古气象内容,主要有气交、大气现象和季节气候划分三个方面。

1.气交:气交是天气与地气的交会之处,也就是人类生存的空间。人类所赖以生存的地球是被大气包裹着的,《素问·五运行大论》说:“帝曰:地之为下,否乎?岐伯曰:地为人之下,太虚之中者也。帝曰:冯乎?岐伯曰:大气举之也。”《素问·六微旨大论》亦说:“上下之位,气交之中,人之居也。”地位于人之下,人类生活的大气层正处于气交之中,主要的大气现象都在这一空间发生,这一自然环境与人类生活关系最为密切,大气时时处在升降不息的运动状态。《素问·六微旨大论》指出:“帝曰;其升降何如?岐伯曰:气之升降,天地之更用也。帝曰:愿闻其用何如?岐伯曰:升已而降,降者谓天;降已而升,升者谓地。天气下降,气流于地;地气上升,气腾于天。故高下相召,升降相因,而变作矣”。并将升降运动的大气分为阴阳两类,如《素问·阴阳应象大论》所云:“积阳为天,积阴为地”,“清阳为天,浊阴为地。”而阴阳两气的交感互用,则产生多种复杂的气象变化。即《素问·五运行大论》所说:“燥以干之,暑以蒸之,风以动之,湿以润之,寒以坚之,火以温之。故风寒在下,燥热在上,湿气在中,火游行其间,寒暑六入,故令虚而生化也。”正由于风、寒、暑、湿、燥、火等各种气象因素随着天体的循环而运行,使自然界发生周期性、节律性改变,直接影响着万物的生生化化和人类的生存。

2.大气现象:《内经》中涉及的大气现象主要有寒、温、风、雨、云、雾、露、霜、雪、雹、雷等。

(1)寒温:寒温是重要气象因素之一,与太阳照射直接相关。《内经》非常重视太阳的光热作用,《素问·生气通天论》强调:“天运当以日光明”,太阳光的照射是生物获得热能的源泉。阳盛则热,阴盛则寒,自然界随着阳气的盛衰消长而表现出温热、寒凉的温度变化。自然气温寒热与人的生理活动关系极为密切。例如《素问·离合真邪论》:“天地温和,则经水安静;天寒地冻,则经水凝泣;天暑地热,则经水沸溢。”随着太阳的运动,气温产生日变化和年变化的规律及气温分布的地域差别,如《灵枢·一日分为四时》说:“以一日分为四时,朝则为春,日中为夏,日入为秋,夜半为冬”。《素问·厥论》则说:“春夏则阳气多而阴气少,……秋冬则阴气盛而阳气衰”。这两种气温变化具有相互对应的特点,连续运转两次便可形成一个周期。《素问·异法方宜论》指出了东方之域为“天地之所始生”,主温;西方之域为“天地之所收引”,主凉;北方之域为“天地所闭藏”,主寒;南方之域为“天地所长养”,主热。可见《内经》对气温地域分布的认识,也符合阴阳的规律。

(2)风:风也是《内经》中研究气象的一个重要内容。《灵枢·九宫八风》是论述风最详尽的篇章。该篇将风分为实风和虚风两种,并提出八风的概念。实风从所居之乡来,是每一季节所出现的当令的风向,主生长,养育万物。如春多东风,夏多南风,秋多西风,冬多北风等。虚风从其冲后来,是时令与风向的相互对冲,常摧残万物,伤人致病。如十一月(子位)刮南风(午位),为子午相冲;二月(卯位)刮西风(酉位),为卯酉相冲等。八风是指来自八个方向的风,属虚风之列,《九宫八风》对八风分别命名,如“风从南方来,名曰大弱风”,“风从西南方来,名曰谋风”等。

(3)云雾、霜露、雨雪:对于云雾、霜露、雨雷等气象要素,《内经》进行了粗略的解释。如《素问·阴阳应象大论》“地气上为云,天气下为雨;雨出地气,云出天气”;《素问·六元正纪大论》“阳明所至,为收为雾露”;《素问·气交变大论》“雨水雪霜不时降”;《素问·脉解》“秋气始上,微霜始下,而方杀万物”等等,这些认识都是直观经验的记载。

3.季节气候的划分:《内经》对气候的时段区划,主要表现在两个方面:一是有候、气、时、岁的不同,即《素问·六节藏象论》所云:“五日谓之候,三候谓之气,六气谓之时,四时谓之岁”。二是运气学说所运用的主运、主气和二十四节气。综合讨论如下:

(1)四时:四时,指春夏秋冬四季,主要反映气温的年周期变化。四季气候变化,寒暑往来,万物随之而有生长收藏的活动。如《素问·四气调神大论》载“春三月,此谓发陈,天地俱生,万物以荣”;“夏三月,此谓蕃秀,天地气交,万物华实”、“秋三月,此谓容平,天气以急,地气以明”;“冬三月,此谓闭藏,水冰地坼”,即用“发陈”、“蕃秀”、“容平”和“闭藏”,概括了万物四季的生化特征。

(2)主运(五季):是按照气候的特征将一年划分为五个阶段,在四季温热凉寒基础上增加了湿。五运每个阶段各主七十三日另五刻,每年约从大寒开始为初运木,主风;春分后十三日起为二运火,主热;芒种后十日起为三运土,主湿;处暑后七日起为四运金,主燥;立冬后四日起为终运水,主寒。

(3)主气(六季):也是按照气候特征将一年划分为六个阶段。六气按五行相生次序,分为六步,每步各有相应的气与之配合,分别为风、热、暑、湿、燥、寒六气,每步约主六十日又八十七刻半,包括四个节气。

(4)二十四节气:二十四节气是用来表示季节的交替和气候变化的时段。它反映季节变化,气温高低,霜露雨雪及其物候规律。二十四节气的确立,是古人用圭表测定日影的方法,测出冬至和夏至,春分和秋分,往后又测出四立,直至二十四节气逐步完善。

(5)候:为计算气候变化最小的区划单位。即在五日中随着气候逐渐变化,万物随之而产生相应的变动,如《礼记·月令》:“立春节,初五日,东风解冻;次五日,蛰虫始振”等。

4.五运六气的天气预测:五运六气根据纪年干支推算六十年的气候周期变化,是《内经》运用医学气象知识进行长期天气预测的实践总结。它的基本概念是五运、六气、十天干、十二地支、六十甲子等。应用纪年的天干地支与五运六气结合进行推算,从宏观方面反映了气候的周期性变化,从而作出长期天气预测。至于五运六气的推算方法,六十年气候变化规律,下篇第八章有详尽介绍,可参阅。

\subsection{《内经》气象医学思想在中医理论体系中的意义}%二、

\subsubsection{《内经》气象医学思想与生理}%(一)

人的生理功能活动,随自然气象春夏秋冬转换而发生生长收藏的相应变化,这些变化主要通过精神活动、脏腑功能,经脉循行,气血运行,津液代谢等方面反映出来。

1.气象与精神活动:《素问·阴阳应象大论》说:“天有四时五行,以生长收藏,以生寒暑燥湿风,人有五脏化五气,以生喜怒悲忧恐”,指出自然阴阳消长,气候寒暑更迭,直接影响脏腑功能活动,使之产生不同的精神情志活动。《素问·四气调神大论》还提出顺应四时气象规律来调摄精神意志活动,认为是“治未病”的关键所在。

2.气象与脏腑功能:“藏象”一词,首见于《素问·六节藏象论》,文中指出:“心者,生之本……为阳中之太阳,通于夏气;肺者,气之本……为阳中之太阴,通于秋气;肾者,主蛰,封藏之本……为阴中之少阴,通于冬气;肝者,罢极之本……为阳中之少阳,通于春气”,说明五脏功能活动具有适应气候的内涵。又《素问·刺禁论》说:“脏有要害,不可不察,肝生于左,肺藏于右,心部于表,肾治于里,脾为之使,胃为之市。”王冰注:“肝象木,王于春,春阳发生,故生于左也。肺象金,王于秋,秋阴收杀,故藏于右也。阳气主外,心象火也。阴气主内,肾象水也。”表明脏腑功能与五方气候有着特定的内在联系,即“五脏应四时,各有收受”(《素问·金匮真言论》。

3.气象与经气活动:《内经》认为经脉的运行,经气的活动,同样受自然气象的影响。《素问·四时刺逆从论》说:“春气在经脉,夏气在孙络,长夏气在肌肉,秋气在皮肤,冬气在骨髓中”。并解释说:“春者,天气始开,地气始泄,冻解永释,水行经通,故人气在脉;夏者,经满气溢,入孙络受血,皮肤充实;长夏者,经络皆盛,内溢肌中;秋者,天气始收,腠理闭塞,皮肤引急;冬者盖戴,血气在中,内着骨髓,通于五脏。”指出人身经脉之气的运行及所在的位置,随四时气候的更迭而变华。《灵枢·阴阳系日月》中还介绍了一年十二个月,阴阳寒热具有消长转化的关系,并以十二个月与十二经配合,通过气候寒温变化来认识经脉之气衰旺的规律。

4.气象与气血运行:《素问·八正神明论》说:“是故天温日明,则人血淖液而卫气浮,故血易泻,气易行;天寒日阴,则人血凝泣而卫气沉”,可见外界气候的变化对人体气血运行的状态影响是显著的。气血运行受气候影响,亦可通过脉象反应出来,如《素问·脉要精微论》云:“四变之动,脉与之上下,以春应中规,夏应中矩,秋应中衡,冬应中权”。还说:“春日浮,如鱼之游在波;夏日在肤,泛泛乎万物有余;秋日下肤,蛰虫将去;冬日在骨,蛰虫周密,君子居室”。

5.气象与津液代谢:《灵枢·五癃津液别》指出:“天暑衣厚则腠理开,故汗出,……天寒则腠理闭,气湿不行,水下留于膀胱,则为溺与气。”春夏阳气发泄,皮肤松弛,所以多汗;秋冬阳气收藏,皮肤致密,所以少汗多溺。说明人体津液代谢的过程,同样受着气候因素的影响。

\subsubsection{《内经》气象医学思想与发病}%(二)

1.气候病因与发病的关系:《素问·五运行大论》说:“五气更立,各有所先,非其位则邪,当其位则正。”《素问·六微旨大论》也说:“其有至而至,有至而不至,有至而大过……至而至者和,至而不至,来气不及也,末至而至,来气有余也。”认为自然界风、热、暑、湿、燥、寒六种气候变化,只有太过或不及,或非其时而有其气,或气候急骤变化,才能成为异常气候,侵犯人体而致病。异常气候是引起疾病发生的外部原因,合称之为六淫。当然六淫是否引起疾病的发生,还要取决于人体正气的强弱。即《灵枢·百病始生》所说:“两虚相得,乃客其形”。与此同时,又从另一方面阐述邪气对发病的重要作用,如《素问·五运行大论》说:“气相得则和,不相得则病……从其气则和,违其气则病”。这种既强调正气又重视邪气的辩证学思想,至今仍有着重要的指导意义。

2.气候病因及其致病特点:《素问·至真要大论》说:“夫百病之生也,皆生于风寒暑湿燥火,以之化之变也。”不同的异常气候,具不同的性质特点,即《素问·五常政大论》谓“寒热燥湿,不同其化也,”分而言之,则如《素问·五运行大论》云:“燥以干之,暑以蒸之,风以动之,湿以润之,寒以坚之,火以温之,……故燥胜则地干,暑胜则地热,风胜则地动,湿胜则地泥,寒胜则地裂,火胜则地固矣”。当六淫侵犯人体而致病,即可表现出不同的病证特点,如《素问·阴阳应象大论》指出:“风胜则动,热胜则肿,燥胜则干,寒胜则浮,湿胜则濡泻”。《素问·生气通天论》亦有具体论述:“因于寒,欲如运枢,起居加惊,神气乃浮,因干暑,汗,烦则喘渴,静则多言,体若燔炭,汗出而散。因于湿,首如裹,湿热不攘,大筋软短,小筋弛长,软短为拘,弛长为痿”,“四时之气,更伤五脏。”后世将《内经》六淫的这些性质和致病特点进行归纳,而指导临床实践。

3.气候病因与发病部位:四时气候病因的性质特点不同,其发病部位各异。《灵枢·四时气》说:“四时之气,各不同形,百病之起,皆有所生。”从总体上指出四时气候病因与发病部位有分属的相关性。这一相关性,既适用于内脏病变,如《素问·咳论》说;“乘秋则肺先受邪,乘春则肝先受之,乘夏则心先受之,乘至阴则脾先受之,乘冬则肾先受之。”亦适用于身形疾患,如《素问·金匮真言论》云:“故春气者,病在头,夏气者,病在脏,秋气者,病在肩背,冬气者,病在四肢”。

4.气候病因与时令病:四时气候春温、夏热、秋凉、冬寒,各有不同特点,机体在异常气候的影响下,容易发生季节性疾病。《素问·金匮真言论》说:“春善病鼽衄,仲夏善病胸胁,长夏善病洞泄寒中,秋善病风疟,冬善病痹厥。”《素问·阴阳应象大论》说:“冬伤于寒,春必温病;春伤于风,夏为飧泄;夏伤于暑,秋必痎疟;秋伤于湿,冬生咳嗽。”前者是季节性多发病,后者是感邪伤正而延缓发病,但都与四时气候病因相关。此外,《内经》还讨论了疾病季节性分类的问题。《素问·热论》说:“凡病伤寒而成温者,先夏至日者为病温,后夏至日者为病暑。”用以区分病性相同而发病时间不同,治疗方法有别的病证。

5.气候病因与地方病:《素问·异法方宜论》在介绍地势特征、气候特点、民风习俗的基础上,提出了东方之人多病痈疡、西方之人多病生于内、北方之人多病胀满、南方之人多病挛痹、中部之人多病痿厥寒热的认识,是地域气候与地方性多发病相关的典型例子。

6.气候病因与疫病:《素问·刺法论》、《索问·本病论》中所述,连续三年的气候异常,可导致人体发生如水疠、金疠、木疠、火疠、土疠等疫病。发病原因主要是由于“气交失易位,气交乃变,变易非常,即四时失序,万化不安,变民病也。”同时强调疫疠的发生,有一个渐变过程,是在气候反复失常条件下,又逢个体正气虚弱,神气失守才会发病,即《灵枢·岁露》所云的“三虚”。此“三虚”相互结合,可产生暴亡的疫病。《内经》指出疫疠发生有其复杂的气候条件,而不同于一般的气候失常,为后世进一步研究疫疠的发病原因,作了重要提示。

7.超年气候与发病:气候的变化,除了因季节的递迁而不同外,还有超年的气候特点,这类气候与疾病发生关系亦异常密切。超年的气候特点可以应用五运六气推算中运的方法进行预测,如甲己为土运,乙庚为金运,丙辛为水运,丁壬为木运,戊癸为火运。凡此十干所统之运,可推测各年的气候特点及其发病规律。

\subsection{《内经》气象医学思想与临床应用}%三、

\subsubsection{气象与疾病的诊断}%(一)

《内经》从色脉与预后方面阐明气象与诊断的关系。

1.气象变化与色脉:察色按脉具有判断死生和决断疑难的重要作用,然而诊察色脉,必须结合阴阳四时的变化规律。就脉而言,《素问·脉要精微论》说“天地之变,阴阳之应,彼春之暖为夏之暑,彼秋之忿为冬之怒,四变之动,脉与之上下。……阴阳有时,与脉为期,期而相失,知脉所分,分之有期,故知死时。”《素问·玉机真脏论》认为春脉如弦,夏脉如钩,秋脉如浮,冬脉如营,揭示了人体平脉随四时气象转换的相应变化规律。若诊得“逆时之脉”,如“春得肺脉,夏得肾脉,秋得心脉,冬得脾脉”或“于春夏而脉沉涩,秋冬而脉浮大”等,是脉不与时相应,为预后不良。以色而论,脏之本色为主,应时之色为客,根据主色和客色的变化,可推测病情的顺逆,即“客胜主善,主胜客恶”(《医宗金鉴·四诊心法要诀》又如《素问·经络论》说:“阴络之色应其经,阳络之色变无常,随四时而行也。寒多则凝泣,凝泣则青黑;热多则淖泽,淖泽则黄赤。……五色俱见者,谓之寒热。”阳络浅表,应时而变。诊察色脉变化,可帮助判定病势顺逆。

2.气候变化与病势转归:对于四时气候变化与疾病转归的关系,《内经》论述颇多,如《素问·阴阳应象大论》说:“阳盛则身热,……能冬不能夏;阴盛则身寒,……能夏不能冬”。指出疾病的发展与四时气候关系极为密切,强调了四时阴阳消长规律对疾病预后的重要影响。又如《素问·三部九候论》说:“帝曰:冬阴夏阳奈何?岐伯曰:九候之脉,皆沉细悬绝者,为阴主冬,故以夜半死;盛躁喘数者,为夏主阳,故以日中死。是故寒热病者,以平旦死。热中及热病者,以日中死。病风者,以日夕死。病水者,以夜半死。”将脉象与证候按寒热虚实分属阴阳,结合昼夜阴阳消长规律,推断死时,这是典型的结合气象判断疾病预后的例子。而《灵枢·顺气一日分为四时》谈到:疾病旦慧、昼安、夕加、夜甚。提示在邪正斗争过程中,正气盛衰与日夜阴阳及气候寒温变化关系十分密切,而且能直接影响疾病的转归与变化。

此外,《内经》指出气候异常致病,在一定条件下其病证性质可循六淫所胜的方向转化,如《素问·六元正纪大论》说:“太阴雨化,施于太阳;太阳寒化,施于少阳;少阴热化,施于阳明;阳明燥化,施于厥阴;厥阴风化,施于太阴,各命其所在以征之也”。六气循五行相胜规律,风向湿,湿向寒,寒向热,热向燥,燥向风方面转化,而病证性质亦可随之改变。由于六淫有“各归不胜而为化”的特点,其所致脏腑器官的病变,由是可发生相应的传化,即肝病传脾,脾病传肾,肾病传心,心病传肺,肺病传肝等,依据这一转化规律,可掌握疾病的传变方向。

\subsubsection{气象与疾病的治疗}%(二)

《内经》根据四时气象特点,提出“必先岁气,无伐天和”(《素问·五常政大论》)的法时而治的思想,内容较为丰富。

1.因时制宜:天时气象对人体生理病理有着必然的联系和影响,故使用药物治疗疾病时,一定要考虑天时气候的变化,遵循法天时而治的原则。

(1)用热远热,用寒远寒:治疗疾病,遣方用药,宜遵循寒者热之,热者寒之的原则,同时注意避免药物性质与气候性质的类从。如《素问·六元正纪大论》说:“用寒远寒、用凉远凉、用温远温、用热远热,食宜同法,”反之则会“不远热而热至,不远寒而寒至。寒至则坚否腹满,痛急下利之病生矣;热至则身热,吐下霍乱,痈疽疮疡,瞀郁注下,瞤瘛肿胀,呕鼽衄,头痛,骨节变,肉痛,血溢血泄,淋闭之病生矣。”(《素问·六元正纪大论》)总之,用药违背时忌,必然加重病情,产生严重后果。

(2)春夏养阳,秋冬养阴:《素问·四气调神大论》云:“夫四时阴阳者,万物之根本也,所以圣人春夏养阳,秋冬养阴,以从其根,故与万物沉浮于生长之门”。由此确立了春夏顺应生长之气以养阳,秋冬顺从收藏之气以养阴的顺时养生法则。后世医家引申作为临床用药的指南。

(3)四时分刺:由于时令气候变化可导致人体经气分布重心的改变,所以针刺必须考虑时令气候的影响。《灵枢·四时气》指出:“四时之气,各有所在,灸刺之道,得气穴为定。故春取经血脉分肉之间,甚者深刺之,间者浅刺之。夏取盛经孙络,取分间,绝皮肤。秋取经腧,邪在府,取之合。冬取井荥,必深以留之。”《素问·诊要经终论》说:“春夏秋冬,各有所刺,法其所在。春刺夏分,脉乱气微,入淫骨髓,病不能愈,令人不嗜食,又且少气,……冬刺秋分,病不已,今人善渴。”总之,春夏秋冬四时寒热相异,人气浮沉及邪气所在的部位不同,治疗上应该根据四时阴阳经气所在确定针刺的部位和浅深,用针违背时忌,必会加重病情,产生严重的后果。

2.因地域气候制宜:气候的特点与地势密切相关,不同地域使用药物时宜尽量考虑气候条件的影响。《素问·五常政大论》说:“东南方阳也,阳者其精降于下,故右热而左温;西北方阴也,阴者其精奉于上,故左寒而右凉,……西北之气散而寒之,东南之气收而温之。”东南方气候温热,气泄于外,寒生于内,故宜收其外泄,温其内寒。西北方气候凉寒,寒固于外,热郁于内,故宜散其外寒,清其内热。地势不同,气候差异亦大,对疾病的影响迥异,务必因地域气候用药,方有成效。

3.六淫所胜用药原则:《内经》根据六淫胜复致病的特点,提出五味调治的规律。例如《素问·至真要大论》说:“风淫所胜,平以辛凉,佐以苦甘,以甘缓之,以酸泻之。热淫所胜,平以咸寒,佐以苦甘,以酸收之。湿淫所胜,平以苦热,佐以酸辛,以苦燥之,以淡泄之。湿上甚而热,治以苦温。佐以甘辛,以汗为故而止。火淫所胜,平以酸冷,佐以苦甘,以酸收之,以苦发之,以酸复之,热淫同。燥淫所胜,平以苦湿,佐以酸辛,以苦下之。寒淫所胜,平以辛热,佐以甘苦,以咸泻之。”根据“五味入胃,各归其所喜”(《素问·至真要大论》)和五行生克原则,并结合大量医疗实践总结出来的六淫所胜的五味用药规律,至今对临床实践仍具有非常重要的指导意义。

\subsubsection{气象与养生}%(三)

1.避邪防病:春温、夏热、秋凉、冬寒,是自然界阴阳二气消长转化的结果,人与自然界息息相通,受自然界的影响,人体生理机能随阴阳的消长运动发生相应的规律性变化,当气候发生异常变化,超过人体适应调节能力时,就可能导致疾病的发生。要预防疾病,必须做到“顺四时而适寒暑”(《灵枢·本神》),使之符合阴阳变化之道,增强适应自然气候变化的能力,同时又要防避四时不正之气的侵袭,如《素问·上古天真论》说:“虚邪贼风,避之有时”,《灵枢·九宫八风》亦告诫:“谨候虚风以避之”,至于有传染性的邪气,因其“皆相染易”,更要“避其毒气”(《素问·刺法论》)。

2.四气调神:《内经》强调人的精神调摄,必须顺应四时阴阳消长规律,才能有益于健康,而调神的方法当结合四季进行,如:“春三月,……夜卧早起,广步于庭,被发缓形,以使志生,生而勿杀,予而勿夺,赏而勿罚,此春气之应,养生之道也。”“夏三月,……夜卧早起,无厌于日,使志无怒,使华英成秀,使气得泄,若所爱在外,此夏气之应,养长之道也。”“秋三月、……天气以急,地气以明,早卧早起,与鸡俱兴,使志安宁,以缓秋刑,收敛神气,使秋气平,无外其志,使肺气清,此秋气之应,养收之道也。”“冬三月,……早卧晚起,必待日光,使志若伏若匿,若有私意,若己有得,去寒就温,无泄皮肤,使气亟夺,此冬气之应。养藏之道也。”(《素问·四气调神大论》)其基本精神,就是顺应春季阳气的生发以舒肝气,顺应夏季阳气的旺盛以养心气,顺应秋季阳气的收敛以养肺气,顺应冬季阳气的闭藏以养肾气,维护人和自然的统一,达到健康长寿的目的。

\zuozhe{(叶庆建)}
\ifx \allfiles \undefined
\end{document}
\fi