% -*- coding: utf-8 -*-
%!TEX program = xelatex
\ifx \allfiles \undefined
\documentclass[draft,12pt]{ctexbook}
%\usepackage{xeCJK}
%\usepackage[14pt]{extsizes} %支持8,9,10,11,12,14,17,20pt

%===================文档页面设置====================
%---------------------印刷版尺寸--------------------
%\usepackage[a4paper,hmargin={2.3cm,1.7cm},vmargin=2.3cm,driver=xetex]{geometry}
%--------------------电子版------------------------
\usepackage[a4paper,margin=2cm,driver=xetex]{geometry}
%\usepackage[paperwidth=9.2cm, paperheight=12.4cm, width=9cm, height=12cm,top=0.2cm,
%            bottom=0.4cm,left=0.2cm,right=0.2cm,foot=0cm, nohead,nofoot,driver=xetex]{geometry}

%===================自定义颜色=====================
\usepackage{xcolor}
  \definecolor{mybackgroundcolor}{cmyk}{0.03,0.03,0.18,0}
  \definecolor{myblue}{rgb}{0,0.2,0.6}

%====================字体设置======================
%--------------------中文字体----------------------
%-----------------------xeCJK下设置中文字体------------------------------%
\setCJKfamilyfont{song}{SimSun}                             %宋体 song
\newcommand{\song}{\CJKfamily{song}}                        % 宋体   (Windows自带simsun.ttf)
\setCJKfamilyfont{xs}{NSimSun}                              %新宋体 xs
\newcommand{\xs}{\CJKfamily{xs}}
\setCJKfamilyfont{fs}{FangSong_GB2312}                      %仿宋2312 fs
\newcommand{\fs}{\CJKfamily{fs}}                            %仿宋体 (Windows自带simfs.ttf)
\setCJKfamilyfont{kai}{KaiTi_GB2312}                        %楷体2312  kai
\newcommand{\kai}{\CJKfamily{kai}}
\setCJKfamilyfont{yh}{Microsoft YaHei}                    %微软雅黑 yh
\newcommand{\yh}{\CJKfamily{yh}}
\setCJKfamilyfont{hei}{SimHei}                                    %黑体  hei
\newcommand{\hei}{\CJKfamily{hei}}                          % 黑体   (Windows自带simhei.ttf)
\setCJKfamilyfont{msunicode}{Arial Unicode MS}            %Arial Unicode MS: msunicode
\newcommand{\msunicode}{\CJKfamily{msunicode}}
\setCJKfamilyfont{li}{LiSu}                                            %隶书  li
\newcommand{\li}{\CJKfamily{li}}
\setCJKfamilyfont{yy}{YouYuan}                             %幼圆  yy
\newcommand{\yy}{\CJKfamily{yy}}
\setCJKfamilyfont{xm}{MingLiU}                                        %细明体  xm
\newcommand{\xm}{\CJKfamily{xm}}
\setCJKfamilyfont{xxm}{PMingLiU}                             %新细明体  xxm
\newcommand{\xxm}{\CJKfamily{xxm}}

\setCJKfamilyfont{hwsong}{STSong}                            %华文宋体  hwsong
\newcommand{\hwsong}{\CJKfamily{hwsong}}
\setCJKfamilyfont{hwzs}{STZhongsong}                        %华文中宋  hwzs
\newcommand{\hwzs}{\CJKfamily{hwzs}}
\setCJKfamilyfont{hwfs}{STFangsong}                            %华文仿宋  hwfs
\newcommand{\hwfs}{\CJKfamily{hwfs}}
\setCJKfamilyfont{hwxh}{STXihei}                                %华文细黑  hwxh
\newcommand{\hwxh}{\CJKfamily{hwxh}}
\setCJKfamilyfont{hwl}{STLiti}                                        %华文隶书  hwl
\newcommand{\hwl}{\CJKfamily{hwl}}
\setCJKfamilyfont{hwxw}{STXinwei}                                %华文新魏  hwxw
\newcommand{\hwxw}{\CJKfamily{hwxw}}
\setCJKfamilyfont{hwk}{STKaiti}                                    %华文楷体  hwk
\newcommand{\hwk}{\CJKfamily{hwk}}
\setCJKfamilyfont{hwxk}{STXingkai}                            %华文行楷  hwxk
\newcommand{\hwxk}{\CJKfamily{hwxk}}
\setCJKfamilyfont{hwcy}{STCaiyun}                                 %华文彩云 hwcy
\newcommand{\hwcy}{\CJKfamily{hwcy}}
\setCJKfamilyfont{hwhp}{STHupo}                                 %华文琥珀   hwhp
\newcommand{\hwhp}{\CJKfamily{hwhp}}

\setCJKfamilyfont{fzsong}{Simsun (Founder Extended)}     %方正宋体超大字符集   fzsong
\newcommand{\fzsong}{\CJKfamily{fzsong}}
\setCJKfamilyfont{fzyao}{FZYaoTi}                                    %方正姚体  fzy
\newcommand{\fzyao}{\CJKfamily{fzyao}}
\setCJKfamilyfont{fzshu}{FZShuTi}                                    %方正舒体 fzshu
\newcommand{\fzshu}{\CJKfamily{fzshu}}

\setCJKfamilyfont{asong}{Adobe Song Std}                        %Adobe 宋体  asong
\newcommand{\asong}{\CJKfamily{asong}}
\setCJKfamilyfont{ahei}{Adobe Heiti Std}                            %Adobe 黑体  ahei
\newcommand{\ahei}{\CJKfamily{ahei}}
\setCJKfamilyfont{akai}{Adobe Kaiti Std}                            %Adobe 楷体  akai
\newcommand{\akai}{\CJKfamily{akai}}

%------------------------------设置字体大小------------------------%
\newcommand{\chuhao}{\fontsize{42pt}{\baselineskip}\selectfont}     %初号
\newcommand{\xiaochuhao}{\fontsize{36pt}{\baselineskip}\selectfont} %小初号
\newcommand{\yihao}{\fontsize{28pt}{\baselineskip}\selectfont}      %一号
\newcommand{\xiaoyihao}{\fontsize{24pt}{\baselineskip}\selectfont}
\newcommand{\erhao}{\fontsize{21pt}{\baselineskip}\selectfont}      %二号
\newcommand{\xiaoerhao}{\fontsize{18pt}{\baselineskip}\selectfont}  %小二号
\newcommand{\sanhao}{\fontsize{15.75pt}{\baselineskip}\selectfont}  %三号
\newcommand{\sihao}{\fontsize{14pt}{\baselineskip}\selectfont}%     四号
\newcommand{\xiaosihao}{\fontsize{12pt}{\baselineskip}\selectfont}  %小四号
\newcommand{\wuhao}{\fontsize{10.5pt}{\baselineskip}\selectfont}    %五号
\newcommand{\xiaowuhao}{\fontsize{9pt}{\baselineskip}\selectfont}   %小五号
\newcommand{\liuhao}{\fontsize{7.875pt}{\baselineskip}\selectfont}  %六号
\newcommand{\qihao}{\fontsize{5.25pt}{\baselineskip}\selectfont}    %七号   %中文字体及字号设置
\xeCJKDeclareSubCJKBlock{SIP}{
  "20000 -> "2A6DF,   % CJK Unified Ideographs Extension B
  "2A700 -> "2B73F,   % CJK Unified Ideographs Extension C
  "2B740 -> "2B81F    % CJK Unified Ideographs Extension D
}
%\setCJKmainfont[SIP={[AutoFakeBold=1.8,Color=red]Sun-ExtB},BoldFont=黑体]{宋体}    % 衬线字体 缺省中文字体

\setCJKmainfont{simsun.ttc}[
  Path=fonts/,
  SIP={[Path=fonts/,AutoFakeBold=1.8,Color=red]simsunb.ttf},
  BoldFont=simhei.ttf
]

%SimSun-ExtB
%Sun-ExtB
%AutoFakeBold:自动伪粗,即正文使用\bfseries时生僻字使用伪粗体;
%FakeBold:强制伪粗,即正文中生僻字均使用伪粗体
%\setCJKmainfont[BoldFont=STHeiti,ItalicFont=STKaiti]{STSong}
%\setCJKsansfont{微软雅黑}黑体
%\setCJKsansfont[BoldFont=STHeiti]{STXihei} %serif是有衬线字体sans serif 无衬线字体
%\setCJKmonofont{STFangsong}    %中文等宽字体

%--------------------英文字体----------------------
\setmainfont{simsun.ttc}[
  Path=fonts/,
  BoldFont=simhei.ttf
]
%\setmainfont[BoldFont=黑体]{宋体}  %缺省英文字体
%\setsansfont
%\setmonofont

%===================目录分栏设置====================
\usepackage[toc,lof,lot]{multitoc}    % 目录(含目录、表格目录、插图目录)分栏设置
  %\renewcommand*{\multicolumntoc}{3} % toc分栏数设置,默认为两栏(\multicolumnlof,\multicolumnlot)
  %\setlength{\columnsep}{1.5cm}      % 调整分栏间距
  \setlength{\columnseprule}{0.2pt}   % 调整分栏竖线的宽度

%==================章节格式设置====================
\setcounter{secnumdepth}{3} % 章节等编号深度 3:子子节\subsubsection
\setcounter{tocdepth}{2}    % 目录显示等度 2:子节

\xeCJKsetup{%
  CJKecglue=\hspace{0.15em},      % 调整中英(含数字)间的字间距
  %CJKmath=true,                  % 在数学环境中直接输出汉字(不需要\text{})
  AllowBreakBetweenPuncts=true,   % 允许标点中间断行,减少文字行溢出
}

\ctexset{%
  part={
    name={,篇},
    number=\SZX{part},
    format={\chuhao\bfseries\centering},
    nameformat={},titleformat={}
  },
  section={
    number={\chinese{section}},
    name={第,节}
  },
  subsection={
    number={\chinese{subsection}、},
    aftername={\hspace{-0.01em}}
  },
  subsubsection={
    number={(\chinese{subsubsection})},
    aftername={\hspace {-0.01em}},
    beforeskip={1.3ex minus .8ex},
    afterskip={1ex minus .6ex},
    indent={\parindent}
  },
  paragraph={
    beforeskip=.1\baselineskip,
    indent={\parindent}
  }
}

\newcommand*\SZX[1]{%
  \ifcase\value{#1}%
    \or 上%
    \or 中%
    \or 下%
  \fi
}

%====================页眉设置======================
\usepackage{titleps}%或者\usepackage{titlesec},titlesec包含titleps
\newpagestyle{special}[\small\sffamily]{
  %\setheadrule{.1pt}
  \headrule
  \sethead[\usepage][][\chaptertitle]
  {\chaptertitle}{}{\usepage}
}

\newpagestyle{main}[\small\sffamily]{
  \headrule
  %\sethead[\usepage][][第\thechapter 章\quad\chaptertitle]
%  {\thesection\quad\sectiontitle}{}{\usepage}}
  \sethead[\usepage][][第\chinese{chapter}章\quad\chaptertitle]
  {第\chinese{section}节\quad\sectiontitle}{}{\usepage}
}

\newpagestyle{main2}[\small\sffamily]{
  \headrule
  \sethead[\usepage][][第\chinese{chapter}章\quad\chaptertitle]
  {第\chinese{section}節\quad\sectiontitle}{}{\usepage}
}

%================ PDF 书签设置=====================
\usepackage{bookmark}[
  depth=2,        % 书签深度 2:子节
  open,           % 默认展开书签
  openlevel=2,    % 展开书签深度 2:子节
  numbered,       % 显示编号
  atend,
]
  % 相比hyperref,bookmark宏包大多数时候只需要编译一次,
  % 而且书签的颜色和字体也可以定制。
  % 比hyperref 更专业 (自动加载hyperref)

%\bookmarksetup{italic,bold,color=blue} % 书签字体斜体/粗体/颜色设置

%------------重置每篇章计数器,必须在hyperref/bookmark之后------------
\makeatletter
  \@addtoreset{chapter}{part}
\makeatother

%------------hyperref 超链接设置------------------------
\hypersetup{%
  pdfencoding=auto,   % 解决新版ctex,引起hyperref UTF-16预警
  colorlinks=true,    % 注释掉此项则交叉引用为彩色边框true/false
  pdfborder=001,      % 注释掉此项则交叉引用为彩色边框
  citecolor=teal,
  linkcolor=myblue,
  urlcolor=black,
  %psdextra,          % 配合使用bookmark宏包,可以直接在pdf 书签中显示数学公式
}

%------------PDF 属性设置------------------------------
\hypersetup{%
  pdfkeywords={黄帝内经,内经,内经讲义,21世纪课程教材},    % 关键词
  %pdfsubject={latex},        % 主题
  pdfauthor={主编:王洪图},   % 作者
  pdftitle={内经讲义},        % 标题
  %pdfcreator={texlive2011}   % pdf创建器
}

%------------PDF 加密----------------------------------
%仅适用于xelatex引擎 基于xdvipdfmx
%\special{pdf:encrypt ownerpw (abc) userpw (xyz) length 128 perm 2052}

%仅适用于pdflatex引擎
%\usepackage[owner=Donald,user=Knuth,print=false]{pdfcrypt}

%其他可使用第三方工具 如:pdftk
%pdftk inputfile.pdf output outputfile.pdf encrypt_128bit owner_pw yourownerpw user_pw youruserpw

%=============自定义环境、列表及列表设置================
% 标题
\def\biaoti#1{\vspace{1.7ex plus 3ex minus .2ex}{\bfseries #1}}%\noindent\hei
% 小标题
\def\xiaobt#1{{\bfseries #1}}
% 小结
\def\xiaojie {\vspace{1.8ex plus .3ex minus .3ex}\centerline{\large\bfseries 小\ \ 结}\vspace{.1\baselineskip}}
% 作者
\def\zuozhe#1{\rightline{\bfseries #1}}

\newcounter{yuanwen}    % 新计数器 yuanwen
\newcounter{jiaozhu}    % 新计数器 jiaozhu

\newenvironment{yuanwen}[2][【原文】]{%
  %\biaoti{#1}\par
  \stepcounter{yuanwen}   % 计数器 yuanwen+1
  \bfseries #2}
  {}

\usepackage{enumitem}
\newenvironment{jiaozhu}[1][【校注】]{%
  %\biaoti{#1}\par
  \stepcounter{jiaozhu}   % 计数器 jiaozhu+1
  \begin{enumerate}[%
    label=\mylabel{\arabic*}{\circledctr*},before=\small,fullwidth,%
    itemindent=\parindent,listparindent=\parindent,%labelsep=-1pt,%labelwidth=0em,
    itemsep=0pt,topsep=0pt,partopsep=0pt,parsep=0pt
  ]}
  {\end{enumerate}}

%===================注解与原文相互跳转====================
%----------------第1部分 设置相互跳转锚点-----------------
\makeatletter
  \protected\def\mylabel#1#2{% 注解-->原文
    \hyperlink{back:\theyuanwen:#1}{\Hy@raisedlink{\hypertarget{\thejiaozhu:#1}{}}#2}}

  \protected\def\myref#1#2{% 原文-->注解
    \hyperlink{\theyuanwen:#1}{\Hy@raisedlink{\hypertarget{back:\theyuanwen:#1}{}}#2}}
  %此处\theyuanwen:#1实际指thejiaozhu:#1,只是\thejiaozhu计数器还没更新,故使用\theyuanwen计数器代替
\makeatother

\protected\def\myjzref#1{% 脚注中的引用(引用到原文)
  \hyperlink{\theyuanwen:#1}{\circlednum{#1}}}

\def\sb#1{\myref{#1}{\textsuperscript{\circlednum{#1}}}}    % 带圈数字上标

%----------------第2部分 调整锚点垂直距离-----------------
\def\HyperRaiseLinkDefault{.8\baselineskip} %调整锚点垂直距离
%\let\oldhypertarget\hypertarget
%\makeatletter
%  \def\hypertarget#1#2{\Hy@raisedlink{\oldhypertarget{#1}{#2}}}
%\makeatother

%====================带圈数字列表标头====================
\newfontfamily\circledfont[Path = fonts/]{meiryo.ttc}  % 日文字体,明瞭体
%\newfontfamily\circledfont{Meiryo}  % 日文字体,明瞭体

\protected\def\circlednum#1{{\makexeCJKinactive\circledfont\textcircled{#1}}}

\newcommand*\circledctr[1]{%
  \expandafter\circlednum\expandafter{\number\value{#1}}}
\AddEnumerateCounter*\circledctr\circlednum{1}

% 参考自:http://bbs.ctex.org/forum.php?mod=redirect&goto=findpost&ptid=78709&pid=460496&fromuid=40353

%======================插图/tikz图========================
\usepackage{graphicx,subcaption,wrapfig}    % 图,subcaption含子图功能代替subfig,图文混排
  \graphicspath{{img/}}                     % 设置图片文件路径

\def\pgfsysdriver{pgfsys-xetex.def}         % 设置tikz的驱动引擎
\usepackage{tikz}
  \usetikzlibrary{calc,decorations.text,arrows,positioning}

%---------设置tikz图片默认格式(字号、行间距、单元格高度)-------
\let\oldtikzpicture\tikzpicture
\renewcommand{\tikzpicture}{%
  \small
  \renewcommand{\baselinestretch}{0.2}
  \linespread{0.2}
  \oldtikzpicture
}

%=========================表格相关===============================
\usepackage{%
  multirow,                   % 单元格纵向合并
  array,makecell,longtable,   % 表格功能加强,tabu的依赖
  tabu-last-fix,              % "强大的表格工具" 本地修复版
  diagbox,                    % 表头斜线
  threeparttable,             % 表格内脚注(需打补丁支持tabu,longtabu)
}

%----------给threeparttable打补丁用于tabu,longtabu--------------
%解决方案来自:http://bbs.ctex.org/forum.php?mod=redirect&goto=findpost&ptid=80318&pid=467217&fromuid=40353
\usepackage{xpatch}

\makeatletter
  \chardef\TPT@@@asteriskcatcode=\catcode`*
  \catcode`*=11
  \xpatchcmd{\threeparttable}
    {\TPT@hookin{tabular}}
    {\TPT@hookin{tabular}\TPT@hookin{tabu}}
    {}{}
  \catcode`*=\TPT@@@asteriskcatcode
\makeatother

%------------设置表格默认格式(字号、行间距、单元格高度)------------
\let\oldtabular\tabular
\renewcommand{\tabular}{%
  \renewcommand\baselinestretch{0.9}\small    % 设置行间距和字号
  \renewcommand\arraystretch{1.5}             % 调整单元格高度
  %\renewcommand\multirowsetup{\centering}
  \oldtabular
}
%设置行间距,且必须放在字号设置前 否则无效
%或者使用\fontsize{<size>}{<baseline>}\selectfont 同时设置字号和行间距

\let\oldtabu\tabu
\renewcommand{\tabu}{%
  \renewcommand\baselinestretch{0.9}\small    % 设置行间距和字号
  \renewcommand\arraystretch{1.8}             % 调整单元格高度
  %\renewcommand\multirowsetup{\centering}
  \oldtabu
}

%------------模仿booktabs宏包的三线宽度设置---------------
\def\toprule   {\Xhline{.08em}}
\def\midrule   {\Xhline{.05em}}
\def\bottomrule{\Xhline{.08em}}
%-------------------------------------
%\setlength{\arrayrulewidth}{2pt} 设定表格中所有边框的线宽为同样的值
%\Xhline{} \Xcline{}分别设定表格中水平线的宽度 makecell包提供

%表格中垂直线的宽度可以通过在表格导言区(preamble),利用命令 !{\vrule width1.2pt} 替换 | 即可

%=================图表设置===============================
%---------------图表标号设置-----------------------------
\renewcommand\thefigure{\arabic{section}-\arabic{figure}}
\renewcommand\thetable {\arabic{section}-\arabic{table}}

\usepackage{caption}
  \captionsetup{font=small,}
  \captionsetup[table] {labelfont=bf,textfont=bf,belowskip=3pt,aboveskip=0pt} %仅表格 top
  \captionsetup[figure]{belowskip=0pt,aboveskip=3pt}  %仅图片 below

%\setlength{\abovecaptionskip}{3pt}
%\setlength{\belowcaptionskip}{3pt} %图、表题目上下的间距
\setlength{\intextsep}   {5pt}  %浮动体和正文间的距离
\setlength{\textfloatsep}{5pt}

%====================全文水印==========================
%解决方案来自:
%http://bbs.ctex.org/forum.php?mod=redirect&goto=findpost&ptid=79190&pid=462496&fromuid=40353
%https://zhuanlan.zhihu.com/p/19734756?columnSlug=LaTeX
\usepackage{eso-pic}

%eso-pic中\AtPageCenter有点水平偏右
\renewcommand\AtPageCenter[1]{\parbox[b][\paperheight]{\paperwidth}{\vfill\centering#1\vfill}}

\newcommand{\watermark}[3]{%
  \AddToShipoutPictureBG{%
    \AtPageCenter{%
      \tikz\node[%
        overlay,
        text=red!50,
        %font=\sffamily\bfseries,
        rotate=#1,
        scale=#2
      ]{#3};
    }
  }
}

\newcommand{\watermarkoff}{\ClearShipoutPictureBG}

\watermark{45}{15}{草\ 稿}    %启用全文水印

%=============花括号分支结构图=========================
\usepackage{schemata}

\xpatchcmd{\schema}
  {1.44265ex}{-1ex}
  {}{}

\newcommand\SC[2] {\schema{\schemabox{#1}}{\schemabox{#2}}}
\newcommand\SCh[4]{\Schema{#1}{#2}{\schemabox{#3}}{\schemabox{#4}}}

%=======================================================

\begin{document}
\pagestyle{main}
\fi
\chapter{《黄帝内经》的时间医学思想}%第四章
时间医学是研究人体生命节律,并用以指导临床诊断、治疗、预防、保健的一门新兴学科。内经》虽尚未形成完整的时间医学学科,但却包含着丰富的有关时间医学的内容。

\section{《内经》对生命基本节律的认识}%第一节

\subsection{亚日节律}%一、

亚日节律是指一日内重复两次至数次的节律变化。《内经》多处提及这类节律,尤以营卫的运行为突出。如《灵枢·卫气》说:“阳主昼,阴主夜。故卫气之行,一日一夜五十周于身,昼日行于阳二十五周,夜行于阴二十五周”。提出卫气的运行存在亚日节律变化。而对于营气运行的论述更为详细,《灵枢·五十营》指出:“气行五十营于身,水下百刻……漏水皆尽,脉终矣”。古代百刻计时制,一昼夜分为百刻,而营行五十周,则每周需时两刻,即28分48秒。至于营气流注的次序,首先从手太阴肺经开始,依次循行到手阳明大肠经、足阳明胃经、足太阴脾经、手少阴心经、手太阳小肠经、足太阳膀胱经、足少阴肾经、手厥阴心包经、手少阳三焦经、足少阳胆经、足厥阴肝经,复行于手太阴肺经,循环无端,无有体止。当营气运行到某经时,某经经气便出现一次髙潮,该经的机能随之旺盛,从而使人体生命活动产生周期变化,其实,这也正是中医针灸临床留针30分种左右的一个重要原因。就某一经而言,这个周期,即属于亚日节律。

\subsection{周日节律}%二、

周日节律是指以二十四小时或接近二十四小时为一个周期的节律变化,又称昼夜节律变化。地球上有昼夜交替,存在着阴阳之气的周日节律变化,人体尤其显著的是阳气,也随之发生相应的改变,而呈现出周日节律变化。如《素问·生气通天论》说:“阳气者,一日而主外,平旦人气生,日中而阳气隆,日西而阳气已虚,气门乃闭”。由于阳气的变化,导致人体疾病亦有周日节律变化,故《灵枢·顺气一日分为四时》云:“夫百病者,多以旦慧、昼安、夕加、夜甚,何也?岐伯曰:四时之气使然……春生、夏长、秋收、冬藏,是气之常也,人亦应之。以一日分为四时,朝则为春,日中为夏,日入为秋,夜半为冬。朝则人气始生,病气衰,故旦慧;日中人气长,长则胜邪,故安;夕则人气始衰,邪气始生,故加;夜半人气入脏,邪气独居于身,故甚也”。另外,人体营卫之气的运行除了存在亚日节律的周期性变化外,也存在着周日节律变化,不仅五十周于身是谓昼夜,而且“夜半而大会”(《灵枢·营卫生会》)。其中卫气“昼日常行于阳,夜行于阴,故阳气尽则卧,阴气尽则寤”(《灵枢·大惑论》,形成了人体寤寐的周日节律变化。

\subsection{周月节律}%三、

周月节律是指以一个恒星月或一个朔望月为一个周期的节律变化。《内经》指出,月满则人体气血旺盛充满,月空则人体气血虚弱、肌肉减、皮肤纵,并提出了相应的治则及违反该治则所引起的疾病,如《素问·八正神明论》云:“凡刺之法,必候日月星辰,四时八正之气,气定乃刺之……月始生,则血气始精,卫气始行;月郭满,则血气实,肌肉坚;月郭空,则肌肉减,经络虚,卫气去,形独居,是以因天时而调血气也”。“月生无泻,月满无补,月郭空无治,是谓得时而调之”。“故曰:月生而泻,是谓脏虚,月满而补,血气扬溢,络有留血,命曰重实;月郭空而治,是谓乱经”。将周月节律变化阐述得十分明确。

\subsection{周年节律}%四、

周年节律是指一年十二个月、或四时、或五时等为一个周期的变化节律。《内经》提出,人体某些节律是以年度为周期的,其中有以二个月为一个阶段者,如《素问·诊要经终论》说:“正月二月,天气始方,地气始发,人气在肝。三月四月,天气正方,地气定发,人气在脾。五月六月,天气盛,地气高,人气在头。七月八月,阴气始杀,人气在肺。九月十月,阴气始冰,地气始闭,人气在心。十一月十二月,冰复,地气合,人气在肾”。有以五时为标准者,如《素问·平人气象论》提出的春季脏真散于肝,夏季脏真通于心,长夏脏真濡于脾,秋季脏真高于肺,冬季脏真下于肾,同时还指出了五时五脏的平、病、死之脉。有以四时为标准者,如《灵枢·五乱》说:“经脉十二者,以应十二月,……分为四时,四时者,春夏秋冬,其气各异”。并提出了脉象的四时周期变化,如《素问·脉要精微论》说:“四变之动,脉与之上下。以春应中规,夏应中矩,秋应中衡,冬应中权”。

至于人体的疾病,《内经》认为也存在着年周期节律变化,如《素问·脏气法时论》指出:“病在肝,愈于夏,夏不愈,甚于秋,秋不死,持于冬,起于春”。“病在心,愈在长夏,长夏不愈,甚于冬,冬不死、持于春,起于夏”。“病在脾,愈在秋,秋不愈,甚于春,春不死,持于夏,起于长夏”。“病在肺,愈在冬,冬不愈,甚于夏,夏不死,持于长夏,起于秋”。“病在肾,愈在春,春不愈,甚于长夏,长夏不死,持于秋,起于东。”

\subsection{超年节律}%五、

超年节律指周期在一年以上的节律变化。《内经》反映超年节律变化的内容主要见于五运六气学说。该学说运用阴阳、五行、六气等理论,并以十天干、十二地支等作为演绎符号,来推论气候变化、生物生化和人体疾病流行之间的关系。它以“甲子”纪天度,认为甲子一周六十年为一个变化周期。《素问》的《六节藏象论》、《天元纪大论》等,详细记载了这一内容。

\section{《内经》对生命活动节律机理及其实质的认识}%第二节

对生命时间特性的本质与形成的认识,目前尚未统一,但有两种主要观点可供参考。其一,认为节律的形成是受地球物理坏境周期性变化为主的外界影响,是由几种外界力——如光、温度、电磁变化,可能还有某些未知的微细的地球物理学的力等与机体代谢相互作用而产生的,即所谓节律形成的外源性观点。其二,认为节律的形成与人体内自发引起的内因性节律相关,是先天性内源性的,是通过遗传获得的。日益增多的各种研究资料说明,生命的节律关系到细胞、器官以至整个生物系统的各个方面,它们相互影响、相互联系;同时,节律的调节还受到外界环境因素的影响。因此,从单一方面解释也许是片面的,生命节律的形成可能是一个多因素的整体效应。《内经》将生命话动节律变化的机理则主要归结为自然界规律的影响,并且认为自然界规律的影响导致了人体的阴阳消长、气血活动、脏腑经脉功能盛衰的节律变化,进而导致了人体生命活动表现出节律变化。

\subsection{《内经》对生命活动节律形成的认识}%一、

人生活在自然界之中,自然界存在着一些规律性的变化,如昼夜交替规律、四季寒暑规律等,这些变化影响着人,导致人产生了节律变化,这是《内经》,整体观思想的一个具体反映。《素问·金匮真言论》说:“平旦至日中,天之阳,阳中之阳也,日中至黄昏,天之阳,阳中之阴也。合夜至鸡鸣,天之阴,阴中之阴也,鸡鸣至平旦,天之阴,阴中之阳也。故人亦应之”。指出地球上有昼夜交替变化规律,故人亦有周日节律变化,有昼精夜眠现象的产生。周月节律是以自然界月相盈亏规律为基础的,至于其机理,《内经》认为与月球引力有关,如《灵枢·岁露论》说:“人与天地相参也,与日月相应也。故月满则海水西盛,人血气积,肌肉充,皮肤致……至其月郭空,则海水东盛,人气血虚,其卫气去,形独居,肌肉减,皮肤纵”。明确指出海水受月球引力的影响而有涨有落,人体生命活动也受月球引力的影响而有周月节律变化。四季阴阳变化使人体产生相应的节律,如《素问·脉要精微论》云:“万物之外,六合之内,天地之变,阴阳之应,彼春之暖,为夏之暑,彼秋之忿,为冬之怒,四变之动,脉与之上下,以春应中规,夏应中矩,秋应中衡,冬应中权。”现代研究证实,冬季温度低而气压高,故人的体表血管收缩而体内血管扩张,夏季温度高而气压低,故人的体表血管扩张而体内血管收缩,因而人体血液的分布随着不同的季节而有其侧重部位,说明人体气血的分布确实存在着周年节律。由于四时阴阳变化是万物生长收藏之本,只有顺应自然界这个规律,才能保持身体健康,因此,《素问·四气调神大论》提出了“春夏养阳,秋冬养阴”的四时养生法则,并制订了具体的方案。

《内经》对遗传、人的先天禀赋问题有明确的论述,如《灵枢·天年》云:“以母为基,以父为楯”。认为父母决定了人的先天,但是就遗传因素在人体生命活动节律中的地位而言,《内经》论述较少。

\subsection{《内经》对生命活动节律实质的认识}%二、

生命节律表现于外是人体生命活动现象的节律性变化,如白昼活动、夜晚睡眠;脉象的春弦夏钩秋毛冬石;疾病的旦慧、昼安、夕加、夜甚等。但究竟是什么引起生命活动现象的节律变化呢?《内经》主要从阴阳消长、气血盛衰及脏腑经脉功能旺盛与否几方面加以阐述。

\subsubsection{阴阳消长变化}%(一)

由于自然界的规律,导致了人体阴阳消长的周期性变化,进而形成了人体生命活动的节律变化。《素问·脉要精微论》说:“冬至四十五日,阳气微上,阴气微下,夏至四十五日,阴气微上,阳气微下”。指出自然界存在着阴阳消长变化的“二至”节律。冬至是阴气盛极,盛极必衰,阳气开始生长的节气,此后阳气逐渐盛长而阴气逐渐衰减,经小寒、大寒、立春等节气到夏至;夏至则是阳气盛极的节气,此后阴气逐渐盛长而阳气渐减,经小暑、大暑、立秋等节气到冬至。即所谓“冬至〜阳生,夏至〜阴生”。自然界的阴阳消长,往复循环,人体的阴阳之气与此相应,也形成了人体阴阳变化的“二至”节律。一天之中的阴阳消长节律变化,其机理亦同。正如《素问·生气通天论》所说:“阳气者,一日而主外,平旦人气生,日中而阳气隆,日西而阳气已虚,气门乃闭”。受此影响,人的劳作活动、精神状态、某些易发病的产生、病情的轻重变化等出现了一定的周期性变化,如《素问·阴阳应象大论》就明确指出阳盛病“能冬不能夏”,阴盛病则“能夏不能冬”,而治疗此类病证也要从调节阴阳入手。

\subsubsection{气血盛衰变化}%(二)

人体周月节律受月相的朔望影响明显,其中主要是指气血的盛衰变化,《素问·八正神明论》与《灵枢·岁露论》均强调了这一点。气血盛衰则使人体表规有肌肉充实、皮肤致密与肌肉消减、皮肤松弛等现象。

现今,对月节律的研究很多,如女性月经周期、妇女免疫机能近月节律的变动,出血患者以圆月时容易发病等,程士德《中医时间证治学纲要·生命节律的现代研究》中指出:“人们依据现有的研究资料,认为人的体液受月球引潮力的影响很大,生物体内月节律的形成可能与此相关”。说明了对《内经》提出的“气血盛衰”导致节律现象产生的研究结果。

\subsubsection{脏腑经脉功能盛衰变化}%(三)

“五脏应四时,各有收受”(《素问·金匮真言论》),因此人体脏腑的这种节律亦称为“五脏主时”节律。《内经》“五脏主时”节律主要包括:周年节律中的五脏应六时、五脏应五时、五脏应四时节律;周日节律中的五脏应日四时段、五脏应日五时段、五脏应日十二辰节律。脏所主之时,则该脏系统功能增强,该脏在全身的作用处于主导地位,并且对外界的敏感性升高。以五脏疾病在不同季节的变化为例,如《素问·脏气法时论》说:“病在肝,愈于夏,甚于秋,秋不死,持于冬,起于春……”。《灵枢·顺气一日分为四时》也说:“脏独主其病者,是必以脏气之所不胜时者甚,以其所胜时者起也”。可以看到,五脏功能盛衰的节律表现为:旺于同本脏阴阳五行属性相一致的时令,即该脏所主之时;衰于该脏所不胜之时令;在生我之时令脏腑之气得到加强而生长,使功能增强;在我生之时令功能逐渐低下。需要说明,这里的五脏,是指以五脏为中心的五个功能活动系统而言。

另外,古代医家在长期的针灸医疗实践中,逐步观察到某经疾病在相应时辰施治较其他时辰疗效显著,并根据《灵枢·经脉》、《灵枢·营气》等篇记载的十二经脉流注的交接次序,制定出十二经脉应十二辰的配属关系。即:手太阴肺配寅、手阳明大肠配卯、足阳明胃配辰、足太阴脾配巳、手少阴心配午、手太阳小肠配未、足太阳膀胱配申、足少阴肾配酉、手厥阴心包配戌、手少阳三焦配亥、足少阳胆配子、足厥阴肝配丑。这一规律是以手太阴肺配寅时为起始的。由于手太阴为十二经之首,而“寅”是一天之中阳气初生之时,正如《灵枢·阴阳系日月》云:“寅者,正月之生阳也”、《史记·律书》也云:“寅,言万物始生”。既然寅应手太阴肺经,则其余各经应时便自然成序。这种配属是对古人临床观察到的经脉在不同时辰中疗效有别的总结,实则是指经脉功能盛衰,其实质是每值某经在其相应的时辰,其功能活动相对旺盛,自身敏感性增强,因而能对针灸等治疗效果产生影响。此外,《内经》所阐述的周年节律中,也有以经脉功能盛衰为主者,如《灵枢·五乱》说:“经脉十二者,以应十二月”。根据《素问·脉解》、《灵枢·阴阳系日月》、《灵枢·经筋》,其经脉盛衰年周期节律如下表:

\begin{table}[htb]%经脉盛衰年周期
	\centering
	\caption{经脉盛衰年周期}\label{tab:经脉盛衰年周期}

	\newcolumntype{M}[1]{>{\centering}m{#1}}
	%\renewcommand\arraystretch{2}
	%\resizebox{\textwidth}{!}{%
	\begin{tabular}{M{9em}|*{10}{M{1em}|}@{}M{2em}@{}|@{}M{2em}@{}|m{7em}}
		\toprule
		\multirow{2}{*}{} & 一 & 二 & 三 & 四 & 五 & 六 & 七 & 八 & 九 & 十 & 十一 & 十二 & \multirow{2}{*}{原文出处} \\ \cline{2-13}
						  & 寅 & 卯 & 辰 & 巳 & 午 & 未 & 申 & 酉 & 戌 & 亥 & 子   & 丑   &  \\
		\midrule
		(一)十二经脉相应十二月盛衰节律 & 左足少阳   & 左足太阳   & 左足阳明   & 右足阳明   & 右足太阳   & 右足少阳   & 右足少阴   & 右足太阴   & 右足厥阴   & 左足厥阴   & 左\ 足\ 太\ 阴     & 左\ 足\ 少\ 阴     & 《灵枢·阴阳系日月》 \\ \hline
		(二)十二经筋相应十二月盛衰节律 & 足少阳之筋 & 足太阳之筋 & 足阳明之筋 & 手阳明之筋 & 手太阳之筋 & 手少阳之筋 & 足太阴之筋 & 足少阴之筋 & 足厥阴之筋 & 手厥阴之筋 & 手\ 太\ 阴\ 之\ 筋 & 手\ 少\ 阴\ 之\ 筋 & 《灵枢·经筋》       \\ \hline
		(三)六经之气盛衰年周期节律     & 太阳       &            & 厥阴       &            & 阳明       &            & 少阴       &            & 少阳       &            & 太\ 阴             &                    & 《素问·脉解》       \\
		\bottomrule
	\end{tabular}
	%}
\end{table}

\section{《内经》时间医学思想的临床应用}%第三节

\subsection{辨证}%一、

根据人体生命活动的时间节律变化来判定疾病的发生及其病位、病情、邪正的趋势等,称为时间辨证。《内经》时间辨证,主要有两种方法,一是按组成人体的各要素与时间的关系分析,如脏腑的时间周期辨证、经脉的时间周期辨证、气血的时间周期辨证、阴阳的时间周期辨证等;一是按照自然时间周期节律分析,如岁气六十年周期节律辨证、周年周期节律辨证、周月周期节律辨证、周日周期节律辨证等。上述两种方法,在临床辨证中并非截然分开,而是相互结合、相互补充的。

\subsubsection{脏腑经脉的时间周期辨证}%(一)

由于《内经》的五脏功能活动系统亦包括经脉在其中,所以经脉辨证不再单列。至于现在针灸临床应用的子午流注等法,其基本理论虽然也是本于《内经》、以时间与经脉对应规律为依据,但其运用则已远超过《内经》所述,故此处亦不再详解。这里所述的脏腑经脉的时间周期辨证,是将五脏病机与时间周期的节律结合起来,从时间和生命的时间节律方面来推求病位、辨别证候,适用于五脏系统的各种时间周期节律,如岁气六十年周期节律、周年节律、周日节律等。这一辨证方法,是以五脏在主时之时段的病变发生发展规律为基础的。

关于某一脏腑病变在其相对应时段的轻重变化规律,《素问·脏气法时论》云:“夫邪气之客于身也,以胜相加,至其所生而愈,至其所不胜而甚,至于所生而持,自得其位而起”。马莳《黄帝内经素问注证发微》曰:“肝病始于春,心病始于夏,脾病始于长夏,肺病始于秋,肾病始于冬者,皆由邪气感于吾身。”又曰:“自得其位而起,肝病起于春,心病起于夏,脾病起于长夏,肺病起于秋,肾病起于冬者,皆得其所应之时而病复起也”。究马莳所注之意,其一,五脏在其所主之时令易感邪发病;其二,五脏之病,在其所主之时令易复发、加重。对于五脏在所主之时令易感邪发病之论,《内经》确有其说,如《素问·咳论》所云:“五脏各以其时受病,非其时各传以与之”。“乘春则肝先受之”。又《素问·风论》亦有类似论述。究其因,一方面,与时令同步相应之脏,即值令之脏,在全身起着主导地位,正如《素问·平人气象论》之春令“脏散于肝”、夏令“脏真通于心”、长夏“脏真濡于脾”、秋令“脏真高于肺”,冬令“脏真下于肾”,故当该时令邪气淫盛侵犯人体则首先影响该脏,其病变表现亦以该脏功能紊乱为主。另一方面,时令与同步相应之脏,二者阴阳五行特性相符,故该时令之邪气易同该脏结合而侵犯人体,正所谓“同气相求”、“以类相从”。其三,在某一时段中,与其相通应的脏腑,其生理功能相对旺盛,但若当旺不旺,则易于发病;然而旺气太过,超过了一定的限度,亦会导致机体内环境失衡而发病。除可表现为主时的脏腑病变外,也可表现为有生克相关脏腑病变,特別是其所克的脏腑。对于马莳注的另一观点,即五脏之病,在其所主之时令易复发或加重,则有不同的看法。山东中医学院等《黄帝内经素问校释》云:“自得其位而起,即至自旺之时病情好转。如肝病至属木之时而起”。实则,这两种情况临床均可见到,如张机《伤寒论》既云“阳明病,欲解时,从申至戌上“(193),又云“日晡所发热者,属阳明也”(240),就说明了这一点。

疾病发生的原因和机制虽有多种多样,但总不外乎正气与邪气两方面势力的对比,《灵枢·顺气一日分为四时》在讨论疾病昼夜轻重规律的机理时云:“朝则人气始生,病气衰,故旦慧;日中人气长,长则胜邪,故安;夕则人气始衰,邪气始生,故加;夜半人气入脏,邪气独居于身,故甚也”。以此理来分析五脏在其所主之时令减轻、痊愈或复发、加重的变化,可做如下分析:在五脏所应之时令,其脏腑系统的精气含量及功能最旺盛,而在其他时段,根据五行的生克关系,则也有多少、盛衰的不同,王玉川《运气探秘》总结归纳为“休”“王”“相”“死”“囚”。五脏所应之时令称为“王”。五脏之病在“王”时容易减轻、痊愈,因该脏在此时正气最强、抗邪最有力,正所谓“人气长,长则胜邪”,此属于疾病的一般发展规律。而包括虚实两类,属于虚者,遇“王”时则正气增强,虚损得到改善,病情缓解或痊愈;而属于实者,正邪相持,遇“王”时则正气旺盛,正胜邪衰,故见缓解或痊愈,如《伤寒论》所论的六经欲解时,即属此类。而五脏之病在“王”时加重或复发,其原因亦应从虚实两方面分析。属于虚者,脏腑之气当旺不旺,不足以值令而发病或加重。如《素问·三部九候论》曰:“其脉乍疏乍数,乍迟乍疾者,日乘四季死”。高世栻《素问直解》注云:“脾脏属土,土灌四旁,若其脉乍疏乍数,乍迟乍疾,乃中土内虚,不能四布,故以一日所乘之四季死。辰戌丑未,寄旺于平旦、日中、日夕、夜半也”。属于实者,则为邪气太盛,正气相对不足,遇“王”时,则正气得到补充,则始与邪气激烈抗争而导致症状突现,这样的病征一般责之为邪气较强盛、或邪著部位较深而不易祛除,如《内经》所述之某些疟证、《伤寒论》之阳明病的日晡所发潮热等即属于此。

根据上述,我们可得出这样一种辨证方法。首先,判定所出现的病证是否属于始发、是否属于周期性的疾病。若是始发,而非周期性发病,则结合发病的时间与脏腑经脉的配属关系,而进行辨证,如春令咳嗽,应结合春应肝,而咳的主要病位在于肺,故辨证病位应在肝肺系统,正如《素问·咳论》所说:“乘春则肝先受之。”进一步则根据症状表现辨其虚实寒热。若属于周期性出现的疾病,则应根据其病情减轻或加重的时间段,推断其所属脏腑,进而根据症状表现判断该脏腑正邪斗争情况、判断其他相关脏腑的情况。一般而言,若症状减轻或消失,则说明该脏腑或为正气不足,或正邪相持,但邪气并非太甚;或该脏腑所胜之脏腑为实证。此时当或用祛邪之法,或用扶正之药,以利用该脏腑正气旺盛之机来抑制消除邪气。笔者曾遇一于姓患者,两年来心烦不乐,以清晨为重,与人言谈多为哭诉,然至下午三时以后则明显减轻。清晨应于肝胆,一诊辨证为肝胆不利,痰热内扰,处以柴芩温胆汤。然一周后再诊则诸症无明显变化,考虑前诊并无错误,但据其症状午后三时顿减的特点分析,是肝失疏泄,病受“金气”所制。因午后3〜7时为申酉,是肺金配属之时辰,故于原方加浙贝、杏仁,以佐金平木,取得了较好疗效。若症状加重或复发,则说明该脏腑邪气旺盛,病情较重;或正气大虚,当旺不旺;或该脏之所不胜脏为正气不足。故此时或当急用祛邪之品,如《伤寒论》日晡所发潮热,属阳明病,急用大承气以下之,或用重剂扶正之品以救其虚衰。

\subsubsection{气血周期节律辨证}%(二)

以气血为基础阐述时间节律,《内经》主要有周月节律和昼夜睡眠节律,其中昼夜睡眠节律变化又与阴阳消长节律密切相关,故另做讨论。周月节律以气血虚实变化为主,辨证方法是:满月时,病情发作或加重者,常为气血壅实之证;若是病情减轻或消失者,常为气血不足之证。月始生或朔月之时,若病情加重或发作,则多属气血不足之虚证;若病情减轻或消失,常为气血壅实,可辨为实证。

\subsubsection{阴阳消长周期节律辨证}%(三)

《内经》涉及该节律的内容主要是周年节律中的“二至”节律及昼夜阴阳消长节律。二至节律辨证方法,一般适用于一年中的二至或一日中的子午两时辰症状出现或加重、减轻或消失的疾病。其病机总属阴阳失调,不能适应自然界阴阳消长交替变化,故辨证应为阴阳失调,治疗应以协调阴阳为主。具体而言,冬至节或子时,是自然界阴气最旺之时,若病情加重或出现,可辨为阴盛或阳虚;若病情减轻或消失,可辨为阴虚或阳盛。夏至节或午时,是自然界阳气最盛之时,若病情加重或出现,可辨为阴虚或阳盛;若病情减轻或消失,可辨为阳虚或阴盛。正如《素问·阴阳应象大论》所云:阳盛病“能冬不耐夏”,阴盛病“能夏不耐冬”。亦有人认为夏至冬至、子时午时为阴阳交替之时,所以此时发病乃阴阳失调之象,故应调和阴阳为主,如岳美中先生用调和阴阳之方剂小柴胡汤治愈午时和子时四肢不自主地下垂软瘫,如无知觉之状的患儿(岳美中,试谈辨证论治和时间空间,上海中医药杂志,1978,复刊号:14)。

此外,还有人从冬至、子时是阳气开始生发之时,即“冬至—阳生”,夏至、午时是阴气开始生发之时,即“夏至—阴生”入手,提出冬至、子时若病情加重或出现,可辨为阴虚或阳盛,若病情减轻或消失,可辨为阴盛或阳虚;夏至或午时,若病情加重或出现可辨为阳虚或阴盛,若病情减轻或消失,可辨为阴虚或阳盛(程士德,中医时间证治学纲要,人民卫生出版社,1994年3月,132页)。两说不同,待进一步研究。

\subsection{治疗}%二、

辨证已定,治法依证而立,用药也就因法而处,故根据前述辨证结果而采用相应治疗方法,自不待言。而对于那些并无时间周期性的蒺病,其治疗亦应考虑时间问题,以免犯“伐天和”之弊,这也是《内经》提出的“因时施治”的重要内容之一。现就因时立法与用药、择时服药两方面,介绍如下。

\subsubsection{因时立法与用药}%(一)

在自然界四时阴阳消长节律的影响下,疾病在春夏季节因阳长而易于热化,于秋冬因阴长而易于寒化,为了防止其热寒之变,保证疗效,《素问·六元正纪大论》提出了“热无犯热,寒无犯寒”及“用寒远寒,用凉远凉,用温远温,用热远热,食宜同法”等用药原则。其大要有二:一是在春夏等阳气旺盛之时令应佐用寒凉,秋冬等阴气旺盛之时令佐用温热;二是若春夏等阳气旺盛之时令必须用热法时,则不可热之太过,而秋冬等阴气旺盛之时令必须用寒法时,亦不可寒之太过。至于其具体方法则有:在原有处方基础上随时令加减用药、选择寒热性缓之药而轻用药量、运用反佐药等。后世医家禀承经训者颇不乏人,如元·朱震亨认为若于夏日火令之时妄投温热,则有虚虚实实之弊;金·刘完素制方用药强调当顺时令而调阴阳;清代医家程钟龄则提出:用药而失四时寒热温凉之宜,乃医家之大误。

升降浮沉是自然界万物的运动形式,人体亦与之相应,因此,在治疗上《内经》也非常重视这一点。《素问·阴阳应象大论》说:“形不足者,温之以气,精不足者,补之以味”。虚则补之之道,不外阴阳二途。阳虚者,于春夏等阳气旺盛之时令,宜用辛甘温热之剂,当升当浮,如李杲说:“辛甘发散,以助春夏生长之用也”。并据此创制相应方剂——补中益气汤,以“辛甘温之剂,补其中,而升其阳”(《脾胃论·饮食劳倦所伤始为热中论》)。因此,补中益气汤不仅是补气升阳的基本方,而且还是春夏阳气旺盛之时令补虚的应时方剂。治阴虚者,秋冬可填养而春夏勿滋腻。如薛生白治下元亏损之虚劳,认为夏月天地之气大泄,质重之补宜缓,而易之清淡气薄、养胃生津、宁神敛液或安中之品,则无抑遏气机、壅滞中焦畅达之弊。待秋后天地之气收降,方可再进温养填补充形之品,以顺秋冬之收藏。秋冬填养,春夏调气,乃治阴损不足,因时用药的原则。

攻邪的方法,一般以汗、吐、下三法为代表,因方药之性能,常具有明显的上升、下降之趋势,故亦应考虑时令阴阳升降之宜忌而用之。如张机《伤寒论》就明确提出:“春夏宜发汗”、“春宜吐”、“秋宜下”的原则,李杲《脾胃论·用药宜禁论》也说:“时禁者,必本四时升降之理,汗、下、吐、利之宜。大法:春宜吐,象万物之发生,耕、耨、科、斫,使阳气之郁者易达也。夏宜汗,象万物之浮而有余也。秋宜下,象万物之收成,推陈致新,而阳气易收也。冬周密,象万物之闭藏,使阳气不动也”。当然,这里的吐、汗、下,其意义是取意于顺应春气的升达、夏气的浮畅、秋气的收降、冬气之闭藏而言,故凡具有春生、夏长、秋收、冬藏之义的各种治法皆可属之。

脏腑疾病的因时治疗,要点有三:其一,直治主时之脏,主要用于具有时间周期性的疾病;其二,在根据疾病证候立法用药基础上兼调主时之脏。如李杲治中风,用羌独愈风汤,并提出此药可常服之,但不可失四时之辅,故春加半夏、人参、柴胡等,应时枢转少阳;夏加石膏、知母、黄芩等,以防火助风势;长夏加防己、白术、茯苓等,健脾利湿,运中洲以达四旁;秋加厚朴、藿香、桂枝,宣肺气之通降,以利中风于秋时缓解;冬加附子、官桂、当归等,补命火、固根底,辅佐主方冬月之用;其三,顾护被克之脏。如李时珍从药之五味方面运用了这一原则,他说:“春,省酸增甘,以养脾气;夏,省苦增辛,以养肺气;长夏,省甘增咸,以养肾气;秋,省辛增酸,以养肝气;冬,省咸增苦,以养心气。”(《本草纲目·序例》)

\subsubsection{择时服药}%(二)

择时服药亦属《内经》“因时而治”内容之一,无论用针用药,都必须随时间的不同而采取不同的措施,否则,将会引起不良的后果,如《灵枢·卫气行》说:“谨候其时,病可与期,失时反候者,百病不治”。张机及后世医家对择时服药都非常重视,在叶天士《临证指南医案》中,仅各种方药的进服时间即记载了近百处。如早用温肾阳之药,晚服补脾气之品;晨滋肾阴,午健脾阳;早服摄纳下焦,暮进纯甘清燥等。其具体择时服药方案,主要有以下几种。

1.因脏腑经脉时间节律服药:这一方法主要分为服用补养剂和攻下剂两种,一般情况下·以祛邪为主的,如行气、活血、祛瘀、散结、导滞、清热、泻火、解毒等药,应在脏腑、经脉主时即其功能活动最旺盛的时令服用,这样可以利用正气祛邪之力,因势利导,充分发挥药物的泻实作用。如手太阴肺应寅时,则张机强调十枣汤应“平旦服”。以补益正气为主的,如补气、养血、滋阴、助阳等,应在脏腑功能相对低下的时令进药,这样既可以对虚证有明显的改善作用,又可以减少有病脏腑经脉昼夜间的虚损差别。如足少阴肾旺于酉时,衰于卯时,故叶天士提出早温肾阳、晨滋肾阴、晨补肾气等时间服药的方法。

2.因气血阴阳盛衰气机升降节律服药:人体气血阴阳盛衰消长、气机升降出入,均与自然界昼夜时辰的阴阳消长、升降相同步,故服药亦应遵循这一规律。一般而言,补阳、行水利湿、催吐和益气药,宜清晨服用;升阳药、发汗解表药,宜于午前服用;清泻大肠通腑之剂,宜午后服用;滋养阴血、清营凉血、攻逐瘀血、化痰豁痰、安神定志等药,宜入夜服用。以符合机体阴阳气血的消长盛衰节律,利用机体营卫的运行、借助人体气机升降之势以提高疗效。

另外,择时服药,《内经》还有对病作有时者当其未发时眼药之论,如疟证的治疗,即属此类,所谓:方其盛时必毁,因其衰也,事必大昌”。

\subsection{预测疾病}%三、

根据人体生命活动节律可以对病情变化进行预测,其超年节律的岁气60年周期节律、周年节律、周月节律、周日节律等,皆可用于预测病情。运气学说提出的超年节律,可以预测每一年、每一时段的气候变化,进而推测病情的发生与演变。而周年节律也可用于病情的推断,如《素问·脏气法时论》说:“病在肝,愈于夏,夏不愈,甚于秋,秋不死,持于冬,起于春”;“病在心,愈在长夏,长夏不愈,甚于冬,冬不死,持于春,起于夏”;“病在脾,愈在秋,秋不愈,甚于春,春不死,持于夏,起于长夏”;“病在肺,愈在冬,冬不愈,甚于夏,夏不死,持于长夏,起于秋”;“病在肾,愈在春,春不愈,甚于长夏,长夏不死,持于秋,起于冬”。这一详细推论,主要根据五脏功能盛衰的周期节律变化及五脏之间的五行生克制化关系而来,适用于以五脏为主的病变,这一推论不仅适合于五脏与季节配属的周年节律变化,也适用于五脏与十二辰配属的周日节律变化。周月节律也可用于推测病情,如《素问·八正神明论》所谓月满则气血多实,月亏则气血多虚,本身就属于预测病情范畴。至于昼夜节律,《灵枢·顺气一日分为四时》明确云:“夫百病者,多以旦慧、昼安、夕加、夜甚”。指出了病情在一天中的变化规律,并以此可以推断病情,这一推断是以阳气昼夜盛衰节律为依据的,较适用于外感病及不以脏主其病者。当然,以时间节律推断预测病情变化,当结合临床具体情况,不可一概而论。

\subsection{养生保健}%四、

《内经》认为百病的发生无外乎两方面,即“或起于阴,或起于阳”,同时又认为,四时阴阳是万物生长化收藏之本,故只有顺应四时阴阳消长变化节律,才能健康无病。《素问·上古天真论》说:“虚邪贼风,避之有时”,即强调养生必须因时避邪,而要做到这一点,就得按照《内经》运气学说所揭示的气候变化节律去预防。对于具体的因时养生方法,《素问·四气调神大论》作了详细说明,其云:“春三月,此谓发陈。天地俱生,万物以荣,夜卧早起,广步于庭,被发缓形,以使志生,生而勿杀,予而匆夺,赏而勿罚,此春气之应,养生之道也。逆之则伤肝,夏为寒变,奉长者少。夏三月,此谓蕃秀。天地气交,万物华实,夜卧早起,无厌于日,使志无怒,使华英成秀,使气得泄,若所爱在外,此夏气之应,养长之道也。逆之则伤心,秋为痎疟,奉收春少,冬至重病。秋三月,此谓容平。天气以急,地气以明,早卧早起,与鸡俱兴,使志安宁,以缓秋刑,收敛神气,使秋气平,无外其志,使肺气清,此秋气之应,养收之道也。逆之则伤肺,冬为飧泄,奉藏者少。冬三月,此谓闭藏。水冰地坼,无扰乎阳,早卧晚起,必待日光,使志若伏若匿,若有私意,若已有得,去寒就温,无泄皮肤,使气亟夺,此冬气之应,养藏之道也。逆之则伤肾,春为痿厥,奉生者少”。其中,涉及到形体锻炼、起居活动、行为活动、情志调节等诸多方面,其目的,就是按照自然界阴阳消长周期长期影响下所形成的人体生命活动节律,去规范人的活动,以达到防病抗衰、保持健康、延年益寿之目的。在因时养生的具体方法上,后世把饮食摄养、服用药饵、气功、针灸等亦纳其中,形成了一套完整的因时养生方法,在临床实践中有着重要的意义。

\section{《内经》时间医学思想的意义及其评价}%第四节

《内经》对人体生命节律的论述,是其理论体系的一个重要学术内容。人体生命具有系统有秩序的节律性变化,这种变化是与宇宙整体的恒动变化互为通应的,其中“四时五脏阴阳”理论是贯穿于《内经》理论体系始终的最重要基本学术思想之一,也是《内经》探索生命节律的主要理论。它把古代哲学中的阴阳五行学说成功地运用到医学领域,提出“五脏应四时,各有收受”(《素问·金匮真言论》),“上应天光星辰历纪,下副四时五行,贵贱更立,冬阴夏阳,以人应之”(《素问·三部九候论》)的思想,从而把人体脏腑气血的活动变化,自然四时昼夜的交相更替和万物的生长化收藏规律,通过阴阳五行的基本原理有机地溶为一体,形成了以天人一体的生命节律思想为指导的独特的理论体系。因此,中医学的理论,如对脏腑经脉气血津液认识、对病因病机病证的理解、对诊断治疗养生的看法,均以“四时五脏阴阳”观点为指导,以生命节律现象为基础。所以说,《内经》时间医学思想是中医学理论体系的重要组成部分,也是这一理论体系形成的重要基础。

深入研究《内经》时间医学思想,对中医学术的发展及临床实践有重要意义。其一,它促进医学模式的转变。仅把人作为生物去研究,这是医学的弊病。而《内经》从自然界影响人,把人与自然界看成整体,从多方面探讨人的生理、病理,促进医学朝先进的医学模式转变。其二,有利于揭示中医理论体系的学术特征。从《内经》时间医学思想看,它从整体角度、运动变化角度、功能角度去认识人体,把人的五脏与四时紧密结合起来,正如恽铁樵先生所说:“《内经》之五脏,非血肉之脏,乃四时之脏”。有利于揭示中医学理论的本质,为更好地发展中医学奠定基础。其三,提高对疾病发展演变的认识,预测病情发展。有利于治疗疾病,阻断其恶化。其四,利于养生防病。根据生命节律养生,人与自然界形成完美的统一,既保障生命健康,又能避免邪气侵犯,从而达到增强体质、延年益寿之目的。

\zuozhe{(翟双庆)}
\ifx \allfiles \undefined
\end{document}
\fi