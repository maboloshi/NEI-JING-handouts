% -*- coding: utf-8 -*-
%!TEX program = xelatex
\ifx \allfiles \undefined
\documentclass[draft,12pt]{ctexbook}
%\usepackage{xeCJK}
%\usepackage[14pt]{extsizes} %支持8,9,10,11,12,14,17,20pt

%===================文档页面设置====================
%---------------------印刷版尺寸--------------------
%\usepackage[a4paper,hmargin={2.3cm,1.7cm},vmargin=2.3cm,driver=xetex]{geometry}
%--------------------电子版------------------------
\usepackage[a4paper,margin=2cm,driver=xetex]{geometry}
%\usepackage[paperwidth=9.2cm, paperheight=12.4cm, width=9cm, height=12cm,top=0.2cm,
%            bottom=0.4cm,left=0.2cm,right=0.2cm,foot=0cm, nohead,nofoot,driver=xetex]{geometry}

%===================自定义颜色=====================
\usepackage{xcolor}
  \definecolor{mybackgroundcolor}{cmyk}{0.03,0.03,0.18,0}
  \definecolor{myblue}{rgb}{0,0.2,0.6}

%====================字体设置======================
%--------------------中文字体----------------------
%-----------------------xeCJK下设置中文字体------------------------------%
\setCJKfamilyfont{song}{SimSun}                             %宋体 song
\newcommand{\song}{\CJKfamily{song}}                        % 宋体   (Windows自带simsun.ttf)
\setCJKfamilyfont{xs}{NSimSun}                              %新宋体 xs
\newcommand{\xs}{\CJKfamily{xs}}
\setCJKfamilyfont{fs}{FangSong_GB2312}                      %仿宋2312 fs
\newcommand{\fs}{\CJKfamily{fs}}                            %仿宋体 (Windows自带simfs.ttf)
\setCJKfamilyfont{kai}{KaiTi_GB2312}                        %楷体2312  kai
\newcommand{\kai}{\CJKfamily{kai}}
\setCJKfamilyfont{yh}{Microsoft YaHei}                    %微软雅黑 yh
\newcommand{\yh}{\CJKfamily{yh}}
\setCJKfamilyfont{hei}{SimHei}                                    %黑体  hei
\newcommand{\hei}{\CJKfamily{hei}}                          % 黑体   (Windows自带simhei.ttf)
\setCJKfamilyfont{msunicode}{Arial Unicode MS}            %Arial Unicode MS: msunicode
\newcommand{\msunicode}{\CJKfamily{msunicode}}
\setCJKfamilyfont{li}{LiSu}                                            %隶书  li
\newcommand{\li}{\CJKfamily{li}}
\setCJKfamilyfont{yy}{YouYuan}                             %幼圆  yy
\newcommand{\yy}{\CJKfamily{yy}}
\setCJKfamilyfont{xm}{MingLiU}                                        %细明体  xm
\newcommand{\xm}{\CJKfamily{xm}}
\setCJKfamilyfont{xxm}{PMingLiU}                             %新细明体  xxm
\newcommand{\xxm}{\CJKfamily{xxm}}

\setCJKfamilyfont{hwsong}{STSong}                            %华文宋体  hwsong
\newcommand{\hwsong}{\CJKfamily{hwsong}}
\setCJKfamilyfont{hwzs}{STZhongsong}                        %华文中宋  hwzs
\newcommand{\hwzs}{\CJKfamily{hwzs}}
\setCJKfamilyfont{hwfs}{STFangsong}                            %华文仿宋  hwfs
\newcommand{\hwfs}{\CJKfamily{hwfs}}
\setCJKfamilyfont{hwxh}{STXihei}                                %华文细黑  hwxh
\newcommand{\hwxh}{\CJKfamily{hwxh}}
\setCJKfamilyfont{hwl}{STLiti}                                        %华文隶书  hwl
\newcommand{\hwl}{\CJKfamily{hwl}}
\setCJKfamilyfont{hwxw}{STXinwei}                                %华文新魏  hwxw
\newcommand{\hwxw}{\CJKfamily{hwxw}}
\setCJKfamilyfont{hwk}{STKaiti}                                    %华文楷体  hwk
\newcommand{\hwk}{\CJKfamily{hwk}}
\setCJKfamilyfont{hwxk}{STXingkai}                            %华文行楷  hwxk
\newcommand{\hwxk}{\CJKfamily{hwxk}}
\setCJKfamilyfont{hwcy}{STCaiyun}                                 %华文彩云 hwcy
\newcommand{\hwcy}{\CJKfamily{hwcy}}
\setCJKfamilyfont{hwhp}{STHupo}                                 %华文琥珀   hwhp
\newcommand{\hwhp}{\CJKfamily{hwhp}}

\setCJKfamilyfont{fzsong}{Simsun (Founder Extended)}     %方正宋体超大字符集   fzsong
\newcommand{\fzsong}{\CJKfamily{fzsong}}
\setCJKfamilyfont{fzyao}{FZYaoTi}                                    %方正姚体  fzy
\newcommand{\fzyao}{\CJKfamily{fzyao}}
\setCJKfamilyfont{fzshu}{FZShuTi}                                    %方正舒体 fzshu
\newcommand{\fzshu}{\CJKfamily{fzshu}}

\setCJKfamilyfont{asong}{Adobe Song Std}                        %Adobe 宋体  asong
\newcommand{\asong}{\CJKfamily{asong}}
\setCJKfamilyfont{ahei}{Adobe Heiti Std}                            %Adobe 黑体  ahei
\newcommand{\ahei}{\CJKfamily{ahei}}
\setCJKfamilyfont{akai}{Adobe Kaiti Std}                            %Adobe 楷体  akai
\newcommand{\akai}{\CJKfamily{akai}}

%------------------------------设置字体大小------------------------%
\newcommand{\chuhao}{\fontsize{42pt}{\baselineskip}\selectfont}     %初号
\newcommand{\xiaochuhao}{\fontsize{36pt}{\baselineskip}\selectfont} %小初号
\newcommand{\yihao}{\fontsize{28pt}{\baselineskip}\selectfont}      %一号
\newcommand{\xiaoyihao}{\fontsize{24pt}{\baselineskip}\selectfont}
\newcommand{\erhao}{\fontsize{21pt}{\baselineskip}\selectfont}      %二号
\newcommand{\xiaoerhao}{\fontsize{18pt}{\baselineskip}\selectfont}  %小二号
\newcommand{\sanhao}{\fontsize{15.75pt}{\baselineskip}\selectfont}  %三号
\newcommand{\sihao}{\fontsize{14pt}{\baselineskip}\selectfont}%     四号
\newcommand{\xiaosihao}{\fontsize{12pt}{\baselineskip}\selectfont}  %小四号
\newcommand{\wuhao}{\fontsize{10.5pt}{\baselineskip}\selectfont}    %五号
\newcommand{\xiaowuhao}{\fontsize{9pt}{\baselineskip}\selectfont}   %小五号
\newcommand{\liuhao}{\fontsize{7.875pt}{\baselineskip}\selectfont}  %六号
\newcommand{\qihao}{\fontsize{5.25pt}{\baselineskip}\selectfont}    %七号   %中文字体及字号设置
\xeCJKDeclareSubCJKBlock{SIP}{
  "20000 -> "2A6DF,   % CJK Unified Ideographs Extension B
  "2A700 -> "2B73F,   % CJK Unified Ideographs Extension C
  "2B740 -> "2B81F    % CJK Unified Ideographs Extension D
}
%\setCJKmainfont[SIP={[AutoFakeBold=1.8,Color=red]Sun-ExtB},BoldFont=黑体]{宋体}    % 衬线字体 缺省中文字体

\setCJKmainfont{simsun.ttc}[
  Path=fonts/,
  SIP={[Path=fonts/,AutoFakeBold=1.8,Color=red]simsunb.ttf},
  BoldFont=simhei.ttf
]

%SimSun-ExtB
%Sun-ExtB
%AutoFakeBold:自动伪粗,即正文使用\bfseries时生僻字使用伪粗体;
%FakeBold:强制伪粗,即正文中生僻字均使用伪粗体
%\setCJKmainfont[BoldFont=STHeiti,ItalicFont=STKaiti]{STSong}
%\setCJKsansfont{微软雅黑}黑体
%\setCJKsansfont[BoldFont=STHeiti]{STXihei} %serif是有衬线字体sans serif 无衬线字体
%\setCJKmonofont{STFangsong}    %中文等宽字体

%--------------------英文字体----------------------
\setmainfont{simsun.ttc}[
  Path=fonts/,
  BoldFont=simhei.ttf
]
%\setmainfont[BoldFont=黑体]{宋体}  %缺省英文字体
%\setsansfont
%\setmonofont

%===================目录分栏设置====================
\usepackage[toc,lof,lot]{multitoc}    % 目录(含目录、表格目录、插图目录)分栏设置
  %\renewcommand*{\multicolumntoc}{3} % toc分栏数设置,默认为两栏(\multicolumnlof,\multicolumnlot)
  %\setlength{\columnsep}{1.5cm}      % 调整分栏间距
  \setlength{\columnseprule}{0.2pt}   % 调整分栏竖线的宽度

%==================章节格式设置====================
\setcounter{secnumdepth}{3} % 章节等编号深度 3:子子节\subsubsection
\setcounter{tocdepth}{2}    % 目录显示等度 2:子节

\xeCJKsetup{%
  CJKecglue=\hspace{0.15em},      % 调整中英(含数字)间的字间距
  %CJKmath=true,                  % 在数学环境中直接输出汉字(不需要\text{})
  AllowBreakBetweenPuncts=true,   % 允许标点中间断行,减少文字行溢出
}

\ctexset{%
  part={
    name={,篇},
    number=\SZX{part},
    format={\chuhao\bfseries\centering},
    nameformat={},titleformat={}
  },
  section={
    number={\chinese{section}},
    name={第,节}
  },
  subsection={
    number={\chinese{subsection}、},
    aftername={\hspace{-0.01em}}
  },
  subsubsection={
    number={(\chinese{subsubsection})},
    aftername={\hspace {-0.01em}},
    beforeskip={1.3ex minus .8ex},
    afterskip={1ex minus .6ex},
    indent={\parindent}
  },
  paragraph={
    beforeskip=.1\baselineskip,
    indent={\parindent}
  }
}

\newcommand*\SZX[1]{%
  \ifcase\value{#1}%
    \or 上%
    \or 中%
    \or 下%
  \fi
}

%====================页眉设置======================
\usepackage{titleps}%或者\usepackage{titlesec},titlesec包含titleps
\newpagestyle{special}[\small\sffamily]{
  %\setheadrule{.1pt}
  \headrule
  \sethead[\usepage][][\chaptertitle]
  {\chaptertitle}{}{\usepage}
}

\newpagestyle{main}[\small\sffamily]{
  \headrule
  %\sethead[\usepage][][第\thechapter 章\quad\chaptertitle]
%  {\thesection\quad\sectiontitle}{}{\usepage}}
  \sethead[\usepage][][第\chinese{chapter}章\quad\chaptertitle]
  {第\chinese{section}节\quad\sectiontitle}{}{\usepage}
}

\newpagestyle{main2}[\small\sffamily]{
  \headrule
  \sethead[\usepage][][第\chinese{chapter}章\quad\chaptertitle]
  {第\chinese{section}節\quad\sectiontitle}{}{\usepage}
}

%================ PDF 书签设置=====================
\usepackage{bookmark}[
  depth=2,        % 书签深度 2:子节
  open,           % 默认展开书签
  openlevel=2,    % 展开书签深度 2:子节
  numbered,       % 显示编号
  atend,
]
  % 相比hyperref,bookmark宏包大多数时候只需要编译一次,
  % 而且书签的颜色和字体也可以定制。
  % 比hyperref 更专业 (自动加载hyperref)

%\bookmarksetup{italic,bold,color=blue} % 书签字体斜体/粗体/颜色设置

%------------重置每篇章计数器,必须在hyperref/bookmark之后------------
\makeatletter
  \@addtoreset{chapter}{part}
\makeatother

%------------hyperref 超链接设置------------------------
\hypersetup{%
  pdfencoding=auto,   % 解决新版ctex,引起hyperref UTF-16预警
  colorlinks=true,    % 注释掉此项则交叉引用为彩色边框true/false
  pdfborder=001,      % 注释掉此项则交叉引用为彩色边框
  citecolor=teal,
  linkcolor=myblue,
  urlcolor=black,
  %psdextra,          % 配合使用bookmark宏包,可以直接在pdf 书签中显示数学公式
}

%------------PDF 属性设置------------------------------
\hypersetup{%
  pdfkeywords={黄帝内经,内经,内经讲义,21世纪课程教材},    % 关键词
  %pdfsubject={latex},        % 主题
  pdfauthor={主编:王洪图},   % 作者
  pdftitle={内经讲义},        % 标题
  %pdfcreator={texlive2011}   % pdf创建器
}

%------------PDF 加密----------------------------------
%仅适用于xelatex引擎 基于xdvipdfmx
%\special{pdf:encrypt ownerpw (abc) userpw (xyz) length 128 perm 2052}

%仅适用于pdflatex引擎
%\usepackage[owner=Donald,user=Knuth,print=false]{pdfcrypt}

%其他可使用第三方工具 如:pdftk
%pdftk inputfile.pdf output outputfile.pdf encrypt_128bit owner_pw yourownerpw user_pw youruserpw

%=============自定义环境、列表及列表设置================
% 标题
\def\biaoti#1{\vspace{1.7ex plus 3ex minus .2ex}{\bfseries #1}}%\noindent\hei
% 小标题
\def\xiaobt#1{{\bfseries #1}}
% 小结
\def\xiaojie {\vspace{1.8ex plus .3ex minus .3ex}\centerline{\large\bfseries 小\ \ 结}\vspace{.1\baselineskip}}
% 作者
\def\zuozhe#1{\rightline{\bfseries #1}}

\newcounter{yuanwen}    % 新计数器 yuanwen
\newcounter{jiaozhu}    % 新计数器 jiaozhu

\newenvironment{yuanwen}[2][【原文】]{%
  %\biaoti{#1}\par
  \stepcounter{yuanwen}   % 计数器 yuanwen+1
  \bfseries #2}
  {}

\usepackage{enumitem}
\newenvironment{jiaozhu}[1][【校注】]{%
  %\biaoti{#1}\par
  \stepcounter{jiaozhu}   % 计数器 jiaozhu+1
  \begin{enumerate}[%
    label=\mylabel{\arabic*}{\circledctr*},before=\small,fullwidth,%
    itemindent=\parindent,listparindent=\parindent,%labelsep=-1pt,%labelwidth=0em,
    itemsep=0pt,topsep=0pt,partopsep=0pt,parsep=0pt
  ]}
  {\end{enumerate}}

%===================注解与原文相互跳转====================
%----------------第1部分 设置相互跳转锚点-----------------
\makeatletter
  \protected\def\mylabel#1#2{% 注解-->原文
    \hyperlink{back:\theyuanwen:#1}{\Hy@raisedlink{\hypertarget{\thejiaozhu:#1}{}}#2}}

  \protected\def\myref#1#2{% 原文-->注解
    \hyperlink{\theyuanwen:#1}{\Hy@raisedlink{\hypertarget{back:\theyuanwen:#1}{}}#2}}
  %此处\theyuanwen:#1实际指thejiaozhu:#1,只是\thejiaozhu计数器还没更新,故使用\theyuanwen计数器代替
\makeatother

\protected\def\myjzref#1{% 脚注中的引用(引用到原文)
  \hyperlink{\theyuanwen:#1}{\circlednum{#1}}}

\def\sb#1{\myref{#1}{\textsuperscript{\circlednum{#1}}}}    % 带圈数字上标

%----------------第2部分 调整锚点垂直距离-----------------
\def\HyperRaiseLinkDefault{.8\baselineskip} %调整锚点垂直距离
%\let\oldhypertarget\hypertarget
%\makeatletter
%  \def\hypertarget#1#2{\Hy@raisedlink{\oldhypertarget{#1}{#2}}}
%\makeatother

%====================带圈数字列表标头====================
\newfontfamily\circledfont[Path = fonts/]{meiryo.ttc}  % 日文字体,明瞭体
%\newfontfamily\circledfont{Meiryo}  % 日文字体,明瞭体

\protected\def\circlednum#1{{\makexeCJKinactive\circledfont\textcircled{#1}}}

\newcommand*\circledctr[1]{%
  \expandafter\circlednum\expandafter{\number\value{#1}}}
\AddEnumerateCounter*\circledctr\circlednum{1}

% 参考自:http://bbs.ctex.org/forum.php?mod=redirect&goto=findpost&ptid=78709&pid=460496&fromuid=40353

%======================插图/tikz图========================
\usepackage{graphicx,subcaption,wrapfig}    % 图,subcaption含子图功能代替subfig,图文混排
  \graphicspath{{img/}}                     % 设置图片文件路径

\def\pgfsysdriver{pgfsys-xetex.def}         % 设置tikz的驱动引擎
\usepackage{tikz}
  \usetikzlibrary{calc,decorations.text,arrows,positioning}

%---------设置tikz图片默认格式(字号、行间距、单元格高度)-------
\let\oldtikzpicture\tikzpicture
\renewcommand{\tikzpicture}{%
  \small
  \renewcommand{\baselinestretch}{0.2}
  \linespread{0.2}
  \oldtikzpicture
}

%=========================表格相关===============================
\usepackage{%
  multirow,                   % 单元格纵向合并
  array,makecell,longtable,   % 表格功能加强,tabu的依赖
  tabu-last-fix,              % "强大的表格工具" 本地修复版
  diagbox,                    % 表头斜线
  threeparttable,             % 表格内脚注(需打补丁支持tabu,longtabu)
}

%----------给threeparttable打补丁用于tabu,longtabu--------------
%解决方案来自:http://bbs.ctex.org/forum.php?mod=redirect&goto=findpost&ptid=80318&pid=467217&fromuid=40353
\usepackage{xpatch}

\makeatletter
  \chardef\TPT@@@asteriskcatcode=\catcode`*
  \catcode`*=11
  \xpatchcmd{\threeparttable}
    {\TPT@hookin{tabular}}
    {\TPT@hookin{tabular}\TPT@hookin{tabu}}
    {}{}
  \catcode`*=\TPT@@@asteriskcatcode
\makeatother

%------------设置表格默认格式(字号、行间距、单元格高度)------------
\let\oldtabular\tabular
\renewcommand{\tabular}{%
  \renewcommand\baselinestretch{0.9}\small    % 设置行间距和字号
  \renewcommand\arraystretch{1.5}             % 调整单元格高度
  %\renewcommand\multirowsetup{\centering}
  \oldtabular
}
%设置行间距,且必须放在字号设置前 否则无效
%或者使用\fontsize{<size>}{<baseline>}\selectfont 同时设置字号和行间距

\let\oldtabu\tabu
\renewcommand{\tabu}{%
  \renewcommand\baselinestretch{0.9}\small    % 设置行间距和字号
  \renewcommand\arraystretch{1.8}             % 调整单元格高度
  %\renewcommand\multirowsetup{\centering}
  \oldtabu
}

%------------模仿booktabs宏包的三线宽度设置---------------
\def\toprule   {\Xhline{.08em}}
\def\midrule   {\Xhline{.05em}}
\def\bottomrule{\Xhline{.08em}}
%-------------------------------------
%\setlength{\arrayrulewidth}{2pt} 设定表格中所有边框的线宽为同样的值
%\Xhline{} \Xcline{}分别设定表格中水平线的宽度 makecell包提供

%表格中垂直线的宽度可以通过在表格导言区(preamble),利用命令 !{\vrule width1.2pt} 替换 | 即可

%=================图表设置===============================
%---------------图表标号设置-----------------------------
\renewcommand\thefigure{\arabic{section}-\arabic{figure}}
\renewcommand\thetable {\arabic{section}-\arabic{table}}

\usepackage{caption}
  \captionsetup{font=small,}
  \captionsetup[table] {labelfont=bf,textfont=bf,belowskip=3pt,aboveskip=0pt} %仅表格 top
  \captionsetup[figure]{belowskip=0pt,aboveskip=3pt}  %仅图片 below

%\setlength{\abovecaptionskip}{3pt}
%\setlength{\belowcaptionskip}{3pt} %图、表题目上下的间距
\setlength{\intextsep}   {5pt}  %浮动体和正文间的距离
\setlength{\textfloatsep}{5pt}

%====================全文水印==========================
%解决方案来自:
%http://bbs.ctex.org/forum.php?mod=redirect&goto=findpost&ptid=79190&pid=462496&fromuid=40353
%https://zhuanlan.zhihu.com/p/19734756?columnSlug=LaTeX
\usepackage{eso-pic}

%eso-pic中\AtPageCenter有点水平偏右
\renewcommand\AtPageCenter[1]{\parbox[b][\paperheight]{\paperwidth}{\vfill\centering#1\vfill}}

\newcommand{\watermark}[3]{%
  \AddToShipoutPictureBG{%
    \AtPageCenter{%
      \tikz\node[%
        overlay,
        text=red!50,
        %font=\sffamily\bfseries,
        rotate=#1,
        scale=#2
      ]{#3};
    }
  }
}

\newcommand{\watermarkoff}{\ClearShipoutPictureBG}

\watermark{45}{15}{草\ 稿}    %启用全文水印

%=============花括号分支结构图=========================
\usepackage{schemata}

\xpatchcmd{\schema}
  {1.44265ex}{-1ex}
  {}{}

\newcommand\SC[2] {\schema{\schemabox{#1}}{\schemabox{#2}}}
\newcommand\SCh[4]{\Schema{#1}{#2}{\schemabox{#3}}{\schemabox{#4}}}

%=======================================================

\begin{document}
\pagestyle{main}
\fi
\chapter{《黄帝内经》的体质医学思想}%第五章

中医学对体质进行探索始于《黄帝内经》。《素问·异法方宜论》、《灵枢·阴阳二十五人》、《通天》等篇,比较系统地论述了有关体质的内容,从而奠定了中医体质学说的理论基础,并在后世得到不断发展。20世纪80年代初期,《中医体质学说》专著问世,它以《内经》体质学说理论作为基石,比较系统地论述了体质的分类、形成,体质与发病,体质与辨证,体质与治疗等内容,对《内经》体质学说的研究起了积极的推动作用。

\section{《内经》体质医学思想的基本内容}%第一节

\subsection{体质的含义}%一、

《内经》蕴藏着丰富的体质医学思想,许多篇章都是专论体质的,如《素问·异法方宜论》、《血气形志篇》、《灵枢·阴阳二十五人》、《通天》、《寿夭刚柔》、《经水》、《骨度》、《肠胃》、《论勇》、《卫气失常》、《逆顺肥瘦》和《五变》等。虽然没有“体质”一词,但记载了不少含义相近的词,如“素”、“质”、“身”、“形”等。《素问·逆调论》说:“是人者,素肾气胜。”《素问·厥论》说:“此人者质壮。”由此,后世一些古医籍也常常称体质为“禀质”、“资质”等。较早提出“体质”一词的如张介宾,他在《景岳全书·杂证谟》中说:“矧体质贵贱尤有不同,凡藜藿壮夫,及新暴之病,自宜消伐”。此后,在清代叶桂、吴瑭等论著中,也相继出现“体质”一词。然而,他们均未阐明体质一词的含义。近十多年,随着体质学说研究的不断深入,医学界对体质的含义也展开了讨论,倾向于这样的认识:

体质是人类的个体在功能、形态、结构上相对稳定的特殊性,它在生理上表现为个体的生理反映的特性,在病理上则表现为个体的发病倾向性。因此体质强调的是个体的形体结构及生理功能的特性。

从《内经》所论述的体质思想现之,则既突出了个体形体结构和生理功能的特性,又强调了其心理、性格、精神面貌、道德品质等方而的特征。而后者是属于“气质”的范畴。因此,《内经》中的体质内容,实际上是将体质与气质融会于一体。这主要源于《内经》“形神合一”的整体观。

\subsection{体质形成的因素}%二、

体质的形成与人之生命发展过程相关,而生命的形成又是一个非常复杂的过程。

《素问·宝命全形论》说:“人以天地之气生,四时之法成”,说明生命的发展与自然界息息相关。《灵枢·天年》又指出:“人之始生,何气筑为基,何立而为楯,何失而死?何得而生?岐伯曰:以母为基,以父为楯,失神者死,得神者生也。”说明生命之初乃由父精母血相合并聚而生。因此,影响体质形成的因素也不外乎先天和后天两个方面。

\subsubsection{先天禀赋}%(一)

先天禀赋是建造人体体质的第一块基石,人体体质的强弱在很大程度上取决于先天因素。先天因素一般是指胎儿出生前影响胎儿的各种因素。如人之胚胎是秉承了父母的遗传基因,而形成了这一胚胎所具有的独特的体质特性,故当婴儿出生时,其体质已能显示出个体的差异性,故《灵枢·寿夭刚柔》说:“人之生也,有刚有柔,有弱有强,有短有长,有阴有阳。”此外,《内经》已认识到胎儿期母体的药食、染病、精神状态等都可能影响子代的体质。如《素问·奇病论》记载:“人生面有病癫疾者,病名曰何?安所得之?歧伯曰:病名为胎病,此得之在母腹中时,其母有所大惊,气上而不下,精气并居,故令子发为癫疾也。”认为原发性癫疾缘由先天所致。这种观点亦使后世医家探索到一些疾病的发生与先天体质相关。如《诸病源候论·漆疮候》曾对漆的过敏作过研究,发现“漆有毒,人有秉性畏漆,但见漆便中其毒”,但“亦有性自耐者,终日浇煮,竟不为害,”故而得出结论:“人无问男女大小,有禀不耐漆者,见漆及新漆器,便著漆毒,”说明秉性为病本。

现代遗传学研究认为:个体从其双亲处继承下来的全部物质及遗传信息均在受精卵的染色体中,染色体内含有的DNA分子可以呈现出无穷无尽的碱基排列方式。由一定结构的DNA便常带来一定的形态结构和生理特性,从而成为各型体质在遗传方面的物质基础。据研究,世界上没有两个人的DNA会有相同的碱基排列次序,因此,严格地说,世界上没有体质完全相同的人。

现代免疫学认为,遗传因素是天然非特异免疫因素中作用最明显的因素,它决定个体来自遗传的免疫差异,遗传对于抗体的种类型别及血清中的含量都起决定性的作用,因而可以用来解释为什么个体对某些疾病反应形式具有体质上的特殊性。

\subsubsection{后天因素}%(二)

后天因素包括地理环境、饮食营养、精神状态、年龄差别、劳逸状况、社会因素、疾病作用以及针药影响等。人从呱呱落地之时起,这些因素即开始对体质的形成发挥重要作用。

1.地理环境:《素问·六节藏象论》曰“天食人以五气,地食人以五味。”人不断地从其赖以生存的自然环境获得维持生命活动的物质,而自然环境时刻也没有忘记对人体进行“雕琢”。《素问·异法方宜论》详细地描述了我国古代五方人民因居处环境、饮食习惯等的不同而造成体质上的差异。内容已见本篇第三章,此不赘述。

地理环境对体质形成的影响,近年多有研究。生态学认为:生物体中所存在的全部化学物都来自土壤、空气和水。不同地区的地壳中所含的化学物质不同,其中水质与体质的关系最为密切,《吕氏春秋·尽数》记载了五种不同水土与人群疾病的关系,说:“轻水所,多秃与瘿人”“重水所多尰与躄人”“甘水所多好与美人”“辛水所多疸与痤人”,“苦水所多尫与伛人”,将各地不同的水质与地方性多发病的关系展示出来。不同的地质及地势高低等均会给人的体质施加影响。这可能与地表元素分布的不均有一定关系,从而成为制约不同区域人体体质发育的重要地质因素。亦有人认为地区性气候类型与长期生活在该地区的人群体质有密切关系,如我国南方多湿热,北方多寒燥,东部沿海为海洋性气候,西部内陆为大陆性气候,都会直接影响人们的体质。

2.饮食习惯:《内经》认为:合理的饮食,充足而全面的营养,可增强人的体质,甚至可使某些病理性体质转变为生理性体质。反之,饥饱不时,五味偏嗜,营养不足,无疑将使体质受到损害。如《素问·痹论》所言:“饮食自倍,肠胃乃伤”;《素问·生气通天论》亦指出“因而饱食,筋脉横解,肠澼为痔”;“味过于酸,肝气乃津,脾气乃绝;味过于咸,大骨气劳,短肌,心气抑;味过于甘,心气喘满,色黑,肾气不衡;味过于苦,脾气不濡,胃气乃厚;味过于辛,筋脉沮弛,精神乃央。”并对恣食肥甘,影响体质作了具体的分析,如:“肥者令人内热,甘者令人中满”(《素问·奇病论》);“高梁之变,足生大丁,受如持虚。”(《素问·生气通天论》)。对此,现代研究指出,长期呈饥饿状态的人其体质易转变为倦㿠质;饮食偏嗜可使体质向不良方向转变。有人认为,过食生冷寒凉会伤脾损胃,产生脾虚体质;过食辛辣炙煿,会酿成火热之体,造成体内阴阳失调,从而使体质衰退。

3.精神状态:精神情志,贵在和调。精神舒畅,则人体气血调畅,脏腑功能协调,体质就会强健。如《灵枢·本脏》篇说:“意志和则精神专直,魂魄不散,悔怒不起,五脏不受邪矣。”长期精神刺激,或突然道到剧烈的精神创伤,一旦超过人体的生理调节能力,就会影响脏腑经络功能和气血运行。《素问·阴阳应象大论》曰:“怒伤肝”,“喜伤心”,“思伤脾”,“忧伤肺”,“恐伤肾”;《素问·举痛论》指出,“怒则气上,喜则气缓,悲则气消,恐则气下,……惊则气乱,思则气结。”精神刺激过久还可进一步引起体质改变,从而形成一定的体质类型,如《内经》认为肝“在志为怒”,“暴怒伤阴、暴喜伤阳”,若经常忿怒,肝气横逆,不仅可伤肝阴肝血,导致阴不制阳,肝阳上亢,形成阳亢的体质,而且还可横逆犯脾,损伤脾胃运化之用,使水谷精气生化乏源,形成脾虚体质,故不可小觑精神因素对体质的影响。

精神因素与体质的关系,从现代有关研究中亦可以得到验证。据报道,人类心理活动能够引起机体内部巨大的生理功能变化,如愉快、兴奋、激动可使肾上腺分泌激素增加,糖类代谢加速,血糖升高,肌肉活动力加强,机体抵抗力增强。反之,剧烈的精神创伤,悲观的情绪,消沉的意念,会引起大脑皮层的功能紊乱和抑制,从而引起一系列相关的脏器功能和器质的病变。

国外有人把人的性格分为三种类型。A型是知足常乐,含蓄安静,小心谨慎;B型的特征是活泼开朗,积极上进;V型的特证是情绪波动,易躁易怒,很不知足。通过对部分对象追踪30年的观察,结果发现,V型的人中77.3\%患有心血管疾病(高血压、冠心病、心肌梗死等)、癌症、良性肿瘤等,难怪国外精神病学专家指出“人精神遭受痛苦,就意味着身体健康遭到至少长达五年的损害。”此言与《内经》的认识完全一致。

4.年龄差异:机体的组织、脏腑功能、气血津液代谢常随人体的生、长、壮、老的变化过程而发生改变,体质从幼弱到壮盛直至虚衰,各个阶段均有其各自的特征。《素问·上古天真论》、《灵枢·天年》等都非常精辟地阐述了人体生命过程中各阶段体质变化的不同特征。如《素问·上古天真论》载女子以七岁、男子以八岁为一阶段,论述了各不同年龄段的生理特征及体质情况。并说明在整个过程中肾气始终在起着主导的作用,体质的强弱是随肾气的盛衰而发生同步的变化。生长发育期间,体质由弱转强,婴儿者,“肉脆血少气弱”(《灵枢·逆顺肥瘦》),易感受外邪。到二七、二八时,肾气盛实,天癸发生作用,表现出青春期男女体质上、生殖机能上的各种显著变化,在以性发育与成熟为主的过程中,人体的各种生理机能进行着整体性的调整,使体质产生明显的改变,例如一些在儿童期体弱多病,患有哮喘、枸偻病或过敏性疾病的人,在青春期不仅病证消失,且身体健壮,体质明显好转。三七、四七、三八、四八时,肾气平均,体质壮盛,此时经过青春期的发育,性机能完全成熟,身高、体重的相对稳定标志着成年期的体形特点,五脏生理功能的旺盛、协调使这一时期人体精力充沛,体质甚佳;女子五七至七七及男子五八至八八,肾气衰,体质由强趋衰,最后至七七、八八之后“肾脏衰”时,则人已经“行步不正”、“筋骨解堕”,完全是一派衰老败坏的体质形态。所以体质有年龄的差异,而其实质还是在于肾气。《灵枢·天年》、《素问·阴阳应象大论》等还以十岁为阶段对人的衰老过程做了介绍:“年四十而阴气自半,起居衰矣”,“五十岁,肝气始衰,肝叶始薄,胆汁始灭,目始不明;……百岁,五脏皆虚,神气皆去,形骸独居而终矣。”观之临床实践,老年人之所以容易患病,乃由体质因素所决定。据调査,60$\sim$64岁的老人患有重要脏器的器质性疾病的占50\%,70岁的占65.9\%,年事越高,百分比越高。

5.性别因素:男女性别不同,其体质亦各有特征,从上述《素问·上古天真论》的论述来看,女子的发育较男子早1$\sim$2年左右,女子是以七为分阶段基数,男子是以八为分阶段基数,除了男女有生殖系统的生理差异外,《内经》还强调了两者的体质的差异。《灵枢·五音五味》篇指出:“妇人之生,有余于气,不足于血。”说明女子因月事以时下,数脱于血,故其体质特征是气盛血虚,后世医家据此强调女子以血为本。而男子因生理功能原因,其精气易亏,故体质特征多见精气虚少,故有“男子以精为本”之说。

从现代研究来看,由于男女性别因素,一些易感的疾病发病率有显著差异,如血液病、心血管病、呼吸系统病、神经系统病等,男性发病率要高于女性31.2\%$\sim$50.2\%。所以男女的体质不同,其发病倾向也是不同的。

6.社会因素:社会因素包括社会的经济、政治和社会风气等各种因素。社会制度、主产力水平、生活状况、思想道德、文化素质等对人的体质都会造成一定的影响。一般来说,社会动乱、战争连年、灾荒饥饿、人民流离失所、饥寒交迫、贫富急剧变化等都对人的体质有不利影响。《素问·疏五过论》指出:那些经历“尝贵后贱”,“尝富后贫”、“暴乐暴苦”、“始乐后苦”的人群,其体质虚衰者多见,“身体日减,气虚无精”,“精气竭绝、形体毁沮”。李东垣所处的年代,就见到京师戒严,被困城中的人们体质普遍急剧下降。但是,尚有一些社会倾向不可忽视,如现代社会中人们过度追求安逸,进门有空调,出门以车代步,家务劳动社会化、电气化,反而体质下降,如易感冒,疲困,纳食不香,失眠,形体呈肥胖型或豆芽型,这是现代文明社会带来的体质变化现象。此外,现代工业发展带来的种种污染已成为日益严重的社会问题,它直接地威胁着人体健康,改变着人的体质。如森林的减少、土地的沙漠化、臭氧层黑洞、二氧化碳增加等,也是影响人体体质不可忽视的因素。

\subsection{体质医学思想的藏象学说基础}%三、

体质既然是一种生理特性,那么这种生理特性的形成必然与其脏腑经络、精气神的功能有关。《内经》中的体质医学思想的基础是藏象理论。

体质的“体”是强调个体在脏腑、气血形质上的差异性,《内经》认为脏腑的形态、大小、质地、位置是体质产生差异的基础,如《灵枢·本脏》说:“视其外应,以知其内脏”,“五脏者,固有大小高下坚脆端正偏倾者,六腑亦有大小长短厚薄结直缓急”。若“五脏皆小者,少病,苦燋心,大愁忧;五脏皆大者,缓于事,难使以忧;五脏皆高者,好高举措;五脏皆下者,好出人下。五脏皆坚者,无病;五脏皆脆者,不离于病;五脏皆端正者,和利得人心;五脏皆偏倾者,邪心而善盗,不可以为人平,反复言语也。”这种观点亦见之于《灵枢·论勇》,认为勇士(体质强者)多因“其心端直,其肝大以坚,其胆满以傍”;怯士(体质弱者)则“肝系缓,其胆不满而纵,肠胃挺,胁下空。”

体质的特性是与个体脏腑、气血功能特点的差异密切相关。如形体的肥瘦泽枯,是由气血的盛衰所致,在《灵枢·阴阳二十五人》中说:“其肥而泽者,血气有余;肥而不泽者,气有余,血不足;瘦而无泽者,气血俱不足。”再如《灵枢·卫气失常》说:“膏者多气,多气者热,热者耐寒。肉者多血则充形,充形则平。脂者,其血清,气滑少,故不能大。”大多数不胖不瘦的人是因其“皮肉脂膏不能相加也,血与气不能相多,故其形不小不大,各自称其身,命曰众人”。“年质壮大”者,说明体质强,那是因其“血气充盛”而产生的体质类型(《灵枢·逆顺肥瘦》)。

\subsection{体质的分类及其特点}%四、

《内经》中对体质的类型已有所论述,因分类的角度不同,分类方法亦异,但却均以阴阳五行、脏腑气血形志作为分类依据。故这些分类方法不仅有理论基础,而且有临床参考价值。

1.五行分类法:这种分类方法主要见于《灵枢·阴阳二十五人》中,此篇将人体的肤色、形体、举止、性格及其对气候耐受力等特点按五行学说,划分为木形、火形、土形、金形、水形等五种不同的体质类型。此五种类型是基本型,在此基础上,还要根据五音太少、阴阳属性以及手足三阳经的左右上下、气血多少盛衰之差异,再将每一类基本型推演成五种亚型,而为二十五种体质类型。故曰:“先立五形金、木、水、火、土,别起五色,异其五形之人,而二十五人具矣。”由于这一分类方法包括了个体之形态、心理特征、性格特点诸方面,故而成为《内经》中最全面的体质分类方法。(表\ref{tab:体质五行分类表})
%\vfill
%\pagebreak[4]
\begin{center}%体质五行分类表
	\renewcommand\baselinestretch{0.9}\small %设置行间距和字号
	\renewcommand\arraystretch{1.3}
	\begin{longtabu}to\textwidth{ccXXXcX[1.1,cp]X[1.1,lp]}
		\caption{体质五行分类表}\label{tab:体质五行分类表} \\
			\toprule
			\multicolumn{5}{c}{基本型体质特点} & \multicolumn{3}{c}{亚型体质特点} \\ \hline
			\rowfont[c]{}类型 & 肤色 & 形态特征 & 举止 & 心理特征 & 五音 & 阴阳上下属性 & 性格 \\
			\midrule
		\endfirsthead
			\multicolumn{8}{r}{\small\sl 接上表}\\
			\toprule
			\multicolumn{5}{c}{基本型体质特点} & \multicolumn{3}{c}{亚型体质特点} \\ \hline
			\rowfont[c]{}类型 & 肤色 & 形态特征 & 举止 & 心理特征 & 五音 & 阴阳上下属性 & 性格 \\
			\midrule
		\endhead
			\bottomrule
			\multicolumn{8}{l}{\small\sl 接下表}\\
		\endfoot
			\bottomrule
		\endlastfoot
		\multirow{5}{*}{木形} & \multirow{5}{*}{苍色} & \multirow{5}{6em}{小头、长面、大肩背,直身,小手足(身材修长俊秀)} & \multirow{5}{6em}{少力} & \multirow{5}{6em}{有才,劳心,多忧,劳于事} & 上角 & 足厥阴 & 佗佗然(雍容自得貌) \\
		 &  &  &  &  & 大角 & 左足少阳之上 & 遗遗然(退让貌) \\
		 &  &  &  &  & 钛角 & 右足少阳之上 & 推推然(勇于进取貌) \\
		 &  &  &  &  & 左角 & 右足少阳之下 & 随随然(柔顺随和貌) \\
		 &  &  &  &  & 判角 & 左足少阳之下 & 栝栝然(方正端直貌) \\
		\multirow{5}{*}{火形} & \multirow{5}{*}{赤色} & \multirow{5}{6em}{广䏖、锐面、小头、好肩背、髀腹,小手足(身材不高,面尖,但肩背肌肉丰满)} & \multirow{5}{6em}{行安地、疾心、行摇} & \multirow{5}{6em}{有气轻财,少信多虑,见事明,好颜急心} & 上微 & 手少阴 & 核核然(真诚朴实貌) \\
		 &  &  &  &  & 质徴 & 左手太阳之上 & 肌肌然(浮躁貌) \\
		 &  &  &  &  & 右徵 & 右手太阳之上 & 鲛鲛然(活跃爽快貌) \\
		 &  &  &  &  & 少徵 & 右手太阳之下 & 慆慆然(乐观喜悦貌) \\
		 &  &  &  &  & 质判 & 左手太阳之下 & 支支颐颐然(怡然自得貌) \\
		\multirow{5}{*}{土形} & \multirow{5}{*}{黄色} & \multirow{5}{6em}{圆面、大头、美肩背、大腹、美股胫、小手足、多肉、上下相称(肥胖丰满,上下匀称)} & \multirow{5}{6em}{行安地、举足浮} & \multirow{5}{6em}{安心,好利人,不喜权势、善附人也} & 上宫 & 足太阴 & 敦敦然(诚实忠厚貌) \\
		 &  &  &  &  & 太宫 & 左足阳明之上 & 婉婉然(婉转和顺貌) \\
		 &  &  &  &  & 少宫 & 右足阳明之上 & 枢枢然(灵活敏捷貌) \\
		 &  &  &  &  & 左宫 & 右足阳明之下 & 兀兀然(勤奋自主貌) \\
		 &  &  &  &  & 加宫 & 左足阳明之下 & 坎坎然(端庄持重貌) \\
		\multirow{5}{*}{金形} & \multirow{5}{*}{白色} & \multirow{5}{6em}{方面、小头、小肩背、小腹、小手足,如骨发踵外} & \multirow{5}{6em}{骨轻(动作轻)} & \multirow{5}{6em}{身清廉。急心,静桿、善为吏} & 上商 & 手太阴 & 敦敦然(敏厚诚实貌) \\
		 &  &  &  &  & 钛商 & 左手阳明之上 & 廉廉然(洁身自好貌) \\
		 &  &  &  &  & 左商 & 右手阳明之上 & 监监然(善于辨察貌) \\
		 &  &  &  &  & 少商 & 右手阳明之下 & 严严然(严肃庄重貌) \\
		 &  &  &  &  & 右商 & 左手阳明之下 & 脱脱然(潇洒超脱貌) \\
		\multirow{5}{*}{水形} & \multirow{5}{*}{黑色} & \multirow{5}{6em}{面不平、大头、廉颐、小肩、大腹、下尻长,背廷延然(面背皆瘦,腹大而尻背修长)} & \multirow{5}{6em}{动手足、发行摇身} & \multirow{5}{6em}{不敬畏,善欺给人,戮死} & 上羽 & 足少阴 & 汗汗然(行为不洁貌) \\
		 &  &  &  &  & 桎之人 & 左足太阳之上 & 安安然(心胸坦荡貌) \\
		 &  &  &  &  & 大羽 & 右足太阳之上 & 颊颊然(得意貌) \\
		 &  &  &  &  & 众之人 & 右足太阳之下 & 洁洁然(性情坦白貌) \\
		 &  &  &  &  & 少羽 & 左足太阳之下 & 纡纡然(纡曲不爽貌) \\
	\end{longtabu}
\end{center}

2.阴阳分类法:将体质按阴阳多少来分类的方法主要见于《灵枢·通天》中,分为太阴之人、少阴之人、太阳之人、少阳之人、阴阳和平之人等五类体质。比较强调个体内阴阳盛衰的差异可以导致个体间在形态、秉性、行为、心理特点、生理特征方面的种种特异性。(表\ref{tab:体质阴阳五态分类表})

\begin{table}[htb!p]%体质阴阳五态分类表
	\centering
	\caption{体质阴阳五态分类表}\label{tab:体质阴阳五态分类表}
	%\resizebox{\textwidth}{!}{%
	%\let\tabu\oldtabu
	\tabulinesep=^5pt_4pt
	%\renewcommand\baselinestretch{1}\small
	\begin{tabu}{p{4em}p{4em}p{10em}p{9em}p{10em}}
		\toprule
		\rowfont[c]{}
		类型         & 阴阳含量 & 生理特征 & 心理特征 & 行为特征 \\
		\midrule
		太阴之人     & 多阴少阳 & 其明血浊,其卫气涩,明阳不和,缓筋而厚皮     & 贪而不仁,下齐湛湛,好内而恶出,心和而不发,不务于时,动而后之(阴险,深藏不露,贪婪)                                         & 黯黯然黑色,念然下意,临临然长大,腘然未偻(皮肤黑,个大而卑恭) \\
		少阴之人     & 多阴少阳 & 小胃而大肠,六腑不调,其阳明脉小,而太阳脉大 & 小贪而贼心,见人有亡,常若有得,好伤好害,见人有荣,乃反愠怒,心疾而无恩(嫉妒,无同情心)                                     & 清然,窃然,固以阴贼,立而躁崄,行面似伏(行为鬼祟、阴险、贼头贼脑貌) \\
		太阳之人     & 多阳少阴 &                                              & 居处于于,好言大事,无能而虚说,志发于四野,举措不顾是非,为事如常自用,事虽败而常无悔(随遇而安,说大话。自信,失败而不追悔) & 轩轩储锗,反身折腘(形容趾高气扬,挺胸撷肚) \\
		少阳之人     & 多阳少阴 & 经小而络大,血在中而气在外,实阴而虚阳       & 諟谛好自责,有小小官,则高自宜,好为外交而不内附(精细,自己抬高自己,好外交而不能踏实做事。)                                 & 立则好仰,行则好摇,其两臂两肘则常出于背 \\
		阴阳平和之人 & 阴阳气和 & 血脉调                                       & 居处安静,无为惧惧,无为欣欣,婉然从物,或与不争,与时变化,尊则谦谦,谭而不治,是谓至治(举止泰然,不慕名利,婉转谦逊)       & 委委然、随随然、颙颙然、愉愉然、䁢䁢然、豆豆然,众人皆曰君子 \\
		\bottomrule
	\end{tabu}
	%}
\end{table}

3.体形分类法:将体质按外表体型肥瘦、壮弱不同区分为不同的类型,这种方法见于《灵枢·逆顺肥瘦》和《灵枢·卫气失常》中。体形的肥瘦壮弱主要与气血的盛衰、多少,运行的滑利、涩滞情况密切相关。(表\ref{tab:肥瘦、壮幼分类表}、表\ref{tab:肥胖体质分类表})

\begin{table}[htb!p]%肥瘦、壮幼分类表
	\centering
	\caption{肥瘦、壮幼分类表}\label{tab:肥瘦、壮幼分类表}
	\begin{tabu}to.87\textwidth{@{}X[c]X[-3,l]X[-3,l]X[-1.5,l]}
		\toprule
		\rowfont[c]{}
		分型     & 形体特征                             & 气血特征                   & 性格特征 \\
		\midrule
		肥人     & 年质壮大,肤革坚固                   & 血气充盈                   &  \\
		壮人     & 广肩、腋项肉薄,厚皮而黑色,唇临临然 & 血黑以浊,气涩以迟         & 贪于取与 \\
		瘦人     & 皮薄、色少、肉廉廉然,薄唇           & 血清气滑,易脱气、损血     & 轻言 \\
		肥瘦适中 & 端正                                 & 血气和调                   & 敦厚 \\
		壮士     & 坚肉缓节,监监然                     & 重则气涩血浊,劲则气滑血清 & \\
		\bottomrule
	\end{tabu}
\end{table}

\begin{table}[htb!p]%肥胖体质分类表
	\centering
	\caption{肥胖体质分类表}\label{tab:肥胖体质分类表}
	\begin{tabu}to.87\textwidth{@{}X[c]X[-3,l]X[-3,l]X[-1.5,l]}
		\toprule
		\rowfont[c]{}类型 & 形态特征               & 生理特征                     & 气血多少 \\
		\midrule
					 膏型 & 肉不坚、皮缓纵腹垂腴   & 肉淖而粗理者身寒,细理者身热 & 多气 \\
					 脂型 & 肌肉坚、皮满,其身收小 & 其肉坚、细理者热             & 血清气滑少 \\
					 肉型 & 皮肉不相离、身体容大   & 粗理者寒                     & 多血 \\
		\bottomrule
	\end{tabu}
\end{table}
4.气质分类法:心理学认为,气质是人的心理特征之一,是指人在生长发育过程中所形成的思维、认识、情感等方面的个体特征,它是人的高级神经活动类型在人的行为和活动中的表现。在《灵枢·通天》的阴阳五态人分类里已包含了气质内容,但因其与阴阳多少关系密切,故又归为“阴阳分类法”;而《灵枢·论勇》中的勇怯分类及《素问·血气形志篇》中的形志苦乐分类则完全显示出此种分类特点,由此可以看出《内经》中体质包含气质、形神气血的思想。

\section{《内经》体质医学思想的临床应用}%第二节

\subsection{体质与疾病}%一、

发病有外感,有内伤。就外感发病而论,体质属于内因范畴,致病邪气则为外因,外因只有通过内因才能起作用。感受外邪后发病与否与体质有重要关系。《灵枢·五变》以匠人伐木类比人之体质与发病的关系。认为“一时遇风,同时得病,其病各异”的根本原因在于体质不同。“匠人磨斧斤砺刀,削斲材木。木之阴阳,尚有坚脆,坚者不入,脆者皮弛,至其交节,而缺斤斧焉。夫一木之中,坚脆不同,坚者则刚,脆者易伤,况其材木之不同,皮之厚薄,汁之多少,而各异耶。夫木之早花先生叶者,遇春霜烈风,则花落而叶萎。久曝大旱,则脆木薄皮者,枝条汁少而叶萎。久阴淫雨,则薄皮多汁者,皮溃而漉。卒风暴起,则刚脆之木,枝折杌伤。秋霜疾风,则刚脆之木,根摇而叶落。凡此五者,各有所伤,况于人乎?”什么样的人容易受邪,受什么样的邪?受邪后发生什么性质的疾病,这在相当程度上决定于体质。故《五变》举例说:“肉不坚,腠理疏,则善病风”;“五脏皆柔弱者,善病消瘅”;“小骨弱肉者,善病寒热”;“粗理而肉不坚者,善病痹”,“皮肤薄而不泽,肉不坚而淖泽,如此则肠胃恶,恶则邪气留止,积聚乃伤。脾胃之间,寒温不次,邪气稍至;稸积留止,大聚乃起”。

体质对疾病的发生起着重要作用,甚至是决定性的作用,然而也不能因此忽视邪气的作用。必须指出,在特殊条件下有时邪气也起到十分重要的作用,如烈性传染病对人体的损害即是如此。此时,即使体质强壮者,亦不能恃强而无所避,故《素问·刺法论》曰:“正气存内,邪不可干,避其毒气。”《素问·上古天真论》亦说,“虚邪贼风,避之有时。”

对内伤病证而言,发病与体质亦有密切关系。《素问·经脉别论》认为,当夜行劳倦、堕坠惊恐、渡水跌仆等情况出现时,“勇者气行则已;怯者则着而为病。”说明当个体处于上述诸种特殊的环境中时,病与不病取决于人之体质强弱,心理素质勇怯等因素。

另外,发病与体质还有同气相求,内外相互感应的关系,故个体的体质特殊性,往往决定着他对某些致病因素的易感性。《灵枢·邪气脏腑病形》说:“邪之中人脏,奈何?岐伯曰:形寒寒饮则伤肺,以其两寒相感,中外皆伤,故气逆而上行。”这是寒邪与体内寒饮同气相求,两气相感的结果。

体质不仅与发病相关,而且与发病之后疾病的变化关系密切。如虚邪之中人,“在肠胃之时,贲响腹胀,多寒则肠鸣飧泄,食不化;多热则溏出糜。”(《灵枢·百病始生》)此“多寒”、“多热”之变化不同,就源于患者体质。阳盛体质者,受邪后易热化,故出现大肠湿热下注之证;阴盛体质者病易寒化,故见脾肾虚寒之飧泄。再如《素问·痹论》也提到同样感受风寒湿之邪,导致痹证,但“阳气少,阴气多”的体质者,表现为肢体骨节寒冷、疼痛剧烈的痛痹;而“阳气多、阴气少”的体质者,则表现为骨节红肿热痛、发热、口干、舌红的热痹。

体质是影响疾病预后的关键,大凡体质壮实者,抗邪有力,病程短,预后良好;体质弱者,抗病能力弱,邪易乘虚内陷,病多难治愈。因此,《素问·评热病论》论劳风时说:“精者三日,中年者五日,不精者七日。”《灵枢·论痛》曰:“同时而伤,其身多热者易已;多寒者难已”,明确提出阳气盛,体质强者易已;阴气盛体质弱者难已。

\subsection{体质与诊断}%二、

辨证论治是中医的特色之一,辨证首先要辨的就是病人的体质,因为同一种疾病,由于个体体质差异,因而对致病因素的反应性各不相同,其临床表现各异,于是形成了不同的“证”,所以辨证的本质,从体质医学思想的角度观之,就是先辨体质,然后确定“证”。《内经》已经初步认识到体质在诊断中以及病证变化过程中的特殊作用,故《内经》诊法中亦突出了察体质的内容。如《素问·经脉别论》说:“诊病之道,观人勇怯骨肉皮肤,能知其情,以为诊法也。”《素问·疏五过论》曰:“圣人之治病也……问年少长,勇怯之理,审于分部,知病本始。”《素问·征四失论》说:“不适贫富贵贱之居,坐之厚薄,形之寒温,不适饮食之宜,不别人之勇怯,不知比类,足以自乱,不足以自明。”这些均说明在诊病时应审察五脏强弱、形之盛衰、年龄、勇怯、饮食、社会经历等因素以了解体质的情况,从而作为辨证的重要依据。不仅如此,《内经》还十分重视体质与疾病生死寿夭的关系。如《素问·三部九候论》说:“决生死奈何?岐伯曰:形盛脉细,少气不足以息者危,形瘦脉大,胸中多气者死,形气相得者生,参伍不调者病。”《灵枢·寿夭刚柔》篇指出形气与精神活动相适应者寿,不相适应者夭;血气盛,充形体,皮固肉坚者寿;血气不能充养形体,皮疏脆者夭。

《内经》辨体质的思想对后世有所启示,《临证指南医案·湿》说:“治法总宜辨体质阴阳,斯可以知寒热虚实之治。若其人色苍赤而瘦,肌肉坚结者,其体属阳,此外感湿邪,必易于化热;若内生湿热,多因膏粱酒醴,必患湿热湿火之症。若其人色白而肥,肌肉柔软者,其体属阴,若外感湿邪不易化热;若内生之湿,多因茶汤生冷太过,必患寒湿之证。”可见叶氏临证时对辨体质论证深有体会。

\subsection{体质与治疗}%三、

1.因体质制宜,辨质论治:“因人制宜”,主要是“因体质制宜”。《素问·三部九候论》说:“必先度其形之肥瘦,以调其气之虚实,实则泻之,虚则补之。”这种观点在《内经》中多处论及,如《灵枢·通天》篇就指出“古之善用针艾者,视人五态乃治之。”《灵枢·逆顺肥瘦》提出,对不同体质的人,针刺的深度、进针速度、留针的时间以及针刺的次数都有不同,如肥人血气充盈,肌肤坚固,可深刺并留针;瘦人皮薄气少,血清气滑,宜浅刺之;婴儿肉脆血少气弱,以毫针浅刺而疾发针。社会地位不同,其体质亦异,针刺时亦应选择不同方法,《灵枢·根结》指出:因人之饮食有“膏粱菽藿之味”之异,社会地位有高下之别,故体质亦有差异,表现在针刺治疗时要注意“气滑则出疾,其气涩则出迟;气悍则针小而入浅,气涩针大而入深,深则欲留,浅则欲疾”;“刺布衣者,深而留之;刺大人者,微而徐之。”《素问·示从容论》则强调年龄、体质不同,治疗部位亦有所宜,“夫年长则求之于腑,年少则求之于经,年壮则求之于脏”。因年长者脾胃运化不及,水谷易停留,多患六腑不通之证,故以通六腑为宜;年少者不耐劳累,过劳则邪气易从经脉而入,故以怯邪通利经脉为宜;壮年者劳伤太过,则易伤精气,故以补五脏为宜。

清代医家徐灵胎在《医学源流论·五方异治论》中说得很明确:体质不同,治法有异,“天下有同此一病,而治此则效,治彼则不效,且不唯无效,而反有大害者,何也?则以病同而人异也。夫七情六淫之感不殊,而交感之人各殊,或身体有强弱,质性有阴阳,生长有南北,性情有刚柔,筋骨有坚脆,肢体有劳逸,年龄有老少,奉养有膏粱藜藿之殊,心境有忧劳和乐之别,更天时有寒暖之不同,受病有深浅之各异,一概施治则病情虽中,而于人之体质迥乎相反,则利害亦相反矣。”

2.体质与治疗反应:体质不同,其对针石治疗或药物治疗的反应性亦不一。《灵枢·论痛》就是专论不同的体质对针石火焫、药物治疗的反应和耐受程度的。其曰:“黑色而美骨者,耐火焫”;“坚肉薄皮者,不耐针石之痛,于火焫亦然”;“胃厚色黑大骨及肥者,皆胜毒;故其瘦而薄胃者,皆不胜毒也。”明确告诉医生,体质强者,其耐痛程度亦高,对药物的耐受力亦大,故可承受刺激强、作用迅速而明显的针石药物治疗;而体质弱者,其痛阈值低,不耐针石之痛,只能用针浅而刺之,不可强刺激;胃气又薄弱,不耐药物之攻伐,故对此类病人,只能缓以图之,不可为求速效而猛药治之。

此外,针刺治疗主要体现于得气与否,而体质不同的人,其得气的迟速亦不相同。《灵枢·行针》曰:“百姓之血气各不同形,或神动而气先针行,或气与针相逢,或针已出气独行,或数刺乃知,或发针而气逆,或数刺病益剧,凡此六者,各不同形。”这就指出了个体体质有异,行针后可有各种不同的反应。“神动而气先针行”,为针刺后得气迅速;“气与针相逢”,说明得气与针刺同步;“针已出气独行”,言虽针已出,但得气之感应依然存在;“数刺乃知”,说明反应迟钝,屡刺方始得气;“发针而气逆”,指出针刺后有可能发生气逆等不良反应;“数刺病益剧”,提示这类体质的病人不适宜针刺,应及时禁用。故医者在治疗时亦须观察病人体质而选择不同方法,不可一律施之。

\subsection{体质与摄生}%四、

摄生讲究的是养神养形,而形神合一是《内经》体质的主要内容。《内经》中的摄生强质思想对当今人们养生仍有指导意义。

1.养神以强壮体质:《素问·上古天真论》中论述养生的诸种方法,其中“恬淡虚无”乃养生一大法则,“恬淡虚无,真气从之,精神内守,病安从来?”思想上清心寡欲,情志和畅,心神内守,肝气条达,肺气宣畅,脾气升清,肾气旺盛,何虑体质不得强健?

养神还需“和于四时”,《素问·四气调神大论》专门论述调神养神与四时相应的方法和意义,通过具体的方法,将精神调摄与四时阴阳盛衰同步,使五脏与四时阴阳消长运动统一起来,体质的保养与自然相和谐。通过四时调神可以达到防病强身、治未病的目的。

2.保精以强壮体质:《内经》认为体质之物质基础是精气,故欲强身健质,应保持肾中精气旺盛。《素问·金匮真言论》说:“夫精者,身之本也。故藏于精者,春不病温”,所以自古至今,有冬令进补的习惯。冬季肾气当令,适当进补,使肾中精气得以及时补充;且冬令又是收藏季节,精气不易妄泄,则对增强体质意义重大。

此外,《内经》认为节欲保精亦是保持正常体质的一个主要内容。如《素问·上古天真论》说:“今时之人不然也,以酒为浆,以妄为常,醉以入房,以欲竭其精,以耗散其真……故半百而衰也”;《灵枢·百病始生》说:“醉以入房,汗出当风,伤脾;用力过度,若入房汗出浴,则伤肾”。《素问·厥论》中更是明确指出了纵欲是损伤体质的重要因素,曰:“前阴者,宗筋之所聚,太阴、阳明之所合也。春夏则阳气多而阴气少,秋冬则阴气盛而阳气衰。此人者质壮,以秋冬夺于所用,下气上争不能复,精气溢下,邪气因从之而上也。气因于中,阳气衰,不能渗营其经络,阳气日损,阴气独在,故手足为之寒也”。由于纵欲太过,使之从“质壮”的体质转变为阳气衰于下的体质,充分说明节欲保精对体质的重要性。对中年人而言,保养精气尤显重要。《素问·阴阳应象大论》说:“年四十,阴气自半也,起居衰矣;年五十,体重,耳目不聪明矣”。中年阴精阳气均已过其半,倘若不知节欲,可直接导致体质虚弱,出现早衰,故本篇又告诫人们:“能知七损八益,则二者可调,不知用此,则早衰之节也。”

此外,《内经》中还有其它一些摄生方法,如“和于术数”、“导引按蹻”、“吐纳精气”、“食饮有节”,“谨和五味”等,均对增强体质有积极意义。

\zuozhe{(周国琪)}
\ifx \allfiles \undefined
\end{document}
\fi