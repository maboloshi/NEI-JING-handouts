% -*- coding: utf-8 -*-
%!TEX program = xelatex
\ifx \allfiles \undefined
\documentclass[12pt]{ctexbook}
%\usepackage{xeCJK}
%\usepackage[14pt]{extsizes} %支持8,9,10,11,12,14,17,20pt

%===================文档页面设置====================
%---------------------印刷版尺寸--------------------
%\usepackage[a4paper,hmargin={2.3cm,1.7cm},vmargin=2.3cm,driver=xetex]{geometry}
%--------------------电子版------------------------
\usepackage[a4paper,margin=2cm,driver=xetex]{geometry}
%\usepackage[paperwidth=9.2cm, paperheight=12.4cm, width=9cm, height=12cm,top=0.2cm,
%            bottom=0.4cm,left=0.2cm,right=0.2cm,foot=0cm, nohead,nofoot,driver=xetex]{geometry}

%===================自定义颜色=====================
\usepackage{xcolor}
	\definecolor{mybackgroundcolor}{cmyk}{0.03,0.03,0.18,0}
	\definecolor{myblue}{rgb}{0,0.2,0.6}

%====================字体设置======================
%--------------------中文字体----------------------
%-----------------------xeCJK下设置中文字体------------------------------%
\setCJKfamilyfont{song}{SimSun}                             %宋体 song
\newcommand{\song}{\CJKfamily{song}}                        % 宋体   (Windows自带simsun.ttf)
\setCJKfamilyfont{xs}{NSimSun}                              %新宋体 xs
\newcommand{\xs}{\CJKfamily{xs}}
\setCJKfamilyfont{fs}{FangSong_GB2312}                      %仿宋2312 fs
\newcommand{\fs}{\CJKfamily{fs}}                            %仿宋体 (Windows自带simfs.ttf)
\setCJKfamilyfont{kai}{KaiTi_GB2312}                        %楷体2312  kai
\newcommand{\kai}{\CJKfamily{kai}}
\setCJKfamilyfont{yh}{Microsoft YaHei}                    %微软雅黑 yh
\newcommand{\yh}{\CJKfamily{yh}}
\setCJKfamilyfont{hei}{SimHei}                                    %黑体  hei
\newcommand{\hei}{\CJKfamily{hei}}                          % 黑体   (Windows自带simhei.ttf)
\setCJKfamilyfont{msunicode}{Arial Unicode MS}            %Arial Unicode MS: msunicode
\newcommand{\msunicode}{\CJKfamily{msunicode}}
\setCJKfamilyfont{li}{LiSu}                                            %隶书  li
\newcommand{\li}{\CJKfamily{li}}
\setCJKfamilyfont{yy}{YouYuan}                             %幼圆  yy
\newcommand{\yy}{\CJKfamily{yy}}
\setCJKfamilyfont{xm}{MingLiU}                                        %细明体  xm
\newcommand{\xm}{\CJKfamily{xm}}
\setCJKfamilyfont{xxm}{PMingLiU}                             %新细明体  xxm
\newcommand{\xxm}{\CJKfamily{xxm}}

\setCJKfamilyfont{hwsong}{STSong}                            %华文宋体  hwsong
\newcommand{\hwsong}{\CJKfamily{hwsong}}
\setCJKfamilyfont{hwzs}{STZhongsong}                        %华文中宋  hwzs
\newcommand{\hwzs}{\CJKfamily{hwzs}}
\setCJKfamilyfont{hwfs}{STFangsong}                            %华文仿宋  hwfs
\newcommand{\hwfs}{\CJKfamily{hwfs}}
\setCJKfamilyfont{hwxh}{STXihei}                                %华文细黑  hwxh
\newcommand{\hwxh}{\CJKfamily{hwxh}}
\setCJKfamilyfont{hwl}{STLiti}                                        %华文隶书  hwl
\newcommand{\hwl}{\CJKfamily{hwl}}
\setCJKfamilyfont{hwxw}{STXinwei}                                %华文新魏  hwxw
\newcommand{\hwxw}{\CJKfamily{hwxw}}
\setCJKfamilyfont{hwk}{STKaiti}                                    %华文楷体  hwk
\newcommand{\hwk}{\CJKfamily{hwk}}
\setCJKfamilyfont{hwxk}{STXingkai}                            %华文行楷  hwxk
\newcommand{\hwxk}{\CJKfamily{hwxk}}
\setCJKfamilyfont{hwcy}{STCaiyun}                                 %华文彩云 hwcy
\newcommand{\hwcy}{\CJKfamily{hwcy}}
\setCJKfamilyfont{hwhp}{STHupo}                                 %华文琥珀   hwhp
\newcommand{\hwhp}{\CJKfamily{hwhp}}

\setCJKfamilyfont{fzsong}{Simsun (Founder Extended)}     %方正宋体超大字符集   fzsong
\newcommand{\fzsong}{\CJKfamily{fzsong}}
\setCJKfamilyfont{fzyao}{FZYaoTi}                                    %方正姚体  fzy
\newcommand{\fzyao}{\CJKfamily{fzyao}}
\setCJKfamilyfont{fzshu}{FZShuTi}                                    %方正舒体 fzshu
\newcommand{\fzshu}{\CJKfamily{fzshu}}

\setCJKfamilyfont{asong}{Adobe Song Std}                        %Adobe 宋体  asong
\newcommand{\asong}{\CJKfamily{asong}}
\setCJKfamilyfont{ahei}{Adobe Heiti Std}                            %Adobe 黑体  ahei
\newcommand{\ahei}{\CJKfamily{ahei}}
\setCJKfamilyfont{akai}{Adobe Kaiti Std}                            %Adobe 楷体  akai
\newcommand{\akai}{\CJKfamily{akai}}

%------------------------------设置字体大小------------------------%
\newcommand{\chuhao}{\fontsize{42pt}{\baselineskip}\selectfont}     %初号
\newcommand{\xiaochuhao}{\fontsize{36pt}{\baselineskip}\selectfont} %小初号
\newcommand{\yihao}{\fontsize{28pt}{\baselineskip}\selectfont}      %一号
\newcommand{\xiaoyihao}{\fontsize{24pt}{\baselineskip}\selectfont}
\newcommand{\erhao}{\fontsize{21pt}{\baselineskip}\selectfont}      %二号
\newcommand{\xiaoerhao}{\fontsize{18pt}{\baselineskip}\selectfont}  %小二号
\newcommand{\sanhao}{\fontsize{15.75pt}{\baselineskip}\selectfont}  %三号
\newcommand{\sihao}{\fontsize{14pt}{\baselineskip}\selectfont}%     四号
\newcommand{\xiaosihao}{\fontsize{12pt}{\baselineskip}\selectfont}  %小四号
\newcommand{\wuhao}{\fontsize{10.5pt}{\baselineskip}\selectfont}    %五号
\newcommand{\xiaowuhao}{\fontsize{9pt}{\baselineskip}\selectfont}   %小五号
\newcommand{\liuhao}{\fontsize{7.875pt}{\baselineskip}\selectfont}  %六号
\newcommand{\qihao}{\fontsize{5.25pt}{\baselineskip}\selectfont}    %七号   %中文字体及字号设置
\xeCJKDeclareSubCJKBlock{SIP}{
	"20000 -> "2A6DF,   % CJK Unified Ideographs Extension B
	"2A700 -> "2B73F,   % CJK Unified Ideographs Extension C
	"2B740 -> "2B81F    % CJK Unified Ideographs Extension D
}
%\setCJKmainfont[SIP={[AutoFakeBold=1.8,Color=red]Sun-ExtB},BoldFont=黑体]{宋体}    % 衬线字体 缺省中文字体

\setCJKmainfont[Path=fonts/,
				SIP={[Path=fonts/,AutoFakeBold=1.8,Color=red]simsunb.ttf},
				BoldFont={simhei.ttf}]{simsun.ttc}
%SimSun-ExtB
%Sun-ExtB
%AutoFakeBold:自动伪粗,即正文使用\bfseries时生僻字使用伪粗体;
%FakeBold:强制伪粗,即正文中生僻字均使用伪粗体
%\setCJKmainfont[BoldFont=STHeiti,ItalicFont=STKaiti]{STSong}
%\setCJKsansfont{微软雅黑}黑体
%\setCJKsansfont[BoldFont=STHeiti]{STXihei} %serif是有衬线字体sans serif 无衬线字体
%\setCJKmonofont{STFangsong}    %中文等宽字体

%--------------------英文字体----------------------
\setmainfont[Path=fonts/,
			 BoldFont={simhei.ttf}]{simsun.ttc}
%\setmainfont[BoldFont=黑体]{宋体}  %缺省英文字体
%\setsansfont
%\setmonofont

%===================目录分栏设置====================
\usepackage[toc,lof,lot]{multitoc}      % 目录(含目录、表格目录、插图目录)分栏设置
	%\renewcommand*{\multicolumntoc}{3} % toc分栏数设置,默认为两栏(\multicolumnlof,\multicolumnlot)
	%\setlength{\columnsep}{1.5cm}      % 调整分栏间距
	\setlength{\columnseprule}{0.2pt}   % 调整分栏竖线的宽度

%==================章节格式设置====================
\setcounter{secnumdepth}{3} % 章节等编号深度 3:子子节\subsubsection
\setcounter{tocdepth}{2}    % 目录显示等度 2:子节

\xeCJKsetup{%
	CJKecglue=\hspace{0.15em},      % 调整中英(含数字)间的字间距
	%CJKmath=true,                  % 在数学环境中直接输出汉字(不需要\text{})
	AllowBreakBetweenPuncts=true,   % 允许标点中间断行,减少文字行溢出
}

\ctexset{%
	part={
		name={,篇},
		number=\SZX{part},
		format={\chuhao\bfseries\centering},
		nameformat={},titleformat={}
	},
	section={
		number={\chinese{section}},
		name={第,节}
	},
	subsection={
		number={\chinese{subsection}、},
		aftername={\hspace{-0.01em}}
	},
	subsubsection={
		number={(\chinese{subsubsection})},
		aftername={\hspace {-0.01em}},
		beforeskip={1.3ex minus .8ex},
		afterskip={1ex minus .6ex},
		indent={\parindent}
	},
	paragraph={
		beforeskip=.1\baselineskip,
		indent={\parindent}
	}
}

\newcommand*\SZX[1]{%
	\ifcase\value{#1}%
		\or 上%
		\or 中%
		\or 下%
	\fi
}

%====================页眉设置======================
\usepackage{titleps}%或者\usepackage{titlesec},titlesec包含titleps
\newpagestyle{special}[\small\sffamily]{
	%\setheadrule{.1pt}
	\headrule
	\sethead[\usepage][][\chaptertitle]
	{\chaptertitle}{}{\usepage}
}

\newpagestyle{main}[\small\sffamily]{
	\headrule
	%\sethead[\usepage][][第\thechapter 章\quad\chaptertitle]
%  {\thesection\quad\sectiontitle}{}{\usepage}}
	\sethead[\usepage][][第\chinese{chapter}章\quad\chaptertitle]
	{第\chinese{section}节\quad\sectiontitle}{}{\usepage}
}

\newpagestyle{main2}[\small\sffamily]{
	\headrule
	\sethead[\usepage][][第\chinese{chapter}章\quad\chaptertitle]
	{第\chinese{section}節\quad\sectiontitle}{}{\usepage}
}

%================ PDF 书签设置=====================
\usepackage[depth=2,        % 书签深度 2:子节
			open,           % 默认展开书签
			openlevel=2,    % 展开书签深度 2:子节
			numbered,       % 显示编号
			atend,
			]{bookmark}     % 相比hyperref,该宏包大多数时候只需要编译一次,
							% 而且书签的颜色和字体也可以定制。
							% 比hyperref 更专业 (自动加载hyperref)

%\bookmarksetup{italic,bold,color=blue} % 书签字体斜体/粗体/颜色设置

%------------重置每篇章计数器,必须在hyperref/bookmark之后------------
\makeatletter
	\@addtoreset{chapter}{part}
\makeatother

%------------hyperref 超链接设置------------------------
\hypersetup{%
	pdfencoding=auto,   % 解决新版ctex,引起hyperref UTF-16预警
	colorlinks=true,    % 注释掉此项则交叉引用为彩色边框true/false
	pdfborder=001,      % 注释掉此项则交叉引用为彩色边框
	citecolor=teal,
	linkcolor=myblue,
	urlcolor=black,
	%psdextra,          % 配合使用bookmark宏包,可以直接在pdf 书签中显示数学公式
}

%------------PDF 属性设置------------------------------
\hypersetup{%
	pdfkeywords={黄帝内经,内经,内经讲义,21世纪课程教材},    % 关键词
	%pdfsubject={latex},        % 主题
	pdfauthor={主编:王洪图}, % 作者
	pdftitle={内经讲义},        % 标题
	%pdfcreator={texlive2011}   % pdf创建器
}

%------------PDF 加密----------------------------------
%仅适用于xelatex引擎 基于xdvipdfmx
%\special{pdf:encrypt ownerpw (abc) userpw (xyz) length 128 perm 2052}

%仅适用于pdflatex引擎
%\usepackage[owner=Donald,user=Knuth,print=false]{pdfcrypt}

%其他可使用第三方工具 如:pdftk
%pdftk inputfile.pdf output outputfile.pdf encrypt_128bit owner_pw yourownerpw user_pw youruserpw

%=============自定义环境、列表及列表设置================
\def\biaoti#1{\vspace{1.7ex plus 3ex minus .2ex}{\bfseries #1}}%\noindent\hei
\def\xiaobt#1{{\bfseries #1}}
\def\xiaojie {\vspace{1.8ex plus .3ex minus .3ex}\centerline{\large\bfseries 小\ \ 结}\vspace{.1\baselineskip}}
\def\zuozhe#1{\rightline{\bfseries #1}}

\newcounter{yuanwen}    % 新计数器 yuanwen
\newcounter{jiaozhu}    % 新计数器 jiaozhu

\newenvironment{yuanwen}[2][【原文】]{%
	%\biaoti{#1}\par
	\stepcounter{yuanwen}   % 计数器 yuanwen+1
	\bfseries #2}
	{}

\usepackage{enumitem}
\newenvironment{jiaozhu}[1][【校注】]{%
	%\biaoti{#1}\par
	\stepcounter{jiaozhu}   % 计数器 jiaozhu+1
	\begin{enumerate}[%
					label=\mylabel{\arabic*}{\circledctr*},before=\small,fullwidth,%
					itemindent=\parindent,listparindent=\parindent,%labelsep=-1pt,%labelwidth=0em,
					itemsep=0pt,topsep=0pt,partopsep=0pt,parsep=0pt]}
	{\end{enumerate}}

%===================注解与原文相互跳转====================
%----------------第1部分 设置相互跳转锚点-----------------
\makeatletter
	\protected\def\mylabel#1#2{% 注解-->原文
		\hyperlink{back:\theyuanwen:#1}{\Hy@raisedlink{\hypertarget{\thejiaozhu:#1}{}}#2}}

	\protected\def\myref#1#2{% 原文-->注解
		\hyperlink{\theyuanwen:#1}{\Hy@raisedlink{\hypertarget{back:\theyuanwen:#1}{}}#2}}
	%此处\theyuanwen:#1实际指thejiaozhu:#1,只是\thejiaozhu计数器还没更新,故使用\theyuanwen计数器代替
\makeatother

\protected\def\myjzref#1{% 脚注中的引用(引用到原文)
	\hyperlink{\theyuanwen:#1}{\circlednum{#1}}}

\def\sb#1{\myref{#1}{\textsuperscript{\circlednum{#1}}}}    % 带圈数字上标

%----------------第2部分 调整锚点垂直距离-----------------
\def\HyperRaiseLinkDefault{.8\baselineskip} %调整锚点垂直距离
%\let\oldhypertarget\hypertarget
%\makeatletter
%   \def\hypertarget#1#2{\Hy@raisedlink{\oldhypertarget{#1}{#2}}}
%\makeatother

%====================带圈数字列表标头====================
%\newfontfamily\circledfont[Path = fonts/]{meiryo.ttc}  % 日文字体,明瞭体
\newfontfamily\circledfont{Meiryo}  % 日文字体,明瞭体

\protected\def\circlednum#1{{\makexeCJKinactive\circledfont\textcircled{#1}}}

\newcommand*\circledctr[1]{%
	\expandafter\circlednum\expandafter{\number\value{#1}}}
\AddEnumerateCounter*\circledctr\circlednum{1}

% 参考自:http://bbs.ctex.org/forum.php?mod=redirect&goto=findpost&ptid=78709&pid=460496&fromuid=40353

%======================插图/tikz图========================
\usepackage{graphicx,subcaption,wrapfig}    % 图,subcaption含子图功能代替subfig,图文混排
	\graphicspath{{img/}}                   % 设置图片文件路径

\def\pgfsysdriver{pgfsys-xetex.def}         % 设置tikz的驱动引擎
\usepackage{tikz}
	\usetikzlibrary{calc,decorations.text,arrows,positioning}

%---------设置tikz图片默认格式(字号、行间距、单元格高度)-------
\let\oldtikzpicture\tikzpicture
\renewcommand{\tikzpicture}{%
	\small
	\renewcommand{\baselinestretch}{0.2}
	\linespread{0.2}
	\oldtikzpicture
}

%=========================表格相关===============================
\usepackage{%
	multirow,                   % 单元格纵向合并
	array,makecell,longtable,   % 表格功能加强,tabu的依赖
	tabu-last-fix,              % "强大的表格工具" 本地修复版
	diagbox,                    % 表头斜线
	threeparttable,             % 表格内脚注(需打补丁支持tabu,longtabu)
}

%----------给threeparttable打补丁用于tabu,longtabu--------------
%解决方案来自:http://bbs.ctex.org/forum.php?mod=redirect&goto=findpost&ptid=80318&pid=467217&fromuid=40353
\usepackage{xpatch}

\makeatletter
	\chardef\TPT@@@asteriskcatcode=\catcode`*
	\catcode`*=11
	\xpatchcmd{\threeparttable}
		{\TPT@hookin{tabular}}
		{\TPT@hookin{tabular}\TPT@hookin{tabu}}
		{}{}
	\catcode`*=\TPT@@@asteriskcatcode
\makeatother

%------------设置表格默认格式(字号、行间距、单元格高度)------------
\let\oldtabular\tabular
\renewcommand{\tabular}{%
	\renewcommand\baselinestretch{0.9}\small    % 设置行间距和字号
	\renewcommand\arraystretch{1.5}             % 调整单元格高度
	%\renewcommand\multirowsetup{\centering}
	\oldtabular
}
%设置行间距,且必须放在字号设置前 否则无效
%或者使用\fontsize{<size>}{<baseline>}\selectfont 同时设置字号和行间距

\let\oldtabu\tabu
\renewcommand{\tabu}{%
	\renewcommand\baselinestretch{0.9}\small    % 设置行间距和字号
	\renewcommand\arraystretch{1.8}             % 调整单元格高度
	%\renewcommand\multirowsetup{\centering}
	\oldtabu
}

%------------模仿booktabs宏包的三线宽度设置---------------
\def\toprule   {\Xhline{.08em}}
\def\midrule   {\Xhline{.05em}}
\def\bottomrule{\Xhline{.08em}}
%-------------------------------------
%\setlength{\arrayrulewidth}{2pt} 设定表格中所有边框的线宽为同样的值
%\Xhline{} \Xcline{}分别设定表格中水平线的宽度 makecell包提供

%表格中垂直线的宽度可以通过在表格导言区(preamble),利用命令 !{\vrule width1.2pt} 替换 | 即可

%=================图表设置===============================
%---------------图表标号设置-----------------------------
\renewcommand\thefigure{\arabic{section}-\arabic{figure}}
\renewcommand\thetable {\arabic{section}-\arabic{table}}

\usepackage{caption}
	\captionsetup{font=small,}
	\captionsetup[table] {labelfont=bf,textfont=bf,belowskip=3pt,aboveskip=0pt} %仅表格 top
	\captionsetup[figure]{belowskip=0pt,aboveskip=3pt}  %仅图片 below

%\setlength{\abovecaptionskip}{3pt}
%\setlength{\belowcaptionskip}{3pt} %图、表题目上下的间距
\setlength{\intextsep}   {5pt}  %浮动体和正文间的距离
\setlength{\textfloatsep}{5pt}

%====================全文水印==========================
%解决方案来自:
%http://bbs.ctex.org/forum.php?mod=redirect&goto=findpost&ptid=79190&pid=462496&fromuid=40353
%https://zhuanlan.zhihu.com/p/19734756?columnSlug=LaTeX
\usepackage{eso-pic}

%eso-pic中\AtPageCenter有点水平偏右
\renewcommand\AtPageCenter[1]{\parbox[b][\paperheight]{\paperwidth}{\vfill\centering#1\vfill}}

\newcommand{\watermark}[3]{%
	\AddToShipoutPictureBG{%
		\AtPageCenter{%
			\tikz\node[%
				overlay,
				text=red!50,
				%font=\sffamily\bfseries,
				rotate=#1,
				scale=#2]
				{#3};
		}
	}
}

\newcommand{\watermarkoff}{\ClearShipoutPictureBG}

\watermark{45}{15}{草\ 稿}    %启用全文水印

%=============花括号分支结构图=========================
\usepackage{schemata}

\xpatchcmd{\schema}
	{1.44265ex}{-1ex}
	{}{}

\newcommand\SC[2] {\schema{\schemabox{#1}}{\schemabox{#2}}}
\newcommand\SCh[4]{\Schema{#1}{#2}{\schemabox{#3}}{\schemabox{#4}}}

%=======================================================

\begin{document}
\pagestyle{main}
\fi
\chapter{《黄帝内经》的社会医学思想}%第七章

社会医学就是从社会角度研究医学问题,即研究社会因素同人类群体或个体健康、疾病发生发展及其防治的关系。社会医学(social medicine)一词,虽由西方学者在十九世纪首先提出,但有关社会医学的一些内容早在中国古代的医学著作中即有所体现。特别是《黄帝内经》,以其成书之早,所载社会医学内容之丰,而独树一帜,值得深入发掘和研究。

\section{《内经》社会医学的基本内容}%第一节

\subsection{《内经》中的社会分期及医学源流说}%一、

《内经》中许多篇章描述了古代社会形态,回顾了医学起源和发展之初的情况,将古代大体分为上古(远古)、中古和暮世三个历史阶段。为知晓这种分期,兹引述三段经文:

《素问·移精变气论》:“往古人居禽兽之间,动作以避寒,阴居以避暑,内无眷慕之累,外无伸宦之形,此恬憺之世,邪不能深入也。故毒药不能治其内,针石不能治其外,故可移精祝由而已。当今之世不然,忧患缘其内,苦形伤其外,又失四时之从,逆寒暑之宜,贼风数至,虚邪朝夕,内至五脏骨髓,外伤空窍肌肤,所以小病必甚,大病必死,故祝由不能已也。”

《素问·汤液醪醴论》:“自古圣人之作汤液醪醴者,以为备耳。夫上古作汤液,故为而弗服也。中古之世,道德稍衰,邪气时至,服之万全。帝曰:今之世不必已何也?岐伯曰:当今之世,必齐毒药攻其中,镵石针艾治其外也……嗜欲无穷,而忧患不止,精气弛坏,荣泣卫除,故神去之而病不愈也。”

《素问·上古天真论》:“上古之入,其知道者,法于阴阳,和于术数,食饮有节,起居有常,不妄作劳,故能形与神俱,而尽终其天年,度百岁乃去。”

从三篇经文可知,远古人穴居野处,为了生存和健康,人们“动作以避寒,阴居以避暑”。氏族部落过着群居生活,劳动果实很少,大家分享后并无剩余,私有观念尚未萌生,堪称“恬憺之世”。当时危害人们健康的主要是外邪和外伤,情志伤脏的内在因素很少,故邪气难于深入,疾病轻浅单纯。另一方面,由于上古时期生产力极不发达,人们屈服于自然力,很难掌握自己的命运,遂令巫术盛行。巫术深入到社会生活的各个方面,“可移精祝由而己”,便是对先人治病采用巫术的记载。马王堆汉墓出土帛书《五十二病方》中有大量“祝词”,至后汉张机《伤寒杂病论》序中谓“卒然遭邪风之气,婴非常之疾,患及祸至,而方震慄,降志屈节,钦望巫祝,告穷归天,束手就败”,说明“祝由”之术在中国历史上延续了很长时间。值得称道的是,《内经》作者很早就看到了医药科学必然战胜巫术的历史趋势,断然称“当今之世”,疾病复杂深重,“祝由不能已也”,祝由疗法应予淘汰。其次,“上古”酿酒术可能还未发明,待其发明时,或已接近“中古”。酒从饮料而渐渐介入医疗,在很长一段时间内,酒剂应用并不广泛,即所谓“为而弗服也”,“以为备耳”。再次,上古人的寿命,据《内经》作者的描绘是很长很长的。认为懂得养生之道者,可“度百岁乃去”;而那些得道的“真人”、“至人”等,则会有不可思议的超常寿命,这恐是作者理想的一种寄托。

“中古”之世,人们的私有观念渐浓,原始公社式的“天下为公”意识已趋淡薄,道德渐衰,疾病种类有所增多,但似不那么深重,醪醴等酒剂正可大显身手,即所谓“服之万全”。此期治病,还采用砭石及药物,祝由疗法仍在通行。

我们据《内经》的描述推测,“上古”似指炎黄前后的很长一段历史时期,私有制尚未产生。“中古”当指禹之后的夏、商及西周时期。由于作者的崇古思想,认为“中古”还不是一个很糟糕的社会,王冰也附会说:“虽道德稍衰,邪气时至,以心犹近道,故服用万全。”实则“中古”已是黑暗的奴隶制社会,残暴的统治者“心犹近道”是不可能的。如果说《内经》所描绘的“上古”、“中古”的社会面貌,皆出自传说或推测,那么对“暮世”的记述则近于写实。“暮世”,即作者所处的时代,故又称为“当今之世”,相当于春秋战国及秦汉时期。《内经》所反映的社会状态及科学文化包括医疗的成就,大多是对“暮世”阶段的写照。

“暮世”是一个多战乱的社会。春秋战国时期华夏大地基本处于分裂状态,诸侯争霸,弱肉强食,战争不断。秦汉是统一的国家,但也经历着王朝的更迭及削平异姓王、同姓王之战。于是《内经》中便有“两军相当,旗帜相望,白刃陈于中野者,此非一日之谋也。能使其民,令行禁止,士卒无白刃之难者,非一日之教也,须臾之得也”之类描述战争场面的句子。甚至言针具亦以兵器相喻:“夫大于针者,惟五兵者焉。五兵者,死之备也,非生之具。且夫人者,天地之镇也,其不可不参乎!夫治民者,亦唯针焉。”(《灵枢·玉版》)战争除使士卒遭白刃之难,战伤累累外,还会改变部分统治者的地位。所谓“封君败伤”、“始富后贫”,说的就是一些王侯失势,顷刻间权势化为乌有,沦为奴隶或平民的情形。

“暮世”是一个灾害频发的社会。除战乱外,灾害饥馑也不时地威胁着国家和人民。《灵枢·岁露》说:“正月朔日,风从南方来,命曰旱乡,从西方来,命曰白骨,将国有殃,人多死亡。正月朔日,风从东方来,发屋,扬抄石,国有大灾也。正月朔日,风从东南方行,春有死亡。正月朔日,天和温不风,籴贱,民不病;天寒而风,籴贵,民多病。此所谓候岁之风,残伤人者也。”自然灾寒影响国民的程度,除自然灾害本身的破坏力外,社会抵御灾害的能力也起着重要作用,灾荒加上粮价暴涨,必然会使疾病更易发生和流行。

“暮世”是一个道德沦丧的社会。在劳苦大众只能以菽藿充饥的同时,一些王公大人却过着“以酒为浆,以妄为常,醉以入房,以欲竭其精,以耗散其真,不知持满,不时御神,务快其心,逆于生乐,起居无节”(《素问·上古天真论》)的生活,他们的寿命缩短,往往“半百而衰”,中寿而亡。疾病种类的增多和病情的严重,促进了医学的发展。春秋战国是中国古代医学快速发展的时期,证实了时代要求是医学发展动力的真理。

\subsection{《内经》有关社会状况与人群疾病的记载}%二、

不同的社会群体,在患病种类上可有不同。

\subsubsection{贫困人群}%(一)

《灵枢·九针十二原》说:“余子万民,养百姓,而收其租税。余哀其不给,而属有疾病。”这里“百姓”与“万民”对举,百姓仍有百官之意,万民即社会底层的劳苦大众。在土地上耕作的农民,由于要缴纳繁重的租税,饱受压迫和剥削,虽终日劳作而不得温饱,普遍营养不良,难以抵御病邪,而接连发生疾病。由于居住和饮食条件较差,劳动者所患多为外感病,特别是受灾年月,瘟疫容易在这一群体中发生和流行。

\subsubsection{富贵人群}%(二)

《内经》中多次提到的君王、公侯、群臣,则是属于压榨劳苦大众的另一群体。他们高高在上,养尊处忧,骄奢淫泆,易患中风、消渴、痈疽、虚损之类的疾病。即如《吕氏春秋·孟春纪第一·本生》所说:“贵富而不知道,适足以为患……出则以车,入则以辇,务以自佚,命之曰招蹶之机;肥肉厚酒,务以自强,命曰烂肠之食;靡曼皓齿,郑卫之音,务以自乐,命曰伐性之斧。三患者贵富之所致也。”

自私有制出现以来,人们吿别了往古的“恬憺之世”失去了质朴无邪的天性。物欲横流,权位纷争,社会动荡,国无宁日,人们的物质生活和精神生活受到了严重影响。那些失侯失王者,削官失势者,仕途不顺者,名利不遂者,穷愁潦倒者,其精神上的创伤不亚于财产的损失。《素问·疏五过论》说:“尝贵后贱,虽不中邪,病从内生,名曰脱营。尝富后贫,名曰失精”,“暴乐暴苦,始乐后苦,皆伤精气”,“封君败伤,及欲侯王,故贵脱势,虽不中邪,精神内伤,身必败亡。始富后贫,虽不伤邪,皮焦筋屈,痿躄为挛”。此“脱营”、“失精”、“痿躄”一类发于内的疾病,实为社会心理因素所致的心身疾病。这一类疾病的危害性在于早期不易察觉,“不在脏腑,不变躯形,诊之而疑,不知病名”,而其预后又十分严重,故引起古代医家的高度重视。

《内经》认为,不良的社会风气会给一定的社会群体带来严重的疾病《素问·腹中论》说:“热中消中,不可服芳草石药。石药发癫,芳草发狂。”这是针对古代上流社会的一种服食之风而说的。服食,亦称服石,即服用玉石一类矿物药以及一些有香气的草药,寒食散(即五石散,由钟乳石、紫石英、白石英、硫黄、赤石脂五种石药组成)为其代表方剂。服食与神仙之术一样,为历代统治者所崇。先秦即已兴起的求仙和服食之风,发端于帝王,风靡于士大夫,流毒于民间。《史记》载秦始皇派徐市、卢生等人赴海上求仙及长生不死之药,汉武帝所宠信的方士李少君、栾大等为之“化丹沙诸药剂为黄金”。服石的结果,往往事与愿违,长寿自是无望,反增诸多疾病。《史记》仓公“诊籍”之二十二已记载了服五石发痈疽而死的案例,谓“中热不溲者,不可服五石。石之为药精悍”,“刚药入则动阳,阴病益衰,阳病益著,邪气流行,为重困于俞,忿为发疽”。《内经》也认为“芳草之气美,石药之气悍”。但是由于劝阻不住那些服石者,而只能发出无奈之叹:“夫热中消中者,皆富贵人也。今禁膏粱,是不合其心;禁芳草石药,是病不愈。”(《素问·腹中论》)《内经》所反对的服石之风,一直延续至隋唐以后。《晋书》记载晋哀帝司马丕、后魏开国皇帝道武帝拓跋珪等皆死于寒食散。魏晋时期医学家皇甫谧,“久婴笃疾”,又服散失度,致“隆冬裸袒食冰,当暑烦闷,加以咳逆”,险些丧命。他致力医学,除撰《针灸甲乙经》外,另著《论寒食散方》二卷。其后,各种“解散”方书纷纷问世,至《隋书·经籍志》收录解散方达二十家之多,大型方书如隋·巢元方《诸病源候论》、唐·王焘的《外台秘要》和日本人丹波康赖的《医心方》也有“解散”专卷。可见,古时的服食和现代吸食鸦片等毒品一样,已成了严重的社会问题。

\subsubsection{五方居民}%(三)

中华古国,幅员广阔,民族众多,四方水土有别,民风民俗各异,患病虽现地区性差异。《素问·异法方宜论》指出,东方临海,居民食鱼嗜咸,“其病皆为痈疡”;西方为沙漠,“水土刚强”,其民食肉类而肥胖,“其病生于内”;北方为高寒地带,“其民乐野处而乳食,脏寒生满病”;南方气候炎热潮湿,“水土弱”,“其民嗜酸而食胕(腐)”,“其病挛痹”;中央为肥沃平原,物产丰富,“其民食杂而不劳”;“其病多痿厥寒热”。此中简要总结了由于地理环境及风俗习惯的不同,而易发生具有地方特点的疾病。在诊治疾病时应予注意。

\subsection{《内经》关于医患关系的讨论}%三、

在医患关系上,《内经》素有以人为本,以病人为本的思想。这种思想在《素问·汤液醪醴论》中表述为:“病为本,工为标。”篇中通过具体病例对此进行了论证:“帝曰:夫病之始生也,极微极精,必先人结于皮肤。今良工皆称曰病成,名曰逆,则针石不能治,良药不能及也。今良工皆得其法,守其数,亲戚兄弟远近,音声日闻于耳,五色日见于目,而病不愈者,亦何暇不早乎?岐伯曰:病为本,工为标,标本不得,邪气不服,此之谓也。”这是一旦发作即已成为重症逆症的病例,它缺乏起病时病情单纯轻浅的一般过程,可以判定它不是外感病,而是个积渐日久而又一朝暴发的内伤病。病人有亲戚朋友相拥,又有延请众多高明医生的财力,表明病家系望族大户,患者为富贵之人。而对这样的病例,众多良医束手无策,遂令人感到病人是根本,医生只居从属地位,再高明的医术在这样病人面前也没有用武之地。那么,病人何以能居“本”位呢?王冰的解释是:“言医与病不相得也……针艾之妙靡容,药石之功匪预,如是则道虽昭著,万举万全,病不许治,欲奚为疗!《五脏别论》曰:‘拘于鬼神者,不可与言至德;恶于针石者,不可与言至巧;病不许治者,病必不治,治之无功’。此皆谓工病不相得,邪气不宾服也。”王氏着眼于医患关系,认为病人失于配合是邪不除、病不愈的主要原因,这是切题之语,但尚不能包括“病为本”的全部内涵。“病为本”,理应包括病人的多方面情况,诸如病人所处的自然、社会及家庭环境,病人的经济状况、职业、经历、嗜好、体质、人格特点以及对健康与疾病的态度等。医工之标,必须随这个病本而转移。否则,仅仅看到疾病的一般表现,不深察病人所固有的特定情况,必然会发生误诊误治或贻误诊疗时机的严重后果。《素问·移精变气论》所说的“逆从倒行,标本不得,亡神失国”,就是指的这种结局。为防止出现人亡神、国不保的严重后果,篇中接着说“去故就新,乃得真人!”王冰注云:“当去故逆理之人,就新明悟之士,乃得至真精晓之人以全已也。”“本”为治疗对象,不能变更,而医生术穷,则须另就高明。以病人为中心,是《内经》诊疗疾病的一个基本出发点。

\section{《内经》社会医学思想的价值}%第二节

\subsection{丰富了中医病因学理论}%一、

《内经》从自然和社会两大方面来探究疾病的发生,不仅认为“人与天地相参也,与日月相应也”(《灵枢·岁露》),更主张“道上知天文,下知地理,中知人事,可以长久,以教众庶,亦不疑殆。医道论篇,可传后世,可以为宝”《素问·著至教论》。《灵枢·逆顺肥瘦》说:“圣人之为道者,上合于天,下合于地,中合于人事,必有明法,以起度数,法式检押,乃后可传焉。”这就是说,《内经》将知天地人视为理论基石,置于法规的高度,并作为医事教育的准绳。天和地,为自然界;人事,指人类社会状态及社会成员的行为,包括社会的政治、经济、文化教育、道德、民俗及人的个体差异(性别、年龄、体质、人格、心理等)。可见,《内经》理论体系源于对人与自然、人与社会关系的认识。

人具有自然(生物)属性,也具有社会属性。谓上古、中古、暮世之人即不同时代的人;谓“君王众庶”、“卑贱富贵”、“愚智贤不肖”、“毚愚仆漏”即社会各色人等。人的健康在受多种自然界因素影响的同时,也受到社会诸多因素的影响,现将《内经》,有关疾病发生的原因,系统归纳如表,而后据经中的有关内容作一简论。

{
	\small\centerline{《内经》病因系统表}\smallskip
	\label{fig:《内经》病因系统表}
	\renewcommand{\baselinestretch}{1}
	\hspace{.1\textwidth}%左侧空白
	\SCh{-7.5ex}{7.8ex}{致病因素}{\smallskip
		\SC{自然环境}{
			天气:风、雨、寒、暑、湿、燥、雾、露等\\
			天象:天宿失度,日月薄蚀,月郭圆缺\\
			地理:地域,地势
		}\\
		\SC{社会环境}{
			政治:政体,战争,职位\\
			经济:经济制度,贫富差异\\
			道徳:嗜欲无穷(物欲、权欲、色欲)\\
			心理:情志异常\\
			信仰:崇拜,迷信\\
			民俗:地区风气,生活习俗\\
			饮食:嗜食,偏食,饱食,冷食\\
			劳伤:体劳,房劳\\
			中毒:毒物(包括药物),醉酒}
	}
	\hfill
}

\subsubsection{气候异常}%(一)

《内经》多篇有“百病之始生”的话题。《灵枢·百病始生》:“夫百病之始生也,皆生于风雨寒暑,清湿喜怒。”《灵枢·口问》:“夫百病之始生也,皆生于风雨寒暑,阴阳喜怒,饮食居处,大惊卒恐。”《灵枢·顺气一日分为四时》:“夫百病之所始生者,必起于燥湿寒暑风雨,阴阳喜怒,饮食居处。”其中的风、雨、寒、暑、湿、燥,属于气候因素。与此相类,《素问·生气通天论》又有“无见雾露”。《素问·阴阳应象大论》依季节概括为寒、暑、燥、湿、风五气。运气诸篇则依三阴三阳律为风、寒、暑、湿、燥、火六气,后世多宗之,而谓之“六淫”。六淫实包含着物理的、化学的和生物性致病因素。正常的天气并不伤人,伤人的是非时之气或当令之气太过。就风而言,前者称为“实风”。后者称为“虚风”。“虚风”可使人生重病,“实风”中人只发轻浅小疾。

\subsubsection{天象异常}%(二)

天象包括天体中日月星辰的运动及发光等。《灵枢·痈疽》说:“天宿失度,日月薄蚀,地失经纪,水道流溢,草萓不成,五谷不殖,径路不通,民不往来,巷聚邑居,则别离异处,血气犹然。”《灵枢·岁露》则说:“月满则海水西盛,人血气积,肌肉充,皮肤致,毛发坚,腠理郄,烟垢著。当是之时,虽遇贼风,其人浅不深。至其月郭空,则海水东盛,人血气虚,其卫气去,形独居,肌肉减,皮肤纵,腠理开,毛发残,膲理薄,烟垢落。当是之时,遇贼风则其入深,其病人也卒暴”。天宿失度,是古人所说的日月星辰有不正常运行,如彗星、流星等。日月薄蚀,即日食月食。日月星辰运行中发生一些奇异现象,会对疾病的发生有一定的影响。这是古人观察事物的入微处,也反映了中国古代天文学的成就。

\subsubsection{地理因素}%(三)

地理包括地域和地势。东西南北中五方,地域不同,水土有别,会有不同的地区多发病,已见于前文引述的《素问·异法方宜论》。《吕氏春秋·季春纪·尽数》也有“轻水所多秃与瘿人,重水所多尰与躄人,甘水所多好与美人,辛水所多疽与痤人,苦水所多尫与伛人。”说明早在战国年间就已看到了各地水质对人健康的不同影响,丨已萌生“地方病”之概念。关于地势高低及地形变化对人的影响,经中亦不乏记载,如《素问·五常政大论》:“地有高下,气有温凉,高者气寒,下者气热,故适寒凉者胀,之温热者疮,……此腠理开闭之常,太少之异也”,“崇高则阴气治之,污下则阳气治之。阳胜者先天,阴胜者后天,此地理之常,生化之道也。……高者其气寿,下者其气夭,地之小大异也”。篇中所言地势与疾病及寿命的关系,基本是符合实际的。

\subsubsection{政治、经济}%(四)

包括国家政体、战争状态及人的阶级地位、经济状况等。中国古代的国家政体都是王权统治的奴隶制或封建制,就战国至秦汉而言主要为封建制国家,但有中央集权与诸侯割剧之别。国家政权的状态关系到社会的稳定与否,动荡的社会往往造成疾病的高发。频仍的战事带来的社会问题更多,军士的伤亡,家园的毁坏,以及领土和财富再分配的战争结局,更令失势者精神遭到重创,引发严重的心身疾病。至于阶级地位,经中多次提到的君王公侯、各级官吏,属于统治阶级,即所谓“大人”;万民、众庶、布衣、黔首,则是被统治阶级,即所谓“小人”。前者富贵,后者贫贱。“膏粱菽藿之味,何可同也”(《灵枢·根结》)。厚味美食,是富贵者的食物;菽藿,豆子和豆叶,糠菜半年粮,是贫贱者的生活。两类人有着不同的体质状态和疾病谱,即使患同种疾病,其临床表现也会有许多不同。

\subsubsection{道德}%(五)

《素问·上古天真论》所规定的道德标准是:“嗜欲不能劳其目,淫邪不能惑其心”、“志闲而少欲,心安而不惧”,“高下不相慕,其民故曰朴”,“适嗜欲于世俗之间,无恚嗔之心,行不欲离子世,举不欲观于俗,外不劳形于事,内无思想之患,以恬愉为务,以自得为功”,如此则“德全不危也”。道德堕落者,“嗜欲无穷,而忧患不止”,私欲膨胀,患得患失,唯名利是务,纵欲贪色,皆伤精伤神,致“精气弛坏,荣泣卫除”,“神去之而病不愈也”。道德衰落,每易罹患疾病,且病情严重复杂,不易治愈。

\subsubsection{心理}%(六)

导致疾病的异常心理因素多与社会有关,《素问·疏五过论》等篇所说的损伤精气之“暴乐暴苦”、“始乐后苦”,伤阴之“暴怒”,伤阳之“暴喜”,令“五脏空虚,血气离守”之“离绝菀结,忧恐喜怒”,以及“忧患不止”,“眷慕之累”,“伸宦之形”等,皆源自国破家亡、权位跌落、仕途坎坷、穷困潦倒、亲人离散等社会事件或个人平生遭际。诸多社会心理因素引起的情志变化可概括为喜、怒,忧、思、悲、恐、惊等。情志失和则伤害藏神的五脏导致精神与躯体病症,如《灵枢·本神》说:“心怵惕思虑则伤神,神伤则恐惧自失,破䐃脱肉,毛悴色夭,死于冬;脾愁忧而不解则伤意,意伤则悗乱,四肢不举,毛悴色夭,死于春;肝悲哀动中则伤魂,魂伤则狂忘不精,不精则不正,当人阴缩而挛筋,两胁骨不举,毛悴色夭,死于秋;肺喜乐无极则伤魄,魄伤则狂,狂者意不存人,皮革焦,毛悴色夭,死于夏;肾盛怒而不止则伤志,志伤则喜忘其前言,腰脊不可以俯仰屈伸,毛悴色夭,死于季夏。”《素问·血气形志篇》也说:“形乐志苦,病生于脉”,“形乐志乐,病生于肉”,“形苦志乐,病生子筋”,“形苦志苦,病生于咽嗌”,“形数惊恐,经络不通,病生于不仁”。七情伤人致病之广、之深、之重,于此可见一斑。

\subsubsection{信仰}%(七)

《内经》虽未明确记载某种宗教信仰,但提到了“拘于鬼神者”,还有“真人”、“至人”、“道者”等称谓。其实,上古时代是一个充满着原始宗教氛围的社会,部族图腾,自然崇拜和祖先(灵魂)崇拜等,使先民形成了根深蒂固的迷信思想和鬼神观念。随着时代的进步,生产的发展,迷信崇拜的市场已逐渐缩小,但未能大范围消除。占卜的普遍应用,方术之士的到处游说,特别是帝王兴师动众的求仙、封禅、祭天、祭祖活动,都为巫术迷信保留了市场,有时还会兴风作浪。迷信阻碍了人们对客观世界的认识和改造,在医学领域里,它阻碍着对疾病的正确认识和治疗。《灵枢·贼风》真实地记载了这种认识疾病的误区:“其毋所遇邪气,又毋怵惕之所志,卒然而病者,其故何也?唯有因鬼神之事乎?”古代医家痛感于此,遂奋起与之抗争。《史记》载战国名医扁鹊(秦越人),把“信巫不信医”作为“六不治”之一。《内经》也庄严声明:“拘于鬼神者,不可与言至德。”(《素问·五脏别论》)说明许多医家已清醒地看到,病人迷信鬼神,不仅会影响治疗,也会因其“疑神疑鬼”而增加一些疾病本不应有的症状,使病情更为复杂。

\subsubsection{民俗}%(八)

即民间的风俗,历代相沿积久而成。《汉书·地理志下》:“凡民函五常之性,而其刚柔缓急,音声不同,系水土之风气,故谓之风;好恶取舍动静亡常,随君上之情欲,故谓之俗。”谓由自然条件不同而形成的习尚叫“风”,由社会环境不同而形成的习尚叫“俗”。《内经》所载五方地域之水土及生活习俗有异而引发不同疾病,已如前述;“随君上之情欲”而兴起服石之风,其后果为明显的药物中毒,又见于中毒项中。

\subsubsection{饮食}%(九)

饮食为病,可因嗜食、偏食、饱食和冷食等所致。如《素问·生气通天论》说:“味过于酸,肝气以津,脾气乃绝。味过于咸,大骨气劳,短肌,心气抑”等,系偏用五味之患;“膏粱厚味,足生大丁”是嗜食之患;“因而饱食,筋脉横解,肠澼为痔,因而大饮,则气逆”,为过食之患。五味之偏又见于《素问·五脏生成论》和《灵枢·五味论》等篇,引起的疾病遍及内脏和筋脉骨肉。故《灵枢·九针论》告诫说:“口嗜而欲食之,不可多也,必自裁也。”暴饮暴食还有更严重的后果:“卒然多食饮则肠满,起居不节,用力过度,则络脉伤……肠胃之络伤,则血溢于肠外,肠外有寒汁沫与血相搏,则并合凝聚不得散,而积成矣。”(《灵枢·百病始生》)即将暴饮暴食作为腹中肿瘤的起因。《灵枢·师传》指出:“食饮者,热无灼灼,寒无沧沧,寒温中适,故气将持,乃不致邪僻也。”恣食生冷除引起腹痛、泄泻之病外,还会引发其他疾患。如《素问·咳论》即有因冷食而发咳的记栽:“其寒饮食入胃,从肺脉上至于肺则肺寒;肺寒则外内合邪,因而客之,则为肺咳。”另外,《内经》还认为病后进食不但会引起某些病的迁延不愈甚至复发。如热病过早吃肉类食品会导致“食复”,例见《素问·热论》;饮食不加节制会导致鼓胀病复发,例见《素问·腹中论》。

\subsubsection{劳伤}%(十)

劳伤可分体劳和房劳两类。《素问·宣明五气》有“久视伤血,久卧伤气,久坐伤肉,久立伤骨,久行伤筋”的五劳之说,是为体劳。经中有关房劳伤精伤脏的记载甚多,如“若醉入房,汗出当风,则伤脾。有所用力举重,若入房过度,汗出浴水,则伤肾”(《灵枢·邪气脏腑病形》),“筋痿者,生于肝,使内也”(《素问·痿论》)等。虽房劳可伤多脏,但按多篇所述,仍以伤肾为主。古养生家和古医家皆视精液为阴精的重要组成部分,故特别重视节欲保精,在临证上往往将肾虚诸症与房事过度联系起来。

\subsubsection{中毒}%(十一)

应包括食物中毒和药物中毒等。食物中毒是容易发生的,惜《内经》未予明确记载,然多次提到的霍乱病则不能完全除外食物中毒。药物中毒之典型案例即前所提到的“石药发癫,芳草发狂”,实为服食不当中毒所致。故《素问·征四失论》有“或伤于毒”之问。另外,醉酒的某些后果亦可视为中毒。《灵枢·论勇》记载了行为反常的“酒悖”:“酒者,水谷之精,熟谷之液也,其气慄悍,其入于胄中则胃胀,气上逆,满于胸中,肝浮胆横。当是之时,固比于勇士,气衰则悔。与勇士同类,不知避之,名曰酒悖也。”《素问·病能论》还记载了“酒风”之病:“有病身热解堕,汗出如浴,恶风少气……病名曰酒风。”嗜酒过度,耗气伤阴,外受风邪,便会有如此表现。

\subsection{完善了中医学防治理论}%二、

防治疾病,《内经》也同样从天地人宏观出发,既重视驱除自然界的致病因素,也重视纠正社会性致病因素造成的后果。《素问·疏五过论》说:“圣人之治病也,必知天地阴阳,四时经纪,五脏六腑,雌雄表里,刺灸砭石,毒药所主,从容人事,以明经道,贵贱贫富,各异品理,问年少长,勇怯之理,审于分部,知病本始,八正九候,诊必副矣。”从容,指举动行为。这里是说,治病的最高法度须将“天地阴阳”和“从容人事”并举,二者相合方能涵盖和指导整个防治理论,任何一方皆不可或缺。

自然与社会并重的医疗思想,贯穿于《内经》防治疾病的理论与实践中。在疾病的预防上,《内经》主张“虚邪贼风,避之有时,恬憺虚无,真气从之,精神内守,病安从来”(《素问·上古天真论》)。虚邪贼风是来自自然界的邪气,要及时避开它,以免伤形;思想要淡泊,摒除由于社会不良影响而滋生的一切私欲杂念,以免伤神。从内外两方面进行养生防病,肯定会收到良好的效果。《素问·宝命全形论》还以官府发布政令的口气说道:“针有悬布天下者五,黔首共余食,莫知之也。一曰治神,二曰知养身,三曰知毒药之真,四曰制砭石大小,五曰知腑脏血气之诊。”所强调的都是诊治疾病的要领,其中前两项治神与养身则是治病的总要求。治神以调节情志,解除悲哀忧愁的困扰,养身可以壮形体,令气血和调而驱邪外出。治神与养身结合非常适于心身疾病的需要,也可视为治病的一般性原则。

《内经》倡导的“病为本,工为标”的思想,又提示医生在全面了解治疗对象的情况之后,要实施个体化治疗。临床工作的最终落脚点是生活在自然环境中的病人,社会人群中的个体。证候的形成具有十分复杂的个体性因素,在这些个性因素中除性别、年龄和体质差异外,还常涉及社会性内容,如政治经济状况、文化修养、宗教信仰、人格类型、心理特点、社会经历和家庭环境等。因此,在治疗时,施针用药的法与方都应避免千篇一律,力求做到因人而异。如同是针刺疗法,对“身体柔脆,肌肉软弱,血气慓悍滑利”的“王公大人”,其手法要轻,勿深刺;对筋强骨硬,肌肤粗糙,血气涩滞的“布衣匹夫之士”,可深刺,并留针。即“刺布衣者,深以留之;刺大人者,微以徐之”(《灵枢·根结》)。同发寒痹之病,也有“刺布衣者,以火焠之;刺大人者,以药熨之”(《灵枢·寿夭刚柔》)的差别。熨法比较温和,达官显贵及一些读书人易于接受;火针峻猛,去病快捷,饱经风霜的劳动者多能耐受。药治亦当如此。对那些骄奢淫逸,危及身体而又不自省者,则可实施开导及训诫之法:“王公大人,血食之君,骄恣纵欲,轻人,而无能禁之,禁之则逆其志,顺之则加其病,便之奈何?治之何先?岐桕曰:人之情,莫不恶死而乐生,告之以其败,语之以其善,导之以其所便,开之以其所苦,虽有无道之人,恶有不听者乎?”(《灵枢·师传》)通过心理疏导,首先让病人做到与医生合作,在心理治疗的基础上,再施以针药,可望获得良好的疗效。

\subsection{构建了较完善的医学模式}%三、

《内经》集中体现了中国古代生命科学的成果,充分阐释了生存于自然环境和社会环境中的人的生命运动规律和疾病的发生发展规律,并以此为基础构建了中医学的医学模式。《内经》非常深刻地认识到人的自然属性,“天覆地载,万物悉备,莫过于人,人以天地之气生,四时之法成”(《素问·宝命全形论》),人的生命活动是与天地息息相关的。人是万物之尊,生灵之首,人的自然属性,也就是生物属性,这是医学模式中的一个要素。还由于人的健康状态及寿命长短与社会环境关系密切,因而社会又是医学模式中的另一要素。于是,这个模式便可表述为自然(生物)社会医学模式。中医学重七情,情志活动是心理活动的主要外在表现。心理活动每有一定的社会背景,换句话说,社会因素对人的影响往往显现于心理变化之中,故又可将上述医学模式称为自然(生物)社会心理医学模式。

按医学模式的说法来自于西医学及医学心理学。西方中世纪以后,医学家广泛采用物理学、化学等学科的先进理论和技术,对人体进行深入的研究,从系统、器官、细胞以至深入到分子水平,在防治疾病上取得了突破。但经典的西方医学习惯于将人只看成是生物的人,忽视了人的社会属性,甚至看不到社会心理因素对人体的作用,很少注意行为和心理过程,有关这种医学模式,被称为生物医学模式。近年来,人们逐步认识到,生物医学模式已不足以阐明人类健康和疾病的全部本质。于是经心理学工作者的努力,新的生物心理社会医学模式被提出来。这种医学模式,要求医学把人看成备一个完整的统一体,人不仅是生物的人,而且是社会的人,人是有心理活动的,心理活动通过神经、内分泌机制与躯体、系统、细胞、分子水平的生物活动相联系,从而影响这些水平的生理功能。心理活动又与社会环境相互影响。普遍认为,这种医学模式能够更全面、更正确地认识人,认识健康和疾病。

显然,《内经》中早已存在的中医学的医学模式,与西医学新改进的医学模式十分接近。应该说这是两千多年前中国古代医学家的真知灼见,也是中国古代灿烂的科学文化的必然反映,如同火药、指南针、造纸术、活字印刷术率先诞生于中国一样。

由于《内经》构建了合于生命活动规律的医学模式,故能在诊察疾病时做到视野开阔,能在宏观水平上从天地自然和社会生活多方面、多角度探求疾病的发生,“诊合微之事,追阴阳之变,章五中之情”(《素问·方盛衰论》),从而,天才地推测出一些疑难病的病因。如论积症的生成,《内经》认为除饮食、劳伤外,忧怒等情志不和也可成为致病因素:“卒然外中于寒,而内伤于忧怒,则气上逆,气上逆则六输不通,温气不行,凝血蕴里而不散,津液涩渗,著而不去,而积皆成矣”(《灵枢·百病始生》)。而晚近西医对恶性肿瘤,通过流行病调査,也显示了精神创伤及焦虑、绝望等不良情绪因素在发病上的重要作用。再如糖尿病,现代学者已重视社会心理因素,认为生活与工作中的重大变故、挫折和心理冲突等,与糖尿病的发生和加剧关系密切,而《内经》对消渴病(其中多为糖尿病)的发生,早已将饮食和不良情志视为重要的原因。《内经》的有关结论,显然得之于长期的、大量的医疗实践,这些结论经反复验证,进而上升为医学思想。

《内经》的这种医学模式,保证了中医学理论的稳定性和疗效的可靠性。其不足之处是长期停留于宏观大体层面,未能切入微观领域,也未能与科技新进展及时结合,这也正是新时代中医学所面临的问题。

\zuozhe{(赵明山)}
\ifx \allfiles \undefined
\end{document}
\fi