% -*- coding: utf-8 -*-
%!TEX program = xelatex
\ifx \allfiles \undefined
\documentclass[draft,12pt]{ctexbook}
%\usepackage{xeCJK}
%\usepackage[14pt]{extsizes} %支持8,9,10,11,12,14,17,20pt

%===================文档页面设置====================
%---------------------印刷版尺寸--------------------
%\usepackage[a4paper,hmargin={2.3cm,1.7cm},vmargin=2.3cm,driver=xetex]{geometry}
%--------------------电子版------------------------
\usepackage[a4paper,margin=2cm,driver=xetex]{geometry}
%\usepackage[paperwidth=9.2cm, paperheight=12.4cm, width=9cm, height=12cm,top=0.2cm,
%            bottom=0.4cm,left=0.2cm,right=0.2cm,foot=0cm, nohead,nofoot,driver=xetex]{geometry}

%===================自定义颜色=====================
\usepackage{xcolor}
  \definecolor{mybackgroundcolor}{cmyk}{0.03,0.03,0.18,0}
  \definecolor{myblue}{rgb}{0,0.2,0.6}

%====================字体设置======================
%--------------------中文字体----------------------
%-----------------------xeCJK下设置中文字体------------------------------%
\setCJKfamilyfont{song}{SimSun}                             %宋体 song
\newcommand{\song}{\CJKfamily{song}}                        % 宋体   (Windows自带simsun.ttf)
\setCJKfamilyfont{xs}{NSimSun}                              %新宋体 xs
\newcommand{\xs}{\CJKfamily{xs}}
\setCJKfamilyfont{fs}{FangSong_GB2312}                      %仿宋2312 fs
\newcommand{\fs}{\CJKfamily{fs}}                            %仿宋体 (Windows自带simfs.ttf)
\setCJKfamilyfont{kai}{KaiTi_GB2312}                        %楷体2312  kai
\newcommand{\kai}{\CJKfamily{kai}}
\setCJKfamilyfont{yh}{Microsoft YaHei}                    %微软雅黑 yh
\newcommand{\yh}{\CJKfamily{yh}}
\setCJKfamilyfont{hei}{SimHei}                                    %黑体  hei
\newcommand{\hei}{\CJKfamily{hei}}                          % 黑体   (Windows自带simhei.ttf)
\setCJKfamilyfont{msunicode}{Arial Unicode MS}            %Arial Unicode MS: msunicode
\newcommand{\msunicode}{\CJKfamily{msunicode}}
\setCJKfamilyfont{li}{LiSu}                                            %隶书  li
\newcommand{\li}{\CJKfamily{li}}
\setCJKfamilyfont{yy}{YouYuan}                             %幼圆  yy
\newcommand{\yy}{\CJKfamily{yy}}
\setCJKfamilyfont{xm}{MingLiU}                                        %细明体  xm
\newcommand{\xm}{\CJKfamily{xm}}
\setCJKfamilyfont{xxm}{PMingLiU}                             %新细明体  xxm
\newcommand{\xxm}{\CJKfamily{xxm}}

\setCJKfamilyfont{hwsong}{STSong}                            %华文宋体  hwsong
\newcommand{\hwsong}{\CJKfamily{hwsong}}
\setCJKfamilyfont{hwzs}{STZhongsong}                        %华文中宋  hwzs
\newcommand{\hwzs}{\CJKfamily{hwzs}}
\setCJKfamilyfont{hwfs}{STFangsong}                            %华文仿宋  hwfs
\newcommand{\hwfs}{\CJKfamily{hwfs}}
\setCJKfamilyfont{hwxh}{STXihei}                                %华文细黑  hwxh
\newcommand{\hwxh}{\CJKfamily{hwxh}}
\setCJKfamilyfont{hwl}{STLiti}                                        %华文隶书  hwl
\newcommand{\hwl}{\CJKfamily{hwl}}
\setCJKfamilyfont{hwxw}{STXinwei}                                %华文新魏  hwxw
\newcommand{\hwxw}{\CJKfamily{hwxw}}
\setCJKfamilyfont{hwk}{STKaiti}                                    %华文楷体  hwk
\newcommand{\hwk}{\CJKfamily{hwk}}
\setCJKfamilyfont{hwxk}{STXingkai}                            %华文行楷  hwxk
\newcommand{\hwxk}{\CJKfamily{hwxk}}
\setCJKfamilyfont{hwcy}{STCaiyun}                                 %华文彩云 hwcy
\newcommand{\hwcy}{\CJKfamily{hwcy}}
\setCJKfamilyfont{hwhp}{STHupo}                                 %华文琥珀   hwhp
\newcommand{\hwhp}{\CJKfamily{hwhp}}

\setCJKfamilyfont{fzsong}{Simsun (Founder Extended)}     %方正宋体超大字符集   fzsong
\newcommand{\fzsong}{\CJKfamily{fzsong}}
\setCJKfamilyfont{fzyao}{FZYaoTi}                                    %方正姚体  fzy
\newcommand{\fzyao}{\CJKfamily{fzyao}}
\setCJKfamilyfont{fzshu}{FZShuTi}                                    %方正舒体 fzshu
\newcommand{\fzshu}{\CJKfamily{fzshu}}

\setCJKfamilyfont{asong}{Adobe Song Std}                        %Adobe 宋体  asong
\newcommand{\asong}{\CJKfamily{asong}}
\setCJKfamilyfont{ahei}{Adobe Heiti Std}                            %Adobe 黑体  ahei
\newcommand{\ahei}{\CJKfamily{ahei}}
\setCJKfamilyfont{akai}{Adobe Kaiti Std}                            %Adobe 楷体  akai
\newcommand{\akai}{\CJKfamily{akai}}

%------------------------------设置字体大小------------------------%
\newcommand{\chuhao}{\fontsize{42pt}{\baselineskip}\selectfont}     %初号
\newcommand{\xiaochuhao}{\fontsize{36pt}{\baselineskip}\selectfont} %小初号
\newcommand{\yihao}{\fontsize{28pt}{\baselineskip}\selectfont}      %一号
\newcommand{\xiaoyihao}{\fontsize{24pt}{\baselineskip}\selectfont}
\newcommand{\erhao}{\fontsize{21pt}{\baselineskip}\selectfont}      %二号
\newcommand{\xiaoerhao}{\fontsize{18pt}{\baselineskip}\selectfont}  %小二号
\newcommand{\sanhao}{\fontsize{15.75pt}{\baselineskip}\selectfont}  %三号
\newcommand{\sihao}{\fontsize{14pt}{\baselineskip}\selectfont}%     四号
\newcommand{\xiaosihao}{\fontsize{12pt}{\baselineskip}\selectfont}  %小四号
\newcommand{\wuhao}{\fontsize{10.5pt}{\baselineskip}\selectfont}    %五号
\newcommand{\xiaowuhao}{\fontsize{9pt}{\baselineskip}\selectfont}   %小五号
\newcommand{\liuhao}{\fontsize{7.875pt}{\baselineskip}\selectfont}  %六号
\newcommand{\qihao}{\fontsize{5.25pt}{\baselineskip}\selectfont}    %七号   %中文字体及字号设置
\xeCJKDeclareSubCJKBlock{SIP}{
  "20000 -> "2A6DF,   % CJK Unified Ideographs Extension B
  "2A700 -> "2B73F,   % CJK Unified Ideographs Extension C
  "2B740 -> "2B81F    % CJK Unified Ideographs Extension D
}
%\setCJKmainfont[SIP={[AutoFakeBold=1.8,Color=red]Sun-ExtB},BoldFont=黑体]{宋体}    % 衬线字体 缺省中文字体

\setCJKmainfont{simsun.ttc}[
  Path=fonts/,
  SIP={[Path=fonts/,AutoFakeBold=1.8,Color=red]simsunb.ttf},
  BoldFont=simhei.ttf
]

%SimSun-ExtB
%Sun-ExtB
%AutoFakeBold:自动伪粗,即正文使用\bfseries时生僻字使用伪粗体;
%FakeBold:强制伪粗,即正文中生僻字均使用伪粗体
%\setCJKmainfont[BoldFont=STHeiti,ItalicFont=STKaiti]{STSong}
%\setCJKsansfont{微软雅黑}黑体
%\setCJKsansfont[BoldFont=STHeiti]{STXihei} %serif是有衬线字体sans serif 无衬线字体
%\setCJKmonofont{STFangsong}    %中文等宽字体

%--------------------英文字体----------------------
\setmainfont{simsun.ttc}[
  Path=fonts/,
  BoldFont=simhei.ttf
]
%\setmainfont[BoldFont=黑体]{宋体}  %缺省英文字体
%\setsansfont
%\setmonofont

%===================目录分栏设置====================
\usepackage[toc,lof,lot]{multitoc}    % 目录(含目录、表格目录、插图目录)分栏设置
  %\renewcommand*{\multicolumntoc}{3} % toc分栏数设置,默认为两栏(\multicolumnlof,\multicolumnlot)
  %\setlength{\columnsep}{1.5cm}      % 调整分栏间距
  \setlength{\columnseprule}{0.2pt}   % 调整分栏竖线的宽度

%==================章节格式设置====================
\setcounter{secnumdepth}{3} % 章节等编号深度 3:子子节\subsubsection
\setcounter{tocdepth}{2}    % 目录显示等度 2:子节

\xeCJKsetup{%
  CJKecglue=\hspace{0.15em},      % 调整中英(含数字)间的字间距
  %CJKmath=true,                  % 在数学环境中直接输出汉字(不需要\text{})
  AllowBreakBetweenPuncts=true,   % 允许标点中间断行,减少文字行溢出
}

\ctexset{%
  part={
    name={,篇},
    number=\SZX{part},
    format={\chuhao\bfseries\centering},
    nameformat={},titleformat={}
  },
  section={
    number={\chinese{section}},
    name={第,节}
  },
  subsection={
    number={\chinese{subsection}、},
    aftername={\hspace{-0.01em}}
  },
  subsubsection={
    number={(\chinese{subsubsection})},
    aftername={\hspace {-0.01em}},
    beforeskip={1.3ex minus .8ex},
    afterskip={1ex minus .6ex},
    indent={\parindent}
  },
  paragraph={
    beforeskip=.1\baselineskip,
    indent={\parindent}
  }
}

\newcommand*\SZX[1]{%
  \ifcase\value{#1}%
    \or 上%
    \or 中%
    \or 下%
  \fi
}

%====================页眉设置======================
\usepackage{titleps}%或者\usepackage{titlesec},titlesec包含titleps
\newpagestyle{special}[\small\sffamily]{
  %\setheadrule{.1pt}
  \headrule
  \sethead[\usepage][][\chaptertitle]
  {\chaptertitle}{}{\usepage}
}

\newpagestyle{main}[\small\sffamily]{
  \headrule
  %\sethead[\usepage][][第\thechapter 章\quad\chaptertitle]
%  {\thesection\quad\sectiontitle}{}{\usepage}}
  \sethead[\usepage][][第\chinese{chapter}章\quad\chaptertitle]
  {第\chinese{section}节\quad\sectiontitle}{}{\usepage}
}

\newpagestyle{main2}[\small\sffamily]{
  \headrule
  \sethead[\usepage][][第\chinese{chapter}章\quad\chaptertitle]
  {第\chinese{section}節\quad\sectiontitle}{}{\usepage}
}

%================ PDF 书签设置=====================
\usepackage{bookmark}[
  depth=2,        % 书签深度 2:子节
  open,           % 默认展开书签
  openlevel=2,    % 展开书签深度 2:子节
  numbered,       % 显示编号
  atend,
]
  % 相比hyperref,bookmark宏包大多数时候只需要编译一次,
  % 而且书签的颜色和字体也可以定制。
  % 比hyperref 更专业 (自动加载hyperref)

%\bookmarksetup{italic,bold,color=blue} % 书签字体斜体/粗体/颜色设置

%------------重置每篇章计数器,必须在hyperref/bookmark之后------------
\makeatletter
  \@addtoreset{chapter}{part}
\makeatother

%------------hyperref 超链接设置------------------------
\hypersetup{%
  pdfencoding=auto,   % 解决新版ctex,引起hyperref UTF-16预警
  colorlinks=true,    % 注释掉此项则交叉引用为彩色边框true/false
  pdfborder=001,      % 注释掉此项则交叉引用为彩色边框
  citecolor=teal,
  linkcolor=myblue,
  urlcolor=black,
  %psdextra,          % 配合使用bookmark宏包,可以直接在pdf 书签中显示数学公式
}

%------------PDF 属性设置------------------------------
\hypersetup{%
  pdfkeywords={黄帝内经,内经,内经讲义,21世纪课程教材},    % 关键词
  %pdfsubject={latex},        % 主题
  pdfauthor={主编:王洪图},   % 作者
  pdftitle={内经讲义},        % 标题
  %pdfcreator={texlive2011}   % pdf创建器
}

%------------PDF 加密----------------------------------
%仅适用于xelatex引擎 基于xdvipdfmx
%\special{pdf:encrypt ownerpw (abc) userpw (xyz) length 128 perm 2052}

%仅适用于pdflatex引擎
%\usepackage[owner=Donald,user=Knuth,print=false]{pdfcrypt}

%其他可使用第三方工具 如:pdftk
%pdftk inputfile.pdf output outputfile.pdf encrypt_128bit owner_pw yourownerpw user_pw youruserpw

%=============自定义环境、列表及列表设置================
% 标题
\def\biaoti#1{\vspace{1.7ex plus 3ex minus .2ex}{\bfseries #1}}%\noindent\hei
% 小标题
\def\xiaobt#1{{\bfseries #1}}
% 小结
\def\xiaojie {\vspace{1.8ex plus .3ex minus .3ex}\centerline{\large\bfseries 小\ \ 结}\vspace{.1\baselineskip}}
% 作者
\def\zuozhe#1{\rightline{\bfseries #1}}

\newcounter{yuanwen}    % 新计数器 yuanwen
\newcounter{jiaozhu}    % 新计数器 jiaozhu

\newenvironment{yuanwen}[2][【原文】]{%
  %\biaoti{#1}\par
  \stepcounter{yuanwen}   % 计数器 yuanwen+1
  \bfseries #2}
  {}

\usepackage{enumitem}
\newenvironment{jiaozhu}[1][【校注】]{%
  %\biaoti{#1}\par
  \stepcounter{jiaozhu}   % 计数器 jiaozhu+1
  \begin{enumerate}[%
    label=\mylabel{\arabic*}{\circledctr*},before=\small,fullwidth,%
    itemindent=\parindent,listparindent=\parindent,%labelsep=-1pt,%labelwidth=0em,
    itemsep=0pt,topsep=0pt,partopsep=0pt,parsep=0pt
  ]}
  {\end{enumerate}}

%===================注解与原文相互跳转====================
%----------------第1部分 设置相互跳转锚点-----------------
\makeatletter
  \protected\def\mylabel#1#2{% 注解-->原文
    \hyperlink{back:\theyuanwen:#1}{\Hy@raisedlink{\hypertarget{\thejiaozhu:#1}{}}#2}}

  \protected\def\myref#1#2{% 原文-->注解
    \hyperlink{\theyuanwen:#1}{\Hy@raisedlink{\hypertarget{back:\theyuanwen:#1}{}}#2}}
  %此处\theyuanwen:#1实际指thejiaozhu:#1,只是\thejiaozhu计数器还没更新,故使用\theyuanwen计数器代替
\makeatother

\protected\def\myjzref#1{% 脚注中的引用(引用到原文)
  \hyperlink{\theyuanwen:#1}{\circlednum{#1}}}

\def\sb#1{\myref{#1}{\textsuperscript{\circlednum{#1}}}}    % 带圈数字上标

%----------------第2部分 调整锚点垂直距离-----------------
\def\HyperRaiseLinkDefault{.8\baselineskip} %调整锚点垂直距离
%\let\oldhypertarget\hypertarget
%\makeatletter
%  \def\hypertarget#1#2{\Hy@raisedlink{\oldhypertarget{#1}{#2}}}
%\makeatother

%====================带圈数字列表标头====================
\newfontfamily\circledfont[Path = fonts/]{meiryo.ttc}  % 日文字体,明瞭体
%\newfontfamily\circledfont{Meiryo}  % 日文字体,明瞭体

\protected\def\circlednum#1{{\makexeCJKinactive\circledfont\textcircled{#1}}}

\newcommand*\circledctr[1]{%
  \expandafter\circlednum\expandafter{\number\value{#1}}}
\AddEnumerateCounter*\circledctr\circlednum{1}

% 参考自:http://bbs.ctex.org/forum.php?mod=redirect&goto=findpost&ptid=78709&pid=460496&fromuid=40353

%======================插图/tikz图========================
\usepackage{graphicx,subcaption,wrapfig}    % 图,subcaption含子图功能代替subfig,图文混排
  \graphicspath{{img/}}                     % 设置图片文件路径

\def\pgfsysdriver{pgfsys-xetex.def}         % 设置tikz的驱动引擎
\usepackage{tikz}
  \usetikzlibrary{calc,decorations.text,arrows,positioning}

%---------设置tikz图片默认格式(字号、行间距、单元格高度)-------
\let\oldtikzpicture\tikzpicture
\renewcommand{\tikzpicture}{%
  \small
  \renewcommand{\baselinestretch}{0.2}
  \linespread{0.2}
  \oldtikzpicture
}

%=========================表格相关===============================
\usepackage{%
  multirow,                   % 单元格纵向合并
  array,makecell,longtable,   % 表格功能加强,tabu的依赖
  tabu-last-fix,              % "强大的表格工具" 本地修复版
  diagbox,                    % 表头斜线
  threeparttable,             % 表格内脚注(需打补丁支持tabu,longtabu)
}

%----------给threeparttable打补丁用于tabu,longtabu--------------
%解决方案来自:http://bbs.ctex.org/forum.php?mod=redirect&goto=findpost&ptid=80318&pid=467217&fromuid=40353
\usepackage{xpatch}

\makeatletter
  \chardef\TPT@@@asteriskcatcode=\catcode`*
  \catcode`*=11
  \xpatchcmd{\threeparttable}
    {\TPT@hookin{tabular}}
    {\TPT@hookin{tabular}\TPT@hookin{tabu}}
    {}{}
  \catcode`*=\TPT@@@asteriskcatcode
\makeatother

%------------设置表格默认格式(字号、行间距、单元格高度)------------
\let\oldtabular\tabular
\renewcommand{\tabular}{%
  \renewcommand\baselinestretch{0.9}\small    % 设置行间距和字号
  \renewcommand\arraystretch{1.5}             % 调整单元格高度
  %\renewcommand\multirowsetup{\centering}
  \oldtabular
}
%设置行间距,且必须放在字号设置前 否则无效
%或者使用\fontsize{<size>}{<baseline>}\selectfont 同时设置字号和行间距

\let\oldtabu\tabu
\renewcommand{\tabu}{%
  \renewcommand\baselinestretch{0.9}\small    % 设置行间距和字号
  \renewcommand\arraystretch{1.8}             % 调整单元格高度
  %\renewcommand\multirowsetup{\centering}
  \oldtabu
}

%------------模仿booktabs宏包的三线宽度设置---------------
\def\toprule   {\Xhline{.08em}}
\def\midrule   {\Xhline{.05em}}
\def\bottomrule{\Xhline{.08em}}
%-------------------------------------
%\setlength{\arrayrulewidth}{2pt} 设定表格中所有边框的线宽为同样的值
%\Xhline{} \Xcline{}分别设定表格中水平线的宽度 makecell包提供

%表格中垂直线的宽度可以通过在表格导言区(preamble),利用命令 !{\vrule width1.2pt} 替换 | 即可

%=================图表设置===============================
%---------------图表标号设置-----------------------------
\renewcommand\thefigure{\arabic{section}-\arabic{figure}}
\renewcommand\thetable {\arabic{section}-\arabic{table}}

\usepackage{caption}
  \captionsetup{font=small,}
  \captionsetup[table] {labelfont=bf,textfont=bf,belowskip=3pt,aboveskip=0pt} %仅表格 top
  \captionsetup[figure]{belowskip=0pt,aboveskip=3pt}  %仅图片 below

%\setlength{\abovecaptionskip}{3pt}
%\setlength{\belowcaptionskip}{3pt} %图、表题目上下的间距
\setlength{\intextsep}   {5pt}  %浮动体和正文间的距离
\setlength{\textfloatsep}{5pt}

%====================全文水印==========================
%解决方案来自:
%http://bbs.ctex.org/forum.php?mod=redirect&goto=findpost&ptid=79190&pid=462496&fromuid=40353
%https://zhuanlan.zhihu.com/p/19734756?columnSlug=LaTeX
\usepackage{eso-pic}

%eso-pic中\AtPageCenter有点水平偏右
\renewcommand\AtPageCenter[1]{\parbox[b][\paperheight]{\paperwidth}{\vfill\centering#1\vfill}}

\newcommand{\watermark}[3]{%
  \AddToShipoutPictureBG{%
    \AtPageCenter{%
      \tikz\node[%
        overlay,
        text=red!50,
        %font=\sffamily\bfseries,
        rotate=#1,
        scale=#2
      ]{#3};
    }
  }
}

\newcommand{\watermarkoff}{\ClearShipoutPictureBG}

\watermark{45}{15}{草\ 稿}    %启用全文水印

%=============花括号分支结构图=========================
\usepackage{schemata}

\xpatchcmd{\schema}
  {1.44265ex}{-1ex}
  {}{}

\newcommand\SC[2] {\schema{\schemabox{#1}}{\schemabox{#2}}}
\newcommand\SCh[4]{\Schema{#1}{#2}{\schemabox{#3}}{\schemabox{#4}}}

%=======================================================

\begin{document}
\pagestyle{main}
\fi
\chapter{《黄帝内经》的成书与流传} %第一章

\section{《内经》的成书}%第一节

\subsection{《内经》的成书年代及作者} %一、

传世本《黄帝内经》由《黄帝内经素问》和《黄帝内经灵枢》两部书组成。《黄帝内经》之名,最早见于《汉书·艺文志·方技略》书目中;《黄帝内经素问》之名,出于唐代医家、《素问》整理者王冰之手;《黄帝内经灵枢》之名,出于宋代医家、《灵枢》整理者史崧之手。无论是汉志所著录的《黄帝内经》,还是王本《素问》、史本《灵枢》,皆未署原作者之名,亦未标记撰著年代,于是关于《内经》的成书年代及作者,便成为千百年来学者医家争讼不已的问题,古今《内经》成书说很多,归纳起来主要有3种:(1)黄帝时书;(2)成书于春秋战国及秦汉之际;(3)汇编成书于西汉。

主张《内经》为黄帝时书者,多为医家。公元三世纪,魏晋人皇甫谧据《针经》、《素问》和《明堂孔穴针灸治要》三部古医书编纂了《针灸甲乙经》,他在书序上说:“《黄帝内经》十八卷,今有《针经》九卷,《素问》九卷,二九十八卷,即《内经》也……又有《明堂孔穴针灸治要》,皆黄帝岐伯选事也。三部同归,文多重复。”《针经》,原称《九卷》,后又有《灵枢》等名。“皆黄帝岐伯选事”,即认为三部书都是黄帝和岐伯的医论,其书自然是黄帝时作。五百年后的唐人王冰,九百年后的宋人史崧皆宗皇甫氏之说,对《内经》是黄帝时书深信不疑。其实,至北宋年间已有一批学者对《素问》是否真为黄帝书产生了疑问,但奉敕校正医书的高保衡、林亿等人仍坚持古医家的看法:“或曰《素问》、《针经》、《明堂》三部之书,非黄帝书,似出于战国。曰:人生天地之间,八尺之躯,脏之坚脆,腑之大小,谷之多少,脉之长短,血之清浊,十二经之血气大数,皮肤包络其外,可剖而视之乎!非大圣上智,孰能知之?战国之人何与焉!大哉《黄帝内经》十八卷,《针经》三卷,最出远古。”(《新校正黄帝针灸甲乙经序》)这种看法一直影响到明清的众多医家。

世人确信《内经》为黄帝时书,究其原因,除他们不谙考据外,与受崇古尊经思想的支配也不无关系。他们认为,最出远古圣人的东西才是最博大最精深最值得研习的。他们提出的依据(或不提依据,仅凭直觉),往往是表面上的,如书名有“黄帝”字样,书中多黄帝与岐伯、伯高、雷公等臣子的问答之语,深奥之语只能出自圣人,而上古圣人又非三皇五帝莫属。民间的传说,书名的假托,竟使许多医家信以为真。如据史实稍加考证,便不会得出“大哉《黄帝内经》,最出远古”的结论。

认为《内经》成书于春秋战国或称周秦之间的,以文史学者居多。如宋代的理学家邵雍和二程兄弟(程颢、程颐),史学家司马光等,率先对《内经》是黄帝书提出了质疑。程颢说:“观《素问》文字气象,只是战国时人作,谓之三坟书则非也。”(《二程全书》)司马光说:“谓《素问》为真黄帝之书,则恐未可。黄帝亦治天下,岂终日坐明堂,但与岐伯论医药针灸耶?此周秦之间,医者依托而取重耳。”(《传家集·书启》)。此后元、明、清各代学者皆踵之而论。《四库全书简明目录》称:“《黄帝素问》原本残阙,王冰采《阴阳大论》以补之。其书云出上古,固未必然,然亦必周秦间人,传述旧闻,著之竹帛。故贯通三才,包罗万变。”清代医家魏荔彤也明确指出:“轩岐之书,类春秋战国人所为,而托于上古。文顺义泽,篇章连贯,读如俨如《礼经》也。”(《伤寒论本义序》)近人赞同此说者很多,也反映到20世纪80年代前全国各中医院校使用的《内经》教材之中。

持《内经》成书于春秋战国说者,显较前一说法进了一步,已从书的内容、文字、笔法等做了初步论证,但论据偏少,结论尚嫌笼统。

认为《内经》成编于西汉者,既有古代学者,更多的则是现代学者。明代学者郎瑛说:“《素问》文非上古,人得知之。以为即全元起所著,犹非隋唐文也。惟马迁刘向近之,又无此等义语。宋·聂吉甫云,既非三代以前文,又非东都以后语,断然以为淮南王之作。予意《鸿烈解》中,内篇文义,实似之矣,但淮南好名之士,即欲藉岐黄以成名,特不可曰述也乎。成医卜未焚,当时必有岐黄问答之书,安得文以成耳。”生活年代相当于我国清朝中叶的日本人丹波元简、丹波元胤父子,力主《内经》为汉代人所撰:“此书实医经之最古者,迨圣之遗言存焉。而晋皇甫谧以下,历代医家断为黄岐所自作,此殊不然也……汉之时,凡说阴阳者,必系之黄帝。《淮南子》云:‘黄帝生明阳。’又云:‘世俗人多尊古而贱今,故为道者,必托之于神农、黄帝,而后能人说’……此经设为黄帝岐伯之问答者,亦汉人所撰述无疑矣。方今医家,或牵合衍赘,以为三坟之一,或诋毁排斥,以为赝伪之书者,俱失焉。”(丹波元简《素问识·素问解题》)这些说法虽有理,但论据仍嫌不足。晚近学者在前人研究的基础上,进一步从《内经》的学术思想、社会背景,所反映的科学技术水平和各种文化现象等,多角度据实考证,特别是采用了20世纪70年代以来的考古新发现,从而得出了《内经》是中国古代医学理论的总集,汇编成书于西汉年间的结论。

另外,还有一些学者对《黄帝内经》在流传中何以分为两部书,存有疑问;有的据《内经》的“五脏五行观”当出现于东汉,而认为其书可能撰著于西汉末至东汉末之间;还有“伪书”说等等。此皆因迄今世上未见《黄帝内经》古本原貌,遂令歧说并出,莫衷一是。

面对着《黄帝内经》这样一部卷帙浩繁的古代医学论文总集,首先要弄清其材料来源和汇编成书是两回事,虽然二者有着密切的联系。全书162篇中,所反映的学术观点、理论水平、技术运用,以及篇幅大小、语言文字等皆存在一定的差异,篇与篇间,乃至灵素两部书之间,还有文字重出及互引互解的现象;全书引用更古的医经文献具名称者多达30余种。说明材料来源久远而地域亦多,自非一时一人之作。一篇文章写成之后,在单篇别行过程中,可能又经多人的传抄、损益,最后方为汇编者所收集、整理、入编。汇编成书之后,才以一个新的统一的书名流传于世。

《黄帝内经》之名,在史籍上首见于《汉书·艺文志》,其《方技略》载有:“《黄帝内经》十八卷、《外经》三十七卷,《扁鹊内经》九卷、《外经》十二卷,《白氏内经》三十八卷、《外经》三十六卷,《旁篇》二十五卷”,合为“医经七家,二百一十六卷”。另有经方十一家,二百七十四卷;房中八家,一百八十六卷;神仙十家,二百零五卷。《汉书·艺文志》是汉书作者班固据《七略》,“删其要,以备篇籍”而成。《七略》则是西汉末刘向、刘歆父子奉诏校书时撰写的我国第一部图书分类目录。其中分工校方技类(医药类)书籍的是朝廷侍医李柱国。史载李柱国校刊医书的时间是在西汉成帝河平三年(公元前26年)。就是说,西汉末成帝年间,《黄帝内经》十八卷本已成编问世,业经李柱国校刊整理,由刘歆著录于《七略》之中。一般认为,李柱国校医书的时间应定为《内经》成书的下限。

《内经》成书的上限,从史料上推,《汉书》之前的《史记》可以说是一个重要的标志。《史记》之前的《左传》、《国语》和《战国策》等先秦史书,记载医事甚少,且未将医学与黄帝联系起来。《史记》记载了上自黄帝下迄汉武帝长达三千多年的历史,书中重笔书写了各个时期科学文化的发展史,对先秦和汉初诸子及其著作皆有介绍,并专为战国的秦越人(扁鹊)、汉初的淳于意(仓公)两位医家作传,但未见有关《内经》之类的书名。可以推想,如果当时《内经》已经成书流传,那么遍览朝廷藏书,考察过全国各地的太史公司马迁,是应该见得到的。在《扁鹊仓公列传》所记公乘阳庆传给仓公及仓公授徒的一批“禁方书”中,有“黄帝、扁鹊之《脉书》”和《上下经》、《五色诊》,《药论》、《石神》、《接阴阳禁书》等。与《内经》相对照,其中有的书如《五色诊》、《奇晐术(奇恒)》、《揆度》、《上下经》,正是《内经》中称引的古医经,传中所记仓公诊病实录即25个“诊籍”中,病名及诊治方法与《内经》同少异多,且不如《内经》理法之完备;其言“砭灸处”,只称部位,未道穴名,有类于《内经》中个别较古老的篇章。此说明《内经》成书的时间要晚于仓公多年。仓公行医的年份处于汉文帝时期,相当于公元前二世纪的上叶,《史记》写成是在作者入狱(时为公元前99年)之后。如此推算,《内经》汇编成书时间当在《史记》之后、《七略》之前的西汉中后期。

20世纪70年代以来,长沙马王堆等数处汉墓出土文物中,发现了大量的简帛医书,内中虽无《黄帝内经》或《素问》之类的篇卷,但其内容与《内经》有一定的关联,成为考定《内经》一些篇章来源和撰著年代的可靠依据。

1973年,于长沙马王堆3号汉墓出土的简帛医书有《足臂十一脉灸经》、《阴阳十一脉灸经》、《五十二病方》、《脉法》、《导引图》等14种(帛书10种,简书4种,书名为整理者所定)。墓主人是西汉初年封于长沙的轪侯利仓之子,下葬于汉文帝十二年(公元前168年)。整理者据帛书的字体近篆及某些结构特点,而认定其抄写年代大约在秦汉之际。帛书与《内经》相较,其内容远为古朴:帛书载有砭石、药物和灸法,无针法;帛书未提五行,采用阴阳也不普遍;帛书仅称“脉”,而无经络之名,无腧穴;帛书病名未与脏腑联系起来,各医方皆无方名,剂型也较简单。再比较《足臂十一脉灸经》、《阴阳十一脉灸经》与《灵枢·经脉》三者,可以看出两灸经显系《经脉》篇的祖本。两灸经记述了经脉理论形成早期阶段的状态;全身仅十一脉,且彼此不相连,仅个别脉连络脏腑,没有形成密布于全身的网络;在经脉走行方向上,《足臂十一脉灸经》全为向心性,由四肢走向胸腹或头部,《阴阳十一脉灸经》只有三脉是离心的,其余八脉也是向心的;脉名虽已分太少阴阳,但仍遗有臂脉、齿脉、肩脉的古老称呼;在病候上,《足臂十一脉灸经》未分“是动”病和“所生”病,病症数量也偏少,《阴阳十一脉灸经》已分出是动病和所产病两类,病候总量仍少于《灵枢·经脉》。而《灵枢·经脉》增手厥阴心主之脉而成十二经脉,六阴脉和六阳脉首尾依次相接,阴脉属脏络腑,阳脉属腑络脏,形成周而复始、如环无端的气血循环径路,且支脉、络脉甚多,在全身构成无处不到的网络。显然,经脉理论经历了由《足臂十一脉灸经》到《阴阳十一脉灸经》再到《灵枢·经脉》的演变(还可能存在其他中间链条)。而这个由简至繁、由低级向高级的发展过程,是不可能在短期内完成的,按上古时期医学前进的步伐,这至少要有数十年上百年的时间。还可以推断,如果轪侯利仓之子这个热衷于收藏古籍并酷爱医学的世袭官吏,生前若能看到像《灵枢·经脉》那样更成熟的经络学文献,一定会视如至宝,加以收罗和珍藏,其不在陪葬品之列的合理解释只能是:《灵枢·经脉》乃至《内经》书尚未诞生。

1983年,湖北荆州地区江陵县张家山三座西汉初古墓中,出土了大批竹简,其中医学文献有《脉书》和《引书》,据考其抄写年代不晚于吕后二年(公元前186年)。《脉书》中包括5种古佚书,其中3种与马王堆汉墓医书《阴阳十一脉灸经》、《阴阳脉死候》、《脉法》大致相同;《引书》内容与马王堆汉墓的《导引图》关系极为密切。长沙和荆州皆为古楚地,荆楚大地有可能是针刺技术及经脉理论的发样地,《素问·异法方宜论》也存有“九针者,亦从南方来”之说。《灵枢·经脉》及《黄帝内经》既不见于张家山汉墓,也更加证实了《内经》成书确在汉代初年之后。

再从《内经》的理论体系特点来看,全书一个重要的学术特征是广泛深入地运用了阴阳学说和五行学说。中国思想发展史表明,阴阳学说和五行学说各有古老的源头,并在很长的历史时期内各自独立地发展着。五行的相克相生体系的提出及与阴阳学说合流,始自战国末期的阴阳家邹衍,“邹衍睹有国者益淫侈,不能尚德”,“乃深观阴阳消息,而作怪迂之变,终始大圣之篇”,“称引天地剖判以来,五德转移,治各有宜,而符应若兹”(《史记·孟子荀卿列传》)。邹衍倡导的五德终始之说,“及秦帝而齐人奏之,故始皇采用之”(《史记·封禅书》)。秦王嬴政采纳了邹氏之说,以国德之水取代了周之火。其后汉承秦制,初为水运,后改土运,遂令阴阳五行成为秦汉时期占统治地位的思想和宇宙观。汉代著名思想家淮南子刘安和董仲舒等人的学说都铭刻着阴阳五行的印记。《淮南子·原道训》:“德优天地而和阴阳,节四时而调五行。”董仲舒《春秋繁露·五行相生》:“天地之气,合而为一,分为阴阳,判为四时,列为五行……五行者,五官也,比相生而间相胜也。”刘、董皆勾画了一个完整的宇宙图式,其中包含着天地、阴阳、四时、五行及事物间的生克制化规律,用以说明宇宙中包括人的生命活动在内的形形色色事物。在这种政治和理论氛围下·一些学术著作也会染上时尚之风。《内经》多篇所采用的,正是汉代流行的阴阳五行学说。《内经》许多篇章大概就是在这种社会背景下最后写定的,文中一些内容甚至语句与淮南子、董仲舒等人的著作类似,也就不足为怪了。

另外,《内经》的语言文字使用,纪时的习惯,以及金属细针(如九针中的毫针、长针、大针)所要求的材质和冶炼术等,多符合汉代的特点,亦可作为成书年代的佐证。

总之,《内经》一书,其材料来源久远,其撰述者众多,汇总成编为一书的时间大约在公元前一世纪的西汉中后期。

\subsection{《内经》书名的含义} %二、

《内经》书名冠以“黄帝”,当是一种崇古假托,也是汉代的时尚。关于“经”字的含义,《说文》:“经,织也。”即布帛的织线为“经”字的本义。陆德明《经典释文》又释为“常也,法也,径也”,即“规范”,当为引申义。古书称“经”者,有《诗经》、《书经》、《易经》、《道德经》等,医书除《内经》外,还有《神农本草经》、《难经》、《中藏经》等,皆为示人以规范的重要典籍。

《内经》之“内”字是与“外”字相对而言。如《汉书·艺文志》所载书目就有内经、外经多种。正如丹波元胤《医籍考》所说:“内外,犹《易》内外卦,及《春秋》内外传,《庄子》内外篇,《韩非子》内外储说,以次第名焉者,不必有深意。”

关于《素问》书名的含义,各家说法颇不一致。全元起说:“素者,本也。问者,黄帝问岐伯也。方陈性情之源,五行之本,故曰《素问》。”(《新校正》引)马莳、吴昆、张介宾等人则认为,《素问》之义即“平素问答之书”。还有的称“天降素女,以治人疾,帝问之,作《素问》”。皆不足取。而林亿等《新校正》之说似近经旨:“按《乾凿度》云:‘夫有形者生于无形,故有太易,有太初,有太始,有太素。太易者,未见气也;太初者,气之始也;太始者,形之始也;太素者,质之始也。’气形质具,而疴瘵由是萌生,故黄帝问此太素,质之始也,《素问》之名,义或由此。”太易、太初、太始、太素为古人探讨天地形成所分的四个阶段。《素问》正是从天地宇宙的宏观出发,运用精气学说和阴阳五行学说,解释和论证天人关系及人的生命活动规律和疾病的发生发展过程的,确有陈源问本之意,可谓名实相符。若以杨上善《黄帝内经太素》的书名而论,则就更近于“太素者,质之始也”的说法了。

对《灵枢》书名的解释,歧义也多。马莳认为本书是医学的门户,故解云:“医无入门,术难精诣……谓之曰《灵柜》者,正以枢为门户,阖辟所系,而灵乃至圣至元之称,此书之切,何以异是。”张介宾则从医学效应解:“神灵之枢要,谓之《灵枢》。”不过,须知《灵枢》之名在文献上首见于王冰《黄帝内经素问》中,王冰将《针经》称为《灵枢》,可能与其崇信道教有关。正如丹波元胤所说:“今考《道藏》令,有《玉枢》、《神枢》、《灵轴》等之经,而又收入是经,则《灵枢》之称,意出子羽流者欤!”羽流,指道士,即指道号启玄子的王冰而言。此说有一定道理。

\section{《内经》的流传} %第二节

\subsection{《内经》的流传} %一、

按《七略》及《汉书·艺文志》所载,《黄帝内经》曾以十八卷本与《黄帝外经》等医经七家一并传世。《七略》之后至东汉末的一段时间内,《内经》是怎样流传的,史无记载。东汉末张机自述为著《伤寒杂病论》,“乃勤求古训,博采众方,撰用《素问》、《九卷》、《八十一难》、《阴阳大论》、《胎胪药录》,并平脉辨证。”(《伤寒论·序》)张机所参考的这批古医书中,有《素问》,有《九卷》,有《八十一难》即《难经》,唯未见《内经》之名。然张机接着又列举了古代名医:“上古有神农、黄帝、岐伯、伯高、雷公、少俞、少师、仲文,中世有长桑、扁鹊,汉有公乘阳庆及仓公。”此中,黄帝、岐伯曾见于《七略》中,黄帝、岐伯和雷公见于《素问》中,黄帝、岐伯、伯高、雷公、少俞和少师见于《灵枢》中。这表明《素问》可为《黄帝内经》之一部,也表明《九卷》可为《黄帝内经》的另一部(该部书当时无正式书名,以卷数名之)。至魏晋年间皇甫谧也予以印证:“按《七略》、《艺文志》,《黄帝内经》十八卷,今有《针经》九卷,《素问》九卷,二九十八卷,即《内经》也。亦有所亡佚。”《针经》是皇甫谧为《九卷》所命之名,恐与该书第一篇《九针十二原》中有“先立针经”之语有关。

《素问》与《九卷》(《针经》)自晋以后的流传惰况,史料上仍有一些记载可循。《隋书·经籍志》录有“《黄帝素问》九卷”(注云:“梁八卷”),“《黄帝针经》九卷”,说明九卷本《素问》在南北朝时已亡佚一卷。《旧唐书·经籍志》著录:“《黄帝素问》八卷,《黄帝针经》十卷,《黄帝九灵经》十二卷(灵宝注)。”《新唐书·艺文志》:“《黄帝针经》十卷,全元起注《黄帝素问》九卷,灵宝注《黄帝九灵经》十二卷,王冰注《黄帝内经素问》二十四卷。”《九灵经》当为《针经》的不同传本。说明至隋唐,《内经》仍以《素问》和《针经》两书分别传世,卷数有少许变化,流传中又有别本新名出现,而《九卷》之旧名,渐从史志及文献上消失了。

历代医家整理《素问》功劳最大的,当推唐代的王冰。他在唐代宗宝应年间,面对残缺不全的八卷《素问》“世本”,对照家藏“张公秘本”,做了大量的补亡、迁移、别目、加字和削繁等工作,使《素问》恢复到八十一篇旧数,重以二十四卷本行世。一般认为,运气七篇和《六节藏象论》中有关运气的一段,皆为王冰补入,其文体与它篇殊不一致。王氏补入运气七篇之后仍缺两篇,即《刺法论》和《本病论》,篇名仅存目录中,后人补出后称为《素问遗篇》。王冰治学严谨,称“凡所加字,皆朱书其文,使今古必分,字不杂糅”。可惜,在后人的传抄中朱墨已经不分了。经过王冰卓有成效的工作,使《素问》较完善的本子得以继续流传。至宋代仁宗嘉祐年间,髙保衡、林亿等人奉朝廷之命,校勘医籍,对已是“文注纷错,义理混淆”的王冰本,再行考正,“正谬误者六千余字,增注义者二千余条”,并定名为《重广补注黄帝内经素问》。林亿等的校本,即今之所见《素问》的原型,宋以后的元、明、清各代皆据此进行翻刻,未再改易。明·顾从德影宋刊本《素问》,堪称善本,为今所据。

《灵枢》为《针经》的另一传本,其名在文献上最早见于王冰次注的《黄帝内经素问》序和注中。王氏称:“《黄帝内经》十八卷,《素问》即其经之九卷也,兼《灵枢》九卷,乃其数焉。”林亿《新校正》则谓:“按《隋书·经籍志》谓之《九灵》,王冰名为《灵枢》。”意为《灵枢》之名是王冰据《九灵经》所改。又《新校正》在校注《素问·调经论》王注时还指出:“详此注引《针经》曰,与《三部九候论》注两引之,在彼云《灵枢》而此曰《针经》,则王氏之意,指《灵枢》为《针经》也。按今《素问》注中引《针经》者,多《灵枢》之文,但以《灵枢》今不全,故未得尽知也。”林亿在此提供了这样的信息:(1)《针经》、《九灵经》、《灵枢》实为一书的不同传本;(2)《灵枢》至宋代已非全帙。史实确是这样,《灵枢》传至宋代已是残本,幸有高丽使者于宋哲宗元祐七年(公元1092年)来华献书,其中就有《黄帝针经》,哲宗于次年正月即诏颁高丽所献《黄帝针经》于天下,使此书复行于世。

北宋末年,由于金兵的南犯和继之而形成的北金南宋的对峙局势,严重地影响了医籍的保存和流传,《针经》等古籍又面临着散佚和失传的厄运。值南宋绍兴二十五年(公元1155年),锦官(成都)人史崧对《灵枢》进行了认真的整理校核,他在书叙中说:“但恨《灵枢》不传久矣,世莫能究……辄不自揣,参对诸书,再行校正家藏旧本《灵枢》九卷,共八十一篇,增修音释,附于卷末,勒为二十四卷。”卷数与王冰本《素问》相同。史崧校正的《灵枢经》文字,后人未再改动,也成为元、明、清续刻的蓝本。

\subsection{《内经》的注家与注本} %二、

《内经》自问世以来,历代医家皆奉为圭臬,演绎发挥、考校编次、注释研究者达二百家以上,著作达四百余部。现将主要的注家与注本介绍如下。

南朝齐梁间人全元起注《素问》,为《内经》最早的注本。《隋志》载:全氏元起注黄帝《素问》八卷。《南史》王僧儒传中也记有此事。林亿等《新校正》云:“隋杨上善为《太素》。时则有全元起者,始为之训解。阙第七一通。”全氏注《素问》时,只存八卷,第七卷已佚,计注释六十八篇。《素问训解》宋时尚存,后佚。《新校正》中保留其篇目。

杨上善撰注《黄帝内经太素》。据考,杨氏为唐初时人,其书将《内经》分为摄生、阴阳、人合、脏腑、经脉、腧穴、营卫气、身度、诊候、证候、设方、九针、补泻、伤寒、寒热、邪论、风论、气论、杂病共十九大类,每类分若干篇目,并加以注释。书中有关《素问》部分保存了王冰改动之前的原貌,具有很髙的文献价值。本书自宋元后已残缺不全,今本系从日本传回,仍有缺卷。

唐·王冰次注《黄帝内经素问》。王冰对《素问》进行整理编次的同时,对全书各篇作了系统而详尽的注释,“敷畅玄言”,对经旨多有发挥。王注《素问》对后人启发和影响很大,成为后人注释《素问》的基础。

明·马莳通注《内经》全书,书名为《黄帝内经素问注证发微》和《黄帝内经灵枢注证发微》。马氏娴于针灸经脉,其所注《灵枢》颇为人称道。

明·吴昆注《素问》,俗称《素问吴注》,是以王冰的二十四卷本为底本加以注释。注文阐发医理深入而不流于空泛。然擅改经文,是其不足处。

明·张介宾著《类经》,将《内经》内容分为摄生、阴阳、藏象、脉色、经络、标本、气味、论治、疾病、针刺、运气、会通共十二大类,三百九十目,是现存全部类分《素问》、《灵枢》最完整的一部书。张氏的分类法扼要而实用,其注文义理周详,明白晓畅,影响很大。

清·张志聪的《黄帝内经素问集注》、《黄帝内经灵枢集注》,是张志聪率门人集体注释的成果,对前人之注,做到了取其精华,厘正误说,新意不少。

清·高世栻著《黄帝素问直解》,乃继张氏《集注》之作。高氏鉴于“隐庵《集注》意义艰深,其失也晦”,故其注着意于简捷明白,要言不繁。

对《内经》进行摘要选注的,以元·滑寿为最早。所著《读素问钞》,将选文分为十二类。该书后经明·汪机增补注释,易名为《续素问钞》。明·李中梓的《内经知要》,系摘《素问》、《灵枢》之要,编纂而成,选文只分八类,执简驭繁,成为医学入门读物。

另外,日人丹波元简所著《素问识》、《灵枢识》,丹波元坚所著《素问绍识》,精选诸家注释,持论公允,注重考据,也具有很大参考价值。

\zuozhe{(赵明山)}
\ifx \allfiles \undefined
\end{document}
\fi