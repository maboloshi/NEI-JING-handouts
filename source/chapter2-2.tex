% -*- coding: utf-8 -*-
%!TEX program = xelatex
\ifx \allfiles \undefined
\documentclass[12pt]{ctexbook}
%\usepackage{xeCJK}
%\usepackage[14pt]{extsizes} %支持8,9,10,11,12,14,17,20pt

%===================文档页面设置====================
%---------------------印刷版尺寸--------------------
%\usepackage[a4paper,hmargin={2.3cm,1.7cm},vmargin=2.3cm,driver=xetex]{geometry}
%--------------------电子版------------------------
\usepackage[a4paper,margin=2cm,driver=xetex]{geometry}
%\usepackage[paperwidth=9.2cm, paperheight=12.4cm, width=9cm, height=12cm,top=0.2cm,
%            bottom=0.4cm,left=0.2cm,right=0.2cm,foot=0cm, nohead,nofoot,driver=xetex]{geometry}

%===================自定义颜色=====================
\usepackage{xcolor}
	\definecolor{mybackgroundcolor}{cmyk}{0.03,0.03,0.18,0}
	\definecolor{myblue}{rgb}{0,0.2,0.6}

%====================字体设置======================
%--------------------中文字体----------------------
%-----------------------xeCJK下设置中文字体------------------------------%
\setCJKfamilyfont{song}{SimSun}                             %宋体 song
\newcommand{\song}{\CJKfamily{song}}                        % 宋体   (Windows自带simsun.ttf)
\setCJKfamilyfont{xs}{NSimSun}                              %新宋体 xs
\newcommand{\xs}{\CJKfamily{xs}}
\setCJKfamilyfont{fs}{FangSong_GB2312}                      %仿宋2312 fs
\newcommand{\fs}{\CJKfamily{fs}}                            %仿宋体 (Windows自带simfs.ttf)
\setCJKfamilyfont{kai}{KaiTi_GB2312}                        %楷体2312  kai
\newcommand{\kai}{\CJKfamily{kai}}
\setCJKfamilyfont{yh}{Microsoft YaHei}                    %微软雅黑 yh
\newcommand{\yh}{\CJKfamily{yh}}
\setCJKfamilyfont{hei}{SimHei}                                    %黑体  hei
\newcommand{\hei}{\CJKfamily{hei}}                          % 黑体   (Windows自带simhei.ttf)
\setCJKfamilyfont{msunicode}{Arial Unicode MS}            %Arial Unicode MS: msunicode
\newcommand{\msunicode}{\CJKfamily{msunicode}}
\setCJKfamilyfont{li}{LiSu}                                            %隶书  li
\newcommand{\li}{\CJKfamily{li}}
\setCJKfamilyfont{yy}{YouYuan}                             %幼圆  yy
\newcommand{\yy}{\CJKfamily{yy}}
\setCJKfamilyfont{xm}{MingLiU}                                        %细明体  xm
\newcommand{\xm}{\CJKfamily{xm}}
\setCJKfamilyfont{xxm}{PMingLiU}                             %新细明体  xxm
\newcommand{\xxm}{\CJKfamily{xxm}}

\setCJKfamilyfont{hwsong}{STSong}                            %华文宋体  hwsong
\newcommand{\hwsong}{\CJKfamily{hwsong}}
\setCJKfamilyfont{hwzs}{STZhongsong}                        %华文中宋  hwzs
\newcommand{\hwzs}{\CJKfamily{hwzs}}
\setCJKfamilyfont{hwfs}{STFangsong}                            %华文仿宋  hwfs
\newcommand{\hwfs}{\CJKfamily{hwfs}}
\setCJKfamilyfont{hwxh}{STXihei}                                %华文细黑  hwxh
\newcommand{\hwxh}{\CJKfamily{hwxh}}
\setCJKfamilyfont{hwl}{STLiti}                                        %华文隶书  hwl
\newcommand{\hwl}{\CJKfamily{hwl}}
\setCJKfamilyfont{hwxw}{STXinwei}                                %华文新魏  hwxw
\newcommand{\hwxw}{\CJKfamily{hwxw}}
\setCJKfamilyfont{hwk}{STKaiti}                                    %华文楷体  hwk
\newcommand{\hwk}{\CJKfamily{hwk}}
\setCJKfamilyfont{hwxk}{STXingkai}                            %华文行楷  hwxk
\newcommand{\hwxk}{\CJKfamily{hwxk}}
\setCJKfamilyfont{hwcy}{STCaiyun}                                 %华文彩云 hwcy
\newcommand{\hwcy}{\CJKfamily{hwcy}}
\setCJKfamilyfont{hwhp}{STHupo}                                 %华文琥珀   hwhp
\newcommand{\hwhp}{\CJKfamily{hwhp}}

\setCJKfamilyfont{fzsong}{Simsun (Founder Extended)}     %方正宋体超大字符集   fzsong
\newcommand{\fzsong}{\CJKfamily{fzsong}}
\setCJKfamilyfont{fzyao}{FZYaoTi}                                    %方正姚体  fzy
\newcommand{\fzyao}{\CJKfamily{fzyao}}
\setCJKfamilyfont{fzshu}{FZShuTi}                                    %方正舒体 fzshu
\newcommand{\fzshu}{\CJKfamily{fzshu}}

\setCJKfamilyfont{asong}{Adobe Song Std}                        %Adobe 宋体  asong
\newcommand{\asong}{\CJKfamily{asong}}
\setCJKfamilyfont{ahei}{Adobe Heiti Std}                            %Adobe 黑体  ahei
\newcommand{\ahei}{\CJKfamily{ahei}}
\setCJKfamilyfont{akai}{Adobe Kaiti Std}                            %Adobe 楷体  akai
\newcommand{\akai}{\CJKfamily{akai}}

%------------------------------设置字体大小------------------------%
\newcommand{\chuhao}{\fontsize{42pt}{\baselineskip}\selectfont}     %初号
\newcommand{\xiaochuhao}{\fontsize{36pt}{\baselineskip}\selectfont} %小初号
\newcommand{\yihao}{\fontsize{28pt}{\baselineskip}\selectfont}      %一号
\newcommand{\xiaoyihao}{\fontsize{24pt}{\baselineskip}\selectfont}
\newcommand{\erhao}{\fontsize{21pt}{\baselineskip}\selectfont}      %二号
\newcommand{\xiaoerhao}{\fontsize{18pt}{\baselineskip}\selectfont}  %小二号
\newcommand{\sanhao}{\fontsize{15.75pt}{\baselineskip}\selectfont}  %三号
\newcommand{\sihao}{\fontsize{14pt}{\baselineskip}\selectfont}%     四号
\newcommand{\xiaosihao}{\fontsize{12pt}{\baselineskip}\selectfont}  %小四号
\newcommand{\wuhao}{\fontsize{10.5pt}{\baselineskip}\selectfont}    %五号
\newcommand{\xiaowuhao}{\fontsize{9pt}{\baselineskip}\selectfont}   %小五号
\newcommand{\liuhao}{\fontsize{7.875pt}{\baselineskip}\selectfont}  %六号
\newcommand{\qihao}{\fontsize{5.25pt}{\baselineskip}\selectfont}    %七号   %中文字体及字号设置
\xeCJKDeclareSubCJKBlock{SIP}{
	"20000 -> "2A6DF,   % CJK Unified Ideographs Extension B
	"2A700 -> "2B73F,   % CJK Unified Ideographs Extension C
	"2B740 -> "2B81F    % CJK Unified Ideographs Extension D
}
%\setCJKmainfont[SIP={[AutoFakeBold=1.8,Color=red]Sun-ExtB},BoldFont=黑体]{宋体}    % 衬线字体 缺省中文字体

\setCJKmainfont[Path=fonts/,
				SIP={[Path=fonts/,AutoFakeBold=1.8,Color=red]simsunb.ttf},
				BoldFont={simhei.ttf}]{simsun.ttc}
%SimSun-ExtB
%Sun-ExtB
%AutoFakeBold:自动伪粗,即正文使用\bfseries时生僻字使用伪粗体;
%FakeBold:强制伪粗,即正文中生僻字均使用伪粗体
%\setCJKmainfont[BoldFont=STHeiti,ItalicFont=STKaiti]{STSong}
%\setCJKsansfont{微软雅黑}黑体
%\setCJKsansfont[BoldFont=STHeiti]{STXihei} %serif是有衬线字体sans serif 无衬线字体
%\setCJKmonofont{STFangsong}    %中文等宽字体

%--------------------英文字体----------------------
\setmainfont[Path=fonts/,
			 BoldFont={simhei.ttf}]{simsun.ttc}
%\setmainfont[BoldFont=黑体]{宋体}  %缺省英文字体
%\setsansfont
%\setmonofont

%===================目录分栏设置====================
\usepackage[toc,lof,lot]{multitoc}      % 目录(含目录、表格目录、插图目录)分栏设置
	%\renewcommand*{\multicolumntoc}{3} % toc分栏数设置,默认为两栏(\multicolumnlof,\multicolumnlot)
	%\setlength{\columnsep}{1.5cm}      % 调整分栏间距
	\setlength{\columnseprule}{0.2pt}   % 调整分栏竖线的宽度

%==================章节格式设置====================
\setcounter{secnumdepth}{3} % 章节等编号深度 3:子子节\subsubsection
\setcounter{tocdepth}{2}    % 目录显示等度 2:子节

\xeCJKsetup{%
	CJKecglue=\hspace{0.15em},      % 调整中英(含数字)间的字间距
	%CJKmath=true,                  % 在数学环境中直接输出汉字(不需要\text{})
	AllowBreakBetweenPuncts=true,   % 允许标点中间断行,减少文字行溢出
}

\ctexset{%
	part={
		name={,篇},
		number=\SZX{part},
		format={\chuhao\bfseries\centering},
		nameformat={},titleformat={}
	},
	section={
		number={\chinese{section}},
		name={第,节}
	},
	subsection={
		number={\chinese{subsection}、},
		aftername={\hspace{-0.01em}}
	},
	subsubsection={
		number={(\chinese{subsubsection})},
		aftername={\hspace {-0.01em}},
		beforeskip={1.3ex minus .8ex},
		afterskip={1ex minus .6ex},
		indent={\parindent}
	},
	paragraph={
		beforeskip=.1\baselineskip,
		indent={\parindent}
	}
}

\newcommand*\SZX[1]{%
	\ifcase\value{#1}%
		\or 上%
		\or 中%
		\or 下%
	\fi
}

%====================页眉设置======================
\usepackage{titleps}%或者\usepackage{titlesec},titlesec包含titleps
\newpagestyle{special}[\small\sffamily]{
	%\setheadrule{.1pt}
	\headrule
	\sethead[\usepage][][\chaptertitle]
	{\chaptertitle}{}{\usepage}
}

\newpagestyle{main}[\small\sffamily]{
	\headrule
	%\sethead[\usepage][][第\thechapter 章\quad\chaptertitle]
%  {\thesection\quad\sectiontitle}{}{\usepage}}
	\sethead[\usepage][][第\chinese{chapter}章\quad\chaptertitle]
	{第\chinese{section}节\quad\sectiontitle}{}{\usepage}
}

\newpagestyle{main2}[\small\sffamily]{
	\headrule
	\sethead[\usepage][][第\chinese{chapter}章\quad\chaptertitle]
	{第\chinese{section}節\quad\sectiontitle}{}{\usepage}
}

%================ PDF 书签设置=====================
\usepackage[depth=2,        % 书签深度 2:子节
			open,           % 默认展开书签
			openlevel=2,    % 展开书签深度 2:子节
			numbered,       % 显示编号
			atend,
			]{bookmark}     % 相比hyperref,该宏包大多数时候只需要编译一次,
							% 而且书签的颜色和字体也可以定制。
							% 比hyperref 更专业 (自动加载hyperref)

%\bookmarksetup{italic,bold,color=blue} % 书签字体斜体/粗体/颜色设置

%------------重置每篇章计数器,必须在hyperref/bookmark之后------------
\makeatletter
	\@addtoreset{chapter}{part}
\makeatother

%------------hyperref 超链接设置------------------------
\hypersetup{%
	pdfencoding=auto,   % 解决新版ctex,引起hyperref UTF-16预警
	colorlinks=true,    % 注释掉此项则交叉引用为彩色边框true/false
	pdfborder=001,      % 注释掉此项则交叉引用为彩色边框
	citecolor=teal,
	linkcolor=myblue,
	urlcolor=black,
	%psdextra,          % 配合使用bookmark宏包,可以直接在pdf 书签中显示数学公式
}

%------------PDF 属性设置------------------------------
\hypersetup{%
	pdfkeywords={黄帝内经,内经,内经讲义,21世纪课程教材},    % 关键词
	%pdfsubject={latex},        % 主题
	pdfauthor={主编:王洪图}, % 作者
	pdftitle={内经讲义},        % 标题
	%pdfcreator={texlive2011}   % pdf创建器
}

%------------PDF 加密----------------------------------
%仅适用于xelatex引擎 基于xdvipdfmx
%\special{pdf:encrypt ownerpw (abc) userpw (xyz) length 128 perm 2052}

%仅适用于pdflatex引擎
%\usepackage[owner=Donald,user=Knuth,print=false]{pdfcrypt}

%其他可使用第三方工具 如:pdftk
%pdftk inputfile.pdf output outputfile.pdf encrypt_128bit owner_pw yourownerpw user_pw youruserpw

%=============自定义环境、列表及列表设置================
\def\biaoti#1{\vspace{1.7ex plus 3ex minus .2ex}{\bfseries #1}}%\noindent\hei
\def\xiaobt#1{{\bfseries #1}}
\def\xiaojie {\vspace{1.8ex plus .3ex minus .3ex}\centerline{\large\bfseries 小\ \ 结}\vspace{.1\baselineskip}}
\def\zuozhe#1{\rightline{\bfseries #1}}

\newcounter{yuanwen}    % 新计数器 yuanwen
\newcounter{jiaozhu}    % 新计数器 jiaozhu

\newenvironment{yuanwen}[2][【原文】]{%
	%\biaoti{#1}\par
	\stepcounter{yuanwen}   % 计数器 yuanwen+1
	\bfseries #2}
	{}

\usepackage{enumitem}
\newenvironment{jiaozhu}[1][【校注】]{%
	%\biaoti{#1}\par
	\stepcounter{jiaozhu}   % 计数器 jiaozhu+1
	\begin{enumerate}[%
					label=\mylabel{\arabic*}{\circledctr*},before=\small,fullwidth,%
					itemindent=\parindent,listparindent=\parindent,%labelsep=-1pt,%labelwidth=0em,
					itemsep=0pt,topsep=0pt,partopsep=0pt,parsep=0pt]}
	{\end{enumerate}}

%===================注解与原文相互跳转====================
%----------------第1部分 设置相互跳转锚点-----------------
\makeatletter
	\protected\def\mylabel#1#2{% 注解-->原文
		\hyperlink{back:\theyuanwen:#1}{\Hy@raisedlink{\hypertarget{\thejiaozhu:#1}{}}#2}}

	\protected\def\myref#1#2{% 原文-->注解
		\hyperlink{\theyuanwen:#1}{\Hy@raisedlink{\hypertarget{back:\theyuanwen:#1}{}}#2}}
	%此处\theyuanwen:#1实际指thejiaozhu:#1,只是\thejiaozhu计数器还没更新,故使用\theyuanwen计数器代替
\makeatother

\protected\def\myjzref#1{% 脚注中的引用(引用到原文)
	\hyperlink{\theyuanwen:#1}{\circlednum{#1}}}

\def\sb#1{\myref{#1}{\textsuperscript{\circlednum{#1}}}}    % 带圈数字上标

%----------------第2部分 调整锚点垂直距离-----------------
\def\HyperRaiseLinkDefault{.8\baselineskip} %调整锚点垂直距离
%\let\oldhypertarget\hypertarget
%\makeatletter
%   \def\hypertarget#1#2{\Hy@raisedlink{\oldhypertarget{#1}{#2}}}
%\makeatother

%====================带圈数字列表标头====================
%\newfontfamily\circledfont[Path = fonts/]{meiryo.ttc}  % 日文字体,明瞭体
\newfontfamily\circledfont{Meiryo}  % 日文字体,明瞭体

\protected\def\circlednum#1{{\makexeCJKinactive\circledfont\textcircled{#1}}}

\newcommand*\circledctr[1]{%
	\expandafter\circlednum\expandafter{\number\value{#1}}}
\AddEnumerateCounter*\circledctr\circlednum{1}

% 参考自:http://bbs.ctex.org/forum.php?mod=redirect&goto=findpost&ptid=78709&pid=460496&fromuid=40353

%======================插图/tikz图========================
\usepackage{graphicx,subcaption,wrapfig}    % 图,subcaption含子图功能代替subfig,图文混排
	\graphicspath{{img/}}                   % 设置图片文件路径

\def\pgfsysdriver{pgfsys-xetex.def}         % 设置tikz的驱动引擎
\usepackage{tikz}
	\usetikzlibrary{calc,decorations.text,arrows,positioning}

%---------设置tikz图片默认格式(字号、行间距、单元格高度)-------
\let\oldtikzpicture\tikzpicture
\renewcommand{\tikzpicture}{%
	\small
	\renewcommand{\baselinestretch}{0.2}
	\linespread{0.2}
	\oldtikzpicture
}

%=========================表格相关===============================
\usepackage{%
	multirow,                   % 单元格纵向合并
	array,makecell,longtable,   % 表格功能加强,tabu的依赖
	tabu-last-fix,              % "强大的表格工具" 本地修复版
	diagbox,                    % 表头斜线
	threeparttable,             % 表格内脚注(需打补丁支持tabu,longtabu)
}

%----------给threeparttable打补丁用于tabu,longtabu--------------
%解决方案来自:http://bbs.ctex.org/forum.php?mod=redirect&goto=findpost&ptid=80318&pid=467217&fromuid=40353
\usepackage{xpatch}

\makeatletter
	\chardef\TPT@@@asteriskcatcode=\catcode`*
	\catcode`*=11
	\xpatchcmd{\threeparttable}
		{\TPT@hookin{tabular}}
		{\TPT@hookin{tabular}\TPT@hookin{tabu}}
		{}{}
	\catcode`*=\TPT@@@asteriskcatcode
\makeatother

%------------设置表格默认格式(字号、行间距、单元格高度)------------
\let\oldtabular\tabular
\renewcommand{\tabular}{%
	\renewcommand\baselinestretch{0.9}\small    % 设置行间距和字号
	\renewcommand\arraystretch{1.5}             % 调整单元格高度
	%\renewcommand\multirowsetup{\centering}
	\oldtabular
}
%设置行间距,且必须放在字号设置前 否则无效
%或者使用\fontsize{<size>}{<baseline>}\selectfont 同时设置字号和行间距

\let\oldtabu\tabu
\renewcommand{\tabu}{%
	\renewcommand\baselinestretch{0.9}\small    % 设置行间距和字号
	\renewcommand\arraystretch{1.8}             % 调整单元格高度
	%\renewcommand\multirowsetup{\centering}
	\oldtabu
}

%------------模仿booktabs宏包的三线宽度设置---------------
\def\toprule   {\Xhline{.08em}}
\def\midrule   {\Xhline{.05em}}
\def\bottomrule{\Xhline{.08em}}
%-------------------------------------
%\setlength{\arrayrulewidth}{2pt} 设定表格中所有边框的线宽为同样的值
%\Xhline{} \Xcline{}分别设定表格中水平线的宽度 makecell包提供

%表格中垂直线的宽度可以通过在表格导言区(preamble),利用命令 !{\vrule width1.2pt} 替换 | 即可

%=================图表设置===============================
%---------------图表标号设置-----------------------------
\renewcommand\thefigure{\arabic{section}-\arabic{figure}}
\renewcommand\thetable {\arabic{section}-\arabic{table}}

\usepackage{caption}
	\captionsetup{font=small,}
	\captionsetup[table] {labelfont=bf,textfont=bf,belowskip=3pt,aboveskip=0pt} %仅表格 top
	\captionsetup[figure]{belowskip=0pt,aboveskip=3pt}  %仅图片 below

%\setlength{\abovecaptionskip}{3pt}
%\setlength{\belowcaptionskip}{3pt} %图、表题目上下的间距
\setlength{\intextsep}   {5pt}  %浮动体和正文间的距离
\setlength{\textfloatsep}{5pt}

%====================全文水印==========================
%解决方案来自:
%http://bbs.ctex.org/forum.php?mod=redirect&goto=findpost&ptid=79190&pid=462496&fromuid=40353
%https://zhuanlan.zhihu.com/p/19734756?columnSlug=LaTeX
\usepackage{eso-pic}

%eso-pic中\AtPageCenter有点水平偏右
\renewcommand\AtPageCenter[1]{\parbox[b][\paperheight]{\paperwidth}{\vfill\centering#1\vfill}}

\newcommand{\watermark}[3]{%
	\AddToShipoutPictureBG{%
		\AtPageCenter{%
			\tikz\node[%
				overlay,
				text=red!50,
				%font=\sffamily\bfseries,
				rotate=#1,
				scale=#2]
				{#3};
		}
	}
}

\newcommand{\watermarkoff}{\ClearShipoutPictureBG}

\watermark{45}{15}{草\ 稿}    %启用全文水印

%=============花括号分支结构图=========================
\usepackage{schemata}

\xpatchcmd{\schema}
	{1.44265ex}{-1ex}
	{}{}

\newcommand\SC[2] {\schema{\schemabox{#1}}{\schemabox{#2}}}
\newcommand\SCh[4]{\Schema{#1}{#2}{\schemabox{#3}}{\schemabox{#4}}}

%=======================================================

\begin{document}
\pagestyle{main2}
\fi
\chapter{藏象}%第二章

藏象是研究脏腑经脉形体官窍的形态结构、生理活动规律及其相互关系的理论,它是《内经》理论体系的核心,也是中医基础理论的基本内容之一。这里所说的形态结构,是指《内经》中的解剖知识;生理活动规律是该理论的重点,它是以五脏为中心、联系诸腑、经脉、形体、官窍等的肝、心、脾、肺、肾五个系统的生理活动。这五个系统不仅都受天地四时阴阳的影响,同时互相之间也紧密联系,从而体现人体局部与整体的生理活动规律。

藏象理论的内容有:五脏、六腑、奇恒之腑、经脉、形体、官窍、精气神。《内经》专论藏象的篇章,《素问》有灵兰秘典论、六节藏象论、五脏生成论、五脏别论、经脉别论、太阴阳明论、解精微论;《灵枢》有本神、骨度、五十营、营气、脉度、营卫生会、决气、肠胃、平人绝谷、海论、五癱津液别、本脏、天年、阴阳二十五人、邪客、通天、卫气行、大惑论等篇。此外,经脉虽也属藏象的内容,但因其具有相对独立性,故另列专章讨论。

\section{素問·靈蘭秘典論}%第一節

\biaoti{【原文】}

\begin{yuanwen}
黃帝問曰:願聞十二藏之相使貴賤\sb{1}何如?岐伯對曰:悉乎哉問也,請遂言之。心者,君主之官\sb{2}也,神明出焉。肺者,相傳之官,治節\sb{3}出焉。肝者,將軍之官,謀慮出焉。膽者,中正之官\sb{4},決斷出焉。膻中者,臣使之官,喜樂出焉。脾胃者,倉廩之官,五味出焉。大腸者,傳道之官,變化出焉。小腸者,受盛之官,化物出焉。腎者,作強之官\sb{5},伎巧出焉。三焦者,決瀆之官,水道出焉。膀胱者,州都\sb{6}之官,津液藏焉,氣化\sb{7}則能出矣。凡此十二官者,不得相失也。故主明則下安,以此養生則壽,歿世不殆,以爲天下則大昌。主不明則十二官危,使道\sb{8}閉塞而不通,形乃大傷,以此養生則殃,以爲天下者,其宗大危,戒之戒之!
\end{yuanwen}

\biaoti{【校注】}

\begin{jiaozhu}
  \item 相使贵贱:相使,相互为用、相互协调的关系。贵贱,即主从。
  \item 君主之官:君主,古代国家的最高统治者。官,即指脏,在此与变文。又人体中具有某些特定功能的部分(器官)也称为官,如《灵枢·五阅五使》把口、耳、鼻、舌、目称为官。心脏统领脏腑百骸,故以君主之官喻之。张介宾注云:“心为一身之君主,禀灵虚而含造化,具一理以应万几,脏腑百骸惟所是命,聪明智慧,莫不由之”。
  \item 治节:治理调节。肺主气而朝百脉,有辅助心脏而治理和调节脏腑气血的功能。张介宾云:“肺主气,气调则营卫脏腑无所不治,故曰治节出焉”。
  \item 中正之官:中正,正直无私,不偏不倚。胆司勇怯而主决断,故称中正之官。
  \item 作强之官:作强,功能作用强大。肾藏精而舍志、主骨生髓、主司生育,其功能强大,故称作强之官。
  \item 州都:州,水中陆地。都,水泽所聚之意。州都,即水液集聚之处。膀胱位居下焦,是三焦水液所归之地,故称之为州都之官。
  \item 气化:此指以肾中阳气为主的心、肝、肺、脾(胃)、肾、三焦等诸脏腑之气,对膀胱所藏津液的蒸化作用。
  \item 使道:即指十二经脉。是十二脏腑相使之道,也就是气血往来、脏腑相互联系的通道。
\end{jiaozhu}

\biaoti{【理论阐释】}

1.十二脏相使说

用十二官相使作比喻,是《内经》论证和解释藏象理论所采用的方法之一,除本篇外,他如《灵枢·邪客》:“心者,五脏六腑之大主也。”《灵枢·五癃津液别》:“五脏六腑,心为之主……肺为之相,肝为之将,脾为之卫,肾为之主外。”《素问·本病论》:“心为君主之官”、“脾为谏议之官”、“肝为将军之官”等,均采用君臣作比喻,来说明脏腑间的相互关系及各自的功能特点,而以《灵兰秘典论》最有代表性。

各脏腑在整体关系中的主从地位,是由其主要功能决定的,心之所以为君主,是因其主神明;肺之所以为相傅,是因其主治节;肝之所以为将军,是因其主谋虑……膀胱之所以为州都之官,是因其藏津液。不过,本文为了说明十二脏腑的相互关系及其各自的功能特点,举有十二官之名称,但在古代朝廷里并无作强、传道等官职,说明比喻方法的采用,是以能够反映医学理论为原则的。

用“十二脏相使”作比喻,不仅可以说明人体是一个相对独立的王国(整体),同时也形象地解释了各脏腑之间相互协调的主从关系,并具体地体现出《内经》从整体角度认识生命规律的学术特点。诸脏腑之间相互为用、相互协调(含主从),形成为统一的有机体,这是藏象理论能够成立的前提之一,也是藏象理论用以分析和认识医学问題的基本观点。正如本文所说,正常人体的生命活动,即是“十二官不得相失”。如“主明则下安”,十二官协调,而能“养生则寿”。反之,若“主不明则十二官危”,脏腑功能紊乱,则“使道闭塞而不通,形乃大伤,以此养生则殃。”

由于本文不仅强调了十二官相使协调的整体性,而且概括地提出了各脏腑的主要功能,因而其在藏象理论中占有重要的地位,其论述对中医理论以及临床医学的发展,具有重要影响。

2.对“心为君主之官”的认识

在十二脏腑整体协调关系中,心脏为什么是“君主之官”?

首先,因为心藏神,而神在人体生命活动中具有主宰或统领的作用。《内经》关于心与神关系的记载十分明确,除本篇外,他如《素问·宣明五气篇》、《素问·调经论》、《灵枢·九针论》等均有“心藏神”的记载。《素问·六节藏象论》云:“心者,生之本,神之变也。”《灵枢·邪客》谓:“心者,五脏六腑之大主也,精神之所舍也。”《灵枢·口问》云:“悲哀愁忧则心动,心动则五脏六腑皆摇。”反映出“心藏神”是藏象理论的重要内容。

“形神合一”是人体生命存在的基础,这一观点在《内经》中也是十分明确的,如《灵枢·天年》说:“血气己和,荣卫已通,五脏已成,神气舍心,魂魄毕具,乃成为人。”《素问·上古天真论》云:“能形与神俱,而尽终其天年。”说明形神兼备才是一个完美而健康的人。但是,就神与形两者的关系而言,气血有形之物(相对神而言)可以化生神,脏腑形骸又是寓神之宅,因而可以说“形为神之体”;而神则有支配和驾驭形体因而保护形体的作用,故可以称“神为形之主”。所以《灵枢·天年》说:“失神者死,得神者生”。《素问·五常政大论》谓:“神去则机息”。都是强调神在人体生命活动中的主宰作用。本文所论“主明则下安,以此养生则寿”,“主不明则十二官危……以此养生则殃”。同样是以心脏“神明出焉”为前提的。又如《灵枢·本脏》所云:“志意者,所以御精神,收魂魄,适寒温,和喜怒者也……志意和,则精神专直,魂魄不散,悔怒不起,五脏不受邪矣。”说明神(志意)是人体适应气候变化、缓冲各种因素刺激引起的情绪波动,从而成为维护内脏功能、身体健康的可靠保证。

其次因为心主血脉《素问·痿论》谓:“心主身之血脉”。而血是奉养精神与形体最为重要的物质,正如《灵枢·营卫生会》所说:“中焦……此所受气者,泌糟粕,蒸津液,化其精微,上注于肺脉,乃化而为血,以奉生身,莫贵于此。”而经脉则是运行气血通道,本篇(《灵兰秘典论》)称为使道。《灵枢·本脏》谓:“经脉者,所以行血气而营阴阳,濡筋骨,利关节者也。”同时,就心脏而言,其主血脉的功能也是其主藏神功能的基础,即如《灵枢·本神》所说:“心藏脉,脉舍神”。

由于心神对全身生命活动的统领作用,以及心主之血脉对脏腑百骸的奉养作用,此两项功能活动是人体生命的根本所在,因而称心为君主之官。

\biaoti{【临证指要】}

\xiaobt{心为君主,不宜受邪}

“心为君主之官”的理论,对中医理论研究和临床实践都有深远的影响。从理论而言,由于心为君主,便不能遭受邪气的直接侵害,否则犹如国家无主,必然危殆。因而产生“心包代心用事”的相关理论,并用来指导临床实践。《灵枢·邪客》云:“心者,五脏六腑之大主也,……邪弗能容,容之则心伤,心伤则神去,神去则死矣。故诸邪之在于心者,皆在于心之包络。”心包络,亦即膻中,为臣使之官。《灵枢·胀论》又谓:“膻中者,心主之宫城也。”故心包有保护心脏,代心受邪的作用。叶桂据此理论在辨证温热病时,有“温邪上受,首先犯肺,逆传心包”之语(《临证指南医案·温热论》);吴瑭据此理论辨证制方,如对热扰神明之证云:“神昏谵语者,清宫汤主之”。自注云:“心阴不能济阳,则心阳独亢,心主言,故谵语不休也。”方论云:“此酸寒甘苦法,清膻中之方也。谓之清宫者,以膻中为心主之宫城也。”(《温病条辨·上焦篇》第16条)。又云:“邪入心包,舌蹇肢厥,牛黄丸主之,紫雪丹亦主之。”注云:“盖舌为心之窍,包络代心用事”。(《温病条辨·上焦篇》第17条)。邪气扰心,用方则称清宫汤,其理论依据正是“包络代心用事”。

其次,根据心神得失,可以诊断病势轻重和推断预后吉凶。《素问·移精变气论》云:“闭户塞牖,系之病者,数问其情,以从其意。得神者昌,失神者亡。”《灵神·本神》谓:“用针者,察观病人之态,以知精神魂魄之存亡得失之意。”均指出医生诊治疾病过程中,必须观察和了解患者“神”的得失存亡,从而判断疗效有无和预后吉凶。若患者神气尚存,则病易愈;若神气已失,则难以治愈而预后不良。若神志昏乱,则说明病势危笃,必须高度警惕。《素问·汤液醪醴论》云:“神去之而病不愈也”。《素问·评热病论》以“失志”作为阴阳交“三死”症之一,都说明在诊治过程中观察神的重要性。

第三、治病过程中不仅不能伤害心脏,还要时刻注意防止病气传入心脏。《素问·刺禁论》:“刺中心,一日死”。告诫医生治病时,当知所禁,不可伤及心脏。《临征指南医案·温热论》云:“若舌绛而干燥者,火邪劫营,凉血清火为要。……大红点者,热毒乘心也,用黄连、金汁。”叶氏此论,意在防止热邪深入营血而伤心神。

\biaoti{【原文】}

\begin{yuanwen}
至道在微,變化無窮,孰知其原!窘乎哉,消者瞿瞿\sb{1},孰知其要!閔閔之當\sb{2},孰者爲良!恍惚之數\sb{3},生於毫釐,毫釐之數,起於度量,千之萬之,可以益大,推之大之,其形乃制。黃帝曰:善哉,余聞精光大道,大聖之業,而宣明大道,非齋戒擇吉日,不敢受也。黃帝乃擇吉日良兆,而藏之靈蘭之室,以傳保焉。
\end{yuanwen}

\biaoti{【校注】}

\begin{jiaozhu}
  \item 消者瞿瞿:消,新校正云:“《太素》作肖者濯”。《素问·气交变大论》作“肖者瞿瞿。”肖者,指有学问、有道德的人。瞿瞿,勤谨的样子。全句谓有学问的人勤勉地研究医学理论。
  \item 闵闵之当:闵闵,深远也。当,事理合宜之意,即妥当、适当。此句言道理深奥而正确。
  \item 恍惚之数:王冰注云:“恍惚者,谓似有似无也。忽,亦数也,似无似有,而毫厘之数生其中。《老子》曰:恍恍惚惚,其中有物。此之谓也。”
\end{jiaozhu}

\biaoti{【理论阐释】}

\xiaobt{关于医学探原的方法}

本文“至道在微,变化无穷,孰知其原。”提出了如何探索医学理论和发展医学科学的问题。指出:第一,医学科学是可知的。尽管医学道理奥妙隐微,人体生命规律变化无穷,但是通过有学问的人们勤奋钻研,终究可以“知其要”、“知其原”。进而使医学科学不断进步,并且会越来越完善和优良;第二,从“数”出发是研究医学的重要方法之一。研究者不仅要详细了解和通晓推引至大、可以尺度斗量的人形制度,更要研究产生人形制度的内在规律。无论人体构造多么复杂,生命过程多么奥妙,但其基本规律无非是从隐微到显著的过程,都会是有“数”可察的。换言之,医学科学是可以引人数学模式的。在《内经》中,已将若干问题运用“数”进行了论证。如关于天体运动规律的解释、关于三阴三阳以及脏腑经脉之数、关于一呼一吸气行尺寸、关于百分之一日的生命节律等。后世医家在理论和临床方面,也都有一定的发挥和发现。如杨上善所撰《黄帝内经太素》,其注文颇重视“数”的计算与推导;灵龟八法、子午流注针法,同样是在经脉气血运行理论基础上发展而来。

\section{素問·六節藏象論(節選)}%第二節

\biaoti{【原文】}

\begin{yuanwen}
帝曰:善。余聞氣合而有形,因變以正名\sb{1}。天地之運,陰陽之化,其於萬物,孰少孰多,可得聞乎?岐伯曰:悉哉問也。天至廣不可度,地至大不可量,大神靈問,請陳其方。草生五色,五色之變,不可勝視,草生五味,五味之美,不可勝極。嗜欲不同,各有所通\sb{2}。天食人以五氣\sb{3},地食人以五味。五氣入鼻,藏於心肺\sb{4},上使五色修明,音聲能彰。五入口,藏於腸胃,味有所藏,以養五氣\sb{5},氣和而生,津液相成,神乃自生\sb{6}。

帝曰:藏象\sb{7}何如?岐伯曰:心者,生之本\sb{8},神之變\sb{9}也,其華在面,其充在血脈,爲陽中之太陽,通於夏氣。肺者,氣之本,魄\sb{10}之處也,其華在毛,其充在皮,爲陽中之太陰\sb{11},通於秋氣。腎者,主蛰,封藏之本\sb{12},精之處也,其華在髮,其充在骨,爲陰中之少陰\sb{13},通於冬氣。肝者,罷極之本\sb{14},魂\sb{15}之居也,其華在爪,其充在筋,以生血氣,其味酸,其色蒼\sb{16},此爲陽中之少陽\sb{17},通於春氣。脾、胃、大腸、小腸、三焦、膀胱者,倉廪之本,營之居也,名曰器,能化糟粕,轉味而入出者也,其華在唇四白,其充在肌,其味甘,其色黃\sb{18},此至陰之類,通於土氣\sb{19}。凡十一藏,取決於膽也\sb{20}。
\end{yuanwen}

\biaoti{【校注】}

\begin{jiaozhu}
  \item 气合而有形,因变以正名:变,变异、变化之意。正名,定正名称,即准确命名。阴阳二气相合而成万物,但诸物所禀阴阳之气有多少之异,根据其阴阳多少而定名称,是对事物命名的基本原则之一,故有阴阳太少等不同名称。
  \item 嗜欲不同,各有所通;诸物之性不一,因而嗜欲不同,各有所通。如味酸青色入于肝,而肝气通于春;赤色苦味入于心,而心气通于夏之类。
  \item 天食人以五气:食,同饲,供养之意。五气,即指寒、暑、燥、湿、风五气,亦即正常气候。《灵枢·九宫八风》:“风从其所居之乡来为实风,主生,长养万物”。
  \item 五气入鼻,藏于心肺:五气经鼻吸入,藏于上焦胸中,上焦为心肺所居。心肺功能正常才能吸纳五气,并运达周身。
  \item 味有所藏,以养五气:味,饮食五味所化生的精微之气。五气,指五脏之气。五脏赖水谷精微的滋养,以维持其功能活动。
  \item 气和而生……神乃自生:张介宾《类经·气味类》注云:“胃藏五味,以养五脏之气,而化生津液以成精,精气充而神自生。”
  \item 藏象:张介宾云:“象,形象也。脏居于内,形见于外,故曰藏象。”此文字表面含义,作为《内经》理论的核心,藏象是研究脏腑经脉形体官窍的形态结构、生理活动规律及其相互关系的理论。尤以生理活动规律是该理论的重点。
  \item 生之本:本,主体、主持。生之本,即指生命活动的主宰。
  \item 神之变:新校正:“详神之变,全元起本并《太素》作神之处。”心藏神,故为神之处。
  \item 魄:五神之一,主要包括痛痒等感觉、本能的动作等。
  \item 阳中之太阴:新校正云:“按太阴《甲乙经》并《太素》作少阴。当作少阴,肺在十二经虽为太阴,然在阳分之中,当为少阴也。”
  \item 封藏之本:肾应冬藏之气,其脏所藏之精为人体生长、发育和生殖的物质基础,宜藏不宜泄,故称肾为封藏之本。
  \item 阴中之少阴:新校正:“按全元起本并《甲乙经》、《太素》少阴作太阴。当作太阴,肾在十二经虽为少阴,然在阴分之中,当为太阴。”
  \item 罢极之本:罢,通疲,怠惰、松弛。极,通亟,紧急、急迫。肝主筋,主司运动,筋收缩则紧张有力,筋弛缓则乏力松弛。二者交替进行,便产生肢体运动。王冰注云:“夫人之运动者,皆筋力之所为也,肝主筋……故曰肝者罢极之本。”
  \item 魂:五神之一,藏于肝血之中,主要包括谋虑、梦幻以及志怒、惊恐之类的情感活动。
  \item 其味酸,其色苍:新校正云:“详此六字当去。……今惟肝脾二脏载其味其色,据《阴阳应象大论》已著色味详矣,此不当出之。今更不添心肺肾三脏之色味,只去肝脾二脏之色味可矣。”
  \item 阳中之少阳:新校正:“按全无起本并《甲乙经》、《太素》作阴中之少阳,当作阴中之少阳。”又《灵枢·阴阳系日月》云:“肝为阴中之少阳。”当从之。
  \item 其味甘,其色黄:六字当删,理见前校注\myjzref{16}。
  \item 此至阴之类,通于土气:《灵枢·阴阳系日月》:“脾为阴中之至阴”。胃、大肠、小肠、三焦、膀胱五腑,皆有传化水谷功能,总统于脾,故称“至阴之类”。
  \item 凡十一脏,取决于胆也:胆为少阳春生之气,春气生则万化安,故云“十一脏取决于胆”。另《黄帝内经研究大成·藏象研究》认为“十一”当校作“土”字,传写之误。“土脏”即指脾脏及胃肠三焦膀胱等传化五腑。决,通也,泄也。胆气疏泻,通降于土脏,土脏则能运化调畅。故言“凡土脏取决于胆也”。
\end{jiaozhu}

\biaoti{【理论阐释】}

1.“四时五脏阴阳”——研究藏象的方法

“四时五脏阴阳”一语,见于《素问·经脉别论》,是《内经》分析和认识医学理论的重要方法之一。本篇正是运用该方法来论证和说明藏象理论,其与《素问·灵兰秘典论》:“十二脏相使”比喻方法并列,而成为《内经》论证藏象问题的两种主要方法。与直接观察脏腑功能活动和对形体解剖度量的方法相比较,该两种方法更具有理论意义,因而也更能深刻地反映人体生命活动的本质。该两种方法的共冋点在于,都是从整体的观点出发;其区别则是“十二脏相使”突出强调人体是一个有机整体的观点,而“四时五脏阴阳”说,则不但从人体本身整体性方面,更着重于将人体与生活环境的四时阴阳作为一个整体来看待。

人体本身以及人与生活环境之所以能够协调统一,而形成为相互联系的整体,本文提出的根据是“嗜欲不同,各有所通”。无论人体还是自然界的万事万物,其各处所、部位(空间)、各季节、时令(时间),所含阴阳多少各有差异,因其差异而嗜欲不同。其嗜欲相应者,便可相通。心脏所含“阳”最多,而为阳中之太阳,便与神、血脉、面、夏气相通;肺脏所含之“阳”略少于心脏,并含较少之“阴”,而为阳中之少阴,便与气、毛、皮、秋气相通……。

人禀“天地之运,阴阳之化”而生,要想从本质上深入地认识人的身体,尤其是十分复杂的生命活动规律,就必须从空间和时间等各方面进行研究,这就是本篇所提示的研究和认识藏象问题的基本方法。

2.本篇确定脏腑阴阳的标准

本篇提出对事物命名的方法之一是“气合面有形,因变以正名”。即根据事物各自的阴阳多少强弱,来确定其名称。人身脏腑同样可以用阴阳来命名,而确定每个脏腑的阴阳多少强弱,有两条可作为标准。一是凡部位在上(胸中)者为阳,而在下(腹中)者便为阴。如心肺二脏在胸中,故为阳,而肝、肾、脾三脏在腹中,则为阴。胃、大肠、小肠、三焦、膀胱等传化五腑,也在腹中,故也属于阴之类。二是根据脏腑的功能特性划分,凡具有升发之性者属阳,而具有潜降之性的属阴;其性质强大的便为“太”,相对较弱的便为“少”。如心主火,其性上炎,其势盛,故称太阳;肝也主生发,有上升之性,但其势温和柔弱,故当为少阳。肾主封藏,作用强大,故称太阴;而肺主敛降,虽也有潜降之性,但其潜降作用较肾脏为弱,故当称少阴。脾脏以及传化五腑,能受纳水谷,运化精微,排出糟粕,又是五脏气机升降的枢纽,含升降于其中,故属至阴。至,始至、到达,而非至极之意。

如此,则心为阳中之太阳,肺为阳中之少阴,肝为阴中之少阳,肾为阴中之太阴,脾为阴中之至阴。胃、肠、三焦、膀胱为“至阴之类”。新校正对本篇有关内容的校勘,也反映了确定脏腑阴阳的该两条标准。而且《灵枢·阴阳系日月》之文更足以为证,云:“其于五脏也,心为阳中之太阳,肺为阳中之少阴,肝为阴中之少阳,脾为阴中之至阴,肾为阴中之太阴”。

\biaoti{【临证指要】}

\xiaobt{藏象是诊治疾病的基本理论}

藏象一词,在《内经》中仅见于本篇的篇名和“藏象何如”句中,作为理论,它在《内经》理论体系中处于核心的位置,因而该理论贯穿于《内经》的全部理论之中。它从“天地之运,阴阳之化”是万物产生的基础的观点出发,而将天、地、人作为一个整体看待;以五脏为中心,整体功能协调统一,作为人体生命活动的基本规律。这些基本观点和在其指导下形成的相关理论,便是中医临床诊治疾病时最常用的理论。人能与自然环境相适应,并做到各脏腑功能协调统一,即“十二脏之相使”,便可使身体保持健康,因而这就是养生防病的基本条件,所谓“以此养生则寿”;从临床辨证而言,脏腑辨证几乎是适用于全部内伤杂病和部分外感性疾病的辨证方法,即使不以脏腑命名的某些辨证方法,如气血津液辨证,经脉辨证等,究其气血、经脉实质,仍属于藏象理论的内容,因而仍然是藏象理论在辨证方面的具体运用;明确病性、病势、病位是诊断的主要内容,其中病位是关键,是病性、病势存在的前提和物质基础。但病位却离不开脏腑、经脉、气血、形体、官窍等藏象所包含的内容与范围;正确诊断是实施治疗的必备条件,只有诊断准确无误,才能保证治疗做到有的放矢。

例如:采用补法,必先知病证为虚,又知其虚在何处(病位);欲用清法,同样必先知其证为热,更需明确其热在何方。

虽然本篇所论与《内经》全部有关藏象记载一样,主要是论证生理问题,但由于“知常达变”,因而可以提示医生有关诊治各方面的基本认识。如因为心为“生之本,神之变(处)”、“通于夏气”,故遇有心神疾病,尤应谨慎,以防伤害生命之本。同时应该注意,心火炽盛之人,遇到夏季,其病有转危的可能。又知,肾脏为“封藏之本,精之处也”,“通于冬气”。因而精气不足和遗泻是肾病常见证候,坚肾固精则是治法之一。若肾中阳气衰弱患者,其病情于冬季亦有加重危险。

\section{素問·五藏別論}%第三節

\biaoti{【原文】}

\begin{yuanwen}
黃帝問曰:余聞方士\sb{1},或以腦髓爲藏,或以腸胃爲藏,或以爲府,敢間更相反,皆自謂是,不知其道,願聞其說。岐伯對曰:腦、髓、骨、脈、膽、女子胞,此六者,地氣之所生也\sb{2},皆藏於陰而象於地\sb{3},故藏而不寫,名曰奇恒之\sb{4}。夫胃、大腸、小腸、三焦、膀胱,此五者,天氣之所生也\sb{5},其氣象天\sb{6},故寫而不藏,此受五藏濁氣,名曰傳化之府,此不能久留,輸寫者也\sb{7}。魄門亦爲五藏使\sb{8},水榖不得久藏。所謂五藏者,藏精氣\sb{9}而不寫也,故滿而不能實\sb{10}。六府者,傳化物\sb{11}而不藏,故實而不能滿也。所以然者,水穀入口,則胃實而腸虛;食下,則腸實而胃虚。故曰實而不能滿,滿而不能實也。
\end{yuanwen}

\biaoti{【校注】}

\begin{jiaozhu}
  \item 方士:通晓方术的人。此指医生。
  \item 地气之所生也:地气,阴气。地气之所生,即禀受于阴、其性属阴之意。
  \item 藏于阴而象于地:两“于”字为音节助词,无义。藏阴,藏蓄阴精。象地,象大地载藏万物。《素问·五运行大论》:“地者,所以载生成之形类也。”
  \item 奇恒之腑:奇,异也。恒,常也。奇恒之腑,即异于恒常之腑,言其与胃肠诸腑不同。王冰注:“脑、髓、骨、脉虽名为腑,不正与神脏为表里。胆与肝合,而不同六腑之传泻。胞虽出纳,纳则受纳精气,出则化出形容,形容之出谓化极而生。然出纳之用有殊于六腑,故言藏而不泻,名曰奇恒之腑也。”
  \item 天气之所生:天气,阳气。天气之所生,即禀受于阳,其性属阳之意。
  \item 其气象天:其性象天气运动不息。《素问·天元纪大论》:“应天之气,动而不息”。
  \item 此不能久留,输泻者也:传化五腑以泻而不藏为其特点,同时也有输送水谷精微、布散津液的作用,故称输泻。《灵枢·经水》谓:“六腑者,受谷而行之,受气而扬之”。
  \item 魄门亦为五脏使:魄,通粕。魄门,即指肛门,因其能排出糟粕而得名。·使,使役、使用之意。为五脏使,一指魄门受五脏支配而启闭;二是肛门能正常排出糟柏与浊气,有利于五脏气机的升降出入。
  \item 藏精气:新校正云:“按全元起本及《甲乙经》、《太素》‘精气’作‘精神’。”作“藏精神”义长。《灵枢·本脏》云:“五脏者,所以藏精神血气魂魄者也。”《灵枢·本神》亦有每一脏所藏之“精”,所藏之“神”的具体记载。
  \item 满而不能实:本篇“满”、“实”二字有特定含义。满,指精神气血等充盈于五脏;实,指水谷及其消化剩余物停留于六腑。王冰注云:“精气为满,水谷为实,但藏精气,故满而不能实”。
  \item 传化物:传,运榆、传送。化物,此指水谷消化后的产物,包括食糜、精气、水液、残渣、大便等。传化物,是对六腑功能的概括,意同《六节藏象论》:“能化糟粕,转味而入出者也。”
\end{jiaozhu}

\biaoti{【理论阐释】}

1.脏腑藏泻的理论意义

五脏和六腑是藏象理论的最主要内容,因此如何确定或划分什么是脏、什么是腑,是一个十分重要的理论问题。否则,“或以脑髓为脏,或以肠胃为脏,或以为腑”,众说纷纭,必将引起学术混乱,便不能形成完善的医学理论,障碍医学科学的发展。以什么为标准来区别脏与腑?由于藏象理论是以讨论“生理活动规律”为重点,因此找出脏与腑“生理活动”各自的共性,便可以妥善地将两者区分开来。本文明确指出五脏的共性是“藏精气而不泻”,六腑的共性是“传化物而不藏”。即脏主藏、腑主泻。这是中医学理论把内脏区分为脏与腑的主要依据,也是贯穿于藏象理论中的一个基本观点。

当然,藏与泻仅是就脏腑的功能特点而言,而在五脏主藏的同时,亦有泻。其在生命活动过程中也产生浊气,其浊气主要输于六腑,经腑排出体外,即经文所谓六腑“此受五脏浊气”。又五脏所藏精气,也会有适当的“泻”,如肝气疏泻、肝血输于经脉;肾藏精,而《素问·上古天真论》谓男子“二八肾气盛,天癸至,精气溢泻,阴阳和,故能有子”;六腑虽以“传化物而不藏”为主,但化物中有精气,其精气输于脏而藏之。又腑之泻中亦有藏,而不可泻之无度,仅是其藏、其留“不能久”而已。诚如姚止庵《素问经注节解》云:“精气主贵而难实,化物至秽而不可久留,其藏其泻,真造化自然之妙用乎!”

2.对“奇恒之腑”的评价

奇恒之腑是藏象理论的内容之一,该理论在《内经》中仅见于本篇。本篇根据生理活动及其特点划分为五脏、六腑之外,又将脑、髓、骨、脉、胆、女子胞归为一类,命名为奇恒之腑。说明此六者在人体生命活动中,具有相当重要的作用,至少在本文作者看来具有仅次于五脏和六腑的作用。这一认识,不同于其他有关藏象文章,因而是篇名称为“别论”的原因之一。

尽管奇恒之腑有“藏而不泻”的功能特点,根据本篇划分脏腑的标准,它应该属于“脏”类,但由于藏象理论以肝、心、脾、肺、肾五脏为核心,地位最高,功能最重要,因而脑髓六者不能与之并列而称脏;同时,此六者本多和五脏相关,如骨、髓、脑属于肾,脉属于心,胆附于肝,子女胞隶属于肝肾二脏,故此六者不能名为脏。欲称之为腑,又与“泻而不藏”的基本条件不符,故以奇恒称之,即不同于常腑之意。

应该看到,本文明确提到五脏、六腑之名,但在《内经》多篇记载中,胆为六腑之一,而本篇则将胆视为奇恒之腑,似相矛盾。尽管王冰曾作解释,亦觉牵强。因此,后世对此“胆”字有多种置疑。不过,从本篇前言胃肠膀胱五者为“传化之腑”,后言“魄门亦为五脏使,水谷不得久藏”记载看,本文作者似将魄门视为六腑之一。联系篇名为《五脏别论》其六腑所指与他篇有别,是不足为怪的。正如清代学者于鬯《香草续校书·内经素问》云:“云化物而不藏,则六腑即上文传化之腑。上文言传化之腑,云‘胃、大肠、小肠、三焦、膀胱’,则止五腑,又云‘魄门亦为五脏使,水谷不得久藏’,则魄门亦实传化之腑之一,合之成六腑……胆亦见上文,乃奇恒之腑,非传化之腑,故舍胆而取魄门为六”。

本文提出奇恒之腑的理论意义,在于强调脑、髓、骨、脉、胆、女子胞在人体生命活动中的重要性,这一认识,在千余年来医学实践尤其是现代临床实践中,已经得到了一定的证明。即以此六者在临床地位而言,脑(髓),中医脑病学已受到广泛关注;脉,心血管疾病早已设立专科诊治;胆,胆和胆系疾病越来越受到临床医生的重视;骨、女子胞,骨科和妇科早已发展为独立的临床学科。

\biaoti{【临证指要】}

1.脏腑藏泻理论的临床应用

本篇指出五脏藏而不泻、六腑泻而不藏的藏象理论,突出了脏和腑各自的生理功能特点,而与生理相对应的,便是病理。即五脏之病以失藏过泻而致虚损之证为多,六腑之病则以失泻过藏而致邪气滞留的实证为主,从而为诊治脏腑疾病提示了基本思路。《内经》以及后世众多医家,在论述五脏病证时,往往虚实并列,而且虚证多于实证,正是由于五脏精神易于失藏过耗而不足;在论述六腑病证时,则多以实邪停滞的实证为主,这是因为六腑失于传泻而腑气滞塞不通所致。因此,补益的法则,多用于五脏疾病的治疗,如养心安神、健脾养营、滋补肝血、补益肺气、固肾填精、滋阴降火等;而通降腑气则是六腑病证共同的基本治则。如和胃止呕、攻下通便、清泻三焦、疏利膀胱、顺气利肠等。而中西医结合应用通下攻泻的中药治疗急腹症获得成功,同样是六腑“传化物而不藏”、“此不能久留,输泻者也”等学术观点临床运用的具体例证。

2.“魄门亦为五脏使”的临床意义

“魄门亦为五脏使”一句,不仅是对传化之腑的补充,同时说明肛门启闭正常,排出糟粕,不单是“腑”的功能,并且是受五脏支配的,而其排出糟粕正常与否,还直接影响五脏气机的升降出入。因此,肛门启闭是否正常,关系到五脏乃至全身的生理、病理状况。张介宾《类经·藏象类》云:“大肠与肺为表里,肺藏魄而主气,肛门失守则气陷而神去,故曰魄门。不独是也,虽诸腑糟粕固由其泻,而脏气升降亦赖以调,故亦为五脏使。”

临床上,通过观察大便情况常可判断疾病虚实寒热,甚至推断预后吉凶。《素问·脉要精微论》:“仓廪不藏者,是门户不要也……得守者生,失守者死”。《素问·玉机真脏论》在论述“五虚死,五实死”时,也特别指出:“浆粥入胃,注泄止,则虚者活;身汗,得后利,则实者活。”治疗方面,不仅大便秘结或泄泻,要根据辨证的结论而分別治疗不同的脏腑,而且某些脏腑的病变也可通过调节肛门启闭收到疗效。如吴瑭创制的宣白承气汤既可治疗肠热便秘,又能治疗肺热痰喘。

\biaoti{【原文】}

\begin{yuanwen}
帝曰:氣口\sb{1}何以獨爲五藏主\sb{2}?岐伯曰:胃者,水穀之海,六府之大源也。五味入口,藏於胃,以養五藏氣,氣口亦太陰也\sb{3}。是以五藏六府之氣味\sb{4},皆出於胃,變見於氣口。故五氣入鼻,藏於心肺,心肺有病,而鼻爲之不利也。

凡治病,心察其下,適其脈\sb{5},觀其志意,舆其病也\sb{6}。拘於鬼神者,不可與言至德;惡於針石者,不可與言至巧;病不許治者,病必不治,治之無功矣。
\end{yuanwen}

\biaoti{【校注】}

\begin{jiaozhu}
  \item 气口:指腕后桡动脉搏动处,是切脉的常用部位。张介宾注云:“气口之义,其名有三:手太阴肺经脉也,肺主诸气,气之盛衰见于此,故曰气口;肺朝百脉,脉之大会聚于此,故曰脉口;脉出太渊,其长一寸九分,故曰寸口。是名虽三而实则一耳”。
  \item 独为五脏主:主,主病。三部九候之脉皆可诊病,但每一部脉只诊一个部位之病,唯独寸口可以诊察五脏之病。
  \item 气口亦太阴也:张介宾注:“气口本属太阴,而曰亦太阴者何也?盖气口属肺,手太阴也;布行胃气,则在于脾,足太阴也……胃气必归于脾,脾气必归于肺,而后行于脏腑营卫,所以气口虽为手太阴,而实即足太阴之所归,故曰气口亦太阴也。”
  \item 气味:此代指脏腑的精气盛衰、功能状况等。
  \item 必察其下,适其脉:适,测也。杨上善《太素·人迎脉口诊》作“必察其上下,适其脉候。”可从。上,指上窍,如鼻息及嗅觉是否正常;下,指下窍,如二便是否通利,魄门启闭是否如常。
  \item 观其志意,与其病也:“也”字,《太素·迎脉口诊》作“能”。能,通态。病态,即证候。杨上善注:“复观其人病态,能可疗以否”。张介宾注云:“是志意关于神气而存亡系之,此志意之不可不察也。”
\end{jiaozhu}

\biaoti{【临证指要】}

\xiaobt{“心肺有病而鼻为之不利”临证举验}

肺有病而鼻不利,临床多见,心有病而鼻不利者相对较少,因而常被忽视,以致误诊、漏诊。王洪图《黄帝医术临证切要·藏象临证发挥》记录了两例心有病而鼻不利的验案:

一例是本有“冠心病”患者,数月来病情平稳,未曾服药。一日突感鼻塞呼吸不利而到某医院就医,医生检査鼻部无异常便未予治疗,但鼻塞症状加剧,乃求治;二例是一“冠心病”患者因胸痛、憋气就诊,而兼有臭觉失灵,鼻不闻香臭。王洪图皆以治疗“冠心病”为主,方用茯苓杏仁甘草汤合旋覆花汤加减。数剂之后,患者在一般症状缓解的同时,鼻塞不利、嗅觉失灵均获解除。

\zuozhe{(王洪图)}

\section{素問·經脈別論(節選)}%第四節

\biaoti{【原文】}

\begin{yuanwen}
黃帝問曰:人之居處動靜勇怯\sb{1},脈\sb{2}爲之變乎?岐伯對曰:凡人驚恐恚勞\sb{3}動靜,皆爲變也。是以夜行則喘出於腎\sb{4},淫氣病肺\sb{5}。有所墮恐,喘出於肝,淫氣害脾\sb{6}。有所驚恐,喘出於肺,淫氣傷心\sb{7}。度水跌仆,喘出於腎與骨\sb{8}。當是之時,勇者氣行則已,怯者則著而爲病也\sb{9}。故曰:診病之道,觀人勇怯骨肉皮膚,能知其情,以爲診法也\sb{10}。故飲食飽甚,汗出於胃\sb{11}。驚而奪精,汗出於心\sb{12}。持重遠行,汗出於腎\sb{13}。疾走恐懼,汗出於肝\sb{14}。搖體勞苦,汗出於脾\sb{15}。故春秋冬夏,四時陰陽,生病起於過用\sb{16},此爲常也。
\end{yuanwen}

\biaoti{【校注】}

\begin{jiaozhu}
  \item 居处动静勇怯:居处,指居住条件及生活环境。动静,指劳作与休息。勇怯,指体质强弱及心理的勇敢与怯懦。
  \item 脉:此指经脉气血。张介宾注:“脉以经脉血气统言之也。”高世栻注:“脉,经脉也。人之经脉,行有常度,如居处之动静,用力之勇怯,经脉亦为之变乎?帝问脉变,所以为经脉之别也。”
  \item 恚劳:恚,恼怒。劳,心劳。恚劳,代指情志失调和思虑过度。
  \item 是以夜行则喘出于肾:张琦注:“凡喘皆肺病而所因不同,故五脏气乘之,皆能为喘,然其因于肺气不降一也。卫气夜行于阴,自足少阴始,肾间动气,呼吸之门,夜行则卫不得藏,而肾气复动,故喘。”肾属阴主闭藏,时应于亥子,夜行扰动肾气,肾失闭藏,纳气无权,致使肺失肃降而喘,故喘出于肾。
  \item 淫气病肺:淫气,指妄行逆乱之气。张志聪注:“肾为本,肺为末,肾气上逆,故淫伤于肺气。”
  \item 有所堕恐,喘出于肝,淫气害脾:丹波元坚注:“‘堕恐’,二字含义似不属,且下有‘惊恐',此‘恐’字疑伪”。据《灵枢·邪气脏腑病形》谓“有所堕坠,……则伤肝”之论,此“堕恐”宜作“堕坠”为是。王冰注:“堕损筋血,因而奔喘,故出于肝也。肝木妄淫,害脾土也。”
  \item 有所惊恐,喘出于肺,淫气伤心:吴昆注:“惊则神越,气乱于胸中,故喘出于肺。心藏神,神乱则邪入,故淫气伤心。”
  \item 度水跌仆,喘出于肾与骨:度,通“渡”。渡水,即步行涉水。张介宾注:“水气通于肾,跌仆伤于骨,故喘出焉。”
  \item 勇者气行则已,怯者则著而为病也:已,止也,指不发病。全句谓体质壮盛而神强气勇者,气血畅行,邪气自无留着之处,故不发病;体质虚弱而神衰气怯者,气血呆滞,邪气着留不去,故发病。
  \item 以为诊法也:张介宾注:“勇者可察其有佘,怯可察其不足,骨可以察肾,肉可以察脾,皮肤可以察肺,望而知其情,即善诊者也。”
  \item 饮食饱甚,汗出于胃:吴昆注:“此下五条,言过用者之损阴也。汗,阴液也。”饮食过饱,胃气蒸腾,胃津发散,故汗出于胃。
  \item 惊而夺精,汗出于心:王冰注:“惊夺心精,神气浮越,阳内薄之,敌汗出于心也。”谓大惊而夺失心精,神失依附而浮越,液失收摄而汗出。
  \item 持重远行,汗出於肾:姚止庵注:“肾主骨,骨劲乃能持重。若所持既重而行又远,则肾惫骨虚,气外泄而为汗矣。”
  \item 疾走恐惧,汗出于肝:吴昆注:“肝主筋而藏魂,疾走则伤筋,恐惧则伤魂,肝受其伤,故汗出于肝。”
  \item 摇体劳苦,汗出于脾:吴昆注:“摇体劳苦,用力勤作也。脾主四肢,故汗出于脾。”
  \item 生病起于过用:过用,泛指七情、劳逸、饮食等过度。张介宾注;“五脏受气,强弱各有常度,若勉强过用,必损其真,则病之所由起也。”
\end{jiaozhu}

\biaoti{【理论阐释】}

1.勇怯与发病

勇、怯,历代医家多以形体的强弱作释,如吴昆谓:“壮者谓之勇,弱者谓之怯”。但是,从《内经》整体思想看,勇怯更重要的内涵当包括心理气质因素在内。如《灵枢》有“论勇”专篇,其勇怯显然是指心理气质而言。原文“勇者气行则已,怯者则着而为病”,深刻揭示了勇怯与发病密切关系。勇者,形体壮盛,神气坚强,正能胜邪,故不易发病;怯者,形体虚衰,神气怯懦,正不胜邪,故易于发病。

精气神是人体生命活动的三大要素,精化气,气生神,神是在精气的基础上化生的,同时神气具有统御调动精气功能的作用,因而神气亦与精气一样,具有抗御外邪的功能。勇怯既与形体盛衰有关,同时又是神气强弱的反映。也可以认为,勇怯是形神合一的综合概念,它具有心身两方面的含义,从心理角度而言,勇怯是人体适应自然变化,缓冲社会因素的心理反映,勇怯形成之后,可以通过影响正气的功能而发挥抗邪防病作用。故《素问·上古天真论》指出:“恬惔虚无,真气从之,精神内守,病安从来。《素问·刺法论》有“气出于脑,即不邪干”;“五气护身之毕,·以想头上如北斗之煌煌,然后可以入于疫室”之说。“气出于脑”,即神气出于脑;“想头上如北斗之煌煌”,亦一种积极的心理活动。均说明心理气质在抗邪防病中的重要作用。

2.“生病起于过用”

原文:“春秋冬夏,四时阴阳,生病起于过用,此为常也。”由此提出了“生病起于过用”的重要发病观。“过用”即使用太过,超越常度。本段的“过用”虽是针对惊恐、恚劳、动静等因素致病而言,但“生病起于过用”的论点,在发病学中实具有普遍指导意义,应视为人体发病的普遍规律。根据《内经》所论,“生病起于过用”的内容,可以从以下五个方面体现出来。

(1)四时气候过用:四季正常气候变化是人体赖以生存的重要条件,若气候反常,风寒暑湿燥火六气太过或不及时,均可造成人体对时气的“过用”,进而导致疾病。如《素问·六节藏象论》说;“未至而至,此谓太过,……命曰气淫。”“至而未至,此谓不及,……命曰气迫。”气候的太过、不及而造成的“气淫”、“气迫”,即是人体对四季“失时反候”之气“过用”致病之例。

(2)精神情志过用:精神情志是生命活动的正常表现之一,适度有益于健康,若精神反常,情志太过,均为“过用”,过则为病。《素问·上古天真论》说:“以欲竭其精,以耗散其真,不知持满,不时御神,务快其心,逆于生乐。”便指出了精神“过用”的危害性。《素问·举痛论》还讨论了情志太过导致气机为病的规律及机理,谓“怒则气上,喜则气缓,悲则气消,恐则气下,……惊则气乱,劳则气耗,思则气结”。亦是七情“过用”所致。

(3)饮食五味过用:民以食为天,饮食五味是维持人体生命活动的后天之本。若暴饮暴食,饥饱失常,或五味偏嗜,饮食不洁,则均构成“过用”,是为发病之因。故《素问·痹论》说:“饮食自倍,肠胃乃伤”。《素问·生天通天论》亦说:“高梁之变,足生大丁,受如持虚”;“阴之五宫,伤在五味。是故味过于酸,肝气以津,脾气乃绝;味过于咸,大骨气劳,短肌,心气抑;味过于甘,心气喘满,色黑,肾气不衡;味过于苦,脾气不濡,胃气乃厚;味过于辛,筋脉沮弛,精神乃央”。说明饮食致病,多由“过用”所致。

(4)劳逸过用:劳此指劳力、劳心及房劳。劳逸太过,即为“过用”。如《素问·调经论》说:“有所劳倦,形气衰少。”《素问·举痛论》说:“劳则喘息汗出,内外皆越。”《素问·宣明五气篇》说:“久视伤血,久卧伤气,久坐伤肉,久立伤骨,久行伤筋。”《灵枢·本神》说:“是故怵惕思虑则伤神,神伤则恐惧,流淫不止。”《素问·痿论》说:“人房太甚,宗筋㢮纵,发为筋痿,及为白淫。”上述所伤,均为劳力、劳心、房劳“过用”所造成。

(5)药物过用:药物各具偏性,“过用”亦能致病。如《素问·腹中论》说:石药发癫,芳草发狂。“《素问·至真要大论》说:“久而增气,物化之常也。气增而久,夭之由也。”《素问·五常政大论》强调:“大毒治病,十去其六;常毒治病,十去其七;小毒治病,十去其八;无毒治病,十去其九;……无使过之,伤其正也。”药物虽有治病却疾之能,然而“过用”则又会伤正致邪,酿生变端,因此不可不慎。

综上所述,凡超越人体正常范围,致使脏腑发生损伤,生理功能、心理活动遭受破坏的各种因素,均属“过用”。从发病学角度而言,无论外感或内伤致病,都符合“生病起于过用”的规律。此外,以医患而言,则除患病者可致“过用”外,医生的治疗不当,亦可造成“过用”之害。因此,无论是用药,或是用针、按摩等,均应适度而不可太过。

\biaoti{【临证指要】}

\xiaobt{喘、汗与五脏}

喘、汗各有所主,一般认为,喘发于肺,汗出于心。由于人是一个有机整体,一旦病及五脏,而又影响肺、心时,则亦可发生气喘与汗出。此时的喘汗,以标本而论,则五脏为本,肺、心为标,因此其治疗重点亦有所不同。

(1)五脏喘:夜行肾气外泄,气失闭藏,背气上逆,气乱于肺,则为肾喘。治宜固摄益肾,潜降纳气,方如金匮肾气丸、参蛤散之类。堕坠伤血损筋,肝气上逆,肺失肃降,则为肝喘。肝喘多实,治宜疏肝活血,降气平喘,方如复元活血汤、五磨饮子之类。惊恐伤神,心神浮越,气乱于肺,则为心喘。治宜重镇安神,降气平喘,方如柴胡桂枝龙骨牡蛎汤、磁硃丸之类。渡水跌仆,水寒袭肾,跌仆伤骨,逆气犯肺,亦为肾喘。治宜温肾散寒,活血降逆,方如真武汤、桃红四物汤之类。

(2)五脏汗:汗为阳气蒸化阴液而成,饮食过饱,胃满气溢,胃津外泄,则为胃汗。治宜化食和胃,清胃泄热,方如保和丸、清胃散之类。胃汗虽不属五脏汗,但与脾有关,并列入五脏讨论,反映了《内经》重视后天脾胃的思想。持重远行,久行伤骨,肾气亦伤,肾液外泄,则为肾汗。治宜滋阴补背,益气止汗,方如六味地黄丸、金匮肾气丸之类。疾走伤筋,恐惧伤魂,肝伤液泄,则为肝汗。治宜滋阴敛液,柔肝止汗,方如一贯煎、酸枣仁汤之类。劳作过度,伤肉损脾,脾虚液泄,则为脾汗。治宜补脾益气,敛阴止汗,方如补中益气汤、玉屏风散之类。

五脏致喘、致汗,反映了《内经》重视五脏整体辨证的思想,他如咳、痿、痹、风、胀等病证的辨证中,《内经》亦运用了五脏整体辨证的方法。这种以五脏为中心的整体辨证思路和模式,对后世产生了深远的影响,至今仍具有十分重要的指导意义。

\biaoti{【原文】}

\begin{yuanwen}
食氣入胃,散精於肝,淫氣於筋\sb{1}。食氣入胃,濁氣歸心\sb{2},淫精於脈\sb{3}。脈氣流經\sb{4},經氣歸於肺,肺朝百脈\sb{5},輸精於皮毛。毛脈合精\sb{6},行氣於府\sb{7}。府精神明,留于四藏\sb{8},氣歸於權衡\sb{9},權衡以平,氣口成寸\sb{10},以決死生。

飲入于胃,遊溢精氣\sb{11},上輸於脾,脾氣散精,上歸於肺\sb{12},通調水道,下输膀胱\sb{13}。水精四布,五經並行\sb{14}。合于四時五藏陰陽,揆度以爲常也\sb{15}。
\end{yuanwen}

\biaoti{【校注】}

\begin{jiaozhu}
  \item 散精于肝,淫气于筋:散,散布、输送。精,食物的精华物质。姚止庵注:“食既入胃,脾为之运,糟粕下行,而其精华则先散布于肝经,以肝为春木,主生发之令故也。”
  \item 浊气归心:浊气,指水谷精气中的浓稠部分。姚止庵注:“食之所化,有清有浊。……浊化血,血有质,心得食气以为养,而血始生焉,故云浊气归心也。”
  \item 淫精于脉:指心将水谷精气中的浓稠部分复输布于脉,因心主脉之故。另说,“精”是指则由谷食精气所化生的营血。营行脉中,故曰“淫精于脉”。上两说可互参。
  \item 脉气流经:张志聪注:“脉气者,水谷之精气而行于经脉中也。”
  \item 经气归肺,肺朝百脉:经气,经脉中流行的气血。朝,上奉、会集之意。王冰注:“言脉气流运,乃为大经,经气归宗,上朝于肺,肺为华盖,位复居高,治节由之,故受百脉之朝会也。”
  \item 毛脉合精:张志聪注:“夫皮肤主气,经脉主血,毛脉合精者,血气相合也。”
  \item 行气于府:“府”,指经脉。《素问·脉要精微论》说:“夫脉者,血之府也。”全句谓心肺之气血相合之后,复运行于经脉之中。
  \item 府精神明,留于四脏:府精,经脉中的气血旺盛。神明,神气精明。留,通“流”。四脏,指心肝脾肾。姚止庵注:“脏本五而此言四者,盖指心肝脾肾言。以肺为诸脏之盖,经气归肺,肺朝百脉,而行气于心肝脾肾,故云留于四脏也。”
  \item 气归于权衡:权衡,即平衡。言精气化为气血,均衡地输布于脏腑组织。
  \item 权衡以平,气口成寸:意谓脏腑精气平衡协调,则经脉之气亦随之盈满而平定,继而反映于气口,所以诊察气口能测知脏腑经脉的变化。
  \item 游溢精气:游,通“游”,流动。溢,渗溢。游溢,形容精气从胃中渗溢流出的状态。精气,此指津液。
  \item 脾气散精,上归于肺:指脾气将津液向上输布散于肺。
  \item 通调水道,下输膀胱:肺气宣降,既可将水液宣布全身,又可将代谢后的水液通过三焦而下输膀胱。张志聪注:“肺应天而主气,故能通调水道而下输膀胱,所谓地气升而为云,天气降而为雨也。”
  \item 水津四布,五行并行:张志聪注:“水津四布者,气化则水行,故四布于皮毛;五行并行者,通灌于五脏之经脉也。”
  \item 合于四时五脏明阳,揆度以为常也:揆度,测度也。意谓饮食精气的生成输布,气血津液的生化运行,均可从测度脉象的变化而得知,在分析时还要结合四时阴阳和五脏阴阳的变化,进行综合判断。
\end{jiaozhu}

\biaoti{【理论阐释】}

1.饮食物的运化及代谢

(1)食物的运化及代谢:食物入胃,经脾胃共同作用化生精气,然后输布全身。其输布的大体过程是:“散精于肝,淫气于筋”;“浊气归心,淫精于脉”;“经气归于肺,……输精于皮毛”。可见,输精的中心是五脏,然后由五脏分别输精于与其相合的形体组织。这里表述的是一种按系统输精的网络观,符合以五脏为中心的五大功能系统理论。在水谷精气的输布过程中,心肺起着关键作用。心主血脉,“浊气归心,淫精于脉”,血气在经脉中周流不休,全赖心气的推动和主持。肺主气,司呼吸,“肺朝百脉”,为“相傅之官”,能协助心治理调节气血,宣发敷布精气,推动血行。二脏密切配合,气血才能在全身输布环流,均衡布散,故原文指出:“毛脉合精,行气于府,府精神明,流于四脏,气归于权衡”。在精气输脏中,原文虽未明言肾、脾二脏,但据文意理解,亦当包括脾肾在内,一是有“府精神明,留于四脏”之语;二是其输精可以依肝、心、肺之例而类推,即精气归脾,输精于肉;精气归肾,输精于骨。

(2)水饮的运化及代谢:水饮入胃,经脾胃共同作用化生精气(津液),然后输布全身。具体过程是:水饮精气输脾归肺,在肺的宣发、肃降作用下,津液经三焦而布散全身,代谢剩余部分经三焦下达而蓄藏于膀胱。在津液的布散下输过程中,肺脾起着重要的作用,故原文强调“脾气散精”,肺“通调水道”。还须明确的是,本段虽未论及肾在津液代謝中的重要作用,但“下输膀胱”一句已寓含着肾与津液代谢的密切关系。《素问·灵兰秘典论》说:“膀胱者,州都之官,津液藏焉,气化则能出矣。”膀胱的气化,除了靠本腑的功能外,主要依赖于肾的气化。膀胱中所藏的津液,在肾的气化作用参与下,才能使清者上升而尿液下泄。故《素问·逆调论》有:“肾者水脏,主津液”,《素问·水热穴论》有水病“其本在肾,其末在肺”及“肾者胃之关也”等论述。

2.切寸口脉诊病的道理

原文谓“权衡以平,气口成寸,以决死生”。为什么切寸口脉能诊病并决死生呢?对此历代论述众多,但以本段原文的精神来看,其道理主要有两点:一是寸口为手太阴肺经所过之部位,脉气旺盛,易于切诊。二是肺主气,朝百脉。气为血帅,在心肺之气的作用下,百脉中的气血都要朝会于肺,然后在肺心气的推动下,又将气血输送于百脉,布散于脏腑组织,所以脏腑之气的盛衰,百脉中气血的变动,均可在气血朝会于肺时,从肺脉的寸口部位反映出来。正如《素问·五脏别论》所说:“五味入口,藏于胃,以养五脏气;气口亦太阴也,是以五脏六腑之气味皆出于胃,变见于气口”。

\biaoti{【临证指要】}

1.肺朝百脉

肺朝百脉,揭示了肺与百脉气血的密切关系。其一,肺与气的虚实和运行有关。肺司呼吸,其吸入的清气是气的重要组成部分,若清气不足,则宗气、营卫、卫气等亦会随之而衰少,此其为致虚的一面。肺主气,其宣降作用能调畅气机,促进气的流通,若肺失主气,宣降不利,则气机失调,易郁易滞,此其为致实的一面。因此,临床上,百脉的气虚或气滞,均可考虑从肺治疗。气虚者,益其肺气;气滞者,调其宣降。其二,肺与血的运行和虚实有关。肺主治节,指肺有助心调节血气的功能,而“肺朝百脉”则是其“治节”作用的具体体现之一。肺对血运的调节,一是通过其宣降作用调畅气机,促进血行来完成,二是通过其所主的宗气和营气推动血行,以及卫气的温通血脉来实现。如肺气失主,宣降不利,宗气、营卫、卫气失行血之职,则血行不利或瘀滞。实当治以宣降肺气,活血化瘀;虚则治以补益肺气,活血行血。肺对血虚的影响,是肺吸入的清气为营血的组成部分,同时肺气又参与了营血的化生,故原文说:“脾气散精,上归于肺”;《灵枢·营卫生会》说:“中焦亦并胃中,出上焦之后,此所受气者,泌糟粕,蒸津液,化其精微,上注于肺脉,乃化而为血,以奉生身,莫贵于此,故独得行于经燧,命曰营气。”由肺所致之血虚,原则上采取益气养血之法,常于补血药中重用益肺气之品,如当归补血汤中重用黄芪之类。

2.肺通调水道

通调水道,是肺的重要生理功能之一。从“脾气散精,上归于肺,通调水道,下输膀胱,水精四布,五经并行”之论看,肺的通调水道作用,主要是通过两大水道系统来实现。一是由肺将津液布散到全身腠理、皮毛、肢体,主要靠“宣发”功能完成,此为外水道系统;二是由肺将水液向下经中、下焦输于膀胱,主要靠“肃降”功能完成,此为内水道系统。两大系统宣降协调,密切配合,维持着人体正常的水液代谢过程以及四时寒暑变迁的适应性调节。由此可见,肺通调水道不仅仅是一个水液下输、废液排泄的过程,更重要的它还是一个津液外布、内濡的新陈代谢过程,体内津液的输布、运行、排泄均与之密切相关。

如果肺失通调,一方面可出现津伤不足的病证,如五体痿之痿躄(皮痿),即肺热津伤,肺失宣发之证。另一方面亦可产生水液停滞的病证,如痰、饮、水肿等。

\section{素問·太陰陽明論}%第五節

\biaoti{【原文】}

\begin{yuanwen}
黃帝問曰:太陰陽明爲表裏,脾胃脈也,生病而異者何也?岐伯對曰:陰陽異位\sb{1},更虛更實\sb{2},更逆更從\sb{3},或從内,或從外,所從不同,故病異名也。帝曰:願聞其異狀也,岐伯曰;陽者,天氣也,主外;陰者,地氣也,主内。故陽道實,陰道虛\sb{4}。故犯賊風虛邪者,陽受之;食飲不節起居不時者,陰受之。陽受之則入六府,陰受之則入五藏\sb{5}。入六府則身熱不時臥\sb{6},上爲喘呼;入五藏則䐜滿閉塞,下爲飧泄,久爲腸澼。故喉主天氣,咽主地氣。故陽受風氣,陰受濕氣\sb{7}。故陰氣從足上行至頭,而下行循臂至指端;陽氣從手上行至頭,而下行至足。故曰陽病者上行極而下,陰病者下行極而上\sb{8}。故傷於風者,上先受之;傷於濕者,下先受之\sb{9}。
\end{yuanwen}

\biaoti{【校注】}

\begin{jiaozhu}
  \item 阴阳异位:张介宾注:“脾为脏,阴也。胃为腑,阳也。阳主外,阴主内,阳主上,阴主下,是阴阳异位也。”
  \item 更虚更实:张介宾注:“阳虚则阴实,阴虚则阳实,是更虚更实也。”
  \item 更逆更从:脾气升胃气降为从,脾气降胃气升为逆。即病者为逆,不病者为从。
  \item 阳道实,阴道虚:道,是发病规律。全句谓属阳的六腑多病外感而为实证,属阴的五脏多病内伤而为虚证。
  \item 阳受之则入六腑,阴受之则入五脏:“阳”,指阳明胃腑。“阴”,指太阴脾脏。张志聪注:“入六腑者,谓阳明为之行气于三阳,阳明病则六腑之气皆为之病矣。……入五脏者,谓太阴为之行气于三阴,太阴病则五脏之气皆为之病矣。”
  \item 不时卧:应据《甲乙经》改作“不得眠。”
  \item 阳受风气,阴受湿气:张介宾注:“风,阳气也,故阳分受之;湿,阴气也,故阴分受之。各从其类也。”
  \item 阳病者上行极而下,阴病者下行极而上:张介宾注:“盖阴气在下,下者必升;阳气在上,上者必降。脾阴胃阳,气皆然也。阳病极则及于下,阴病极则及于上,极则变也。非惟上下,表里亦然。
  \item 伤于风者,上先受之;伤于湿者,下先受之:张介宾注:“阳受风气,故上先受之;阴受湿气,故下先受之。然上非无湿,下非无风,但受有先后耳。曰先受之,则后者可知矣。”
\end{jiaozhu}

\biaoti{【理论阐释】}

1.关于“阳道实,阴道虚”

“阳道实,阴道虚”,高度概括了五脏与六腑的生理病理特点。从生理上看,五脏属阴,六腑属阳,二者各具有不同的生理功能和特点。五脏贮藏精气,藏而不泻,静而“主内”,易于耗伤,故多不足;六腑传化水谷,泻而不藏,动而“主外”,易于积滞,故多有余。从病理上看,外感之邪首先侵犯阳经、阳腑,多见邪气有余的实证;内伤之因首先侵犯阴经、阴脏,多见正气不足的虚证。因此临床上五脏病多虚,六腑病多实;《伤寒论》中三阳病多实,三阴病多虚。而后世又有“实则太阳,虚则少阴”,“实则阳明,虚则太阴”的说法。

从原文“阳者,天气也,主外;阴者,地气也,主内”来看,“阳道实,阴道虚”,从一个方面概栝了阴阳的各自基本特性,亦可以广泛应用于各种阴阳事物和现象之中。一般而言,凡事物之属于阳者,必有温热、躁动、刚悍、充实、有余、向外、向上等特性;而事物之属于阴者,必有寒凉、安静、柔顺、虚弱、不足、向内、向下等特性。因此,天地、日月、男女、阳气阴精、六腑五脏等都具有阳道实阴道虚的特点,在分析人体的生理病理时切切不可忽略这一点。

2.不同病因伤人的部位规律

不同性质的邪气,对人体部位的侵犯有一种趋向性。这种趋向性表现为以类相从、同气相求的规律。病因中六淫属阳,饮食不节、起居不时属阴;脏腑中六腑属阳,五脏属阴;经脉中阳脉属阳,阴经属阴。故“犯贼风虚邪者,阳受之;食饮不节、起居不时者,阴受之。阳受之则入六腑,阴受之则入五脏”。可见,属阳的邪气,多侵犯属阳的部位,而属阴的邪气,则多侵犯属阴的部位。由于阳阴属性的划分具有相对性,因此外感或内伤之邪还有阴阳之别,其所伤部位也各不相同,如“故阳受风气,阴受湿气”,“伤于风者,上先受之;伤于湿者,下先受之。”

《素问·阴阳应象大论》说:“天之邪气,感则害人五脏;水谷之寒热,感则害于六腑。”从另一角度讨论了邪气伤人的规律。此文乍看上去,似与本篇“故犯贼风虚邪者,阳受之;食饮不节、起居不时者,阴受之。阳受之,则入六腑;阴受之,则入五脏”之义相反,然实则相反相成。张琦《素问释义》注云:“以邪气言,邪气无形故入脏,水谷有形故入腑;以表里言,腑阳主外,故贼风虚邪从外而受,脏阴主内,故饮食不节从内而受;实则脏腑皆当有之。盖内外之邪,病情万变,非一端可尽,故广陈其义耳。”

\biaoti{【临证指要】}

1.“阳道实,阴道虚”

“阳道实,阴道虚”,揭示了五脏六腑的病理、病证规律,是辨证所必须掌握的重要内容。从临床来看,由于五脏多虚证,六腑多实证,因此在治疗上,五脏病证当以扶正补虚为本,六腑病证当以祛邪通降为先。如《伤寒论》邪气入里化热,侵犯阳明之经,证见身大热、大汗出、烦渴引饮、舌苔黄燥、脉洪大等,治宜清热生津,以白虎汤祛邪为先;邪传阳明胃腑,邪热与肠中糟粕搏结而证见日晡潮热,手足濈然汗出,脐腹胀满疼痛,大便秘结,舌苔黄厚干燥,边尖起芒刺,甚则焦黑燥裂,脉沉迟而实或滑数等,治当清热通腑,以承气汤通降为要。若太阴为饮食所伤,证见脘腹胀满,大便泄泻,时腹自痛,喜温喜按,苔白脉濡或迟缓,治当温运脾阳,以理中汤类温补建中为主。基于“阳道实,阴道虚”的特点,在治疗上对于六腑的虚证,要补中不忘通降;对于五脏的实证,也要泻中不忘补益,此所谓顺脏腑之性而为,也寓“治病必求于本”之意。

2.“阳受风气,阴受湿气”

风为阳邪,其性轻扬,故风邪易伤人体属阳的上部、外部;湿为阴邪,其性沉滞,故湿邪易伤人体属阴的下部、内部。原文云:“伤于风者,上先受之;伤于湿者,下先受之。”临床上,头部的头痛、头晕及体表的瘙痒肿痛等症多为风邪所致;足部的重痛、肿胀,脾胃的胀满、恶呕多为湿邪所致。不仅如此,同为湿邪,还有天湿、地湿的区别。雨湿来自天,多伤人之头部,水湿出自地,多伤人之足部。故《灵枢·百病始生》指出:“清湿袭虚,则病起于下;风雨袭虚,则病起于上。”因此,六淫之邪伤人,其头部、体表的病证,要多考虑风邪为患;足部、体内的病证,要多考虑湿邪所伤。即使同感湿邪,头部之病多为天之雨湿所伤,足部之病多为地之湿气所为。

\biaoti{【原文】}

\begin{yuanwen}
帝曰:脾病而四支不用何也?岐伯曰:四支皆稟氣於胃,而不得至經\sb{1},必因於脾,乃得禀也。今脾病不能爲胃行其津液,四支不得稟水穀氣,氣日以衰,脈道不利,筋骨肌肉,皆無氣以生,故不用焉。帝曰:脾不主時何也?岐伯曰:脾者土也,治中央\sb{2},常以四時長四藏,各十八日寄治\sb{3},不得獨主于時也。脾藏者常著胃土之精也,\sb{4},土者生萬物而法天地,故上下至頭足,不得主時也。帝曰:脾與胃以膜相連耳,而能爲之行其津液何也?岐伯曰:足太陰者三陰也,其脈貫胃屬脾絡嗌,故太陰爲之行氣于三陰\sb{5}。陽明者表也,五藏六府之海也,亦爲之行氣于三陽\sb{6}。藏府各因其經\sb{7}而受氣于陽明,故爲胃行其津液。四支不得稟水穀氣,日以益衰,陰道不利,筋骨肌肉無氣以生,故不用焉。
\end{yuanwen}

\biaoti{【校注】}

\begin{jiaozhu}
  \item 至经:《太素》作“径至”,可从。径至,直接到达。
  \item 治中央:治,主、旺也。张介宾注:“五脏所主,故肝木主春而旺于东,心火主夏而旺于南,肺金主秋而旺于西,肾水主冬而旺于北,唯脾属土而蓄养万物,故位应中央,寄旺四时各一十八日。”
  \item 各十八日寄治:寄,暂居的意思。土之正位在中央,而于四季立春、立夏、立秋、立冬之前,土旺主事各十八日,故曰“寄治”。张志聪注:“春夏秋冬,肝心肺肾之所主也。土位中央,灌溉于四脏,是以四季月中各旺十八日。是四时之中皆有土气,而不独主于时也。五脏之气各主七十二日以成一岁。”
  \item 常著胃土之精也:著,昭著,常著胃土之精,即常使胃中水谷精气布达昭著于全身的意思。姚止庵注:“胃主受,脾主运,胃受水谷而脾为之运化,使之著见于一身,是胃土之精实由脾著也。”
  \item 太阴为之行气于三阴:吴昆注:“为之,为胃也。三阴,太、少、厥也。脾为胃行气于三阴,运阳明之气入于诸阴也。”
  \item 亦为之行气于三阳:为之,为胃。张介宾注:“虽阳明行气于三阳,然赖脾气而后行,故曰亦也”。三阳,三阳经及六腑。
  \item 脏腑各因其经:姚止鹿注:“其经,即脾经也。言五脏六腑必借脾之运化,而后得胃气以为养。胃之津液,亦必借脾之运化,而后得遍及于五脏六腑也。”
\end{jiaozhu}

\biaoti{【理论阐释】}

1.脾与胃的关系

脾主运,胃主纳,二者虽然生理分工不同,但是它们之间的关系却密不可分,具体表现为以下几个方面:

(1)组织结构方面:脾与胃同居中焦,“以膜相连”,“太阴阳明为表里,经脉相互络属,脾“脉贯胃属脾络嗌”。可见,脾与胃在组织结构方面的联系是同居中焦,经脉相贯,脏腑相连,表里相合,阴阳相从。

(2)生理功能方面:胃主受纳,为“五脏六腑之海”,脾主运化,为“胃行其津液”,二者既分工又协作,共同完成对水谷的消化、吸收、输布等功能。具体来说,胃受纳水谷后,在脾的协同下,化生水谷精气,但胃中水谷精气不能“径至”四肢及脏腑,“必因于脾,乃得禀也”,因此,胃中水谷精气还必须在脾气的作用下,通过“太阴为之行气于三阴”,复通过阳明“为之行气于三阳”,从而“脏腑各因其经受气于阳明”,这就是脾为胃行其津液的道理,亦即脾胃为“后天之本”的理论渊源。后世将其生理关系总结为脾主升,胃主降,相反相成。脾气升,则水谷之精微得以输布;胃气降,则水谷及其糟粕才得以下行。胃属燥土,脾属湿土,胃喜润恶燥,脾喜燥恶湿,燥湿相济,阴阳相合,才能完成饮食物的运化过程。由此说明脾胃阴阳相合,升降相因,燥湿相济,密不可分。

(3)病理方面:脾与胃的病理虽然各具特点,所谓“阳道实,阴道虚”,然而二者之间却相互影响,不可分割。如“四肢皆禀气于胃”,若脾病“不能为胃行其津液”。则会影响胃中水谷精气向四肢的输送,出现“四肢不用”,亦可影响胃的受纳,引起纳呆、腹胀、恶呕等症状;相反,若胃病失纳降,则脾运亦会受到影响,从而出现腹胀便秘、四肢不用等症。又如,在伤寒病变的过程中,通过一定条件催化,太阴病可以转化为阳明病,阳明病亦可转化为太阴病。以上说明临床上脾病可及胃,胃病可以及脾的道理。

2.脾胃为后天之本

基于脾与胃的密切关系,言脾必然及胃,论胃也必然及脾,脾与胃常常合而论之。如《素问·灵兰秘典论》说:“脾胃者,仓廪之官,五味出焉”。胃主受纳,脾主运化,二者分工不同,但胃纳和脾运必须相互配合,相辅相成,才能共同完成饮食物的消化吸收及精微物质的输布,从而使五脏六腑及各组织器官得到水谷精气的充养。原文所谓胃“为五脏六腑之海”,此胃实包括脾在内。正是由于人体出生之后,其生长发育和生命活动的维持,主要依赖于水谷精气,而水谷精气的化生,又依赖于脾胃纳运的相互配合,二者缺一不可,故称脾胃为“后天之本”,“气血生化之源”,突出了脾胃在人体的特殊重要地位。

在“后天之本”的作用中,相对胃而言,脾居于主导地位,是脾功能系统的主体和中心。具体表现在以下三个方面:一是脾具有运化水谷精微的功能。胃受纳水谷,化生水谷精气,脾除了参与化生水谷精气外,还能将胃中水谷精气输布于脏腑及周身,故原文说:“四肢皆東气于胃,而不得至经,必因于脾,乃得禀也”;“脾脏者,常著胃土之精也。”《素问·玉机真脏论》亦说:“脾为孤脏,中央土以灌四傍。”可见,胃化生水谷精气,必须脾气的参与;其水谷精气的布散更赖脾气的转输。二是脾能运化水液,调节胃中津液代谢。如原文说:“饮入于胃,游溢精气,上输于脾,脾气散精,上归于肺,通调水道,下输膀胱。”说明胃中津液的代谢亦有依赖于脾的运化,“脾气散精”,才能使胃中津液布达周身。三是脾运有助于胃纳。胃纳是脾运的基础,脾运是胃纳的前提,只有脾气健运,才能胃纳正常。相反,若脾病亦会影响胃纳,而出现不能食之证。正如赵献可《医贯》所说:“不能食者,脾之病,脾主浇灌四旁,与胃行其津液者也。”

\biaoti{【临证指要】}

1.“脾病而四肢不用”

四肢不用,指四肢不能正常运动。四肢运动与五脏均有关,为何独重脾胃呢?如杨上善《黄帝内经太素·脏腑之一》云:“五脏皆连四肢,何因脾病独四肢不用也?”“脾病而四肢不用”的论点,是从脾胃为水谷精气生化之源的角度提出的,反映了“后天之本”的重要性。从总体而言,四肢之所以能正常运动,是由于它不断地得到胃中水谷精气的充养,然而胃中水谷精气“不得径至”于四肢,“必因于脾,乃得稟也”。若“脾病不能为胃行津液”,则“四肢不得禀水谷气”,进而“脉道不利,筋骨肌肉皆无气以生”,四肢失养,故不用。临床上,痿证多可从脾胃治疗,故《素问·痿论》有“治痿独取阳明”之说。尽管致痿原因多端,涉及脏腑广泛,然而依临床所见,由脾胃所致痿证的比例较大,而且,各种痿证后期均会涉及脾胃,出现脾胃虚损之证。因此,重视脾胃在治痿中的地位实具有十分重要的意义。据《素问·痿论》所述,其五体痿均由脏热精伤所致,肺热津伤可致痿躄,心热血伤可致脉痿,脾热精伤可致肉痿,肝热血伤可致筋痿,肾热精伤可致骨痿,然而其被耗伤的精气血津液,均可从脾胃化生的水谷精气中得到补充,所以在治疗各种痿证时注意调理脾胃,亦不失为明智之举。

2.胃为“五脏六腑之海”

海,有汇聚之意。阳明胃主受纳水谷,脏腑气血均海于水谷所化,故称胃为“五脏六腑之海”。然胃中水谷精微不能直接到达五脏六腑,亦“必因于脾,乃得禀也”,因此,此胃当包括脾在内,或曰脾胃“为五脏六腑之海”。《灵枢·五味》亦有相同论述,曰:“胃者,五脏六腑之海也,水谷皆入于胃,五脏六腑皆禀气于胃。”《素问·玉机真脏论》亦说:“五脏者,皆禀气于胃。胃者,五脏之本也”。

脾胃之气及其正常的功能活动,又称为胃气。人以胃气为本,“有胃气则生,无胃气则死”,所以中医诊治疾病均十分重视胃气,常把“保胃气”作为重要的治疗原则。胃气对于维持机体正常的生命活动,促进疾病的痊愈与正气的恢复,均至关重要。历代不少医家提出了“治病当以脾胃为先”的观点,临床上对虚损病证的治疗常可通过调补脾胃而获效。如清·王三尊《医权初编》说:“凡饮食先入于胃,俟脾胃运化,……若脾胃有病,或虚或实,一切饮食药饵,皆不运化,安望精微输肺而布各脏耶?是知治病当以脾胃为先。若脾胃它脏兼而有病,舍脾胃而治它脏,无益也。又一切虚证,不问在气在血,在何脏腑,而只专补脾胃,脾胃一强,则饮食自倍,精血日旺,阳生而阴亦长矣。”此外,诸病日久不愈,也可通过调治脾胃来治疗。明·周子干《慎斋遗书·辨证施治》指出:“诸病不愈,必寻到脾胃之中,方无一失。何以言之?脾胃一伤,四脏皆无生气,故疾病日多矣。万物从土而生,亦从土而归。‘补肾不若补脾’,此之谓也。治病不愈,寻到脾胃而愈者甚多。凡见咳嗽、自汗、发热、脾虚生痰,不必理痰清热,土旺而痰消热退,四君子汤加桂、姜、陈皮、北五味子,后调以参苓白术散。”

在临床上还要深刻理解脾胃与五脏之气的互含互治关系。张介宾在《景岳全书·杂证谟》中对此有十分精辟的论述,说:“脾胃有病,自宜治脾,然脾为土脏,灌溉四旁,是以五脏中皆有脾气,而脾胃中亦皆有五脏之气,此其互为相使,有可分而不可分者在焉。故善治脾者,能调五脏,即所以治脾胃也。能治脾胃,而使食进胃强,即所以安五脏也。”可见,五脏之气源于脾胃之气,治脾胃即所以治五脏;脾胃之气滋于五脏之气,治五脏即所以治脾胃。以上充分体现了脾胃与五脏之间的辩证关系。

\section{靈樞·本神}%第六節

\biaoti{【原文】}

\begin{yuanwen}
黃帝問于岐伯曰:凡刺之法,先必本於神\sb{1},血脈營氣精神,此五藏之所藏也。至其淫泆離藏\sb{2}則精失,魂魄飛揚,志意恍亂,智慮去身者,何因而然乎?天之罪舆?人之過乎?何謂德、氣、生、精、神、魂、魄、心、意、志、思、智、慮?請問其故。岐伯答曰:天之在我者德也,地之在我者氣也\sb{3},德流氣薄而生者也\sb{4}。故生之來謂之精\sb{5},兩精相搏謂之神\sb{6},隨神往來者謂之魂,並精而出入者謂之魄\sb{7}。所以任物者謂之心\sb{8},心有所憶謂之意\sb{9},意之所存謂之志\sb{10},因志而存變謂之思\sb{11},因思而遠慕謂之慮\sb{12},因慮而處物謂之智\sb{13}。故智者之養生也\sb{14},必順四時而適寒暑,如喜怒而安居處,節陰陽而調剛柔\sb{15}。如是則僻邪\sb{16}不至,長生久視\sb{17}。
\end{yuanwen}

\biaoti{【校注】}

\begin{jiaozhu}
  \item 先必本于神:“先必”,《甲乙经》作“必先”,马莳等注本亦同,可参。本于神,即以神气为根本。此“神”主要指病人的神气,也包括医生的神气在内。
  \item 淫泆离脏:淫泆,指神志过激,失去控制。离脏,指神气失藏而耗散于外。
  \item 天之在我者德也,地之在我者气也:德,此作“生机”、“规律”解。《易·系辞下》说:“天地之大德曰生。”张介宾注:“肇生之德本乎天。”气,指构成形体的物质。张介宾注:“成形之气本乎地。”
  \item 德流气薄而生者也:薄,通“搏”。德流气薄,即天之生机与地之精气相结合。生,指产生了新的生命活动个体。
  \item 生之来谓之精:谓孕育新生命的原始物质叫做精。黄元御注:“精者,生化之始基也,故生之方来谓之精,人身形象之根源,神气之室宅也。”
  \item 两精相搏谓之神:张介宾注:“两精者,阴阳之精也搏,交结也。……凡万物生成之道,莫不阴阳交而后神明见。故人之生也,必合阴阳之气,构父母之精,两精相博,形神乃成。所谓天地合气,命之曰人也。”可见,男女两精结合而产生的生命活动叫做神。
  \item 随神往来者谓之魂,并精而出入者谓之魄:神,神气,此指精神话动。精乃构成形体的基本物质,此代指形体。并,作依附解。张介宾注:“精对神而言,则神为阳而精为阴;魄对魂而言,则魂为阳而魄为阴,故魂则随神而往来,魄则并精而出入。……盖神之为德,如光明爽朗,聪慧灵通之类皆是也。魂之为言,如梦寐恍惚,变幻游行之境皆是也。神藏于心,故心静则神清;魂随乎神,故神昏则魂荡。……盖精之为物,重浊有质,形体因之而成也。魄之为用,能动能作,痛痒由之而觉也。精生于气,故气聚则精盈;魄并于精,故形强则魄壮。”可见,全句谓伴随神气而产生的谋虑、梦幻等精神活动,称为“魂”;依附形体而产生的本能感觉和动作等功能活动·称为“魄”。
  \item 所以任物者谓之心:任,担当、接受。任物,接受、处理事物。张介宾注:“心为君主之宫,统神灵而参天地,故万物皆其所任。”
  \item 心有所忆谓之意:意,意念,为思维活动的开始。张介宾注:“忆,思忆也。谓一念之生,心有所向而未定者,曰意。”
  \item 意之所存谓之志:意念确定,志向形成,叫做志。李中梓注:“意已决而确然不变者,志也。”
  \item 因志而存变谓之思:围绕已定的志向反复计度、比较称为思。李中梓注:“志虽定而反复计度者,思也。”
  \item 因思而远慕谓之虑:由近及远,多方分析,深谋远虑,称为虑。杨上善注:“求变之思,逆慕将来,谓之虑也。”
  \item 因虑而处物谓之智:在深谋远虑的基础上正确地处理事物,称为智。李中梓注:“虑而后动,处事灵巧者,智也。”
  \item 故智者之养生也:智者,善于正确思考而处理事物的明智者。此指善于养生的人。养生,保养生命,防病抗衰,延年益寿的意思。又称为“摄生”、“道生”。
  \item 节阴阳而调刚柔:阴阳,此指男女。刚柔,代指阴阳。全句谓节制房事,调和阴阳。
  \item 僻邪:僻,邪也。僻邪,同义复词,即邪气。
  \item 生长久视:视,活也。长生久视,即健康长寿的意思。
\end{jiaozhu}

\biaoti{【理论阐释】}

1.神的概念

神的概念,十分广泛而丰富,就其内容来看,主要包括以下三个方面。

(1)指自然界事物的运动变化及规律:万事万物都无时无刻不处于运动变化之中,其运动变化及其规律性,均可以用“神”来概括。如《素问·天元纪大论》说:“故物生谓之化,物极谓之变,阴阳不测谓之神,神用无方谓之圣。”王冰注:“由圣与神,故众妙无能出幽玄之理。深乎妙用,不可得而称之。”可见,万事万物的生长变化,均源于内部的阴阳运动,因其运动变化的高深莫测,故以神称之。《素问·阴阳应象大论》说:“阴阳者,天地之道也,……神明之府也。”此即以事物内部玄妙莫测的阴阳变化规律,称为“神”。

(2)对人体生命活动及现象的高度概括:本段“两精相搏谓之神”,以及《灵枢·天年》“何者为神?岐伯曰:血气已和,荣卫已通,五脏已成,神气舍心,魂魄毕具,乃成为人。”其神的含义即指人体的生命活动及其外在表现。

(3)指人的精神、意识、思维、情志活动:此神乃狭义之神。如《素问·灵兰秘典论》“心者,君主之官也,神明出焉”之“神”,《素问·宣明五气》“心藏神”之“神”,本段之“神、魂、魄、心、意、志、思、智、虑”,以及《素问·阴阳应象大论》“人有五脏化五气,以生喜怒悲忧恐”之“喜怒悲忧恐”等均属此类。

2.思维活动的过程

人类的思维活动,是在心的主导之下,由五脏配合完成的。故原文说:“所以任物者谓之心”。《内经》将整体思维过程分为五个阶段,并分别用五个字来进行概括。

(1)意;“心有所忆谓之意”。心接受事物,并且对事物产生初步印象或念头,叫做意。此为思维活动的第一个阶段。

(2)志:“意之所存谓之志”。根据意念而确定的志向或打算,叫做志。此为思维活动的第二个阶段。

(3)思:“因志而存变谓之思,根据所立的志愿对事物反复地分析、比较,叫做思。此为思维活动的第三个阶段。

(4)虑:“因思而远慕谓之虑”。通过周密思考以计划未来的行动,叫做虑。此为思维活动的第四个阶段。

(5)智:“因虑而处物谓之智”。在深谋远虑的基础上正确地处理事物,叫做智。此为思维活动的第五个阶段。

本段对思维过程的论述,与现代心理学所表达的认知活动包括感觉、知觉、记忆、比较、分析、综合、判断等过程十分近似。如“任物”,相当于感觉;“意”、“志”,相当于知觉、记忆;“思”,相当于比较、分析;“虑”,相当于综合;“智相当于判断。充分反映出古人对思维过程的重视及研究水平。

\biaoti{【临证指要】}

\xiaobt{“凡刺之法,先必本于神”}

神是生命活动的主宰及集中体现,针刺及其它治法的使用,都必须在充分调动和发挥神气作用的前提下,才能取得最佳治疗效果。因此本篇开篇即提出“凡刺之法,先必本于神”。本文虽言针刺,但实则赅药物、推拿等治法在内。下段原文在讨论了五志(七情)伤脏之后,再次强调“是故用针者,察观病人之态,以知精神魂魄之存亡,得失之意,五者以伤,针不可以治之也。”“五者以伤”,本指五脏精气与神气两伤,但结合“精神魂魄之存亡”来看,主要当指神气。说明五脏神气的损伤,会对治疗造成十分严重的后果。

《素问·汤液醪醴论》对患者精神心理状态在治疗中的作用进行了讨论,指出:“帝曰:形弊血尽而功不立者何?岐伯曰:神不使也。帝曰:何谓神不使?岐伯曰:针石,道也。精神不进,志意不治,故病不可愈。今精坏神去,荣卫不可复收。何者?嗜欲无穷,而忧患不止,精气㢮坏,荣涩卫除,故神去之而病不愈也。”可见,“精神不进,志意不治”,“嗜欲无穷,而忧患不止”等不良的精神心理状态,可导致“精气㢮坏,荣涩卫除”等心身变化,从而影响治疗效果。

此外,“本于神”之“神”,除了指病人五脏的神气及精神心理状态外,还应当包括医生的神气及精神心理状态在内。医生之神既可以提高辨证论治的质量,又可调动病人的神气,并激发病人产生良好的精神心理状态,因此也是临证时所必须重视的问题之一。如针刺的“得气”是疗效好坏的关键,而“得气”与否,以及“得气”的质量等均与病人和医生的神气密切相关。病人信任医生,密切配合,保持良好心态;医生神志专注,操作精细入微,便容易“得气”,并且“得气”的质量高。否则,则结果完全相反。故《灵枢·九针十二原》说:“小针之要,易陈而难入,粗守形,上守神。”

基于上述,在临床上要以察神治神为首务,以“本于神”的精神,提高治疗效果。具体方法,一是察病人之神,以了解其机体状态。故原文说:“是故用针者,察观病人之态”。二是调动病人之情,解除心理障碍,充分发挥其主观能动作用。此亦即《素问·汤液醪醴论》所说“病为本,工为标”,“标本不得,邪气不服”的意思。三是用目光制约、调理病人的心理活动,促进其气血流通,从而为治疗服务。故《素问·针解》说:“必正其神者,欲瞻病人,目制其神,令气易行也。”四是调整医生之神,使医患之神高度统一,提高治疗效果。如《素问·宝命全形论》说:“凡刺之真,必先治神,五脏已定,九候已备,后乃存针,众脉不见,众凶弗闻,内外相得,无以形先,可玩往来,乃施于人。……经气已至,慎守勿失,深浅在志,远近若一,如临深渊,手如握虎,神无营于众物。”说明医生的全神贯注,志定神凝是治疗的重要条件。

\biaoti{【原文】}

\begin{yuanwen}
是故怵惕思慮\sb{1}者則傷神,神傷則恐懼,流淫而不止\sb{2}。因悲哀動中者,竭絕而失生\sb{3}。喜樂者,神憚散而不藏\sb{4}。愁憂者,氣閉塞而不行\sb{5}。盛怒者,迷惑而不治\sb{6}。恐懼者,神蕩憚而不收\sb{7}。

心怵惕思慮則傷神,神傷則恐懼自失\sb{8},破䐃脫肉\sb{9},毛悴色夭\sb{10},死於冬\sb{11}。脾愁憂而不解則傷意,意傷則悗亂\sb{12},四肢不舉,毛悴色夭,死於春。肝悲哀動中則傷魂,魂傷則狂忘不精,不精則不正\sb{13},當人陰縮而筋攣,兩脅骨不舉\sb{14},毛悴色夭,死於秋。肺喜樂無極則傷魄,魄傷則狂,狂者意不存人\sb{15},皮革焦,毛悴色夭,死于夏。腎盛怒而不止則傷志,志傷則喜忘其前言,腰脊不可以俛仰屈伸\sb{16},毛悴色夭,死于季夏\sb{17}。恐懼而不解則傷精\sb{18},精傷則骨痠痿厥\sb{19},精時自下。是故五藏主藏精者也,不可傷,傷則失守而陰虚\sb{20},陰虚則無氣,無氣則死矣\sb{21}。是故用鍼者,察観病人之態,以知精神魂魄之存亡得失之意,五者以傷\sb{22},鍼不可以治之也。
\end{yuanwen}

\biaoti{【校注】}

\begin{jiaozhu}
  \item 怵惕思虑:张介宾注:“怵,恐也。惕,惊也。”怵惕思虑,即惊恐、焦虑。
  \item 流淫而不止:流淫,指滑精之类。张介宾注:“流淫,谓流泄淫溢,如下文所云‘恐惧而不解则伤精’、‘精时自下’者是也。思虑而兼怵惕,则神伤而心怯,心怯则恐惧,恐惧则伤肾,肾伤则精不固。盖以心肾不交,故不能收摄如此。”
  \item 因悲哀动中者,竭绝而失生:因,可据《太素》删,与前后句文例一律。中,内脏。生,生机。全句谓悲哀太过,损伤内脏,可导致精气竭绝而丧失生机。张介宾注:“竭者绝之渐,绝则尽绝无余矣。”
  \item 喜乐者,神惮散而不藏:惮散,即涣散之意。全句意为喜乐太过,心神涣散而失于内藏。黄元御注:“喜乐伤心,君火升泄,故神明惮散而不藏。”
  \item 愁忧者,气闭塞而不行:黄元御注:“愁忧伤脾,中气不运,故土气闭塞而不行。”
  \item 盛怒者,迷惑而不治:迷惑,指神识迷乱不清。张介宾注:“怒则气逆,甚者必乱,故致昏迷惶惑而不治。不治,乱也。”
  \item 神荡惮而不收:神气动荡耗散而不能收持。
  \item 恐惧自失:因恐惧而失去自控能力。
  \item 破䐃脱肉:指肌肉消瘦、脱陷。张介宾注:“䐃者,肉结聚之处。心虚则脾虚,故破䐃脱肉。”
  \item 毛悴色夭:皮毛憔悴,色泽枯槁。
  \item 死于冬:张介宾注:“火衰畏水,故死于冬。”下文类此。
  \item 愁忧不解则伤意,意伤则悗乱:悗乱,心胸烦乱。张介宾注:“忧则脾气不舒,不舒则不能运行,故悗闷而乱。”
  \item 魂伤则狂忘不精,不精则不正:忘,《太素》、《甲乙经》均作“妄”,可从。狂妄不精,指神志狂乱,呆滞愚笨。不正,言行举止失常。张介宾注:“魂伤则为狂为忘而不精明,精明失则邪妄不正。”
  \item 当人阴缩而筋挛,两胁骨不举:当,副词,作“则”字解。不举,《太素》、《千金要方》均无“不”字,可从。杨上善注:“肝足厥阴脉环阴器,故魂肝伤,宗筋缩也。肝又主诸筋,故挛也。肝在两胁,故肝病两胁骨举也。”
  \item 魄伤则狂,狂者意不存人:张介宾注:“喜本心之志,而亦伤肺者,暴喜伤阳,火邪乘金也。肺藏魄,魄伤则神乱而为狂。意不存人者,旁若无人也。”
  \item 志伤则喜忘其前言,腰脊不可以俛仰屈伸:杨上善注:“肾志伤,故喜忘。肾在腰脊之中,故肾病不可以俯仰屈伸也。”
  \item 季夏:农历六月,即长夏,为湿土主事之时。
  \item 恐惧而不解则伤精:张介宾注:“盖盛怒虽云伤肾,而恐惧则肾脏之本志,恐则气下而陷,故能伤猜。”
  \item 骨痠痿厥:痿厥,偏义复词,此指肢体痿弱、无力运动的痿证。
  \item 伤则失守而阴虚:伤五脏则精气失守,精属阴,故云阴虚。张介宾注:“此总结上文而言五脏各有其精,伤之则阴虚,以五脏之精皆阴也。”
  \item 阴虚则无气,无气则死矣:张介宾注:“阴虚则无气,以精能化气也。气聚则生,气散则死。”
  \item 五者以伤:“以”,通“已”。五者,指五脏的精气和神气。全句言五脏的精气和神气俱已耗伤。
\end{jiaozhu}

\biaoti{【理论阐释】}

\xiaobt{情志太过伤脏的规律}

七情虽产生于五脏,但七情太过反过来又会损伤五脏。其所伤的规律是:

(1)五志首先伤心:情志发于心,复为心所统,故五志过激均可首先伤及心脏,而表现出心功能失调的病证。如“怵惕思虑则伤神,神伤则恐惧,流淫而不止”;“喜乐者,神惮散而不藏”;“盛怒者,迷惑而不治”;“恐惧者,神荡惮而不收”。说明不仅喜乐太过伤心,而且怵惕思虑、盛怒、恐惧等情志太过亦可伤及心脏。

(2)五志自伤本脏:五志分属于五脏,五志过激,多可伤及本脏。具体所伤是,喜伤心,怒伤肝,悲忧伤肺,思虑伤脾,惊恐伤肾。如本段“恐惧而不解则伤精”,即属五志自伤本脏之例。

(3)五志互伤它脏:五志伤心和五志自伤本脏,属于五志伤脏的一般规律;而本处所论五志互伤它脏,则属于五志伤脏的特殊情况。如本段的“怵惕思虑”,伤心神;“愁忧不解”,伤脾意;“悲哀动中”,伤肝魂;“喜乐无极”,伤肺魄;“盛怒不止”,伤肾志,便均属于它脏之志伤及本脏之神。从其所伤来看,似缺乏规律性,难以掌握,但依据“邪之所凑,其气必虚”这一论点,可以认为,脏虚之处,即为情伤之所。因此,五志互伤它脏,多为本脏不虚而它脏先虚所致。

\biaoti{【原文】}

\begin{yuanwen}
肝藏血,血舍魂,肝氣虛則恐,實則怒\sb{3}。心藏脈\sb{4},脈舍神,心氣虛則悲,實則笑不休。肺藏氣\sb{7},氣舍魄,肺氣虛則鼻塞不利\sb{6},少氣,實則喘喝胸盈仰息\sb{7}。腎藏精,精舍志,腎氣虛則厥,實則脹,五藏不安。必審五藏之病形,以知其氣之虛實,謹而調之也。
\end{yuanwen}

\biaoti{【校注】}

\begin{jiaozhu}
  \item 肝气虚则恐,实则怒:肝气虚则血亏魂怯,故恐;肝气实则本志有余,故怒。
  \item 脾藏营:指脾化生水谷精气,而水谷精气又为营血化生之源。张介宾注:“营出中焦,受气取汁,变化而赤是谓血,故曰脾藏营。”
  \item 经溲不利:“经”,《甲乙经》作“泾”,当从。泾,指小便。溲,为二便之通称。泾溲不利,即二便不利。
  \item 心藏脉:藏,作“主持”理解。心藏脉,即心主血脉。
  \item 肺藏气;藏,主持之意。肺藏气,即肺主气。
  \item 鼻塞不利:此四字,《素问·调经论》王冰注引文等均作“鼻息利”,可从。
  \item 喘喝胸盈仰息:喘促而喝喝有声,胸部胀满,仰面喘息。形容邪气壅肺,肺气不利而呼吸高度困难的状态。张介宾注:“胸盈,胀满也。仰息,仰面而喘也。”
\end{jiaozhu}

\biaoti{【理论阐释】}

\xiaobt{神与脏腑}

神是在脏腑精气的基础上产生的,精气充养脏腑、组织、器官等形体便产生了神的活动。因此,形、神是构成生命活动的两大基本要素,或者说精、气、神是构成生命活动的三大要素。精气化生于脏腑,藏于五脏,所以神与五脏的关系尤为密切。神与脏腑的关系,可以概括为下述三个方面。

(1)神主于心:心藏神,心为神的主宰。《内经》反复强调心与神的密切关系。如《素问·灵兰秘典论》说:“心者,君主之官,神明出焉。”《素问·六节藏象论》说:“心者,生之本,神之变也。”《素问·宣明五气篇》说:“心藏神”。张介宾在《类经·藏象类》中作了进一步的阐释,谓:“人身之神,惟心所主,故本经曰:心藏神。又曰:心者君主之官,神明出焉。此即吾身之元神也。外如魂魄志意五神五志之类,孰匪元神所化而统乎一心?是以心正则万神俱正,心邪则万神俱邪,迨其变态,莫可名状。”可见,张氏认为凡神皆心所主,而其表现则千变万化,多彩多姿。

(2)神分属于五赃:神由精气所化,精气藏于五脏,所以神又分属于五脏,而五脏又有“五神脏”之称。正如王冰注《素问·六节藏象论》所说,“神脏五者:一肝,二心,三脾,四肺,五肾也,神藏于内,故以名焉。”神与五脏的配属关系,即本篇所言:“肝藏血,血舍魂”;“脾藏营,营舍意”;“心藏脉,脉舍神”;“肺藏气,气舍魄”;“肾藏精,精舍志”。亦《素问·宣明五气篇》所说:“心藏神,肺藏魄,肝藏魂,脾藏意,肾藏志,是谓五脏所藏”。

(3)神寄于脑髓:脑髓属奇恒之府,因其为肾精所生,故与神亦有密切的关系。《素问·脉要精微论》说:“头者,精明之府,头倾视深,精神将夺矣。”脑与神的关系,除了肾藏精,精生髓,脑为髓海之外,还与五脏六腑的精气均上充于脑密切相关。“头者,精明之府”,便指出了脑为五脏六腑精气、神明会聚的地方。但是,以《内经》的本义而论,脑为奇恒之府,仍属肾及五脏所主,在功能系统上则隶属于肾系统。

后世对脑髓与神的认识,在《内经》论述的基础上不断有所深化。孙思邈《千金要方》说:“头者,人之元首,人神之所注。”李时珍《本草纲目·辛夷》说:“脑为元神之府”。张志聪《素问集注》亦说:“诸阳之神气,上会于头,诸髓之精上注于脑,故头为精髓神明之府。髓海不足,则头为之倾,神气衰弱。”清·程文囿《医述》引《会心录》还说:“盖脑为神脏,谓之泥丸官,而精髓藏焉。……脑脏病,则神气失守。”王清任在《医林改错》中更是力排众说,直谓“灵机记性,不在心在脑”。

关于心、脑在生神、主神中的作用与地位,仍是一个有待继续研究探讨的问题。城然,脑藏精生神,是与神生成密切相关的脏腑之一,但是其精气源于肾和五脏,且脑不参与化生精气,因此,脑在生神、主神中的作用和地位,是远远无法与五脏相比的。也就是说,脑在生神、主神方面处于从属于五脏尤其是心、背的地位。决不可因有“脑为元神之府”、“脑为神脏”及“灵机记性不在心在脑”等说法,而将“脑神说”凌驾于“心神说”之上。

(4)神与胆相关:胆属六腑之一,又属于奇恒之腑,因其内藏精汁,故可生神而主决断。神在决断方面的功能则主要分属于胆。《素问·灵兰秘兰论》论:“胆者,中正之官,决断出焉。”因胆具“决断”之能,复通心气之用,故可协助心调节五脏六腑的神志等功能活动,而又有“凡十一脏,取决于胆也”之说。

\section{靈樞·營衛生會}%第七節

\biaoti{【原文】}

\begin{yuanwen}
黃帝問于岐伯曰:人焉受氣?陰陽焉會?何氣爲營?何氣爲衛?營安從生?衛於焉會?老壯不同氣,陰陽異位,願聞其會。岐伯答曰:人受氣于谷,榖入於胃,以傳與肺\sb{1},五藏六府,皆以受氣,其清者爲營,濁者爲衛\sb{2},營在脈中,衛在脤外,營周不休,五十而復大會\sb{3}。陰陽相貫,如環無端\sb{4}。衛氣行于陰二十五度,行于陽二十五度\sb{5},分爲畫夜,故氣至陽而起,至陰而止\sb{6}。故曰:日中而陽隴\sb{7}爲重陽,夜半而陰隴爲重陰。故太陰主內,太陽主外\sb{8},各行二十五度,分爲畫夜。夜半爲陰隴,夜半後而爲\sb{9}陰衰,平旦陰盡而陽受氣\sb{10}矣.日中爲陽隴,日西而陽衰,日入陽盡而陰受氣矣。夜半而大會,萬民皆臥,命曰合陰\sb{11}。平旦陰盡而陽受氣,如是無已,與天地同紀\sb{12}。

黃帝曰:老人之不夜瞑者,何氣使然?少壯之人不畫瞑者,何氣使然?岐伯答曰:壯者之氣血盛,其肌肉滑,氣道通\sb{13},營衛之行,不失其常,故畫精\sb{14}而夜瞑。老人之氣血衰,其肌肉枯,氣道澀,五藏之氣相搏\sb{15},其營氣衰少而衛氣內伐\sb{16},故畫不精,夜不瞑。
\end{yuanwen}

\biaoti{【校注】}

\begin{jiaozhu}
  \item 以传与肺:“以”,《甲乙经》及王冰注《素问·平人气象论》引《灵枢》文均作“气”义胜,可从。意为水谷精气经脾上传于肺。
  \item 清者为营,浊者为卫:清、浊,此指气的性质刚柔而言。唐宗海《中西汇通医经精义》说:“清浊以刚柔言,阴气柔和为清,阳气刚悍为浊。”
  \item 五十而复大会:指营气在一昼夜运行五十周次后,而与卫气会合一次。因是五十周次后的会合,故称大会。张介宾注:“营气之行,周流不休,凡一昼一夜五十周于身而复为大会。”
  \item 阴陌相贯,如环无端:张介宾注:“其十二经脉之次,则一阴一阳,一表一里,迭行相贯,终而复始,故曰如环无端也。”
  \item 卫气行于阴二十五度,行于阳二十五度:阴、阳乃相对而言,此处阳指体表,阴指体内脏腑。二十五度,即二十五周。
  \item 气至阳而起,至阴而止:卫气行至阳分体表,则人起而活动;行至阴分体内,则人卧而休息。张志聪注:“气至阳则卧起而目张,至阴则体止而目瞑。”
  \item 陇:通“隆”,盛也。
  \item 太阴主内,太阳主外:张介宾注:“内言营气,外言卫气。营气始于手太阴而复会于太阴,故太阴主内。卫气始于足太阳而复会于太阳,故太阳主外。”
  \item 为:《甲乙经》无,可从。
  \item 平旦阴尽而阳受气:平旦之时,卫气从阴出于阳,阴分卫气已衰,而阳分开始接受卫气。
  \item 夜半而大会,万民皆卧,命曰合阴:夜半子时为阴气最盛之时,营卫二气俱行于阴而大会,故曰合阴。
  \item 与天地同纪:纪,法则,规律。全句谓卫气的运行与天地阴阳的变化规律相一致。
  \item 气道通:营卫之气运行的道路通畅。
  \item 昼精:白天精力充沛,精神饱满。
  \item 五脏之气相搏:搏,《医统》本作“搏”,可从。相搏,即不相调和。五脏之气相搏,即五脏的功能不相协调。
  \item 营气衰少而卫气内伐:营气衰少,指营卫俱衰。卫气内伐,指卫气内扰而营卫运行紊乱。
\end{jiaozhu}

\biaoti{【理论阐释】}

1.营卫的运行

营卫皆为水谷精气所化,“清者为营,浊者为卫”,由于其气的性质不同,所以营卫的运行路线和规律也各不相同。

(1)营气的运行:“营在脉中”,昼夜运行于人身五十周次,终而复始。《灵枢·营气》说:“营气之通,内谷为宝,谷入于胃,乃传之肺,流溢于中,布散于外,精专者行于经隧,常营无已,终而复始。”其具体的循行路线为:“故气从太阴出,注手阳明,……上行至肝,从肝上注肺,上循喉咙,入颃颡之窍,究于畜门。其支别者,上额循巅下项中,循脊入骶,是督脉也,络阴器,上过毛中,入脐中,上循腹里,入缺盆,下注肺中,复出太阴。此营气之所行也,逆顺之常也。”可见,营气运行的主体路线是循十二经脉之顺序运行,始于肺,终于肝,复还于肺。其支别的路线则是从足厥阴别出,循督脉,过任脉,复入于手太阴经。

此外,《灵枢·脉度》及《灵枢·五十营》等篇还提出了营气行于二十八脉之说。如《灵枢·五十营》说:“黄帝曰:余愿闻五十营奈何?岐伯答曰:天周二十八宿,……日行二十八宿,人经脉上下、左右、前后二十八脉。”这二十八脉,是指十二经脉左右各一、任督脉各一、跷脉左右各一。但跷脉有阴跷、阳跷二脉,其计数方法是,男子只计阳跷脉,女子只计阴跷脉。计数者为经脉,包括在二十八脉之内;不计数者为络脉,排除在二十八脉之外。故《灵枢·脉度》说:“男子数其阳,女子数其阴,当数者为经,不当数者为络也。”(图\ref{fig:营气运行图})

%\begin{figure}[htb]%营气运行图
%  \centering
%  % Requires \usepackage{graphicx}
%  \includegraphics[width=0.7\textwidth]{营气运行图.pdf}\\
%  \caption{营气运行图}\label{fig:营气运行图}
%\end{figure}

\begin{figure}[htb]%营气运行图
  \centering
  \begin{tikzpicture}[
    %两个结点距离
    node distance=1.5,
    %连接线式样
    line/.style={->,shorten >=1pt,>=stealth',},   %粗 thick
    line2/.style={<-,shorten >=1pt,>=stealth',},
    dotted line/.style={->,shorten >=1pt,>=stealth',dashed},
    % hv path 表示一个结点到另一个结点是先水平再垂直。vh 相反。
    %skip loop 表示 垂直-水平-垂直 vskip loop 表示水平-垂直-水平
    hv path/.style={to path={-| (\tikztotarget)}},
    vh path/.style={to path={|- (\tikztotarget)}},
    %环连接
    v loop/.style={to path={-- ++(0,#1) -| (\tikztotarget)}},
    h loop/.style={to path={-- ++(#1,0) |- (\tikztotarget)}},
    ]
    %////////////////////////////////////////////////
    \matrix[row sep=10,column sep=30] {
      \node (fj) {手太阴肺经};    & \node (dcj) {手阳明大肠经}; & \node (wj) {足阳明胃经};   \\
      \node (xcj) {手太阳小肠经}; & \node (xj) {手少阴心经经};  & \node (pj) {足太阴脾经经}; \\
      \node (pgj) {足太阳膀胱经}; & \node (sj)  {足少阴肾经};   & \node (xbj) {手厥阴心包经};\\
      \node {}; &&\\
      \node (gj) {足厥阴肝经};    & \node (dj) {足少阳胆经};    & \node (sjj) {手少阳三焦经};\\
    };
    \node (zbz) [below left=.5of gj,yshift=-2mm] {支别者};
    \node (rm)  [left=1.8of xcj] {任脉};
    \node (dm)  [below=of rm] {督脉};
    \node (jm)  [below right=.08of pgj,yshift=0mm] {蹻脉};
    %///////////////////////////////////////////////
    \path
      (fj) edge [line] (dcj)
      (dcj) edge [line] (wj)
      (wj.east) edge [line,h loop=1] (pj.east)
      (pj) edge [line] (xj)
      (xj) edge [line] (xcj)
      (xcj.west) edge [line,h loop=-1] (pgj.west)
      (pgj) edge [line] (sj)
      (sj) edge [line] (xbj)
      (xbj.east) edge [line,h loop=.8] (sjj.east)
      (sjj) edge [line] (dj)
      (dj) edge [line] (gj)
      (gj.west) edge [line,h loop=-1.5] ($(fj.west)-(0,.8mm)$)
      (gj.south) edge [line,v loop=-.5] (dm.south)
      (dm.north) edge [line] (rm.south)
      (rm.north) edge [line,vh path] ($(fj.west)+(0,.5mm)$)
      (sj.south) edge [dotted line,vh path] (jm.east)
      (jm.west) edge [dotted line,hv path] (pgj.south)
    ;
  \end{tikzpicture}
  \caption{营气运行图}\label{fig:营气运行图}
\end{figure}

(2)卫气的运行:《内经》关于卫气运行的记载,散见于多篇之中。由于卫为水谷之悍气,不受脉道的约束,所以其运行路线呈现多样化特征。归纳起来有三个方面:

一是卫行脉外,与营并行。《灵枢·卫气》说:“其浮气之不循经者为卫气,其精气之行于经者为营气。阴阳相随,外内相贯,如环之无端。”说明卫气与营气阴阳相互依随,脉内外互相贯通,有如圆环之无端一样地运行不息。临床常见营血至则卫气亦至,营血虚则卫气亦不足的情况,亦反映了卫气偕同营气运行的状态。张志聪对此亦有明确论述,指出:“营卫相将,卫随营行者也。”

二是卫行脉外,昼行于阳,夜行于阴,各二十五周。《灵枢·卫气行》专篇论述了卫气的昼夜运行路线,说:“故卫气之行,一日一夜五十周于身,昼日行于阳二十五周,夜行于阴二十五周,周于五脏。”又说:“是故平且阴尽,阳气出于目,目张则气上行于头,循项下足太阳,……其始入于阴,常从足少阴注于肾,肾注于心,心注于肺,肺注于肝,肝注于脾,脾复注于肾为周。”这一条路线与营气的运行不相同。(图\ref{fig:卫气运行图})

%\begin{figure}[htb]%卫气运行图
%  \centering
%  % Requires \usepackage{graphicx}
%  \includegraphics[width=0.65\textwidth]{卫气运行图.pdf}\\
%  \caption{卫气运行图}\label{fig:卫气运行图}
%\end{figure}

\begin{figure}[htb]%卫气运行图
  \centering
  \begin{tikzpicture}[
    %两个结点距离
    node distance=1,
    %连接线式样
    line/.style={->,shorten >=1pt,>=stealth',},   %粗 thick
    line2/.style={<-,shorten >=1pt,>=stealth',},
    %环连接
    v loop/.style={to path={-- ++(0,#1) -| (\tikztotarget)}},
    h loop/.style={to path={-- ++(#1,0) |- (\tikztotarget)}}
    ]
    %////////////////////////////////////////////////
    \node (zty) {足太阳};
    \node (sty) [right=of zty] {手太阳};
    \node (zsy) [right=of sty] {足少阳};
    \node (sym) [below=.5 of zty] {手阳明};
    \node (zym) [right=of sym] {足阳明};
    \node (ssy) [right=of zym] {手少阳};
    \node (zsys) [below=2.5 of zty] {足少阴(肾)};
    \node (x) [right=of zsys] {心};
    \node (f) [right=of x] {肺};
    \node (p) [below right=0.5 of zsys,xshift=-10mm] {脾};
    \node (g) [right=1.8of p] {肝};
    \node (yq) [below left=.5of sym,xshift=5mm] {阳蹻};
    \node (ynq) [left=1.4of yq] {阴蹻};
    \node (zx) [above=.98of ynq,text width=4em] {昼行于阳二十五周};
    \node (yx) [below=.98of ynq,text width=4em] {夜行于阳二十五周};
    %///////////////////////////////////////////////
    \path
      (zty) edge [line] (sty)
      (sty) edge [line] (zsy)
      (zsy.east) edge [line,h loop=1] (ssy.east)
      (ssy) edge [line] (zym)
      (zym) edge [line] (sym)
      ($(sym.west)+(0,.25mm)$) edge [line,h loop=-1] (zty.west)
      (zsys) edge [line] (x)
      (x) edge [line] (f)
      (f.east) edge [line,h loop=1] (g.east)
      (g) edge [line] (p)
      ($(zsys.west)-(0,.25mm)$) edge [line2,h loop=-.59] (p.west)
      ($(sym.west)-(0,.25mm)$) edge [line, h loop=-1]  ($(zsys.west)+(0,.25mm)$)
      ($(zsys.west)-(.6,.5)$) edge [line,h loop=-.7] ($(zty.west)-(1,.5)$)
    ;
  \end{tikzpicture}
  \caption{卫气运行图}\label{fig:卫气运行图}
\end{figure}

三是卫行脉外,散行于肌肉、皮肤、胸腹、脏腑。如《灵枢·邪客》说:“卫气者,出其悍气之慓疾,而先行于四末分肉皮肤之间而不休者也。”《素问·痹论》亦说:“卫气者,水谷之悍气也。其气慓疾滑利,不能入于脉也,故循皮肤之中,分肉之间,熏于肓膜,散于胸腹。”可见,卫为悍气,具有“慓疾滑利”之性,因而不受约束而散行全身。

卫气的三种运行途径,体现了其分布广泛,运行迅速,应激能力强的特点,是完成温煦、卫外等功能的前提和基础。

2.营卫的会合

营卫在运行中的会合,包括营气自会、卫气自会及营卫交会等三种情况。

(1)营气自会:营气的运行,始于手太阴,终于足厥阴,而复会于手太阴。也就是说,营气运行一周后在手太阴相会一次,一昼夜相会五十次。亦如张介宾所说:“营气始于手太阴,而复会于太明。”

(2)卫气自会:卫气中昼行于阳、夜行于阴的部分,始于足太阳,终于足少阴,而复会于足太阳。卫气运行一周后在足太阳相会一次,一昼夜亦相会五十次。故张介宾说:“卫始于足太阳,而复会于太阳。”

(3)营卫交会:营卫在运行中,虽然“阴阳异位”,各行其道,然而二者并非绝对分开,互不相涉,而是相互贯通,不断交会的。其交会的形式主要表现在两个方面。

其一,营卫脉内外交会。营行脉中,卫行脉外,在运行中二气相互感应、贯通、交会。正如张介宾所说:“虽卫主气而在外,然亦何尝无血;营主血而在内,然亦何尝无气,故营中未必无卫,卫中未必无营,但行于内者便谓之营,行于外者便谓之卫,此人身阴阳交感之道,分之则二,合之则一而已。”营卫分行,不断交会,以互促互化,从而维持人体的勃勃生机。

其二,营卫五十而复大会。营卫在运行五十周次后有一次大的会合,便称为大会。原文说:“营在脉中,卫在脉外,五十而复大会。”又说:“夜半而大会、“常与营俱行于阳二十五度,行于阴亦二十五度,一周也,故五十度而复大会于手太阴矣。”可见,《内经》认为营卫的大会,是在各自运行五十周之后,于夜半子时而会合于手太阴。故张介宾说:“大会,言营卫阴阳之会也。营卫之行,表里异度,故尝不相值;惟于夜半子时,阴气已极,阳气将生,营气在阴,卫气亦在阴,……营卫皆归于脏,而会于天一之中也。”

\biaoti{【临证指要】}

\xiaobt{营卫运行与睡眠的关系及意义}

营卫运行与睡眠的关系十分密切,正如原文所说:“老人不夜瞑者,何气使然?少壮之人不昼瞑者,何气使然?岐伯答曰:壮者之气血盛,其肌肉滑,气道通,营卫之行,不失其常,故昼精而夜瞑。老者之气血衰,其肌肉枯,气道涩,五脏之气相搏,其营气衰少而卫气内伐,故昼不精,夜不瞑。”说明人体正常睡眠的条件是气血盛,营卫强,气血运行之道通畅,营卫能够正常运行而阴阳相交。

营卫之气的运行,是昼行于阳,夜行于阴。行于阳则阳分之气盛,阳主动而兴奋,故白天精力充沛,神气健旺,而不思睡;行于阴则阴分之气盛,阴主静而抑制,故夜晚目瞑而寐,安然熟睡。故《灵枢·口问》说:“卫气昼日行于阳,夜半行于阴,阴者主夜,夜者主卧。”还说:“阳气尽,明气盛,则目瞑;阴气尽而阳气盛,则寤矣。”相反,如果营卫失调,运行之道不通,则营卫阴阳不能相交,故不寐或寐而不安。《灵枢·大惑论》说:“卫气不得入于阴,常留于阳。留于阳则阳气满,阳气满则阳跷盛,不得入于阴则阴气虚,故目不瞑矣。”《灵枢·邪客》亦说:“今厥气客于五脏六腑,则卫气独行其外,行于阳,不得入于阴。行于阳则阳气盛,阳气盛则阳跷陷;不得入于阴,阴虚,故目不瞑。”说明卫气不能入于阴分而与营气相交,导致阳分之气盛,而阴分之气虚是失眠的重要机理。由此推之,凡外感、内伤等因素,一旦扰乱了营卫的正常运行,均有可能导致失眠或嗜睡等证。

临床上,对于睡眠失常的病证,除了治心治肝等外,还应当辨证运用调和营卫之法。如《内经》用半夏秫米汤治失眠,《金匮要略》用桂枝龙骨牡蛎汤治失眠、梦交,《三因极一病证方论》用温胆汤治虚烦不眠、惊悸不宁等,皆与调和营卫之法有关。

\biaoti{【原文】}

\begin{yuanwen}
黃帝曰:願聞營衛之所行,皆何道從來?岐伯答曰:營出於中焦\sb{1},衛出於下焦\sb{2}。黃帝曰:願聞三焦之所出\sb{3}。岐伯答曰:上焦出於胃上口\sb{4},並咽\sb{5}以上貫膈而布胸中,走腋循太陰之分而行\sb{6},還至陽明,上至舌,下足陽明\sb{7}。常舆營俱行于陽二十五度,行于陰亦二十五度\sb{8},一周也\sb{9},故五十度而復大會于手太陰矣\sb{10}。黃帝曰:人有熱飲食下胃,其氣未定,汗則出\sb{11},或出於面,或出於背,或出於身半,其不循衛氣之道而出,何也?岐伯曰:此外傷於風,內開腠理,毛蒸理泄,衛氣走之\sb{12},固\sb{13}不得循其道。此氣慓悍滑疾,見開而出,故不得從其道,故命曰漏泄\sb{14}。

黃帝曰:願聞中焦之所出。岐伯答曰:中焦亦並胃中\sb{15},出上焦之後\sb{16},此所受氣者\sb{17},泌糟粕,蒸津液\sb{18},化其精微,上注於肺脈,乃化而爲血\sb{19},以奉生身,莫貴於此,故獨得行於經隧,命曰營氣。黃帝曰:夫血之與氣,異名同類,何謂也?岐伯答曰:營衛者精氣也\sb{20},血者神氣也\sb{21},故血之輿氣,異名同類焉。故奪血者無汗,奪汗者無血\sb{22},故人生有兩死而無兩生\sb{23}。

黃帝曰:願聞下焦之所出。岐伯答曰:下焦者,別回腸,注於勝胱而滲入焉\sb{24}。故水榖者,常並居於胃中,成糟粕而俱下於大腸,而成下焦,滲而俱下,濟泌別汁\sb{25},循下焦而滲入膀胱焉。黃帝曰:人飲酒,酒亦入胃,穀未熟\sb{26}而小便獨先下,何也?岐怕答曰:酒者熟榖之液也,其氣悍以清\sb{27},故後穀而入,先榖而液出\sb{28}焉。黃帝曰:善。余聞上焦如霧,中焦如漚,下焦如瀆\sb{29},此之謂也。
\end{yuanwen}

\biaoti{【校注】}

\begin{jiaozhu}
  \item 营出于中焦:出,输出。营出于中焦,营气从中焦输出。杨上善注:“故营出中焦者,出胃中口也。”
  \item 卫出于下焦:下·《太素》及《千金要方》均作“上”,可从,以与下文合。卫出于上焦,卫气从上焦输出。杨上善注:“卫出上焦者,出胃上口也。”张志聪注:“卫气,阳明水谷之悍气,从上焦而出,卫于表阳,故曰卫出上焦。”
  \item 三焦之所出:上、中、下三焦所输出水谷精气的情况。
  \item 胃上口:即胃上脘贲门处。
  \item 咽:此指食道。
  \item 循太阴之分而行:循,沿着。分,指循行部位。全句言卫气自腑中经腋,沿着手太阴肺经循行的部位运行。
  \item 下足阳明:指卫气从口舌部向下沿着足阳明胃经循行的部位运行,而进入卫气循环之中,故对以下循行路线略而未言。
  \item 常与营俱行于阳二十五度,行于阴亦二十五度:指卫气常与营气一起,昼行于体表阳分二十五周,夜行于体内阴分二十五周。
  \item 一周也:刘衡如《灵枢经》校勘本曰:“译文义,疑是后人沾注。”可参。
  \item 五十度而复大会于手太阴矣:张介宾注:“大会,言营卫阴阳之会也。”杨上善注:“故一日一夜行五十周,平旦会于手太阴脉也。”
  \item 其气未定,汗则出:其气未定,指饮食物在胃中尚未完成水谷精微的生化过程。汗则出,随汗而外出。
  \item 毛蒸理泄,卫气走之:风热侵入,蒸迫皮毛,腠理开泄,卫气外散。
  \item 固:《甲乙经》作“故”,可参。
  \item 漏泄:病证名,又称漏泄风。杨上善注:“言卫气勇急,遂不循其道,即出其汗,谓之漏泄风也。”张介宾注:“此即热食之气也,出不由度,故曰漏泄。”
  \item 中焦亦并胃中:胃中,《甲乙经》及《太素》均作“胃口”,可参。中焦亦并胃中,言中焦之营气亦从胃中(胃口)输出。
  \item 出上焦之后:张介宾注:“后,下也。”谓中焦输出营气的部位,在于上焦所出部位的下方。”
  \item 此所受气者:此,指中焦。受气,接受谷食之气。
  \item 泌糟粕,蒸津液:泌,滤出。蒸,蒸化、腐熟。黄元御注:“泌,分也。泌糟粕者,犹酒既酿熟,与糟粕分别之也。”
  \item 上注于肺脉,乃化而为血:张志聪注:“上注于肺脉,奉心神化赤而为血。”
  \item 营卫者精气也:张介宾注:“营卫之气虽分清浊,然皆水谷之精华,故曰营卫者精气也。”
  \item 血者神气也:张志聪注:“血者中焦之精汁,奉心神而化赤,神气之所化也。”
  \item 夺血者无汗,夺汗者无血:夺,大量丧失。无,通“毋”。“无”后之“汗'“血”,活用为动词。血、津、汗同源,津血互化,汗源于津,故血大伤者不要再发其汗,汗大出津伤者不要再伤其血。张介宾注:“然血化于液,液化于气,是血之与气,本为同类,而血之与汗,亦非两种。但血主营,为阴为里,汗属卫,为阳为表,一表一里,无可并攻,故夺血者无取其汗,夺汗者无取其血。”
  \item 人生有两死而无两生:“人”后之“生”字,《甲乙经》无,可从。两死,指夺血亡阴死,夺汗亡阳死。死,危重。全句意谓夺血亡阴和夺汗亡阳均可致死而无生机。张介宾注:“若表里倶夺,则不脱于阴,必脱于阳,脱阳亦死,脱阴亦死,故曰人生有两死。然人之生也,阴阳之气皆不可无,未有孤阳能生者,亦未有孤阴能生者,故曰无两生也。”
  \item 别回肠,注于膀胱而渗入焉:别回肠,指下焦所输出的津液自小肠与回肠相连接的阑门处别出。
  \item 而成下焦,渗而倶下,济泌别汁:“而成下焦,渗而俱下,济”,刘衡如《灵枢经》校勘本认为:“此九字《素问·咳论》王注无,疑是后人沾注,应加括号,则文义俱畅。”可参。泌别汁,即以过滤的方式而使津液从回肠别出。
  \item 谷未熟:指水谷尚未被胃完全腐熟、消化。
  \item 其气悍以清:清,《甲乙经》及《太素》等均作“滑”,义胜,可从。其气悍以滑,言酒为水谷酿成的精华之气,具有慓悍、滑利、急速的特性。
  \item 故后谷而入,先谷而液出:液,《太素》无,可从。张介宾注:“盖以酒之气悍,则直连下焦;酒之质清,故速行无滞。故后谷而入,先谷而出也。”
  \item 上焦如雾,中焦如沤,下焦如渎:上焦如雾,形容上焦心肺宣发敷布水谷精气的功能,如同雾露一样,弥漫灌溉周身。中焦如沤,形容中焦脾胃腐熟水谷、吸收输布精微的功能,如同沤渍食物一样,使之变化而出。下焦如渎,形容下焦肾膀胱,排泄水液糟粕的功能,如同沟渠一祥,畅通无阻。
\end{jiaozhu}

\biaoti{【临证指要】}

\xiaobt{“夺血者无汗,夺汗者无血”}

一般来看,血与汗似不相涉,但血汗同源,气化相通,故血与汗的关系实密不可分。血与汗通过津液作中介而相互影响。津血同源而异流,在运行输布过程中,可以互渗互化,即血渗络外而为津,津还络中而为血。《灵枢·邪气脏腑病形》说:“十二经脉三百六十五络,其血气皆上熏于面而走空窍,……其气之津液皆上熏于面。”血液在经络中运行,从孙络渗出于脉外,与脉外之津液化合,以濡润脏腑组织及皮肤肌腠为津液。同样,脏腑组织及皮肤肌腠的津液,亦可由孙络渗入经络之中,与经络中运行的血液化合,并在心气的作用下化赤为血。故《灵枢·痈疽》说:“肠胃受谷,……中焦出气如露,上注溪谷而渗孙脉,津液和调,变化而赤为血。”

由于血与津可以互渗互化,而汗又为津液蒸化出于体表而成,所以血便通过津液而与汗发生密切关系。一般而言,血足则津充而汗源充足,血虚则津少而汗源不充,为了避免“虚虚”之诫,所以原文强调“夺血者无汗,夺汗者无血。”其意是说大失血者,其津液已伤,不宜再发其汗;大汗伤津液者,其血液已伤,不宜再伤其血。临床上,对于大失血之人,不可再夺其汗,如《伤寒论》有“衄家不可汗”、“疮家不可汗”、“亡血家不可汗”之告诫。对于大出汗之人,则应避免再伤其血,如妄用活血化瘀、刺络出血等法。因特殊情况而非用不可者,则当辅以补血或益津之法。如对大失血而兼外感者,治宜养血益律,兼以发汗;对大出汗而兼血瘀者,治宜补血益津,兼以化瘀。

\section{靈樞·決氣}%第八節

\biaoti{【原文】}

\begin{yuanwen}
黃帝曰:余聞人有精、氣、津、液、血、脈,余意以爲一氣耳,今乃辨爲六名\sb{1},余不知其所以然\sb{2}。岐伯曰:兩神相搏\sb{3},合而成形,常先身生,是謂精\sb{4}。何謂氣?岐伯曰:上焦開發,宣五榖味\sb{5},熏膚充身澤毛,若霧露之溉,是謂氣\sb{6}。何謂津?岐伯曰:腠理發泄,汗出溱溱\sb{7},是謂津。何謂液?岐伯曰:榖入氣滿\sb{8},淖澤\sb{9}注於骨,骨屬\sb{10}屈伸,泄澤,補益腦髓,皮膚潤澤\sb{11},是謂液。何謂血?岐伯曰:中焦受氣取汁,變化而赤\sb{12},是謂血。何謂脈?壅遏營氣,令無所避\sb{13},是謂脈\sb{14}。

黃帝曰:六氣者,有餘不足,氣之多少,腦髓之虛實,血脈之清濁\sb{15},何以知之?岐伯曰:精脫者,耳聾\sb{16};氣脫者,目不明\sb{17};津脫者,腠理開,汗大泄\sb{18};液脫者,骨屬屈伸不利,色夭,腦髓消,脛痠,耳數嗚\sb{19};血脫者,色白,夭然不澤,其脈空虚\sb{20},此其候也。黃帝曰:六氣者,貴賤何如?

岐伯曰:六氣者,各有部主\sb{21}也,其貴賤善惡,可爲常主\sb{22},然五穀與胃爲大海也\sb{23}。
\end{yuanwen}

\biaoti{【校注】}

\begin{jiaozhu}
  \item 辨为六名:辨,别也。张介宾注:“六者之分总由气化,故曰一气。而下文云六气者,亦以形不同而名则异耳,故当辨之。”
  \item 然:其后应据《太素》补“愿闻何谓精?”五字,以使文义完整。
  \item 两神相搏:两神,指男女两性。搏,交合。两神相搏,指男女两性精气相合。杨上善注:“雄雌二灵之别,故曰两神。阴阳二神相得,故谓之薄(通‘搏’)。”
  \item 合而成形,常先身生,是谓精:精,指男女生殖之精。张介宾注:“凡阴阳合而万形成,无不先从精始,故曰常先身生是谓精。”
  \item 上焦开发,宣五谷味:上焦,主要指肺。开发,开通散发。宣,宣发布散。全句言肺宣发布散水谷精气于全身。张介宾注:“上焦,胸中也。开发,通达也。宣,布散也。”
  \item 若雾露之溉,是谓气:溉,灌溉。杨上善注:“若雾露之溉万物,故谓之气,即卫气也。”
  \item 汗出溱溱:溱溱,众盛貌。黄元御注:“涣然流漓之象。”汗出溱溱,形容汗出很多的样子。
  \item 谷入气满:气满,胃中水谷精气满溢。张介宾注:“谷入于胃,其气满而化液。”
  \item 淖泽:《灵枢经》音释:“淖,浊也。泽,液也。”淖泽,指水谷津液中质较稠浊的部分。
  \item 骨属:骨与骨连接处,即关节。
  \item 泄泽,补益脑髓,皮肤润泽:泄泽,即流出汁液。杨上善注:“五谷之精膏注于诸骨节中,其汁淖泽,因屈伸之动,流汁上补于脑,下补诸髓,旁益皮肤,令其润泽。”
  \item 中焦受气取汁,变化而赤:受气,接受水谷精气。取汁,吸取水谷精气中精专之汁。变化而赤,将精专之汁上送肺脉,经心化赤。杨上善注:“五谷精汁在于中焦,注于手太阴脉中变赤,循脉而行,以奉生身,谓之为血也。”
  \item 壅遏营气,令无所避:壅遏,约束。避,逃遁,引申为散失。杨上善注:“壅遏营血之气,日夜营身五十周,不令避散,故谓之脉也。”
  \item 脉:指约束营血使之正常运行的脉气。张介宾注:“非气非血,而所以通乎气血者也。”
  \item 血脉之清浊:指血液的有余不足,即血量的多与少,血质的清稀与稠浊。
  \item 精脱者,耳聋:脱,失去,此言虚之甚。张介宾注:“肾藏精,肾者耳之窍,故精脱则耳聋。”
  \item 气脱者,目不明:张介宾注:“五脏六腑精阳之气,皆上注于目而为晴,故阳气脱则目不明。”
  \item 津脱者,腠理开,汗大泄:此言腠理开、汗大泄,既是津脱的表现,又是导致津脱的原因。
  \item 液脱者,骨属屈伸不利,色夭,脑髓消,胫痠,耳数鸣:张介宾注:“液所以注骨益脑而泽皮肢者,液脱则骨髓无以充,故屈伸不利而脑消胫痠。皮肤无以滋,故色枯而夭。液脱则阴虚,故耳鸣也。”
  \item 其脉空虚:此前,《甲乙经》有“脉脱者”三字可从,以全六脱之候。
  \item 六气者,各有部主:谓六气分别有其主管的部位和所主的脏腑。张介宾注:“部主谓各部所主也。如肾主精,肺主气,脾主津液,肝主血,心主脉也。”
  \item 其贵贱善恶,可为常主:贵贱,言地位主次。善恶,言正常、异常。可为常主,是说六气的主次常变,分别由其所主的脏腑决定。
  \item 五谷与胃为大海也:谓六气以胃受纳的五谷作为生化的源泉。
\end{jiaozhu}

\biaoti{【理论阐释】}

1.气的概念

《内经》中气的概念十分广泛,凌耀星《实用内经词句辞典》认为,《内经》中气的所指包括极细微的物质、地球周围的大气、节气、气候、气势、呼吸之气、气质、病邪、药性、针刺的得气感、正气、水谷精气、元气、卫气、营气、宗气、五脏之气、脉气、矢气、神气、运气等21个方面。但归纳起来,《内经》中“气”的运用不外自然界与人体两大方面。在自然界,气主要指大气、六气(风、寒、暑、湿、燥、火)、六淫(六气太过)。在人体,气主要指4个方面内容:一是指体内流动着的精微物质,如水谷精气、元气等;二是指脏腑组织的功能活动,如五脏之气、六腑之气、经络之气等;三是指脏腑经络功能失调所出现的病理变化和症状,如肺气不降、胃气上逆、嗳气、矢气等;四是指体内存在的不正常之气,即邪气,如滞气、湿气、火热之气等。本篇所言之“六气”,既指体内的精微物质,如精、气、血、律、液,又指由精微物质构成的组织,如脉。由此可见,《内经》所论人体生理之气,有物质和功能之分,从物质而言,又有广义和狭义之分。狭义之气,仅指在体内流动着的属阳的极微细的气态精微物质,如元气、卫气、营气、宗气等。广义之气,则除此之外,还包括以下两方面内容:一是体内属阴的液态精微物质,如精、血、津、液;二是由上述各种精微物质构成的各种组织、器官及其生理功能,如筋气、脉气、骨气、肉气等,由于它们均由水谷精气这一气所化,故可以“气”称之。

2.律液的概念及其与精血的关系

津液是人体富有滋润濡养作用的正常液体。其中清而稀薄的为津,浊而稠厚的为液。津流动性大,主要布散于体表皮肤、肌肉和孔窍等部位,并能渗入脉中,以滋润全身。液流动性较小,灌注于骨节、脏腑、脑、髓等组织,以滋润脏腑组织。二是虽然在性质、分布上有所不同,但均源于水谷,化生于中焦脾胃,异名而同类,故津液往往联名并称。

津液与精血的关系,泛言之,它们之间源同流别,可以互生互化,具体而论,则津与血,液与精的关系最为密切。血在脉中流动,环周不休,津以其流动性大的特点,而同气相求,渗入于脉中,随血流动,并且在心气的作用下化赤为血。《灵枢·痈疽》说:“中焦出气如露,上注溪谷,而渗孙脉,律液和调,变化而赤为血。”同样,运行于脉中之血,从孙络渗出于脉外,与脉外的津液化合,便可成为津液。故《灵枢·邪气脏腑病形》说:“十二经脉三百六十五络,其血气皆上于面而走空窍,……其气之津液皆上熏于面。”精藏于肾,聚于脑髓,液以其流动性较小的特点,亦同气相求,灌注于骨节、脑、髓、脏腑之中,如《灵枢·五癃津液别》说:“五谷之津液,和合而为膏者,内渗入于骨空,补益脑髓,而下流阴股。”津液在滋润人体脏腑组织的过程中,其滑利关节的律液(主要指液),一部分渗入骨空,与髓液化合,补益脑髓,并下流于肾中,转化为肾精。基于同样的道理,肾精化髓充脑,其中一部分亦可与充养脑髓的液相合而化为液,由此可见,精中有液,液中有精,二者可以互促互化。《灵枢·口问》说:“液者,所以灌精濡空窍者也。”液能灌精,说明液中有精,液中之精华部分即是精。由于液能补脑益髓,并入肾化为精,所以精中有液,而且补精药多能滋液,如熟地、枸杞子、肉苁蓉、黄精等均能补精滋液。

\biaoti{【临证指要】}

1.“精脱者,耳聋”

肾开窍于耳,肾精耗脱,髓海空虚,耳失所养,就可出现耳鸣、耳聋之症。要确诊精脱之耳聋,必须有肾精亏虚的症状出现,即除耳聋之外,兼有而目晦暗,头目眩晕,腰膝酸软,遗精早泄,脉沉细等症。治宜补肾填精,益耳通窍。临床上一般可用六味地黄丸加减,可酌情加入柴胡、磁石(即耳聋左慈丸),以疏肝镇肝;或加枸杞、肉苁蓉、人参,以补肾益气;或加石菖蒱、远志,以化痰开窍。

耳聋之原因较为复杂,聋有久暴,病有虚实。一般来说,暴聋属实,多责在外感或痰热之邪阻闭耳窍,证有风、火、痰、瘀之别;久聋属虚,多责在肾虚,另与肝、脾、心、肺有关,证有肾肝气血之分。如《医钞类编·耳病门》说:“耳聋耳鸣,有痰,有火,有气虚,有阴虚,有肝火,少壮多属痰火,中年必是阴虚。”程文囿《医述·耳》亦说:“虚聋由渐而成,必有兼证可辨,如而颊黧黑者精脱,少气嗌干者肺虚,目善恐者肝虚,心神恍惚惊悸烦躁者心虚,四肢懶倦眩晕少食者脾虚。”说明精脱耳聋只是虚证耳聋中的主要证型之一,而且必有肾虚精亏之证可凭,切不可将精脱作为耳聋的惟一病机对待。

耳聋的治疗难度较大,特别是慢性耳聋往往非一日可愈,需要坚持较长时间的治疗。临床除内服中药外,一般应配合针灸疗法以提高疗效。常用穴位有听宫、听会、耳门、翳风、风池、中渚、外关、曲池等。

2.“气脱者,目不明”

气与目的关系十分密切,《素问·脉要精微论》说:“夫精明五色者,气之华也。”《灵枢·大惑论》亦说:“五脏六腑之精气,皆上注于目而为之精。”目“视万物,别白黑,审短长”(《素问·脉要精微论》)等视觉功能,全赖气之充养。如果元气亏虚或脏气久衰,目失气养,则可出现视物不明等症状。《类经·藏象类》注:“五脏六腑精阳之气,皆上注于目而为睛,故阳气脱则目不明。”可见,目之视觉功能与五脏之气的关系密切。在五脏之中,目与肝的关系又最为密切。《灵枢·脉度》说,“肝气通于目,肝和则目能辨五色矣。”目为肝所主的官窍,肝气、肝血的充养是维持目视觉功能正常的重要条件,一旦肝气血不足,目失所养,则可出现视物不明,甚至雀盲、暴盲等病证。故《素问·脏气法时论》说:“肝病者,目䀮䀮无所见。”《灵枢·天年》亦说:“五十岁,肝气始衰,肝叶始薄,胆汁始灭,目始不明。”

目不明临床上有急慢性之分,急性多由气血脱失所导致,气脱之目不明,当伴有头昏、气短、乏力、脉弱等气虚证,血脱之目不明,则伴有头昏心悸、面白无华、爪甲苍白、舌淡脉细等血虚证,而且均有造成急性气血脱失的因素存在。慢性则因肝肾阴虚,精血不足所致,常伴头目眩晕,耳鸣耳闭,腰膝酸软等症。治疗上,急性气脱者,可用补中益气汤加减,成独参汤治疗,血脱者宜用四物汤加参、芪化裁。慢性一般可用杞菊地黄丸合四物汤加减,秦伯未《中医临证备要》指出,若因视力减退而成为“远视”或“近视”,前人多从水火偏盛偏衰立论,认为不能远视乃气虚血盛,用定志丸(菖蒲、远志、茯神、人参);不能近视乃血虚气盛,用地芝丸(熟地、天冬、枳壳、菊花)。

\xiaojie

本章指出了藏象理论的概念与内容,选择了在记载藏象理论方面有代表性的《内经》文章八篇(含节选),内容涉及:1.脏与腑的区别,五脏、六腑、奇恒之腑的功能特点;2.肝、心、脾、肺、肾五脏和胆、胃、大肠、小肠、三焦、膀胱六腑的生理功能;3.天之五气入鼻藏于心肺,以养全身。地之五味入口之后,经过消化、精微输布、糟粕排出,维持人体正常生理活动;4.强调脾不主时及脾胃为气血化源、五脏六腑之海的特殊作用;5.上、中、下三焦的功能特点及其部位界线,营、卫二气的功能及其生成与运行;6.五脏所藏及精、神、魂、魄、意、志、思、虑、智的概念,以及神志受伤常见症状;7.精、气、津、液、血、脉的概念,相互关系及其虚实证候。

在全部论述中,突出“整体”的观点,不仅论述脏腑时采用“十二脏相使”和“四时五脏阴阳”的方法,即使气血津液等同样有“血之与气,异名同类”,精、气、津、液、血、脉“余意以为一气耳”之论。此外,重视“神”的观点,也是十分明确的。所谓:“凡刺之法,先必本于神”,“故用针者,察观病人之态,以知精神魂魄之存亡得失之意。”

\zuozhe{(邱幸凡)}
\ifx \allfiles \undefined
\end{document}
\fi