% -*- coding: utf-8 -*-
%!TEX program = xelatex
\ifx \allfiles \undefined
\documentclass[draft,12pt]{ctexbook}
%\usepackage{xeCJK}
%\usepackage[14pt]{extsizes} %支持8,9,10,11,12,14,17,20pt

%===================文档页面设置====================
%---------------------印刷版尺寸--------------------
%\usepackage[a4paper,hmargin={2.3cm,1.7cm},vmargin=2.3cm,driver=xetex]{geometry}
%--------------------电子版------------------------
\usepackage[a4paper,margin=2cm,driver=xetex]{geometry}
%\usepackage[paperwidth=9.2cm, paperheight=12.4cm, width=9cm, height=12cm,top=0.2cm,
%            bottom=0.4cm,left=0.2cm,right=0.2cm,foot=0cm, nohead,nofoot,driver=xetex]{geometry}

%===================自定义颜色=====================
\usepackage{xcolor}
  \definecolor{mybackgroundcolor}{cmyk}{0.03,0.03,0.18,0}
  \definecolor{myblue}{rgb}{0,0.2,0.6}

%====================字体设置======================
%--------------------中文字体----------------------
%-----------------------xeCJK下设置中文字体------------------------------%
\setCJKfamilyfont{song}{SimSun}                             %宋体 song
\newcommand{\song}{\CJKfamily{song}}                        % 宋体   (Windows自带simsun.ttf)
\setCJKfamilyfont{xs}{NSimSun}                              %新宋体 xs
\newcommand{\xs}{\CJKfamily{xs}}
\setCJKfamilyfont{fs}{FangSong_GB2312}                      %仿宋2312 fs
\newcommand{\fs}{\CJKfamily{fs}}                            %仿宋体 (Windows自带simfs.ttf)
\setCJKfamilyfont{kai}{KaiTi_GB2312}                        %楷体2312  kai
\newcommand{\kai}{\CJKfamily{kai}}
\setCJKfamilyfont{yh}{Microsoft YaHei}                    %微软雅黑 yh
\newcommand{\yh}{\CJKfamily{yh}}
\setCJKfamilyfont{hei}{SimHei}                                    %黑体  hei
\newcommand{\hei}{\CJKfamily{hei}}                          % 黑体   (Windows自带simhei.ttf)
\setCJKfamilyfont{msunicode}{Arial Unicode MS}            %Arial Unicode MS: msunicode
\newcommand{\msunicode}{\CJKfamily{msunicode}}
\setCJKfamilyfont{li}{LiSu}                                            %隶书  li
\newcommand{\li}{\CJKfamily{li}}
\setCJKfamilyfont{yy}{YouYuan}                             %幼圆  yy
\newcommand{\yy}{\CJKfamily{yy}}
\setCJKfamilyfont{xm}{MingLiU}                                        %细明体  xm
\newcommand{\xm}{\CJKfamily{xm}}
\setCJKfamilyfont{xxm}{PMingLiU}                             %新细明体  xxm
\newcommand{\xxm}{\CJKfamily{xxm}}

\setCJKfamilyfont{hwsong}{STSong}                            %华文宋体  hwsong
\newcommand{\hwsong}{\CJKfamily{hwsong}}
\setCJKfamilyfont{hwzs}{STZhongsong}                        %华文中宋  hwzs
\newcommand{\hwzs}{\CJKfamily{hwzs}}
\setCJKfamilyfont{hwfs}{STFangsong}                            %华文仿宋  hwfs
\newcommand{\hwfs}{\CJKfamily{hwfs}}
\setCJKfamilyfont{hwxh}{STXihei}                                %华文细黑  hwxh
\newcommand{\hwxh}{\CJKfamily{hwxh}}
\setCJKfamilyfont{hwl}{STLiti}                                        %华文隶书  hwl
\newcommand{\hwl}{\CJKfamily{hwl}}
\setCJKfamilyfont{hwxw}{STXinwei}                                %华文新魏  hwxw
\newcommand{\hwxw}{\CJKfamily{hwxw}}
\setCJKfamilyfont{hwk}{STKaiti}                                    %华文楷体  hwk
\newcommand{\hwk}{\CJKfamily{hwk}}
\setCJKfamilyfont{hwxk}{STXingkai}                            %华文行楷  hwxk
\newcommand{\hwxk}{\CJKfamily{hwxk}}
\setCJKfamilyfont{hwcy}{STCaiyun}                                 %华文彩云 hwcy
\newcommand{\hwcy}{\CJKfamily{hwcy}}
\setCJKfamilyfont{hwhp}{STHupo}                                 %华文琥珀   hwhp
\newcommand{\hwhp}{\CJKfamily{hwhp}}

\setCJKfamilyfont{fzsong}{Simsun (Founder Extended)}     %方正宋体超大字符集   fzsong
\newcommand{\fzsong}{\CJKfamily{fzsong}}
\setCJKfamilyfont{fzyao}{FZYaoTi}                                    %方正姚体  fzy
\newcommand{\fzyao}{\CJKfamily{fzyao}}
\setCJKfamilyfont{fzshu}{FZShuTi}                                    %方正舒体 fzshu
\newcommand{\fzshu}{\CJKfamily{fzshu}}

\setCJKfamilyfont{asong}{Adobe Song Std}                        %Adobe 宋体  asong
\newcommand{\asong}{\CJKfamily{asong}}
\setCJKfamilyfont{ahei}{Adobe Heiti Std}                            %Adobe 黑体  ahei
\newcommand{\ahei}{\CJKfamily{ahei}}
\setCJKfamilyfont{akai}{Adobe Kaiti Std}                            %Adobe 楷体  akai
\newcommand{\akai}{\CJKfamily{akai}}

%------------------------------设置字体大小------------------------%
\newcommand{\chuhao}{\fontsize{42pt}{\baselineskip}\selectfont}     %初号
\newcommand{\xiaochuhao}{\fontsize{36pt}{\baselineskip}\selectfont} %小初号
\newcommand{\yihao}{\fontsize{28pt}{\baselineskip}\selectfont}      %一号
\newcommand{\xiaoyihao}{\fontsize{24pt}{\baselineskip}\selectfont}
\newcommand{\erhao}{\fontsize{21pt}{\baselineskip}\selectfont}      %二号
\newcommand{\xiaoerhao}{\fontsize{18pt}{\baselineskip}\selectfont}  %小二号
\newcommand{\sanhao}{\fontsize{15.75pt}{\baselineskip}\selectfont}  %三号
\newcommand{\sihao}{\fontsize{14pt}{\baselineskip}\selectfont}%     四号
\newcommand{\xiaosihao}{\fontsize{12pt}{\baselineskip}\selectfont}  %小四号
\newcommand{\wuhao}{\fontsize{10.5pt}{\baselineskip}\selectfont}    %五号
\newcommand{\xiaowuhao}{\fontsize{9pt}{\baselineskip}\selectfont}   %小五号
\newcommand{\liuhao}{\fontsize{7.875pt}{\baselineskip}\selectfont}  %六号
\newcommand{\qihao}{\fontsize{5.25pt}{\baselineskip}\selectfont}    %七号   %中文字体及字号设置
\xeCJKDeclareSubCJKBlock{SIP}{
  "20000 -> "2A6DF,   % CJK Unified Ideographs Extension B
  "2A700 -> "2B73F,   % CJK Unified Ideographs Extension C
  "2B740 -> "2B81F    % CJK Unified Ideographs Extension D
}
%\setCJKmainfont[SIP={[AutoFakeBold=1.8,Color=red]Sun-ExtB},BoldFont=黑体]{宋体}    % 衬线字体 缺省中文字体

\setCJKmainfont{simsun.ttc}[
  Path=fonts/,
  SIP={[Path=fonts/,AutoFakeBold=1.8,Color=red]simsunb.ttf},
  BoldFont=simhei.ttf
]

%SimSun-ExtB
%Sun-ExtB
%AutoFakeBold:自动伪粗,即正文使用\bfseries时生僻字使用伪粗体;
%FakeBold:强制伪粗,即正文中生僻字均使用伪粗体
%\setCJKmainfont[BoldFont=STHeiti,ItalicFont=STKaiti]{STSong}
%\setCJKsansfont{微软雅黑}黑体
%\setCJKsansfont[BoldFont=STHeiti]{STXihei} %serif是有衬线字体sans serif 无衬线字体
%\setCJKmonofont{STFangsong}    %中文等宽字体

%--------------------英文字体----------------------
\setmainfont{simsun.ttc}[
  Path=fonts/,
  BoldFont=simhei.ttf
]
%\setmainfont[BoldFont=黑体]{宋体}  %缺省英文字体
%\setsansfont
%\setmonofont

%===================目录分栏设置====================
\usepackage[toc,lof,lot]{multitoc}    % 目录(含目录、表格目录、插图目录)分栏设置
  %\renewcommand*{\multicolumntoc}{3} % toc分栏数设置,默认为两栏(\multicolumnlof,\multicolumnlot)
  %\setlength{\columnsep}{1.5cm}      % 调整分栏间距
  \setlength{\columnseprule}{0.2pt}   % 调整分栏竖线的宽度

%==================章节格式设置====================
\setcounter{secnumdepth}{3} % 章节等编号深度 3:子子节\subsubsection
\setcounter{tocdepth}{2}    % 目录显示等度 2:子节

\xeCJKsetup{%
  CJKecglue=\hspace{0.15em},      % 调整中英(含数字)间的字间距
  %CJKmath=true,                  % 在数学环境中直接输出汉字(不需要\text{})
  AllowBreakBetweenPuncts=true,   % 允许标点中间断行,减少文字行溢出
}

\ctexset{%
  part={
    name={,篇},
    number=\SZX{part},
    format={\chuhao\bfseries\centering},
    nameformat={},titleformat={}
  },
  section={
    number={\chinese{section}},
    name={第,节}
  },
  subsection={
    number={\chinese{subsection}、},
    aftername={\hspace{-0.01em}}
  },
  subsubsection={
    number={(\chinese{subsubsection})},
    aftername={\hspace {-0.01em}},
    beforeskip={1.3ex minus .8ex},
    afterskip={1ex minus .6ex},
    indent={\parindent}
  },
  paragraph={
    beforeskip=.1\baselineskip,
    indent={\parindent}
  }
}

\newcommand*\SZX[1]{%
  \ifcase\value{#1}%
    \or 上%
    \or 中%
    \or 下%
  \fi
}

%====================页眉设置======================
\usepackage{titleps}%或者\usepackage{titlesec},titlesec包含titleps
\newpagestyle{special}[\small\sffamily]{
  %\setheadrule{.1pt}
  \headrule
  \sethead[\usepage][][\chaptertitle]
  {\chaptertitle}{}{\usepage}
}

\newpagestyle{main}[\small\sffamily]{
  \headrule
  %\sethead[\usepage][][第\thechapter 章\quad\chaptertitle]
%  {\thesection\quad\sectiontitle}{}{\usepage}}
  \sethead[\usepage][][第\chinese{chapter}章\quad\chaptertitle]
  {第\chinese{section}节\quad\sectiontitle}{}{\usepage}
}

\newpagestyle{main2}[\small\sffamily]{
  \headrule
  \sethead[\usepage][][第\chinese{chapter}章\quad\chaptertitle]
  {第\chinese{section}節\quad\sectiontitle}{}{\usepage}
}

%================ PDF 书签设置=====================
\usepackage{bookmark}[
  depth=2,        % 书签深度 2:子节
  open,           % 默认展开书签
  openlevel=2,    % 展开书签深度 2:子节
  numbered,       % 显示编号
  atend,
]
  % 相比hyperref,bookmark宏包大多数时候只需要编译一次,
  % 而且书签的颜色和字体也可以定制。
  % 比hyperref 更专业 (自动加载hyperref)

%\bookmarksetup{italic,bold,color=blue} % 书签字体斜体/粗体/颜色设置

%------------重置每篇章计数器,必须在hyperref/bookmark之后------------
\makeatletter
  \@addtoreset{chapter}{part}
\makeatother

%------------hyperref 超链接设置------------------------
\hypersetup{%
  pdfencoding=auto,   % 解决新版ctex,引起hyperref UTF-16预警
  colorlinks=true,    % 注释掉此项则交叉引用为彩色边框true/false
  pdfborder=001,      % 注释掉此项则交叉引用为彩色边框
  citecolor=teal,
  linkcolor=myblue,
  urlcolor=black,
  %psdextra,          % 配合使用bookmark宏包,可以直接在pdf 书签中显示数学公式
}

%------------PDF 属性设置------------------------------
\hypersetup{%
  pdfkeywords={黄帝内经,内经,内经讲义,21世纪课程教材},    % 关键词
  %pdfsubject={latex},        % 主题
  pdfauthor={主编:王洪图},   % 作者
  pdftitle={内经讲义},        % 标题
  %pdfcreator={texlive2011}   % pdf创建器
}

%------------PDF 加密----------------------------------
%仅适用于xelatex引擎 基于xdvipdfmx
%\special{pdf:encrypt ownerpw (abc) userpw (xyz) length 128 perm 2052}

%仅适用于pdflatex引擎
%\usepackage[owner=Donald,user=Knuth,print=false]{pdfcrypt}

%其他可使用第三方工具 如:pdftk
%pdftk inputfile.pdf output outputfile.pdf encrypt_128bit owner_pw yourownerpw user_pw youruserpw

%=============自定义环境、列表及列表设置================
% 标题
\def\biaoti#1{\vspace{1.7ex plus 3ex minus .2ex}{\bfseries #1}}%\noindent\hei
% 小标题
\def\xiaobt#1{{\bfseries #1}}
% 小结
\def\xiaojie {\vspace{1.8ex plus .3ex minus .3ex}\centerline{\large\bfseries 小\ \ 结}\vspace{.1\baselineskip}}
% 作者
\def\zuozhe#1{\rightline{\bfseries #1}}

\newcounter{yuanwen}    % 新计数器 yuanwen
\newcounter{jiaozhu}    % 新计数器 jiaozhu

\newenvironment{yuanwen}[2][【原文】]{%
  %\biaoti{#1}\par
  \stepcounter{yuanwen}   % 计数器 yuanwen+1
  \bfseries #2}
  {}

\usepackage{enumitem}
\newenvironment{jiaozhu}[1][【校注】]{%
  %\biaoti{#1}\par
  \stepcounter{jiaozhu}   % 计数器 jiaozhu+1
  \begin{enumerate}[%
    label=\mylabel{\arabic*}{\circledctr*},before=\small,fullwidth,%
    itemindent=\parindent,listparindent=\parindent,%labelsep=-1pt,%labelwidth=0em,
    itemsep=0pt,topsep=0pt,partopsep=0pt,parsep=0pt
  ]}
  {\end{enumerate}}

%===================注解与原文相互跳转====================
%----------------第1部分 设置相互跳转锚点-----------------
\makeatletter
  \protected\def\mylabel#1#2{% 注解-->原文
    \hyperlink{back:\theyuanwen:#1}{\Hy@raisedlink{\hypertarget{\thejiaozhu:#1}{}}#2}}

  \protected\def\myref#1#2{% 原文-->注解
    \hyperlink{\theyuanwen:#1}{\Hy@raisedlink{\hypertarget{back:\theyuanwen:#1}{}}#2}}
  %此处\theyuanwen:#1实际指thejiaozhu:#1,只是\thejiaozhu计数器还没更新,故使用\theyuanwen计数器代替
\makeatother

\protected\def\myjzref#1{% 脚注中的引用(引用到原文)
  \hyperlink{\theyuanwen:#1}{\circlednum{#1}}}

\def\sb#1{\myref{#1}{\textsuperscript{\circlednum{#1}}}}    % 带圈数字上标

%----------------第2部分 调整锚点垂直距离-----------------
\def\HyperRaiseLinkDefault{.8\baselineskip} %调整锚点垂直距离
%\let\oldhypertarget\hypertarget
%\makeatletter
%  \def\hypertarget#1#2{\Hy@raisedlink{\oldhypertarget{#1}{#2}}}
%\makeatother

%====================带圈数字列表标头====================
\newfontfamily\circledfont[Path = fonts/]{meiryo.ttc}  % 日文字体,明瞭体
%\newfontfamily\circledfont{Meiryo}  % 日文字体,明瞭体

\protected\def\circlednum#1{{\makexeCJKinactive\circledfont\textcircled{#1}}}

\newcommand*\circledctr[1]{%
  \expandafter\circlednum\expandafter{\number\value{#1}}}
\AddEnumerateCounter*\circledctr\circlednum{1}

% 参考自:http://bbs.ctex.org/forum.php?mod=redirect&goto=findpost&ptid=78709&pid=460496&fromuid=40353

%======================插图/tikz图========================
\usepackage{graphicx,subcaption,wrapfig}    % 图,subcaption含子图功能代替subfig,图文混排
  \graphicspath{{img/}}                     % 设置图片文件路径

\def\pgfsysdriver{pgfsys-xetex.def}         % 设置tikz的驱动引擎
\usepackage{tikz}
  \usetikzlibrary{calc,decorations.text,arrows,positioning}

%---------设置tikz图片默认格式(字号、行间距、单元格高度)-------
\let\oldtikzpicture\tikzpicture
\renewcommand{\tikzpicture}{%
  \small
  \renewcommand{\baselinestretch}{0.2}
  \linespread{0.2}
  \oldtikzpicture
}

%=========================表格相关===============================
\usepackage{%
  multirow,                   % 单元格纵向合并
  array,makecell,longtable,   % 表格功能加强,tabu的依赖
  tabu-last-fix,              % "强大的表格工具" 本地修复版
  diagbox,                    % 表头斜线
  threeparttable,             % 表格内脚注(需打补丁支持tabu,longtabu)
}

%----------给threeparttable打补丁用于tabu,longtabu--------------
%解决方案来自:http://bbs.ctex.org/forum.php?mod=redirect&goto=findpost&ptid=80318&pid=467217&fromuid=40353
\usepackage{xpatch}

\makeatletter
  \chardef\TPT@@@asteriskcatcode=\catcode`*
  \catcode`*=11
  \xpatchcmd{\threeparttable}
    {\TPT@hookin{tabular}}
    {\TPT@hookin{tabular}\TPT@hookin{tabu}}
    {}{}
  \catcode`*=\TPT@@@asteriskcatcode
\makeatother

%------------设置表格默认格式(字号、行间距、单元格高度)------------
\let\oldtabular\tabular
\renewcommand{\tabular}{%
  \renewcommand\baselinestretch{0.9}\small    % 设置行间距和字号
  \renewcommand\arraystretch{1.5}             % 调整单元格高度
  %\renewcommand\multirowsetup{\centering}
  \oldtabular
}
%设置行间距,且必须放在字号设置前 否则无效
%或者使用\fontsize{<size>}{<baseline>}\selectfont 同时设置字号和行间距

\let\oldtabu\tabu
\renewcommand{\tabu}{%
  \renewcommand\baselinestretch{0.9}\small    % 设置行间距和字号
  \renewcommand\arraystretch{1.8}             % 调整单元格高度
  %\renewcommand\multirowsetup{\centering}
  \oldtabu
}

%------------模仿booktabs宏包的三线宽度设置---------------
\def\toprule   {\Xhline{.08em}}
\def\midrule   {\Xhline{.05em}}
\def\bottomrule{\Xhline{.08em}}
%-------------------------------------
%\setlength{\arrayrulewidth}{2pt} 设定表格中所有边框的线宽为同样的值
%\Xhline{} \Xcline{}分别设定表格中水平线的宽度 makecell包提供

%表格中垂直线的宽度可以通过在表格导言区(preamble),利用命令 !{\vrule width1.2pt} 替换 | 即可

%=================图表设置===============================
%---------------图表标号设置-----------------------------
\renewcommand\thefigure{\arabic{section}-\arabic{figure}}
\renewcommand\thetable {\arabic{section}-\arabic{table}}

\usepackage{caption}
  \captionsetup{font=small,}
  \captionsetup[table] {labelfont=bf,textfont=bf,belowskip=3pt,aboveskip=0pt} %仅表格 top
  \captionsetup[figure]{belowskip=0pt,aboveskip=3pt}  %仅图片 below

%\setlength{\abovecaptionskip}{3pt}
%\setlength{\belowcaptionskip}{3pt} %图、表题目上下的间距
\setlength{\intextsep}   {5pt}  %浮动体和正文间的距离
\setlength{\textfloatsep}{5pt}

%====================全文水印==========================
%解决方案来自:
%http://bbs.ctex.org/forum.php?mod=redirect&goto=findpost&ptid=79190&pid=462496&fromuid=40353
%https://zhuanlan.zhihu.com/p/19734756?columnSlug=LaTeX
\usepackage{eso-pic}

%eso-pic中\AtPageCenter有点水平偏右
\renewcommand\AtPageCenter[1]{\parbox[b][\paperheight]{\paperwidth}{\vfill\centering#1\vfill}}

\newcommand{\watermark}[3]{%
  \AddToShipoutPictureBG{%
    \AtPageCenter{%
      \tikz\node[%
        overlay,
        text=red!50,
        %font=\sffamily\bfseries,
        rotate=#1,
        scale=#2
      ]{#3};
    }
  }
}

\newcommand{\watermarkoff}{\ClearShipoutPictureBG}

\watermark{45}{15}{草\ 稿}    %启用全文水印

%=============花括号分支结构图=========================
\usepackage{schemata}

\xpatchcmd{\schema}
  {1.44265ex}{-1ex}
  {}{}

\newcommand\SC[2] {\schema{\schemabox{#1}}{\schemabox{#2}}}
\newcommand\SCh[4]{\Schema{#1}{#2}{\schemabox{#3}}{\schemabox{#4}}}

%=======================================================

\begin{document}
\pagestyle{main2}
\fi
\chapter{病因病机}%第四章

病因,即生病的原因,也就是破坏人体相对动态平衡而导致发生疾病的原因。《内经》所论述的病因有外感时邪、情志过激、饮食失调、劳逸失度、起居无节、跌仆损伤以及病气遗传等方面。病机,就是疾病发生发展变化的规律。包括阴阳盛衰、正邪虚实、升降出入失调等。有关病因、病机的记载,主要见于《素问·生气通天论》、《素问·玉机真脏论》、《素问·至真要大论》、《素问·举痛论》以及《灵枢·百病始生》等篇。

\section{素問·生氣通天論}%第一节

\biaoti{【原文】}

\begin{yuanwen}
黃帝曰:夫自古通天者,生之本,本於陰陽\sb{1}。天地之間,六合\sb{2}之内,其氣九州\sb{3}、九竅、五藏、十二節\sb{4},皆通乎天氣,其生五,其氣三\sb{5},数犯此者,則邪氣傷人,此壽命之本也。

蒼天之氣,清淨則志意治\sb{6},順之則陽氣固,雖有賊邪\sb{7},弗能害也。此因時之序\sb{8}。故聖人傳精神\sb{9},服天氣\sb{10},而通神明\sb{11}。失之,則內閉九竅,外壅肌肉,衛氣散解\sb{12},此谓自傷,氣之削\sb{13}也。
\end{yuanwen}

\biaoti{【校注】}

\begin{jiaozhu}
  \item 生之本,本于阴阳:生命的根本在于枬阳双方的协调统一。
  \item 六合:指东、南、西、北、上、下六方,即整个宇宙。
  \item 九州:王冰注:“九州,谓冀、兖、青、徐、扬、荆、豫、梁、雍也。”地有九州,人有九窍。然俞樾《内经辩言》注:“九州即九窍,……古谓窍为州。”如此,“九州”与下文“九窍”义重,疑衍。
  \item 十二节:即双侧腕、肘、肩、踝、膝、髋等十二个大关节。
  \item 其生五,其气三:其,指自然界的阴阳。五,即木火土金水五行。三,即三阴三阳。全句意为自然界的阴阳化生木火土金水五行,分为三阴三阳。
  \item 清净则志意治:净,通静。志意,指人的精神活动。治,正常。即自然界阴阳之气清静而无异常变化,则有利于人的精神保持正常。
  \item 贼邪:贼,伤害也。贼邪,即伤害人的邪气。
  \item 此因时之序:因,顺也。意即顺应四时气候变化的规律而养生。
  \item 传精神:俞樾《内经辩言》注:“传,读为抟,聚也。”传精神,即聚精神,全神贯注之义。
  \item 服天气:服,顺也。意即顺应自然界阴阳之气的变化。
  \item 通神明:通,此处作统一解。神明,即阴阳变化。通神明,言人体阴阳之气与自然界阴阳之气变化统一起来。
  \item 卫气散解:指卫气离散耗解而不固。
  \item 气之削:即阳气被削弱。
\end{jiaozhu}

\biaoti{【理论阐释】}

\xiaobt{关于“生气通天”}

本节重点阐述人体生命的根本在于阴阳二气的协调,而且人体阴阳之气与自然界阴阳是相互通应的,亦要达到协调统一。所以养生应顺应天地阴阳的变化,如《素问·四气调神大论》中“春夏养阳,秋冬养阴”之法。传精神,服天气,而通神明”,乃本篇眼目,昭示养生要旨,内则精神专一,外则顺应自然,保持人与自然的和谐,如此“则阳气固,虽有贼邪弗能害也。”若违背了这一规律,内有脏腑阴阳气血失调,九窍功能障碍;外有肌肉壅阻而不滑利,卫气不固而腠理疏松,则邪气为害,正气削弱,疾病丛生而短折寿命。

“天人相应”为《内经》基本学术思想之一,而“生气通天”是《内经》中“天人相应”观的组成部分。说明人体内的各种生理机能无不与自然界息息相通。如《灵枢·岁露论》曰:“人与天地相参也,与日月相应也;”《素问·金匮真言论》指出:“五脏应四时,各有收受。”《灵枢·五癃津液别》论津液气化受四时影响,“天暑衣厚则腠理开,故汗出”;“天寒则腠理闭,气湿不行,水下留于膀胱,则为溺与气。”《素问·八正神明论》认为血气的运行、盛衰变化与月亮盈亏关系甚密,谓“月始生,则血气始精,卫气始行;月郭满,则血气实,肌肉坚;月郭空,则肌肉减,经络虚,卫气去。”人生存于自然之中,与自然环境有如此息息相关之联系,故在此特别强调了养生必须以“生气通天”为要领。

\biaoti{【原文】}

\begin{yuanwen}
陽氣者,若天與日,失其所\sb{1}則折壽而不彰\sb{2}。故天運\sb{3}當以日光明。是故陽因而上\sb{4},衛外者也。
\end{yuanwen}

\biaoti{【校注】}

\begin{jiaozhu}
  \item 失其所:所,场所。谓阳气运行、作用失常,失去其应居之所。
  \item 折寿而不彰:折寿,即短寿;不彰,不显著。指人身若阳气功能失常,可导致短折寿命的结果。高世栻注:“通体之气,经脉之气,各有其所。若失其所,则运行者不周于通体,旋转者不循行于经脉,故短折其寿,而不彰著于人世矣。”
  \item 天运:即天体的运行。
  \item 阳因而上:因,顺应,依顺。言阳气顺应其上升外越之性,而具有卫外的作用。
\end{jiaozhu}

\biaoti{【理论阐释】}

\xiaobt{阳气的重要作用}

本节把人之阳气比做天体中的太阳,据此可见,人之阳气具有天之阳气诸多特点和功用。天之阳气能给自然界带来光明、温暖。主司天体的运行,可蒸腾气化水液,使万物生长化收藏;没有太阳,也就没有自然万物蓬勃之象。比类于人体,则人之阳气具有护卫生命,促进机体生命活动运转不息的作用,若阳气“失其所”,则人之寿命不保而早夭。人体之温暖,五脏功能之运转,津液之气化,对外界虚邪贼风之抵御,均赖阳气的温煦和推动。可见,在人体的阴阳平衡中,阳气起着主导的作用。这种重视阳气的观点启发后世医家,并成为温补学派的重要理论依据。明代医家张介宾在《类经附翼·求正录》中说:“阳化气,阴成形。形本属阴,而凡通体之温者,阳气也;一生之活者,阳气也;五官五脏之神明不测者,阳气也。及其既死,则身冷如冰,灵觉尽灭,形固存而气则去,此以阳脱在前,而阴留在后。”强调“天之运,人之命,天之根本,总在太阳无两也。”并进而提出了“天之大宝,只此一丸红日;人之大宝,只此一息真阳”(《类经附翼·大宝论》)的著名论点。

\biaoti{【原文】}

\begin{yuanwen}
因於寒,欲如運樞\sb{1},起居如驚,神氣乃浮\sb{2};因於暑,汗,煩則喘喝,靜則多言\sb{3},體若燔炭,汗出而散;因於濕,首如裹,濕熱不攘\sb{4},大筋緛短,小筋㢮長\sb{5},緛短爲拘,㢮長爲痿;因於氣\sb{6},爲腫。四維相代\sb{7},陽氣乃竭。

陽氣者,煩勞則張\sb{8},精絕,辟積\sb{9}於夏,使人煎厥\sb{10}。目盲不可以視,耳閉不可以聽,潰潰乎若壞都,汩汩乎不可止\sb{11}。陽氣者,大怒則形氣絕\sb{12},而血菀\sb{13}上,使人薄厥\sb{14}。有傷於筋,縱,其若不容\sb{15}。汗出偏沮\sb{16},使人偏枯。汗出見濕,乃生痤疿\sb{17}。高梁之變,足生大丁\sb{18},受如持虚\sb{19}。勞汗當風,寒薄爲皶\sb{20},鬱乃痤。

陽氣者,精則養神,柔則養筋\sb{21}。開合不得,寒氣從之,乃生大僂\sb{22};陷脈爲瘻\sb{23},留連肉腠,俞氣化薄\sb{24},傳爲善畏,及爲驚駭\sb{25};營氣不從,逆於肉理,乃生癰腫;魄汗\sb{26}未盡,形弱而氣爍\sb{27},穴俞以閉,發爲風瘧\sb{28}。

故風者,百病之始也,清靜則肉腠閉拒,雖有大風苛毒\sb{29},弗之能害\sb{30}。此因時之序也。故病久則傳化,上下不並\sb{31},良醫弗爲。故陽畜稹病死,而陽氣當隔,隔者當寫\sb{32}。不亟正治,粗乃敗之\sb{33}。

故陽氣者,一日而主外,平旦人氣生,日中而陽氣隆,日西而陽氣已虛,氣門\sb{34}乃閉。是故暮而收拒,無擾筋骨,無見霧露,反此三時\sb{35},形乃困薄\sb{36}。
\end{yuanwen}

\biaoti{【校注】}

\begin{jiaozhu}
  \item 欲如运枢:运,运转。枢,户枢,即门轴。欲如运枢,是指卫阳之气如户枢般开合运转自如。
  \item 起居如惊,神气乃浮:惊,卒暴之意。神气,即指阳气。浮,浮越于表。意为生活起居正常规律被扰,邪气侵犯,卫阳之气则上浮与邪气抗争。
  \item 烦则喘喝,静则多言:烦,烦躁不安。喘喝,气喘息急,喝喝有声。烦则喘喝,为阳热内盛所致;静,相对烦而言,指神昏嗜卧。多言,如神昏谵语、郑声之类。静则多言,为暑邪伤神所致。
  \item 攘:除也。
  \item 大筋緛短,小筋拖长:緛,收缩。㢮,同弛,松弛。本句作互文解,即太筋、小筋或为收缩变短,或为弛缓变长。
  \item 气:即风。高世栻注:“气,犹风也,《阴阳应象大论》云:‘阳之气,以天地之疾风名之。’故不言风而言气。”
  \item 四维相代:四维,四方四时,此处指上文所言的风寒暑湿等四时邪气。代,更代。意为寒、暑、湿、风(气)四种邪气更替伤人。
  \item 张:鸱张亢盛。
  \item 辟积:辟,通襞,即衣裙褶。辟积,重复之义。
  \item 煎厥;病名。指过度烦劳,阳气鸱张亢盛,火炎则水干,阴精虚衰,又逢盛夏阳热之气,则两热相合,如煎如熬,以致阴气竭绝而昏厥的病证。
  \item 溃溃乎若坏都,汩汩乎不可止:溃溃,是形容洪水泛滥的样子。都,防水之堤。汩汩,水急流的声音。本句以洪水决堤来形容煎厥发病来势凶猛发展迅速的病证特点。
  \item 形气绝:形,即形体,此处主要指脏腑经络。绝,阻滞隔绝。马莳注谓:此“绝”是“阻绝之义,非断绝之谓。”形气绝,即脏腑经络之气阻绝不通。
  \item 菀:同郁。
  \item 薄厥:病名。指大怒而气血上逆所致的昏厥病证。张介宾注:“相迫曰薄,气逆曰厥,气血俱乱,故为薄厥。”
  \item 不容:容,通“用”。不容,即不用,指肢体不能随意运动。
  \item 汗出偏沮:沮,阻也,阻止。汗出偏沮,指汗出受阻而半侧身体无汗的症状。
  \item 痤疿:痤,即小疖。疿,即汗疹,痱子。
  \item 高梁之变,足生大丁:高,通膏,即脂膏类食物。梁,通粱,即精细的食物。膏粱,在此指肥甘厚味。足,俞樾《内经辨言》云:疑‘是’字之误。上云乃生痤疿,此云是生大丁,语意一律,‘是’误为足”。胡澍《素问校义》亦云:“足,当作是字之误也。是,犹则也。”丁,通疔,此泛指疮疡。吴昆注:“膏粱之人,内多滞热,故其病变,能生大疔。”
  \item 受如持虚:形容得病容易,犹如持空虚之器受物一样。
  \item 皶:即生长于面部的粉刺;若生于鼻部则为酒齄鼻。
  \item 精则养神,柔则养筋:当作“养神则精,养筋则柔”解。精,指精神爽慧。柔,即筋脉柔和,活动自如。此句提示阳气具有温养精神、筋脉作用。
  \item 大偻:偻,曲背之义。大偻,指阳气不能温养筋脉导致形态伛偻,不能直立的病证。
  \item 陷脉为瘘:陷脉,指邪气内陷经脉。瘘,瘘管,即病理性管道。
  \item 俞气化薄:俞,同腧,即腧穴,为经脉气血输注之处。化,传化、传入的意思。薄,迫也。俞气化薄,指邪气从腧穴传入而内迫五脏。
  \item 传为善畏,及为惊骇:五脏主藏神,脏气被邪所迫,阳气不能养神,故见心神不安之善畏、惊骇。
  \item 魄汗:魄,通白,魄汗即白汗。白汗,指汗出不因暑热所致,即自汗也。
  \item 形弱而气烁:形弱,在此指腠理不固,自汗出而易感受外邪,形体虚弱。烁,消烁。气烁,此指阳气被邪热所消耗。
  \item 风疟:疟疾之一,因感受风邪,寒热往来,恶风汗出而名之。
  \item 大风苛毒:苛,暴也。大风苛毒,指致病性强的病邪。
  \item 弗之能害:即“弗能害之”,弗,通勿。
  \item 上下不并:并,交并,交通之意。上下不并,指人体阳气上部与下部不相交通,相互阻隔的病理变化。
  \item 阳气当隔,隔者当写:前一“当”字,通挡。阳气挡隔,对此应当采用泻法,祛火以使蓄积的阳气得以畅通。
  \item 不亟正治,粗乃败之:亟,急也。粗,粗工,即医疗水平低劣的医生。全句意为不能迅速给予正确治疗,这样水平低劣的医生只能使病情败坏、恶化。
  \item 气门:此处指汗孔。王冰注:“所以发泄经脉营卫之气,故谓之气门也。”
  \item 三时:即上文的平旦、日中、日西。
  \item 形乃困薄,指形体困顿而衰薄。马莳注:“未免困窘而衰薄矣。”
\end{jiaozhu}

\biaoti{【理论阐释】}

1.阳气的生理功能

(1)“阳气者,精则养神,柔则养筋”

提示阳气的温养作用,阳气旺盛可使人精神聪慧、饱满,筋脉柔软屈伸自如。若阳气发生病变,则温养作用下降,精神、筋脉失养出现病变。如煎厥之病,因烦劳阳气亢盛精绝而至精神昏聩,再如薄厥,因大怒阳气上逆迫血上行,至血菀伤神伤筋,神伤则厥,筋伤则纵而不用;大偻之病,乃阳气不得养筋所致;魄汗则为阳气不能温固肌表所起。

(2)阳气随昼夜阴阳消长而变化

阳气在一昼夜中有生发,隆盛、虚衰的变化规律,人身阳气与自然界阴阳变化息息相关。提示人要随自然界的阴阳变化来调节生活起居,以保持阳气的充沛,防止疾病的发生。病理上,《灵枢·顺气一日分为四时》有“旦慧、昼安、夕加、夜甚”的疾病变化规律,其内在机制也是阳气昼夜消长的节律性变化。这种认识与现代人体生物钟理论相似,值得进一步探讨和研究。

2.阳气的病理

(1)阳气卫外失常,外邪侵袭

寒邪易伤阳气,故首论“因于寒”;暑为阳邪,两伤气津,实证是邪热内盛而多汗烦喘,虚证则神失所养而神昏谵语;湿邪重浊,困遏阳气,清阳不升,故头重,湿邪郁而化热,流连筋脉,致阳气不能温养,故或为挛急,或为筋痿;风性轻扬,易致头面浮肿。如风寒暑湿,交替为病,阳气反复受损,可使阳气衰竭。

(2)阳亢精绝

平素烦劳过度,阳气过亢,虚火上炎,阴精亏损,复加暑热煎灼,致阴精衰惫,发生突然昏厥,古人名为“煎厥”。其临床表现除昏厥外,还有耳闭、目盲。此病虽来势骤急,但以阴精亏损为本,故为至虚危候。

(3)阳气厥逆

由大怒而致气上逆,血随气升,气血逆乱,出现突然昏厥,古人名为“薄厥”。其临床表现除昏厥外,可见筋脉弛纵不收,类似于现代脑中风。其发病暴卒,故其初病为气血壅阻之实证。

(4)阳气偏阻

阳气不足,不能温运全身,偏阻一侧,表现为半侧身体汗出,半侧无汗,有可能出现局部肢体枯萎不用的病证。

(5)阳气郁遇

汗出而阳气宣泄之时,猝被湿邪所郁遏,宣泄不畅,易生疖子、痱子。或形劳汗出,坐卧当风,风寒迫聚于皮腠,形成粉刺,郁而化热而成疮疖。

(6)阳热内盛

嗜食膏粱厚味,阳热蓄积,热毒逆于肉里,易生疔疮,而腐肉酿脓。

(7)阳气开合不得

阳气不足,邪气入里留恋不去,会导致各种病证。如邪入筋,阳虚寒邪痹阻于背,筋失温养,不能运动自如,出现背曲不能直立之症。邪入脉中,阳虚邪陷经脉,经脉败漏,日久成痿管,久不收口。邪入脏腑,阳虚邪气留恋肉腠,由俞穴侵入,内传迫及五脏,影响其藏神功能,而出现善畏、惊骇等症。邪入肉里,营卫失调,营气不从,阻逆于肌肉之间,发生痈肿。邪入俞穴,阳气被热邪所耗伤,汗出不止,风邪入侵,俞穴闭阻,发生风疟。

(8)阳气阻隔,上下不相交通

阳气蓄积于一处则病情危重,当急用泻法以祛除实火,疏通阳气,当能挽救。

3.对“因于寒,欲如运枢”的理解

“因于寒,欲如运枢”历代有不同见解。王冰注云:“言因天之寒,当深居周密,如枢纽之内动;不当烦扰筋骨,使阳气发泄于皮肤,而伤于寒毒也。”认为是指冬季不宜剧烈运动,只能像转动门轴那样活动一下关节即可。此为一。《新校正》云:“按全元起本作‘连枢’。元起云:阳气定如连枢者,动系也。”据此,“欲如运枢”当作“定如连枢”,意谓阳气被寒邪束缚,不得畅运,犹如门枢被固定,开合不得。此为二;明·吴昆则将“欲如运枢,起居如惊,神气乃浮”三句移至上文“阳因而上,卫外者也”之后,而把“体若燔炭,汗出而散”移至“因于寒”下,此为三。第三种解释即补充说明了阳气的卫外作用,又将受于寒邪后机体的抵抗能力和症状表现凸现出来,比较符合临床实际。

\biaoti{【临证指要】}

1.薄厥和煎厥

薄厥来势凶猛,可见突然昏愦,目闭耳聋,筋脉弛纵,半身不遂,口舌歪斜,二便失禁。类似于临床中风病,这是中医对脑中风证的最早记载。张介宾曰:“厥逆之证,危证也。盖厥者,尽也;逆者,乱也。即气血败乱之谓也。故《内经》特重而详言之”(《景岳全书·杂证谟》)。当此之时,可用镇肝熄风汤合安宫牛黄丸治之,待病情控制后再以补阳还五汤治其“有伤于筋,纵,其若不容”的瘫痪症状。张介宾又认为薄厥亦有轻证,如“气实而厥者,其形气愤然勃然,脉沉弦而滑,胸膈喘满,此气逆证也。经曰:大怒则形气绝,而血苑于上,即此类也。治宜以排气饮或四磨饮,或八味顺气丸,苏合香丸之类,先顺其气,然后随其虚实而调理之”(《景岳全书·杂证谟》)。这类患者未必有筋脉弛纵证候。其证类似于今之癔病。

煎厥一证强调了阴虚阳亢的内在因素,又突出了暑热相合的外因,这很符合临床暑厥的病证。暑厥的患者多为体质虚弱,能冬不能夏的老人、幼儿、久病或产后妇女,临床常见暑厥病势凶猛,而无手足弛缓的瘫痪症状,但亦有老年患者,病情复杂,暑厥时又并发脑血管意外,则可有手足瘫痪的表现,临床须仔细观察,及时抢救。

2.“因于湿,首如裹,湿热不攘,大筋緛短,小筋弛长”

临床一般遇筋脉病变常以肝风内动、热极生风或外感风邪猝伤筋脉辨证,而本句原文则提示湿邪亦可致筋脉病变,由于湿为阴邪,若留连日久,阳气被阻或损伤,则筋脉失养,而有拘挛弛缓之症。此种观点在《内经》中还见于《至真要大论》“诸痉项强,皆属于湿。”故治疗此类筋脉病变可用祛湿之法,如《丹溪心法》中加味二妙散治疗筋脉弛缓;王海藏用神术汤加羌活、独活、麻黄治筋脉拘挛而发热无汗反恶寒的刚痉;用白术汤加桂心、黄芪治疗筋脉拘急而发热汗出不恶寒的柔痉。

3.“因于暑,汗,烦则喘喝;静则多言”

感受署邪之后,汗出是其主症之一。“烦则喘喝;静则多言”,张介宾分析:“暑有阴阳之证,阳证因于中热,阴证因于中寒……此节所言,言暑之阳者也。故为汗出烦躁,为喘,为大声呼喝。若其静者,亦不免于多言。盖邪热伤阴,精神内乱,故言无伦次也。”暑热熏蒸,汗出烦躁喘喝,此为实证;若暑热过伤心神,神伤嗜卧,表情淡漠,郑声多言,多为气阴两虚之证。实证宜祛暑清热;虚证应清暑益气。

\biaoti{【原文】}

\begin{yuanwen}
岐伯曰:陰者藏精而起亟\sb{1}也;陽者衛外而爲固也。陰不勝其陽,則脈流薄疾\sb{2},並乃狂\sb{3};陽不勝其陰,則五藏氣爭\sb{4},九竅不通。是以聖人陳陰陽,筋脈和同\sb{5},骨髓堅固,氣血皆從。如是則內外調和,邪不能害,耳目聰明,氣立如故\sb{6}。

風客淫氣,精乃亡\sb{7},邪傷肝也。因而飽食,筋脈横解\sb{8},腸澼爲痔\sb{9};因而大飲,則氣逆;因而強力\sb{10},腎氣乃傷,高骨\sb{11}乃壞。

凡陰陽之要,陽密乃固\sb{12}。兩者不和,若春無秋,若冬無夏,因而和之,是謂聖度。故陽強不能密,陰氣乃絕\sb{13};陰平陽秘,精神乃治\sb{14};陰陽離決,精氣乃絕\sb{15}。

因於露風\sb{16},乃生寒熱。是以春傷於風,邪氣留連,乃爲洞泄\sb{17};夏傷於暑,秋爲痎瘧、秋傷於濕,上逆而咳\sb{19},發爲痿厥\sb{20};冬傷於寒,春必溫病\sb{21}。四時之氣,更傷五藏\sb{22}。
\end{yuanwen}

\biaoti{【校注】}

\begin{jiaozhu}
  \item 起亟:亟,频数.起亟,指阴精不断地起而与阳气相应,说明阴为阳之基。
  \item 脉流薄疾:薄,迫也。脉流薄疾,指阳气亢盛,使脉中气血流动急迫而快疾。
  \item 并乃狂:并,交并,引申为重复、加甚之意。并乃狂,指阳气亢盛而致神志狂乱。
  \item 五脏气争:争,不和之意。五脏气争,指五脏功能失调,气机失和。
  \item 筋脉和同:和同,即和谐,协调。筋脉和同,指筋脉功能协调。
  \item 气立如故:立,反训为行。气立如故,指脏腑经络之气运行正常。一说气立,谓人需依赖自然四时之气才能有此生命。吴昆云:“气立者,人受天地之气以立命,故有生谓之气立。”
  \item 风客淫气,精乃亡:凤邪侵袭人体,而为淫乱之气。风为阳邪,易使阴精耗散。高世栻注:“风为阳邪,风客淫气,则阴精消烁,故精乃亡。”
  \item 筋脉横解:横,放纵的意思。解,通懈,即松弛。筋脉横解,即筋脉弛纵不收。
  \item 肠澼为痔:肠澼,即下利脓血的痢疾等病。为,与也。痔,痔疮。饮食过饱,肠胃乃伤,湿热下注则为痢疾;迫于魄门,日久成痔。
  \item 强力:勉强用力,劳力过度,又指房室太过,王冰注:“强力,谓强力入房也。”
  \item 高骨:即腰间脊骨。
  \item 阴阳之要,阳密乃固:要,关键,要领。阴精与阳气关系的关键,在于阳气致密于外,阴气才能固守于内。
  \item 阳强不能密,阴气乃绝;阳强,阳气亢盛。阳气若致密于外,则阴气能固守于内。今阳气亢盛,不能为阴气致密于外,则阴气亦不能内守而外泄,以至衰竭亡绝。张介宾注:“强,亢也。孤阳独月,不能固密,则阳气耗而竭绝矣。《痹论》曰:阴气者,静则神藏,躁则消亡。躁即阳强不密之谓。”
  \item 阴平阳秘,精神乃治:阴平阳秘为互文,即阴阳平秘。平秘,平和协调之意。治,正常。
  \item 阴阳离决,精气乃绝:离,分离。决,决裂。阴阳分离决裂,则孤阳不生独阴不长,精气无以滋生而竭绝。
  \item 露风:泛指一般外感病的致病因素,如下文所言风、暑、湿、寒诸邪。又,露,触冒。露风,即触冒风邪之意。
  \item 洞泄:指水谷不化,下利无度的重度泄泻。
  \item 痎疟:疟疾的总称。
  \item 秋伤于湿,上逆而咳:张介宾注:“湿土用事于长夏之末,故秋伤于湿也。秋气通于肺,湿郁成热,则上乘肺金,故气逆而为咳嗽。”
  \item 痿厥:偏义复词,偏在“痿”,即肢体枯萎不用的病证。
  \item 冬伤于寒,春必温病:因冬季养生不当,感受寒邪,阴精亏虚,至春天阳气升发,或又感新邪,发为温病。
  \item 四时之气,更伤五脏:更,更替。指四时不正之气,交替地损伤五脏。即与前文“四维相代,阳气乃竭”同义。
\end{jiaozhu}

\biaoti{【理论阐释】}

1.阴精与阳气的关系

(1)相互为用,相互依存

“阴者藏精而起亟也,阳者卫外而为故也。”阴是内藏的精气,不断地起而供给阳气之用;阳气能保卫体表,抵御外邪,使机体固密,保护阴精的正常化生。《素问·阴阳应象大论》所谓“阴在内,阳之守也,阳在外,阴之使也。”如果阴精和阳气这种互根互用的关系遭到破坏,则“孤阳不生,独阴不长”,临床可见“阳损及阴”,“阴损及阳”的病理变化。

(2)互相制约

“阴不胜其阳,则脉流薄疾,并乃狂;阳不胜其阴,则五脏气争,九窍不通。”提示阴阳之间存在着相互制约关系,阴不胜阳则阳偏盛,阳不胜阴则阴偏盛。

(3)阴平阳秘,精神乃治

《老子》说:“万物负阴而抱阳,冲气以为和。”阴阳之间的和谐协调,是万物自身运动所形成的最佳状态。它体现着阴阳双方在相互消长的状态中彼此相互作用,保持着稳定性。对人体来说,阴平阳秘是健康。所以“圣人”养生“陈阴阳,筋脉和同,骨髓坚固,气血皆从。如是则内外调和,邪不能害,耳目聪明,气立如放。”

2.阴阳失和的病理

阴不胜其阳,则阳用事,可出现“脉流薄疾,并乃狂”等病证。大凡狂证阳多阴少,阳与阳相并,阳气蓄积,内扰神明,而出现精神狂妄,奔走呼叫,登高而歌,弃衣而走的证候。其脉多见弦滑而急数之象,若狂妄之势退去,则薄急之脉亦会有所减缓。如若阳不胜其阴,则阴用事,而见“五脏气争,九窍不通。”若“阳强不能密,阴气乃绝”,如煎厥一证,即阳气鸱张,致使阴精亏虚,待暑热之气相逼时,则阳气亢极,而导致“阴阳离决,精气乃绝”的局面。

\biaoti{【临证指要】}

\xiaobt{阴阳互根理论的临床应用}

张介宾根据《内经》阴阳互根互用的理论,对于阴精阳气偏衰的病证进行调治。他认为这一理论可以进一步扩展、深化,广泛应用于阴阳、精气、气血二者同虚证之中,依据“阴阳互根”、“精气互化”、“气血互生”之理,创制了左归丸、右归丸等著名方别,其在《景岳全书·补略》中提出:“善补阳者,必于阴中求阳,则阳得阴助而生化无穷;善补阴者,必于阳中求阴,则阴得阳升而源泉不竭。”“气因精而虚者,自当补精以化气;精因气而虚者,自当补气以生精。”其在左归丸中用鹿角胶,正体现了“阳中求阴”、“补气以生精”之义;而右归丸中用熟地、山萸,正是“阴中求阳”、“补精以化气”也。

\biaoti{【原文】}

\begin{yuanwen}
陰之所生,本在五味\sb{1};陰之五宮\sb{2},傷在五味。是故味過於酸,肝氣以津,脾氣乃絕\sb{3};味過於鹹,大骨氣勞,短肌,心氣抑\sb{4};味過於甘,心氣喘滿,色黑,腎氣不衡\sb{5};味過於苦,脾氣不濡,胃氣乃厚\sb{6};味過於辛,筋脈沮弛,精神乃央\sb{7}。是故謹和五味\sb{8},骨正筋柔,氣血以流,腠理以密,如是則骨氣以精\sb{9},謹道如法,長有天命\sb{10}。
\end{yuanwen}

\biaoti{【校注】}

\begin{jiaozhu}
  \item 阴之所生,本在五味:阴,即阴精。五味,即酸苦甘辛咸,此处泛指饮食物。言阴精的产生,本源于饮食五味。
  \item 阴之五宫:五宫,即五脏。阴之五宫,即藏蓄阴精的五脏。
  \item 味过于酸,肝气以津,脾气乃绝:津,满溢、过盛之意。酸味本有滋养肝脏的作用,但酸味太过,会导致肝气过亢,肝木乘脾土,而使脾气衰竭。
  \item 味过于咸,大骨气劳,短肌,心气抑:张志聪注:“大骨,腰高之骨,肾之府也。过食咸则伤肾,故骨气劳伤;水邪盛则侮土,故肌肉短缩;水上凌心,故心气抑郁也。”
  \item 味过于甘,心气喘满,色黑,肾气不衡:甘,《太素》作“苦”可从。喘,此指心跳急促。满,通懑,烦闷也。衡,平也。苦入心,味过于苦则反伤心气,故心跳急促而烦闷;黑为水色,火不足则水气乘之,故面见黑色,心火虚衰则肾水偏盛,故言肾气不衡。
  \item 味过于苦,脾气不濡,胃气乃厚:苦,《太素》作“甘”且无“不”字,可从。濡,湿也。厚,此指胀满。甘入脾,味过于甘则伤脾生湿,湿阻脾胃则生胀满,《奇病论》曰:“甘者令人中满。”
  \item 味过于辛,筋脉沮弛,精神乃央:沮,败坏。央,通殃。辛入肺,味过于辛则伤肺,肺伤则津液不布,筋失所养而败坏弛缓;辛性走散,神气耗伤,故殃及精神。
  \item 谨和五味:谨慎地调和饮食五味。
  \item 骨气以精:骨气,泛指上文的骨、筋、气、血、腠理。精,强盛。骨气以精,是言骨、筋、气、血、腠理均得到五味的滋养而强盛不衰。
  \item 天命:自然赋予人的寿限。
\end{jiaozhu}

\biaoti{【理论阐释】}

1.阴之所生,本在五味;阴之五宫,伤在五味

饮食五味是人赖以生存的基本条件,是五脏精气之本源。但是,“水能载舟,亦能覆舟”,若饮食太过,也可成为损伤五脏精气的重要原因。饮食所伤,除能直接伤害肠胃以影响五脏外,还可通过五味与五脏的相合关系,引起相关脏腑发生病理变化,又进一步影响到其他脏腑。

酸味先走肝,可养肝资筋,但酸味太过,则肝气亢盛,易乘脾土,致脾气衰竭。咸味先走肾,可养肾资骨,若咸味太过,损伤肾气,大骨气劳,气化失司,水邪偏盛,侮土则短肌,凌心则心气抑。苦味先走心,可养心资血,若苦味太过,损伤心气,则心悸烦闷。若心肾相交,则水火既济,今心火不足,则肾气不衡,而水气上乘,故色黑。甘味先走脾,可养脾资肉,若甘味太过,损伤脾气,脾失健运,则湿阻中焦而脘腹胀满。辛味先走肺,可养肺资气,若辛味太过,肺气受损,津液不布,肝筋失养,故筋脉沮弛,肝主魂,肺主魄,魂隗失藏,故精神乃殃。

在《内经》中不仅有五味入五脏,五味各走其所喜,五味伤五脏的理论,还有五味所禁的内容,如《素问·宣明五气》提出:“五味所禁,辛走气,气病无多食辛;咸走血,血病无多食咸;苦走骨,骨病无多食苦;甘走肉,肉病无多食甘;酸走筋,筋病无多食酸,是为五禁,无令多食。”学习时,可结合而参考之。

2.五味理论的现代研究

《内经》五味入五脏的理论是中医理论的特色之一,不仅自古以来指导着医疗实践,而且亦得到当今医学界的重视,对此进行了一些研究。根据中国预防医学科学院营养与食品卫生研究所编著的食物成分表中的食物营养成分来看,肉类中性温的羊肉,性平的猪肉、鸡肉,性凉的兔肉,和性微寒的鸭肉在某些营养素的含量上的确有所不同,如寒凉性的食物脂肪含量低,维生素E和硒含量较高;在脂肪酸含量方面,性平凉的食物其饱和脂肪酸较性温的食物为多。但根据同类食物各种营养成分比较,未见它们之间有明显的差异。因此,食物的食性并不是单纯的营养成分和含量多少的区别,而是食物中其它功能性的成分在影响它的食性。所以,在这方面还值得进一步研究。如《内经》曰:“多食咸,则血脉凝泣而变色。”有人进行了动物实验研究,结果表明高盐饮食对动物全血粘度、纤维蛋白原含量及平均血压、高血压发生率均显著髙于正常饮食组。

根据现代营养学的研究,随着经济的发展,生活的改善,人们饮食结构的变化,由于五味偏嗜,营养失调造成的肥胖、高血压、心血管疾病等“现代文明病”的发病率有逐年升高的趋势,而高糖、高盐、高脂肪食物是主要因素。因此五味偏嗜、五味入五脏的理论可以与现代营养学密切结合起来,共同探索建立中医营养治疗学。

\biaoti{【临证指要】}

\xiaobt{五味理论的临床意义}

中药有五味之分,饮食亦有五味之分,饮食中的五昧既具有营养作用同时亦具有治疗作用。所以在日常生活中应注意调和饮食,避免太过,过之则会带来危害。如:

(1)辛味食物

如辣椒、胡椒、葱、蒜等,具有行气、行血、发散作用。除调味之外,也可用以治疗气血阻滞、外邪束表之证。辛味能刺激胃肠蠕动,增加消化液分泌,促进血液循环和机体代谢。现代药理研究认为此类食物多含有辣椒碱,可引起粘膜皮肤的烧灼感,从而反射性地提高体温与血压。但过食辛辣则对眼病、口腔炎、痔疮患者不利。

(2)甘味食物

如蜂蜜、饴糖、甘草等,具有和中缓急、补益作用,通常可用以治疗虚证、拘急痉挛、疼痛、脾胃虚寒等,而且这类食物对金属类毒物具有一定的解毒作用。

(3)酸味食物

如乌梅、山楂、石榴等,具有增加食欲、健脾开胃、收敛、固涩作用,且可增强肝脏功能,提高钙、磷的吸收率,有利于食物的消化和防止消化道感染。通常可用以治疗汗证、泄泻、遗精、带下等病证。但过食酸味,可损伤胃粘膜而致溃疡发生;影响牙齿的坚固,使消化功能紊乱。

(4)苦味食物

如杏仁、苦瓜、马兰等,具有宣泄、清热、燥湿、消炎、抗菌等作用,通常可用以治疗热证、心烦、湿证、咳喘等。而苦味过浓可抑制味觉神经,导致呕吐恶心。

(5)咸味食物

如海带、海蛰、海藻等,具有散结、软坚作用,通常可用以治疗瘰疬、痰核、瘿瘤等。现代药理学认为此类食物中的钾、钠氧化物、溴化物及碘化物含量较高。咸可调味,增加食欲,促进水盐代谢,但食咸过量可使水盐代谢紊乱,血容量增加而血压升高。有调査发现,喜食盐或口重者,患食管癌的可能性比一般人高12.3倍,因此,“谨和五味”理论,值得在临床及生活中切实遵循。

\section{素問·玉機真藏論(節選)}%第二节

\biaoti{【原文】}

\begin{yuanwen}
五藏受氣於其所生\sb{1},傳之於其所勝\sb{2},氣舍於其所生,死於其所不勝\sb{3}。病之且死,必先傳行,至其所不勝,病乃死\sb{4}。此言氣之逆行\sb{5}也,故死。肝受氣於心,傳之與脾,氣舍於腎,至肺而死。心受氣於脾,傳之於肺,氣舍於肝,至腎而死。脾受氣於肺,傳之於腎,氣舍於心,至肝而死。肺受氣於腎,傳之於肝,氣舍於脾,至心而死。腎受氣於肝,傳之於心,氣舍於肺,至脾而死。此皆逆死\sb{6}也。一日一夜五分之,此所以占死生之早暮也\sb{7}。

黃帝曰:五藏相通,移皆有次;五藏有病,則各傳其所勝\sb{8}。不治,法三月,若六月,若三日,若六日,傳五藏而當死\sb{9}。是順傳所勝之次。故曰:別於陽者,知病從來,別於陰者,知死生之期\sb{10},言知\sb{11}至其所困而死\sb{12}。
\end{yuanwen}

\biaoti{【校注】}

\begin{jiaozhu}
  \item 五脏受气于其所生:受气,遭受病气。所生,指我生之脏。全句指五脏从其所生的子脏接受病气,即子病传母。如心病传肝。
  \item 传之于其所胜:所胜,即我克之脏。本句为插入语,言五脏疾病的一般传变规律是相克而传,即下文所说的顺传,如肝病传脾等。
  \item 气舍于其所生,死于其所不胜:舍,留止也。所生,此处指生我之脏,即:母脏。所不胜,指克我之脏。全句言病气的留舍按子病传母的方式传变,若传至克我之脏时,就有死亡的可能。如肝病气留舍于母脏肾,进而传至肺,因肺金克肝木,故肝病传至肺时就有死亡的可能。
  \item 病之且死,必先传行,至其所不胜,病乃死:疾病发展到将要死亡之时,一般来说,病气将传克我之脏。如心病传肝,再传至肾,肾为心之所不胜,故心病传至肾,就有死亡的可能。
  \item 气之逆行:指病气的逆传,即上文子病传母的疾病传变方式,因其与一般相克而传的顺传方式不同,故曰“逆行”。
  \item 逆死:逆行传变至克我之脏,预后不良,有死亡的可能。与上文“气之逆行”同义。
  \item 一日一夜五分之,此所以占死生之早暮也:占,预测。死生,偏义复词,即死亡。朝暮,即早晚,这里引申为时辰。全句言一昼夜十二时辰分属五脏,据此可以预测出五脏病气逆传至其所不胜而死的大约时辰。
  \item 五脏相通,移皆有次:五脏有病,则各传其所胜:此言五脏疾病相克而传的顺传方式。五脏之气相互贯通,五脏之气的转移有一定的次序,故五脏有病一般传其所胜之脏,如肝病传脾等。《新校正》云:“上文既言逆传,下文所言乃顺传之次也。”
  \item 不治,法三月,若六月,若三日,若六日,传五脏而当死:此指五脏病气各传其所胜,推测其死期的约略时数。张介宾注:“病不早治,必至相传,远则三月、六月,近则三日、六日,五脏传遍,于法当死。所谓三六者,盖天地之气,以六为节,如三阴三阳,是为六气,六阴六阳,是为十二月,故五脏相传之数,亦以三六为尽。若三月而传遍,一气一脏也;六月而传遍,一月一脏也;三日者,昼夜各一脏也;六日者,一日一脏也。脏惟五而传遍以六者,假令病始于肺,一也;肺传肝,二也;肝传脾,三也;脾传肾,西也;肾传心,五也;心复传肺,六也。是谓六传。六传已尽,不可再传,故《难经·五十三难》曰:一脏不再伤,七传者死也。”
  \item 别于阳者,知病从来,别于阴者,知死生之期:阳,指胃气脉;阴,言真脏脉。吴昆云:“阳,至和之脉,有胃气者也。阴,至不和之脉,真脏偏胜,无胃气者也。言能别于阳和之脉者,则一部不和便知其病之从来;别于真脏五阴脉者,则其死生之期可预知也。”
  \item 知:《甲乙经》无此字,可从。
  \item 至其所困而死:指至其所不胜的脏气当旺之时令则死,如脾病至肝当旺之时,则土不胜木克,故死。张介宾注:“至其所困而死,死于其所不胜也,凡年、月、日、时,其候皆然。”
\end{jiaozhu}

\biaoti{【理论阐释】}

1.五脏疾病的传变方式

本段主要论述五脏疾病的两种传变方式及其预后。一为逆行传变,即子病传母的疾病传变方式,如肝传肾、肾传肺、肺传脾、脾传心。因与相克而传的顺传方式不同,故曰“逆行”。若进一步传变至克己之脏,脏气被克,正气更虚,则预后差。如肝病传到肺、肺病传到心、心病传到肾、肾病传到脾、脾病传到肝等。二为顺传,即相克关系而传变的方式,如肝传脾、脾传肾、肾传心、心传肺、肺传肝等。待五脏传遍,脏气已竭,就要死亡。这是以五行生克关系说明人体是一个统一的整体,五脏之间在生理病理上都有着密切的联系,任何一脏发病,皆能传变至其他脏腑,传变的速度有快有慢,慢则三个月、六个月传遍五脏,快则三、六日传遍五脏,因此在诊断时,既要了解各脏腑病变“至其所困而死”的基本规律,又要“一日一夜五分之”来测候病甚及死亡的早晚,从而做到诊断明确,能根据病情,预见其传变,及早采取治疗措施,避免病情恶化。《素问·脏气法时论》中也有“邪气之客于身也,以胜相加,至其所生而愈,至其所不胜而甚,至于所生而持,自得其位而起”的论述。可互参。

2.关于“一日一夜五分之,所以占死生之早暮也”

结合《素问·生气通天论》有关阳气昼夜消长变化的论述来看,一日之间阴阳消长变化与人体的机能活动确有密切关系,尤其在病理过程中表现更为明显,如“旦慧、昼安、夕加、夜甚”(《灵枢·顺气一日分为四时》),因此,如何正确把握疾病发展的规律,是古今医家研究的重要课题。在此基础上,预测疾病的死生,是完全有可能的。

\biaoti{【原文】}

\begin{yuanwen}
是故風者,百病之長\sb{1}也。今風寒客於人,使人毫毛畢直,皮膚閉而爲熱\sb{2},當是之時,可汗而發也。或痹不仁、腫痛\sb{3},當是之時,可湯熨及火灸刺而去之\sb{4}。弗治,病人舍於肺,名曰肺痹\sb{5},發咳上氣。弗治,肺即傳而行之肝,病名曰肝痹,一名曰厥\sb{6},脅痛出食\sb{7},當是之時,可按若刺\sb{8}耳。弗治,肝傳之脾,病名曰脾風\sb{9},發癉\sb{10},腹中熱,煩心,出黃\sb{11},當此之時,可按可藥可浴。弗治,脾傳之腎,病名日疝瘕\sb{12},少腹冤熱\sb{13}而痛,出白\sb{14},一名曰蠱\sb{15},當此之時,可按可藥。弗治,腎傅之心,病筋脈相引而急,病名日瘛\sb{16},當此之時,可灸可藥。弗治,滿十日,法當死\sb{17}。腎因傅之心,心即復反傳而行之肺,發寒熱,法當三歲死\sb{18},此病之次\sb{19}也。

然其卒發者,不必治於傳\sb{20},或其傅化有不以次。不以次人者,憂恐悲喜怒\sb{21},令不得以其次,故令人有大病矣。因而喜,大虛,則腎氣乘矣\sb{22},怒則肝氣乘矣,悲則肺氣乘矣,恐則脾氣乘矣,憂則心氣乘矣,此其道也。故病有五,五五二十五變\sb{23},及其傳化。傳,乘之名也\sb{24}。
\end{yuanwen}

\biaoti{【校注】}

\begin{jiaozhu}
  \item 风者,百病之长:长,首也。风为六淫之首,常为外邪致病的先导,又善行数变,故称百病之长。
  \item 皮肤闭而为热:寒邪束表,腠理闭塞,故皮肤闭而无汗;卫阳之气内郁,故发热。
  \item 痹不仁、肿痛:风寒留舍经脉,闭阻脉道,气血运行不畅,故见麻痹不仁、肢体肿胀疼痛诸症。
  \item 可汤熨及火灸刺而去之:汤,用热水洗浴。熨,用布裹热药在体表来回温熨。火灸,用火熏灼。刺,针刺。去之,即祛除病邪。
  \item 肺痹:病证名。指因肺气闭阻不通,以发咳上气为主症的病证。
  \item 病名曰肝痹,一名曰厥:肝痹,病证名。此指以胁痛出食为主症的病证。厥,逆也。
  \item 胁痛出食:胁痛,肝病也。出食,食入而出,脾病也。胁痛出食,肝病传脾之兆。
  \item 可按若刺:按,按摩。若,与也。言肝痹可用按摩与针刺进行治疗。
  \item 脾风:病证名。此指以发瘅,腹中热,烦心,出黄为主症的病证。王冰注:“肝气应风,木胜脾土,土受风气,故曰脾风。盖为风气通肝而为名也。”
  \item 发瘅:瘅,通疸。发瘅,发为黄疸。王冰注:“脾之为病,善发黄疽,故发瘅也。”
  \item 出黄:指小便色黄。
  \item 疝瘕:病证名。此指脾经湿热下注于肾,湿热结聚少腹,气机被阻而以少腹冤热而痛、出白为主症的病证。
  \item 冤热:吴昆注:“冤热,烦热也。”
  \item 出白:张介宾注:“溲出白浊也。”
  \item 蛊:毒虫,在此为病证名。因病邪深入,真阴亏蚀,形体消瘦,如被蛊虫吸蚀一样,故名。张介宾注:“热结不散,亏蚀其阴,如虫之吸血,故亦名曰蛊。”
  \item 病筋脉相引而急,病名曰瘛:瘛,筋脉抽搐之证。心主血脉,心血不足,不能濡养筋脉,筋脉失养而为瘛。
  \item 满十日,法当死:吴昆注:“满十天则天干一周,五脏生意皆息,故死。”
  \item 法当三岁死:滑寿《读素问钞》注:“三岁,当作三日。”可从。此时病气由心再传至肺,使肺气更衰,甚至败绝,故曰三日而死。
  \item 此病之次:次,次序。言五脏疾病相胜而传的次序。高世拭注:“上文五脏相通,移皆有次者,相生之次也;此病之次,乃相胜之次也。”
  \item 然其卒发者,不必治于传:卒发,突然发作,指暴发的急病。此类疾病发病急骤,不按一般的传变规律,故治疗当根据病因、症状具体分析,不必拘泥于相传之次。
  \item 不以次人者,忧恐悲喜怒:忧恐悲喜怒等情志致病,直接损伤五脏之气,故不依次相传。王冰注:“忧恐悲喜怒,发无常分,触遇则发,故令病气亦不次而生。”
  \item 因而喜,大虚,则肾气乘矣:乘,相胜太过,以强凌弱也。心属火,心火不足,肾水乘而胜之。吴昆注;“喜则气缓,故过于喜,令心火虚,虚则肾气乘之,水胜火也。”下文诸“乘”皆同此类传变方式。
  \item 故病有五,五五二十五变:人有五脏,一脏有病则兼传其他四脏。每一脏病变有五,故五脏病变谓五五二十五变。
  \item 传,乘之名也:传,传变。乘,以强凌弱。疾病传变往往趁虚而传,含有以强凌弱的意思。吴昆注:“言传者,亦是相乘之异名耳。”
\end{jiaozhu}

\biaoti{【理论阐释】}

\xiaobt{关于随证而治}

本段阐述了外邪侵犯人体是由表入里,病情由轻及重,并以顺传为例说明五脏疾病相互传变的规律,提示对于疾病要做到及时诊断、及时治疗。当病邪尚在浅表时,就积极采用针刺、火灸、按摩、药物、汤浴、熨敷等各种治疗方法祛邪外出,恢复正气。一旦病邪入里入脏,更要既病防变,在掌握五脏疾病相克而传规律的基础上,进行有针对性的治疗。张仲景所说的“见肝之病,知肝传脾,当先实脾”,就是基于五脏疾病相克而传规律之上提出来的。

但是,临床上疾病多种多样,人的体质又有差异,证候表现不一,病情十分复杂。疾病的发展变化不是“逆行”、“顺传”两种方式所能涵盖的。如骤然暴发的疾病,并没有由表入里的过程,如伤寒直中,温疫暴发等,所以说“然其卒发者,不必治于传”。再如,七情致病,由内而发,随触而动,故发病亦不以其次。提示临床诊治疾病不可拘泥于五行关系下的“逆行”、“顺传”,更要从实际出发,灵活运用。

\biaoti{【临证指要】}

疾病传变的方式有顺传有逆传,此段原文举例了顺传(相胜而传)的临床表现和治疗方法,临床可结合具体情况用之。

如风寒初客肌表,卫阳被遏,腠理闭固,则畏寒发热,治宜发汗解表,麻黄汤主之。风寒留舍于经脉之中,气血运行不利,麻本不仁而为肿痛,可用温经散寒行气活血之汤熨、火灸法治之。邪入舍于肺,肺失宣发肃降,咳逆气上,宜降气平喘,三拗汤、桂枝加厚朴杏子汤等。传入于肝,两胁胀痛,肝木横逆克土,胃气上逆而呕吐,方以疏肝降逆之柴胡疏肝散加姜半夏、旋覆花、代赭石。传入于脾,脾湿郁而化热,形成湿热黄疸,腹中热,烦心小便黄,以茵陈五苓散治之。传之于肾,则有疝瘕之证,可见少腹引痛,烦热,小溲白浊等症状,此肾经湿热为病,可用萆薢分清饮清理之。传之于心,心火亢盛,邪热鸱张,热极生风,筋脉收引拘急抽瘛,可清心泻火,养阴熄风,清宫汤之类治之。若不能及时施治,而传遍五脏,则生命危殆。

\biaoti{【原文】}

\begin{yuanwen}
黃帝曰:余聞虛實以決死生,願聞其情。岐伯曰:五實死,五虛死。帝曰:願聞五實五虛。岐伯曰:脈盛、皮熱、腹脹、前後\sb{1}不通、問瞀\sb{2},此謂五實;脈細、皮寒、氣少、泄利前後、飲食不入,此謂五虛。帝曰:其時有生者何也?岐伯曰:漿粥入胃,泄注止,則虛者活\sb{3};身汗得後利,則實者活\sb{4}。此其候也。
\end{yuanwen}

\biaoti{【校注】}

\begin{jiaozhu}
  \item 前后:指大小便。
  \item 闷瞀:即胸中郁闷,眼目昏花。
  \item 浆粥入胃,泄注止,则虚者活:五脏之气,都是由胃气资生,今饮食能入,泄泻得止,为胃气来复的表现,所以五虚证预后转好。
  \item 身汗得后利,则实者活:身汗可解在表之实,后利能去在里之实,邪去则正安,所以五实证预后转好。
\end{jiaozhu}

\biaoti{【理论阐释】}

\xiaobt{关于五实证和五虚证}

本段论述了五实证和五虚证的内容及其预后的判断。《素问·通评虚实论》云:“邪气盛则实,精气夺则虚。”五实,是五脏邪气壅盛的实证。心主脉,心气实则脉盛;肺主皮毛,肺气实则皮热;脾主运化,脾气实则腹胀;肾主二阴,肾气实则二便不通;肝开窍于目,肝气实则闷瞀。五虚,是五脏精气虚损的虚证。心气虚则脉细,肺气虚则皮寒,肝气虚则气少乏力,肾气虚则二便不禁,脾气虚则不欲饮食。五虚死,五实死,说明其预后多不良,但又并非是绝对的。在一定条件下,其证尚有治愈之机。五实证生之转机在于“身汗得后利”,身汗则表邪解,后利则里邪除,使邪有出路,内外通和,提示实证治疗应以祛邪为先。五虚证生的转机在于“浆粥入胃,泄注止”,浆粥入胃则脾气渐运,气血生化有源;泄注止则肾气渐固,精气得以内藏。提示脾肾二脏对于五脏虚证的治疗有着决定性意义,联系《灵枢·本神》有关脾肾二脏功能失调导致五脏不安的论述,说明《内经》作者已经对先天之本肾、后天之本脾对调节全身脏腑功能的重要作用有了较为深刻的理解。

当然,临床上单纯的实证和虚证并不多见,疾病的症状往往错综复杂,病机虚实互见,故扶正祛邪,谁先谁后,孰重孰轻,当仔细斟酌,以免犯虚虚、实实之戒。

\biaoti{【临证指要】}

\xiaobt{使五实证、五虚证者“活”的临床指导意义}

五实证仅是本篇举实证之要而言,并非概括临床上全部危急之实证。“邪气盛则实”,“身汗得后利则实者活”,强调五实证的治疗转机,实质上是启发医生治疗实证要用“通”法,使邪有出路,邪去则正安。故临床上对危急的实证,如小大不利、高热不退、胸满气逆、腹胀如鼓、胃脘壅滞、欲吐不得、胎衣不下等,可采用泻法,以三承气汤攻之则小便通,大便利,腹胀除,壅滞去;催吐法亦是祛邪之通途,邪气在上而不得下,可因势利导,“其上者,因而越之,”用瓜蒂散、盐汤涌吐而祛邪,以少腹逐瘀汤攻下胎衣,为“留者攻之”法添一注脚。五虚证病情重笃,但只要胃气尚存,五脏精气仍有重建之希望,故“浆粥入胃,泄注止”为五虚证治疗指明了方向,虽正气已处衰微之时,然浆粥能够入胃,则说明残微的胃气有重振的机会,可谓开源有望;“泄注止”则使精气漏泄之处得到固摄,是节流之法;临床上可用健运益胃,收敛止泄的方剂治之,如参苓白术散合附子理中汤。

\zuozhe{(周国琪)}

\section{素問·舉痛論(節選)}%第三節

\biaoti{【原文】}

\begin{yuanwen}
余知百病生於氣也\sb{1}。怒則氣上,喜則氣緩,悲則氣消,恐則氣下,寒則氣收,炅則氣泄,驚則氣亂,勞則氣耗,思則氣結,九氣不同,何病之生?岐估曰:怒則氣逆,甚則嘔血及飧泄\sb{2},故氣上矣。喜則氣和志達,榮衛通利,故氣緩\sb{3}矣。悲則心系急\sb{4},肺布葉舉\sb{5},而上焦不通,榮衛不散,熱氣在中,故氣消矣。恐則精卻\sb{5},卻則上焦閉,閉則氣還,還則下焦脹,故氣不行\sb{7}矣。寒則腠理閉,氣不行,故氣收\sb{8}矣。炅則腠理開,榮衛通,汗大泄,故氣泄\sb{9}。驚則心無所倚,神無所歸,慮無所定,故氣亂矣。勞則喘息汗出,外內皆越\sb{10},故氣耗矣。思則心有所存,神有所歸,正氣留而不行,故氣結矣。
\end{yuanwen}

\biaoti{【校注】}

\begin{jiaozhu}
  \item 百病生于气也:百病,泛指多种疾病。谓多种疾病的发生,都是由于气的失常所致。
  \item 呕血及飧泄:大怒伤肝,肝气上逆,血随气涌,故甚则呕血。肝气横逆,乘犯脾土,则为飧泄。又“飧泄”,《甲乙经》、《太素》均作“食而气逆”,可参。
  \item 气缓:含两义,即适度的喜能使气和志达,喜太过则气涣散不能收持。张介宾注:“气脉和调,故志畅达,荣卫通利,故气徐缓。然喜甚则气过于缓而渐至涣散,故《调经论》曰:喜则气下。《本神》篇曰:喜乐者,神惮散而不藏。义可知也”。
  \item 心系急:心系,指心与其它脏器相连系的络脉。急,拘急、牵引。
  \item 肺布叶举;谓肺叶张大。
  \item 恐则精却:却,退却,精气衰退之意。肾在志为恐,主藏精,恐惧太过则耗伤肾精,故致精却。
  \item 气不行:林亿《新校正》云:“当作‘气下行’也。”又高世栻注:“恐伤肾而上下不交,故气不行。不行者,不行于上也。恐则气下,以此故也。”
  \item 气收:卫阳郁遏之谓。张介宾注:“寒束于外则玄府闭密,阳气不能宜达,故收敛于中而不得散也。”
  \item 气泄:热则汗出,气随汗泄,故称气泄。
  \item 外内皆越:越,散越之意。指人体正气外内两方面消耗亏损。马莳注:“人有劳役,则气动而喘息,其汗必出于外,夫喘则内气越,汗则外气越,故气以之而耗散也。”
\end{jiaozhu}

\biaoti{【理论阐释】}

1.百病生于气

气是脏腑经络组织功能活动的体现,同时又是构成和维持人体生命活动的最基本物质。气布散于全身,无处不有,无时不在,运行不息,不断地推动和激发脏腑经络组织器官的生理活动。气的活动正常,就是人体生理;气的活动失常,则为人体病理。正如张介宾所说:“气之在人,和则为正气,不和则为邪气。凡表里虚实,逆顺缓急,无不因气而至,故百病皆生于气。”(《类经·疾病类》)原文列举外感六淫、情志过激、劳倦太过等因素影响人体,或致脏腑组织耗气太过,或致脏腑之气升降出入失常,出现气“上、缓、消、下、乱、结、收、泄、耗”等不同病理改变,产生诸多病证。文中总结了气机失调的九种病机模式,强调了气机逆乱是百病产生的根源。九气为病中,情志致病占了六种,可见情志因素在《内经》病因学中占有相当重要的地位。

《内经》主要以两种形式归纳气的失常,一是耗用太过致使气虚,二是病因干扰致使气机失调。

(1)气虚

气虚的形成原因主要有两方面:一是气的生成不足。如禀赋不足,先天精气匮乏;脾胃虚弱,纳运失常,水谷精气亏虚;肺之功能减弱,吸入清气减少,致使气的生化乏源。二是气的消耗太多。如后天调养失宜,邪伤正气,久病、重病消耗等。本篇所言,则为耗用太过致虚。如外感火热邪气,致腠理开疏,汗太泄,气随津泄造成气虚,即“炅则腠理开,荣卫通,汗大泄,故气泄”。又情志郁而化热,热则伤津耗气而致气虚,即“悲则心系急,肺布叶举,而上焦不通,荣卫不散,热气在中,故气消矣。”此外,劳倦太过,致喘息汗出而消耗精气,即“劳则喘息汗出,内外皆越,故气耗矣”。气的功能主要表现在推动、温煦、防御、固摄、气化等方面,因此,气虚常出现为推动无力,温煦失职,防御功能减退,固摄失常,气化不足等病理改变。临床以神疲乏力,头昏自汗,声低息微,饮食减少,容易感冒,舌淡脉虚等为主要表现。原文中的汗大泄,喘息汗出,即是气的功能减弱所致。

(2)气机失调

是指气的升降出入运动失常的病理。在疾病过程中,由于致病因素的影响,或脏腑功能发生障碍,导致气运行不畅或升降出入运动失去协调。气机失调在本篇中的表现主要有气机郁滞、气机逆乱、气机下陷和气机闭阻等方面。

气机郁滞:指气的运行不畅,或停滞郁阻的病理状态。气机郁滞的形成多因情志不遂而脏气不舒所致,如本篇所言“思则气结”,《灵枢·本神》“愁忧者,气闭塞而不行”均属此列。此外,痰湿、瘀血、结石、宿食、虫积等有形之邪阻滞脏腑经络以及外邪入侵,如“寒则气收”等,亦可引起气机郁滞的发生,气机郁滞的病机以全身气机不畅或局部气机郁阻为特征,因气机郁滞所在部位的不同,其证候表现各具特点,但临床总以胀闷、疼痛为主。

气机逆乱:逆之含义有二,一是方向相反,现代中医界认为以不降反升或上升太过称上逆。二是抵触、不顺从、妄行称逆乱。《内经》所论的气机逆乱,既有全身阴阳、清浊、营卫之气运行逆乱,也包括脏腑经络之气妄行反作。本篇所言气机上逆、气机紊乱等,当属脏腑气机逆乱之类。气机上逆,指气的上升运动太过或下降运动不及的病理状态。升降运动是脏腑的特性,由于病因影响,可致脏腑气机失常,如胃、肺、心之气易失降而上逆,肝气易上升太过而冲逆。原文“怒则气逆,甚则呕血及飧泄,故气上矣”,即为大怒致肝气上逆,血随气升而病。若因意外的非常事故干扰人体,机体自身无法调控,致脏腑气机紊乱,气血失调,心失所养,神无所归,亦会产生“气乱”的病证,即如“惊则心无所倚,神无所归,虑无所定,故气乱矣”。

气机下陷:指气下降运动太过或上升运动不及的病理状态,多由气虚病变发展而来。气陷以脾、肾两脏为常见,如《素问·阴阳应象大论》指出;“清气在下,则生飧泄”,《灵枢·口问》说:“上气不足,脑为之不满,耳为之苦鸣,头为之苦倾,目为之眩”,本篇载“恐则精却,却则上焦闭,闭则气还,还则下焦胀,故气下(见注\myjzref{7})行矣。”皆论脾、肾气虚不足,升举、封藏失职,而表现出眩晕,飧泄,二便失禁,遗精滑泄等气陷的病证。

气机闭阻:指全身气机闭郁或重要脏腑气机闭塞不行的病理状态。轻者昏厥呈一过性,重者多以突然意识丧失,呼吸窒息,二便不通或四肢厥逆为特征。《内经》讨论的暴厥、薄厥、尸厥以及本篇的大厥即是以阴阳气血逆乱,闭阻不行为其病机,其证尤甚于“气机郁滞”。

2.情志致病

《素问·阴阳应象大论》云:“人有五脏化五气,以生喜怒悲忧恐”,指出只有通过脏腑功能活动才能产生各种情志变化,中医将这些情志变化概称为“五志”或“七情”。情志活动在病因或非病因之间具有相对性,即人体正气旺盛,气血和平,脏腑功能协调,对外界各种刺激反应正常,此属生理范围;只有当外界各种刺激过于强烈或持久,超过了机体生理和心理的适应调节限度,引起脏腑气血功能紊乱,这时的情志活动便会成为内源性病因,导致疾病的发生。《内经》认为脏腑气血的盛衰常变以及体质的强弱是决定情志是否致病的主要因素,如《素间·经脉别论》所言:“勇者气行则已,怯者则着而为病也”。《内经》论述情志致病的特点,归纳大约有四:

一是首先犯心,多产生精神情志病证。《灵枢·本神》指出:“所以任物者谓之心”;《灵枢·邪客》亦说:“心者,五脏六腑之大主也,精神之所舍也”。张介宾解释:“心为五脏六腑之大主,而总统魂魄,兼该志意,故忧动于心则肺应,思动于心则脾应,怒动于心则肝应,恐动于心则肾应,此所以五志唯心所使也”。由于心主宰统帅着人的精神情志活动,故当情志活动失常导致人体发病,则首先会影响到心及其主宰的精神意识思维活动,产生各种精神情志病证。如《灵枢·口问》云:“悲哀愁忧则心动,心动则五脏六腑皆摇”,张介宾亦认为“情志之伤,虽五脏各有所属,然求其所由,则无不从心而发”。《类经·疾病类》)常发的精神病证,如《灵枢·巅狂》言:“狂言,惊,善笑,好歌乐,妄行不休者,得之大恐”;《灵枢·本神》说:“心,休惕思虑则伤神,神伤则恐惧自失”;“肺,喜乐无极则伤魄,魄伤则狂,狂者意不存人”等皆是也。

二是直接损伤内脏,导致脏腑气机紊乱。由于五脏与情志活动有对应配属的关系,故不同情志所伤之脏不同,其气机失调的表现也各异。如《素问·阴阳应象大论》云:“怒伤肝”、“喜伤心”、“思伤脾”、“忧伤肺”、“恐伤肾”;本文云:“怒则气上、喜则气缓,悲则气消,恐则气下”,“惊则气乱”,“思则气结”等。

三是加重病情。一般来说,适度喜有益于病体的康复,某些情志活动能缓解甚至消除另一些情志性疾患,但是不良的情志刺激则往往会加重病情,甚至导致死亡,正如《素问·汤液醪醴论》所说:“嗜欲无穷而忧患不止,精气弛坏,荣泣卫除,故神去之而病不愈也”。

四是发无常分,触遇则发。《素问·玉机真脏论》云:“忧恐悲喜怒,令不得以其次”,指出情志过激发病,因情而发,证候无常,在时间和部位上均无规律可循。

3.关于惊与恐

惊与恐,同属七情范畴,《内经》认为二者致病的病机迥然有别,如本文所云:“恐则气下”、“惊则气乱”。溯其原因,惊是突然遭受意外的非常事故,情志刺激超越了机体的适应限度,故致“气乱”;恐多因惧怕受到曾经遭受的或耳闻的可怕经历而产生,故致“精却”、“气下”。作为外来的情志刺激,惊是突暴而短暂的,恐是长期而过度的;惊来自外界,恐多发自内心。正如《儒门事亲·内伤形》云:“惊者为阳,从外入也。恐则为阴,从内出。惊者为自不知故也,恐者自知也”。治疗上二者亦不相同,对于恐的治疗,《素问·阴阳应象大论》提出“思胜恐”;而对于惊,《素问·至真要大论》则提出“惊者平之”。

\biaoti{【临证指要】}

\xiaobt{七情致病治重理气调神}

情志为病各具特点,证候表现亦非常复杂,但其病机主要责之于五脏气机失调和心神失常。本篇所言:“怒则气上、喜则气缓、悲则气消、恐则气下”、“惊则气乱”、“思则气结”;《灵枢·寿夭刚柔》云:“忧愁忿怒伤气,气伤脏,乃病脏”。均表现出五脏气机失调,升降出入阻滞和紊乱的病理,甚者可损伤脏腑气血阴阳,出现气虚、血虚、阴虚、阳虚等病证。故临床治疗情志所致病证,首当调理脏腑之气机,《素问·至真要大论》中“结者散之……逸者行之,惊者平之”等法,主要为调理气机而设。又由于心为五脏六腑之大主,为“君主之官,神明出焉”(《素问·灵兰秘典论》),故情志过激致病,常使心神受干扰或损伤,而后致脏腑气机失调,出现各脏腑功能失常及精神情志异常的病证,因此治疗情志之病,还当重视养心调神。例如,刘冠军《现代针灸医案选·气厥》载以疏肝理气,开窍醒神法治气厥病案:张某,女,28岁。因情志抑郁,经常胸闷不舒,头痛头晕,今与其夫发生口角,突然晕倒,不省人事,四肢厥冷,口噤拳握,呼吸气促,给予指切人中,按压肘窝、腿弯后苏醒,醒后哭啼不休、少时又突然抽搐,口噤不开,呼吸气促,脉沉弦。析为气机逆乱,上壅心胸,蒙闭窍隧的气厥病,治以疏肝理气,开窍醒神,针刺人中、四关,两次即愈。

神志之病除用针刺或药物治疗外,还可根据五行相胜理论,运用“以情胜情”的调神方法,如《素问·阴阳应象大论》提出“怒伤肝,悲胜怒……喜伤心,恐胜喜……思伤脾,怒胜思……忧伤肺,喜胜忧……恐伤肾,思胜恐”。历代医者屡用不鲜,医案中多有记载,兹举一例示之:一女新嫁后,其夫经商二年不归,因不食,困卧如痴,无他病,多向里床坐。丹溪诊之,肝脉弦出寸口,曰:此思男子不得,气结于脾,药独难治,得喜可解。不然,令其怒。脾主思,过思则脾气结而不食,怒属肝本,木能克土,怒则肝气升发而冲开脾气矣。其父掌其面,呵责之,号泣大怒,至三时许,令慰解之,与药一服,即索粥食矣。朱曰:思气虽解,必得喜,庶不再结。乃诈以夫有书,旦夕且归。后三月,夫果归而愈。(《古今医案按·七情·思》)

\section{素問·調經論(節選)}%第四節

\biaoti{【原文】}

\begin{yuanwen}
黃帝問曰:余聞刺法\sb{1}言,有餘寫之,不足補之,何謂有餘?何謂不足?岐伯對曰:有餘有五,不足亦有五,帝欲何問?帝曰:願盡聞之。岐伯曰:神\sb{2}有餘有不足,氣有餘有不足,血有餘有不足,形有餘有不足,志有餘有不足。凡此十者,其氣不等\sb{3}也。帝曰:人有精氣津液,四支九竅,五藏十六部\sb{4},三百六十五節\sb{5},乃生百病,百病之生,皆有虛實。今夫子乃言有餘有五,不足亦有五,何以生之乎?岐伯曰:皆生於五藏也。夫心藏神,肺藏氣,肝藏血,脾藏肉,腎藏志,而此成形\sb{6}。志意通,內連骨髓,而成身形五藏\sb{7}。五藏之道,皆出於經隧,以行血氣,血氣不和,百病乃變化而生,是故守經隧\sb{8}焉。
\end{yuanwen}

\biaoti{【校注】}

\begin{jiaozhu}
  \item 刺法:古代有关针刺方法的文献。
  \item 神:《甲乙经》神下补“有”,于义为顺。下文“气”、“血”、“形”、“志”并同。
  \item 其气不等:气,指脏气。即五脏之气各有虚实不同。张介宾注:“神属心,气属肺,血属肝,形属脾,志属肾,各有虚实,故其气不等。”
  \item 十六部:张志聪注:“十六部者,十六部之经脉也,手足经脉十二,跷脉二,督脉一,任脉一,共十六部。”又《黄帝内经研究大成·藏象研究》认为十六部指:毛、皮、络、.经、腠、肉、脉、筋、骨、上、下、内、外、左、右、中部。
  \item 三百六十五节:节,此指穴位言。即三百六十五俞穴。
  \item 而此成形:张琦《素问释义》认为“四字衍”。《甲乙经》无此四字,张说可参。
  \item 志意通,内连骨髓,而成身形五脏:志意,代指五神;骨髓,代指五体;言神对形体内脏的作用。张介宾注:“志意者,统言人身之五神也。骨髓者,极言深邃之化生也。五神藏于五脏而心为之主,故志意通调,内连骨髄,以成身形五脏,则互相为用矣。”
  \item 守经隧:守,防守、保卫之意,引申为保持。经隧,指经脉。守经璲,即保持经脉的通畅。
\end{jiaozhu}

\biaoti{【理论阐释】}

\xiaobt{经络的作用及“守经隧”的意义}

经络运行人身气血,调节机体阴阳,使体内脏腑与五官九窍、皮肉筋骨相耳配合,协调一致,构成一个有机的整体。可见经络既是人体组织结构的重要组成部分,又是维持人体脏腑组织器官生理活动不可缺少的信息系统,还是气血运行的重要径路,故经络在人体整个生命活动中发挥着极为重要的作用。《内经》论经脉的作用,主要有以下三个方面:

一是运行气血,防御外邪。人体五脏六腑,形体官窍进行正常的生理活动,离不开气血的濡润和滋养,而气血必须通过经络的输注,才能通达全身,正如《灵枢·本脏》说:“经络者,所以行血气而营阴阳,濡筋骨,利关节者也”。经络将气血输送到全身各处,“内溉脏腑,外濡肌腠”(《灵枢·脉度》),发挥其营养脏腑组织、维持生命活动的作用。《素问·气穴论》认为孙脉能“溢奇邪”、“通荣卫”,说明经络既是运行气血的通路,还是邪气入侵的途径,故外邪入侵,通过经络布散,可由表入里,由浅入深,累及脏腑,产生各种疾病。如《素问·皮部论》说:“邪客于皮肤则腠理开,开则邪入客于络脉,络脉满则注于经脉,经脉满则入舍于腑脏也”。所以经脉的功能正常,经气和利,气血调畅,脏腑组织得养,则能抵御外邪的侵袭。本篇所言:“五脏之道,皆出于经隧,以行血气,血气不和,百病乃变化而生”,即强调了经脉与脏腑之间的密切联系,同时认为经气不调,则气血不和,可令邪气内入,导致百病发生。

二是内属脏腑,外连肢体官窍。人体经络贯穿于内脏和体表之间,如《灵柩·经脉》记载了十二正经的循行,指出每一条经脉都络属相应的脏腑,从而构成脏腑之间的表里关系。脏腑与外周肢节,五官九窍的联系,也主要通过十二经脉的联络作用来实现,《灵枢·海论》云“夫十二经脉者,内属于脏腑,外络于肢节”。正是通过经络的沟通联结,使脏腑组织器官成为统一协调的有机整体。

三是协调阴,调理虚实。疾病的产生,多由于“血气不和”所致,而气血必由经脉而输布周身,当经脉中气血偏盛或不足时,首先可导致经气偏盛偏衰,如《灵枢·经脉》云:“大肠手阳明之脉,……气有余则当脉所过者热肿,虚则寒栗不复”,同时经气的盛衰,又可累及相关脏腑,导致该脏腑功能失常,产生成实或虚的证候。如《灵枢·经脉》云:“胃足阳明之脉,……气盛则身以前热,其有余于胃,则消谷善饥;溺色黄,气不足则身以前皆寒栗,胃中寒则胀满。”此外,由于多种病因的复杂作用,还可以出现人体一部分经气偏盛而另一部分经气偏虚的虚实错杂的病理状态,如本篇所云:“气血以并,阴阳相倾,气乱于卫,血逆于经,血气离居,一实一虚”;《素问·五脏生成论》亦云:“是以头痛巅疾,下虚上实,过在足少阴巨阳”。通过调理经脉,补虚泻实,令气血调畅,五脏安定,阴阳才能恢复协调平衡状态。由于经络有运行气血,联络脏腑组织,以及协调阴阳,调理虚实,防御外邪的重要作用,故《灵枢·经脉》强调:“经脉者,所以能决死生,处百病,调虚实,不可不通”,此即原文“守经隧”的意义所在。

\biaoti{【原文】}

\begin{yuanwen}
帝曰:神有餘不足何如?岐伯曰:神有餘則笑不休,神不足則悲。血氣未并\sb{1},五藏安定,邪客於形,洒淅起於毫毛,未入於經絡也,故命曰神之微\sb{2}。帝曰:補寫奈何?岐伯曰:神有餘,則寫其小絡之血,出血,勿之深斥\sb{3},無中其大經,神氣乃平。神不足者,視其虛絡,按而致之\sb{4},刺而利之\sb{5},無出其血,無泄其氣,以通其經,神氣乃平。帝曰:刺微奈何?岐伯曰:按摩勿釋\sb{6},著針勿斥\sb{7},移氣於不足\sb{8},神氣乃得復。

帝曰:善。有餘\sb{9}不足奈何?岐伯曰:氣有餘則喘咳上氣,不足則息利少氣\sb{10}。血氣未并,五藏安定,皮膚微病,命曰白氣微泄\sb{11}。帝曰:補寫奈何?岐伯曰:氣有餘,則寫其經隧\sb{12},無傷其經,無出其血,無泄其氣。不足,則補其經遂,無出其氣。帝曰:刺微奈何?岐伯曰:按摩勿釋,出鍼視之,曰我將深之,適人必革\sb{13},精氣自伏。邪氣散亂,無所休息,氣泄腠理,真氣乃相得。

帝曰:善。血有餘不足奈何?岐伯曰:血有餘則怒,不足則恐。血氣未并,五藏安定,孫鉻水溢\sb{14},則經有留血\sb{15}。帝曰:補寫奈何?岐伯曰:血有餘,則寫其盛經\sb{16},出其血。不足,則視其虛經,內鍼其脈中,久留而視,脈大,疾出其鍼,無令血泄。帝曰:刺留血奈何?岐伯曰:視其血絡,刺出其血,無令惡血得入於經,以成其疾。

帝曰:善。形有餘不足奈何?岐伯曰:形有餘則腹脹,涇溲不利\sb{17},不足則四支不用。血氣未并,五藏安定,肌肉蠕動,命曰微風\sb{18}。帝曰:補寫奈何?岐伯曰:形有餘則寫其陽經,不足則補其陽絡\sb{19}。帝曰:刺微奈何?岐伯曰:取分肉間,無中其經,無傷其絡,衛氣得復,邪氣乃索\sb{20}。

帝曰:善。志有餘不足奈何?岐伯曰:志有餘則腹脹飧泄,不足則厥。血氣未并,五藏安定,骨節有動\sb{21}。帝曰:補寫奈何?岐伯曰:志有餘則寫然筋\sb{22}血者,不足則補其復溜\sb{23}。帝曰:刺未并奈何?岐伯曰:即取之,無中其經,邪所乃能立虛\sb{24}。
\end{yuanwen}

\biaoti{【校注】}

\begin{jiaozhu}
  \item 血气未并:并,合并、偏聚。血气未并,指气血无偏盛偏衰之象。
  \item 神之微:微,指表浅。神之微,指神的病变在肌表毫毛,未入经脉脏腑。
  \item 勿之深斥:深,深刺。斥,开拓扩大之意。意即不要深刺和摇大针孔。
  \item 按而致之:按,按摩。按摩穴位,使气血通达,充实于虚络。
  \item 刺而利之:利,疏利,和畅。针刺令经脉气血和畅。
  \item 按摩勿释:勿释,不放手。指按摩时间延长些。
  \item 著针勿斥:留针而不宜摇大针孔。
  \item 移气于不足:邪在皮毛,表卫不足,针刺引导正气于肌表,故谓移气于不足。
  \item 有余:参《太素》及吴注本,前而佚“气”字,当补。
  \item 息利少气:呼吸通畅,但气短无力。
  \item 白气微泄:肺主气,其色白,故以白气代称肺气。白气微泄,指肺气微虚。王冰注:“肺合皮毛,其色白,故皮肤微病,命曰白气微泄。”
  \item 经隧:即经脉。此指邪客而有余之脉。
  \item 适人必革:革,变革。全句意为:持针佯言深刺,待病人精神状态发生改变,意志内守时才入针浅刺。
  \item 孙络水溢:水,《甲乙经》、《太素》均作“外”,可从。孙络外溢,此指邪气充斥络脉,象水满外溢一样地流入经脉。
  \item 经有留血:指经脉血行留滞不畅。
  \item 盛经:与下句的“虚经”皆系指肝经之虚实。
  \item 泾溲不利:王冰注:“泾,大便,溲,小便也。”即指二便不利。
  \item 肌肉蠕动,命曰微风:即风邪入侵肌肉,肌肉似有虫爬行的感觉,因其属于风邪为患的轻症,故名微风。
  \item 阳经……阳络:指足阳明胃经及其络脉。因足阳明胃经与足太阴脾经为表里,故脾病可取阳明治之。
  \item 邪气乃索:索,散也。指邪气乃消散。
  \item 骨节有动:动,指变动,异常变化。此句指骨节间发生病变。
  \item 然筋:即然谷穴,是足少阴经之荥穴。
  \item 复溜:穴名,在足内踝上二寸处,属足少阴肾经。
  \item 邪所乃能立虚:邪所,指邪居之处。虚,指邪气去。意指经过针刺,病邪很快祛除。
\end{jiaozhu}

\biaoti{【理论阐释】}

\xiaobt{五脏虚实的病机}

原文对“神、气、血、形、志”有余、不足的论述,实质上是讨论心、肺、肝、脾、肾五脏系统的虚实病证。虚实病证的形成,从本文“血气未井”为“神之微”推之,当为“血气已并”所致,其实者为气机阻滞,其虚者乃精气不足,具体病机则需结合脏腑气血阴阳盛衰和致病因素的影响进行分析。

(1)神病

心藏神,主神明,心病则神志失常,喜笑不休或悲。此为心之虚实病证举例。《灵枢·本神》云:“心藏脉,脉舍神,心气虚则悲,实则笑不休”;“喜乐者,神惮散而不藏”。《素问·阴阳应象大论》亦云:“暴喜伤阳”。皆明确指出伤及神明会出现情感异常的表现。心气虚实在临床常见有心经火盛,或痰火扰心,神不安舍及营阴不足,心神失养,神气涣散不收等证候。

(2)气病

肺藏气,司呼吸,肺病则呼吸异常,喘咳上气和息利少气。此为肺之虚实病证举例。《灵枢·本神》亦云:“肺藏气,……肺气虚则鼻塞不利,少气,实则喘喝胸盈仰息”。均是对肺病在呼吸方面的表现,肺气虚实在临床常见邪壅于上,影响肺之宣降,致肺气上逆,发为喘咳。或肺伤气虚,呼吸无力等病证。

(3)血病

肝藏血,主疏泄,肝病则情志失常,易怒或恐。此为肝主疏泄功能失常在情志方面的表现。《灵枢·本神》所云:“肝藏血,血舍魂,肝气虚则恐,实则怒”与之同理。肝气抑郁不伸故多怒,怒复激动肝气肝火,二者互为因果。临床以肝火上炎,肝阳上亢易见多怒;肝气不足,疏泄失职,气机不畅,或肝虚及肾,子盗母气易见善恐。

(4)形病

脾藏营,主肌肉,充形体,脾病则运化失常,形体、四肢失养。腹胀,泾溲不利和四肢不用均为脾之虚实病证举例。《灵枢·本神》云:“脾藏营,……脾气虚则四肢不用,五脏不安,实则腹胀,泾溲不利”,亦指脾病运化失常,形体、四肢失养的病证。临床常见邪气滞脾,运化失司,气机不利,而腹胀不通,或脾虚不运,水谷精气不足,四肢失养而肢体痿弱不用。

(5)志病

肾藏精,主水,志为肾之神而寄居于精,腹胀、飧泄和厥,是肾之虚实病证举例。《灵枢·本神》云:“肾藏精,精舍志,肾气虚则厥,实则胀,五脏不安”,亦指肾失气化,阴阳失调所致气机方面的病证。临床常见肾精气不足,导致阴阳失调,甚至逆乱而发生厥病;或肾脏受邪,关门不利,水液停聚,而发生飧泄、腹部胀满等病证。

\biaoti{【临证指要】}

\xiaobt{五脏虚实证治及临床意义}

神志之病,多由心发,心之实证可出现笑不休,心之虚证可产生悲,证之临床,完全吻合。心实者多是心火亢盛,神志被扰,出现发狂一类精神病。治以泻心火,安心神,每获良效。例如,万密斋治程氏子,未一岁,多笑,知其心火有余,令以川连、栀子、辰砂为丸,服之。三日后,笑渐少。(《续名医类案·咳嗽》)心虚者是心血不足,神无所养,神志不宁,出现悲伤多哭等情志病证,证类《金匮要略》之脏躁。治疗当以益心宁神为主,甘麦大枣汤用之颇验。

肺以清肃下降为顺,宣发为和。喘咳上气,息利少气均属肺之宣降失常所致,但有虚实之不间,“喘咳上气”,乃肺部受邪,宣肃失司,肺气逆上而发,属肺气壅实证,治宜祛邪肃肺降气以平喘止咳,效可应手。“息利少气”,多由久病伤肺或水不润金而成,为肺气不足之候。治宜补肺益气,或培土以生金,或滋肾水以养母气,当缓缓图功。例:陆祖愚治唐鸣和,平时有火症,因试事成痰火咳嗽,日夜吐黄痰二三碗,气逆喘急,饮食不进,服枳、梗、二陈尤甚,改服参、术几危。脉之,两手俱洪滑而数,乃用茯苓、桑皮、贝母、芩、连、花粉、元参、枳壳,加牛黄、竹沥,二三剂胸宽气缓,七八剂痰乃色白,去牛黄,三十余剂而安。(《续名医类案·喘》)本案情志郁热化火成痰,壅滞于肺而致喘咳,陆氏治以清肺泄热,降气豁痰,痰火去则喘咳渐平。

肝主疏泄喜条达,藏血而主怒。肝之实证,多见情志不达,肝郁气滞;或肝郁化火,肝火上炎:或暴怒伤肝,肝气逆上;或肝阳化风,阳升风动等,因肝气有余,血实气逆故易怒。治当平肝、情肝、泻肝。肝之虚证,可分肝气不足或肝血不足,二者常相互影响,肝气不足则肝血不生,血虚不能养肝,致使肝气更虚,或子盗母气则恐俱。治宜养血柔肝、疏肝,佐以安神定志,诸症可平。兹举肝实证一例示之:丹溪治一妇人,年十九岁,气实多怒不发,忽一日大发,叫而欲厥,盖痰闭于上,火起于下,上冲故也。与香附末五钱,甘草三钱,川芎七钱,童便、姜汁煎。又与青黛、人中白、香附末为丸,稍愈,后大吐乃安,复以导痰汤加姜炒黄连……当归龙荟丸。(《古今医案按·七情·怒》)此系郁怒伤肝,气郁痰结,痰火上扰,蒙闭清窍所致。证属实火,治疗当泻火调气涤痰,先祛其痰火,调其气机,继以凉肝豁痰,养心安神法而收效。

脾主运化水谷情微,充养身形五脏,故形病多关于脾。脾土壅滞,气机不畅,或湿邪困脾,碍脾健运,故为腹部胀满,二便不利。治宜祛邪运脾,脾得健运,其气自和。

若脾虚,运化失职,水谷精气衰少,四肢失养,久则痿弱不用,临床所见痿证,大多属于此类。治当补脾益气,健运升清,此亦为“治痿独取阳明”之法。例:单某,女20岁,四肢逐渐痿软无力两月余,神经科诊为“重症肌无力”。症见手指无力提笔写字,下午走路常易跌倒,右眼皮下垂,倦怠嗜睡,苔薄白,脉沉细。综合病情,属气血不足,脾虚肾亏之痿证。用炙黄芪15克、炒当归12克、丹参30克、红花9克、川芎12克、白菊花12克、枸杞12克、黄精15克、玉竹15克、桂枝6克、鹿角片15克、炙甘草6克,治20日,诸证好转,原方增减,以其肌肉筋骨得其所养,而渐趋恢复。(董建华《中国现代名中医医案精华(一)·杨继苏医案》)

肾的精气亏虚,厥气易于逆上,“阳气衰于下,则为寒厥”(《素问·厥论》),“肾气虚则厥”(《灵枢·本神》),故厥证常可表现为肢体厥冷,甚或出现阴阳之气不相顺接之昏厥。肾为胃之关,邪实于肾则关门不利,可见腹胀、飧泄等症。兹举病案证之:王某,男,60岁。素有腰膝酸痛、头晕、失眠、耳鸣、咽干等症。最近因思想紧张,恐怖不解,遂至卒然昏倒,诊为脑溢血。症见面红,痰声漉漉,牙关紧闭,舌红赤,脉弦大。患者素禀肾阴亏损,肝阳上亢,复因恐怖伤肾,肾精倍损,水不涵木,肝阳愈亢,阳热上冲,风痰交阻,此病肾阴亏损为本,肝风挟痰是标,治以滋养肾阴为主,潜阳熄风、豁痰开窍为辅,用六味地黄以养肾阴;加牡蛎、龙骨、白芍以养肝潜阳熄风;再加石菖蒲、远志、竹茹以豁痰开窍。致使阴足阳潜、风静痰消,诸症可冀缓解。(董建华《中国现代名中医医案精华(二)·李斯炽医案》)

\biaoti{【原文】}

\begin{yuanwen}
帝曰:善。余已聞虛實之形,不知其何以生。岐伯曰:氣血以并,陰陽相傾\sb{1},氣亂於衛,血逆於經\sb{2},血氣離居,一實一虛\sb{3}。血并於陰,氣并於陽,故爲驚狂\sb{4}。血并於陽,氣并於陰,乃爲炅中\sb{5}。血并於上,氣并於下,心煩惋善怒\sb{6}。血并於下,氣并於上,亂而喜忘\sb{7}。帝曰:血并於陰,氣并於陽,如是血氣離居,何者爲實?何者爲虛?岐伯曰:血氣者,喜溫而惡寒,寒則泣不能流,溫則消而去之\sb{8},是故氣之所并爲血虛,血之所并爲氣虛。帝曰:人之所有者,血與氣耳。今夫子乃言血并爲虛,氣并爲虛,是無實乎?岐伯曰:有者爲實,無者爲虛,故氣并則無血\sb{9},血并則無氣\sb{10},今血與氣相失,故爲虛焉。絡之舆孫脈俱输於經,血與氣并,則爲實焉。血之與氣,并走於上,則爲大厥\sb{11},厥則暴死,氣復反則生,不反則死。

帝曰:實者何道從來?虛者何道從去?虛實之要,願聞其故。岐伯曰:夫陰舆陽皆有俞會\sb{12},陽注於陰,陰滿之外\sb{13},陰陽勻平,以充其形,九候若一\sb{14},命曰平人。夫邪之生也,或生於陰,或生於陽。其生於陽者,得之風雨寒暑。其生於陰者,得之飲食居處,陰陽喜怒\sb{15}。帝曰:風雨之傷人奈何?岐伯曰:風雨之傷人也,先客於皮膚,傳入於孫脈,孫脈滿則傳入於絡脈,絡脈滿則輸於大經脈,血氣與邪并客於分腠之間,其脈堅大,故曰實。

實者外堅充滿\sb{16},不可按之,按之則痛。帝曰:寒濕之傷人奈何?歧伯曰:寒濕之中人也,皮膚不收\sb{17},肌肉堅緊,榮血泣,衛氣去,故曰虛。虛者聶辟氣不足\sb{18},按之則氣足以溫之,故快然而不痛。帝曰:善。陰之生實\sb{19}奈何?岐伯曰:喜怒不節則陰氣上逆\sb{20},上逆則下虛,下虛則陽氣走之\sb{21},故曰實矣。帝曰:陰之生虛奈何?岐伯曰:喜則氣下\sb{22},悲則氣消,消則脈虛空,因寒飲食,寒氣熏滿\sb{23},則血泣氣去,故曰虛矣。
\end{yuanwen}

\biaoti{【校注】}

\begin{jiaozhu}
  \item 气血以并,阴阳相倾:并,合并,引伸为偏盛。倾,倾陷,倾斜,此指失调。全句指人体气血阴阳出现偏胜偏衰的病理。如气并于血,则气实而血虚;血并于气,则血实而气虚。
  \item 气乱于卫,血逆于经:卫属气,气乱于卫,故为气实。经行血,血逆于经,故为血实。
  \item 血气离居,一实一虚:张志聪注:“血并于气,则血离其居;气并于血,则气离其居矣。血离其居则血虚而气实,气离其居则气虚而血实,故曰一实一虚。盖有者为实,无者为虚也”。气血运行失调,不循常道而逆乱,即可产生血虚气实或气虚血实的病理。
  \item 血并于阴,气并于阳,故为惊狂:血属阴,气属阳。血并于阴则阴盛,“重阴者癫”;气并于阳则阳盛,“重阳者狂”。
  \item 血并于阳,气并于阴,乃为炅中:炅(jiǒnɡ音炯),热之义。炅中,即热中,指热盛于里。张志聪注:“炅,热也。血并于阳,则阴虚而生内热矣。气并于阴,则阳气内盛而为热中矣。”
  \item 血并于上,气并于下,心烦惋善怒:上、下,指心、肝。惋,音义同闷。《甲乙经》作“闷”,《太素》作“悗”,惋、闷、悗三字通。血为阴,并于上部,则心火为阴所蔽,故烦惋,气为阳,并于下部,则肝木为阳所炙,故为善怒。
  \item 血并于下,气并于上,乱而喜忘:血并于下则阴气不升,气并于上则阳气不降,阴阳离散,故神乱而喜忘。
  \item 温则消而去之:此句有二释:一从生理解,马莳注:“温则消释易行”。认为寒则凝而收引,血流缓慢,甚则凝涩不通;温则血行通利。二从病理释,黄元御《素问悬解》注:“温则消而去之,血气涣散,因而成虚。”即过热则消灼阴血而致血虚。二解均通,可互参。
  \item 气并则无血:无,此作“少”解,无血,即血少,血分不足。气并于血,气分偏胜,血分相对不足。
  \item 血并则无气:无气,即少气。血并于气,血分偏胜,气分相对不足。
  \item 大厥:指突然昏倒,不省人事的晕厥证。
  \item 夫阴与阳,皆有俞会:阴,指阴经;阳,指阳经。俞,指俞穴。会,会合之意。俞会,即阴阳经气输注会合之处。张志聪注:“俞者,谓三百六十五俞穴,乃血脉之所流注;会者,谓三百六十五会,乃神气之所游行,皆阴阳气血之所输会者也。”
  \item 阳注于阴,阴满之外:这里的阴阳,指阴经阳经。之,至也。外,此指阳经而言。全句即人体气血运行,阳经满溢,可注于阴经,阴经充满,可注于阳经。
  \item 九候若一:九候,古代诊脉部位,即上中下三部,每部各有天地人三候,合称为九候。若一,此指协调。三部九候之脉相互协调,是平人之脉象。
  \item 阴阳喜怒:阴阳,此指房事。喜怒,泛指七情。
  \item 外坚充满:坚,疑为“邪”之误。即外邪充满。
  \item 皮肤不收:《甲乙经》、《太素》无“不”字。杨上善注:“皮肤收者,言皮肤急而聚也。”与寒主凝滞、收引之特性相合。”
  \item 聂辟气不足:聂辟,王冰注:“聂,谓聂皱。辟,谓辟迭也。”指皮肤松弛皱起似有折迭,是正气虚衰的表现。
  \item 阴之生实:实,指邪气盛。此言内伤之实证。
  \item 阴气上逆:阴气,此指肝经之气上逆。
  \item 下虚则阳气走之:杨上善《黄帝内经太素·虚实所生》注:“人有喜怒,不能自节,故怒则阴气上,阴气上则上逆,或呕血,或不能食。阴气既上则是下虚,下虚则阳气乘之,故名曰阴实也。”即肝气上逆则下虚,阴虚于下,阳邪必凑之,所以为实。
  \item 喜则气下:《素问·举痛论》:“喜则气缓”。盖“缓”、“下”皆情志过喜引起心气的异常变化,只是气机的变化不同而已。
  \item 寒气熏满:《甲乙经》作“动脏”可从。即寒邪伤及内脏。
\end{jiaozhu}

\biaoti{【理论阐释】}

\xiaobt{虚实的病机}

关于虚实的病机,《内经》从两方面阐释:一是从邪正盛衰立论。邪正相搏贯穿疾病的全过程,邪正双方力量对比的消长盛衰变化,可致机体表现出或实或虚的病理状态,如《素问·通评虚实论》云“邪气盛则实,精气夺则虚”。在邪正相搏过程中,邪气亢盛,而正气未衰的病理状态为实,由此表现出的证候为实证。若以正气不足为矛盾主要方面,而邪气也不盛的病理状态为虚,由此表现出的证候为虚证。二是从气血逆乱,阴阳失衡立论。本篇从人体阴阳气血分布和运行的常变,探讨虚实病证产生的机理。认为虚实的产生,主要是由于“气血以并,阴阳相倾”而致。具体列举了“血并于阴,气并于阳”;“血并于阳,气并于阴”;“血并于上,气并于下”;“血并于下,气并于上”等多种情况。但其虚实病机均不外“有者为实,无者为虚”的道理。其“有者”、“无者”系指气血的偏盛偏衰而言,即“气并”或“血并”于某处,该部位即为“实”;反之,血或气离散于某处,该处即为“虚”。原文指出:“络之与孙脉俱输于经”,“血之与气,并走于上,则为大厥”,即气血并居于一个部位则为实之例证。结合临床“气并”可为气滞、气逆、气郁、气结、气闭;“血并”即成血瘀、血热、血寒;“气与血并”,则为气滞血瘀、气逆血涌、血寒气结等。由于气血相并的部位不同,可以表现出惊狂、喜怒、心烦惋、善忘等不同的病证。而气血离于某处,该处即产生气虚、气耗、气陷、气脱或血虚、血枯、血脱,甚至出现气不摄血、气随血脱、气血两虚的病理。此外,还有“一实一虚”同时存在的情况,称之虚实错杂。因为一方面的偏盛有余,容易导致另一方面的偏衰不足,即形成“一实一虚”。如气之所并为气实血虚,血之所并为血实气虚,气并入血分则血实气虚,血并入气分则气实血虚等。原文讨论的神、气、血、形、志五者有余不足,即由气血逆乱,阴阳偏盛偏衰所导致。

\biaoti{【临证指要】}

1.血气者,喜温而恶寒

寒温变化直接影响着气血的运行,经文“血气者,喜温而恶寒,寒则泣不能流,温则消而去之”,充分说明了血气有喜温恶寒的特性,遇寒则凝,得温则行,为临床治疗血气方面的病证指明了方向。如血气凝涩所致病证,因于寒者,理当温通;因于热者,亦当照顾其喜温恶寒的特性,不可过用寒凉,或于寒凉之中辅以温通之药。例:刘冠军《现代针灸医案选》治痛痹案:杨某,女,64岁。一月前因着凉致右上肢疼痛,虽经针药兼治,但肘关节疼痛不减,疼处不移,得热痛减,遇冷则剧,并日趋加重。査:右肘关节轻度肿胀,皮肤微凉,按之略有压痛,活动受限,右上肢不敢提物,舌质淡,苔薄白,脉沉紧。诊为痛痹,治宜温经通络散寒,用温针疗法。取左侧曲池、手三里穴,每日一次,一个疗程后诸证消失,功能恢复正常。按:温针能使局部红晕,血行旺盛,热可深透肌腠,内注筋骨,温通经脉,祛散寒邪,故用温针治疗痛痹效果满意。此亦验证“血气者,喜温而恶寒,……温则消而去之”在临床治疗中发挥的指导作用。

2.关于大厥

《内经》论厥,总以气血运行逆乱为其病机。认为人之气血,周流全身,循环不息,

贵在平衡协调,一有逆乱,百病乃变化而生。如因外感或内伤等致病因素干扰人体,引起气血并聚,逆乱于上,则可扰乱神明,出现突然昏倒,不省人事之大厥证。上逆之气血如能复返下行,即可生还;若气血依然上涌,则预后不良。临床治疗此证,当急则治其标,以平肝熄风,开窍通关为法,待病势平稳,再审因论治。后世医家对本证病机与治疗,发挥颇多,论述精辟,如张锡纯在《医学衷中参西录·医方》中指出:“《内经》调经论曰:‘血之与气,并走于上,则为大厥,厥则暴死,气反则生,气不反则死,’盖血不自升,必随气而上升,上升之极,必致脑中充血,至所谓气反则生,气不反则死者,盖气反而下行,血即随之下行,故其人可生。若其气上行不反,血必随之充而益充,不至血管破裂不止,犹能望其复苏乎?读此节经文,内中风之理明,脑充血之理亦明矣。”张氏根据“大厥”之病理,创制镇肝熄风汤为治疗主方,对某些昏厥,包括脑血管意外等疾患的治疗,至今仍有积极的指导意义。

\biaoti{【原文】}

\begin{yuanwen}
帝曰:經言\sb{1}陽虛則外寒,陰虛則内熱,陽盛則外熱,陰盛則内寒,余已聞之矣,不知其所由然也。岐伯曰:陽受氣于上焦,以溫皮膚分肉之間,今寒氣在外,則上焦不通,上焦不通,則寒氣獨留於外,故寒慄\sb{2}。帝曰:陰虛生内熱奈何?岐伯曰:有所勞倦,形氣衰少,榖氣不盛\sb{3},上焦不行,下脘不通\sb{4}。胃氣熱\sb{5},熱氣熏胸中,故內熱。帝曰:陽盛生外熱奈何?岐伯曰:上焦不通利,則皮膚緻密,腠理閉塞,玄府\sb{6}不通,衛氣不得泄越,故外熱。帝曰:陰盛生內寒奈何?岐伯曰:厥氣上逆\sb{7},寒氣積於胸中而不寫,不寫則溫氣去\sb{8},寒獨留,則血凝泣,凝則脈不通,其脈盛大以澀\sb{9},故中寒\sb{10}。
\end{yuanwen}

\biaoti{【校注】}

\begin{jiaozhu}
  \item 经言:指古代医经所论。
  \item 寒气在外,……故寒慄:指外感寒邪初期,恶寒症状产生的机理。张介宾注:“寒气在外,阻遏阳道,故上焦不通,卫气不温于表,而寒气独留,乃为寒慄”。
  \item 谷气不盛:谷气,水谷精气。谓脾胃运化无力,水谷精气不足。
  \item 上焦不行,下脘不通:劳倦伤脾,脾气不足,不能转输,致清气不能开,浊气不能降。高世栻注:“上焦不能宣五谷味,故上焦不行,下脘不能化谷之精,故下脘不通。”
  \item 胃气热:胃为水谷气血之海,清浊升降失常,滞于中焦而生热。张志聪注:“胃为阳热之府,气留而不行,则热气熏于胸中,而为内热矣。”
  \item 玄府:即汗孔。
  \item 厥气上逆:下焦阴寒之气逆行于上。
  \item 温气去:温气,即阳气。去,消散。言阳气被寒邪所迫而散失。
  \item 脉盛大以涩:寒邪积留胸中,脉象紧而有力,故为实大;气血运行不利,故脉见涩象。
  \item 中寒:胸中寒盛,故称中寒。
\end{jiaozhu}

\biaoti{【理论阐释】}

\xiaobt{阴阳虚实致内外寒热的机理}

文中讨论了阳虚则外寒,阴虚则内热,阳盛则外热,阴盛则内寒的机理,其论与后世所说的阳虚生外寒,阴虚生内热,阳盛则热,阴盛则寒,在概念和病机方面都不完全一致。

(1)阳虚则外寒

本文所论“阳虚则外寒”,是由寒邪侵犯人体,阻遏卫气,令卫气不能宣达肌表,表卫不足,致使寒邪独留于体表而产生外寒。此“寒”并非虚寒,实为外感寒邪早期出现的恶寒症状。因此,“阳虚则外寒”是指外感表证中恶寒症状产生的机理。所谓“阳虚”仅指肌表卫阳受损或卫阳为寒邪遏阻不能宣达,肌表失于阳气温煦而言。治宜辛温发散,透解表邪。表证除则恶寒止。而后世临床常说的“阳虚生外寒”,是指阳气不足,不能温煦肌腠,而出现的畏寒。由于阳气虚衰,机能减退,产热量不足,故产生畏寒肢冷等虚寒证候。治宜温补阳气,阳气足则畏寒除。

(2)阴虚生内热

本文所论“阴虚生内热”,是因劳倦太过,损伤脾气,使脾的运化升清功能失调,清阳不升,则浊阴不降,谷气留而不行,郁而化热,熏蒸于胸中,产生内热。此种内热,实际是脾气虚发热,脾居于内属阴,故称脾气虚为阴虚。李东垣所说的“气虚发热”即指此言,倡用升阳益气、甘湿除热治气虚发热,是这一理论的发展。而后世临床常说的“阴虚发热”,是指阴气不足,濡养滋润功能减退,阴不制阳,阳气相对亢奋,而出现的热证。多见于肺胃或肝肾之阴不足,虚火内生之午后潮热,颧红盗汗,口燥咽干,舌红少津,脉象细数等。治宜滋阴增液为主,清热降火为辅,以冀壮水之主以制阳光。

(3)阳盛生外热

本文所论“阳盛生外热”,是指寒邪外束肌表,致上焦不通,腠理闭塞,卫气郁遏肌腠而致的表证发热。这种发热实为寒邪侵犯肌表,在恶寒之后出现的发热,即是前述外感病恶寒症之进一步发展,宜辛温发散治之。如《素问·阴阳应象大论》云:“体若燔炭,汗出而散。”而后世临床常说的“阳盛则热”包括里热证和表热证的发热,主要由阳邪亢盛,机能亢奋,邪正相搏激烈所致。表现或表热,或里热,或表里俱热,治疗以清热为主,在表者解表,在里者清里。

(4)阴盛则内寒

本文所论“阴盛则内寒”,是因寒气积于胸中,致使血脉凝涩不畅,耗损胸中阳气,从而产生内寒。此内寒虽属于阳虚阴寒之邪过盛所致,但其寒气局限于胸中,如“胸痹”、“心痛”等多属此证。可用瓜蒌薤白桂枝汤之属治之。后世临床常说的“阴盛则寒”,泛指一切脏腑之寒证,治宜温中散寒。

\biaoti{【临证指要】}

\xiaobt{甘温除热的临床意义}

李东垣以《内经》“有所劳倦,形气衰少,谷气不盛,上焦不行,下脘不通,胃气热,热气熏胸中,故内热”作为甘温除热的立论依据,指出脾胃气虚而生大热者应从三个方面来认识:一是脾胃气虚导致清气不升,浊阴不降,清浊相干,谷气不行,混于胃中,郁而生热,这是“胃病而生大热”的缘由;二是由于脾胃气虚,元气不足导致了阴火上冲,阴火伤其生发之气,日渐煎熬,导致了血虚,血虚则气无所附,引起虚阳亢奋而外浮发热;三是脾胃之气下流,谷气不升,无阳以护其荣卫,荣卫不和则不任风寒,乃生寒热。除第三种情况有外邪挟杂所谓不任风寒而有表证外,其余二种情况均属里证,而此三者又均与元气不足有关。对气虚发热证的治疗,东垣认为宜遵《内经》‘劳者温之,损者温(益)之’之旨,唯以辛甘温之剂,补其中而升其阳,甘寒以泻其火则愈,故创立补中益气汤,开甘温除大热之先河。后世医家宗其义而用之,多获良效。兹举例证之:陈某,女,26岁。1979年5月初诊:近半年多来,右胁部疼痛,经常发热在37$\sim$38.5℃之间,多发于下午及夜间,伴食欲不振、心悸、四肢无力、睡眠欠佳、大便稀而粗糙。舌质红,苔薄黄,脉沉细。证属湿热困脾,中气不足,师甘温除热法,以补中益气汤加减,服十五剂,热退,余症大减,拟上方加减为蜜丸,以资巩固。(赵金铎《医话医论荟要·徐振盛医话》)

\section{素問·至真要大論(節選)}%第五節

\biaoti{【原文】}

\begin{yuanwen}
帝曰:善。夫百病之生也,皆生於風寒暑濕燥火,以之化之變\sb{1}也。經言盛者寫之,虛者補之、余鍚以方士\sb{2},而方士用之尚未能十全,余欲令要道\sb{3}必行,桴鼓相應\sb{4},猶拔刺雪汙\sb{5},工巧神聖\sb{6},可得聞乎?岐伯曰:審察病機\sb{7},無失氣宜\sb{8},此之謂也。

帝曰:願聞病機何如?岐伯曰:諸風掉眩\sb{9},皆屬於肝。諸寒收引\sb{10},皆屬於腎。諸氣膹鬱\sb{11},皆屬於肺。諸濕腫滿,皆屬於脾。諸熱瞀瘛\sb{12},皆屬於火\sb{13}。諸痛癢瘡,皆屬於心\sb{14}。諸厥固泄\sb{15},皆屬於下\sb{16}。諸痿喘嘔,皆屬於上\sb{17}。諸禁鼓慄\sb{18},如喪神守\sb{19},皆屬於火。諸痙項強\sb{20},皆屬於濕。諸逆衝上\sb{21},皆屬於火。諸脹腹大\sb{22},皆屬於熱。諸躁狂越\sb{23},皆屬於火。諸暴強直\sb{24},皆屬於風。諸病有聲,鼓之如鼓\sb{25},皆屬於熱。諸病胕腫\sb{26},疼酸驚駭,皆屬於火。諸轉反戾\sb{27},水液\sb{28}渾濁,皆屬於熱。諸病水液,澄澈清冷\sb{29},皆屬於寒。諸嘔吐酸,暴注下迫\sb{30},皆屬於熱。取《大要》\sb{31}曰:謹守病機,各司其屬,有者求之,無者求之\sb{32},盛者責之,虛者責之\sb{33},必先五勝\sb{34},疎其血氣,令其調達,而致和平。此之謂也。
\end{yuanwen}

\biaoti{【校注】}

\begin{jiaozhu}
  \item 之化之变:化,指化生,气化,是由六气推移产生的正常作用。变,即变动,是六气互为胜负而产生的变动。本句意指风寒暑湿燥火的化生和变异。王冰注:“风寒暑湿燥火,天之六气也,静而顺者为化,动而变者为变,故曰之化之变。”
  \item 锡以方士:锡,通“赐”。传授之意,方士,即医工,医生。意即得虚实补泻之治法,传授给医生们。
  \item 要道:要,切要,重要。道,理也,术也。要道,指医学中重要的理论与技术。
  \item 桴鼓相应:桴,即鼓槌。喻用桴打鼓而声应,形容疗效显著。
  \item 拔刺雪汙:汙,同“污”。雪,洗也。即洗雪污垢。《灵枢·九针十二原》“刺虽久犹可拨也,污虽久犹可雪也。”
  \item 工巧神圣:指医生高明的诊断技术。《难经·六十一难》:“望而知之谓之神,闻而知之谓之圣,问而知之谓之工,切脉而知之谓之巧。”
  \item 病机:机,机要、关键。病机,指疾病发生发展变化的关键所在。张介宾注:“机者,要也,变也,病变所由出也。”
  \item 无失气宜:气宜,六气主时之所宜。无失气宜,即审察病机要从六气主时出发,治疗也不要违背这一规律。
  \item 掉眩;掉,摇也,指肢体不自主地摇摆或震颤。眩,眩晕、指视物动幻不定。吴昆注“掉,摇也。眩,昏乱旋运而目前玄也。乃风木动摇蔽翳之象。”
  \item 收引:收,敛也。引,急也。谓筋脉敛缩牵引拘急。
  \item 膹郁:膹,气逆喘急;郁,痞闷。膹郁,为胸闷喘息之证。王冰注:“膹谓膹满,郁谓奔迫。”
  \item 瞀瘛:瞀(mào音冒),昏闷不清。瘛(chì音斥)抽掣,手足时伸时缩。
  \item 皆属于火:《黄帝素问直解》火作“心”,可参。
  \item 皆属于心:《黄帝素问直解》心作“火”,高世枕注:“火,旧本作心,今改。诸痛痒疮,皆属手少阳三焦之火。”可参。
  \item 厥固泄:厥,在病证多指昏厥和肢厥,《素问·厥论》载有暴不知人之昏厥和阳气衰于下的寒厥、阴气衰于下的热厥。固,二便不通。泄,二便泻利不禁。
  \item 下:指人身下部的脏腑,如肾、肝、膀胱、大小肠等。
  \item 上:指人身上部的脏腑,如肺、心、胃上口等。
  \item 禁鼓慄:禁,通“噤”,指牙关紧闭。鼓,击也,此指上下牙齿叩击。慄,战慄,即身体战抖。
  \item 如丧神守:形容鼓颔战慄而不能自控。吴昆注:“神能御形,谓之神守,禁鼓慄则神不能御形,如丧其神守矣。”
  \item 痉项强:痉,病名,症见牙关紧急、项背强急、角弓反张。项强,项部强硬不舒,动转困难。项强可为独立证候,也可为痉病的症状。
  \item 逆冲上:指气机急促上逆而致的病证,如急性呕吐、呃逆等。
  \item 胀腹大:指腹部胀满膨隆。
  \item 躁狂越:躁,躁扰不宁。狂,语言行为错乱。越,举止越常。
  \item 暴强直:暴,猝然;强直,四肢伸而不屈,身躯仰而不俯。暴强直,指突然发作的全身筋脉挛急,伸而不屈。
  \item 鼓之如鼓:谓叩击腹部如鼓之有声。
  \item 胕肿:胕肿二字在《内经》中屡见,但其含义不尽相同。如胕通“腐”,胕肿,即皮肉肿胀溃烂。胕又通“跗”,跗即足背,胕肿,即为足胫肿。胕还通“浮”,胕肿,指浮肿。本条“胕肿”与火邪有关,释为腐肿较胜。
  \item 转反戾:转,扭转。反,背反张。戾,身屈曲。转反戾,指腰身扭转、背反张、身屈曲等各种病态。
  \item 水液:主要指尿液,亦可包括涕、唾、涎、痰、白带等分泌物。
  \item 澄澈清冷:形容水液清稀而寒冷。
  \item 暴注下迫:暴注,突然剧烈的泄泻。下迫,指欲便不能,肛门窘迫疼痛,即里急后重。程士德《素问注释汇粹》:“火性急速,故暴注泄泻,火郁于内,故里急后重。”
  \item 《大要》:古医经名。
  \item 有者求之,无者求之:求,探求,辨识。有者求之,指有外邪者,当求其外感何邪。无者求之,指无外邪者,当求其内伤何因。张介宾注:“有者言其实,无者言其虚,求之者,求有无之本也。”此“本”,当指病因而言,与下文成连接关系。
  \item 盛者责之,虚者责之:责,追究,分析之意,谓分析病证虚实的机理。
  \item 必先五胜,五,五运五行之气。胜,更胜。即必须首先掌握天之五气与人之五脏间的五行更胜关系。
\end{jiaozhu}

\biaoti{【理论阐释】}

1.对十九病机的理解

病机十九条原文的理解:任应秋《病机临证分析·前言》指出:“对病机十九条‘诸’、‘皆’、‘属’三字,要活泼地理解它,不能解得太死煞了。诸,众也,仅表示不定的多数,不能释为‘凡’字。凡者,为统计及总指一切之词,以此释之,未免释之太泛。皆者,乃‘同’字之义,与‘诸’字正成相对之词儿。属,近也,犹言‘有关’,不必释为‘隶属’之属。‘诸风掉眩,皆属于肝’,即是说有许多振掉和眩晕的风病,同样是有关于肝。第必须辨其为肝虚、肝实、肝寒、肝热而治之。假使简单地解释为:一切振掉、眩晕的风病,都是肝病,这便毫无辨证的余地,徒见其以词害意而已。”任氏此论符合经旨,对“诸”、“皆”二字的理解,已得到多数医家的认同,但在“属”字的理解上,仍当反映出隶属关系较妥,即“属”前的病证应隶属于“属”后的病机,若仅释为“有关”,还是缺乏肯定性和确切性。仍以“诸风掉眩,皆属于肝”为例,此句可释为:多种风病而见振掉、眩晕之类证候者,大多属于肝的病变。

关于五脏病机中的“风”、“寒”、“湿”、“气”、“热”的理解,有两种意见:一是将其理解为病因或病邪。二是将其理解为病证。因为“病机十九条的句式结构是统一的,有章可循的,即‘诸’字后叙述病证及其证候特点,‘皆属于’之后则归纳出前述病证的病位或病邪。照此体例,论五脏病机五句中的‘风’、‘寒’、‘湿’、‘气'、‘热’,就不应如有些医家那样看作病因或病邪,而应一律视为病证名称”。依据“病机十九条”文例,前言病证,后言病机,故以后说更为贴切。(王洪图主编《黄帝内经研究大成·理论研究》)

原文“诸热瞀瘛,皆属于火。诸痛痒疮,皆属于心”的校释:清·高世栻认为应将“火”与“心”二字互易其位,校为“诸热瞀瘛,皆属于心。诸痛痒疮,皆属于火”,如此处理,文理、医理皆通。又有成肇智在《湖北中医杂志》1986年第6期撰文指出:“热病而出现瞀(神识昏糊)、瘛(筋脉抽搐)等严重证候,即后世所谓‘高热痉厥’,多从心或心包论治。这既同临床相合,又同‘诸寒收引,皆属于肾’一条相对,而且热·病与心脏俱属于火,亦与其它几脏病机多属五行同类相仿一致。”同时他认为‘诸痛痒疮,皆属于心’的‘心’是否为‘火’字之误,却值得商榷。‘诸痛痒疮’是谓多种疼痛而兼疮疡的病证。这些病证主要因邪气阻滞经络,气血不通所致,《素问·举痛论》、《灵枢·痈疽》等篇皆有详细的论述。由于心主血脉而通神明,因此这些病证属于心是顺理成章的。王冰亦注此说:‘心寂则痛微,心躁则痛甚,百端之起,皆自心生,痛痒疮疡生于心也’。另一方面,引起疮疡疼痛者非止火邪一种,例如《素问·至真要大论》说:‘阳明司天,燥淫所胜,民病……疡疮痤痈’,‘太阳司天,寒淫所胜,……发为痈疡,民病厥心痛,……病本于心’等。同时,本条位于论述‘五脏病机’的五条和上下病机的两条之间,是本条非论邪气‘火’而论病位‘心’的又一佐证。至于‘五脏病机’条文中,何以属于心者独占两条,这从心为‘五脏六腑之大主’和心包代心受邪而二者分司君相二火的《内经》学术观点中不难得到解释。”

2.掌握病机的重要性及方法

“审察病机,无失气宜”是本节的提纲。文中指出,一般医生虽然懂得“百病”多由于六气的异常变化所致,也知道补虚泻实的治则,但治病仍“未能十全”,其原因就是没有掌握好病机。医生治病,必须“审察病机”,即着重把握疾病变化的关键所在,同时还要“必先五胜”、“无失气宜”,结合气候变化考虑立法制方,才能获得满意的疗效。怎样才能掌握好病机呢?“谨守病机,各司其属”,是对掌握病机方法的高度概括。归纳原文,兹从探求病因、辨明病性、整体定位三个方面分述掌握病机的具体方法。

①探求病因

病机十九条在分析疾病表现证候时,非常重视因、机、证三者的有机联系,本篇“必伏其所主,而先其所因”,强调欲治病求本,必先探求致病之因。病因有外感、内伤之分,原文“有者求之”,指有外邪者,当求其外感何邪;“无者求之”,指无外邪者,当辨明内伤何因。求因必以审证为前提,即根据疾病的症状表现,探求致病原因,如“诸躁狂越,皆属于火”,从躁动不宁,狂言骂詈,殴人毁物,逾垣上屋等证候表现分析,得出火邪为患之病因。

②辨明病性

无论内伤、或外感发病,均可发生寒热虚实的病理变化,故在探求病因的基础上,辨别病证的属性实为重要。病证寒热虚实变化是病机的关键所在。“盛者责之”,意谓对于实证,要辨明何邪盛及其邪实之机;“虚者责之”,则是对于虚证,要辨明何气虚及其正虚之理。又据阴静而阳躁,及水体清,火体浊等特性,得出水液澄澈清冷属寒;水液混浊属热的结论。

③整体定位

本篇讨论病机所属之病证较为广泛,因此,必须从复杂的病证中,通过“审证求因”,找出其与五脏和六气的所属关系。此外,还要“必先五胜”,“无失气宜”,根据五行更胜规律辨明五运六气的司值胜复和五脏六腑的盛衰乘侮,从人与自然及人体脏腑整体性上作出全面的分析判断。

此外,文中还提出“谨守病机,各司其属”,揭示以病机为纲分析归类临床症状,要具体情况具体分析,不可泥守一端。如相同或相似的症状,可有不同的病机,十九条中,“掉眩”、“收引”、“暴强直”、“痉项强”、“转反戾”、“瞀瘛”均为筋脉拘挛、强急、抽搐之症,其病因病机就各有肝、肾、风、湿、热、火的所属;反之,不同的症状,其病因病机则可基本相同,如“瞀瘛”、“禁鼓慄”、“躁狂越”、“胕肿,疼酸惊骇”、“逆冲上”等证,均由火邪所致。这些范例为临床辨证论治既奠定了理论也提供了方法。

\biaoti{【临证指要】}

\subsection*{一、脏腑病机及其临床意义}

1.诸风掉眩,皆属于肝

多种肢体动摇不定和头目眩晕的风证,大都属于肝的病变。肝藏血,主身之筋膜,开窍于目,若肝有病变,木失滋荣,伤及所合之筋,所主之窍,则见肢体摇摆震巅,目眩头晕。兹举例证之:贾某,男,45岁。头晕胀痛,游移而无定所,左肩升抬转侧感痛,甚或臂手指节麻木,舌微燥红,脉息虚弦。风气通于肝,肝生筋,肝血不足,洒陈式微,是以麻木,肝风肝阳随络上激元神之府,是以头晕胀痛游移不定。拟养营调络,柔肝熄风法。防风5克、天麻9克、枸杞子9克、谷精6克、桑枝9克、桑叶络6克、钩藤9克、秦艽9克、当归须9克、阿胶9克(烊冲)、丝瓜络6克、炙甘草3克,上方服6贴,头晕停,诸证已。(董建华《中国现代名中医医案精华(一)·吴考盘医案》)

2.诸寒收引,皆属于肾

多种身体蜷缩,四肢拘急不舒,关节屈伸不利的寒性病证,大都属于肾的病变。肾藏精,为水火之宅。若肾阳虚衰、温煦无力,气血运行不畅,筋脉失养故筋脉拘挛,关节屈伸不利。举例肾虚“大偻”治验:杨某,男,26岁。腰背佝偻疼痛九年,因惊恐复受寒后致腰部不适而疼痛,痛渐趋加重而腰渐弯曲,至成伛偻之状,诊见腰脊疼痛佝倭,行动困难,伴两胁胀痛不舒,形体瘦削,苔薄白,脉弦细。治宜补肾气,温经脉,祛寒邪,方用独活寄生汤去防风、桂心、茯苓,加狗脊、路路通、乌蛸蛇,6付,辅以气功疗法。二诊腰胁疼痛明显减轻,睡卧腰脊稍能伸展,效不更方,续治12日病愈,该案系惊恐复感寒凉,经络受邪,营卫不从,内伤于肾,羁延进展,而成痼疾。补益肾气治其内,温经散寒治其外,标本兼顾,九年痼疾尽除。(王洪图《黄帝医术临证切要·临证发挥篇》)

3.诸气膹郁,皆属于肺

多种呼吸喘促,胸部胀闷之类的气病,大都属于肺的病变。肺主气,司呼吸,故气之为病,首责于肺。肺病宣降失常,气壅郁于胸或上逆,则见呼吸喘息,胸中窒闷,痞塞不通。案例,杨某,男,54岁。咳嗽喘息,痰稠难咯色白,入夜尤甚,发作时大小便次数增多或有失禁,伴有畏寒感,舌淡苔白,脉沉细滑。治以温化,用小青龙汤加减:炙麻黄6克、姜半夏10克、茯苓12克、桂枝6克、杏仁10克、白术10克、五昧10克、橘红12克、橘络6克、补骨脂10克、甘草5克,3剂,药后咳喘渐平,疲转清易咯,守前方加减再进16剂,喘平寐安,诸证皆除。(董建华《中国现代名中医医案精华(一)李振华医案》)盖喘咳初起,病多在肺,母病及子,肾气亦损,气逆更甚。以小青龙汤温肺平喘为主,辅以补骨脂益其元而助纳气,药证相合,效如桴鼓。

4.诸湿肿满,皆属于脾

多种浮肿和脘腹胀满之类的湿病,大都属于脾的病变。脾主运化水湿,主四肢,应大腹,若脾失健运,水津失布,内聚中焦或泛溢肌肤,则见脘腹胀满,四肢浮肿。程士德《内经理论体系纲要·病机学说》谓临床水肿之因于陴者,多由湿邪困脾,脾气不运,水湿内聚,三焦决渎失司,膀胱气化失常,水湿渗溢肌肤而成水肿之证。如再水聚于腹,则浮肿与腹满可并见,此属实证,后世归为“阳水”类。如果中阳不足,脾阳不运,水湿不化,以致水邪泛滥而成水肿,其特点是肿以下半身为甚,按之凹陷不起。伴见脾气虚的脾失健运,面色萎黄,神倦纳呆,便溏,小便不利等证,此属虚证,后世归为“阴水”类。由于脾病而见腹满的,可见于脾气虚,运化失职,腹满多见于食后,且“腹满时减复如故”。如饮食不节,食滞内停,亦见脘腹胀满,其特点是厌食,嗳腐酸臭,脘腹胀满不减的,辨证则属于实证。此外,脾胃湿郁,郁而化热,熏蒸肝胆,以致气机升降失常,水液代谢障碍,水停腹脘而成腹胀,此病又在肝肺两脏。

5.诸痛痒疮,皆属于心

多种疮疡及其痛痒之类的热证,大都属于心的病变。疮疡,包括痈、疽、疖、疔、丹毒等,痛和痒是其主要症状。心为阳脏主火,主身之血脉,若心阳偏亢,火热炽盛,气血运行不畅,壅滞于肌肤局部可形成疮疡肿痛之疾。如《素问·生气通天论》云:“营气不从,逆于肉理,发为痈肿”。例:薛立斋治一男子,患痈肿硬疼痛,发热烦躁,饮冷,脉沉实,大便秘,乃邪在脏也。用内疏黄连汤疏通之,以绝其源。先投1剂,候行一次,势退一二,再进1剂,诸症悉退,乃用黄连消毒散四剂而消。(《续名医类案·痈疸》)立斋治痈所用二方均以黄连为主,意在清心以泻火,绝其痈毒之源,故取效颇捷。

6.诸厥固泄,皆属于下

多种厥病及二便不通或二便泻利不禁诸证,大都属下部之病变。《素何·厥论》云:“阳气衰于下则为寒厥,阴气衰于下则为热厥”,《灵枢·本神》又云:“肾气虚则厥,均与肾相关,是厥病发于下部之例。固为二便不通,泄为二便泄利失禁。由于肾、膀胱、大肠皆在人身下部,当肾失司二阴之职及肾与膀胱气化、大肠传导功能失调,则可见二便不通、二便泻利不禁等证候。

7.诸痿喘呕,皆属于上

多种痿及喘、呕诸证,大都为上部的病变。肺主宣降,为华盖之脏,如《素问·痿论》说:“肺者,脏之长也,为心之盖也”,胃主降浊,以降为和。肺、胃位于中、上二焦,若肺气热“五脏因肺热叶焦,发为痿躄”(《素问·痿论》),肺失清肃,肺气上逆则喘,胃失和降,胃气上逆则呕。

\subsection*{二、六淫病机及其临床意义}

(一)属火、热类

1.诸热瞀瘛,皆属于火

多种神识昏闷,肢体抽掣之类的热证,大都属于(心)火的病变。(此条属心,是校勘后的文字。)心藏神,主身之血脉,若火热扰心,伤及神明,上逆头目,灼伤阴血,筋脉失养,可见神识昏闷,肢体抽掣。

2.诸禁鼓慄,如丧神守,皆属于火

口噤、鼓颔、战慄,不能自控的,大都为火邪所致。火热郁闭,不得外达,阳盛格阴,火极似水,上扰神明,神失主持,故见口噤、鼓颔、战慄不能自控。

3.诸逆冲上,皆属于火

呕、哕等气逆上冲诸证,大都为火邪所致。火性炎上,易扰气机,引起脏腑气机逆乱,向上冲逆,如肺气上逆,胃气上逆则可产生如急性呕吐,喘促气急等病证。

4.诸躁狂越,皆属于火

躁狂失常诸证,大都为火邪所致。火属阳,性主动,张介宾《类经·疾病类》注“热盛于外,则肢体躁扰;热盛于内,则神志躁烦。”故火热伤人,扰乱神明,多见烦躁不宁,狂言骂詈,殴人毁物,逾垣上屋等症。

5.诸病胕肿,疼酸惊骇,皆属于火

皮肤肿胀疡溃,疼痛酸楚及惊骇不宁诸证,大都为火邪所致。胕肿,此当为腐肿,火热壅滞皮肉筋脉,致血瘀肉腐,令患处红肿溃烂,疼痛或酸楚;内迫脏腑,扰神则惊骇不宁。

6.诸胀腹大,皆属于热

腹部胀大诸证,大都为热邪所致。外感邪热传里,壅结胃肠,致气机升降失常,热结腑实,可见腹部胀满膨隆,疼痛拒按,大便难下。

7.诸病有声,鼓之如鼓,皆属于热

腹中肠鸣有声,腹胀如鼓诸症,大都为热邪所致。积热壅滞胃肠,致气机不利,传化迟滞,故症见肠鸣有声,腹胀中空如鼓。本篇又载:“热淫所胜,……民病腹中常鸣,气上冲胸,……少腹中痛,腹大。”其意相近。

8.诸转反戾,水液浑浊,皆属于热

转筋拘挛、背反张、身曲不能直,以及小便浑浊诸证,大都为热邪所致。热伤津血,筋脉失养出现筋脉拘挛,或弛张失度而扭转,甚至背反张,身躯曲而不直等症。热盛煎熬津液,令尿液等排泄物黄赤浑浊。

9.诸呕吐酸,暴注下迫,皆属于热

呕吐吞酸,暴泻及里急后重诸证,大都为热邪所致。邪热犯胃,或食积化热,胃失和降而逆上,见呕吐酸腐。热走肠间,传化失常,则暴泻如注,势如喷射,或肛门灼热窘迫,欲便而不能便。

以上九条,较为广泛地论述火热邪气的致病机理及常见病证,其中论火五条,论热四条,概括原文所述,火热致病特点可归纳为以下六方面:①火性炎热燔灼,内扰心神。如见瞀瘛,禁鼓慄,如丧神守,狂越等。②火性炎上,致气机上冲或逆乱。如见逆冲上,呕,腹胀大,腹如鼓等。③火热消灼阴津。如水液浑浊,吐酸等。④火热燔灼筋脉,引动内风。如见转反戾等。⑤火热灼伤血肉,郁生疮疡。如见胕肿疼酸等。⑧火性急迫,为病多急暴,如呕逆、暴注下迫等。火之与热,皆为阳邪,致病特点相似,因其所犯脏腑经脉部位不同,症状同中有异,医家称之为同机异症。结合临床实践,外感火热之邪,或六淫郁滞从阳化热化火,或五志过极、饮食积滞而化热化火,均可以产生火热病证,临床表现甚多,是为十九病机中详论火热病机的道理所在。

(二)属风、寒、湿类

1.诸暴强直,皆属于风

突然发作的筋脉强直、角弓反张诸证,大都为风邪所致。风性主动,善行数变,风气通于肝,风邪内袭,伤肝及筋,多见颈项,躯干、四肢关节突然出现拘急抽搐、强直不柔之症。叶庆莲曾治一女性患者,20岁。暴风骤至,大雨欲来,大声呼叫其弟回家,突然口不能闭而歪斜,时有抽动,伴舌强语謇。暴雨稍停即到医院求治。诊为风中经络,用牵正散加川芎8克、路路通12克、地龙10克、白芍15克、甘草6克,3剂,日一付,水煎分3次服。又针刺地仓、合谷、夹车、肝俞穴。二诊,口型基本正常,语言流利,守方再服2剂以资巩而。

2.诸病水液,澄澈清冷,皆属于寒

机体分泌物、排泄物呈清稀寒冷诸证,如痰涎清稀、小便清长、大便稀薄等,多由寒邪传里,阴盛伤阳,阳虚气化无权,水津失于温化而成。说明寒证的特点,不但形体寒慄,且体内分泌物、排泄物亦具有寒凉清冷的特性。证之临床,不论外寒内寒,都可出现这些特征,如胃寒或脾气虚寒见呕吐清水涎沫;寒邪犯肺,肺气失宣见涕液、痰液清稀淡薄;肾阳不足,膀胱虚寒见小便清长而频数,大肠虚寒见大便清稀不腥臭。反之,热证多表现出分泌物、排泄物黄浊稠厚腐臭等特征,故临床以分泌物、排泄物是否清冷来判断证候的寒热属性。

3.诸痉项强,皆属于湿

发痉,项强诸证,大都为湿邪所致。《素问·生气通天论》云:“阳气者,精则养神,柔则养筋”。而湿为阴邪,其性粘滞,最易阻遏阳气,筋脉失却温养,可见筋脉拘急,身体强直,角弓反张,或项强不舒,屈颈困难等症。《内经》论痉,提到风、寒、热、燥多方面,而仅本条将痉归属于湿,注家认识颇不一致。徐忠可《金匮要略论注》注:“诸痉项强,皆属于湿……然则疼证之湿,从何来乎?不知痉之根源,由亡血阴虚,其筋易强,而痉之湿,乃即汗之余气,搏寒为病也。故产后血虚多汗则致之;太阳病汗太多则致之;风病原有汗,下之而并耗其内液则致之;疮家发汗则致之;此仲景明知有湿而不专论湿,谓风寒去而湿自行耳”。徐氏之论,着眼于汗,指出汗之余气即为湿,搏寒则为痉。此注深得经旨,亦符合临床实际。(傅贞亮等《黄帝内经素问析义》)

\zuozhe{(叶庆莲)}

\section{靈樞·百病始生(節選)}%第六节

\biaoti{【原文】}

\begin{yuanwen}
黃帝問于岐伯曰:夫百病之始生也,皆生於風雨寒暑,清濕\sb{1}喜怒。喜怒不節則傷藏,風雨則傷上,清濕則傷下。三部之氣\sb{2},所傷異類,願聞其会\sb{3}。岐伯曰:三部之氣各不同,或起于陰,或起于陽,請言其方。喜怒不節則傷藏,藏傷則病起于陰也;清濕襲虛\sb{4},則病起於下;風爾襲虛,則病起於上,是謂三部。至於其淫泆\sb{5},不可勝數。
\end{yuanwen}

\biaoti{【校注】}

\begin{jiaozhu}
  \item 清湿:清,寒也。清湿,即寒湿之邪。
  \item 三部之气:指伤于上部的风雨之邪,伤于下部的寒湿之气,以及伤于五脏的暴喜暴怒之气。
  \item 会:领会。领会其中的道理。
  \item 袭虚:袭,侵袭;虚,乘人体之虚。言外邪乘人之虚而侵袭机体。
  \item 淫泆:淫,浸淫;泆,溢也,含蔓延扩散之义。淫泆,言病邪逐步浸淫、传变扩散。
\end{jiaozhu}

\biaoti{【理论阐释】}

\xiaobt{病因与发病部位的关系}

根据不同的病因和不同的病位,将发病分为三部:风雨主要病发于上部,寒湿起病多在下部,而情志不节则多见内伤脏气。说明发病初期起病部位与病邪作用途径相应,这种认识有助于临床的病因辨证。至于病邪入里,邪气蔓延淫泆之时,则其传变多样,病变部位各异,要具体情况具体分析。这种“三部之气”的分类法为后世医家认识病因奠定了基础。汉代张仲景按病因的传变概括为三条途径,他说:“千般疢难,不越三条:一者,经络受邪入脏腑,为内所因也;二者,四肢九窍,血脉相传,壅塞不通,为外皮肤所中也;三者,房室、金刃、虫兽所伤。以此详之,病由都尽。”宋代陈言则更明确地提出“三因学说”,他说:“六淫,天之常气,冒之则先自经络流入,内合于腋腑,为外所因;七情,人之常性,动之则先自脏腑郁发,外形于肢体,为内所因;其如饮食饥饱,叫呼伤气,金疮蹐折,疰忤附着,畏压溺等,有悖常理,为不内外因”。

\biaoti{【临证指要】}

“喜怒不节则伤脏,风雨则伤上,清湿则伤下”;“三部之气,所伤异类”,说明邪气不同,所伤人体病位亦不同,换言之,人体各部位对病邪的易感性不同。故临床常见风雨之邪伤人大多始于头面部,咽喉部,如风寒头痛,鼽衄,喉痹,咳嗽等;而久居阴冷潮湿之地,或长期水中作业,则腰痠骨痹身重肢疼,多见下肢病证;喜怒不节则或见两胁胀痛,泛酸呕吐,为肝气郁结,横逆犯胃;或见胸痛心烦,心悸气逆,为心气痹阻;或见神志狂乱,谵妄昏瞀,为心神惮散等,均为气机逆乱,直接影响五脏功能所致。

\biaoti{【原文】}

\begin{yuanwen}
黃帝曰:余固不能數,故問先師\sb{1},願卒\sb{2}聞其道。岐伯曰:風雨寒熟,不得虛,邪不能獨傷人。卒然\sb{3}逢疾風暴雨而不病者,蓋無虛,故邪不能獨傷人。此必因虛邪\sb{4}之風,舆其身形,兩虛相得,乃客其形\sb{5}。兩實相逢,衆人肉堅\sb{6}。其中於虛邪也,因于天時,與其身形,參以虛實,大病乃成\sb{7}。氣有定舍,因處爲名\sb{8},上下中外,分爲三員\sb{9}。
\end{yuanwen}

\biaoti{【校注】}

\begin{jiaozhu}
  \item 先师:即黄帝对岐伯的尊称。《太素》作“天师”。杨上善注:“天师,尊之号也”。
  \item 卒:尽也。详尽之义。
  \item 卒然:卒,同猝,突然之义。
  \item 虚邪:可以使人致病的四时不正之气。因乘人之正气虚而侵入,故称虚邪。王冰注:“邪乘虚入,是谓虚邪。”
  \item 两虚相得,乃客其形:两虚,虚邪和正气虚弱。相得,相合。客,侵犯。言邪气与正气虚弱两种情况相合,虚邪就会侵犯人体致病。
  \item 两实相逢,众人肉坚:两实,言六气正常和正气充实;肉坚,指肌肤固密不易受邪发病。
  \item 参以虚实,大病乃成:参,参合。虚,正气虚。实,邪气盛实。正气虚与邪气实两种情况相参合,外感病证即形成。
  \item 气有定舍,因处为名:气,邪气。定舍,停留之处。即根据邪气入侵后停留的部位命名疾病。
  \item 三员:即上文所言三部之气。
\end{jiaozhu}

\biaoti{【理论阐释】}

1.关于虚邪

虚邪为“八正之虚邪气也”,说明是反常的气候条件所产生的使人致病的因素,如暑天寒凉、冬行夏令均为气候“至而未至”,故为反常;再如风起则“发屋、折树木、扬沙石”(《灵枢·岁露》),下雨则“雷殷气交,埃昏黄黑”,“击石飞空,洪水乃从,川流漫衍”(《素问·六元正纪大论》),则是气候“至而太过”的表现,亦属反常气候。所以虚邪袭人,其性暴烈,致病性强,邪正斗争剧烈,实证较多,并有逐步深入之势。当机体的抗御能力降低时,虚邪则可侵入,甚至长驱直入。反之,体质强壮,卫气固表,腠理固密,任其虚邪来势凶猛,仍可不病。

2.外感病的发病机理

可从如下几方面理解:

(1)外邪不遇正虚不发病“风雨寒热”等一般性致病因素,在正气不虚,抗病力强时,不会致病,所谓“邪不能独伤人”。

(2)“两虚相得,乃客其形”发病必须有虚邪贼风侵袭的外部条件,还要有正气亏虚的内部原因,两者相合,外感疾病才会发生。

(3)“两实相逢,众人肉坚”外有正常气候环境,内部正气旺盛,内外环境都不存在发病的条件,自然不会有外感病证发生。

因此“两虚相得,乃客其形”,是本节的关键,充分反映了《内经》外感发病的基本观点。在两虚之中,正虚是起主导作用的,在正气虚的前提下外邪才可能侵袭人体。这种重视内因的发病学观点在《内经》中还见于《素问·评热病论》“邪之所凑,其气必虚”;《素问·刺法论》“正气存内,邪不可干”;《素问·上古天真论》“虚邪贼风,避之有时,恬淡虚无,真气从之,精神内守,病安从来”等原文中,其主要精神就是突出正气在发病过程中的决定作用。这些理论有效地指导着中医学的预防、养生,以及早期治疗等临床实践。

\biaoti{【原文】}

\begin{yuanwen}
是故虛邪之中人也,始於皮膚,皮膚緩\sb{1}則腠理開,開則邪從毛髮入,入則抵深,深則毛髪立,毛髪立則淅然\sb{2},故皮膚痛\sb{3};留而不去,則傳舍於絡脈,在絡之時,痛於肌肉,其痛之時息\sb{4},大經乃代\sb{5};留而不去,傳舍於經,在經之時,洒淅喜驚\sb{6};留而不去,傳舍於輸\sb{7},在輸之時,六經不通四肢,則肢節痛,腰脊乃強;留而不去,傳舍於伏衝之脈\sb{8},在伏衝之時,體重身痛;留而不去,傳舍於腸胃,在腸胃之時,賁響腹脹\sb{9},多寒則腸鳴飧泄,食不化;多熱則溏出麋\sb{10}。留而不去,傳舍於腸胃之外,募原之間\sb{11},留著於脈,稽留而不去,息而成積\sb{12}。或著孫脈,或著絡脈,或著經脈,或著輸脈\sb{13},或著於伏衝之脈,或著於膂筋\sb{14},或著於腸胃之募原,上連於緩筋\sb{15},邪氣淫泆,不可勝論\sb{16}。
\end{yuanwen}

\biaoti{【校注】}

\begin{jiaozhu}
  \item 皮肤缓:缓者,不坚也,此指表虚。
  \item 淅然:形容怕冷的样子。
  \item 皮肤痛:张介宾注:“寒邪伤卫则血气凝滞,故皮肤为痛。”
  \item 时息:息,止也。此指疼痛时作时止。《甲乙经》作“其病时痛时息。”
  \item 大经乃代:大经,指经脉。代,是替代。大经乃代,指原来邪气留存于络脉之处,现在病位已由经脉代替了,也即邪气进一步深入的意思。张介宾注:“络浅于经,故痛于肌肉之间。若肌肉之痛时渐止息,是邪将去络而深,大经代受之矣。”
  \item 洒淅喜惊:洒浙,寒冷不安的样子。喜,《甲乙》、《太素》均作“善”可参。洒淅喜惊,指恶寒怕冷,精神惊恐不安状。
  \item 输:即下文之“输脉”。指背部的足太阳膀胱经之脉。因其上有五脏六腑之输穴,故称输脉。
  \item 伏冲之脉:指冲脉隐行于脊柱内的部分,部位较深,所以叫伏冲之脉。
  \item 贲响腹胀:即肠鸣腹胀。贲,同“奔”。贲响,有气攻冲而鸣响。
  \item 溏出麋:溏,指大便稀薄。麋,同“糜”,指大便糜烂、腐败、恶臭难闻。
  \item 募原之间:募与膜通,募原又称膜原,此指肠胃之间的脂膜。张志聪注:“募原者,肠胃外之膏膜。”
  \item 息而成积:息,生长。积,积块。此言邪气留于肠外募原之间的脉中,气血凝聚而不行,日久生长出积块。杨上善云:“传入肠胃之间,长息成于积病。”
  \item 输脉:足太阳膀胱经脉。
  \item 膂筋:即附于脊膂之筋。
  \item 缓筋:循于腹内之筋,指足阳明之筋。
  \item 邪气淫泆,不可胜论:张介宾注:“邪之所著则留而为病,无处不到,故淫泆不可胜数。”
\end{jiaozhu}

\biaoti{【临证指要】}

1.病邪所伤部位不同,症状表现各异

病邪伤人,部位不同,其症各异。如同一病邪,侵犯到皮毛则为恶寒毛发竖立,皮肤痛;深入至输脉,则肢节痛,腰脊强;深至冲脉,则体重身痛,至肠外募原之间,甚至可患大积大聚。再如疼痛一证,病邪相同,如均为寒邪,寒邪客于厥阴之脉,则表现为胁肋与少腹相引而痛;寒邪客于背俞之脉,则见心与背相引而痛;寒邪客于阴股,则见少腹与阴股相引而痛。所以辨证不仅要辨病因,还需辨病位。

2.邪气留连,日久转化

病邪久留,随人阴阳盛衰的不同情况病性会发生转化。如本段所示:虚邪之入于肠胃,“多寒则肠鸣飧泄,食不化;多热则溏出糜。”此多热、多寒明显是指阴阳偏胜的不同体质。阳盛之质则病从热化,大便稀薄糜烂且秽臭,属于热泻之类;阴盛阳虚体质则病从寒化,表现为完谷不化的泄泻,伴有肠鸣腹胀,属脾肾阳虚,健运乏力。故病邪虽同,但因患者体质不同,其证可迥异,对此医家临证时不可不察。

\biaoti{【原文】}

\begin{yuanwen}
黃帝曰:積\sb{1}之始生,至其已成,奈何?岐伯曰:積之始生,得寒乃生,厥乃成積\sb{2}也。

黃帝曰:其成積奈何?岐伯曰:厥氣生足悗\sb{3},悗生脛寒,脛寒則血脈凝澀,血脈凝澀則寒氣上入於腸胃,入於腸胃則䐜脹,䐜脹則腸外之汁沫迫聚不得散\sb{4},日以成積。卒然多食飲,則腸滿,起居不節,用力過度,則絡脈傷。陽絡\sb{5}傷則血外溢,血外溢則衄血\sb{6};陰絡\sb{5}傷則血內溢,血內溢則後血\sb{7}。腸胃\sb{8}之絡傷,則血溢於腸外,腸外有寒,汁沫與血相搏,則并合凝聚不得散,而積成矣。卒然外中於寒,若內傷於憂怒,則氣上逆,氣上逆則六輸\sb{9}不通,溫氣不行,凝血蕴裏\sb{10}而不散,津液澀滲\sb{11},著\sb{12}而不去,而積皆成矣。
\end{yuanwen}

\biaoti{【校注】}

\begin{jiaozhu}
  \item 积:肿块,积聚之证。
  \item 厥乃成积:厥,厥逆。寒邪厥逆于上,气血郁滞不行,日久渐成积。
  \item 厥气生足悗:厥气,寒邪厥逆向上。悗,同“闷”。足闷,指足部痠困,活动不利。
  \item 肠外之汁沫迫聚不得散:汁沬,津液。迫聚,迫使其凝聚。全句意谓寒邪上逆,迫使肠外的津液凝聚,形成痰湿。
  \item 阳络、阴络:在上在表的络脉为阳络;在下在里的络脉为阴络。
  \item 衄血:指皮肤及五官七窍的出血。如鼻衄、舌衄、肌衄、齿衄等。
  \item 后血:指前后二阴出血。
  \item 胃:《甲乙经》、《太素》作“外”可参。
  \item 六输:六,六经;输,同“俞”,即俞穴。
  \item 凝血蕴里:蕴,积也,积聚。里,《甲乙经》、《太素》作“裹”。凝结之血积聚相裹而不得散解。
  \item 津液涩渗:涩渗,《甲乙经》作“凝涩”。即津液凝涩停聚。
  \item 著:留著。指气滞,血瘀,津液凝滞皆留著于肠外而不去。
\end{jiaozhu}

\biaoti{【理论阐释】}

\xiaobt{积的病因病机}

积的病因主要是寒邪外袭,七情不和,饮食失调,起居不节,用力过度等因素。“积之始生,得寒乃生”,明确指出寒邪是积证的重要原因。寒为阴邪,其性凝滞收引,而人身气血津液皆喜温而恶寒,寒则涩而不能流。寒邪久留而不去,使气滞血凝痰湿停留,日久成积。七情不和亦与积证有关,“卒然外中于寒,内伤于忧怒则气上逆”,可使内脏气机逆乱,营血、津液运行障碍,结聚日久而成积。暴饮暴食、寒温不适等损伤胃肠功能,是肠胃积证的主要病因。

积的病机主要是寒凝、气滞、血瘀、津液凝涩,积聚而不散,日久而成。关于积的论述,在《内经》中还见于《灵枢·水胀》、《灵枢·刺节真邪》等篇中,可互参。

\biaoti{【临证指要】}

1.积的病因理论对预防癌症的积极意义

积为赘生的肿物,现今临床所称之癌症,多属于“积”的范畴。本篇说“积之始生,得寒乃生”,此“寒”可看作广义的,泛指外邪,强调外邪是积证发病的重要因素。从目前临床观之,气候异常变化、紫外线的过度照射、日光的曝晒、空气的污染均可致癌;但《内经》中又指出七情过激,饮食失调,起居不节,用力过度等因素均可导致积证发生,这些致病因素都是从生活方式方面提出的,需要引起足够的重视。据医学研究报道,直接与遗传、职业有关的癌是存在的,但为数不多,而与人类生活方式或是生活中的行为有关的癌,却占到了80\%。据此,医学界提出了“生活方式癌”的概念。这一认识与《内经》的论述有惊人的相似之处。如认为抽烟、酗酒、熬夜、吃夜宵、在车水马龙的街头呼吸汽车排出的尾气,或在家吸入炒菜时释放的油烟雾气等,都有着致癌的可能性。生活水平的提高,生活方式的改变,社会的转型,生活工作压力的增加,心理负荷的加重,饮食中各种添加剂的加入,营养过剩等,这些与衣食住行有关的不健康因素都有可能引发癌变。上述现代研究说明《内经》对此病的认识是具有一定临床基础的。

2.积的病机理论对临床的指导意义

本篇归结了积的病机主要是寒凝、气滞、血瘀、津液涩渗,四者并合凝聚不得散,日久成积。这一理论长期指导着中医治疗积聚的实践。如对邪毒壅盛的积聚,用攻毒散结法,选取干蟾皮、斑蝥、露蜂房、七叶一枝花、半枝莲之类的药物。气滞血瘀的积聚,用理气活血散结法,选取黄药子、槟榔、枳实、枳壳,桃仁、红花、三棱、莪术之类药物;血瘀甚者,用理气逐瘀散结法,加用水蛭、虻虫、乳香、没药等药物;若有气滞血瘀与痰浊凝结者,可在理气活血的基础上,再合用化痰散结法,以胆南星、生半夏、木馒头、海藻、昆布、象贝母之类药物治之;如肿块坚硬,加用软坚散结法,如穿山甲、皂角刺、夏枯草、山慈菇之类药物。总之,中医临床治疗积聚,治法多样,药物繁多,但万变不离其宗,始终是抓住寒凝、气滞、血瘀、津液涩渗四者为纲。当然,《内经》还告诫;“大积大聚,其可犯也,衰其大半而止,过者死”(《素问·六元正纪大论》);“大毒治病,十去其六;常毒治病,十去其七;小毒治病,十去其八;无毒治病,十去其九;谷肉果菜,食养尽之,无使过之,伤其正也(《素问·五常政大论》)。所以治疗积聚时要防止攻邪太过,损伤正气,可攻邪与扶正兼顾之。

\biaoti{【原文】}

\begin{yuanwen}
黃帝曰:其生於陰\sb{1}者,奈何?岐伯曰:憂思傷心;重寒傷肺;忿怒傷肝;醉以入房,汗出當風傷脾;用力過度,若入房汗出浴,則傷腎,此內外三部之所生病者也\sb{2}。

黃帝曰:善。治之奈何?岐伯答曰:察其所痛,以知其應\sb{3},有餘不足,當補則補,當寫則寫,毋逆天時,是謂至治。
\end{yuanwen}

\biaoti{【校注】}

\begin{jiaozhu}
  \item 生于阴:指内伤五脏的病证。本篇言:“喜怒不节则伤脏,脏伤则病起于阴也”。
  \item 此内外三部之所生病者也:此句与篇首第一段的内容相呼应,总结上下内外的三部之气。
  \item 察其所痛,以知其应:痛,此指病候;应,相应病变脏器部位。言审察病候,以定病位。
\end{jiaozhu}

\biaoti{【理论阐释】}

\xiaobt{内伤五脏的病因}

七情太过、重寒、房劳、劳倦等因素均可造成五脏病变。根据原文举例,有以下二点启示;一是脏病常由内外合邪所致。如重寒伤肺,正可以解释肺部病变的重要病因,这也是《内经》论述五脏病变时比较强调的一个方面,除了本篇所论,在《灵枢·邪气脏腑病形》中亦有类似观点:“愁忧恐惧则伤心;形寒寒饮则伤肺,以其两寒相感,中外皆伤,故气逆而上行;有所堕坠,恶血留内,若有所大怒,气上而不下,积于胁下,则伤肝;有所击仆,若醉入房,汗出当风,则伤脾;有所用力举重,若入房过度,汗出浴水,则伤肾”。二是五脏病的致病原因各有其特点。如心肝之病多伤于精神情感失调,肺病多伤于寒邪,脾病多伤于饮食不节,肾病多伤于劳倦、房室。这些都为后世的脏腑辨证提供了依据。

\biaoti{【临证指要】}

“重寒伤肺”和“形寒寒饮则伤肺,以其两寒相感,内外皆伤,故气逆而上行”,对临床认识咳喘病证,尤其是小儿的支气管哮喘,老人的老年慢性支气管炎等有重要指导价值。此类病人大多素有痰湿寒饮内伏,成为发病之夙根,外有寒邪相袭,或饮食寒凉,以致内外合邪,中外皆伤,故喘咳易反复发作。临床对这类病证治疗,发作时祛除外寒,温化内饮为主;缓解时以补益肺肾之气为重。此外,嘱患者不得过食生冷肥甘之物,以免内寒痰湿更甚,夙根难除。

\xiaojie

本章主要讨论了《内经》中有关病因病机学说的主要内容和学术思想。

1.病因:《内经》将致病因素主要分为三大类:风雨伤上;清湿伤下;喜怒不节、用力过度、入房汗出、五味太过则伤脏。

2.发病:外感发病是由于“两虚相得,乃客其形”,突出正气虚弱在发病中的重要意义,体现了内因为主的发病学观点。内伤发病则是机体内部本身的功能有余不足、气机升降出入失调等所致。

3.传变:疾病传变有外感、内伤之不同。外感疾病传变,大多从皮毛而入,由表入里,逐步深入,最后可达肠胃之外,募原之间,“邪气淫泆,不可胜数”。五脏疾病传变,有逆传,即子病传母;有顺传,即所胜而传,体现了传变的规律性。亦有不以次相传的,体现了传变的特殊性。

4.病理:

(1)阳气失常病机

阳气的病理表现多种多样,可有卫外失常,阳气厥逆,阴虚阳亢,阳热内盛,阳气抑遏,阳气蓄积,以及病久传化等等。

(2)阴精阳气失调的病机

阴精阳气失调的病机包括阴阳互根互用关系的失常,阴阳相互制约关系的失衡,如出现“阳强不能密,阴气乃绝”,甚至见“阴阳离决,精气乃绝”等病变。

(3)五脏虚实的病机

五实证是邪壅五脏,邪无出路的实证,其转机在于使邪有出路,攻邪外出;五虚证是五脏精气虚少,胃气衰败的虚证,其转机在于使胃气来复,则精气生化有源。

(4)九气病机

“百病皆生于气”,气机失调是疾病的主要病机之一。情志剧烈变化可引起气机升降出入逆乱,寒热、劳倦等因素亦可使气机运动失常。

(5)阴阳寒热盛衰的病机

“阳虚则外寒”,“阳盛则外热”均为外感疾病恶寒发热的病机;“阴虚则内热”乃劳倦伤脾,脾气不运,胃中谷气郁而化热所生;“阴盛則内寒”属阴寒上逆,胸阳受损,血脉凝涩之病机。

(6)五脏病机

《至真要大论》病机十九条所阐述的五脏病机,体现了辨证定位,分析病机的思想。如肝风内动,肾阳虚衰,脾湿内困,肺气膹郁,心火亢盛均是五脏病变的典型病机。

(7)六淫病机

病机十九条中皆属于火、热的病机最多,体现了火热致病的普遍性,并表现了火热之邪的病机特点,即火热燔灼,火性炎上,火热伤律,火性急速,且易扰乱心神,可腐肉酿脓,热极生风,扰乱胃肠气机,甚至引起阳盛格阴,真热假寒等证候。寒邪致病,阳气不能蒸腾水液,水液澄澈清冷是其特点。风邪、湿邪可以损伤筋脉。

(8)积的病机

积“得寒乃生,厥乃成积”。外感寒邪,卒然多饮食,起居不节,用力过度,内伤忧怒皆可使气滞、血瘀、津液凝泣,并合凝聚而日久成积。

\zuozhe{(周国琪)}
\ifx \allfiles \undefined
\end{document}
\fi