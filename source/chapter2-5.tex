% -*- coding: utf-8 -*-
%!TEX program = xelatex
\ifx \allfiles \undefined
\documentclass[draft,12pt]{ctexbook}
%\usepackage{xeCJK}
%\usepackage[14pt]{extsizes} %支持8,9,10,11,12,14,17,20pt

%===================文档页面设置====================
%---------------------印刷版尺寸--------------------
%\usepackage[a4paper,hmargin={2.3cm,1.7cm},vmargin=2.3cm,driver=xetex]{geometry}
%--------------------电子版------------------------
\usepackage[a4paper,margin=2cm,driver=xetex]{geometry}
%\usepackage[paperwidth=9.2cm, paperheight=12.4cm, width=9cm, height=12cm,top=0.2cm,
%            bottom=0.4cm,left=0.2cm,right=0.2cm,foot=0cm, nohead,nofoot,driver=xetex]{geometry}

%===================自定义颜色=====================
\usepackage{xcolor}
  \definecolor{mybackgroundcolor}{cmyk}{0.03,0.03,0.18,0}
  \definecolor{myblue}{rgb}{0,0.2,0.6}

%====================字体设置======================
%--------------------中文字体----------------------
%-----------------------xeCJK下设置中文字体------------------------------%
\setCJKfamilyfont{song}{SimSun}                             %宋体 song
\newcommand{\song}{\CJKfamily{song}}                        % 宋体   (Windows自带simsun.ttf)
\setCJKfamilyfont{xs}{NSimSun}                              %新宋体 xs
\newcommand{\xs}{\CJKfamily{xs}}
\setCJKfamilyfont{fs}{FangSong_GB2312}                      %仿宋2312 fs
\newcommand{\fs}{\CJKfamily{fs}}                            %仿宋体 (Windows自带simfs.ttf)
\setCJKfamilyfont{kai}{KaiTi_GB2312}                        %楷体2312  kai
\newcommand{\kai}{\CJKfamily{kai}}
\setCJKfamilyfont{yh}{Microsoft YaHei}                    %微软雅黑 yh
\newcommand{\yh}{\CJKfamily{yh}}
\setCJKfamilyfont{hei}{SimHei}                                    %黑体  hei
\newcommand{\hei}{\CJKfamily{hei}}                          % 黑体   (Windows自带simhei.ttf)
\setCJKfamilyfont{msunicode}{Arial Unicode MS}            %Arial Unicode MS: msunicode
\newcommand{\msunicode}{\CJKfamily{msunicode}}
\setCJKfamilyfont{li}{LiSu}                                            %隶书  li
\newcommand{\li}{\CJKfamily{li}}
\setCJKfamilyfont{yy}{YouYuan}                             %幼圆  yy
\newcommand{\yy}{\CJKfamily{yy}}
\setCJKfamilyfont{xm}{MingLiU}                                        %细明体  xm
\newcommand{\xm}{\CJKfamily{xm}}
\setCJKfamilyfont{xxm}{PMingLiU}                             %新细明体  xxm
\newcommand{\xxm}{\CJKfamily{xxm}}

\setCJKfamilyfont{hwsong}{STSong}                            %华文宋体  hwsong
\newcommand{\hwsong}{\CJKfamily{hwsong}}
\setCJKfamilyfont{hwzs}{STZhongsong}                        %华文中宋  hwzs
\newcommand{\hwzs}{\CJKfamily{hwzs}}
\setCJKfamilyfont{hwfs}{STFangsong}                            %华文仿宋  hwfs
\newcommand{\hwfs}{\CJKfamily{hwfs}}
\setCJKfamilyfont{hwxh}{STXihei}                                %华文细黑  hwxh
\newcommand{\hwxh}{\CJKfamily{hwxh}}
\setCJKfamilyfont{hwl}{STLiti}                                        %华文隶书  hwl
\newcommand{\hwl}{\CJKfamily{hwl}}
\setCJKfamilyfont{hwxw}{STXinwei}                                %华文新魏  hwxw
\newcommand{\hwxw}{\CJKfamily{hwxw}}
\setCJKfamilyfont{hwk}{STKaiti}                                    %华文楷体  hwk
\newcommand{\hwk}{\CJKfamily{hwk}}
\setCJKfamilyfont{hwxk}{STXingkai}                            %华文行楷  hwxk
\newcommand{\hwxk}{\CJKfamily{hwxk}}
\setCJKfamilyfont{hwcy}{STCaiyun}                                 %华文彩云 hwcy
\newcommand{\hwcy}{\CJKfamily{hwcy}}
\setCJKfamilyfont{hwhp}{STHupo}                                 %华文琥珀   hwhp
\newcommand{\hwhp}{\CJKfamily{hwhp}}

\setCJKfamilyfont{fzsong}{Simsun (Founder Extended)}     %方正宋体超大字符集   fzsong
\newcommand{\fzsong}{\CJKfamily{fzsong}}
\setCJKfamilyfont{fzyao}{FZYaoTi}                                    %方正姚体  fzy
\newcommand{\fzyao}{\CJKfamily{fzyao}}
\setCJKfamilyfont{fzshu}{FZShuTi}                                    %方正舒体 fzshu
\newcommand{\fzshu}{\CJKfamily{fzshu}}

\setCJKfamilyfont{asong}{Adobe Song Std}                        %Adobe 宋体  asong
\newcommand{\asong}{\CJKfamily{asong}}
\setCJKfamilyfont{ahei}{Adobe Heiti Std}                            %Adobe 黑体  ahei
\newcommand{\ahei}{\CJKfamily{ahei}}
\setCJKfamilyfont{akai}{Adobe Kaiti Std}                            %Adobe 楷体  akai
\newcommand{\akai}{\CJKfamily{akai}}

%------------------------------设置字体大小------------------------%
\newcommand{\chuhao}{\fontsize{42pt}{\baselineskip}\selectfont}     %初号
\newcommand{\xiaochuhao}{\fontsize{36pt}{\baselineskip}\selectfont} %小初号
\newcommand{\yihao}{\fontsize{28pt}{\baselineskip}\selectfont}      %一号
\newcommand{\xiaoyihao}{\fontsize{24pt}{\baselineskip}\selectfont}
\newcommand{\erhao}{\fontsize{21pt}{\baselineskip}\selectfont}      %二号
\newcommand{\xiaoerhao}{\fontsize{18pt}{\baselineskip}\selectfont}  %小二号
\newcommand{\sanhao}{\fontsize{15.75pt}{\baselineskip}\selectfont}  %三号
\newcommand{\sihao}{\fontsize{14pt}{\baselineskip}\selectfont}%     四号
\newcommand{\xiaosihao}{\fontsize{12pt}{\baselineskip}\selectfont}  %小四号
\newcommand{\wuhao}{\fontsize{10.5pt}{\baselineskip}\selectfont}    %五号
\newcommand{\xiaowuhao}{\fontsize{9pt}{\baselineskip}\selectfont}   %小五号
\newcommand{\liuhao}{\fontsize{7.875pt}{\baselineskip}\selectfont}  %六号
\newcommand{\qihao}{\fontsize{5.25pt}{\baselineskip}\selectfont}    %七号   %中文字体及字号设置
\xeCJKDeclareSubCJKBlock{SIP}{
  "20000 -> "2A6DF,   % CJK Unified Ideographs Extension B
  "2A700 -> "2B73F,   % CJK Unified Ideographs Extension C
  "2B740 -> "2B81F    % CJK Unified Ideographs Extension D
}
%\setCJKmainfont[SIP={[AutoFakeBold=1.8,Color=red]Sun-ExtB},BoldFont=黑体]{宋体}    % 衬线字体 缺省中文字体

\setCJKmainfont{simsun.ttc}[
  Path=fonts/,
  SIP={[Path=fonts/,AutoFakeBold=1.8,Color=red]simsunb.ttf},
  BoldFont=simhei.ttf
]

%SimSun-ExtB
%Sun-ExtB
%AutoFakeBold:自动伪粗,即正文使用\bfseries时生僻字使用伪粗体;
%FakeBold:强制伪粗,即正文中生僻字均使用伪粗体
%\setCJKmainfont[BoldFont=STHeiti,ItalicFont=STKaiti]{STSong}
%\setCJKsansfont{微软雅黑}黑体
%\setCJKsansfont[BoldFont=STHeiti]{STXihei} %serif是有衬线字体sans serif 无衬线字体
%\setCJKmonofont{STFangsong}    %中文等宽字体

%--------------------英文字体----------------------
\setmainfont{simsun.ttc}[
  Path=fonts/,
  BoldFont=simhei.ttf
]
%\setmainfont[BoldFont=黑体]{宋体}  %缺省英文字体
%\setsansfont
%\setmonofont

%===================目录分栏设置====================
\usepackage[toc,lof,lot]{multitoc}    % 目录(含目录、表格目录、插图目录)分栏设置
  %\renewcommand*{\multicolumntoc}{3} % toc分栏数设置,默认为两栏(\multicolumnlof,\multicolumnlot)
  %\setlength{\columnsep}{1.5cm}      % 调整分栏间距
  \setlength{\columnseprule}{0.2pt}   % 调整分栏竖线的宽度

%==================章节格式设置====================
\setcounter{secnumdepth}{3} % 章节等编号深度 3:子子节\subsubsection
\setcounter{tocdepth}{2}    % 目录显示等度 2:子节

\xeCJKsetup{%
  CJKecglue=\hspace{0.15em},      % 调整中英(含数字)间的字间距
  %CJKmath=true,                  % 在数学环境中直接输出汉字(不需要\text{})
  AllowBreakBetweenPuncts=true,   % 允许标点中间断行,减少文字行溢出
}

\ctexset{%
  part={
    name={,篇},
    number=\SZX{part},
    format={\chuhao\bfseries\centering},
    nameformat={},titleformat={}
  },
  section={
    number={\chinese{section}},
    name={第,节}
  },
  subsection={
    number={\chinese{subsection}、},
    aftername={\hspace{-0.01em}}
  },
  subsubsection={
    number={(\chinese{subsubsection})},
    aftername={\hspace {-0.01em}},
    beforeskip={1.3ex minus .8ex},
    afterskip={1ex minus .6ex},
    indent={\parindent}
  },
  paragraph={
    beforeskip=.1\baselineskip,
    indent={\parindent}
  }
}

\newcommand*\SZX[1]{%
  \ifcase\value{#1}%
    \or 上%
    \or 中%
    \or 下%
  \fi
}

%====================页眉设置======================
\usepackage{titleps}%或者\usepackage{titlesec},titlesec包含titleps
\newpagestyle{special}[\small\sffamily]{
  %\setheadrule{.1pt}
  \headrule
  \sethead[\usepage][][\chaptertitle]
  {\chaptertitle}{}{\usepage}
}

\newpagestyle{main}[\small\sffamily]{
  \headrule
  %\sethead[\usepage][][第\thechapter 章\quad\chaptertitle]
%  {\thesection\quad\sectiontitle}{}{\usepage}}
  \sethead[\usepage][][第\chinese{chapter}章\quad\chaptertitle]
  {第\chinese{section}节\quad\sectiontitle}{}{\usepage}
}

\newpagestyle{main2}[\small\sffamily]{
  \headrule
  \sethead[\usepage][][第\chinese{chapter}章\quad\chaptertitle]
  {第\chinese{section}節\quad\sectiontitle}{}{\usepage}
}

%================ PDF 书签设置=====================
\usepackage{bookmark}[
  depth=2,        % 书签深度 2:子节
  open,           % 默认展开书签
  openlevel=2,    % 展开书签深度 2:子节
  numbered,       % 显示编号
  atend,
]
  % 相比hyperref,bookmark宏包大多数时候只需要编译一次,
  % 而且书签的颜色和字体也可以定制。
  % 比hyperref 更专业 (自动加载hyperref)

%\bookmarksetup{italic,bold,color=blue} % 书签字体斜体/粗体/颜色设置

%------------重置每篇章计数器,必须在hyperref/bookmark之后------------
\makeatletter
  \@addtoreset{chapter}{part}
\makeatother

%------------hyperref 超链接设置------------------------
\hypersetup{%
  pdfencoding=auto,   % 解决新版ctex,引起hyperref UTF-16预警
  colorlinks=true,    % 注释掉此项则交叉引用为彩色边框true/false
  pdfborder=001,      % 注释掉此项则交叉引用为彩色边框
  citecolor=teal,
  linkcolor=myblue,
  urlcolor=black,
  %psdextra,          % 配合使用bookmark宏包,可以直接在pdf 书签中显示数学公式
}

%------------PDF 属性设置------------------------------
\hypersetup{%
  pdfkeywords={黄帝内经,内经,内经讲义,21世纪课程教材},    % 关键词
  %pdfsubject={latex},        % 主题
  pdfauthor={主编:王洪图},   % 作者
  pdftitle={内经讲义},        % 标题
  %pdfcreator={texlive2011}   % pdf创建器
}

%------------PDF 加密----------------------------------
%仅适用于xelatex引擎 基于xdvipdfmx
%\special{pdf:encrypt ownerpw (abc) userpw (xyz) length 128 perm 2052}

%仅适用于pdflatex引擎
%\usepackage[owner=Donald,user=Knuth,print=false]{pdfcrypt}

%其他可使用第三方工具 如:pdftk
%pdftk inputfile.pdf output outputfile.pdf encrypt_128bit owner_pw yourownerpw user_pw youruserpw

%=============自定义环境、列表及列表设置================
% 标题
\def\biaoti#1{\vspace{1.7ex plus 3ex minus .2ex}{\bfseries #1}}%\noindent\hei
% 小标题
\def\xiaobt#1{{\bfseries #1}}
% 小结
\def\xiaojie {\vspace{1.8ex plus .3ex minus .3ex}\centerline{\large\bfseries 小\ \ 结}\vspace{.1\baselineskip}}
% 作者
\def\zuozhe#1{\rightline{\bfseries #1}}

\newcounter{yuanwen}    % 新计数器 yuanwen
\newcounter{jiaozhu}    % 新计数器 jiaozhu

\newenvironment{yuanwen}[2][【原文】]{%
  %\biaoti{#1}\par
  \stepcounter{yuanwen}   % 计数器 yuanwen+1
  \bfseries #2}
  {}

\usepackage{enumitem}
\newenvironment{jiaozhu}[1][【校注】]{%
  %\biaoti{#1}\par
  \stepcounter{jiaozhu}   % 计数器 jiaozhu+1
  \begin{enumerate}[%
    label=\mylabel{\arabic*}{\circledctr*},before=\small,fullwidth,%
    itemindent=\parindent,listparindent=\parindent,%labelsep=-1pt,%labelwidth=0em,
    itemsep=0pt,topsep=0pt,partopsep=0pt,parsep=0pt
  ]}
  {\end{enumerate}}

%===================注解与原文相互跳转====================
%----------------第1部分 设置相互跳转锚点-----------------
\makeatletter
  \protected\def\mylabel#1#2{% 注解-->原文
    \hyperlink{back:\theyuanwen:#1}{\Hy@raisedlink{\hypertarget{\thejiaozhu:#1}{}}#2}}

  \protected\def\myref#1#2{% 原文-->注解
    \hyperlink{\theyuanwen:#1}{\Hy@raisedlink{\hypertarget{back:\theyuanwen:#1}{}}#2}}
  %此处\theyuanwen:#1实际指thejiaozhu:#1,只是\thejiaozhu计数器还没更新,故使用\theyuanwen计数器代替
\makeatother

\protected\def\myjzref#1{% 脚注中的引用(引用到原文)
  \hyperlink{\theyuanwen:#1}{\circlednum{#1}}}

\def\sb#1{\myref{#1}{\textsuperscript{\circlednum{#1}}}}    % 带圈数字上标

%----------------第2部分 调整锚点垂直距离-----------------
\def\HyperRaiseLinkDefault{.8\baselineskip} %调整锚点垂直距离
%\let\oldhypertarget\hypertarget
%\makeatletter
%  \def\hypertarget#1#2{\Hy@raisedlink{\oldhypertarget{#1}{#2}}}
%\makeatother

%====================带圈数字列表标头====================
\newfontfamily\circledfont[Path = fonts/]{meiryo.ttc}  % 日文字体,明瞭体
%\newfontfamily\circledfont{Meiryo}  % 日文字体,明瞭体

\protected\def\circlednum#1{{\makexeCJKinactive\circledfont\textcircled{#1}}}

\newcommand*\circledctr[1]{%
  \expandafter\circlednum\expandafter{\number\value{#1}}}
\AddEnumerateCounter*\circledctr\circlednum{1}

% 参考自:http://bbs.ctex.org/forum.php?mod=redirect&goto=findpost&ptid=78709&pid=460496&fromuid=40353

%======================插图/tikz图========================
\usepackage{graphicx,subcaption,wrapfig}    % 图,subcaption含子图功能代替subfig,图文混排
  \graphicspath{{img/}}                     % 设置图片文件路径

\def\pgfsysdriver{pgfsys-xetex.def}         % 设置tikz的驱动引擎
\usepackage{tikz}
  \usetikzlibrary{calc,decorations.text,arrows,positioning}

%---------设置tikz图片默认格式(字号、行间距、单元格高度)-------
\let\oldtikzpicture\tikzpicture
\renewcommand{\tikzpicture}{%
  \small
  \renewcommand{\baselinestretch}{0.2}
  \linespread{0.2}
  \oldtikzpicture
}

%=========================表格相关===============================
\usepackage{%
  multirow,                   % 单元格纵向合并
  array,makecell,longtable,   % 表格功能加强,tabu的依赖
  tabu-last-fix,              % "强大的表格工具" 本地修复版
  diagbox,                    % 表头斜线
  threeparttable,             % 表格内脚注(需打补丁支持tabu,longtabu)
}

%----------给threeparttable打补丁用于tabu,longtabu--------------
%解决方案来自:http://bbs.ctex.org/forum.php?mod=redirect&goto=findpost&ptid=80318&pid=467217&fromuid=40353
\usepackage{xpatch}

\makeatletter
  \chardef\TPT@@@asteriskcatcode=\catcode`*
  \catcode`*=11
  \xpatchcmd{\threeparttable}
    {\TPT@hookin{tabular}}
    {\TPT@hookin{tabular}\TPT@hookin{tabu}}
    {}{}
  \catcode`*=\TPT@@@asteriskcatcode
\makeatother

%------------设置表格默认格式(字号、行间距、单元格高度)------------
\let\oldtabular\tabular
\renewcommand{\tabular}{%
  \renewcommand\baselinestretch{0.9}\small    % 设置行间距和字号
  \renewcommand\arraystretch{1.5}             % 调整单元格高度
  %\renewcommand\multirowsetup{\centering}
  \oldtabular
}
%设置行间距,且必须放在字号设置前 否则无效
%或者使用\fontsize{<size>}{<baseline>}\selectfont 同时设置字号和行间距

\let\oldtabu\tabu
\renewcommand{\tabu}{%
  \renewcommand\baselinestretch{0.9}\small    % 设置行间距和字号
  \renewcommand\arraystretch{1.8}             % 调整单元格高度
  %\renewcommand\multirowsetup{\centering}
  \oldtabu
}

%------------模仿booktabs宏包的三线宽度设置---------------
\def\toprule   {\Xhline{.08em}}
\def\midrule   {\Xhline{.05em}}
\def\bottomrule{\Xhline{.08em}}
%-------------------------------------
%\setlength{\arrayrulewidth}{2pt} 设定表格中所有边框的线宽为同样的值
%\Xhline{} \Xcline{}分别设定表格中水平线的宽度 makecell包提供

%表格中垂直线的宽度可以通过在表格导言区(preamble),利用命令 !{\vrule width1.2pt} 替换 | 即可

%=================图表设置===============================
%---------------图表标号设置-----------------------------
\renewcommand\thefigure{\arabic{section}-\arabic{figure}}
\renewcommand\thetable {\arabic{section}-\arabic{table}}

\usepackage{caption}
  \captionsetup{font=small,}
  \captionsetup[table] {labelfont=bf,textfont=bf,belowskip=3pt,aboveskip=0pt} %仅表格 top
  \captionsetup[figure]{belowskip=0pt,aboveskip=3pt}  %仅图片 below

%\setlength{\abovecaptionskip}{3pt}
%\setlength{\belowcaptionskip}{3pt} %图、表题目上下的间距
\setlength{\intextsep}   {5pt}  %浮动体和正文间的距离
\setlength{\textfloatsep}{5pt}

%====================全文水印==========================
%解决方案来自:
%http://bbs.ctex.org/forum.php?mod=redirect&goto=findpost&ptid=79190&pid=462496&fromuid=40353
%https://zhuanlan.zhihu.com/p/19734756?columnSlug=LaTeX
\usepackage{eso-pic}

%eso-pic中\AtPageCenter有点水平偏右
\renewcommand\AtPageCenter[1]{\parbox[b][\paperheight]{\paperwidth}{\vfill\centering#1\vfill}}

\newcommand{\watermark}[3]{%
  \AddToShipoutPictureBG{%
    \AtPageCenter{%
      \tikz\node[%
        overlay,
        text=red!50,
        %font=\sffamily\bfseries,
        rotate=#1,
        scale=#2
      ]{#3};
    }
  }
}

\newcommand{\watermarkoff}{\ClearShipoutPictureBG}

\watermark{45}{15}{草\ 稿}    %启用全文水印

%=============花括号分支结构图=========================
\usepackage{schemata}

\xpatchcmd{\schema}
  {1.44265ex}{-1ex}
  {}{}

\newcommand\SC[2] {\schema{\schemabox{#1}}{\schemabox{#2}}}
\newcommand\SCh[4]{\Schema{#1}{#2}{\schemabox{#3}}{\schemabox{#4}}}

%=======================================================

\begin{document}
\pagestyle{main2}
\fi
\chapter{病证}%第五章

病,是指人体在生理和心理即形与神的失常状态。《内经》言病,多以“疾”、“病”和“候”字称之。“证”字仅见于《素问·至真要大论》。“症”字晚出,不见于经。“证”与“候”同义,故后人常合称“证候”,成为中医学疾病理论的特色与优势。

《内经》有关病证的内容极为丰富,所载病证多达350余种,涵盖了临床各科,对许多疾病还辟专篇进行了系统深入的阐述。经中多采用脏腑分证、经络分证和病因分证等方法对证候予以分类,成为辨证体系的雏型。

本章仅节选其中8篇专论疾病的原文。就内容而言,有外感热病类疾病(《素问·热论》、《素问·评热病论》),有外感风邪所致的风类疾病(《素问·风论》),有咳类疾病(《素问·咳论》),有疼痛类疾病(《素问·举痛论》),有痹类疾病(《素问·痹论》有痿类疾病(《素问·痿论》),以及水肿、肤胀、癥瘕类疾病(《灵枢·水胀》),凡70余病。这些病名,除反映了《内经》关于疾病的命名规律,疾病的分类和诸病的临床表现外,还涉及疾病的演变规律,疾病的诊断和鉴别诊断,以及对疾病的预后、治疗、护理等方面的基本认识。

\section{素問·熱論}%第一節

\biaoti{【原文】}

\begin{yuanwen}
黃帝問曰:今夫熱病,皆傷寒\sb{1}之類也,或愈或死,其死皆以六七日之間,其愈皆以十日以上者,何也?不知其解,願聞其故。

岐伯對曰:巨陽者,諸陽之屬也。其脈連於風府,故為諸陽主氣也\sb{2}。人之傷於寒也,則為病熱,熱雖甚不死,其兩感\sb{3}於寒而病者,必不免於死。
\end{yuanwen}

\biaoti{【校注】}

\begin{jiaozhu}
  \item 伤寒:病名,外感性热病的总称。
  \item 巨阳者,诸阳之属也。其脉连于风府,故为诸阳主气也:巨阳,即太阳。属,统帅,聚会之意。风府,督脉穴。杨上善《太素·热病决》注:“诸阳者,督脉、阳维脉也。督脉,阳脉之海;阳维,维诸阳脉,总会风府,属于太阳。故足太阳脉为诸阳主气。”
  \item 两感:表里两经同时感受邪气而发病。如太阳与少阴两感,阳明与太阴两感,少阳与厥阴两感。
\end{jiaozhu}

\biaoti{【理论阐释】}

1.关于伤寒

伤寒可分为广义伤寒和狭义伤寒,“今夫热病者,皆伤寒之类也。”明确指出一切外感热病,皆属于伤寒的范畴。将外感性热病命名为伤寒,是从病因角度而言,寒在此泛指四时邪气。谓之热病,则是以症状特点来命名。因为发热是外感病的共同特征和主要症状,故泛称外感病为热病,又叫外感热病。《难经·五十八难》曰:“伤寒有五:有中风,有伤寒,有湿温,有热病,有温病。”从中可知本篇所说之伤寒,就是“伤寒有五”之广义伤寒。而狭义伤寒,就是五者之中由感受寒邪引起的外感性热病。

2.热病的预后

“其两感于寒而病者,必不免于死”,以及下文“其不两感于寒者……病日已矣”。指出外感病的预后和疾病的转归,关系到受邪部位、感邪轻重、病邪性质、体质强弱等方面的因素,即与邪正力量的消长变化有关。文中“人之伤于寒也,则为病热,热虽甚不死”,其中之热是寒邪束表,汗孔闭塞,卫气内郁,不得宣泄,邪正交争所致,此时若能正确运用汗法,则邪随汗解,诸症消除。正如《素问·生气通天论》所说:“体若燔炭,汗出而散”。两感于寒,则是表里两经同时受邪,邪气迅速内传,伤及脏腑及营卫气血,邪充内外,邪盛正衰,若救治不及时,“必不免于死”。文中“死”与“不死”是指病情之轻重,预后之好坏而言。其实就是“两感于寒”者,只要救治及时,方法得当,亦可有生还之机。

\biaoti{【原文】}

\begin{yuanwen}
帝曰:願聞其狀。岐伯曰:傷寒一日\sb{1},巨陽受之,故頭項痛,腰脊強;二日,陽明受之,陽明主肉,其脈俠鼻絡於目,故身熱\sb{2},目疼而鼻乾,不得臥也;三日,少陽受之,少陽主膽\sb{3},其脈循脇絡於耳,故胸脇痛而耳聾;三陽經絡皆受其病,而未入於藏者,故可汗而已\sb{4}。四日,太陰受之,太陰脈布胃中絡於嗌,故腹滿而嗌乾;五日,少陰受之,少陰脈貫腎絡於肺,繫舌本,故口燥舌乾而渴;六日,厥陰受之,厥陰脈循陰器而絡於肝,故煩滿而囊缩\sb{5}。三陰三陽,五藏六府皆受病,榮衛不行,五藏不通,則死矣\sb{6}。

其不兩感於寒者,七日\sb{7},巨陽病衰,頭痛少愈;八日,陽明病衰,身熱少愈;九日,少陽病衰,耳聾微聞;十日,太陰病衰,腹減如故,則思飲食;十一日,少陰病衰,渴止不滿\sb{8},舌乾已而嚏\sb{9};十二日,厥陰病衰,囊縱,少腹微下\sb{10},大氣\sb{11}皆去,病日已矣。

帝曰:治之奈何?岐伯曰:治之各通其藏脈\sb{12},病日衰已矣。其未滿三日者,可汗而已;其滿三日者,可泄而已\sb{13}。

帝曰:熱病已愈,時有所遺\sb{14}者,何也?岐伯曰:諸遺者,熱甚而強食之,故有所遺也。若此者,皆病已衰,而熱有所藏,因其穀气相薄,兩熱相合,故有所遺也。帝曰:善。治遺奈何?岐伯曰:視其虛實,調其逆從,可使必已矣。帝曰:病熱當何禁之?岐伯曰:病熱少愈,食肉則復\sb{15},多食則遺,此其禁也。
\end{yuanwen}

\biaoti{【校注】}

\begin{jiaozhu}
  \item 一日:一日与下文二日、三日、四日、五日、六日,都是指外感热病传变的次序与发展的阶段,而不能简单地理解成具体的日数。
  \item 身热:此谓身体发热,按之烫手,愈按愈热。张介宾注:“伤寒多发热,而独此云身热者,盖阳明主肌肉,身热尤甚也。”
  \item 少阳主胆:胆《甲乙经》、《太素》均作“骨”,可从。《灵枢·经脉》说:“胆足少阳之脉……是主骨所生病者”可证。少阳胆与厥阴肝相表里,而肝主筋,筋会于骨,所以少阳主骨。此可与上文“阳明主肉”相应。
  \item 未入于脏者,故可汗而已:张介宾注:“三阳为表属腑,邪在表而未入于三阴之脏者,皆可汗而散也。”
  \item 烦满而囊缩:满,同“懑”,烦懑之意。囊缩,指阴囊收缩。
  \item 三阴三阳,……则死矣:此虽属“不两感于寒者”,但邪气深重仍可致正气衰竭而亡。故张介宾注:“伤寒邪在经络,本为表证……若六经传遍而邪不退,则深入于腑,腑不退则深至于脏,故五脏六腑皆受病矣。邪盛于外则营卫不行,气竭于内则五脏不通,故六七日间致死也。”另一说认为此属“两感于寒者”,故死。可参。
  \item 七日:七日与下文八日、九日、十日、十一日、十二日均指热病过程中,邪退正复疾病转愈的概数,其时间长短取决于邪正力量的对比。
  \item 不满:丹波元简云:“《甲乙》、《伤寒例》并无‘不满’二字,上文不言腹满,此必衍文。”可参。
  \item 嚏:少阴病邪气初退,正气未复,故嚏。《灵枢·口问》云:“阳气和利,满于心,出于鼻,故为嚏。”
  \item 囊纵,少腹微下:阴囊收缩及少腹拘急的症状渐见舒缓。
  \item 大气:指邪气。
  \item 治之各通其脏脉:通,疏通,调治。脏脉,脏腑之脉。即调治病变所在之脏腑经脉。
  \item 其未满三日者,可泄而已:未满三日,言病犹在三阳之表;已满三日,指病已入三阴之里。《内经》治热病主要用针刺疗法发汗泄热,张琦注:“经言刺法,故曰通其脏脉,三日以前,病在三阳,故可汗;三日以后,病在三阴,故可泄。泄谓泄越其热,非攻下之谓也。”
  \item 遗:病邪遗留未尽,迁延不愈。杨上善《太素·热病决》注:“遗,余也。大气虽去,犹有残热在脏腑之内外,因多食,以谷气热与故热相薄,重发热病,名曰余热病也。”
  \item 复:指病愈而复发。
\end{jiaozhu}

\biaoti{【理论阐释】}

\xiaobt{热病的六经主证、传变、治疗、禁忌及意义}

本节指出热病中病情简单,发病比较典型的非“两感于寒者”的一类病征,其主症一按各经脉循行部位来归纳,如太阳病的“头项痛,腰脊强”、少阳病的“胸胁痛而耳聋”、少阴病的“口燥,舌干而渴”;二按经脉循行结合脏腑功能而论之,如阳明病的“目痛鼻干,身热不得卧”、太阴病的“腹满而咽干”、厥阴病的“烦满,囊缩”。

其传变,原文指出伤寒在经之邪有向里传与不向里传的区别,其向里传的规律为由表及里,由阳入阴;其次序从太阳始,依次传及阳明、少阳、太阴、少阴、厥阴。邪若不内传,各经缓解的时间大约在受病的第七天。

其治疗总的精神是“各通其脏脉”,突出一个“通”字,具体“其未满三日者”,邪尚在三阳之表,可用汗法;“其满三日者”邪热已滞于三阴之里,应用泄热之法,体现出对外感热病的治疗以“祛邪为主”的原则。

《素问·热论》中所列举的六经症状皆为实证、热证,未及虚证、寒证;治疗提出汗、泄两法。其理论为后世热病的辨证论治奠定了坚实的基础,后汉·张机《伤寒论》在《热论》六经分证基础上,进一步确立了六经辨证纲领,在病证上补充了虚证和寒证;在原六经单传的基础上提出越经、直中、合病、并病等多种传变形式;对热病治法则较《热论》更加全面。关于《热病》禁忌热论提出“食复”,《伤寒论》又提出“劳复”,从而使热病的证治理法更加完善。

\biaoti{【临证指要】}

1.六经热病的治法

本节所论六经热病的临床特征,符合临床实际,用药物治疗同样可以收到一定效果。

太阳病:太阳病由外邪侵犯太阳经脉所致。临床特点为头项痛,腰脊强,发热,恶寒。治法与寒邪所伤之“伤寒病”同。可用麻黄汤或杏苏散之类。

阳明病:阳明病由太阳经病邪传入阳明经所致。其临床特点为身热,目痛,鼻干,不得卧。治法为解肌散热,用柴葛解肌汤。若里热偏盛,伤律耗液,用白虎汤。

少阳病:少阳病由太阳经病邪传至少阳经所致。其临床特点为胸胁痛,耳聋。治法为和解少阳,用小柴胡汤。

太阴病:太阴病由太阳经病邪传至太阴经所致,其临床特点为腹满,咽干。至此,邪已入里,可用针刺泄热,其有热结者,也可用泻下法,选小承气汤或增液承气汤之类。

少阴病:少阴病出太阳经病邪传至少阴经所致。其临床特征为口燥,舌干而渴。邪热入里,热盛伤津,针刺治疗可用泄热法。药物治疗可选用黄连阿胶汤或增液承气汤之类。

厥阴病:厥阴病由太阳经病邪传至厥阴经所致。其临床特征为烦满而囊缩,在女子如李梃《医学入门》所说:“阴户急痛引少腹。”针刺治疗当用泄热法,药物治行可选四逆散及金铃子散之类。

2.热病的饮食护理

疾病的饮食护理在治疗中占重要地位,合理的饮食可固护胃气,增强正气,有利于疾病的恢复,反之,可致疾病反复或加重。我国古代虽医护不分,但《内经》中的护理思想已十分明确,就热病而言,原文指出遗热的原因是“热甚而强食”所致,在热病过程中,由于热邪炽盛,伤及胃气,消化机能低下,勉强进食,食入之水谷之气与热邪相薄,会使热病迁延不愈。若热势稍减,就急于补益,进食肉类等不易消化之品或大量进食,就会使热病复发如故。因此热病期间饮食调养十分重要,一要少食,二要清淡,待热邪已去,胃气恢复,方可进补。但临床还应视具体情况而定,对身体虚弱赢瘦者又当权变,故张介宾《类经·疾病类》说:“凡病后脾胃气虚,未能消化饮食,故于肉食之类皆当从缓,若犯食复,为害非浅。其有挟虚内馁者,又不可过于禁制,所以贵得宜也。”

\biaoti{【原文】}

\begin{yuanwen}
帝曰:其病兩感於寒者,其脈應與其病形何如?岐伯曰:兩感於寒者,病一日,則巨陽與少陰俱病,則頭痛口乾而煩滿;二日,則陽明舆太陰俱病,則腹滿身熱,不欲食,譫言\sb{1};三日,則少陽與厥陰俱病,則耳聾囊縮而厥\sb{2},水漿不入,不知人,六日死。帝曰:五藏已傷,六府不通,榮衛不行,如是之後,三日乃死,何也?岐伯曰:陽明者,十二經脈之長也,其血氣盛,故不知人三日,其氣乃盡,故死矣。
\end{yuanwen}

\biaoti{【校注】}

\begin{jiaozhu}
  \item 谵言:即谵语。
  \item 厥,指手足逆冷。
\end{jiaozhu}

\biaoti{【临证指要】}

\xiaobt{保胃气是治疗热病的根本}

该段“两感于寒者”是外感热病中最为严重的一类病证。表里两经同病,说明邪盛正衰,三日六经俱病,腑脏皆伤,说明起病急、发展快,“六日死”提示病情重、预后差。这是对前文“两感于寒而病者,必不免于死”的进一步阐释。其死最终是由于胃气的衰败。阳明属胃,为水谷气血之海,十二经脉气血皆赖以注,故《素问·血气形志篇》云:“阳明常多气多血”,《素问·太阴阳明论》又云:“阳明者,五脏六腑之海”,《素问·玉机真脏论》亦云:“五脏者,皆禀气于胃,胃者五脏之本也。”在两感热病中,阳明之气衰,气血化源少,《灵枢·营卫生会》曰:“血者,神气也”,神气先绝,“故不知人”;三日之后,阳明经气竭绝,脏腑经脉无所受气,故生命垂危。前文曰病遗及食复的原因是饮食不当伤及胃气所致,此又指出“水浆不入,不知人,六日死”的严重证候,前后呼应,均说明热病预后好坏,与阳明胃气盛衰存亡密切相关,此即“有胃气则生,无胃气则死”。纵观张仲景《伤寒论》立法处方,无不把保胃气,存津液,作为治病的根本原则,发汗必滋化源,清下不伤胃气,如用桂枝汤则“服已须臾,啜热稀粥”用白虎汤必用粳米;调胃承气汤中必用甘草等,无不反映了在治疗热病过程中固护胃气的重要性,其目的是提高正气,治病达邪。

\biaoti{【原文】}

\begin{yuanwen}
凡病傷寒而成溫\sb{1}者,先夏至日者為病溫,后夏至日者為病暑,暑當與汗皆出,勿止\sb{2}。
\end{yuanwen}

\biaoti{【校注】}

\begin{jiaozhu}
  \item 温:指温热病而言。
  \item 暑当与汗皆出,勿止:汗出,则暑邪随之外泄,故不可止之。张介宾注:“暑气侵入,当令有汗,则暑随汗出,故曰勿止。”
\end{jiaozhu}

\biaoti{【理论阐解】}

本段所论之温病、暑病,可从两方面来理解均:一均由伏邪所致,即同是感受寒邪,邪气伏藏体内而不发,至来年春夏发病,虽感受病因相同,但因发病时间和发病特点不同而有温病和暑病质区别。以季节而言,温病发于夏至以前,暑病发于夏至以后,如马莳《黄帝内经素问注证发微》注:“此言温病暑病各有其时也。……其有所谓温病者,则夏至以前者为病温,后夏至日者为病暑”。以发热轻重而言,温病发热较轻,暑病发热较重,如高世栻《黄帝素问直解》注:“温,犹热也。暑,热之极也”。二是结合句首“今夫热病者,皆伤寒之类也”,可将此理解为广义伤寒,即冬日感寒为伤寒(狭义);春日夏日感时邪分别为温病,暑病。这两种理解均对后世温病学说的形成和发展有很大影响。如伏气温病学说,就导源于此,在各种温病中,明确提出并公认有“伏邪”者,仅见于“春温”与“伏暑”;如其病因为四时不同的时邪,就会有不同特点的外感热病发生。

\biaoti{【临证指要】}

\xiaobt{暑病的治疗}

暑病专发夏季,有明显的季节性,暑天气候炎热,机体为适应外环境的变化,汗孔开张,汗出散热,以维持正常体温的恒定。如《灵枢·五癃津液别》说:“天暑衣厚则腠理开,故汗出。”若感受暑邪,暑热迫津外泄,则汗出,此汗出有利于暑热之邪的排出。若误用止汗收敛之法,会致闭门留寇,邪陷心包的危重证候。文中“暑当与汗皆出,勿止。”指出暑病的治疗大法,且勿见汗止汗,当以清热为主,热清汗出,脉静身凉。此外暑热致汗出,往往可使气随汗泄,故暑邪致病有易耗气伤阴的特点,治当以清热涤暑,益气生阴为大法,清暑益气汤为常用之方。

\section{素問·評熱病論}%第二節

\biaoti{【原文】}

\begin{yuanwen}
黃帝問曰:有溫病者,汗出輒復熱\sb{1},而脈躁疾\sb{2}不為汗衰,狂言不能食,病名為何?岐伯對曰:病名陰陽交\sb{3},交者死也。帝曰:願聞其說。岐伯曰:人所以汗出者,皆生於穀,榖生於精\sb{4},今邪氣交爭於骨肉而得汗者,是邪却而精勝也。精勝,則當能食而不復熱。復熱者,邪氣也。汗者,精氣也,今汗出而輒復熱者,是邪勝也,不能食者,精無俾\sb{5}也。病而留者,其壽可立而傾也。且夫《熱論》\sb{6}曰:汗出而脈尙躁盛者死。今脈不與汗相應,此不勝其病也,其死明矣。狂言者,是失志,失志者死。今見三死\sb{7},不見一生,雖愈必死也。
\end{yuanwen}

\biaoti{【校注】}

\begin{jiaozhu}
  \item 汗出辄复热:辄,即立刻。谓汗出热退后又立即再度发热。
  \item 脉躁疾:谓阳邪入脉,脉象躁动不安而急数。
  \item 阴阳交:阴阳在此分别指阳热邪气和阴精正气。交,交结,交争。阴阳交,指阳热邪气入于阴分,邪正交结不解,正不胜邪的危重病证。
  \item 谷生于精:“于”为助词,无义。即水谷是精气化生之源。张介宾注:“谷气内盛则生精,精气外达则为汗”。
  \item 精无俾:俾,补充,补益之意。即精气得不到补益充养。
  \item 《热论》:指《灵枢·热病》篇或古经《热论》。《热病》谓“热病已得汗而脉尚躁盛,此阴脉之极也,死;得其汗而脉静者,生。”与本篇所载“汗出而脉尚躁盛者死”,文意相近。
  \item 三死:指汗出复热而不能食、脉躁疾、狂言三症。杨上善《太素·热病决》注:“汗出而热不衰,死有三候:一不能食,而犹脉躁,三者失志。汗出而热,有此三死之候,未见一生之状,虽差必死。”
\end{jiaozhu}

\biaoti{【理论阐释】}

\xiaobt{阴阳交的病机特点}

本段经文对阴阳交的病机、症状、预后进行了系统的论述,其中病机为重要方面。从症状分析:汗出辙复热,是正不胜邪所致;不能食,说明里热燔灼,劫伤胃阴;狂言失志,是由于肾精受损,阴精不足,热扰神明;脉躁疾,是阴不制阳,邪热充斥脉道。从预后看,热留伤精“其寿可立而倾也”,说明病情严重、凶险,预后不良。整个疾病过程紧紧围绕阳热邪盛,阴精不足,阴精正气不能制服阳热邪气这一病机来认识疾病的严重性,强调了阳邪与阴精双方的胜负存亡在温热病转归中所起的决定性作用。

本段提出温病邪留立倾与《素问·热论》中所载的非两感伤寒“热虽甚不死”,二者区别的关键在于病机不同。前者指在温热病的过程中,人体感受温热病邪,邪正交争于骨肉之间,邪盛正虚,阴精正气无力祛邪,汗出而邪热不退的危重病理变化,治疗颇为棘手;后者则指寒邪伤于肌表,卫气郁遏不得达表,邪盛正未衰的病理变化,此时若用发汗解表之法,邪随汗解,则脉静身凉,病即痊愈,如张介宾《类经·疾病类》说:“寒散则热退,故虽甚不致死。”从中可知,“病机”是诊治疾病的主要依据,治疗温热病应谨守“阳盛精衰”这一基本病机,方能临证不惑。

\biaoti{【临证指要】}

本段经文对后世温病学派以很大启迪。温病学派一致认为汗出病减为隹兆,反之其证凶险。观温病危重证候,不外乎高热反复,阴耗液枯,动风动血,热扰神明等几方面。在本篇阳热之邪须赖阴精以制胜的观点启发下,温病学派结合临床,制定出一系列相应的治疗措施,将“保津液”列为温病治疗之首务,力倡“热病以救阴为先”,提出“救阴以泄热为要”的扶正祛邪兼治的基本法则,使温病类似阴阳交的危重证候的治疗取得了很大进展。临床治疗可根据不同情况,选择适宜方药,如热入营阴可用清营汤;热陷心包,用清宫汤送服安宫牛黄丸或至宝丹、紫雪丹;热闭心包兼腑实的,可用牛黄承气汤;热盛动风用羚羊钩藤汤;后期热灼真阴见证,可用黄连阿胶汤、加减复脉汤等。其总的治疗精神是以清热滋阴为主,这也是“存得一分津液,便有一分生机”的意义所在。

\biaoti{【原文】}

\begin{yuanwen}
帝曰:有病身熱,汗出煩滿,煩滿不為汗解,此為何病?岐伯曰:汗出而身熱者,風也;汗出而煩滿不解者,厥\sb{1}也。病名曰風厥\sb{2}。帝曰:願卒聞之。岐伯曰:巨陽主氣\sb{3},故先受邪,少陰舆其為表裏也,得熱則上從之\sb{4},從之則厥也。帝曰:治之奈何?岐伯曰:表裏刺之\sb{5},飲之服湯\sb{6}。
\end{yuanwen}

\biaoti{【校注】}

\begin{jiaozhu}
  \item 厥:逆也,此指少阴之气上逆。
  \item 风厥:指风袭太阳,精亏不足,引动少阴虚火上逆而致汗出、发热、烦闷不除的病证。
  \item 巨阳主气:足太阳经主宰全身阳经之气,阳主表。张介宾《类经·疾病类》注:“巨阳主气,气言表也。”
  \item 上从之:太阳与少阴相表里,太阳受邪发热,少阴虚火随之上逆。
  \item 表里刺之:即泻太阳,补少阴。张介宾《类经·疾病类》注:“阳邪盛者阴必虚,故当泻太阳之热,补少阴之气,合表里而刺之也。”
  \item 饮之服汤:谓在表里刺之的同时配合以汤药内服。马莳《素问注证发微》注:“又当饮之以汤剂,以止逆上之肾气,则可以治斯疾也。”
\end{jiaozhu}

\biaoti{【理论阐释】}

\xiaobt{关于风厥病名}

风厥病名《内经》有三处提到,但这三处所指不同。《素问·阴阳别论》曰:“二阳一阴发病,主惊骇,背痛,善噫,善欠,名曰风厥。”张介宾《类经·疾病类》注:“二阳,胃与大肠也;一阴,肝与心主也。”张志聪《黄帝内经素问集注》注:“此厥阴风木厥逆之为病也,风木为病,肝及胃土,故名风厥”。即肝气郁滞,横逆犯胃,胃失和降,故致噫、欠诸症。《灵枢·五变》曰:“人之善病风厥漉汗者……肉不坚,腠理踈也。”此指素体虚弱,卫外不固,易感风邪出现的病证。本篇之风厥指太阳少阴并病,少阴虚火上逆的病证。以上三篇所指风厥,病名虽同,其实各异。

\biaoti{【临证指要】}

本篇风厥病因为太阳受风所致,风为阳邪,其性开泄,风邪袭表,故常多汗,汗多必耗阴精,太阳与少阴为表里,阴伤精亏,邪入少阴,少阴经气上逆,从而出现热病变证,而有阴虚于里,风袭于表的特点。当表里兼治,扶正祛邪同用,针刺可选太阳经的风门穴和少阴经的太溪穴。药物治疗当滋阴解表,方选加减葳蕤湯之类。

\biaoti{【原文】}

\begin{yuanwen}
帝曰:勞風\sb{1}為病何如?岐伯曰:勞風法在肺下\sb{2},其為病也,使人強上冥視\sb{3},唾出若涕,惡風而振寒,此為勞風之病。帝曰:治之奈何?岐伯曰:以救俛仰,\sb{4},巨陽引\sb{5}。精者三日,中年者五日,不精者七日\sb{6}。咳出青黄涕,其狀如濃,大如彈丸,從口中若鼻中出,不出則傷肺,傷肺則死也。
\end{yuanwen}

\biaoti{【校注】}

\begin{jiaozhu}
  \item 劳风:杨上善《太素·热病说》注:“劳中得风为病,名曰劳中,亦曰劳风。”即因劳而虚,因虚而感受风邪为劳风。
  \item 法在肺下:法,常也。肺下,指肺部。指劳风病位在肺部。
  \item 强上冥视:于鬯《香草续较书》云:“上,疑‘工’字之误。工,盖‘项’字之借。项谐‘工’声,故借工为‘项’,强工者,强项也”。强上,即项强。冥视,谓视物不清。
  \item 以救俛仰:俛,同“俯”。俯仰有两种解释,一指患者呼吸困难。尤在泾《医学读书记》说:“肺主气而司呼吸,风热在肺,其液必结,其气必壅,是以俯仰皆不顺利,故曰当救俯仰也。救俯仰者,即利肺气、散邪气之谓乎。”二指项背强急,如高世栻《素问直解》注:“治之之法,当调和经脉以治俯仰,经脉调和,则俯仰自如,强上可愈。”两说可并存。
  \item 巨阳引:在足太阳经上取穴针刺,以引动经气的治疗方法。
  \item 精者三日,中年者五日,不精者七日:精者,谓精气旺盛之人。此谓年轻力壮,精气充沛者,病愈快;中、老年人,精气渐衰,转愈日数延长。三日、五日、七日乃指病情缓解的大约日数。
\end{jiaozhu}

\biaoti{【理论阐释】}

\xiaobt{劳风的病因病机治疗及预后}

本病的病因病机是因劳受风,表邪未解,又入里化热,致使肺失清肃,痰热壅滞。因里热不除,俯仰不能,甚者会出现痰阻气道之危候。太阳表邪入里,但仍有部分表邪不散,故不但强上冥视,恶风振寒不解,还可致在表之邪再度入里,化热伤肺。治疗时既要宣肺利气,排除痰液,通畅气道,“以救俛仰”,同时又要祛散表邪,通利经气,亦即“巨阳引”,两方面同时并举,使内外邪气俱解,这是热病变证表里双解的典型范例。这种全方位认识疾病情况并在治疗时采取多方兼治的思路,给后世医家以启示,从而为诸多颇具成效的表里双解方剂的创立奠定了基础。

关于劳风的预后,本段提出两点,第一与人体精气盛衰,年龄大小有关。年轻气血旺盛体质强壮者,抗邪有力,邪气容易祛除,故病易愈,病程短,预后好;年老气血不足体质校差者,抗病力弱,邪易乘虚内陷,故病难治,病程长,预后不良。等二与能否及时排除痰液有关,痰出邪去则正安,否则,痰阻气道,蕴结为脓,伤肺而死。说明对危重病人不仅要及时正确地治疗,更要注重随时观察病情。

\biaoti{【临证指要】}

本文所述劳风病因、病位、临床表现及预后与张仲景《金匮要略·肺痿肺痈咳嗽上气病脉证治第七》论述“肺痈”一病颇为相似。该篇云“风舍于肺,其人则咳,口干喘满,咽燥不渴,多唾浊沫,时时振寒,热之所过,血为之凝滞,蓄结痈脓,吐如米粥,始萌可救、脓成则死。”在治疗上张仲景结合临床实际,创桔梗汤解毒排脓治“咳面胸满,振寒脉数,咽干不渴,时出浊唾腥臭,久久吐脓如米粥者”。对痰闭气阻,热毒壅滞之“肺痈,咳逆上气,喘鸣迫塞,葶苈大枣泻肺汤主之。”这些对临床均有较大的指导意义。

\biaoti{【原文】}

\begin{yuanwen}
帝曰:有病腎風\sb{1}者,面胕痝然壅\sb{2},害于言\sb{3},可刺不?岐伯曰:虛不當刺,不當刺而刺,后五日,其氣必至\sb{4}。帝曰:其至何如?岐伯曰:至必少氣時熱,時熱從胸背上至頭,汗出手熱,口乾苦渴,小便黃,目下腫,腹中鳴,身重難以行,月事不來,煩面不能食,不能正偃\sb{5},正偃則咳,病名曰風水\sb{6},論在《刺法》\sb{7}中。

帝曰:願聞其說。岐伯曰:邪之所湊,其氣必虛。陰虛者陽必湊之,故少氣時熱而汗出也。小便黃者,少腹中有熱也。不能正偃者,胃中不和也。正偃則咳甚,上迫肺也。諸有水氣者,微腫先見於目下也。帝曰:何以言?岐伯曰:水者,陰也;目下,亦陰也;腹者,至陰之所居,故水在腹者,必使目下腫也。眞氣上逆,故口苦舌乾\sb{8},臥不得正偃,正偃則咳出清水也。

諸水病者,故不得臥,臥則驚,驚則咳甚也。腹中鳴者,病本於胃也。薄脾\sb{9}則煩不能食。食不下者,胃脘隔也。身重難以行者,胃脈在足也。月事不來者,胞脈閉也\sb{10},胞脈者,屬心而絡於胞中,今氣上迫肺,心氣不得下通,故月事不來也。帝曰:善。
\end{yuanwen}

\biaoti{【校注】}

\begin{jiaozhu}
  \item 肾风:谓风邪客肾,主水之功能失常所致面目浮肿,妨害语言的一种疾病。
  \item 面胕痝然壅:胕,同“浮”。王冰注:“痝然,肿起貌。壅,谓目下壅,如卧蚕形也。”
  \item 害于言:指妨碍语言。王冰注;“肾之脉,从肾上贯肝膈,入肺中,循喉咙,侠舌本,故妨害于语言。”
  \item 不当刺而刺,后五日,其气必至:气,病气。至,考《易·坤卦》:“至哉坤元”注:“至,谓至极也。”极有“甚”义。故至指病情加重。句意为:肾之精气不足,风邪侵袭而成肾风,肾风虚不当刺,不当刺而刺,则真气愈虚,脏气五日一周,复归于肾,邪气必随之入肾,引起严重的变证。
  \item 正偃;偃,倒下。正偃,即仰卧,平卧。
  \item 风水:指由肾风误刺而引起的比肾风严重的水肿病。
  \item 《刺法》:张介宾注:“即《水热穴论》也。”
  \item 真气上逆,故口苦舌干:真气,指心脏之真气。心属火,其气上逆,所以口苦舌干,张志聪注:“真气者,脏真之心气也。心属火而恶水邪,水气上乘,则迫其心气上逆,是以口苦舌干。”
  \item 薄脾:薄,迫。薄脾,即犯脾。
  \item 月事不来者,胞脉闭也:胞脉,即胞宫之络脉。高世栻注“胞脉主冲任之血,月事不来者,乃胞脉闭也。中焦取汁,奉心化赤,血归胞中。故胞脉者,属心而络于胞中。今水气上迫肺,心气不得下通,故月事不来也。”
\end{jiaozhu}

\biaoti{【理论阐释】}

\xiaobt{“邪之所凑,其气必虚”的临床意义}

本节经文,意在阐发风水病肾阴不足,水不制火,而发“时热”的机理。就本篇而言,意在说明:精不胜邪,阴阳交争不解的“阴阳交”;少阴之气虚于内,风热之邪胜于外的“风厥”;劳伤肺肾,复受风邪的“劳风”;不当刺而刺,损伤正气,阴虚者阳必凑之的“肾风”,均由正不胜邪所致。但它的真正意义尚不止于此,它可以说明疾病发生的共同特征,从而阐明一个重要的发病学观点:正气不足是发病的内在根据,在邪正斗争胜负中,正气旺盛与否是决定发病与不发病的关键。这一观点在《内经》多篇中反复强调,《素问·遗篇·刺法论》说:“正气存内,邪不可干”。《灵枢·百病始生》篇说:“风雨寒热,不得虚,邪不能独伤人。此必因虚邪之风,与其身形,相虚相得,乃客其形”,从而突出了正气在发病中的主导地位。既然正气的强弱可决定发病与不发病,那么保护正气显得尤为重要,而要保护正气提高机体的抗病能力,就当注重养生,其具体方法在《素问·上古天真论》、《素问·四气调神大论》、《素问·阴阳应象大论》等均有详细论述。因此说,“邪之所凑,其气必虚”,不仅是中医发病学中的重要观点之一,也对养生学有一定贡献。

\biaoti{【临证指要】}

\xiaobt{肾风、风水的治疗}

《内经》虽对肾风、风水病因病机临床表现论述较多,但未及此两病的具体治法,因二者均有水肿一症,当按水肿病进行治疗。关于水肿病的治疗,《素问·水热穴论》和《素问·骨空论》均提出针刺“水俞五十七处”。《素问·汤液醪醴论》提出:“开鬼门”、“洁净府”、“去宛陈莝”等从脏腑气血调整的行之有效的治疗方法,历代医家本着《内经》的理论,在治法上不断发展,张仲景因势利导,就近祛邪,在《金匮要略·水气病脉证并治第十四》指出:“诸有水者,腰以下肿,当利小便;腰以上肿,当发汗乃愈”。利小便,常用肾气丸、防己茯苓汤之类;发汗,常用越婢汤、大小青龙汤之类,利小便兼发汗,则用五苓散之类。从临床方面有较大的指导意义。

\section{素問·咳論}%第三節

\biaoti{【原文】}

\begin{yuanwen}
黃帝問曰:肺之令人咳,何也?岐伯對曰:五藏六府皆令人咳,非獨肺也。帝曰:願聞其狀。岐伯曰:皮毛者肺之合也,皮毛先受邪氣,邪氣以從其合也。其寒飲食入胃,從肺脈上至於肺\sb{1},則肺寒,肺寒則外內合邪,因而客之,則為肺咳。五藏各以其時受病\sb{2},非其時,各傳以與之\sb{3}。人與天地相參,故五藏各以治時\sb{4},感於寒則受病,微則為咳,甚者為泄、為痛。乘\sb{5}秋則肺先受邪,乘春則肝先受之,乘夏則心先受之,乘至陰則脾先受之,乘冬則腎先受之。

帝曰:何以異之?岐伯曰:肺咳之狀,咳而喘息有音,甚則唾血。心咳之狀,咳則心痛,喉中介介如梗狀\sb{6},甚則咽腫喉痹。肝咳之狀,咳則兩脇下痛,甚則不可以轉,轉則兩胠\sb{7}下滿。脾咳之狀,咳則右脇下痛陰陰引肩背,甚則不可以動,動則咳劇\sb{8}。腎咳之狀,咳則腰背相引而痛,甚則咳涎。

帝曰:六府之咳奈何?安所受病?岐伯曰:五藏之久咳,乃移於六府。脾咳不已,則胃受之,胃咳之狀,咳而嘔,嘔甚則長蟲\sb{9}出。肝咳不已,則膽受之,膽咳之狀,味嘔膽汁。肺咳不已,則大腸受之,大腸咳狀,咳而遺失\sb{10}。心咳不已,則小腸受之,小腸咳狀,咳而失氣,氣舆咳俱失。腎咳不已,則勝胱受之,膀胱咳狀,咳而遺溺。久咳不已,則三焦受之,三焦咳狀,咳而腹滿,不欲食飲\sb{11}。此皆聚於胃,關於肺,使人多涕唾而面浮腫气逆也\sb{12}。

帝曰:治之奈何?岐伯曰:治藏者治其俞\sb{13},治府者治其合\sb{13},浮腫者治其經\sb{13}。帝曰:善。
\end{yuanwen}

\biaoti{【校注】}

\begin{jiaozhu}
  \item 其寒饮食入胃,从肺脉上至于肺:杨上善《太素·咳论》注:“人肺脉手太阴,起于中焦,下络大肠,还循胃口,上膈属肺。寒饮寒食入胃,寒气循肺脉上入肺中。”
  \item 五脏各以其时受病:其时,指五脏主旺的季节,谓五脏各在其所主的时令感受病邪而发病。
  \item 非其时,各传以与之:非其时,指非肺所主之秋季。之,指肺,张志聪《素问集注》注:“如非其秋时,则五脏之邪,各传与之肺而为咳也。”
  \item 治时:治,主也。治时,即所主的时令。
  \item 乘:趁着,凭借。
  \item 介介如梗状:介,通“芥”,小草、杂草。形容咽部不舒如有物梗阻。
  \item 胠:指腋下胁上的部位。
  \item 咳则右胁下痛阴阴引肩背,……动则咳剧:阴阴,即隐隐。脾肺为母子关系,又同为太阴经,经气相通,脾病及肺,致肺气逆滞而咳。姚止庵《素问经注节解》注:“脾气连肺,故痛引肩背也。按右者肺治之部,肺主气,脾者气之母,脾病则及于肺,故令右胁下痛。肩背者,肺所主也,动则气愈逆,故咳剧。”
  \item 长虫:指蛔虫。《说文·虫部》:“蛕,腹中长虫也”。蛕,是“蛔”的异体字。
  \item 遗失:失,皇甫谧《甲乙经》及杨上善《太素·咳论》均作“矢”,可从。遗矢,即大便失禁。
  \item 久咳不已,……不欲食饮:久咳,指上述各种咳嗽而言。三焦总司一身之气化,故久咳不已,皆可传于三焦。姚止庵《素问经注节解》注:“此总论久咳之为害也。囊括一身,以气为用者也。所以咳在三焦,则气壅闭而不行,故令腹满而不思饮食。”
  \item 此皆聚于胃,关于肺,使人多涕唾而面浮肿气逆也:张介宾《类经·疾病类》注:“诸咳皆聚于胃,关于肺者,以胃为五脏六腑之本,肺为皮毛之合,如上文所云皮毛先受邪气及寒饮食入胃者,皆肺胃之候也,阳明之脉起于鼻,会于面,出于口,故使人多涕唾而面浮胂。肺为脏腑之盖而主气,故令人咳而气逆。”
  \item 俞、合、经:指五输穴中的输穴、合穴、经穴。如脾俞太白、胃合三里、三焦经支沟等。
\end{jiaozhu}

\biaoti{【理论阐释】}

1.咳的病因病机

本篇认为咳嗽的病因为风寒之邪,其受邪途径一从皮毛而入,二伤生冷饮食。风寒外袭,“皮毛先受邪气”,而“皮毛者,肺之合也”邪气由表及里内舍其合以伤肺;胃伤生冷,其寒“从肺脉上至于肺,则肺寒”。肺为娇脏,不耐寒热,“形寒寒饮则伤肺,以其两寒相感,中外皆伤,故气逆而上行”(《灵枢·邪气脏腑病形》)发为咳嗽。十分明确地提出了“内外合邪”的致病观和肺胃是成咳之源的病机,为咳病的治疗和预防提供了理论依据。

虽本篇认为风寒客肺是导致咳嗽之主因,但在《内经》其他篇中,尚有湿、热、火,燥诸邪外袭皆可致咳之记载。《素问·生气通天论》的:“秋伤于湿,上逆而咳”是伤于湿;《素问·气交变大论》的:“岁火太过,炎暑流行,肺金受邪。民病疟,少气咳喘”是伤于暑;《素问·至真要大论》的“少阴司天,热淫所胜……寒热咳喘”是伤于热;“少阴司天,火淫所胜……疮疡咳唾血”,是伤于火;“阳明司天,燥淫所胜,……咳”,是伤于燥,说明六淫邪气皆能致咳,并非仅见风寒之邪。后世将咳嗽的病因分为两大类,一为外感,二为内伤,外感由六淫邪气所致,内伤由脏腑功能失调所致,其中外感中确以风寒之邪为多见,内伤中由饮食所伤,中焦失运,痰湿蕴肺为多见,说明本篇提出的外感寒邪,内伤生冷影响肺胃是导致咳病两大因素的观点是值得重视的。

2.咳病与感邪轻重的关系

咳病的发生不但与感邪性质有关,而且与感邪轻重有关。邪轻病轻,邪重病重,“感于寒则受病,微则为咳,甚则为泄、为痛。”微即感邪较轻,病在肺脏,仅为咳嗽,甚即感邪较重,邪犯部位深在,他脏受邪波及于肺或肺脏有病影响他脏,症状复杂。如张介宾《类经·疾病类》曰:“邪微者,浅在表,故为咳。甚者深而入里,故为泄为痛。”这是《内经》发病学的又一重要观点。指出了五脏咳均有疼痛症状,六府咳均有外泄症状,使痛、泄成为五脏六腑咳的辨证要点。

3.咳与脏腑的关系

(1)咳为肺之本病

本篇首先肯定了“肺之令人咳”。《灵枢·九针论》明确指出:“肺主咳。”《素问·宣明五气》篇亦曰:“肺为咳。”均强调咳病病位主要在肺。《内经》其他篇章对肺病表现的论述,大多涉及咳,如《素问·刺热论》曰“肺热病者,……热争则喘咳。”《灵枢·胀论》曰:“肺胀者,虚满而喘咳。”《灵枢·五邪》曰:“邪在肺,则病皮肤痛,寒热,上气喘,汗出,咳动肩背。”可见,咳是肺病之主症。肺有主气司呼吸,主宣发肃降之功能,所以纵然外感内伤均可致咳,但其根本机制是肺在致病因素作用下,宣降失常,肺气上逆而引起。故陈修园《医学三字经》指出:“诸气上逆于肺,则呛而咳,是咳嗽不止于肺,而亦不离于肺。”

(2)五脏六腑皆令人咳

虽咳为肺之本病,但人是一个有机的整体,一旦得病,他脏病可影响本脏,本脏病可传至他脏。“五脏各以其时受病,非其时,各传以与之”,说明咳病并非只见肺脏疾患,它脏有疾累及于肺,均能导致肺失宣降发为咳嗽。临证诸如脾虚生湿,湿痰蕴肺;肝火上冲,气逆犯肺;肾虚水犯,水寒射肺;肾阴亏虚,子盗母气;胃寒停饮,饮邪迫肺等。张志聪《黄帝内经素问集注》曰:“肺主气而位居尊高,受百脉朝会,是咳虽肺证,而五脏六腑之邪皆能上归于肺而为咳”。此外,肺脏有病久咳不愈又可并发他脏疾患,如本篇所称“五脏之久咳,乃移于六腑”,“脾咳不止,则胃受之…肝咳不已,则胆受之”。经文的“五脏六腑皆令人咳,非独肺也”的著名观点,成为后性医家论咳的准则。

(3)咳与肺胃关系密切

虽“五脏六腑皆令人咳”,而以肺胃两脏关系最为密切,肺与咳的关系前已论及,胃与咳的关系可从以下几方面来分析:其一,肺之经脉“起于中焦,下络大肠,还循胃口”(《灵枢·经脉》),肺胃同有主降之特性,所以胃受外邪或接受其他脏腑内传而聚于胃之邪气,均可使胃失和降并可通过肺脉使邪气上传于肺,使肺气不降而发为咳嗽。其二,胃为五脏六腑之海,与脾同居中焦,为气血化生之源。若脾胃运化失司,气血化生乏源,一方面可导致土不生金,使肺之气阴不足,宣降失常而病咳;另一方面,由于营卫之气不充,卫外御邪能力减弱,则易使外邪侵犯皮毛,内舍于肺而发为咳嗽。其三,胃主纳,脾主运,若脾胃受伤,水津失运,停聚而生痰成饮,痰饮上逆于肺,亦可生为咳嗽。因此,陈修园《医学三字经》说:“《内经》虽分五脏诸咳,而所尤重者,在‘聚于胃关于肺’六字。”其歌诀谓:“气上呛,咳嗽生,肺最重,胃非轻。”

\biaoti{【临证指要】}

1.脏腑咳的辨治

本篇提出的咳病针刺治疗原则,虽然简单,但对咳病的辨证论治有较大的指导意义。由于“五脏六腑皆令人咳,非独肺也”,故在诊断时应既重视主证又照顾兼证,首先辨明咳之在脏在腑,然后采取相应的治疗措施,以协调脏腑间关系的平衡为目的,后世医家在《内经》针刺治疗基础上,创制了不少颇具成效的方药。如肺咳,王肯堂《证治准绳·杂病诸气门》提出用麻黄汤。心咳,用桔梗汤。肝咳,用小柴胡汤。脾咳用升麻汤。肾咳,用麻黄附子细辛汤。

关于六腑咳,王肯堂《证治准绳·杂门诸气门》提出:“胃咳用乌梅丸,胆咳用黄芩加半夏生姜汤,大肠咳用赤石脂禹余粮汤、桃花汤,小肠咳用芍药甘草汤,膀胱咳用茯苓甘草汤,三焦咳用钱氏异功散”。秦伯未《内经类证·咳嗽病类》则指出:“咳时小便不禁,用五苓散加党参,咳时频转矢气,且欲大便用补中益气汤加麦冬、五味子。”

2.咳与季节气候的关系

五脏各以治时感邪发病,是《内经》四时五脏发病的基本观点。五脏在其旺时主持一身,其气亦布于一身,当其时邪气侵入人体对,邪气首先与人身主时之脏气相接触,使该脏受伤而发病。故本篇从“人与天地相参”的整体观出发,提出“五脏各以其时受病,非其时各传以与之”的发病观,这一观点对咳病的治疗有一定的提示,清代医家林佩琴根据四时生长收藏,阴阳升降之理,指出不同季节治疗咳病的用药规律:“以四对论之,春季咳,木气升也,治宜兼降,前胡、杏仁、海浮石、瓜蒌仁之属;夏季咳,火气炎也,治宜兼凉,沙参、花粉、麦冬、知母、玄参之属;秋季咳,燥气乘金也,治宜清润,玉竹、贝母、杏仁、阿胶、百合、枇杷膏之属;冬季咳,风寒侵肺也,治宜温散,苏叶、川芎、桂枝、麻黄之属”(《类征治裁》)。很好地发挥了本篇季节与咳病关系的理论。

3.咳之“聚于胃,关于肺”对治疗与预防的启示

经文在“此皆聚于胃,关于肺”之后,还有“使人多涕唾而面浮肿气逆”的证候描述。临床咳病日久,每见浮肿之象,此为外寒内饮之邪气壅闭肺胃所致,与《金匮要略·痰饮咳嗽病脉证并治第十二》所述“咳逆倚息,短气不得卧,其形如肿,”的支饮颇为一致,张仲景在治疗支饮的方剂中,如小半夏汤、小半夏加茯苓汤、厚朴大黄汤、泽泻汤、葶苈大枣泻肺汤、小青龙汤等,亦无不从肺胃着手。再观临床由肺胃所致咳嗽是最常见的咳嗽,除上述治饮之方外,清燥救肺汤、麦门冬汤、沙参麦冬汤等,也都是咳病治在肺胃的常用方剂。可见本篇“此皆聚于胃,关于肺”这一咳病辨治纲领的提出,确为后世对咳病的治疗,起到了执简驭繁的作用。

“聚于胃,关于肺”还为咳病预防提供了理论依据,咳病由“外内合邪”所致,故应外避虚邪贼风,内调饮食,忌食生冷寒冻,从而减少咳病的发生。

\section{素問·舉痛論(節選)}%第四節

\biaoti{【原文】}

\begin{yuanwen}
黃帝問曰:余聞善言天\sb{1}者,必有驗\sb{2}於人;善言古者,必有合於今;善言人者,必有厭\sb{3}於己。如此,則道不惑而要數極\sb{4},所謂明也。今余間於夫子,令言\sb{5}而可知,視\sb{5}而可見,捫\sb{5}而可得,令驗於己,而發蒙解惑,可得而聞乎?岐伯再拜稽首對曰:何道之問也?帝曰:願聞人之五藏卒痛,何氣使然?歧伯對曰:經脈流行不止,環周不休。寒氣入經而稽遲,泣而不行,客於脈外則血少,客於脈中則气不通\sb{6},故卒然而痛。

帝曰:其痛或卒然而止者,或痛甚不休者,或痛甚不可按者,或按之而痛止者,或按之無益者,或喘動應手\sb{7}者,或心與背相引而痛者,或脇肋舆少腹相引而痛者,或腹痛引陰股\sb{8}者,或痛宿昔\sb{9}而成積者,或卒然痛死不知人有少間復生者,或痛而嘔者,或腹痛而後泄者,或痛而閉不通者,凡此諸痛,各不同形,別之奈何?

岐伯曰:寒氣客於脈外則脈寒,脈寒則縮踡,縮踡則脈絀急\sb{10},絀急則外引小絡,故卒然而痛,得炅\sb{11}則痛立止;因重中於寒,則痛久矣。寒氣客於經脈之中,與炅气相薄則脈滿,滿則痛而不可按也。寒氣稽留,炅气從上\sb{12},則脈充大而血氣亂,故痛甚不可按也。寒氣客於腸胃之間,膜原\sb{13}之下,血不得散,小鉻急引故痛。按之則血氣散,故按之痛止。寒氣客於俠脊之脈\sb{14},則深按之不能及,故按之無益也。寒氣客於衝脈,衝脈起於關元\sb{15},隨腹直上,寒氣客則脈不通,脈不通則氣因之,故喘動應手矣。寒氣客於背俞之脈\sb{16}則脈泣,脈泣則血虚,血虚則痛,其俞注於心,故相引而痛。按之則熱氣至,熱氣至則痛止矣。寒氣客於厥陰之脈,厥陰之脈者,絡陰器,繫於肝,寒氣客於脈中則血泣脈急,故脇肋與少腹相引痛矣。厥氣\sb{17}客於陰股,寒氣上及少腹,血泣在下相引,故腹痛引陰股。寒氣客於小腸膜原之間,絡血之中,血泣不得注於大經,血氣稽留不得行,故宿昔而成積矣。寒氣客於五藏,厥逆上泄,陰氣竭,陽氣未入\sb{18},故卒然痛死不知人,氣復反則生矣。寒氣客於腸胃,厥逆上出,故痛而嘔也。寒氣客於小腸,小腸不得成聚,故後泄腹痛矣。熱氣留於小腸,腸中痛,癉熱\sb{19}焦渴則堅乾不得出,故痛而閉不通矣。

帝曰:所謂言而可知者也,視而可見,奈何?岐伯曰:五藏六府,固盡有部\sb{20},視其五色,黃赤為熱,白為寒,青黑為痛,此所謂視而可見者也。帝曰:捫而可得,奈何?岐伯曰:視其主病之脈,堅而血及陷下者\sb{21},皆可捫而得也。帝曰:善。
\end{yuanwen}

\biaoti{【校注】}

\begin{jiaozhu}
  \item 天:指天地阴阳自然之理。
  \item 验:检验、验证。
  \item 厌:《说文解字·厂部》:厌,“一曰合也”。与上文“验”、“合”之义相通,有参照之意。
  \item 道不惑而要数极:道,道理,规律。数,理也。要数,即要理,大理。杨上善《太素·邪客》注:“得其要理之极,明达故也。”意为对道理认识明确,对事物变化的规律掌握十分透彻。
  \item 言、视、扪:“言”即问诊;“视”即望诊;“扪”即切诊、按诊。
  \item 客于脉外则血少,客于脉中则气不通:此属互文句式。脉内与脉外、气与血互补。
  \item 喘动应手:喘,亦动之义。谓血脉搏动按之急促应手。
  \item 阴股:指大腿内侧。
  \item 宿昔:同义复词,经久的意思。
  \item 绌急:屈曲拘急。
  \item 炅:热也。
  \item 从上:上,疑为“之”字之误。
  \item 膜原:张介宾《类经·疾病类》注:“膜,筋膜也。原,育之原也。”又在《痹论》注中云:“育者,凡腔腹肉理之间,上下空隙之处,皆谓之肓。”又在《瘘论》注中云:“盖膜犹幕也,凡肉理脏腑之间,其成片联络薄筋,皆谓之膜,所以屏障血气者也。凡筋膜所在之处,脉络必分,血气必聚,故又谓之膜原,亦谓之脂膜。”
  \item 侠脊之脉:指脊柱两旁深部之经脉。
  \item 冲脉起于关元:关元,任脉穴,在脐下三寸。马莳注:“按《骨空论》云:冲脉起于气冲,今曰关元者,盖任脉当脐中而上行,则本起于气冲而与任脉并行,故谓之起于关元亦可也。”
  \item 背俞之脉:指足太阳经脉,其行于背部的部分有五脏六腑之俞穴,故名之。
  \item 厥气:张介宾《类经·疾病矣》注:“寒逆之气也。”按前后文例,“厥气”似应与下句“寒气”互易,成“寒气客于阴股,厥气上及少腹”,于理较顺。
  \item 厥逆上泄,阴气竭,阳气未入:泄,即向上泄越。竭,音义均同“遏”,即遏止,阻隔不通。寒气客于五脏,阴气阻绝于内,阳气泄越于外,阴阳之气不相顺接,暂处离绝状态,故卒然痛死不知人。
  \item 瘅热:热甚也。
  \item 五脏六腑,固尽有部:固,明抄本作“面”,可从。张志聪注:“五脏六腑之气色,皆见于面,而各有所主之部位。”
  \item 坚而血及陷下者:此指局部按诊。按之坚硬,局部血脉壅盛,属实;按之陷下濡软,为虚。
\end{jiaozhu}

\biaoti{【理论阐释】}

1.疼痛的病因病机

疼痛是临床常见的病症之一,《内经》认为引起疼痛的原因很多,如有六淫外袭、瘀血、虫积等。然其主要因素,则为寒邪。《素问·痹论》曰:“痛者寒气多也,有寒故痛也。”本篇言:“寒气入经而稽迟,泣而不行……故卒然而痛”。所列举的诸种疼痛惟“热气留于小肠”主乎热,其余均由寒邪所致。寒邪之所以引起疼痛,主要病机为寒性凝滞收引,使血脉缩踡绌急,气血滞涩不畅。所谓寒,包括外寒如冻伤,感受寒邪,饮食生冷等,也包括内寒,即机体阳气虚衰,温煦气化功能减弱,出现颤抖、畏寒肢冷、肤色苍白、排泄物澄澈清冷等。

痛病的病机本篇概括为“客于脉外则血少,客于脉中则气不通”此两句原文相互补充,即邪气侵犯经脉内外,既可导致气血不通,亦可导致气血衰少,二者均可引发疼痛,换言之,疼痛的病机无外乎虚实两端。结合《内经》其他篇章,又可分为以下几种:其一,不通作痛。本篇云:“寒气客则脉不通,脉不通则气因之,故喘动应手矣。”又曰:“热气留于小肠,肠中痛,瘅热焦渴,则坚干不得出,故痛而闭不通矣。”不通则痛,包括气机不通、气血不通、腑气不通等,其原因有寒邪凝滞所致,也可因燥屎内结、食积、结石等实物堵塞所为。其二,脉络拘急收引作痛。原文曰:“寒气客于脉外则脉寒,脉寒则缩踡,缩踡则脉绌急,绌急则外引小络,故卒然而痛。”此种疼痛亦常表现出牵引疼痛,如寒客厥阴之脉,“血泣脉急”,可致胁肋与少腹相引而痛。其三,失养作痛。气虚、血虚、阴精亏少或因血脉不畅,脏腑组织不能得到充足的气血濡养,而致不荣作痛。如本篇言寒邪客于背俞之脉,“脉泣血虚”而致心与背相引而痛;《灵枢·五癃津液别》则指出:“髓液皆减而下,下过度则虚,虚故腰背痛而胫酸。”其四,气逆作痛。本篇曰:“寒气稽留……炅气从上(之),则脉充大而血气乱,故痛甚不可按也。”《素问·骨空论》则指出:“督脉者……此生病从少腹上冲心而痛,不得前后,为冲疝。”它如肝气上逆、肝气横逆、冲脉气逆等,均可致痛。此外,《灵枢·周痹》还提出肌肉分裂作痛,其文曰:“风寒湿气客于外分肉之间,迫切而为沫,沫得寒则聚,聚则排分肉而分裂也,分裂则痛。”

2.疼痛的诊断要点

疼痛的临床表现多种多样,《内经》所论疼痛主要有头痛、心胸痛、胁痛、胃脘痛、腹痛、腰痛等,然本篇所论十四种疼痛,皆属胸腹腔内脏腑疼痛。其疼痛诊断与辨证的要点是:

(1)疼痛的时间特点与程度:疼痛的时间特点与程度,常可反映病情的轻重。如“其痛或卒然而止者”,“按之痛止者”,其疼痛较轻,痛有休止,说明病情轻浅;若“痛甚不休”,“因重中于寒,则痛久也”,说明病情较重;若“卒然痛死不知人,气复反则生”,说明阴阳有暂时离绝之势,病情更甚。

(2)疼痛对按压的反应:疼痛对按压的反应,常可反映疾病的虚实,病位的浅深。痛而拒按者,是寒热搏结,按之使气血更加逆乱。按之痛不减者,因寒客部位深在,按之不及病所。按之痛减者,一是使原本凝聚之气血,得以疏散,二是使壅闭之阳气,得以通达。按之搏动应手者,是邪客深部冲脉使血滯而上逆之故。

(3)疼痛的牵引部位:根据经脉的循行路线及与脏俯关系,牵引性疼痛所发生的部位,常可作为确定病位的依据。如心与背相引而痛,是寒客背俞之脉。胁肋与少腹相引而痛及少腹痛引阴股,是寒伤厥阴之脉,因肝足厥阴之脉“循股阴入毛中,过阴器,抵小腹,挟胃属肝络胆,上贯膈,布胁肋。”

(4)疼痛的寒热属性:一般而言,疼痛喜温、喜按,“得炅则痛立止”者,为寒证,常伴有面白、身寒肢冷、舌淡脉迟等一系列寒象;疼痛喜冷、拒按,得凉缓解者,为热证,常伴有面赤、身热、口渴、尿赤、舌红脉数等一系列热象。

(5)疼痛的兼症辨证:疼痛的兼症,亦是辨别疼痛病位及其寒热虚实的重要依据。痛兼积聚,乃邪客小肠膜原之间,寒凝血滞之故;痛伴昏厥,是寒邪伤脏,阴阳之气不相交通所为;痛兼呕吐,为寒犯肠胃,失于和降;痛兼腹泻,为寒犯小肠,清浊不分,水谷混杂,并走大肠;痛兼便秘,乃热灼肠液,津伤化燥所致。

\biaoti{【临证指要】}

\xiaobt{关于“客于脉外则血少,客于脉中则气不通”}

该原文概括了疼痛属实属虚的两种病机,具有纲领性意义,它为疼痛之分证与治法奠定了坚实的理论基础,使得后世医家对疼痛的认识,有章可循,有法可遵,张介宾深得经旨,批评了后世治痛只泻不补的错误倾向,在《类经·疾病类·六十六》中说:“后世治痛之法,有曰痛无补法者,有曰通则不痛,痛则不通者,有曰痛随利减者,人相传诵,皆以此为不易之法,凡是痛证无不执而用之。……然痛证亦有虚实,治法亦有补泻,其辨之之法,不可不详。”并根据自己的临床经验指出辨痛虚实之方法,曰:“凡痛而胀闭者多实,不胀不闭者多虚。痛而拒按者为实,可按者为虚。喜寒者多实,爱热者多虚。饱而甚者多实,饥而甚者多虚。新病壮年者多实,愈攻愈剧者多虚。”在治疗上提出:“故凡表虚而痛者,阳不足也,非温经不可;里虚而痛者,阴不足也,非养营不可;上虚而痛者,心脾受伤也,非补中不可;下虚而痛者;脱泄亡阴也,非速救脾肾、温补命门不可。”可谓对《内经》疼痛虚实病机和诊治理、法的发挥与运用。

\zuozhe{(段延萍)}

\section{素問·風論}%第五節

\biaoti{【原文】}

\begin{yuanwen}
黃帝問曰:風之傷人也,或爲寒熱,或爲熱中\sb{1},或爲寒中\sb{2},或爲癘風\sb{3},或爲偏枯\sb{4},或爲風也\sb{5},其病各異,其名不同,或內至五藏六府,不知其解,願聞其說。岐伯對曰:風氣藏於皮膚之間,內不得通,外不得泄。風者善行而數變\sb{6},腠理開則洒然寒,\sb{7},閉則熱而悶,其寒也則衰食飲;其熱也則消肌肉,故使人怢慄\sb{8}而不能食,名曰寒熱。風氣與陽明入胃,循脈而上至目內眥。其人肥,則風氣不得外泄,則爲熟中而目黃;人瘦則外泄而寒,則爲寒中而泣出。風氣與太陽俱入,行諸脈俞,散於分肉之間,與衛氣相干,其道不利,故使肌肉憤䐜而有瘍,衛氣有所凝而不行,故其肉有不仁也。癘者,有榮氣熱胕,其氣不清,故使其鼻柱壞而色敗,皮膚瘍潰。風寒客於脤而不去,名曰癘風,或名曰寒熱。

以春甲乙傷於風者爲肝風\sb{9},以夏丙丁傷於風者爲心風,以季夏戊己傷於邪者爲脾風,以秋庚辛中於邪者爲肺風,以冬壬癸中於邪者爲腎風。風中五藏六府之俞,亦爲藏府之風,各人其門戶,所中則爲偏風\sb{10}。風氣循風府而上,則爲腦風\sb{11}。風入系頭\sb{12},則爲目風、眼寒。飲酒中風,則爲漏風\sb{13}。入房汗出中風,則爲內風\sb{14}。新沐中風,則爲首風。久風人中,則爲腸風、飧泄。外在腠理,則爲泄風。故風者,百病之長也。至其變化,乃爲他病也,無常方,然致有風氣也。

帝曰:五藏風之形狀不同者何?願聞其診及其病能。岐伯曰:肺風之狀,多汗惡風,色皏然白,時咳短氣,晝日則差,暮則甚,診在眉上,其色白。心風之狀。多汗惡風,焦絕\sb{15},善怒嚇,赤色,病甚則言不可快,診在口,其色赤。肝風之狀,多汗惡風,善悲,色微蒼,嗌乾善怒,時憎女子,診在目下,其色青。脾風之狀,多汗惡風,身體怠墯,四支不欲動,色薄微黃,不嗜食,診在鼻上,其色黄。腎風之狀,多汗惡風,面痝然浮腫,脊痛不能正立,其色炲,隱曲不利\sb{16},診在肌上\sb{17},其色黑。胃風之狀,頸多汗,惡風,食飲不下,鬲塞不通,腹善滿,失衣則䐜脹,食寒則泄,診形瘦而腹大。首風之狀,頭面多汗惡風,當先風一日則病甚,頭痛不可以出內,至其風日,則病少愈。漏風之狀,或多汗,常不可單衣,食則汗出,甚則身汗,喘息惡風,衣常濡,口乾善渴,不能勞事。泄風之狀,多汗,汗出泄衣上,口中乾,上漬,其風不能勞事,身體盡痛則寒。帝曰:善!
\end{yuanwen}

\biaoti{【校注】}

\begin{jiaozhu}
  \item 热中:病名。是风邪入侵于胃,以目黄为主症的病。
  \item 寒中:病名。是风邪入侵,邪从寒化,以双目流泪为主症的病。
  \item 疠风:病名,又称“大风”、“癞风”、“大麻风”,即今之麻风病。
  \item 偏枯:病名。指一侧肢体瘫痪。
  \item 或为风也:指下文脑风、目风、漏风、内风、首风、肠风、泄风以及脏腑之风等多种风证而言。
  \item 风者善行而数变:此指风邪致病有病位游走不定、症状变化无常的特点。善行,指行动不居;数变,指变化多端。
  \item 洒(xiǎn音显)然寒:形容病人恶风怕冷的样子。洒然,寒冷貌。
  \item 怢(tū音突)慄:突然战慄。怢,不经意。
  \item 以春甲乙伤于风者为肝风:古代以干支纪年、月、日、时。五脏分属五行而主五季五时,春季与甲乙日属木,分属于肝,故春、甲乙日受邪入肝而名肝风。
  \item 偏风;有两解:一是与“偏枯”同,指风邪中于人体左侧或右侧俞穴而导致偏枯之症;二是指风邪偏中于人体某脏某部,均谓之偏风。
  \item 脑风;病名。指风犯脑髓,以脑部疼痛为主症的病。
  \item 系头:即目系。
  \item 漏风:指饮酒后感受风邪所致汗多如漏的病证。同《素问·病能论》之酒风。
  \item 内风:内,房内,指男女房事。入房汗出,气精两虚,风邪直中,故名内风。
  \item 焦绝:指唇舌焦燥。张介宾《类经·疾病类》注:“焦绝者,唇舌焦操,津液干绝也。”
  \item 隐曲不利:隐曲,指前阴为隐蔽委曲之处。隐曲不利,当包括性机能衰退及小便不利等病变。
  \item 诊在肌上:肌,《太素》作“颐”,可从。
\end{jiaozhu}

\biaoti{【理论阐释】}

\xiaobt{风邪性质和致病特征}

风为春季主气,五行属木,应脏于肝,正常不易使人生病。若气候变化异常,风气太过,可导致机体发病,呈现轻扬开泄、善行数变、为百病之长、主动的致病特点。其一,风为阳邪,易伤阳位。《素问·太阴阳明论》云:“阳受风气”,“伤于风者,上先受之”,“故犯贼风虚邪者,阳受之。”说明风邪伤人以头面、鼻咽、肺、体表皮毛等阳位为主。其二,风性开泄,易伤卫表。《素问·骨空论》云:“风从外入,令人振寒,汗出,头痛,身重,恶寒。”《灵枢·营卫生会》所述因风面毛蒸理泄的漏泄及本篇所述脏腑风、泄风、漏风,均见汗出恶风的症状,说明风袭肌表,内开腠理,营卫失和,表现出风阳开泄,使人腠理疏松,而有出汗、恶风的致病特点。其三,风性善行数变。风邪伤人,病位游走不定,病状变化无常,发病突然,病势急暴迅猛。故曰“风气胜者为行痹也”(《素问·痹论》;“邪风之至,疾如风雨”(《素问·阴阳应象大论》);“诸暴强直,皆属于风”(《素问·至真要大论》)。可见,风邪伤人,外面五体,内而五脏,无处不到,具有善动不居,易行而无定处的“善行”特征,发病急,变化多的“数变”特点,并因感风时间长短、体质因素差异而呈现各异病变。如《素问·生气通天论》所论风邪所伤有感而即发的寒热病等,也有伏而后发的洞泄。《灵枢·岁露论》云:“贼风邪气之中人也,不得以时,然必因其开也,其入深,其内极病,其病人也卒暴;因其闭也,其入浅以留,其病也徐以迟。”《灵枢·五变》云:“肉不坚,腠理疏,则善病凤”,“一时遇风,同时得病,其病各异”等,均说明皮肤开泄,卫气不固是遭致风邪内侵的重要因素。体质不同,受风之后,发病各异。“善行者,无处不到,数变者,证不一端。风之为邪,其厉矣哉”(《素问经注节解·三卷》)。其四,风性主动。“风胜则动”(《素问·阴阳应象大论》);“厥阴所至为挠动,为迎随”(《素问·六元正纪大论》);“诸风掉眩,皆属于肝”(《素问·至真要大论》)等论述,说明风邪常致头目、肢节动摇不定,其机理与“风气通于肝”、“风伤筋”密切相关。其五,风为百病之长。风邪致病极为广泛,常在不同时日,不同条件下侵袭机体,成为外感病邪之首。《素问·骨空论》、《素问·生气通天论》亦有“风者,百病之始也”之论。深究风为“百病之长”的机理,还可从风性开泄为据论之。风邪伤人之后,使人的腠理松弛,玄府不固,外邪易于入侵,加之卫气随着津液而外泄,抵御外邪之力受损,所以其他邪气便乘虚入侵。此可谓“百病之长”的道理之一。另外,在不同气候条件下,其他邪气常会依附于风而伤人,风便成为其他外邪入侵时的载体。可见,在诸种外邪之中,风邪是最易伤人之邪,故谓之为“百病之长”。

\biaoti{【临证指要】}

本节论述了外感风邪節所致的18种外感疾病。这18种因风邪所致疾病,可归纳为四类:

其一,以风邪所伤部位命名。此类风病有九:

(1)肝风

五脏风之一,是风邪犯肝所致,有多汗,恶风,面色青,咽干,善怒等症。姚止庵解其机理曰:“肝虽主风,然更感风,则内外相通而气大泄,故亦多汗恶风也。悲者,肺志,肝何以悲?盖肝病则木虚而受制于金,故善悲。苍者,木病之色也。肝脉循喉咙之后,入颃颡,风动火炽,故嗌干。肝志怒,木气急出。木急则气阳志亢,故不欲近女子也”(《素问经注节解·三卷》)。治宜用甘麦大枣汤加桑叶、荆芥疏风,白芍平肝敛阴。

(2)肺风

五脏风之一,是风邪犯肺,肺失宣降之故,常见汗出,恶风,咳嗽,气喘,且有昼轻夜甚的症状特点。治宜益肺气,祛风邪,可选玉屏风散加杏仁、桑叶治之。

(3)心风

五脏风之一,是风邪犯心,心阳偏亢,火热炽盛之故,常见多汗,恶风,唇干,善怒或惊吓,面赤,舌强等症。治宜清心泻热兼安心神为法。

(4)脾风

五脏风之一,是风邪犯脾,失于运化而成。常见多汗,恶风,身体倦怠,四肢乏力,面色淡黄,食欲不振等症。治以健脾祛邪,可用藿香正气散为主方。

(5)肾风

五脏风之一,是风邪入肾,肾脏功能受损所致。常见多汗,恶风,腰脊疼痛,阳痿,小便不利,色黑等症。张介宾解析曰:“风邪入肾,则挟水气上升,故面为浮肿。肾脉贯脊属肾,故令脊痛不能正立。肾主水,故色黑如炱。肾开窍于二阴,故为隐曲不利。肌肉本主于脾,今其风水合邪,反侮于土,故诊见肌上,色当黑也”。治以祛风行水之法,可用越婢汤为主方。

(6)胃风

胃风是素体中阳不足,风邪犯胃而成。姚止庵分析其病机时指出:“胃者,土也。风木乘之,则胄液外泄,胃脉诊于人迎,人迎在颈,故风入胃而颈独多汗恶风也。胃主饮食,其治上连膈,下连腹,故病则上下不通而满也。胃喜暖而恶寒,故䐜胀而泄也。凡若此者,土被木刑,肌肉消削,中气不运,形瘦腹大,是其诊也”。治疗可选理中汤为主方,加防风、白芍。

(7)肠风

肠风是风邪内犯肠道,致使肠道传导失常所致的病。多为感风日久,风邪内传所致。《内经》认为,小肠“主液”,大肠“主津”(《灵枢·经脉》)。风邪犯肠,津液不化,故本病以飧泄为主症。故原文说:“久风入中,则为肠风、飧泄。”治宜疏风止泻,地榆、防风、黄芩等可选用。

(8)首风(头风)

首风,又叫头风,是指头部受风,以头痛,头面多汗,恶风为症状的病。所以李中梓在阐述头痛用风药的机理时指出:“头痛自有多因,而古方每用风药者何也?高巅之上,惟风可到,味之薄者,阴中之阳,自地升天者也。在风寒湿者,固为正用;即虚与热者,亦假引经”(《医宗必读·头痛》)。

(9)脑风

脑风是指风邪循督脉入脑髓,以头晕微痛,巅顶痛,脑户多冷,项背怯寒,时流清涕为主症的病。姚止庵析之曰;“脑风者,风人于脑,触风则头晕微痛,时流清涕。”治疗宜以祛风益肾为法。

(10)目风(眼寒)

目风是指风邪入脑犯目系,以目痒、目痛、遇寒流泪为主症的病。治以温散风寒为主。

其二,以疾病的性质命名。此类风病有三:

(1)寒热

本篇之寒热病是风邪犯表,症见恶寒发热之病。故治以发散表邪为法。《灵枢·寒热病》所言之寒热,既有外感病,也有内伤病,显然与本篇有别。

(2)热中

热中病是外感风邪化热,以发热、目黄为主症的病。阳明为气血俱盛之经,风为阳邪,若其人肥则腠理致密,阳气不得宣泄,郁而化热,则为发热之症。肥人多湿,湿热交蒸,上薰于目,则见目黄。其湿热兼有表证者,可选麻黄连翘赤小豆汤;湿热并重者可用茵陈蒿汤

(3)寒中

寒中,是阳气素虚,风邪外袭,邪从寒化之证,以汗出,恶风,流泪为主症。可选麻黄附子细辛汤以温阳解表。

其三,以临床表现特征命名。此类风病有四:

(1)偏风

偏风病多为外风引动内风,以半身不遂为主症的病。治疗可酌用补阳还五汤益气活血。

(2)漏风

漏风,亦名酒风。是因饮酒后感受风邪所致。《素问·病能论》说:“有病身热懈堕,汗出如浴,恶风少气。此为何病?岐伯曰:病名曰酒风。”治以泽泻饮祛风除湿。

(3)泄风

泄风是素体阳虚,表卫不固,风邪侵袭,腠理疏松开泄所致的病。可酌选玉屏风散治之。

(4)疠风

疠风,即后世所称之麻风病。风邪犯脉,入里化热,气凝血热为其主要病机。《诸病源候论》提出用雷丸治之。

其四,以感风途径命名。此类风病有二:

(1)内风

此风病是入房伤肾,汗出复感风邪的病。临证当分辨其为气虚外感、阳虚外感,或为阴虚外感,随证治之。

(2)首风(见前)。

\section{素問·痹論}%第六節

\biaoti{【原文】}

\begin{yuanwen}
黃帝問曰:痹之安生?岐伯對曰:風寒濕三氣難至合而爲痹也。其風氣勝者爲行痹\sb{1},寒氣勝者爲痛痹\sb{2},濕氣勝者爲著痹\sb{3}也。

帝曰;其有五者何也?岐伯曰:以冬遇此者爲骨痹\sb{4},以春遇此者爲筋痹\sb{4},以夏遇此者爲脈痹\sb{4},以至陰\sb{5}遇此者爲肌痹\sb{4},以秋遇此者爲皮痹\sb{4}。

帝曰:內舍\sb{6}五藏六府,何氣使然?岐伯曰:五藏皆有合\sb{7},病久而不去者,內舍於其合也。故骨痹不已,復感於邪,內舍於腎;筋痹不已,復感於邪,內舍於肝;脈痹不已,復感於邪,內舍於心;肌痹不已,復感於邪,內舍於脾;皮痹不已,復感於邪,內舍於肺。所謂痹者,各以其時重感於風寒濕之氣\sb{8}也。

凡痹之客五藏者,肺痹者,煩滿喘而嘔;心痹者,脈不通,煩則心下鼓\sb{9},暴上氣而喘,嗌乾,善噫\sb{10},厥氣\sb{11}上則恐;肝痹者,夜臥則驚,多飲數小便,上爲引如懷\sb{12};腎痹者,善脹,凥以代踵,脊以代頭\sb{13};脾痹者,四支解堕\sb{14},發欬嘔汁,上爲大塞\sb{15};腸痹者,數飲而出不得\sb{16},中氣喘争\sb{17},時發飧泄;胞痹者\sb{18},少腹膀胱按之內痛,若沃以湯\sb{19},濇於小便,上爲清涕\sb{10}。

陰氣\sb{21}者,靜則神藏,躁則消亡\sb{22}。飲食自倍,腸胃乃傷。淫氣\sb{23}喘息,痹聚在肺;淫氣憂思,痹聚在心;淫氣遺溺,痹聚在腎;淫氣乏竭\sb{24},痹聚在肝;淫氣肌絕\sb{25},痹聚在脾。

諸痹不已,亦益內\sb{26}也。其風氣勝者,其人易已也。

帝曰:痹,其時有死者,或疼久者,或易已者,其故何也?岐伯曰:其入藏者死,其留連筋骨間者疼久,其留皮膚間者易已。

帝曰:其客於六府者,何也?岐伯曰:此亦其食飲居處,爲其病本也。六府亦各有俞,風寒濕氣中其俞,而食飲應之,循俞而入,各舍其府也。

帝曰:以鍼治之奈何?岐伯曰:五藏有俞\sb{27},六府有合\sb{28},循脈之分,各有所發\sb{29},各隨其過則病瘳也\sb{30}。
\end{yuanwen}

\biaoti{【校注】}

\begin{jiaozhu}
  \item 行痹:以肢节酸痛,游走无定处为特点的痹病,也称为风痹。
  \item 痛痹:以疼痛剧烈为特点的痹病,也称寒痹。
  \item 著痹:以痛处重滞固定,或顽麻不仁为特点的痹病,也称湿痹。
  \item 骨痹、筋痹、脉痹、肌痹、皮痹:此根据风寒湿三气侵入人体的不同季节,以及五脏应五时、合五体的理论进行命名的五种痹病,合称为五体痹。
  \item 至阴:指长夏季节。
  \item 内舍:指病邪入内,稽留潜藏之意。
  \item 五脏皆有合:谓五脏都有与之相联系的五体。《素问·五脏生成》“心之合脉也,肺之合皮也,肝之合筋也,脾之合肉也,肾之合骨也。”
  \item 各以其时重感于风寒湿之气:谓各在其所主的时令季节,又重复感受了风寒湿邪。
  \item 心下鼓:指心悸。鼓,动也。
  \item 嗌干,善噫:谓咽干、嗳气。
  \item 厥气:逆气。
  \item 上为引如怀:谓腹部胀大如引满之弓,有似怀孕之状。引,《说文》:“开弓也。”
  \item 凥以代踵,脊以代头:凥,尾骶部。踵,足跟。凥以代踵,指足不能行,以凥代之。脊以代头,指头俯不能仰,以致脊高于头。
  \item 四支解堕:四肢倦怠乏力。支,通“肢”。解,通“懈”。
  \item 大塞,大,当作“不”,形误。不,通“否”。否,通“痞”。大塞,即痞塞。又《太素》“塞”作“寒”。
  \item 出不得:谓小便不通。
  \item 中气喘争:谓肠中之气斡旋转动剧烈,即肠鸣,争,《三因方》卷三《叙论》引“争”作“急”,义胜。
  \item 胞痹:即膀胱痹。胞,脬也,指膀胱。
  \item 若沃以汤:谓如热水灌之,有灼热感。《说文》:“沃,灌溉也。”汤,热水。
  \item 上为清涕:马莳注:“膀胱之脉,上额交巅,上入络脑,故邪气上蒸于脑而为清涕也。”
  \item 阴气:指五脏的精气。
  \item 静则神藏,躁则消亡;张介宾《类经·疾病类》注:“人能安静,则邪不能干,故精神完固而内藏。若躁扰妄动,则精气耗散,神志消亡,故外邪得以乘之,五脏之痹因而生矣。”
  \item 淫气:指五脏失和之气。
  \item 乏竭:《大素》作“渴乏”。森立之《素问考注》注:“渴乏者,渴燥匮乏之义,内渴乏,故引饮甚多也,是亦邪结饮闭在肝经之证。”一说为疲乏力竭。
  \item 肌绝:肌肉消瘦。
  \item 益内:谓病甚向内发展。益,通“溢”,蔓延之意。
  \item 五脏有俞:谓五脏各有输穴。俞,此指“五腧穴”中的输穴。
  \item 六腑有合:谓六腑各有合穴。
  \item 各有所发:各经受邪,均在经脉所循行的部位发生病变而出现症状。
  \item 各随其过则病瘳(chōu抽)也:各随其病变部位而治之则病愈。过,指病变。瘳,病愈。随,《太素》、《甲乙经》均作“治”,于义为长。
\end{jiaozhu}

\biaoti{【理论阐释】}

本段经文系统地论述了痹病的病因、病机、分类、证候、传变、治疗及预后等,可谓是《内经》论述痹病的重要文献。

1.论痹的涵义

痹,病名,是指因感受风寒湿邪所致的,气血凝滞,经脉闭塞不通的一类疾病。此处是以病机言其病名的。论“痹”是《内经》中的重要命题。除本篇专论外,尚有40余篇涉及与痹有关的内容,其中以痹为名的病,约有50余种。就《内经》所论“痹”之涵义,主要有四:一为病名,泛指风寒湿邪所致气血经脉闭阻不通的一类疾病。本篇即属于此;二指痛风历节病,如《灵枢·寒热病》篇之“骨痹举节不用而痛”者是,本节之“行痹”、“痛痹”、“著痹”,亦属于此;三指闭塞不通之病机。如《素问·阴阳别论》之“一阴一阳结,谓之喉痹”,以及《素问·至真要大论》之“食痹则呕”等语中的“痹”,即属于此。四是对阴分病的泛称。如《灵枢·寿夭刚柔》篇说:“病在阳者名曰风,病在阴者名曰痹。”《素问·宣明五气》篇亦有类似之说,曰:“邪入于阴则痹。”

本篇所论之痹,是指风寒湿邪气侵犯人体,导致经络脏腑痹阻不通,所引起的以肢体关节疼痛痠楚、麻木不仁、沉重,以及脏腑功能障碍,气机阻滞不畅为特点的一类疾病。其中包括的内容极其广泛。既有形体疾病,又有脏腑功能障碍等全身性多系统、多种类的疾病,不能单纯地认为是风湿性关节炎。

2.论痹之发病机理

本篇开章以痹之发病起论,切入论痹主题,从痹之发病原因、发病季节等方面,论述了痹病的发病机理。

(1)病因

风寒湿邪所伤,是致痹的主要原因,原文用“风寒湿三气杂至合而为痹也”句概之。此语提示:痹病属外感病范畴;外邪致病的相兼性;致痹之邪是诸邪杂合的“复合致病因子”,非单一外邪所犯。这就从病因方面提示该病的复杂性和治疗该病的困难性。

(2)季节与发病

外邪致病有明显的季节性。本篇从五体痹的发生,详述了其发病有明显的季节性。五体内合五脏,外应于四时,所以在不同季节感受风寒湿致痹之邪时,就会在不同部位发生相关的痹病。所以张志聪说:“皮肉脉筋骨,五脏之外合也。五脏之气,合于四时五行,故各以其时而受病,同气相感也”(《素问集注·卷五》)。详言之,骨痹,是因冬气通于肾,肾主骨,冬季肾气衰而感受风寒湿邪,侵犯于骨而发;筋痹,是因春气通于肝,肝主筋,春季肝气衰而感受邪气,侵犯于筋而发;脉痹,是因夏气通于心,心主脉,夏季心气不足,感受邪气,伤于血脉而发;肌痹,是因长夏之气通于脾,脾主肌肉,脾气衰,则长夏感受邪气,侵及肌肉而发;皮痹,因秋气通于肺,肺主皮毛,秋季肺气不足,感受邪气,皮表受邪而发。

五脏痹的发病,同样也受季节因素的影响。

(3)五脏痹的发病机理

“邪之所凑,其气必虚”(《素问·评热病论》),是《内经》的基本发病观。痹邪之所以能内传五脏,不但有其风寒湿邪之外因,而且有其内脏先伤之内在发病基础。就本篇全文精神而言,邪传于内而成五脏痹的机理有四:一是体痹病久不愈之因;二是各脏在其所应的季节重新感受风寒湿邪,此即原文所言的“所谓(五脏)痹者,各以其时重感于风寒湿之气也”;三是五脏阴精先伤,是痹传五脏的内在病理基础。因为五脏藏精舍神,宜静藏而忌躁扰。如若五脏所藏之神躁扰妄动,必然招致所藏阴精,消损耗伤,亦会使其相应之体的痹邪内传舍之,而成五脏痹。故曰:“五脏有合,病久而不去者,内舍于其合也。”四是营卫失常。营卫之气是生命活动的重要物质,“和调于五脏”,“熏于肓膜,散于胸腹”,也是五脏赖以濡养和温煦的物质基础,因此,营卫失调,必致五脏受损,也是五脏痹形成的机理之一。

(4)六腑痹的发病机理

痹邪之所以能伤及六腑而成六腑痹病,其机理亦不外乎是内外因素之两端。就外部原因而言,是风寒温致痹之邪伤犯了分布于肌表的六腑经脉(即所谓“六府亦各有俞”之意)。经脉不但是人体气血运行的通路,身体各种信息传导的通路,也是人体在病理状态下,邪气传送的通道,所以原文说,邪气可以“循俞而入”。就内在的病理基础而言,一者是“饮食居处,为其病本也。”所谓“饮食”,是指饮食不节,原文以“饮食自倍,肠胃乃伤”为例示之,凡饮食不节制,无论是过饱、过饥、偏嗜等,只要是造成了肠胃的损伤,皆可成为引发六腑痹的内在基础。所谓“居处”,是指起居失常,生活无规律,以及不利于健康的居住环境。其二,营卫失调亦应当是其形成的内在条件之一。因为营卫之气,可以“洒陈于六府”,“熏于肓膜,散于胸腹”,同样也是六腑活动的重要物质基础,因而营卫失调,也必然会招致六腑机能障碍,成为风寒湿邪内传六腑的条件之一。

3.论痹病分类

《内经》虽然论述了不同类型的痹病,但就其分类而言,以本篇较为完整而系统。原文是以病因、症状特点、病性,以及病位四方面为据进行分类的。

一是依据病因分类。风寒湿邪夹杂致痹时,其间的比例轻重互有不同,因此本篇依据感邪偏重的不同而进行分类。正因为邪气的性质不同,其致病特点必然有别,故尔有行痹、痛痹、著痹之殊。后世将其分别命名为风痹、寒痹、湿痹,即源于此。因为风邪善行而数变,当其偏胜所致之痹,就有游走性疼痛的特点,故谓之“行痹”;寒邪凝滞主疼痛,是外邪中最易致痛之邪,当其偏胜所致之痹,以疼痛剧烈为主要症状,故谓之“痛痹”;湿邪具有“重浊”、“粘滞”之性,伤人后与所在病位的组织器官胶著难去,当其偏胜所致之痹,以肢体、关节沉重、痠困、固定不移为特点,故曰“著痹”。正如马莳所云:“风寒湿之三邪气错杂而致,则合之于体而痹生,合之于经而痹分,故曰合而为痹也。其风气胜者,则风以阳经而受之,故当为行痹之证,如虫行于头而四体也。其寒气胜者,则寒以阴经受之,故当为痛痹之证,寒气伤血而伤处作痛也。其湿气胜者,则湿以皮肉筋脉而受之,故当为著痹之证,当沉著不去,而举之不痛也”。

二是以痹病的主要症状特点分类。不同的致病因素,或者同一邪气所伤部位不同,或者同一疾病发生于不同类型的个性体质,其表现的症状特点有较大的差异。本节正是基于这一认识,将痹病分别以症状呈游走不定的特征,命名为“行痹”;以疼痛剧烈为特征者命名为“痛痹”;以症状固定不移,胶着难愈为特征者命名为“著病”。《灵枢·周痹》篇则以疼痛症状环绕流动者命为“周痹”;以疼痛部位广泛命名为“众痹”等,均体现《内经》以症状作为痹病分类命名的方法。

三是病性分类。所谓病性分类,即是以痹病的寒、热性质为分类依据的方法。病因虽同,但由于患者体质的差异,患病后就可能有热化、寒化的区别。痹邪所伤,若其人是“阳气少,阴气多”的偏明体质,邪气便从阴化寒,即成为“寒痹”病。痹邪所伤,若其人是“阳气多,阴气少”的偏阳体质,邪气便会从阳化热,即可成为“热痹”病。

四是根据病位分类。本篇将痹病按病位分为三类:一类是五体痹,分别是皮痹、肌痹、脉痹、筋痹、骨痹。此种分类,不但体现了邪伤形体不同层次所患的不同痹病,而且体现了痹邪由皮→肌肉→脉→筋→骨的由浅入深的演变过程;同时提示病位深浅不同,预后有别,此可从“其入脏者死,其留连筋骨间者疼久,其留皮肤间者易已”的原文证之。一类是五脏痹。五脏外合五体,体痹病久不愈,若内脏失常,重感风寒湿邪时,体痹便会内传与之相合的脏而发生五脏痹。此五者分别是肺痹、脾痹、心痹、肝痹、肾痹。另一类是六腑痹,原文以肠痹(大肠痹、小肠痹)、胞(膀胱)痹例之。

4.论痹病预后

文中对痹病的预后,主要是从感邪之性质、病位之深浅,以及病程之长短而加以分析的。从病邪性质而论,由于风为阳邪,其性轻扬,易于驱除,故风邪偏胜者易于痊愈;如系寒湿之阴邪侵犯而偏胜者,由于寒性凝滞,湿性粘腻,难以驱除,故难于治愈。从病位深浅而论,邪在皮肤之间者,病位轻浅,邪气易于祛散,故容易治愈;若邪气留滞于筋骨之间,阻塞经络气血,使其凝滞不通,病位较深,邪气既不易外出,又不内入,病邪不易祛散,故病变难以治愈;若邪气侵及五脏,损伤精神气血,病位更为深在,正虚邪盛,预后较差。从病程长短而论,病程短者,邪气轻而病位浅,易于治愈;病程长,邪气深入,病情缠绵持久,难以治愈,预后亦较差。所以王冰注曰:“入脏者死,以神去也。筋骨疼久,以其定也。皮肤易已,以浮浅也。由斯深浅,故有是不同。”《内经》痹证预后的理论,既符合临床实践,又强调早期治疗的重要性,是值得借鉴的。

5.论痹病的刺治原则

关于痹病的治疗,本节虽然仅论针刺治疗,但其基本精神却有广泛意义。原文明确提出了两条基本原则;一是辨证论治。“五脏有俞,六腑有合,循脉之分。”说明痹病发生的部位,分别其属脏、属腑,病在何经,而后在其相应经脉之“俞穴”、“合穴”及相应经脉取穴刺治。二是根据疼痛部位,在痛处取穴。此即《灵枢·经筋》所说的“以知为数,以痛为输”的取穴原则。

\biaoti{【临证指要】}

痹病的治疗,要遵循辨证论治的原则:一是要审因论治,若以风邪偏盛所致的行痹,宜缪刺法,方选防风汤加味;若以寒邪偏盛而致的痛痹,可用火焠热熨法,方如乌头汤、甘草附子汤之类;若以湿邪偏盛所致的著痹,可选蠲痹汤。二是脏腑定位论治。如心痹之用苓桂术甘汤、瓜蒌薤白半夏汤;肝痹之用肝痹散(《辨证奇闻》),肺痹用肺痹汤等(《辨证奇闻》)。

\biaoti{【原文】}

\begin{yuanwen}
帝曰:榮衛之氣,亦令人痹乎?岐伯曰:榮者,水榖之精氣也,和調於五藏,灑陳於六府\sb{1},乃能人於脈也,故循脈上下,貫五藏絡六府也。衛者,水榖之悍氣也,其氣慄疾滑利,不能入於脈也,故循皮膚之中,分肉之間,熏於肓膜\sb{2},散於胸腹;逆其氣則病,從其氣則愈,不與風寒濕氣合,故不爲痹。

帝曰:善。痹,或痛,或不痛,或不仁,或寒,或熱,或燥,或濕,其故何也?岐伯曰:痛者,寒氣多也,有寒故痛也。其不痛不仁\sb{3}者,病久入深,榮衛之行濇,經絡時疎,故不通\sb{4}。皮膚不營,故爲不仁。其寒者,陽氣少,陰氣多,與病相益\sb{5},故寒也.其熱者,陽氣多,陰氣少,病氣勝,陽遭陰\sb{6},故爲痹熱。其多汗而濡\sb{7}者,此其逢濕甚也。陽氣少,陰氣盛,兩氣相感\sb{8},故汗出而濡也。

帝曰:夫痹之爲病,不痛何也?岐伯曰:痹在於骨則重,在於脈則血凝而不流,在於筋則屈不伸,在於肉則不仁,在於皮則寒。故具此五者,則不痛\sb{9}也。凡痹之類,逢寒則蟲\sb{10},逢热则縱。帝曰:善。
\end{yuanwen}

\biaoti{【校注】}

\begin{jiaozhu}
  \item 和调于五胜,洒陈于六腑:谓营气运行布散于五脏六腑。姚止庵注:“和调者,运行无间;洒陈者,遍满不遗,然惟和调,故能洒陈也。”
  \item 熏于肓膜:熏,温煦之意。肓膜,张介宾《类经·疾病论》注:“凡腔腹肉理之间,上下空腺之处,皆谓之肓。”“盖膜犹幕也,凡肉理之间,脏腑内外其成片联络薄筋,皆谓之膜。”
  \item 不痛不仁:杨上善《太素·痹论》注:“仁者,亲也,觉也。营卫及经络之气疏涩,不营皮肤,神不至于皮肤之中,故皮肤不觉痛痒,名曰不仁。”
  \item 不通:《太素》、《甲乙经》均作“不痛”,可从。
  \item 阳气少,阴气多,与病相益:“阳气少,阴气多”,指人的体质偏于阴盛。病,此指风寒湿邪。意指素体阴盛者,再感受风寒湿邪,其寒更甚。益,增加、助长之意。
  \item 阳遭阴:遭,《甲乙经》作“乘”。乘,战而胜之也。言病人素体阳盛阴虚,感受风寒湿邪后,阴不胜阳,邪从阳化热,故为痹热。
  \item 濡:湿润。
  \item 两气相感:指人体偏盛的阴气与以湿邪为主的风寒湿邪相互作用。
  \item 具此五者,则不痛:张琦注:“五者具,则自皮入骨,前所谓病久入深,明不痛之为重也。”
  \item 虫:《甲乙经》、《太素》均作“急”又。孙诒让《礼迻》:“虫,当为痋之借字……段玉裁《说文》注谓:‘痋’,即疼字。”孙校为是。
\end{jiaozhu}

\biaoti{【理论阐释】}

1.营卫与痹病发生的关系

营行脉中,卫行脉外,阴阳相贯,气调血畅,濡养四肢百骸、脏腑经络。营卫之气与人体防御功能有着密切的关系。营卫和调,则卫外御邪能力强,邪气不易侵入人体;若营卫不和,腠理疏松,防御功能减退,则风寒湿邪侵袭,易使脉络闭阻,气血凝滞,而形成痹病,故原文曰:“逆其气则病”,“不与风寒湿气合,故不为痹”。说明营卫失调是痹病发生的内在因素之一。林佩琴发挥说:“诸痹……良由营卫先虚,腠理不密,风寒湿乘虚内袭,正气为邪所阻,不能宣行,因而留滞,气血凝涩,久而成痹”(《类证治裁·痹证论治》)。姚止庵也说:“水谷之精气为荣,荣行脉内,贯通脏腑,无处不到。水谷之悍气为卫,卫行脉外,屏藩脏腑,捍御诸邪。邪欲中入,必乘卫气之虚而入,入则由络抵经,由腑入脏。是风寒湿之为痹也,皆因卫虚,不能悍之于外,以致内入,初非与风寒湿相合而然,是故痹止于荣而不及卫也”(《素问经注节解·三卷》)。可见,痹病的发生与营卫失调有着十分密切的关系,临证治疗时,应当以“从其气则愈”的观点,作为重要的指导原则。

2.痹病的发病类型与体质的关系

《内经》作者十分重视体质在疾病中的重要作用。不但认为体质的强弱决定着感邪与否、发病与否,如“勇者气行则已,怯者则著而为病”(《素问·经脉别论》)者是;而且认为体质决定着感邪的轻重、所感邪气的性质、邪气入侵后伤害的部位;同时已经认识到体质可以影响疾病性质的转化。本节原文就是基于对体质与疾病性质关系的这一认识,指出同样是“风寒湿”致痹之邪,但有偏寒(邪从寒化,而生寒痹病)、偏热(邪从热化而生热痹)的不同性质。原文十分明确地指出:“其寒者,阳气少,阴气多,与病相益,故寒也;其热者,阳气多,阴气少,病气胜,阳遭阴,故为痹热。”张志聪则体悟出原文的真谛,从体质阴阳类型的视角诠解此处痹之寒热,认为:“此言寒热者,由人身之阴阳气化也”(《素问集注·卷五》)。可见,凡属偏阴质的人,致痹之邪便易从阴寒化,而成寒痹病;若逢偏阳质之人,患痹后邪气易从阳化热而成热痹病。

本节突出了《内经》对痹病个体化特征的现点。认为,虽然属同一致病因素,但由于患者不同的个性体质差异的缘故,其病位深浅亦有区别,因而有“或痛、或不仁、或寒、或热、或燥、或湿”不同类型的个体化症状特点。这不仅是中医体质学研究的重要内容,而且对于分析痹病的不同类型、痹病的演变转归,以及痹病的辨证用药均有重要的指导意义。

3.痹病与季节气候的关系

本篇从两方面论述了痹病与季节气候的关系。首先认为痹病的发病与季节气候的关系十分密切,篇首即云:“以冬遇此者为骨痹,以春遇此者为筋痹,以夏遇此者为脉痹,以至阴遇此者为肌痹,以秋遇此者为皮痹。”五体痹的发病如此,五脏痹的发病亦是如此,各脏之痹,皆在内脏先伤,以及已患体痹的基础上,“各以其时重感于风寒湿之气”的缘故。其次,认为各种类型的痹病,其病情的变化常受气候因素的影响而有相应的波动。“凡痹之类,逢寒则虫(“疼”的诈字),逢热则纵”,即是这一观点的体现,也非常符合临床所见。

\biaoti{【临证指要】}

\xiaobt{调和营卫法治疗痹病}

在《内经》营卫失调为痹证发生主要内在因素的理论指导下,后世医家在论治痹证时,十分重视调和营卫治法。张仲景在论述历节“疼痛如掣”时,认为其病机为“风血相搏”(《金匮要略·中风厉节病脉证并治第五》),应用桂枝芍药知母汤治疗历节痛,方中用桂枝、芍药、甘草、白术调和营卫,就是突出治疗痹证应用调和营卫扶助正气的原则。朱丹溪在论治痛风时指出:“气行脉外,血行脉内,昼行阳二十五度,夜行阴二十五度,此平人之造化也。得寒则行迟而不及,得热则行速太过。内伤于七情,外伤于六气,则血气之运或迟或速,而病作矣”(《格致余论·痛风论》),朱氏所言气血,即荣血卫气。在治疗上,现今一般在发作期间,以祛邪为主,在静止期,则以调营卫,养气血、补肝肾为主,即在祛风、逐寒、化湿的同时,加入活血、行血、通经、化瘀之品,如桂枝、当归、乳香、没药、赤芍、红花、桃仁、五灵脂、蒲黄、地黄等药,且常用酒以提高活血功效。

\section{素問·痿論}%第七節

\biaoti{【原文】}

\begin{yuanwen}
黃帝問曰:五藏使人痿,何也?岐伯對曰:肺主身之皮毛,心主身之血脈,肝主身之筋膜\sb{1},脾主身之肌肉,腎主身之骨髓。故肺熱葉焦\sb{2},則皮毛虛弱急薄\sb{3},著\sb{4}則生痿躄\sb{5}也。心氣熱,則下脈厥而上,上則下脈虛,虛則生脈痿,樞折挈\sb{6},脛縱而不任地也。肝氣熱,則膽泄口苦,筋膜乾,筋膜乾則筋急而攣,發爲筋痿。脾氣熱,則胃乾而渴,肌肉不仁,發爲肉痿。腎氣熱,則腰脊不舉\sb{7},骨枯而髓減,發爲骨痿。

帝曰:何以得之?岐伯曰:肺者,藏之長\sb{8},爲心之蓋也,有所失亡,所求不得,則發肺鳴\sb{9},鳴則肺熱葉焦。故曰五藏因肺熱葉焦\sb{10},發爲痿躄,此之謂也。悲哀太甚,則胞絡絕\sb{11},胞絡絕則陽氣內動,發則心下崩\sb{12},數溲血也。故《本病》曰:大經空虛,發爲肌痹\sb{13},傳爲脈痿。思想無窮,所願不得,意淫於外,入房太甚,宗筋\sb{14}弛縱,發爲筋痿,及爲白淫\sb{15}。故《下經》曰:筋痿者,生於肝,使內\sb{16}也。有漸\sb{17}於濕,以水爲事,若有所留,居處相濕\sb{18},肌肉濡漬,痹而不仁,發爲肉痿。故《下經》曰:肉痿者,得之濕地也。有所遠行勞倦,逢大熱而渴,渴則陽氣內伐\sb{19},內伐則熱舍於腎,腎者水藏也,今水不勝火\sb{20},則骨枯而髓虛,故足不任身,發爲骨痿。故《下經》曰:骨痿者,生於大熱也。

帝曰:何以別之?岐伯曰:肺熱者,色白而毛敗;心熱者,色赤而絡脈溢\sb{21};肝熱者,色蒼而爪枯;脾熱者,色黃而肉蠕動\sb{23};腎熱者,色黑而齒槁。

帝曰:如夫子言可矣。論言治痿者,獨取陽明何也?岐伯曰:陽明者,五藏六府之海,主閏\sb{23}宗筋,宗筋主束骨而利機關\sb{24}也。衝脈者,經脈之海也,主滲灌谿谷\sb{25},舆陽明合於宗筋,陰陽揔宗筋之會\sb{26},會於氣街,而陽明爲之長\sb{27},皆屬於帶脈而絡於督脈。故陽明虛,則宗筋縱,帶脈不引,故足痿不用也。帝曰:治之奈何?岐伯曰:各補其滎而通其俞\sb{28},調其虛實,和其逆順,筋脈骨肉,各以其時受月\sb{29},則病已矣。帝曰:善。
\end{yuanwen}

\biaoti{【校注】}

\begin{jiaozhu}
  \item 筋膜:森立之《素问考注》曰:“筋与膜同类而异形,所以连缀脏腑,维持骨节,保养䐃肉,为之屈伸自在者也。”
  \item 肺热叶焦:《太素》、《甲乙经》“肺”下并有“气”字。叶焦,是形容肺叶受热灼伤,津液损耗的病理状态。
  \item 急薄:形容皮肤干枯无泽,拘急不舒的样子。
  \item 著:有甚之意。
  \item 痿躄:指四肢萎废不用的病。躄,指下肢行动不便。
  \item 枢折挈:枢,枢轴,转轴,在此指关节。折,断也。挈,提也,用手提物曰挈。枢折挈,是形容关节弛缓,不能提举活动,犹如枢轴折断不能活动之状。
  \item 腰脊不举:谓腰脊不能活动。
  \item 肺者,脏之长:指肺居于人体五脏的上部,又朝百脉,布津液达于全身。
  \item 肺鸣:指因肺气不畅而致的喘咳声音。
  \item 故曰五脏因肺热叶焦:《甲乙经》无此九字。
  \item 胞络绝:谓心包之络脉阻绝不通。
  \item 心下崩:指心气上下不通,心阳妄动,迫血下行而尿血。崩,大量出血。
  \item 肌痹:《太素》作“脉痹”。按下文有“肌肉濡渍,痹而不仁,发为肉痿”之论述,此乃言经脉空虚,渗灌不足,血行涩滞,痹而不通,发为脉痹。
  \item 宗筋:指众筋,泛指全身筋膜。于鬯《香草续校书》曰:“宗当训众。《广雅·释诂》云:‘宗,众也。’”
  \item 白淫:此指男子滑精一类病症。
  \item 使内:谓入房。
  \item 渐:浸渍之意,《广雅·释诂》:“渐,渍也。”
  \item 相湿:《甲乙经》作“伤湿”可从。
  \item 阳气内伐:谓阳热邪气内侵,使津液耗伤。伐,侵也。
  \item 水不胜火:谓肾之阴精受损,不能制胜于火热之邪。
  \item 络脉溢:指表浅部位的血络充血。
  \item 肉蠕动:蠕,《太素》作“濡”,濡亦有软意。动,郭霭春《黄帝内经素问校注》疑为“蠕”之旁记,误入正文。肉蠕,即肌肉软弱。
  \item 闰:《甲乙经》作“润”即润养之意。
  \item 主束骨而利机关:谓筋具有约束骨节而使关节屈伸灵活的作用。机关,此指关节。
  \item 渗灌谿谷:谓渗透滋灌腠理肌肉及骨节。
  \item 阴阳揔宗筋之会:阴阳,指阴经、阳经。揔,音义同“总”,会聚也。宗筋,此特指前阴。张介宾注云:“宗脉聚于前阴,前阴者,足之三阴、阳阴、少阳及冲、任、督、蹻九脉之所会也。”
  \item 阳明为之长:指诸经在主润众筋功用方面,阳明经有主导作用。
  \item 各补其荥而通其俞:吴昆注:“十二经有荥有俞,所溜为荥,所注为俞。补,致其气也。通,行其气也。”如肺经荥穴鱼际、俞穴太渊,大肠经荥穴二间、俞穴三间等。
  \item 各以其时受月:分别以各脏所主季节进行针刺治疗。高世栻注:“肝主之游,心主之脉,肾主之骨,脾主之肉,各以其四时受气之月而施治之则病已矣。受气者,筋受气于春,脉受气于夏,骨受气于冬,肉受气于长夏也。”又张志聪注:“《诊要经终》篇曰:正月二月,人气在肝;三月四月,人气在脾;五月六月,人气在头;七月八月,人气在肺;九月十月,人气在心;十一月十二月,人气在肾。”又《太素》“月”作“日”。森立之《素问考注》:“谓肝木痿证,以甲乙日刺之也。他四脏皆仿此。”
\end{jiaozhu}

\biaoti{【理论阐释】}

1.痿病概念及其与痹病关系

(1)痿病概念:痿,是指肌肉萎缩,四肢不能随意运动的病。《内经》又称之为“痿躄”、“痿疾”、“瘘易”等。从其症状特点言,《内经》所载痿病,有弛缓不收性(“胫纵”)和挛缩不伸性(“筋急而挛”)两类。

(2)痿病、痹病的区别及联系:古人常将痹病、痿病混称。《说文》云:“痿,痹也。”《内经》亦有之,如《素问·气交变大论》说:“岁火不及,寒乃大行”,“复则……暴挛痿痹,足不任身。”《灵枢·阴阳二十五人》亦曰:“善痿厥足躄。”之所以有这种痹、痿混称现象,是因为:一则痿和痹的病因均与感染湿邪有关。如《说文》:“痿,痹也。”“痹,湿病也。”二则痿和痹均属肢体病,病本虽在内脏,但症状多表现于肢体的筋骨肌肉。三则两病均可有肌肤不仁、“足不任身”等某些相似的症状。四是痹病日久,可以演变为肌肉萎缩,肢体运动障碍之痿病。如本篇所言的“发为肌(当为‘脉’——编者)痹,传为脉痿”及“肌肉濡渍,痹而不仁,发为肉痿”,即是其例。临证中,由顽固痹病演变为手足萎废者亦不鲜见。

痿和痹毕竟是两个不同类型的疾病,二者的不同点在于:其一,病源不同。痹病纯属外感风寒湿邪所得,而痿病有外感,如感热、伤湿;亦有内伤,如情志所伤、房劳所伤等,其二,病性不同。痹病以阴寒性质为多见,虽有热痹,此不过是缘于患者体质而病从热化。本节所论痿病,则以阳热为主。其三,病传有别。痹病是外邪先犯形体,体痹病久不愈,内传五脏而致五脏痹。但是,痿病则相反,先有“肺热叶焦”、五脏有热,消灼精、血、津液,五体失养,发为五体痿。所以张志聪注云:“夫五脏各有所合,痹从外而合病于内,外所因也,痿从内而合病于外,内所因也。”“夫形身之所以能举止动静者,由脏气之煦养于筋脉骨肉也。是以脏病于内,则形痿于外矣”(《素问集注·卷五》)。其四,症状特点有别。痿病以手足萎弱无力,不能随意运动为主,一般无疼痛、痠楚等症,病情与季节气候无明显的相关性。痹病则不然,是以肢节疼痛、痠楚、困重、麻木为主症,病情变化常受季节气候的影响。

2.痿病的病因病机

(1)痿病的发病原因:就痿病的病因而言,本篇认为致痿原因有六:一是情志不遂。原文分别从“有所亡失,所求不得”,“悲哀太甚”,“思想无穷,所愿不得”方面,强调情志所伤、气郁生热而成痿病。二为形劳过度,耗气劫阴,阴不制阳,阳亢生热致痿,如“远行劳倦”者是。三为房事过度,劫耗肾阴,阴虚生热成痿,如“意淫地外,入房太甚”者是。四为外感热邪,伤津耗液而成痿,如“逢大热面渴”者是。五为湿邪浸渍,久留化热致痿,如“有渐于湿,以水为事,若有所留,居处相湿”者是。此外,还有脾胃损伤,阳明虚损致痿。

(2)痿病的发病机理:痿病形成的机理甚为复杂,原文从五个方面加以论述:

其一,五脏气热致痿。由于五脏分主五体,肺主皮毛,心主血脉,肝主筋膜,脾主肌肉,肾主骨髓。所以,五脏气热,熏灼津液,导致筋、脉、肉、皮、骨五体失养,从而发生五体痿。至于形成五脏气热的原因,本篇所论有情志太过,劳伤过度,湿邪浸淫和触冒暑热等因素。王肯堂认为:“是用五志、五劳、六淫,从脏气所要者,各举其一以为例耳。若会通八十一篇而言,便见五劳、五志、六淫,尽得成五脏之热以为痿也”(《证治准绳·杂病》)。

其二,肺热叶焦致痿。本篇提出“五脏因肺热叶焦,发为痿躄”。说明痿的病变部位虽在四肢,但产生的根源却在五脏,而五脏之中尤以肺为关键。《素问·经脉别论》曰;“食气入胃,浊气归心,淫精于脉。脉气流经,经气归于肺,肺朝百脉,输精于皮毛。”《灵枢·决气》亦云:“上焦开发,宣五谷味,熏肤,充身,泽毛。”可见,全身各脏腑组织所需的营养物质,都是经肺的敷布而获得的,所以说,“肺为脏之长也”。如果肺叶受到邪热熏灼,使津伤叶焦,高源化绝,则筋脉皮毛骨肉失养,势必导致痿病。丹波元坚亦云:“思虑忿怒,五志之火内炽,消烁肺金,故喘息有音,而肺叶焦枯。肺所以行营卫,治阴阳,饮食之精,必自肺家传布,变化津液,灌输脏腑,肺脏一伤,五脏无所禀受,故因之以成瘘躄也”(《素问绍识·卷三》)。均说明肺热熏灼是致痿的重要病机。

其三,脾胃气虚致痿。《素问·五脏别论》说,“胃者,水谷之海,六腑之大源也……五脏六腑之气味,皆出于胃。”人有四海,胃居其一。脾胃为气血津液之化源,后天之本。人的脏腑之气,筋脉肌肉,四肢百骸,无不赖脾胃化生的水谷精气以资养。故本篇下文曰:“阳明者,五脏六腑之海,主润宗筋,宗筋主束骨利机关也。”《素问·太阴阳明论》亦指出:“四肢皆禀气于胃,而不得至经,必因于脾,乃得禀也。”正由于如此,本篇指出:“阳明虚则宗筋纵,带脉不引,故足痿不用也。”《素问·脏气法时论》更为明确地说:“脾病者,身重,善肌肉痿,足不收引。”以上所论,说明脾胃为水谷气血精微之源,是肺中精津的化生之处,如果脾胃气虚,不能化生水谷精微,精血津液亏虚,使筋骨肌肉失养,从而发生痿病。后世在论治痿病时,特别强调从阳明脾胃调治,其机理即源于此。

其四,肝肾亏虚致痿。由于肝主筋,为罢极之本;肾主骨,为作强之官。肝肾功能正常,精血充盛,则筋骨劲强,活动正常。如果因各种原因,使肝肾受损,精血亏虚,复因阴虚内热,使筋骨经脉失去濡养,则会导至痿病。本篇所述的“思想无穷,所愿不得,意淫於外,入房太甚,宗筋弛纵,发为筋痿”;“有所远行劳倦……阳气内伐,内伐则热舍于肾……发为骨痿”即属这一类型。《内经》认为思想无穷,情志过用,房室不节,形体劳伤,均可导致肝肾精血亏虚而发生痿病。《灵枢·本神》还认为:“恐惧而不解则伤精,精伤则骨酸痿厥,精时自下。”

其五,湿邪浸淫致痿。《内经》认为湿邪浸淫是形成痿证的主要外在因素。本文提出:“有渐于湿,以水为事,若有所留,居处相湿,肌肉濡渍,痹而不仁,发为肉痿。”此外,《素问·生气通天论》亦云:“湿热不攘,大筋緛短,小筋弛长,緛短为拘,弛长为痿。”指出湿热浸淫可致痿。《灵枢·九宫八风》曰:“犯其雨湿之地,则为痿。”《素问·气交变大论》亦曰:“岁土太过,雨湿流行,肾水受邪,民病……足痿不收。”《素问·六元正纪大论》还曰:“太阳司天之政……民病寒湿,发肌肉痿,足痿不收。”这均说明长期因气候,或居处环境,或受湿邪,或者寒湿之邪,均能导致痿病的发生。

综上所述,《内经》论痿有虚实之分,实证因暑热、湿热和寒湿所致,虚证多因脾肺虚弱,肝肾精血亏虚所致,病位涉及五脏,以肺、脾、肝、肾为主。后世医家仍宗《内经》五脏失常而致痿的总体思路,证之于临床,该病多由热、虚、痰、瘀而成,其病位则与肺、胃、肝、肾最为密切。《临证指南医案·痿》说:“夫痿证之旨,不外乎肝、肾、肺、胃四经之病。盖肝主筋,肝伤则四肢不为人用,而筋骨拘挛;肾藏精,精血相生,精虚不能灌溉诸末,血虚则不能营养筋骨;肺主气,为高清之脏,肺虚则高源化绝,化绝则水涸,水涸则不能濡润筋骨;阳明为宗筋之长,阳明虚则宗筋纵,宗筋纵则不能束筋骨以流利机关,此不能步履、萎弱筋缩之症作矣。”此论颇为中肯允当。

3.痿病的辨证分类

本篇主要论述了五脏郁热痿和(脾)胃虚弱(阳明虚)痿、湿热痿、肝肾亏虚痿几种类型。现结合《内经》相关篇章的原文述之。

(1)五脏郁热痿:原文以五脏主五体的生理为论证基础,详论了五脏因外感、内伤生热,致使精、气、血、津液耗伤,不能营养滋润五体,从而形成痿躄、筋痿、脉痿、骨痿、肉痿,并简要地叙述了五体痿的诊断要点。

痿躄,是以皮肤干枯不荣,肌肉枯萎,四肢痿弱,不能站立和行走。伴有咳喘为主要症状特点的病。其基本病机为“五脏因肺热叶焦”。

筋痿,是以筋脉拘急,肢体拘挛为主症的病。常伴见面色青、爪甲干枯、口苦等症。本病是因情志不遂,气郁化火;加之房事过度,损伤肝肾精血,以致筋脉失养之故。

骨痿,是以腰脊不能伸举,下肢痿软不能站立和行走为主要症状的病。常伴见面色黧黑少泽,牙齿干枯等症。本病是因“远行劳倦”,又感触热邪,以致伤肾髓减,损骨所致。

脉痿,是以关节松弛痿软,下肢软弱不能站立,行走,关节不能收提为特点的病。可伴见面赤,体表络脉充血的症状。本病是因心之邪热灼伤阴血,或悲哀太甚,心包络阻隔不通,阳热内扰,热迫血出,以致脉道空虚而成。

肉痿,是以面黄肌瘦,麻木不仁,下肢痿弱无力为特点的病。常伴见口渴、肌肉蠕动的症状。本病是因湿邪浸渍,郁而化热,湿热交蒸,熏灼于脾,使脾失于运化,水谷精微不能营养肢体肌肉之故。

(2)湿热痿:湿热痿,是以肢体痿弱无力,下肢为甚,伴有肤色黄微肿,肌肤麻木,口渴脘痞,小便色黄涩痛,舌红苔黄腻,脉濡数为特点的病。上述所言的“肉痿”病,即属于此。此类痿病,多因感受湿邪,湿阻阳气,郁而化热,湿热交蒸而成。《内经》对此认识一致。

(3)脾胃虚弱痿;此类痿病,是以肢体痿软无力,甚则肌肉消瘦,常伴有食少,腹胀,便溏,面色萎黄,或淡白无华等症。脾胃为气血化生之源,筋骨肌肉,四肢百骸,皆赖其养。脾胃受损不足,不能化生气血,日久则见本类疾病。所以本篇原文说:“阳明虚则宗筋纵,带脉不引,故足痿不用。”《素问·太阴阳明论》也说:“今脾病不能为胃行其津液,四肢不得禀水谷气,气日益衰,脉道不利,筋骨肌肉皆无气以生,故不用焉。”《证治汇补·痿躄》进一步发挥说:“气虚痿者,因饥饿劳惓,胃气一虚,肺气先绝,百骸溪谷,皆失所养,故宗筋弛纵,骨节空虚。凡人病后手足痿弱者,皆属气虚。”

(4)肝肾亏损痿:肝肾亏损类痿病,是以下肢痿弱无力,不能站立,甚则不能行走的病。常伴有腰脊痠软,遗精,早泄,头晕,目眩,耳鸣,或耳聋等肝肾亏损之症;本病多因劳伤肝肾、精血亏虚、筋骨失养所致。本节所言“思想无穷,所愿不得,意淫于外,入房太甚……筋痿者,生于肝,使内也”即属其例。所以,张介宾指出:痿病,“元气败伤则精虚不能灌溉,血虚不能营养者亦不少矣”(《景岳全书·痿证》)。

4.痿病的治疗

(1)治痿取阳明:篇中提出“治痿独取阳明”的问题,突出调理脾胃在痿病治疗中的重要性,并详述其机理。要言之有三:其一,“阳明者,五脏六腑之海”,乃人身气血津液化生之源泉;其二阳明“主润宗筋,宗筋主束骨面利机关”。《素问.五脏生成》篇EUK诸筋者,皆属于节/阳明盛,气血充,诸筋得以濡养,则关节滑利,运动自如;其三,阴经阳经总会于宗筋,合于阳明。会于前阴者虽有九脉,但冲脉、阳明脉占重要地位,而冲脉通过气街与阳明相会,以接受阳明的气血,故冲脉之气血本之于阳明。所以高世栻说:“阳明者,胃也,受盛水谷,故为五脏六腑之海,皮、肉、筋、脉、骨,皆资于水谷之精,故阳明主润宗筋……痿则机关不利,筋骨不和,皆由阳明不能濡润,所以治痿独取阳明也”(《素问直解》)。当然,“治痿独取阳明”是治痿的重要原则,而不是唯一的原则,应当根据痿病发病之脏不同,辨证治疗,即原文所说的“各补其荥而通其俞”之意。诚如张介宾所说:“盖治痿者,当取阳明,又必察其所受之经而兼治之也”。

(2)辨证论治:本篇指出:“各补其荥,而通其俞,调其虚实,和其逆顺”,强调治痿必须辨证论治。虚则补之,实则泻之。例如对痿躄的治疗,当详辨其虚实。如系肺有热邪,可取肺经输穴太渊穴,采用泻的针刺方法;若为肺气虚弱,可取肺经之荥穴鱼际,采用针刺补法。所以张介宾说:“如筋痿者,取阳明、厥阴之荥俞;脉痿者,取阳明、少阴之荥命;肉痿、骨痿其治皆然”。

(3)因时制宜:原文指出:对痿病辨证论治的同时,还要考虑时间因素的影响,“筋脉骨肉,各以其时受月,则病已矣”,即强调了因时制宜的原则。

\biaoti{【临证指要】}

1.痿病的治疗

关于痿病的治疗,应当根据“各补其荥而通其俞,调其虚实,和其逆顺”的论治原则,如肝阴不足,虚热内炽的筋痿,可选《辨证奇闻》的伐木汤;肾虚所致的骨痿,可选用大补阴丸。若为湿热致痿者,治疗当以清热化湿为法,可选加味二妙散,如果湿象不著者,辛燥之品当慎用。若是脾胃虚弱致痿者,临证用药要细辨是气虚或是阴虚,气虚者可选补中益气汤之类,阴虚者可选琼玉膏(《洪氏集验方》)之属,如果是肝肾亏虚,精血不足而成痿者,可用虎潜丸或六味地黄丸加牛骨髓、鹿筋胶、龟板等。

2.“治痿独取阳明”的临床应用

本论所述“治痿独取阳明”法则,后世医家甚为重视,一直指导着临床实践,且取得良好的疗效。一般认为本法不仅是补阳明之虚,而且也包括泻阳明之实。治阳明虚弱之痿,多为阳明气血虚弱,五脏失其化源,宗筋失养所致。治应补益阳明,使气血充盛,宗筋得滋,痿病则愈。常用如下三法:一是补益阳明之气:多因饮食劳倦,脾胃衰弱气虚血弱,宗筋失其温养,骨节空虚,关节弛纵无力。兼有食少便溏,肌肉萎缩,舌苔薄白,脉细。治宜健脾益气,以复阳明之正气。方用补中益气汤。二是补养阳明之血:此系阳明血虚,筋脉肌肉失去濡养,以致肢体萎软无力,面色萎黄、肢体肌肤麻木不仁,舌质淡红,脉细弱。治宜健脾养血,方用人参养荣汤合六君子汤。例如林佩琴曾治张氏,“四肢萎弱,动履艰难,脉涩且弱,为营虚之候。经言天癸将绝,系太冲脉衰……因知冲为血海,隶于阳明,阳明虚则冲脉不荣,而宗筋弛纵,无以束筋骨利机关。法当调补营血,以实奇经。人参、茯苓、杞子、牛膝、当归、杜仲、木瓜、山药,酒蒸熟地、姜、枣水煎,十数服,渐愈”(《类证治裁·痿》)。三是滋补阳明之阴:由于胃阴耗损,失其柔润筋脉之功,上无供心肺而毛脉枯萎,下不得充肝肾而筋骨萎弱,中焦缺其自养则宗筋弛纵。常有下肢萎弱无力,腰膝痠困,口干舌燥,脉细数,舌质红。治宜培中养胃,滋胃益阴,方用沙参麦冬饮加减。

后世发挥了治阳明邪实之痿。由于邪气壅结阳明,中焦失和,气血化源失常,亦可致痿。若阳明热盛成痿者,治宜泻阳明实热,方用白虎汤或承气汤加减;若湿热阻滞中焦,宜清化阳明湿热,方用二妙散加减。

\section{靈樞·水脹}%第八節

\biaoti{【原文】}

\begin{yuanwen}
黃帝問於岐伯曰:水\sb{1}舆膚脹、鼓脹、腸覃、石瘕、石水\sb{2},何以別\sb{3}之?岐伯答曰:水始起也,目窠上微腫\sb{4},如新臥起之狀,其頸脈動\sb{5},時咳,陰股間寒\sb{6},足脛瘇\sb{7},腹乃大,其水已成矣。以手按其腹,隨手而起,如裹水之狀,此其候\sb{8}也。

黃帝曰:膚脹\sb{9}何以候\sb{10}之?岐伯曰:膚脹者,寒氣客於皮膚之間,𪔣𪔣然不堅\sb{11},腹大,身盡腫,皮厚\sb{12},按其腹窅而不起\sb{13},腹色不變,此其候也。

鼓脹\sb{14}何如?岐伯曰:腹脹身皆大,大與膚脹等,色蒼黄,腹筋起\sb{15},此其候也。
\end{yuanwen}

\biaoti{【校注】}

\begin{jiaozhu}
  \item 水:指水胀病。
  \item 石水:病名.下文未见论及,疑原文有脱漏。石水主要表现为少腹肿,脉沉或徵大。病机为阴盛阳虚。
  \item 别:鉴別。
  \item 目窠(kē科)上微肿;即眼睑下轻微浮肿,如卧蚕状。
  \item 颈脉动:颈脉,指喉结旁之人迎脉。颈脉动,是指水湿内停,内泛血脉,脉中水气涌动,故可见颈脉搏动异常明显的状态。
  \item 阴股间寒:即大腿内侧因水湿所伤而感寒冷。
  \item 足胫瘇:泛指下肢足部浮肿。瘇,同“肿”。
  \item 候:指征候,即临床表现。
  \item 肤胀:病名,指因寒气客于皮肤之内,出现肿胀症状的一类疾病。
  \item 候;在此作诊察解。
  \item 𪔣𪔣(kōnɡkōnɡ空空)然不坚;指腹腔胀气,外观膨隆,叩击时呈鼓音状。𪔣𪔣,鼓音。
  \item 皮厚:张介宾《类经·疾病类》注:“然有水则皮泽而薄,无水则皮厚。”
  \item 窅(yǎo咬)而不起:此谓凹陷不起。窅本谓目深貌,此借喻凹陷。
  \item 鼓胀:病名,以腹部胀大如鼓,肤色苍黄,腹筋暴露为特征的一类疾病。
  \item 腹筋起:指腹壁有脉络显露、凸起。
\end{jiaozhu}

\biaoti{【理论阐释】}

1.水胀

本节以疾病鉴别诊断为切入视角,论述了水胀、肤胀、鼓胀三病的病因病机、症状特点,以及各病间的“两两鉴别”,为后世临床鉴别诊断理论的发展,做出了示范。

水胀是阳气不能蒸化,水湿停聚体内所致的病。该病初起症见目窠微肿,此为水湿上泛之故;继则因水气逆于阳明,而有人迎脉搏动明显之症;水湿逆犯于肺,宣降失常,故有时时咳嗽;阳气被水湿郁遏而失于温煦,故阴股间逆冷不温;水聚腹腔,但未停皮下,所以症见腹部膨隆胀大,腹壁无压痕,按诊时就象按压裹水的皮囊一样。杨上善对此有精辟的论说:“水病之状,候有六别:一者,目窠微肿;二者,足阳明人迎之脉,视见其动,不待按之;三者,胀气循足少阴脉上冲于肺,故时有咳;四者,阴下阴股间冷;五者,脚胻肿起;六者,腹如囊盛水状,按之不坚,去手即起。此之六种,水病候也”(《黄帝内经太素·气论》)。在水胀的这些症状中,最具鉴别诊断意义的症状是:“以手按其腹,随手而起,如裹水之状。”

《内经》以发病途径为据,将水肿病分为外感性水肿和内伤性水肿两大类别。

(1)外感性水肿。是指感受外邪而致的水肿病。细言之有如下三种:

①肾风。是外感风邪,伤犯于肾,肾不能蒸化水液而致,患者以恶风、多汗、小便不利,腰脊痛,色黑,以头面、上半身肿甚为特点的水肿病。如《素问》的《风论》、《评热病论》、《奇病论》所论者是。

②风水。是外感风邪,致使水湿不化,风邪与水湿相互搏结而成。初起头面浮肿,渐及全身,小便不利,咳嗽,身痛,恶风,脉浮。此即《素问·水热穴论》之风水病。还有如《灵枢经》的《论疾诊尺》、《九针论》也有论述。

③涌水。是寒邪伤肺,下传于肾,致使肺失宣降,肾失蒸化,水液停聚体内而成。由于肺与大肠相表里,故涌水病患者,可有咳嗽,气喘,腹大如水囊,全身浮肿,肠鸣等症。正如《素问·气厥论》所论者是。

(2)内伤性水肿。所谓内伤性水肿病,是指因七情、饮食劳倦,或失治、误治而成的水肿病。此类水肿有如下四病:

①风水。《素问·评热病论》所载之风水与《素问·水热穴论》之风水不同,属于肾风(外感性水肿)误刺伤肾发展而成的内伤性水肿病。其基本病机是误刺损伤肾阴,阴虚阳盛,阴阳失衡,致使肾主水液功能失常,水液不化,停蓄体内而成本病。症见浮肿、少气、时热、心烦、口干口苦、尿少色黄、肠鸣、不能食、不能仰卧、咳吐清涎、闭经等。所以《素问·评热病论》说:“有病肾风者……虚不当刺,不当刺而刺,后五日,其气必至”,“至必少气时热,时热从胸背上至头,汗出手热,口干苦渴,小便黄,目下肿,腹中鸣,身重难以行,月事不来,烦而不能食,不能正偃,正偃則咳,病名日风水。”

②溢饮。溢饮病是肝失疏泄,气机郁滞,气不行津液,而致水泛溢于肌肤、肠胃之外的水肿病。本病以全身浮肿,皮色光亮、多饮为主要症状。结合病位在肝的特点,究其病因,当属情志所伤而致者为多。《素问·脉要精微论》指出:“肝脉……其耎而散,色泽者,当病溢饮。溢饮者,渴暴多饮,而易入肌皮、肠胃之外也。”

③石水。石水病是由肾阳受损,阳虚阴盛,阳不化水,水湿内停所致。以少腹肿、下肢肿甚为其特点,如《素问·阴阳别论》说:“阴阳结斜,多阴少阳,曰石水,少腹肿。”《灵枢·邪气脏腑病形》也说:肾脉“微大为石水,起脐以下至少腹,腄腄然。”

④《素问·汤液醪醴论》之水肿病。篇中原文虽然未明言其病因,但以“其有不从毫毛而生”作为该肿病分类的界定,就明示该篇所论之水肿病,属于内伤而非外感所致。若结合篇名“汤液醪醴”,及上文“神不使”之论,那么该病之发病原因,当与饮酒过度,情志所伤,以致于损伤五脏,气化失常,特别是肺、脾、肾三脏失常为其主要病理基础,故原文以“五脏阳以竭”,“气拒于内而形施于外”突出其病机。由于五脏全面损伤,所以水湿停聚后所致的浮肿症状也就尤为显著,故原文以“津液充郭”、“四极急而动中”、“形不可与衣相保”句,强调其全身高度浮肿的症状特点。因此,《素问·汤液醪醴论》说:“其有不从毫毛而生,五脏阳以竭也。津液充郭,其魄独居,孤精于内,气耗于外,形不可与衣相保。此四极急而动中,是气拒于内而形施(yī易,改变、变化——编者注)外。治之奈何?岐伯曰:平治于权衡。去宛陈莝,微动四极,温衣,缪刺其处,以复其形;开鬼门,洁净府,精以时服,五阳已布,疏涤五脏。故精自生,形自盛,骨肉相保,巨气乃平。”

本篇下文所论之鼓胀病亦属此类。《内经》有关水肿病的分类理论,成为后世研究水肿病的主要理论基础。

2.肤胀

肤胀,是以外感寒邪,阳气阻遏,水湿留而不行所致的病。寒为阴邪,其性凝滞,收引,易伤阳气。内脏阳气为寒邪所伤;阳不化水,故可致水湿停聚,或留于皮肤肌腠,或滞于腹内。湿阻气机,气郁滞塞,而成肤胀之病。气、水滞留于皮下,则见“身尽肿,皮厚”之症;气、水滞于腹内,则见“腹大”;气,水滞于皮下,按之气散不能猝聚,故按之则“窅而不起”;由于气滞皮下,故其“腹色不变”。杨上善指出:“肤胀,凡有五别:一者,寒气循于卫气,客于皮肤之间;二者,为肿不坚;三者,腹大身肿;四者,皮厚,按之不起;五者,“腹色不变”(《太素·气论》)。其中“按之不起”是与水肿病的鉴别要点;“腹色不变”是与鼓胀病的鉴别要点。张介宾对此颇有见地指出:“以手按其腹,随手而起者属水,窅而不起属气,此固然也。然按气囊者,亦随手而起。又水在肌肉之中,按而散之,猝不能聚,如按糟囊者,亦窅而不起,故未可以起与不起为水、气之的辨,但当察其皮厚色苍,或一身尽肿,或自上而下者,多属气;若皮薄色泽,或肿有分界,或自下而上者,多属水也”(《类经·疾病类》)。

《内经》论肤胀,亦从内伤、外感两分之。如《灵枢·胀论》即是。认为“寒气”致之者,即为外感之“胀”。《素问·至真要大论》还认为外感湿邪、热邪亦可致“胀”。《素问·太阴阳明论》认为内伤饮食也可为“胀”。就其病机而言,外感致之者,多与卫气失常,营卫不和有关,此正如张介宾论:“是以凡病胀者,皆发于卫气也”(《类经·疾病类》);内伤致之者,则以脏腑气机失调为主,临证中又有标本虚实之别,不可不察。

3.鼓胀

鼓胀,是指水湿内停,以腹胀身肿,肤色苍黄,腹壁青筋暴露为特点的病。从“色苍黄”症状分析其病机,当属肝郁犯脾,本土不和,气滞温停,血络瘀阻为其发病机理。水湿浸渍肌肤,故见全身浮肿,水湿停聚,加之肝失疏泄,气机郁滞,气、水充斥腹腔,故见“腹胀身皆大,大与肤胀等”的特点;肝脾不和,故其“色苍黄”;水湿及郁滞不通之气机,停阻于脉络,故有腹壁青筋暴露之“腹筋起”之症。另据《素问·腹中论》记载,“有病心腹满,旦食则不能暮食,此为何病?岐伯对曰:病名为鼓张。”又说:“其时有复发者,何也?岐伯曰:此饮食不节,故时有病也。虽然其病且已时,故当病气聚于腹也。”因此,张介宾注曰:“胀病多反复也。鼓胀之病,本因留滞,故不可复纵饮食也。病虽将愈而复伤其脾,所以气复聚也。”不但提出了饮食不节为鼓胀的主要病因,而且明示该病的反复发作特点。

肤胀与鼓胀的鉴别:肤胀与鼓胀均有腹大身尽肿的共同症状。二者主要的鉴别要点是:肤胀之腹色不变;鼓胀之色苍黄,青筋暴起。因为肤胀以气滞为主,故腹胀而肤色不变;鼓胀是气血瘀滞为主,故腹胀大色苍黄;且因血络瘀阻不畅,故腹部青筋暴起。故李中梓说:“鼓胀与肤胀,大同小异,只以色苍黄,腹筋起为别耳”。

\biaoti{【临证指要】}

水胀治疗。据水胀症状的描述,实质是指风水。对于外感所致的水肿,张仲景最先提出了证治原则时指出“诸有水者,腰以下肿,当利小便;腰以上肿,当发汗乃愈”(《金匮要略·水气病脉证并治》)的治疗原则。对风水表虚证用防己黄芪汤;有郁热者,用越婢汤;脉浮的用杏子汤;脉沉的用麻黄附子细辛汤。对于内伤性水肿,历代医家有以五脏论治的,如仲景有“心水”、“肝水”、“肺水”、“脾水”、“肾水”、“里水”之论。其中张介宾更以肺、脾、肾三脏论之,认为:“凡水肿等证,乃肺、脾、肾三脏相干之病。盖水为至阴,故其本在肾;水化于气,故其标在肺;水惟畏土,故其制在脾。今肺虚则气不化精而化水,脾虚则土不制水而反克,肾虚则水无所主而妄行”(《景岳全书·肿胀》)。说明肺、脾、肾三脏是水肿病形成的关键。此外,心肝失常,亦可致生水肿,肝主疏泄气机,气亦行水。故肝郁气滞,气机郁滞,气不行水,亦可致成水胀。此类水肿,临证多以阴水辨治。

肤胀的治疗,《灵枢·胀论》指出:“无问虚实,工在疾泻。”关键在于泻实去邪。针刺可取足三里,使用泻法。后世多以脏腑辨证治之。如因肺失宣降所致者,治宜宣降肺气,方用越婢加半夏汤;有因肝郁气滞所致,治宜疏肝理气,方用柴胡疏肝散;有因脾虚湿困所致,治当健脾除湿,方用香砂平胃散;有因肾气虚衰所致,治宜温补肾气,方用金匮肾气丸;有因邪滞胃中,宿食不化,阻滞化热所致,治宜消食导滞,方用保和丸、枳实导滞丸,等等。

纵观历代医家之论,鼓胀病多缘于酒食不节、情志所伤、劳欲过度,感染血吸虫,以及黄疸、积聚、积聚病失治。其病机主要涉及肝、脾、肾功能障碍,导致气滞、血瘀、水停,滞积于腹内而成本病。

临证治疗鼓胀时,当先明辨虚实。病之初期,多属实证,可据病情,选用行气、利水、消瘀、化积等治法以消其胀。如为气滞湿阻者,可用柴胡疏肝散合平胃散;若属寒凝气滞者,可选实脾饮加减,以温阳散寒,化湿行水;若为湿热蕴结者,当选中满分消丸,以清热利湿,逐水消胀;若为肝郁脾虚血瘀者,当用当归、赤芍、丹皮、桃仁、红花、丹参、甲珠、白术、泽泻、青皮、牡蛎等药,以活血化瘀,行气利水消胀。若患者体质尚可,正气未衰,或虚之不甚者,可抽放腹水,以治其标,以缓其急。

\biaoti{【原文】}

\begin{yuanwen}
腸覃\sb{1}何如?岐伯曰:寒氣客於腸外,舆衛氣相搏,氣不得榮\sb{2},因有所系\sb{3},癖而內著\sb{4},惡氣\sb{5}乃起,瘜肉\sb{6}乃生。其始生也,大如雞卵,稍以益大,至其成,如懷子之狀,久者離歲\sb{7},按之則堅,推之則移,月事以時下\sb{8},此其候也。

石瘕\sb{9}何如?岐伯曰:石瘕生於胞中\sb{10},寒氣客於子門\sb{11},子門閉塞,氣不得通,惡血當瀉不清\sb{12},衃以留止\sb{13},日以益大,狀如懷子,月事不以時下\sb{14}。皆生於女子,可導而下\sb{15}。

黃帝曰:膚脹,鼓脹,可刺邪\sb{16}?岐伯曰:先瀉其脹之血絡\sb{17}。後調其經,刺去其血絡\sb{18}也。
\end{yuanwen}

\biaoti{【校注】}

\begin{jiaozhu}
  \item 肠覃(xùn训):病名。系指生长于肠外的如菌状的肿瘤。
  \item 气不得荣:气,此指卫气。荣,通“营”,运行之意。
  \item 因有所系:谓寒邪与卫气被束留于局部。
  \item 癖(pǐ劈)而内著(着zhuó):此指寒邪在体内停留。癖,原系积久成习的嗜好,此作积久。著,着的本字,附着不移之意。
  \item 恶气:指寒邪与卫气搏结所产生的一种能形成肠覃的致病因素。
  \item 瘜肉:即寄生的恶肉。因其非正常的肉,故称“恶肉”。
  \item 久者离岁:指病程长,超过一年。
  \item 月事以时下:指月经能按时来潮。
  \item 石瘕:病名。寒气侵入胞官,恶血停积而成的肿块,质硬如石,故名石瘕。
  \item 胞中:子官内。胞,指子宫。
  \item 子门:即子宫口。
  \item 气不得通,恶血当泻不泻:马詩注:“寒气客于子门,子门闭塞,气不得通子外,恶血之在内,当泻不泻。”
  \item 衃(pēi胚)以留止:指凝集之败血滞留不得出。衃,凝固而呈紫黑色之死血。
  \item 月事不以时下:即月经不能按时来潮。
  \item 可导而下:指用破血逐瘀的方法治疗。
  \item 邪:同“耶”,疑问词。
  \item 先泻其胀之血络:《甲乙经》、《太素》均作“先刺其腹之血络。”腹之血络,指腹壁胀起之血络。
  \item 刺去其血络:《甲乙经》、《太素》均作“亦刺去其血脉,”与上文“先泻”相对应,则有“后刺”之意。
\end{jiaozhu}

\biaoti{【理论阐释】}

肠蕈和石瘕,属于积聚病的范畴。积聚,是以腹内有积块,或痛或胀为主要特征的一类疾病。《内经》中有近30篇经文涉及于此,尤其以本篇以及《灵枢·刺节真邪》、《灵枢·胀论》、《素问·奇病论》、《素问·腹中论》、《灵枢·百病始生》等篇的论述最为集中而深刻。《内经》所论积聚有三类:其一为“瘕”类病症。此类又有水瘕、石瘕、血瘕;其二为“积”类疾病。此类有伏梁、肥气、息贲、肠覃、奔豚。其三为“瘤”类疾病。此类有筋瘤、肠瘤、昔瘤。《内经》认为“积”和“聚”都是有体实质性包块的疾病。自《难经·五十五难》以后,才将二者分而待之。认为腹腔中的包块固定不移,形状大小可以触摸者,属于积病,病位在血分,属于五脏失常所致;认为腹腔中的包块散聚无定者,属于聚病,病程在气分,属于六腑失常所致。

《内经》认为积聚形成的病因病机主要有三:一是外感寒邪,气滞津停血瘀而成。《灵枢·百病始生》篇说:“积之始生,得寒乃生。”又说:“寒,汁沫与血相搏,则并和凝聚不得散,而积成矣。”《素问·举痛论》及本篇,均持此论。因为寒性收引凝滞,而人体之津液、气、血皆有“喜温而恶寒,寒则泣不能流”(《素问·调经论》)的特性。所以,寒邪所伤,久留不去,就会引起气滞、津停、血凝的病理变化,三者互结,日久成积。本篇所论肠覃、石瘕之成因,皆如此。二是七情刺激,气滞、津停、血瘀而成。《灵枢·百病始生》篇说:“若内伤于忧怒,则气上逆,气上逆则六输不通,温气不行,凝血蕴里而不散,津液涩渗,著而不去,而积皆成矣。”大量的临床资料显示,不良的情绪刺激是肿瘤形成的重要因素之一,尤其是胃肠肿瘤、肝肿瘤、子宫肌瘤、乳腺肿瘤、卵巢肿瘤的形成,与精神因素有明显的相关性。其三是饮食不节,起居失常,劳倦太过,损伤肝、脾、肾,致使气血郁滞,水湿停聚而成积病。可见,积病总以气机郁滞、水湿停聚、血行瘀阻为其主要病机。三者互相影响,相互作用,常呈恶性循环,愈积愈甚,因此本病表现为渐进性发展和进行性加重。

本节所论之肠覃与石瘕,均系女子下腹部的肿瘤,属于妇科癥瘕病的范畴。故李中梓说:“此二证惟女人有之,故曰皆生于女子也”(《内经知要·病能》)。其中肠覃是因寒气与卫气相搏,病变是以气凝为主,病位在于肠外及子门之外的下腹部肿块。石瘕是寒气凝滞经血,使恶血不去,衃血留止,病变以血瘀为主,病位在胞中。肠覃与石瘕均有腹部胀大,按之坚硬的症状,但肠覃病变不在子宫内,不影响月经,故月事按时来潮;石瘕病变在子宫内,对月经有影响,故月事不以时下。这是二者的主要鉴别点。

\biaoti{【临证指要】}

肠覃和石瘕病的治疗,本篇指出“可导而下”的原则。杨上善谓此可“针刺导下之”,张介宾谓此指“导血之剂下之”(《类经·疾病类》),丹波元简又谓:“导,谓坐导药,其病在胞中,故用坐药以导下之”(《灵枢识·卷五》)。盖导者,消导之意也。导而下之,即以消导通下之法治之。即《素问·至真要大论》所谓“坚者削之”,“留者攻之”。对此寒凝气血、瘀滞积块的病症,法当消导与通下,或内服活血逐瘀消积之药,或外用坐药治之。

\xiaojie

本章虽然仅选《内经》病证内容8篇,但却能从中窥视《内经》疾病观之一斑。就本章所论疾病而言,包括了外感病、内伤病、肢体病、内脏病、妇科病、五官疾病、疮疡病7类。就疾病的命名而言,基本反映了《内经》疾病的命名规律,其中有以病因为据命名的,如伤寒病、温病、暑病、劳风等;有以病机为据命名的,如阴阳交、痹病等;有以主症为据命名的,如咳病、痛病、痿病、水胀、肤胀、鼓胀等;有以病因与病位结合命名的,如心风、肝风、首风等;有以主症与病位结合命名的,如心咳、肝咳、胃咳、三焦咳等;有以病机与病位结合命名的,如五脏痹、六腑痹等;有以疾病的性质命名的,如寒中、热中、寒痹、热痹等;有以疾病的某些特征命名的,如漏泄、疠风等。详细描述了各种病证的临床表现、特征、分类,并对有关病因、病机、诊断以及防治原则进行了阐述。对中医临床医学的发展奠定了基础。

\zuozhe{(张登本)}
\ifx \allfiles \undefined
\end{document}
\fi