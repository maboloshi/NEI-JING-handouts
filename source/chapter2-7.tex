% -*- coding: utf-8 -*-
%!TEX program = xelatex
\ifx\allfiles\undefined
\documentclass[draft,12pt]{ctexbook}
%\usepackage{xeCJK}
%\usepackage[14pt]{extsizes} %支持8,9,10,11,12,14,17,20pt

%===================文档页面设置====================
%---------------------印刷版尺寸--------------------
%\usepackage[a4paper,hmargin={2.3cm,1.7cm},vmargin=2.3cm,driver=xetex]{geometry}
%--------------------电子版------------------------
\usepackage[a4paper,margin=2cm,driver=xetex]{geometry}
%\usepackage[paperwidth=9.2cm, paperheight=12.4cm, width=9cm, height=12cm,top=0.2cm,
%            bottom=0.4cm,left=0.2cm,right=0.2cm,foot=0cm, nohead,nofoot,driver=xetex]{geometry}

%===================自定义颜色=====================
\usepackage{xcolor}
  \definecolor{mybackgroundcolor}{cmyk}{0.03,0.03,0.18,0}
  \definecolor{myblue}{rgb}{0,0.2,0.6}

%====================字体设置======================
%--------------------中文字体----------------------
%-----------------------xeCJK下设置中文字体------------------------------%
\setCJKfamilyfont{song}{SimSun}                             %宋体 song
\newcommand{\song}{\CJKfamily{song}}                        % 宋体   (Windows自带simsun.ttf)
\setCJKfamilyfont{xs}{NSimSun}                              %新宋体 xs
\newcommand{\xs}{\CJKfamily{xs}}
\setCJKfamilyfont{fs}{FangSong_GB2312}                      %仿宋2312 fs
\newcommand{\fs}{\CJKfamily{fs}}                            %仿宋体 (Windows自带simfs.ttf)
\setCJKfamilyfont{kai}{KaiTi_GB2312}                        %楷体2312  kai
\newcommand{\kai}{\CJKfamily{kai}}
\setCJKfamilyfont{yh}{Microsoft YaHei}                    %微软雅黑 yh
\newcommand{\yh}{\CJKfamily{yh}}
\setCJKfamilyfont{hei}{SimHei}                                    %黑体  hei
\newcommand{\hei}{\CJKfamily{hei}}                          % 黑体   (Windows自带simhei.ttf)
\setCJKfamilyfont{msunicode}{Arial Unicode MS}            %Arial Unicode MS: msunicode
\newcommand{\msunicode}{\CJKfamily{msunicode}}
\setCJKfamilyfont{li}{LiSu}                                            %隶书  li
\newcommand{\li}{\CJKfamily{li}}
\setCJKfamilyfont{yy}{YouYuan}                             %幼圆  yy
\newcommand{\yy}{\CJKfamily{yy}}
\setCJKfamilyfont{xm}{MingLiU}                                        %细明体  xm
\newcommand{\xm}{\CJKfamily{xm}}
\setCJKfamilyfont{xxm}{PMingLiU}                             %新细明体  xxm
\newcommand{\xxm}{\CJKfamily{xxm}}

\setCJKfamilyfont{hwsong}{STSong}                            %华文宋体  hwsong
\newcommand{\hwsong}{\CJKfamily{hwsong}}
\setCJKfamilyfont{hwzs}{STZhongsong}                        %华文中宋  hwzs
\newcommand{\hwzs}{\CJKfamily{hwzs}}
\setCJKfamilyfont{hwfs}{STFangsong}                            %华文仿宋  hwfs
\newcommand{\hwfs}{\CJKfamily{hwfs}}
\setCJKfamilyfont{hwxh}{STXihei}                                %华文细黑  hwxh
\newcommand{\hwxh}{\CJKfamily{hwxh}}
\setCJKfamilyfont{hwl}{STLiti}                                        %华文隶书  hwl
\newcommand{\hwl}{\CJKfamily{hwl}}
\setCJKfamilyfont{hwxw}{STXinwei}                                %华文新魏  hwxw
\newcommand{\hwxw}{\CJKfamily{hwxw}}
\setCJKfamilyfont{hwk}{STKaiti}                                    %华文楷体  hwk
\newcommand{\hwk}{\CJKfamily{hwk}}
\setCJKfamilyfont{hwxk}{STXingkai}                            %华文行楷  hwxk
\newcommand{\hwxk}{\CJKfamily{hwxk}}
\setCJKfamilyfont{hwcy}{STCaiyun}                                 %华文彩云 hwcy
\newcommand{\hwcy}{\CJKfamily{hwcy}}
\setCJKfamilyfont{hwhp}{STHupo}                                 %华文琥珀   hwhp
\newcommand{\hwhp}{\CJKfamily{hwhp}}

\setCJKfamilyfont{fzsong}{Simsun (Founder Extended)}     %方正宋体超大字符集   fzsong
\newcommand{\fzsong}{\CJKfamily{fzsong}}
\setCJKfamilyfont{fzyao}{FZYaoTi}                                    %方正姚体  fzy
\newcommand{\fzyao}{\CJKfamily{fzyao}}
\setCJKfamilyfont{fzshu}{FZShuTi}                                    %方正舒体 fzshu
\newcommand{\fzshu}{\CJKfamily{fzshu}}

\setCJKfamilyfont{asong}{Adobe Song Std}                        %Adobe 宋体  asong
\newcommand{\asong}{\CJKfamily{asong}}
\setCJKfamilyfont{ahei}{Adobe Heiti Std}                            %Adobe 黑体  ahei
\newcommand{\ahei}{\CJKfamily{ahei}}
\setCJKfamilyfont{akai}{Adobe Kaiti Std}                            %Adobe 楷体  akai
\newcommand{\akai}{\CJKfamily{akai}}

%------------------------------设置字体大小------------------------%
\newcommand{\chuhao}{\fontsize{42pt}{\baselineskip}\selectfont}     %初号
\newcommand{\xiaochuhao}{\fontsize{36pt}{\baselineskip}\selectfont} %小初号
\newcommand{\yihao}{\fontsize{28pt}{\baselineskip}\selectfont}      %一号
\newcommand{\xiaoyihao}{\fontsize{24pt}{\baselineskip}\selectfont}
\newcommand{\erhao}{\fontsize{21pt}{\baselineskip}\selectfont}      %二号
\newcommand{\xiaoerhao}{\fontsize{18pt}{\baselineskip}\selectfont}  %小二号
\newcommand{\sanhao}{\fontsize{15.75pt}{\baselineskip}\selectfont}  %三号
\newcommand{\sihao}{\fontsize{14pt}{\baselineskip}\selectfont}%     四号
\newcommand{\xiaosihao}{\fontsize{12pt}{\baselineskip}\selectfont}  %小四号
\newcommand{\wuhao}{\fontsize{10.5pt}{\baselineskip}\selectfont}    %五号
\newcommand{\xiaowuhao}{\fontsize{9pt}{\baselineskip}\selectfont}   %小五号
\newcommand{\liuhao}{\fontsize{7.875pt}{\baselineskip}\selectfont}  %六号
\newcommand{\qihao}{\fontsize{5.25pt}{\baselineskip}\selectfont}    %七号   %中文字体及字号设置
\xeCJKDeclareSubCJKBlock{SIP}{
  "20000 -> "2A6DF,   % CJK Unified Ideographs Extension B
  "2A700 -> "2B73F,   % CJK Unified Ideographs Extension C
  "2B740 -> "2B81F    % CJK Unified Ideographs Extension D
}
%\setCJKmainfont[SIP={[AutoFakeBold=1.8,Color=red]Sun-ExtB},BoldFont=黑体]{宋体}    % 衬线字体 缺省中文字体

\setCJKmainfont{simsun.ttc}[
  Path=fonts/,
  SIP={[Path=fonts/,AutoFakeBold=1.8,Color=red]simsunb.ttf},
  BoldFont=simhei.ttf
]

%SimSun-ExtB
%Sun-ExtB
%AutoFakeBold:自动伪粗,即正文使用\bfseries时生僻字使用伪粗体;
%FakeBold:强制伪粗,即正文中生僻字均使用伪粗体
%\setCJKmainfont[BoldFont=STHeiti,ItalicFont=STKaiti]{STSong}
%\setCJKsansfont{微软雅黑}黑体
%\setCJKsansfont[BoldFont=STHeiti]{STXihei} %serif是有衬线字体sans serif 无衬线字体
%\setCJKmonofont{STFangsong}    %中文等宽字体

%--------------------英文字体----------------------
\setmainfont{simsun.ttc}[
  Path=fonts/,
  BoldFont=simhei.ttf
]
%\setmainfont[BoldFont=黑体]{宋体}  %缺省英文字体
%\setsansfont
%\setmonofont

%===================目录分栏设置====================
\usepackage[toc,lof,lot]{multitoc}    % 目录(含目录、表格目录、插图目录)分栏设置
  %\renewcommand*{\multicolumntoc}{3} % toc分栏数设置,默认为两栏(\multicolumnlof,\multicolumnlot)
  %\setlength{\columnsep}{1.5cm}      % 调整分栏间距
  \setlength{\columnseprule}{0.2pt}   % 调整分栏竖线的宽度

%==================章节格式设置====================
\setcounter{secnumdepth}{3} % 章节等编号深度 3:子子节\subsubsection
\setcounter{tocdepth}{2}    % 目录显示等度 2:子节

\xeCJKsetup{%
  CJKecglue=\hspace{0.15em},      % 调整中英(含数字)间的字间距
  %CJKmath=true,                  % 在数学环境中直接输出汉字(不需要\text{})
  AllowBreakBetweenPuncts=true,   % 允许标点中间断行,减少文字行溢出
}

\ctexset{%
  part={
    name={,篇},
    number=\SZX{part},
    format={\chuhao\bfseries\centering},
    nameformat={},titleformat={}
  },
  section={
    number={\chinese{section}},
    name={第,节}
  },
  subsection={
    number={\chinese{subsection}、},
    aftername={\hspace{-0.01em}}
  },
  subsubsection={
    number={(\chinese{subsubsection})},
    aftername={\hspace {-0.01em}},
    beforeskip={1.3ex minus .8ex},
    afterskip={1ex minus .6ex},
    indent={\parindent}
  },
  paragraph={
    beforeskip=.1\baselineskip,
    indent={\parindent}
  }
}

\newcommand*\SZX[1]{%
  \ifcase\value{#1}%
    \or 上%
    \or 中%
    \or 下%
  \fi
}

%====================页眉设置======================
\usepackage{titleps}%或者\usepackage{titlesec},titlesec包含titleps
\newpagestyle{special}[\small\sffamily]{
  %\setheadrule{.1pt}
  \headrule
  \sethead[\usepage][][\chaptertitle]
  {\chaptertitle}{}{\usepage}
}

\newpagestyle{main}[\small\sffamily]{
  \headrule
  %\sethead[\usepage][][第\thechapter 章\quad\chaptertitle]
%  {\thesection\quad\sectiontitle}{}{\usepage}}
  \sethead[\usepage][][第\chinese{chapter}章\quad\chaptertitle]
  {第\chinese{section}节\quad\sectiontitle}{}{\usepage}
}

\newpagestyle{main2}[\small\sffamily]{
  \headrule
  \sethead[\usepage][][第\chinese{chapter}章\quad\chaptertitle]
  {第\chinese{section}節\quad\sectiontitle}{}{\usepage}
}

%================ PDF 书签设置=====================
\usepackage{bookmark}[
  depth=2,        % 书签深度 2:子节
  open,           % 默认展开书签
  openlevel=2,    % 展开书签深度 2:子节
  numbered,       % 显示编号
  atend,
]
  % 相比hyperref,bookmark宏包大多数时候只需要编译一次,
  % 而且书签的颜色和字体也可以定制。
  % 比hyperref 更专业 (自动加载hyperref)

%\bookmarksetup{italic,bold,color=blue} % 书签字体斜体/粗体/颜色设置

%------------重置每篇章计数器,必须在hyperref/bookmark之后------------
\makeatletter
  \@addtoreset{chapter}{part}
\makeatother

%------------hyperref 超链接设置------------------------
\hypersetup{%
  pdfencoding=auto,   % 解决新版ctex,引起hyperref UTF-16预警
  colorlinks=true,    % 注释掉此项则交叉引用为彩色边框true/false
  pdfborder=001,      % 注释掉此项则交叉引用为彩色边框
  citecolor=teal,
  linkcolor=myblue,
  urlcolor=black,
  %psdextra,          % 配合使用bookmark宏包,可以直接在pdf 书签中显示数学公式
}

%------------PDF 属性设置------------------------------
\hypersetup{%
  pdfkeywords={黄帝内经,内经,内经讲义,21世纪课程教材},    % 关键词
  %pdfsubject={latex},        % 主题
  pdfauthor={主编:王洪图},   % 作者
  pdftitle={内经讲义},        % 标题
  %pdfcreator={texlive2011}   % pdf创建器
}

%------------PDF 加密----------------------------------
%仅适用于xelatex引擎 基于xdvipdfmx
%\special{pdf:encrypt ownerpw (abc) userpw (xyz) length 128 perm 2052}

%仅适用于pdflatex引擎
%\usepackage[owner=Donald,user=Knuth,print=false]{pdfcrypt}

%其他可使用第三方工具 如:pdftk
%pdftk inputfile.pdf output outputfile.pdf encrypt_128bit owner_pw yourownerpw user_pw youruserpw

%=============自定义环境、列表及列表设置================
% 标题
\def\biaoti#1{\vspace{1.7ex plus 3ex minus .2ex}{\bfseries #1}}%\noindent\hei
% 小标题
\def\xiaobt#1{{\bfseries #1}}
% 小结
\def\xiaojie {\vspace{1.8ex plus .3ex minus .3ex}\centerline{\large\bfseries 小\ \ 结}\vspace{.1\baselineskip}}
% 作者
\def\zuozhe#1{\rightline{\bfseries #1}}

\newcounter{yuanwen}    % 新计数器 yuanwen
\newcounter{jiaozhu}    % 新计数器 jiaozhu

\newenvironment{yuanwen}[2][【原文】]{%
  %\biaoti{#1}\par
  \stepcounter{yuanwen}   % 计数器 yuanwen+1
  \bfseries #2}
  {}

\usepackage{enumitem}
\newenvironment{jiaozhu}[1][【校注】]{%
  %\biaoti{#1}\par
  \stepcounter{jiaozhu}   % 计数器 jiaozhu+1
  \begin{enumerate}[%
    label=\mylabel{\arabic*}{\circledctr*},before=\small,fullwidth,%
    itemindent=\parindent,listparindent=\parindent,%labelsep=-1pt,%labelwidth=0em,
    itemsep=0pt,topsep=0pt,partopsep=0pt,parsep=0pt
  ]}
  {\end{enumerate}}

%===================注解与原文相互跳转====================
%----------------第1部分 设置相互跳转锚点-----------------
\makeatletter
  \protected\def\mylabel#1#2{% 注解-->原文
    \hyperlink{back:\theyuanwen:#1}{\Hy@raisedlink{\hypertarget{\thejiaozhu:#1}{}}#2}}

  \protected\def\myref#1#2{% 原文-->注解
    \hyperlink{\theyuanwen:#1}{\Hy@raisedlink{\hypertarget{back:\theyuanwen:#1}{}}#2}}
  %此处\theyuanwen:#1实际指thejiaozhu:#1,只是\thejiaozhu计数器还没更新,故使用\theyuanwen计数器代替
\makeatother

\protected\def\myjzref#1{% 脚注中的引用(引用到原文)
  \hyperlink{\theyuanwen:#1}{\circlednum{#1}}}

\def\sb#1{\myref{#1}{\textsuperscript{\circlednum{#1}}}}    % 带圈数字上标

%----------------第2部分 调整锚点垂直距离-----------------
\def\HyperRaiseLinkDefault{.8\baselineskip} %调整锚点垂直距离
%\let\oldhypertarget\hypertarget
%\makeatletter
%  \def\hypertarget#1#2{\Hy@raisedlink{\oldhypertarget{#1}{#2}}}
%\makeatother

%====================带圈数字列表标头====================
\newfontfamily\circledfont[Path = fonts/]{meiryo.ttc}  % 日文字体,明瞭体
%\newfontfamily\circledfont{Meiryo}  % 日文字体,明瞭体

\protected\def\circlednum#1{{\makexeCJKinactive\circledfont\textcircled{#1}}}

\newcommand*\circledctr[1]{%
  \expandafter\circlednum\expandafter{\number\value{#1}}}
\AddEnumerateCounter*\circledctr\circlednum{1}

% 参考自:http://bbs.ctex.org/forum.php?mod=redirect&goto=findpost&ptid=78709&pid=460496&fromuid=40353

%======================插图/tikz图========================
\usepackage{graphicx,subcaption,wrapfig}    % 图,subcaption含子图功能代替subfig,图文混排
  \graphicspath{{img/}}                     % 设置图片文件路径

\def\pgfsysdriver{pgfsys-xetex.def}         % 设置tikz的驱动引擎
\usepackage{tikz}
  \usetikzlibrary{calc,decorations.text,arrows,positioning}

%---------设置tikz图片默认格式(字号、行间距、单元格高度)-------
\let\oldtikzpicture\tikzpicture
\renewcommand{\tikzpicture}{%
  \small
  \renewcommand{\baselinestretch}{0.2}
  \linespread{0.2}
  \oldtikzpicture
}

%=========================表格相关===============================
\usepackage{%
  multirow,                   % 单元格纵向合并
  array,makecell,longtable,   % 表格功能加强,tabu的依赖
  tabu-last-fix,              % "强大的表格工具" 本地修复版
  diagbox,                    % 表头斜线
  threeparttable,             % 表格内脚注(需打补丁支持tabu,longtabu)
}

%----------给threeparttable打补丁用于tabu,longtabu--------------
%解决方案来自:http://bbs.ctex.org/forum.php?mod=redirect&goto=findpost&ptid=80318&pid=467217&fromuid=40353
\usepackage{xpatch}

\makeatletter
  \chardef\TPT@@@asteriskcatcode=\catcode`*
  \catcode`*=11
  \xpatchcmd{\threeparttable}
    {\TPT@hookin{tabular}}
    {\TPT@hookin{tabular}\TPT@hookin{tabu}}
    {}{}
  \catcode`*=\TPT@@@asteriskcatcode
\makeatother

%------------设置表格默认格式(字号、行间距、单元格高度)------------
\let\oldtabular\tabular
\renewcommand{\tabular}{%
  \renewcommand\baselinestretch{0.9}\small    % 设置行间距和字号
  \renewcommand\arraystretch{1.5}             % 调整单元格高度
  %\renewcommand\multirowsetup{\centering}
  \oldtabular
}
%设置行间距,且必须放在字号设置前 否则无效
%或者使用\fontsize{<size>}{<baseline>}\selectfont 同时设置字号和行间距

\let\oldtabu\tabu
\renewcommand{\tabu}{%
  \renewcommand\baselinestretch{0.9}\small    % 设置行间距和字号
  \renewcommand\arraystretch{1.8}             % 调整单元格高度
  %\renewcommand\multirowsetup{\centering}
  \oldtabu
}

%------------模仿booktabs宏包的三线宽度设置---------------
\def\toprule   {\Xhline{.08em}}
\def\midrule   {\Xhline{.05em}}
\def\bottomrule{\Xhline{.08em}}
%-------------------------------------
%\setlength{\arrayrulewidth}{2pt} 设定表格中所有边框的线宽为同样的值
%\Xhline{} \Xcline{}分别设定表格中水平线的宽度 makecell包提供

%表格中垂直线的宽度可以通过在表格导言区(preamble),利用命令 !{\vrule width1.2pt} 替换 | 即可

%=================图表设置===============================
%---------------图表标号设置-----------------------------
\renewcommand\thefigure{\arabic{section}-\arabic{figure}}
\renewcommand\thetable {\arabic{section}-\arabic{table}}

\usepackage{caption}
  \captionsetup{font=small,}
  \captionsetup[table] {labelfont=bf,textfont=bf,belowskip=3pt,aboveskip=0pt} %仅表格 top
  \captionsetup[figure]{belowskip=0pt,aboveskip=3pt}  %仅图片 below

%\setlength{\abovecaptionskip}{3pt}
%\setlength{\belowcaptionskip}{3pt} %图、表题目上下的间距
\setlength{\intextsep}   {5pt}  %浮动体和正文间的距离
\setlength{\textfloatsep}{5pt}

%====================全文水印==========================
%解决方案来自:
%http://bbs.ctex.org/forum.php?mod=redirect&goto=findpost&ptid=79190&pid=462496&fromuid=40353
%https://zhuanlan.zhihu.com/p/19734756?columnSlug=LaTeX
\usepackage{eso-pic}

%eso-pic中\AtPageCenter有点水平偏右
\renewcommand\AtPageCenter[1]{\parbox[b][\paperheight]{\paperwidth}{\vfill\centering#1\vfill}}

\newcommand{\watermark}[3]{%
  \AddToShipoutPictureBG{%
    \AtPageCenter{%
      \tikz\node[%
        overlay,
        text=red!50,
        %font=\sffamily\bfseries,
        rotate=#1,
        scale=#2
      ]{#3};
    }
  }
}

\newcommand{\watermarkoff}{\ClearShipoutPictureBG}

\watermark{45}{15}{草\ 稿}    %启用全文水印

%=============花括号分支结构图=========================
\usepackage{schemata}

\xpatchcmd{\schema}
  {1.44265ex}{-1ex}
  {}{}

\newcommand\SC[2] {\schema{\schemabox{#1}}{\schemabox{#2}}}
\newcommand\SCh[4]{\Schema{#1}{#2}{\schemabox{#3}}{\schemabox{#4}}}

%=======================================================

\begin{document}
\pagestyle{main2}
\fi
\chapter{论治}%第七章
论治是《内经》学术体系的重要组成部分。除了疾病治疗思想外,主要包括治则和治法两部分内容。

治则,即是治疗疾病的法则与准绳,如《内经》的治病求本、本标先后、协调阴阳、扶正祛邪、因势利导是治疗疾病的总法则,而针对疾病表里的解表清里、针对气血的补气活血、针对病情微甚的正治反治等则是治病的具体法则。治法,即治疗疾病的方法与手段。《内经》中治法非常丰富,主要包括针灸疗法,药物疗法,饮食疗法,导引、按蹻、吐纳,精神疗法。其它诸如手术疗法,药熨,溃浴,束指,吹耳,刺鼻,饥饿,冷疗,负重运动等皆有记载,散见于《内经》多篇,主要有:《素问》的《四气调神大论》、《阴阳应象大论》、《生气通天论》、《移精变气论》、《异法方宜论》、《汤液醪醴论》、《脏气法时论》、《标本病传论》、《八正神明论》、《宝命全形论》、《病能论》、《调经论》、《至真要大论》、《五常政大论》、《灵枢》的《九针十二原》、《逆顺》、《官能》、《经脉》、《终始》、《四时气》、《顺气一日分为四时》、《刺节真邪》、《水胀》等篇。

\section{素問·陰陽應象大論(節選)}%第一節

\biaoti{【原文】}

\begin{yuanwen}
故曰:病之始起也,可刺而已;其盛,可待衰而已\sb{1}。故其輕而揚之\sb{2},因其重而減之\sb{3},因其衰而彰之\sb{4}。形不足者,溫之以氣;精不足者,補之以味\sb{5}。其髙者,因而越之\sb{6};其下者,引而竭之\sb{7};中滿者,寫之於內\sb{8}。其有邪者,漬形以爲汗\sb{9};其在皮者,汗而發之\sb{10};其慓悍者,按而收之\sb{11};其實者,散而寫之\sb{12}。審其陰陽,以別柔剛\sb{13},陽病治陰,陰病治陽\sb{14}。定其血氣\sb{15},各守其鄉。血實宜決之\sb{16},氣虛宜引之\sb{17}。
\end{yuanwen}

\biaoti{【校注】}

\begin{jiaozhu}
	\item 其盛,可待衰而已:在某些特殊情况下,邪势太盛,不宜用针刺直接攻邪,应等待病势稍衰而后刺之。
	\item 因其轻而扬之:轻,是指病邪质轻。扬,轻扬宣散之意。张介宾注:“轻者浮于表,故宜扬之。扬者散也”。
	\item 因其重而减之:重,病邪重浊。减,逐步减轻之意。张介宾:“重者实于内,故宜减之。减者泻也。”
	\item 因其衰而彰之:指正衰,用补益法彰之。
	\item 形不足者,温之以气;精不足者,补之以味:张介宾注:“以形精言,则形为阳,精为阴;以气味言,则气为阳,味为阴。阳者卫外而为固也,阴者藏精而起亟也。故形不足者,阳之衰也,非气不足以达表而温之;精不足者,阴之衰也,非味不足以实中而补之。阳性暖,故曰温,阴性静,故曰补。
	\item 其高者,因而越之:越之,这里指的是吐法。吴昆云:“高,胸之上也。越之,吐之也。此宜于吐,故吐之。”
	\item 其下者,因而竭之:吴昆云:“下,脐之下也。或利小便,或通其大便,皆引而竭之。竭,尽也。”
	\item 中满者,泻之于内:中焦痞满,应从内部消散病邪,即消法。吴昆云:“此不在高,不在下,故不可越,亦不可竭,但泻之于内,消其坚满是也。”一说《伤寒论》泻心汤是其例。
	\item 渍形以为汗:以汤液浸渍使其出汗,包括薰蒸、浸浴等治法。
	\item 其在皮者,汗而发之:邪在皮肤,取汗而发散之。
	\item 其慓悍者,按而收之:慓悍,指邪气急猛。按,抑制。收,收敛,制伏的意思。张介宾注:“凡邪气之急利者,按得其状,则可收而制之矣。”
	\item 其实者,散而泻之:吴昆云:“表实则散,里实则泻。”
	\item 柔刚:柔属阴,刚属阳,即阴阳之意。张介宾注:“形证有柔刚,气味尤有柔刚。柔者属阴,刚者属阳,知柔刚之化者,知阴阳之妙用矣,故必审而别之。”
	\item 阳病治阴,阴病治阳:张介宾注:“阳胜者阴必病,阴胜者阳必病。如《至真要大论》曰:诸寒之而热者取之阴,热之而寒者取之阳。启玄子曰:壮水之主,以制阳光;益火之源,以消阴翳。皆阳病治阴,阳病治阳之道也。”
	\item 定其血气,各守其乡:诸经皆有血气,宜安定之,使之各守其位,不得出位乘侮。
	\item 决之:指放血之法。
	\item 气虚宜引之:引,即是升提补气法。张介宾注:“,《甲乙经》作掣,挽也。气虚者,无气之渐,无气则死矣,故当挽回其气而引之使复也。如上气虚者升而举之,下气虚者纳而归之,中气虚者温而补之,是皆掣引之义也。”
\end{jiaozhu}

\biaoti{【理论阐释】}

\xiaobt{因势利导治疗原则}

“因势利导”作为《内经》的治疗原则之一,以本段记载和论述最为突出,它包含两方面的含义。

一是根据邪气的部位施治,尤其是以实邪为主的病证,应根据邪气所在部位和性质而采取相应措施,使之从最简捷的途径,以最快的速度排出体外,以免病邪深入而过多地损伤正气。如本段所云:“因其轻而扬之,因其重而减之……其高者因而越之,其下者引而竭之,中满者泻之于内,其有邪者渍形以为汗,其在皮者汗而发之。”说明因邪气质轻,而用扬散之法,如风邪宣散之类;邪气性质重浊者,而用逐渐衰减之法,如湿邪可淡渗,癥瘕宜消坚之类;邪之部位在上焦(高)者,因其在上之势,越而出之,如涌吐之类;邪居下焦(下)者,因其在下之势,引而下出,如利尿、攻逐、导便,灌肠之类;中脘痞满者,则分消于内而泻之,如仲景泻心汤之类;邪在皮、在表,则因其在外之势,而或用汤渍或用药取汗,如发散风寒表邪之类。

二是根据邪正盛衰而择时治疗。尤其是对某些周期性发作的疾病,应在其未发病之前治疗,因为这个阶段的邪气较弱,正气相对旺盛。如能给以适宜的治疗,则可收到良好的治疗效果。如《素问·疟论》说:“方其盛时必毁,因其衰也,事必大昌”。《灵枢·逆顺》:“无刺熇熇之热,无刺漉漉之汗,无刺浑浑之脉”,“方其盛也,勿敢毁伤,刺其已衰,事必大昌”。言邪气猖厥之时,暂无施治,待其病气衰再行治疗,才能取得好的效果。本段也说:“其盛,可待衰而已”。当然,“待衰”而治只是特殊病证具有反复发作,时轻时重特征者而言。

\biaoti{【临证指要】}

\xiaobt{张机对“因势利导”治则的运用}

医圣张仲景“勤求古训,博采众方”,撰著《伤寒杂病论》而为后世垂范,于今所见《伤寒论》和《金匮要略》二者各种治病原则,皆本于《黄帝内经》,尤以《阴阳应象大论》本段所载“因势利导”的运用最为突出。

“汗而发之”。应用于在皮、在表之病,张仲景凡用汗法,必有表证存在。如《伤寒论》51条云:“脉浮,病在表,可发汗。”根据病情分别设有麻黄汤、桂枝汤、葛根汤等方。

“其高者,因而越之”。实邪停于上焦,应用吐法上越而出。仲景书中论及吐法虽仅数条,却明确地指出了邪气所在部位,病邪性质及临床症状特点,并以瓜蒂散为催吐代表方剂。《伤寒论》171条云:“……寸脉微浮,胸中痞硬,气上冲咽喉不得息者,此为胸中有寒(邪)也,当吐之,宜瓜蒂散。”实邪位居胸中,因其“高”而用吐法,其配伍正是运用本篇前文所说:“酸苦涌泻为阴”之理,而以味苦之瓜蒂,味酸之赤小豆组成。

“其下者,引而竭之。”实邪位于下焦,因其在下之势,用通利二便之法以排出之。如《伤寒论》381条云:“伤寒哕而腹满,视其前后,知何部不利,利之即愈。”其用内服药从大便泻出者,包括胃肠中有燥屎之承气汤证,有下焦蓄血之抵当汤证,有邪气由阳入结于里的大陷胸汤等证,有“留饮”不去的甘遂半夏汤等证。利小便之法,有膀胱蓄水的五苓散证。药物“引”导,从下排出实邪法,制有蜜煎导(坐药),土瓜根、猪胆汁及醋灌肠法。

“中满者,泻之于内”。中焦气机转枢不利,引起心下胀满痞闷,治疗当调气机以消除痞满。仲景据《内经》这一原则,又结合痞满的寒热虚实不同性质,而制有诸“泻心汤”。方名称“泻”者,正取本段经文“泻之于内”的“泻”;“心”,即是“心下”,亦即指胃脘,而属于中焦。仲景以泻心汤治疗心下痞,正是本《内经》“中满者,泻之于内”之旨。

《素问·疟论》的“因其衰也,事必大昌”,主要用于周期发作性疾病,在病未发作之前,亦即邪气较弱之时施治。《伤寒论》54条云:“病人脏无他病,时发热自汗出而不愈者。此卫气不和也,先其时发汗则愈,宜桂枝汤。”《金匮要略》用蜀漆散治牡疟,于“未发前,以浆水服半钱”;治温疟,于“临发前,服一钱匕”皆属此意。

\section{素問·異法方宜論}%第二節

\biaoti{【原文】}

\begin{yuanwen}
黃帝問曰:醫之治病也,一病而治各不同,皆愈何也?岐伯對曰:地勢使然\sb{1}也。故東方之域,天地之所始生\sb{2}也。魚鹽之地,海濱傍水。其民食魚而嗜鹹,皆安其處,美其食。魚者使人熱中\sb{3},鹽者勝血\sb{4}。故其民皆黑色踈理\sb{5},其病皆爲癰瘍。其治宜砭石。故砭石者\sb{6},亦從東方來。
\end{yuanwen}

\biaoti{【校注】}

\begin{jiaozhu}
	\item 地势使然:地势,指地表高低起伏的状态和位于地表面所有固定性物体的总体,如居民地、道路、江河、森林等。地势不同,气温有别,而发生的疾病各有特点,《素问·五常政大论》曰:“地有高下,气有温凉,高者气寒,下者气热。故适寒凉者胀,之温热者疮。”
	\item 东方之域,天地之所始生也:张介宾注:“天地之气,自东而升,为阳生之始。故发生之气,始于东方,而在时则为春。”域,疆界、境地或自然条件形成的特征性地理区域。
	\item 热中:又称中热。一般指脏腑有热。
	\item 盐者胜血:杨上善《黄帝内经太素·知方地》注:“盐,水也;血者,火也。水以克火,故胜血而人色黑也。”
	\item 踈理:踈,“疏”的俗字(《广韵》)。疏理,即腠理疏松。
	\item 砭石:自然形成或人为加工制成的尖石或石片,用以治病。
\end{jiaozhu}

\biaoti{【原文】}

\begin{yuanwen}
西方者,金玉之域,沙石之處,天地之所收引\sb{1}也。其民陵居而多風\sb{2},水土剛強,其民不衣而褐薦\sb{3},其民華食而脂肥\sb{4}。故邪不能傷其形體,其病生於内,其治宜毒藥\sb{5}。故毒藥者,亦從西方來。
\end{yuanwen}

\biaoti{【校注】}

\begin{jiaozhu}
	\item 天地之所收引:张介宾注:“天地之气,自西而降,故为天地之收引,而在时则应秋。”
	\item 陵居而多风:张介宾注:“陵居,高处也,故多风。”
	\item 其民不衣而褐薦:褐(hè音贺),兽毛或粗麻制成的短衣;薦(jiàn音见),兽所食之草也,《庄子齐物论》:“麋鹿食薦。”其民不衣而褐薦,言西方之人不讲究穿衣而披兽皮或着麻草编织的短衣。
	\item 华食而脂肥:王冰注:“华,谓鲜美,酥酪骨肉之类也。以食鲜美,故人体脂肥。”脂肥,言身体健壮。《灵枢·卫气失常》曰:“肉坚,皮满者肥,……脂者,其肉坚。”
	\item 毒药:指作用峻烈的药物。
\end{jiaozhu}

\biaoti{【原文】}

\begin{yuanwen}
北方者,天地所閉藏\sb{1}之域也。其地高,陵居,風寒冰例。其民樂野處而乳食,藏寒生滿病\sb{2},其治宜灸焫\sb{3}。故灸焫者,亦從北方來。
\end{yuanwen}

\biaoti{【校注】}

\begin{jiaozhu}
	\item 北方者,天地所闭藏:张介宾注:“天之阴在北,故其气闭藏,而在时则应冬。”
	\item 脏寒生满病:张介宾注:“地气寒,乳性亦寒,故令人脏寒。脏寒多滞,故生胀满。”
	\item 灸焫:焫,烧也。王冰注:“火艾烧灼,谓之灸焫。”艾灸、火针、焠针,皆属此法。
\end{jiaozhu}

\biaoti{【原文】}

\begin{yuanwen}
南方者,天地所長養\sb{1},陽之所盛處也,其地下\sb{2},水土弱,霧露之所聚也。其民嗜酸而食胕\sb{3}。故其民皆緻理\sb{4}而赤色。其病孿痺\sb{5},其治宜微滅\sb{6}。故九鍼者\sb{7},亦從南方來。
\end{yuanwen}

\biaoti{【校注】}

\begin{jiaozhu}
	\item 南方者,天地所长养:张介宾注:“天之阳在南,故万物长养,而在时则应夏。”
	\item 其地下:言南方地势低。
	\item 胕:同腐,指经发酵的食物,如豉、鲊、曲、酱类食物。
	\item 致理:腠理致密,皮肤细腻。
	\item 挛痹:拘挛疼痛。
	\item 微针:指针体细小,加工精细的针具。秦汉时已有金属针具,较古之石针、骨针精细,故称之微针。
	\item 九针:《内经》时期将针具规定为九种,即镵针、员针、𫔂针、锋针、铍针、员利针、毫针、长针、大针(《灵枢·九针十二原》、《灵枢·九针论》)。
\end{jiaozhu}

\biaoti{【原文】}

\begin{yuanwen}
中央者,其地平以濕,天地所以生萬物也衆。其民食雜而不勞\sb{1},故其病多痿厥寒熱\sb{2},其治宜導引按蹻\sb{3}。故導引按蹻者,亦從中央出也。

故聖人雜合以治,各得其所宜\sb{4},故治所以異而病皆愈者,得病之情,知治之大體也\sb{5}。
\end{yuanwen}

\biaoti{【校注】}

\begin{jiaozhu}
	\item 其民食杂而不劳:王冰注:“四方辐辏而万物交归,故人食杂而不劳也。”
	\item 其病多痿厥寒热:吴昆注:“湿伤筋,则病瘘弱;湿伤足,则病下厥,谓逆冷也。中央当南北之冲,水火之所交袭,故病寒热。”
	\item 导引按蹻:王冰注:“导引,谓摇筋骨,动肢节;按,谓抑按皮肉;蹻,谓捷举手足。”
	\item 圣人杂合以治,各得其所宜:张志聪注:“天有四时之气,地有五方之宜,民有居处衣食之殊,治有针灸药饵之异,故圣人或随天之气,或合地之宜,或随人之病,或用针灸、毒药,或以导引按摩,杂合以治,各得其宜。”
	\item 得病之情,知治之大体:大体,指重要的义理;有关大局的道理,用于医学可引申为治则治法。
\end{jiaozhu}

\biaoti{【理论阐释】}

1.地理、地势与发病

本篇以“地势使然”简洁地回答了“一病而治各不同”的道理。进一步分析五方的地势不同而有地理、气候、物产差异性,这些差异性决定五方之人的居住条件与环境,饮食结构及饮食习惯各自不同。天人两方面因素直接影响人体的形质强弱和发生疾病的种类与性质,因而五方之人得病各异,治法各有所宜。

地势有高低,地域有南北,气候有寒温,病发有不同。如本篇指出,北方“地高陵居,风寒冰冽”,“病生于内”。而南方则“其地下,水土弱,雾露之所聚”,“其病挛痹”。皆在说明地势与气温特点和发病具有相关性。《素问·五常政大论》“地有高下,气有温凉,高者气寒,下者气热。故适寒凉者胀,之温热者疮。”

饮食结构与习惯不同而病发有异。如东方之域,“其民食鱼而嗜咸”,其病以“痈疡”居多;而北方之域,由于“乐野处而乳食”病以“脏寒生满病”为主。地势不同,所在地域人们的饮食结构以及饮食习惯表现为单一性或偏嗜现象。《内经》中有关饮食五味所伤论述较多,如《素问·生气通天论》“高梁之变,足生大丁。”《素问·奇病论》“肥者令人内热,甘者令人中满。”《素问·五脏生成论》、《素问·生气通天论》均记载偏嗜五味太过引起脏腑气血发病等。

本篇中砭石、毒药、灸焫、微针、导引、按蹻,是针对五方地域性常见病、多发病而在实践中创建的治疗工具与方法,对不同疾病各有其治疗的优势。

2.圣人杂合以治

“圣人杂合以治”,圣人之所以能杂合以治,因为圣人能“得病之情,知治之大体”。也就是说,一个高明的医生能把握疾病的发生,及病情的变化,掌握治疗的法则,而在选择具体治法上则灵活变通。“东方之域……其民皆为痈疡,其治宜泛石。”砭石固然是治疗痈疡的有效方法与手段,但是砭石也不完全适用于痈疡的各阶段,张志聪《黄帝内经素问集注》注:“如痈病之热毒盛于外者,治宜针砭;毒未尽出者,治以毒药;阴毒之内陷者,又宜于艾焫也。”

本篇中虽说东方之域其病皆痈疡,但文中指出“鱼者使人热中”,治热中之病,除贬石之外,还可用药物疗法。同样北方之域“脏寒生满病”,亦不限于灸焫,亦可针刺或药物治疗。

总之“圣人杂合以治”,是根据病情,确定治则而选用适宜的治疗手段,并可结合运用多种疗法以提高疗效。

\biaoti{【临证指要】}

\xiaobt{杂合以治,各得其所宜}

本篇提出五种治疗方法,以适应五方常见病与多发病的治疗。文中最后提出“圣人杂合以治,各得其所宜”,倡导各治法结合运用,并使所运用的治法对治疗的对象发挥其应有的作用。

“杂合以治”的运用在《内经》中已经得到证实,如:药物与食疗结合治疗疾病,《素问·五常政大论》“大毒治病,十去其六;常毒治病,十去其七;小毒治病,十去其八;无毒治病,十去其九。谷肉果菜,食养尽之,无使过之,伤其正也。不尽,行复如法。”又有针刺与汤液或热饮结合治疗,如《素问·评热病论》风厥的治疗为“表里刺之,饮之服汤。”针砭与药物、灸法结合运用,如《灵枢·禁服》“代则取血络且饮药”,“紧则灸刺且饮药”,“不盛不虚,以经取之,名曰经刺……所谓经治者,饮药,亦曰灸刺。”杂合以治,并非治疗手段在形式上的结合,而是根据病情的需要,根据各种疗法的治疗作用,合理的配合,而达到治疗疾病的目的。《素问·汤液醪醴论》治疗阳虚水肿,即将按摩、温覆、药物、针刺、食疗综合运用,共奏扶正而驱邪的治疗作用。

\zuozhe{(王贵臣)}

\section{素問·湯液醪醴論(節選)}%第三節

\biaoti{【原文】}

\begin{yuanwen}
帝曰:上古聖人,作湯液醪醴\sb{1},爲而不用何也?岐伯曰:自古聖人之作湯液醪醴者,以爲備耳。夫上古作湯液,故爲而弗服也。中古之世,道德稍衰\sb{2},邪氣時至,服之萬全。帝曰:今之世不必已何也?岐伯日:當今之世,必齊毒藥攻其中,鑱石鍼艾治其外\sb{3}也。
\end{yuanwen}

\biaoti{【校注】}

\begin{jiaozhu}
	\item 汤液醪醴:汤液,用五谷煎煮的液体;醪醴,是指用谷类加工制作的酒类。醪,汁滓酒也(《说文》),即酒酿,又称浊酒;醴,“酿之一宿而成醴,有酒味而已也。”(《释名·释饮食》)。
	\item 道德稍衰:道,原指人行的道路,借用为事物运动变化所必须遵循的普遍规律。“德”和“得”意义相近。对于“道”的认识修养有得于己,亦称为“德”。道德稍衰,即人的修养,品德渐渐变差。
	\item 必齐毒药攻其中,镵石针艾治其外:齐,同“并”之义,《楚辞·九歌·云中君》:“与日月齐光。”必齐毒药攻其中,镵石、针艾治其外,意思是用汤液醪醴还必须配合毒药攻除内里的疾病,配合针灸治疗外部疾病。镵石,尖锐或有刃的石制工具。用于治疗疾病称之为镵石,又称针石。针艾,即针刺及艾灸。
\end{jiaozhu}

\biaoti{【理论阐释】}

1.上古圣人作汤液醪醴,为而不用

上古圣人作汤液醪醴,“为而不用”,“为而弗服”,仅是“以为备耳”。王冰注:“言圣人愍念生灵先防萌渐,陈其法制,以备不虞耳。”又曰:“圣人不治已病,治未病,故但为备用而不服也。”从中可以领悟,古人注重防病于未然,之所以“为而弗服”是因为上古之人善于适应自然环境,调控精神、嗜欲,身心健康而无病。亦如《素问·移精变气论》所言:“往古人……内无眷慕之累,外无伸宦之形。此恬憺之世,邪不能深入也。”

2.当今之世,汤液醪醴的应用

当今之世(指《内经》形成的历史时期)汤液醪醴不能单独发挥其治疗作用。根据疾病所在,必须配合药物,或针刺、艾灸治疗。究其原因,如《素问·移精变气论》所言:“当今之世不然,忧患缘其内,苦形伤其外,又失四时之从,逆寒暑之宜,贼风数至,虚邪朝夕,内至五脏骨髓,外伤空窍肌肤,所以小病必甚,大病必死。”病至如此程度,则非汤液醪醴之所能及。而与药物及针刺、艾灸配合治疗,则既能驱邪,疏通血脉,又能扶正,达到治疗目的。故本文提出“当今之世,必齐毒药攻其中,镵石针艾治其外。”

本段经文虽然用“上古”与“当今”疾病有轻重不同,意在说明养生的重要性,但却也反映出汤液醪醴与药物、针灸结合运用于治疗之中,是医学由单一治疗方法向综合治疗的一大发展。

\biaoti{【临证指要】}

\xiaobt{关于醪醴的应用}

醪醴(即酒类)在医学里应用甚久,尤其是与药物配合使用而治病《内经》中已经得以证实,如《素问·腹中论》治疗鼓胀,应用鸡矢醴。鸡矢,即鸡矢白;醴,即酒。李时珍《本草纲目》引何大英云:“用腊月鸡矢白半斤,袋盛,以酒醅一斗,渍七日,温服三杯,日三。”张仲景《伤寒杂病论》中瓜蒌薤白白酒汤是最典型的醪醴与药物配合使用的代表方。后世各类药酒、酊剂品种繁多,使用范围非常广范。

\biaoti{【原文】}

\begin{yuanwen}
帝曰:其有不從毫毛而生,五藏陽以竭\sb{1}也,津液充郭,其魄獨居\sb{2}。孤精於內\sb{3},氣耗於外,形不可與衣相保\sb{4}。此四極急而動中\sb{5},是氣拒於內而形施於外\sb{6}。治之柰何?岐伯曰:平治于權衡\sb{7},去宛陳莝\sb{8},微動四極,溫衣,缪刺其處,以復其形\sb{9}。開鬼門、潔淨府\sb{10},精以時服\sb{11},五陽已布,踈滌五藏\sb{12}。故精自生,形自盛,骨肉相保,巨氣乃平\sb{13}。帝曰:善。
\end{yuanwen}

\biaoti{【校注】}

\begin{jiaozhu}
	\item 五藏阳以竭:水肿发生的原因。以,同“已”。竭,尽也;又亡也。王冰注:“不从毫毛,言生于内也。阴气内盛,阳气竭绝,不得入于腹中,故言五脏阳已竭也。”
	\item 津液充郭,其魄独居:津液,水谷化生的液态精微物质。郭,同“廓”,此指形体。魄,同“粕”,此言津液之糟粕,即水气。阳气虚不能化气行水,水液停留,充斥周身,故言津液充郭,其魄独居。
	\item 孤精于内:阳气亏粍,水精(津液)不行而为水气,故孤精于内。
	\item 形不可与衣相保:王冰注:“水满皮肤,身体否肿,故云形不可与衣相保。”
	\item 此四极急而动中:四极,四肢。急,窘也。中,胸腹中。水邪四西溢,外则四肢肿急,内则动于胸腹而致气急咳啭。
	\item 气拒于内而形施于外:气,水气。施,易也。水气内停,形体肿急而变易其状态。
	\item 平治于权衡:权衡,秤锤与秤杆,此有衡量揆度之义。衡连揆度病情,以平调阴阳的偏盛偏衰。吴昆:“平治之法,当如权衡,阴阳各得其平,勿令有轻重低昂也。
	\item 去宛陈莝:宛,音义同“郁'。莝(cuò音错),斩也。张介宾:“宛,积也。陈,久也。莝斩草也。谓去其水气之陈积,欲如斩草而渐除之也。”又新校正云:“《太素》莝作‘茎’”。
	\item 缪刺其处,以复其形:缪刺,病在左而刺右,病在右而刺左的刺络法。又据《素问·缪刺论》曰:“有痛而经不病者缪刺之,因视其皮部有血络者尽取之,此缪刺之数也。”据此,缪刺又是针对皮部血脉(血络)采取解结或放血疗法。缪刺其处,以复其形,即通过缪刺(解结、故血)使血脉恢复正常状态。
	\item 开鬼门,洁净府:鬼门,汗孔;净府,膀胱。开鬼门,洁净府,即发汗、利小便。以此法消散水气,祛除水肿。
	\item 精以时服:精,精良的食物,即富含营养,补益精气的食物,如鱼豆类等。精以食服,指按时令服用精良的食物,属于饮食疗法的范畴。
	\item 五阳已布,踈滌五脏:《黄帝内经太素·知汤药》作“服五汤,有五疏,修五脏”。五汤,即五谷汤液。杨上善注:“五汤,五味汤也,有,通“又”,《书尧典》“朞三百有六旬有六日。”疏,通“蔬”,《论语述而》“饭疏食饮水,曲肱而枕之。”修*,整饬貌,《苟子修身》“见善修然。”服五汤能生精,精能化为气,以补五脏阳气之虚衰。
	\item 巨气乃平:王冰注:“大经脉气乃平復尔。”
\end{jiaozhu}

\biaoti{【理论阐释】}

1.“五脏阳以竭”与水肿的发生

水肿是水液滞留于体内,泛溢于皮肤分肉之间,或脏腑之内而产生的以肿胀为临床特征的疾病。引起水肿的原因,不外外因与内因两类,外因如风寒之邪客于肌表皮肤分肉之间,阻滞或凝聚津液而引起水肿。《素问·水热穴论》记载“勇而劳甚则肾汗出,肾汗出逢于风,内不得入于脏腑,外不得越于皮肤,客于玄府,行于皮里,传为胕肿,本之於肾,名曰风水。”引起水肿的内因,即本文所言“五脏阳以竭”,阳气虚不能化气行水,使水液内停、泛溢而为肿。

人体水液的代谢,与五脏六腑及经脉之气的功能密切相关,《素问·经脉别论》记载“饮入于胃,游溢精气,上输于脾,脾气散精,上归于肺,通调水道,下输膀胱。水精四布,五经并行。合于四时五脏阴阳,揆度以为常也。”又《素问·水热穴论》“肾者,至阴也,至阴者,盛水也。肺者太阴也,肾者冬脉也,故其本在肾,其末在肺,皆积水也”。又“肾者胃之关也,关门不利,故聚水以从其类也,上下溢于皮肤,故为浮肿。”

2.去宛陈莝

去宛陈莝,在古汉语属于“错综”语序,是常见的修辞手法,与海滨傍水形式相同。《黄帝内经研究大成》将“平治于权衡,去宛陈莝”并收于治疗总则文中。

去宛陈莝,从杨上善注为:宛陈,指血脉中的郁滞,在治疗上提出“刺去之”。又《灵枢·九针十二原》有“宛陈则除之”的针刺治疗原则。《灵枢·小针解》解曰:“宛陈则除之者,去血脉也”;《小针解》所言“血脉'”是指颜色、质地、形态爰发生异常改变的血络。《灵枢·血络论》“血脉者,盛坚以赤,上下无常处,小者如针,大者如筋,则而泻之万全也。”至于去除宛陈的方法,在内经中主要是针刺放血或解结及缪刺。《灵枢·刺节真邪》曰:“一经上实下虚而不通者,此必有横络盛加于大经,令之不通,视而泻之,此所谓解结也。”又曰:“脉淖泽者,刺而平之;坚紧者,破而散之,气下乃止,此所谓解结者也。”《素问·调经论》记载:“血有余,则泻其盛经出其血。”盛经,《素问·水热穴论》解释曰:“所谓盛经者,阳脉也。”阳脉即指皮部表浅的血脉。《素问·缪刺论》“视其皮部有血络者尽取之,此缪刺之数也。”在《内经》中,针刺放血、解结、缪刺,其治疗作用是开通郁阻的血脉,为针刺调经之法。《灵枢·水胀》篇关于肤胀、鼓胀的治疗亦有“先泻其胀(当为“腹”)之血络,后调其经,(亦)刺去其血络也”的记载,提示了刺血络与调经实施的先后及治疗中的关系。

3.阳虚水肿的治疗

本篇中指出阳虚水肿的治则与治法。其治则是“平治于权衡,去宛陈莝”,其治法是“微动四极;温衣;缪刺其处;开鬼门,洁净府;精以时服。”文中治法及作用包括以下几方面:

①微动四极:即轻微活动四肢。其作用是疏通气血,振奋阳气。既有利于经脉中气血津液的流通,又可促进阳气的化气行水之功。

②温衣:即加衣温覆。其作用为保护阳气,消散寒湿之气。

③缪刺其处:即用针刺实施解结法或放血法去除血络中的郁阻,恢复血脉的正常状态,使经络疏通。既有利经脉中气血津液的转输,又为其它治疗奠定基础。

④开鬼门,洁净府:即发汗、利小便。是本篇中消除水肿主要治疗手段。

⑤精以时服:即餐服精美食物。以之益气养精,是本病扶正的重要措施。

通过以上诸法综合治疗,达到扶正驱邪,消除水肿的目的。治法中的“缪刺其处”,“开鬼门、洁净府”,“精以时服”正是“当今之世,必齐毒药攻其中,镵石针艾治其外”的例证。

\biaoti{【临证指要】}

1.“去宛陈莝”的临床运用

去宛陈莝,可理解为去蓄积之水,也可理解为是“菀陈则除之”治则的另一种语言表述,是针对皮部血络的病理改变提出的治疗原则。去宛陈莝所釆用的针刺解结或放血疗法,适用于多种疾病的治疗或辅助治疗。除解除络脉的自身郁阻,还可以通过解除络脉的郁阻,减轻、消除郁阻络脉对经脉的压迫,而适用于多种疾病的治疗。如:《素问·刺腰痛篇》所举诸腰痛的治疗均有针刺放血,并提出“血变而止”针。其中记载“解脉令人腰痛,痛引肩,目䀮䀮然,时遗溲,刺解脉,在膝筋肉分间郄外廉之横脉出血,血变而止。解脉令人腰痛如引带,常如折腰状,善恐,刺解脉,在郄中结络如黍米,刺之血射以黑,见赤血而已。”《灵枢·癲狂》记载“暴仆,四肢之脉皆胀而纵,脉满,尽刺之出血。”“狂而新发……先取曲泉左右动脉及盛者见血,有顷已。”鼓胀、肤胀的治疗中解结或放血作为调经治疗水肿的基本疗法。疟疾的治疗亦用放血或解结法治疗,《素问·刺疟篇》曰:“疟发身方热,刺跗上动脉,开其空,出其血,立寒。……诸疟而脉不见,刺十指间出血,血去必已。先视身之赤如小豆者尽取之。……十二疟者……先其发时如食顷而刺之,一刺则衰,二刺则知,三刺则已。不已刺舌下两脉出血,不已刺郄中盛经出血……舌下两脉者,廉泉也。”

宛陈作为治疗的对象,是人体中的病理产物。它不只限于瘀血,其它诸如水气、痰饮、燥屎、宿食及尿中砂石等均可视为宛陈之物,所以王冰注为“去积久之水物,犹如草茎之不可久留于身中也。”丹波元坚引初和甫曰:“去宛陈莝,谓涤肠胃中腐败也。”治疗“宛陈”也不限于计刺解结或放血。现代中医临床应用最多的是药物疗法。如活血化瘀,软坚散结,化痰涤痰,攻逐水饮,下气通便等均应视为对去宛陈莝(菀陈则除之)治法的发挥。

2.开鬼门、洁净府

“开鬼门、洁净府”是本篇消除水肿的两个基本方法与途径。

《内经》指出人摄入水饮,在脏腑之气作用下形成津液,通过经脉转输到身体各部。而津液的主要代谢外排方式是汗与尿,如《灵枢·决气》曰:“腠理发泄,汗出溱溱是谓津。”《素问·经脉别论》曰:“饮入于胃……通调水道,下输膀胱。”

受某些条件或病因的影响,津液代谢失常形成水气,甚至出现水液停留而发生水肿,《灵枢·五隆津液别》记载“天寒衣薄,则为溺与气,天热衣厚,则为汗……邪气内逆,则气为之闭塞而不行,不行则为水胀。”因此,发汗、利小便是促进津液代谢和消除水肿的两种有效方法与途径。在临床应用时可根据病情,或发汗或利小便,或二者并用。张仲景《金匮要略·水气病脉证并治》指出:“诸有水者,腰以下肿,当利小便;

腰以上肿,当发汗乃愈。”《医宗金鉴》:“诸有水者,谓诸水病也。治水之病,当知表里上下分消之法。腰以上肿者,水在外,当发其汗乃愈,越婢、青龙汤证也。腰以下肿者,水在下,当利小便乃愈,五苓、猪苓等汤证也。”观仲景水气病治疗汤方,除个别发汗解表治皮水外,多属于发汗利小便结合运用方剂。且一般均有补益正气作用的药物,或兼有温阳行气的药物,临证应予重视。

\section{素問·藏氣法時論(節選)}%第四節

\biaoti{【原文】}

\begin{yuanwen}
肝主春\sb{1},足厥陰少陽主治\sb{2},其日甲乙\sb{3}。肝苦急,急食甘以緩之\sb{4}。心主夏,手少陰太陽主治,其日丙丁。心苦緩,急食酸以收之\sb{5}。脾主長夏,足太陰陽明主治,其日戊己。脾苦濕,急食苦以燥之\sb{6}。肺主秋,手太陰陽明主治,其日庚辛。肺苦氣上逆,急食苦以泄之\sb{7}。腎主冬,足少陰太陽主治,其日壬癸。腎苦燥,急食辛以潤之\sb{8},開腠理,致津液,通氣也\sb{9}。
\end{yuanwen}

\biaoti{【校注】}

\begin{jiaozhu}
	\item 肝主春:五脏应四时(五时),据“人禀天地之气生,四时之法成”的理论,则知五时主五脏。“肝主春”是古汉语的被动句式,即春主肝。以下仿此。
	\item 足厥阴少阳主治:足厥阴肝经、足少阳胆经主治肝病。王冰注:“厥阴肝脉,少阳胆脉。肝与胆合,故治同。”以下仿此。
	\item 其日甲乙:甲乙丙丁戊己庚辛壬癸,为十天干,古代用以纪日、纪月、纪年。其日甲乙,即甲乙日主肝。以下仿此。
	\item 肝苦急,急食甘以缓之:苦,恶也,又厌也。以下四脏所苦均仿此。急,缩也,即收缩、拘紧、不舒展之义;又快也。本段及下段“急食”中的“急”均取此义。缓,柔也,又舒也,即柔软、舒展之义。张介宾注:“肝为将军之官,其志怒,其气急。急则自伤,反为所苦,故宜食甘以缓之,则急者可平,柔能制刚也。”
	\item 心苦缓,急食酸以收之:缓,放纵也。此言病则涣散不收。《灵枢·本神》云:“喜乐者,神惮散而不藏。”张介宾注:“心藏神,其志喜,喜则气缓而心虚神散,故宜食酸以收之。”
	\item 脾苦湿,急食苦以燥之:张介宾注:“脾以运化水谷,制水为事,湿盛则反伤脾土,故宜食苦温以燥之。”
	\item 肺苦气上逆,急食苦以泄之:张介宾注:“肺主气,行治节之令,气病则上逆于肺,故宜急食苦以泄之。”
	\item 肾苦燥,急食辛以润之:张介宾注:“肾为本脏,藏精者也。阴病者苦燥,故宜食辛以润之。”
	\item 开腠理,致津液,通气也:此言辛味的作用。张介宾注:“盖辛从金化,水之母也。其能开腠理致津液者,以辛能通气也。水中有真气,惟辛能达之,气至水亦至,故可以润肾之燥。”
\end{jiaozhu}

\biaoti{【理论阐释】}

1.五脏与四时的关系

《内经》提出“人以天地之气生,四时之法成”,“人生于地,悬命于天”的理论。强调“天人相应”。五脏应四时(五时),即是以天人相应、阴阳五行为理论基础形成的具有代表人与自然关系的具体医学理论。《素问·六节脏象论》指出“心者……通于夏气;肺者……通于秋气;肾者……通于冬气;肝者……通于春气;脾者·……通于土气(即至阴长夏之气)。”就发病关系而言,《素问·金匮真言论》记栽“东风生于春,病在肝;……南风生于夏,病在心;……中央为土,病在脾。”《素问·咳论》指出五脏各以其时受病,非其时各传以与之。”不但阐发了四时之气与五脏的发病关系的一般规律,而且也说明在不应时的情况下,邪气可以通过其它关系与途径侵犯它脏。就养生而言,《素问·四气调神大论》依据此理论,倡导适应四时生长收藏的规律养生。总结出“四时阴阳者,万物之根本,所以圣人春夏养阳,秋冬养阴,以从其根”的养生法则。

2.五脏所苦的治疗

对五脏所苦文中明确指出从两个方面进行治疗:其一,是表里相合两经主治。针刺是主要的治疗方法。由于“十二经脉者,内属于腑脏,外络于肢节。”(《灵枢·海论》)。而能“决死生,处百病,调虚实”(《灵枢·经脉》),所以五脏病可取各自经脉治疗之。至于五脏病取其相合之经,则是由于脏腑功能相合,经脉表里络属关系决定的。取相合经脉,有从阳引阴之意,以求阴阳的平衡。其二,五脏病在本文中还提出药食五味的治疗内容,药、食均有五味,五味各有其作用,即本篇所言“辛散、酸收、甘缓、苦坚、咸软”。文中五脏所苦的药食治疗,是釆取逆五脏之所苦而从五脏之所欲选择相应作用的药味。通过治疗,既使五脏所苦得以解除,同时也有防止五脏所苦之病发生的作用。

\biaoti{【临证指要】}

\xiaobt{脏腑病针刺与药食治疗}

针刺、药食是治疗疾病的主要方法,临床运用必须以疾病的证候为依据,而选择相应的经脉和药物。

五脏病的治疗,虽然本段只言取经,而未言取穴,历代注家,也没有提供治疗的具体的取穴。但可参照《咳论》、《痹论》、《痿论》取五脏经脉的荥穴、俞穴,取六府经脉的合穴。其理论依据,则是《灵枢·邪气脏府病形》提供的“荣俞治外经,合治内府”,以及《灵枢·九针十二原》曰:“五脏有疾也,应出十二原,……凡此十二原者,主治五脏六府之有疾也。”从《灵枢·九针十二原》、《灵枢·本输》记载可知,五脏经(即阴经)的原穴与俞穴是同一穴位,所以阴经的俞穴既可治外经病,又可治内脏病。其取阳经合穴,则尚有防止脏病传腑的治疗作用。

本段提出用药食治疗,是根据“五味各走其所喜”的五味入五脏原理发挥药、食四气、五味的作用而调治疾病。《灵枢·五味》曰:“五味各走其所喜。谷味酸,先走肝;谷味苦,先走心;谷味甘,先走脾;谷味辛,先走肺;谷味咸,先走肾。”在临床上除应注重药食之味外,还要注意把握使用药、食五味的量与强度,无太过,无不及。其不及则药食气味不达病所而病不愈,五味太过则亦害而无益,正如《素问·至真要大论》曰:“五味入胃,各归所喜……久而增气,物化之常也。气增而久,夭之由也。”因而临证调养疾病,药食用量,必须适度,中病即止。

\biaoti{【原文】}

\begin{yuanwen}
肝欲散,急食辛以散之,用辛補之,酸寫之\sb{1}。……心欲耎\sb{2},急食鹹以耎之,用鹹補之,甘寫之……。脾欲緩,急食甘以緩之,用苦寫之,甘補之……。肺欲收,急食酸以收之,用酸補之,辛寫之。……腎欲堅,急食苦以堅之,用苦補之,鹹寫之……。肝色青,宜食甘\sb{3},粳米\sb{4}牛肉棗葵\sb{5}皆甘;心色赤,宜食酸\sb{6},小豆犬肉李韭皆酸;肺色白,宜食苦\sb{7},麥羊肉杏薤\sb{8}皆苦;脾色黃,宜食鹹\sb{9},大豆豕肉栗藿\sb{10}皆鹹;腎色黑,宜食辛\sb{11},黃黍\sb{12}雞肉桃葱皆辛。辛散酸收甘緩苦堅鹹耎。
\end{yuanwen}


\biaoti{【校注】}

\begin{jiaozhu}
	\item 用辛补之,酸泻之:张介宾注:“顺其性者为补,逆其性者为泻。”余脏准此。
	\item 耎:同软。
	\item 肝色青,宜食甘:吴昆注:“肝苦急,急食甘以缓之是也。”
	\item 粳米:粳,《玉篇》:“不黏稻,亦作秔。”《灵枢·五味》记载“五谷,秔米甘,麻酸,大豆咸,麦苦,黄黍辛”。故知粳、秔同。
	\item 葵:植物名,即“冬葵”,为我国古代主要蔬菜之一。元·王祯《农书》称葵为“百菜之王”。
	\item 心色赤,宜食酸:吴昆注:“心苦缓,急食酸以收之是也。”
	\item 肺色白,宜食苦:吴昆注:“肺苦气上逆,急食苦以泻之是也。”
	\item 薤:俗称小根蒜,可食。其地下鳞茎干燥加工,即中药“薤白”。
	\item 脾色黄,宜食咸:吴昆注:“脾苦湿,咸能泄湿,故食之。瓜果肉菜得盐而湿出,理可知矣。”
	\item 藿:豆叶,《广雅·释草》“豆角谓之荚,其叶谓之藿。”
	\item 肾色黑,宜食辛:吴昆注:“肾苦燥,急食辛以润之是也。”
	\item 黄黍:张介宾注:“黄黍,即小米,北方谓之黄米。”
\end{jiaozhu}

\biaoti{【理论阐释】}

\xiaobt{五脏所欲与五味补泻}

本段论五味辛散、咸软、甘缓、酸收、苦坚的不同作用,而五脏各有其所欲。经文中应用作用相反的药味,一补一泻相配合调治疾病。其补其泻以适应五脏之性与否为分辨,正如张介宾《类经·疾病类》云:“顺其性者为补,逆其性者为泻。”

调治五脏所欲之药的五味搭配体现了组方的君、臣配伍关系。以文中“肝欲散,急食辛以散之,用辛补之,酸泻之”为例。“急食辛以散之”即用辛味疏散肝气,是治病的主要部分(君)。“用辛补之”则是从其肝之所欲,增加散气之功,可视为補助药(臣)。酸味主收敛,与“肝欲散”忤逆,又有碍辛散之功,故称“酸泻之”。就病与治关系而言,用酸收从其病;就用药配伍而言,用酸收以制辛散的太过。因此可以认为,调治中用酸味,具有反佐的作用,而为佐药。

五味顺逆补泻之义,又见于《素问·至真要大论》,如曰:“木位之主,其泻以酸,其补以辛。火位之主,其泻以甘,其补以咸。土位之主,其泻以苦,其补以甘。金位之主,其泻以辛,其补以酸。水位之主,其泻以咸,其补以苦。”

\biaoti{【原文】}

\begin{yuanwen}
毒藥攻邪\sb{1},五穀\sb{2}爲養,五果\sb{3}爲助,五畜\sb{4}爲益,五菜\sb{5}爲充。氣味合而服之,以補精益氣\sb{6}。此五者,有辛酸甘苦鹹,各有所利,或散或收,或缓或急,或堅或耎。四時五藏,病隨五味所宜也\sb{7}。
\end{yuanwen}

\biaoti{【校注】}

\begin{jiaozhu}
	\item 毒药攻邪:张介宾注:“药以治病,因毒而能,……是凡可辟邪去正者,均可畚称为毒药,故曰毒药攻邪也。”
	\item 五谷:粳米、麻、大豆、麦、黄黍。
	\item 五果:枣、李、栗、杏、桃。
	\item 五畜:牛、犬、猪、羊、鸡。
	\item 五菜:即五蔬。葵、韭、藿、薤、葱。
	\item 气味合而服之,以补精益气:王冰注:“气为阳化,味曰阴施。气味合和则补益精气矣。……形不足者,温之以气;精不足者,补之以味。由是则补精益气,其义可知。”
	\item 四时五脏,病随五味所宜也:王冰注:“用五味而调五脏。配肝以甘,心以酸,脾以咸,肺以苦,肾以辛者,各随其宜,欲缓欲收欲软欲泄欲散欲坚而为用,非以相生相养而为义也。”
\end{jiaozhu}

\biaoti{【理论阐释】}

\xiaobt{毒药攻邪,食物养正}

本段论药物与食疗配合应用而调治疾病,强调了五谷、五果、五畜、五菜的补益作用,以及五味的治疗作用。

就疾病而言,疾病的发生与否取决于正气与邪气两个要素,而此正气为决定因素。《素问(遗篇)·刺法论》曰:“正气存内,邪不可干。”《素问·评热病论》言:“邪之所凑,其气必虚。”说明一旦发病,人体正气都有不同程度的损伤。就药物治疗而言,毒药虽能攻邪,同时也会克伐人体正气,就人的生命活动而言,饮食精微物质是生命的根本,如饮食摄入不足或不进食则使正气衰减,其至死亡。《灵枢·五味》指出“天地之精气,其大数常出三入一,故谷不入,半日则气衰,一日则气少矣。”又《灵枢·平人绝谷》记载“平人不食饮七日而死者,水谷精气津液皆尽故也。”

无论养生保健还是治疗疾病,均应注意药食五味“合”和而服之。

本篇中气味相合,有两种形式,其作用不一。

其一,相同气味的相合。如“肝色青,宜食甘,粳米牛肉枣葵皆甘。即谷肉果菜中同一气味配合应用。可以增进这一气味的作用。

其二,不同气味相合,有主有次。如,“肝欲散,急食辛以散之,用辛补之,酸泻之”,其中“食辛以散之”,“用辛补之”是同一气味的增补与加强,发挥辛味发散的主要调治作用;“酸泻之”是用酸味起收敛的作用,防止辛散太过,反伤肝气。属于不同气味的搭配使用。

\biaoti{【临证指要】}

\xiaobt{毒药攻邪,五谷为养}

毒药与五谷(包括五果、五畜、五菜)各具寒热温凉之性与酸、苦、甘、辛、咸五味。寒热温凉分别阴阳,互相制胜;五味各有所喜、所归,而分别具有“辛散酸收甘缓苦坚咸软”的不同作用。毒药与五谷、五果、五畜、五菜在其性、味、所归、所喜等方面具有共性,其差别在于有毒无毒,毒性大小和对疾病治疗作用的缓急、强弱及益害多少。临证中应用不同毒性药物治病,目的是祛除病邪,但不可忽略毒药伤人的危害性。为了缓和其峻猛,消减毒性,常采用以下几种方式:

1.药、食共组一方。如十枣汤中用大枣;瓜蒂散中用赤小豆;其它诸如桂枝、建中、白虎、白通加猪胆汁汤、当归生姜羊肉汤等均属毒药或药物与谷、肉、果、菜共用的组方。谷、肉、果、菜在方中除了具有补益正气之外,尚有解毒、缓和药物峻烈之性的作用,如大枣缓和大戟、甘遂、芫花之毒,赤小豆与瓜蒂配伍,有酸苦涌泻之意,而又有治疗作用;猪胆汁的咸寒佐制葱姜附大热之性,又具反佐意义。

2.药、食分方并用。谷菜果畜类不与药物共方,在治疗中与药物配合使用,发挥其扶正、和助药力以祛邪的作用。《素问·汤液醪醴论》阳虚水肿用药物“开鬼门、洁净府”,而又提出“精以时服”(《太素》有“服五汤,有五疏”),即用由五谷制成具有辛温之性,既有振奋阳气、疏通经络,又有助于药物发散水气作用的醪醴,又用“五蔬(蔬)”养其正气。桂枝汤方后“啜热稀粥”亦属药、食分方并用,其热稀粥,既能扶正又能助药以解表邪,而有一举两得之功。

3.药、食分用。《素问·五常政大论》“大毒治病,十去其六;常毒治病,十去其七;小毒治病,十去其八;无毒治病,十去其九;谷肉果菜,食养尽之。无使过之,伤其正也。不尽,行復如法。”可知,先药治,后食疗。根据药物毒性的大小,决定药物治疗的时间,祛病的程度。其余下之病则借助谷肉果菜的辛散、酸收、甘缓、苦坚、咸软的功效去除之。谷肉果菜既发挥了治疗作用,又发挥其扶正将养之功。

\zuozhe{(杨旭)}

\section{素問·標本病傅論(節選)}%第五節

\biaoti{【原文】}

\begin{yuanwen}
黃帝問曰:病有標本\sb{1},刺有逆從\sb{2},奈何?岐伯對曰:凡刺之方,必別陰陽\sb{3},前後相應\sb{4},逆從得施\sb{5},標本相移\sb{6},故曰:有其在標而求之於標,有其在本而求之於本;有其在本而求之於標,有其在標而求之於本。故治有取標而得者,有取本而得者,有逆取而得者,有從取而得者。故知逆與從,正行無問\sb{7},知標本者,萬舉萬當,不知標本,是謂妄行。

夫陰陽逆從,標本之爲道也。小而大\sb{8},言一而知百病之害\sb{9},少而多\sb{8},淺而博\sb{8},可以言一而知百也。以淺而知深,察近而知遠,言標與本,易而勿及\sb{10}。治反爲逆,治得爲從\sb{11},先病而後逆者治其本\sb{12},先逆而後生病者治其本,先寒而後生病者治其本,先病而後生寒者治其本,先熱而後生病者治其本,先熱而後生中滿者治其標\sb{13},先病而後泄者治其本,先泄而後生他病者治其本\sb{14},必且調之,乃治其他病。先病而後生中滿着治其標,先中滿而後煩心者治其本。人有客氣有同氣\sb{15},小大不利治其標\sb{16},小大利治其本。病發而有餘,本而標之\sb{17},先治其本,後治其標。病發而不足,標而本之\sb{18},先治其標,後治其本。謹察間甚\sb{19},以意調之,間者并行,甚者獨行\sb{20}。先小大不利而後生病者,治其本。
\end{yuanwen}

\biaoti{【校注】}

\begin{jiaozhu}
	\item 病有标本:标本是一个相对的概念。病有标本,主要是指病发先后,先病为本,后病为标。
	\item 刺有逆从:指针刺等治法有逆治和从治的不同。逆治是病在本而治标,病在标而治本;从治为病在标而治标,病在本而治本。
	\item 必别阴阳:张介宾注:“阴阳二字,所包者广,知经络时令,气血疾病,无所不在。”
	\item 前后相应:前后,指先病后病。
	\item 逆从得施:张介宾注:“或逆或从,得施其法”。
	\item 标本相移:先治本病或先治标病,不是固定不变的,急则治其标,缓则治其本,须视具体的情况而定。
	\item 正行无问:此是正确施行之法,不必问之于人。
	\item 小而大、小而多、浅而博:言掌握了逆从标本之理,就可以使人们对疾病的认识由小到大,由少到多,由浅薄到广博。
	\item 言一而知百病之害:一,指阴阳逆从标本之理。疾病种类虽多,不外阴阳;病证虽杂,不离标本;治法虽众,无非逆从。故言一阴阳逆从标本之理,便可触类旁通,尽知多种疾病危害。
	\item 易而勿及:标本逆从之理是容易理解,但实际具体运用,却不那么容易。
	\item 治反为逆,治得为从:治疗违反标本之理,则为治之逆;符合标本之理,则为治之从。此逆从指治疗效果的成败,与逆治、从治不同。
	\item 先病而后逆者治其本:先病者为本,后病者为标,治其本,是治其病之本源。
	\item 先热而后生中满者治其标;中满为腑气不行,水谷难入之危急证候,故应先治其中满。张介宾《类经·标本类》注:“诸病皆先治本,而惟中满者先治其标,盖以中满为病,其邪在胃,胃者脏腑之本也,胃满则药食之气不能行,而脏腑皆失其所禀,故先治此者,亦所以治本也。”
	\item 先泄而后生他病者治其本:高世栻注:“先泄而后生他病者,治其先泄之本,先泄则中土先虚,既治其本,必且调之,乃治其他病,所以重其中土也。”
	\item 人有客气有同气:《新校正》:“按全元起本‘同’作‘固’。”为是。客气为新感之邪气,固气为原本体内的邪气。先受病为本,后受病为标,则客气为致病之标,固气为致病之本。
	\item 小大不利治其标:大小便不通,应先治其标病。张介宾《类经·标本类》注:“即先有他病,而后为小大不利者,亦先治其标。诸皆治本,此独治标,盖二便不通,乃危急之候,虽为标病,必先治之,此所谓急则治其标也。”
	\item 本而标之:先治其本,而后治其标。
	\item 标而本之:先治其标,而后治其本。
	\item 间甚:间者言病情轻浅,甚者言病情深重。
	\item 间者并行,甚者独行:并行,即标本同治;独行,指单治标或单治本。张介宾《类经·标本类五》:“病浅者可以兼治,故曰并行,病甚者难容杂乱,故曰独行。”
\end{jiaozhu}

\biaoti{【理论阐释】}

1.标本含义

标与本相对而言,常用来概括说明事物的本质与现象,因果关系及病变过程中矛盾的主次关系等。就其本义,本是指草木之根;标又称末,为草木枝叶末梢。《说文》云:“本,本下曰本,从本,一在其下。”《素问·移精变气论》“治以草苏草荄之枝,本末为助,”亦有此义。《内经》在对标本概念的运用中,十分重视标本含义的相对性,并用以比喻事物上与下,内与外,先与后,病与医等相对应双方的主次先后及轻重缓急,具体如下。

(1)六气阴阳标本:风热火湿燥寒六气为本,三阴三阳为标。《素问·六微者大论》云:“少阳之上,火气治之,中见厥阴;阳明之上,燥气治之,中见太阴,……所谓本也,本之下,中之见也,见之下,气之标也,本标不同,气应异象。”是谓标本中气之论。《素问·至真要大论》提出:“少阳太阴从本,少阴太阳从本从标,阳明厥阴不从标本,从乎中也。”说明六气阴阳标本可以推测六气及其所致的气候、病候的变化规律。

(2)医患标本:病在先,医在后。病为主,医为次。医生所施行的治疗方法需通过病人起作用,即《素问·汤液醪醴论》云:“病为本,工为标,标本不得,邪气不服。”

(3)体内结构标本:内在脏器为本,外在形体为标。

(4)病脏间标本:在水液代谢及水肿病的病机中。《素问·水热穴论》云:“其本在肾,其末在肺。”“标本俱病,故肺为喘呼,肾为水肿,肺为逆不得卧,分为相输俱受者,水气之所留也。”

(5)疾病先后标本:原发病、先发病为本,后发病、继发病为标。

后世医家在《内经》的基础上,又进一步扩大了标本的范围,如称正气为本,邪气为标;病因为本,病症为标;旧病为本,新病为标;里病为本,表病为标;在证候上急者为标,缓则为本;外为标,内为本;阳为标,阴为本;六腑为标,五脏为本等。还可以从矛盾运动的法则来认识标本,“本”就是能够反映疾病的本质,即亟待解决的主要矛盾和矛盾的主要方面,而“标”是指疾病反应在外的征象,是次要矛盾和矛盾的次要方面。

2.标本治则

大凡治病,有从本治,有从标治的从治法,有见标从本,见本从标的逆治法,以及标本先后,标本缓急的标本兼治,标本相移的不同情况。从本而治为治疗常法,是治疗疾病的一般规律,本病既愈,标病自除。即所谓“疏其源而流自通。”从标而治为特殊情况下的治疗变法,为权宜之计,标证解除,亦需治本。

诸病多从本,唯“中满”与“大小不利”二症,无论是属标、属本,均需先治。中满属胃气壅滞,水浆难入,药食不纳,则后天化源竭绝,气机转输失主,故先治。相反,泄泻一证,无论先后,“必且调之,乃治其他病”,否则后天之本己衰,诸证难以彻底治愈。体现了《内经》重视脾胃为脏腑之本,气血生化之源的理论观点。对张仲景保胃气的治则以及李东垣重视培土的思想有较大的影响。二便不通反映脾肾二脏功能失常,气机紊乱,亦为危急之候。张介宾《类经·标本类》云:“诸皆治本,此独治标,盖二便不通,乃危急之候,虽为标病,必先治之,此所谓急则治其标也。”后世引申为“急则治其标,缓则治其本”的治疗原则。

有些病证治本则妨碍治标,治标则妨碍治本,应据标本先后而调整。《灵枢·师传》:“春夏先治其标,后治其本;秋冬先治其本,后治其标。”《灵枢·终始》:“病先起于阴者,先治其阴而后治其阳;病先起于阳者,先治其阳而后治其阴。”《灵枢·五色》:“病生于内者,先治其阴,后治其阳,反者益甚;其病生于阳者,先治其外,后治其内,反者益甚。”《素问·至真要大论》:“从内之外者,调其内,从外之内者,治其外;从内之外而盛于外者,先调其内而后治其外;从外之内而盛于内者,先治其外而后调其内。”可见,病先发先治,后发后治乃为常规治法,表现为阴阳内外标本治法。四时气候不同,疾病特征有别,治分标本先后。即春夏气血浮于外,故应先治外而后治内,标而本之;秋冬气血沉于内,故当先治其内,后治其外,本而标之。邪正虚实不同,标本先后各异。原发病邪气有余者,必侮其他脏腑及经脉,是病从本而传于标,故宜先治其本病,而后治其标病;原发病正气不足者,必受他脏他经而侮之,病从标传本,故宜先治他脏经气乘侮之标,而后治正气不足之本。说明当病情复杂多变,标病本病的主从关系发生改变的时候,治疗的重点也要随之加以调整。

病情有间甚之殊,标本有缓急之别,“间者并行,甚者独行”代表《内经》标本治则思想,即本急标缓则治本,标急本缓则治标,标本同等而其势不甚则标本同治。如《素问·评热病论》:治风厥,“表里刺之,饮之服汤”,既治发热之表,又治烦闷之里,属标本同治之“并行”。《素问·病能论》治怒狂阳厥,“服以生铁洛为饮”,取其一味生铁洛,气寒质重,下气急速,而获专攻,属“甚者独行”。

3.治病求本与标本逆从

《素问·阴阳应象大论》云:“治病必求于本”是言治疗疾病必须探求阴阳这个万事万物的根本,能够把握病因病机,疾病本质的阴阳属性,准确的辨证求因,审因论治,以平调气血阴阳而获效。张介宾《类经·论治类》“见痰休治痰,见血休治血,无汗不发汗,有热莫攻热,喘生休耗气,遗精不涩泄,明得个中趣,方是医中杰”,体现了治本之要妙。标本逆从之“本”是与标相对而言,代表疾病发生发展过程中的主要矛盾和矛盾的主要方面,即正气、病因病机、内脏病、旧病、主要症状等,从本论治是强调抓住疾病的主要矛盾,属治疗常法,与治病求阴阳之本虽有联系,但不能互相包容。

\biaoti{【临证相要】}

1.“间者并行,甚者独行”应用示范

《伤寒论》301条:“少阴病,始得之,反发热,脉沉者,麻黄附子细辛汤主之。”本条所论少阴病兼表的证治,因里阳虚不太甚而现表里同病,宜表里同治,以麻黄附子细辛汤温经发汗,表里双解,属间者并行之类。《伤寒论》91条:“伤寒,医下之,续得下利清谷不止,身疼痛者,急当救里;后身疼痛,清便自调者,急当救表,救里宜四逆汤,救表宜桂枝汤”。此乃为甚者独行的应用。

2.痞满先治以保胃气病案举隅

江瓘《名医类案·痞满》:东垣治一贵妇,八月中,先因劳逸饮食失节,加之忧思,病结痞,心腹胀满,旦食则不能暮食,两胁刺痛,诊其脉,弦而细。至夜,浊阴之气当降而不降,腹胀尤甚。大抵阳主运化,饮食劳倦损伤脾胃,阳气不能运化精微,聚而不散,故为胀满。先灸中脘,乃胃之募穴,引胃中升发之气上行阳道。又以木香顺气汤助之,使浊阴之气自此而降矣。

\zuozhe{(王贵臣)}

\section{素問·五常政大論(節選)}%第六節

\biaoti{【原文】}

\begin{yuanwen}
帝曰:有毒無毒,服有約\sb{1}乎?岐伯曰:病有久新,方有大小,有毒無毒\sb{2},固宜常制\sb{3}矣。大毒治病,十去其六;常毒治病,十去其七;小毒治病,十去其八;無毒治病,十去其九;穀肉果菜,食養盡之\sb{4},無使過也,傷其正也。不盡,行復如法\sb{5}。
\end{yuanwen}

\biaoti{【校注】}

\begin{jiaozhu}
	\item 约:高世栻注:“约,规则也”。
	\item 有毒无毒:有毒,指药性峻烈的药物。无毒,指性味平和的药物。
	\item 常制:即服药的一般常规。张介宾注:“病重者宜大,病轻者宜小,无毒者宜多,有毒者宜少,皆常制之约也。”
	\item 大毒治病,……食养尽之:张介宾注:“药性有大毒,常毒,小毒,无毒之分,去病有六分、七分、八分、九分之约者,盖以治病之法,药不及病,则无济于事,药过于病,则反伤其正而生他患矣。故当知约制,而进止有度也。”“然无毒之葯,性虽平和,久而多之,则气有偏胜,必有偏绝,久攻之则脏器偏弱,既弱且困,不可长也,故十去其九而止。病已去其八九而有余未尽者,则当以谷、肉、果、菜,饮食之类培养正气,而余邪自尽矣”。
	\item 行复如法:言病邪尚未尽者,仍重复上法治疗。
\end{jiaozhu}

\biaoti{【理论阐释】}

\xiaobt{攻邪养正}

攻邪养正或曰扶正祛邪,是适用于多种疾病的一项治疗原则。此段经文提出了正与邪、治与养、攻与补即攻邪与养正的关系问题,并论述了用药治病的规则与饮食调养的作用。病有新旧之异,方有大小之别,药有崚缓之分。药虽能治病,但对人体正气也会带来一定损害。因此,应根据药性的峻缓和毒性的有无或大小,而决定治病用药法度及饮食调养。这些,直至今天都是临床应用的基本原则。

\zuozhe{(杨旭)}

\section{素問·至真要大論(節選)}%第七節

\biaoti{【原文】}

\begin{yuanwen}
寒者熱之,熱者寒之\sb{1},微者逆之,甚者從之\sb{2},堅者削之\sb{3},客者除之\sb{4},勞者溫之\sb{5},結者散之\sb{6},留者攻之\sb{7},燥者濡之\sb{8},急者緩之\sb{9},散者收之\sb{10},損者溫之\sb{11},逸者行之\sb{12},驚者平之\sb{13},上之下之,摩之浴之,薄之劫之,開之發之\sb{14},適事爲故。

帝曰:何謂逆從?岐伯曰:逆者正治,從者反治\sb{15},從少從多,觀其事也。帝曰:反治何謂?岐伯曰:熱因熱用,寒因寒用\sb{16},塞因塞用,通因通用\sb{17},必伏其所主,而先其所因\sb{18},其始則同,其終則異\sb{19},可使破積,可使潰堅,可使氣和,可使必已。帝曰:善。氣調而得者\sb{20},何如?岐伯曰:逆之,從之,逆而從之,從而逆之,踈氣令調,則其道也。

帝曰:論言治寒以熱,治熱以寒,而方士不能廢繩墨\sb{21}而更其道也。有病熱者,寒之而熱;有病寒者,熱之而寒。二者皆在,新病复起,奈何治?岐伯曰:諸寒之而熱者取之陰\sb{22},熱之而寒者取之陽\sb{22},所謂求其屬也\sb{24}。
\end{yuanwen}

\biaoti{【校注】}

\begin{jiaozhu}
	\item 寒者热之,热者寒之:指治寒病用温热法,治热病用寒凉法。也就是以热治寒,以寒治热的正治法。
	\item 微者逆之,甚者从之:张介宾注:“病之微者,如阳病则热,阴病则寒,真形易见,其病则微,故可逆之。逆即上文之正治也。病之甚者,如热极反寒,寒极反热,假证难辨,其病则甚,故当从之。从即下文之反治也。”
	\item 坚者削之:指体内有坚积之病,如癥块之类,当用削伐之法。
	\item 客者除之:客,侵犯之意。外邪入侵,用驱除病邪的方法。
	\item 劳者温之:指虚劳之病,用温补法。
	\item 结者散云:指气血郁结,或痰浊,邪气内结等,用消散法。
	\item 留者攻之:指病邪留而不去,如留饮、蓄血、停食、便闭等,用攻下法。
	\item 燥者濡之:指伤津粍气一类干燥病证,用滋润生津等濡润之法。
	\item 急者缓之:指拘急痉挛一类的疾病,用舒缓法。
	\item 散者收之:指精气粍散之病,如自汗、盗汗等用收敛法。
	\item 损者温之:虚损怯弱之病,用温养补益法。
	\item 逸者行之:过于安逸,则气血凝滞不畅,须用行气活血法。
	\item 惊者平之:指惊悸不安一类的病证,用镇静安神之法。
	\item 上之下之,摩之浴之,薄之劫之,开之发之:上之,指病邪在上者,用涌吐法使之上越。下之,指病邪在下者,用攻下法使之下夺。摩之,指按摩法。浴之,指药物浸洗和水浴法,薄之,指侵蚀法。劫之,指强行制止的劫夺法。开之,指开泄法。发之,指发散法。
	\item 逆者正治,从者反治:张介宾注,“以寒治热,以热治寒,逆其病者,谓之正治。以寒治寒,以热治热,从其病者,谓之反治。”
	\item 热因热用,寒因寒用:原本作“热因寒用,寒因热用”,据下文“塞因塞用,通因通用”句改为“热因热用,寒因寒用”。即以热药治疗真寒假热证,以寒药治疗真热假寒证。
	\item 塞因塞用,通因通用:高世栻:“补药治中满,是塞因塞用也。攻药治下利,是通因通用也。”
	\item 必伏其所主,而先其所因:张介宾注:“必伏其所主者,制病之本也。先其所因者,求病之由也。”
	\item 其始则同,其终则异:高世栻注:“热治热,寒治寒,塞用塞,通用通,是其始则同。热者寒,寒者热,塞者通,通者塞,是其终则异。塞因塞用,则正气自强,故可使破积,可使溃坚。通因通用,则邪不能容,故可使气和,可使必已,此反治之道也。”
	\item 气调而得者:张介宾注:“气调而得者,言气调和而偶感于病,则或因天时,或因意料之外者也。若其治法,亦无过逆从而已,或可逆者,或可从者,或先逆而后从者,或先从而后逆者,但疏其邪气而使之调和,则治道尽矣。”
	\item 绳墨:这里是准则的意思。
	\item 寒之而热者取之阴:指由阴虚而引起的发热证,用苦寒泻热而热不退,省当用补阴法治疗。
	\item 热之而寒者取之阳:指因阳虚而引起的寒证,用辛热散寒而寒不去,当用补阳法治疗。
	\item 求其属:推求疾病的本质究属于阴,属于阳。王冰注:“言益火之源,以消阴翳,壮水之主,以制阳光,故曰求其属也。”
\end{jiaozhu}

\biaoti{【理论阐释】}

\xiaobt{治法逆从}

治法逆从包括逆治法与从治法。“逆者正治,从者反治。”故逆治法又叫正治法,从治法又叫反治法。这是《素问·至真要大论》提出的治疗理论。在疾病的发展过程中,疾病的外在表现与体证,在一般的情况下与疾病的本质是一致的,特别是一些初起的或较轻的疾病更少有例外,其治疗用药的药性逆其表象即可,即“微者逆之”;某些重病,久病则可能出现与本质不相一致的症状与体征,即产生假象,此时用药则须顺其表象,即“甚者从之”。从治的结果是“其始则同,其终则异。”这是因为,无论哪种方法,欲克制病邪,获得疗效,都必须“伏其所主,而先其所因”,即必须针对疾病的本质,消除疾病的原因。因此,可以说正治法与反治法是治病求本原则的两种表面相反而实则归一的表现形式。实质上,两者都是针对疾病的本质而治。然而反治法这种形式又不是可有可无的,它的提出便于临证警惕并识别假象,洞察本质,从而不失时机地使用针对病本的药物。针对寒热虚实的真与假,《内经》不仅有“寒者热之,热者寒之”,“盛者泻之,虚则补之”等正治法,更提出“寒因寒用,热因热用,塞因塞用,通因通用”的反治法。临证准确自如地使用正治、反治两法亦诚非易事,正如张介宾《类经·论治类》所说:“寒热有真假,虚实亦有真假。真者正治,知之无难,假者反治,乃为难耳。”他进而列举了“阳证似阴,火极似水”,阳盛格阴的假寒证,“阴证似阳,水极似火”,阴盛格阳的假热证和“至虚有盛候”的假实证,“大实有羸状”的假虚证的多种表现,指出“不可不辨其真耳”,“世有不明真假本末而曰知医者,余未敢许也”。可见,正反之用的前提在于辨明真假,抓住本质,不为假象所迷惑。

\biaoti{【临证指要】}

\xiaobt{关于反治法的运用}

张仲景《伤寒论》中有许多反治法的记载,即:

热因热用:《伤寒论》317条“少阴病,下利清谷,里寒外热,手足厥逆,脉微欲绝,身反不恶寒,其人面色赤,或腹痛,或干呕,或咽痛,或利止脉不出者,通脉四逆汤主之。”身热而赤而反用姜附重剂,以期立挽阴盛格阳之势。

寒因寒用:《伤寒论》350条“伤寒,脉滑而厥者,里有热,白虎汤主之。”表象为手足厥冷,用辛寒清热之白虎汤,实治其里热。

塞因塞用:《伤寒论》273条“太阴之为病,腹满而吐,食不下,自利益甚,时腹自痛”,治以理中汤之类,此虽腹满,但须参、姜、术、甘、温中散寒,健脾燥湿,则胀满自利自消。

通因通用:《伤寒论》321条“少阴病,自利清水,色纯清,心下必痛,口干燥者,急下之,宜大承气汤。”自利清水,反用大承气峻下,治其热结旁流,而旁流是假,热结是真,热结得下,旁流亦自止。

\biaoti{【原文】}

\begin{yuanwen}
帝曰:善。方制君臣何謂也?岐伯曰:主病之謂君,佐君之謂臣,應臣之謂使\sb{1}。

帝曰:氣有多少,病有盛衰,治有緩急,方有大小,願聞其約\sb{2},奈何?岐伯曰:氣有高下,病有遠近,證有中外,治有輕重,適其至所爲故\sb{3}也。大要曰:君一臣二,奇之制也\sb{4};君二臣四,偶之制也\sb{5};君二臣三,奇之制也;君二臣六,偶之制也。故曰:近者奇之,遠者偶之,汗者不以奇,下者不以偶\sb{6},補上治上制以緩,補下治下制以急\sb{7}。急則氣味厚,緩則氣味薄,適其治所,此之謂也。病所遠,而中道氣味之者,食而過之\sb{8},無越其制度也。是故平氣之道\sb{9},近而奇偶,制小其服也;遠而奇偶,制大其服也。大則數少;小則數多,多則九之,少則二之。奇之不去則偶之\sb{10},是謂重方;偶之不去,則反佐\sb{11}以取之,所謂寒熱溫涼,反從其病也。
\end{yuanwen}

\biaoti{【校注】}

\begin{jiaozhu}
	\item 主病之谓君,佐君之谓臣,应臣之谓使:君、臣、佐、使是立方之准则。张介宾注:“主病者,对证之要药也,故谓之君。君者,味数少而分量重,赖之以为主也。佐君者谓之臣,味数稍多而分量稍轻,所以匡君之不迨也。应臣者谓之使,数可出入而分量更轻,所以备通行向导之使也。”
	\item 约:犹准则也。
	\item 适其至所为故:王冰注:“脏位有高下,腑气有远近,病证有表里,药用有轻重,调其多少,知其紧慢,令药气至病所为故,勿太过与不及也。”
	\item 奇制也:即奇方。王冰注:“奇,谓古之单方。”
	\item 偶制也:即偶方。王冰注:“偶,谓古之复方也。”
	\item 汗者不以奇,下者不以偶:张琦疑“奇”“偶”二字误倒。《素问吴注》和《类经》均改为“汗者不以偶,下者不以奇。”可从。王冰注:“汁药不以偶方,气不足以外发泄。下药不以奇制,药毒攻而致过。”奇方药少力专,偶方药多而力广。
	\item 补上治上制以缓,补下治下制以急:吴昆云:“补上治上治以缓,恐其下迫也。补下治下治以急,恐其中留也。制急方气味薄,则力与缓等,制缓方而气味厚,则势与急同,故急则气味厚,缓则气味薄。总之适至病所耳。”高世栻注:“治之缓急,因病之上下以为用,病在上而补上治上,则制方以缓。病在下而补下治下,则制方以急。制以忽,则气味宜厚,气味厚,则能下行也。制以缓,则气味宜薄,气味薄,则能上行也。”二说可互参。
	\item 病所远,而中道气味之者,食而过之:马莳注:“病所远,而药食气味止于中道。”如病在上焦者,应先食物而后服药,病在下焦者,应先服药而后食物,以免食物阻隔药物之气味,使其药效中途消失。这是饭前服药或饭后服药的一种常法。
	\item 平气之道:张介宾注:“平气之道,平其不平之谓也。”
	\item 奇之不去则偶之:王冰注:“方与其重也,宁轻;与其毒也,宁善;与其大也,宁小。是以奇方不去,偶方主之。”
	\item 反佐:指处方中药物组成的反佐法,即在寒药方中佐以热药,在热药方中佐以寒药。如白通加猪胆计汤,用猪胆汁即为反佐。
\end{jiaozhu}

\biaoti{【理论阐释】}

1.制方法则

《内经》根据药物性能作用和病变特点提出两种制方法则。其一是以药物作用的主次确立“君、臣、佐、使”的组方原则。提出“主病之谓君,佐君之谓臣,应臣之谓使”。指出方剂中“君”是针对主证,起主要作用的药物;“臣”是协同和加强君药功效的药物;“使”是引药达于病所或调和诸药的药物。一般处方除必须确定君药外,其它臣、佐、使之药是否需要,以及使用的药味和用量多少,可根据病情而定。这一制方法则,一直沿用至今。其二是因病制方。选药组方,是以疾病的客观存在为依据的,不论选用任何药为君,为臣,为佐,为使,以及每类药物的味数与用量,都以适合病情为原则。正如经文所述“气有高下,病有远近,治有轻重,适其至所为故也”,“有毒无毒,所治为主,适大小为制也”,即是此意。

2.方剂分类

《内经》根据君、臣、佐、使各类药物的味数与用量,将方剂分为大、小、缓、急、奇、偶、重七类。

大方和小方是根据药味数的多少而分,即“君一臣二,制之小也;君一臣三佐五,制之中也;君一臣三佐九,制之大也”。凡臣、佐之药味数多者即为大方,味数少者即为小方。大方用于治疗较为复杂严重的疾病,小方用于比较单纯或轻浅的疾病。如张志聪《黄帝内经素问集注》所言“病之微者,制小其服。病之甚者,制大其服。”

奇方和偶方,是以药物味数的单数或双数来区分的。如一味君药,二味臣药,总数是三,为奇数,则称为奇方;二味君药,四味臣药,总数是六,为偶数,则称偶方。奇方的药味为单数,治疗作用单一而轻;偶方的药味为双数,治疗作用较多而大。奇方和偶方的作用并不是绝对的,其各方功效之强弱,还与用药的分量有关。原文中提到的大、小二字。大,指用量大,而其味数少,则药力专一,故能治部位较“远”的病证;小,指用量小,而其味数较多,则药力轻散,故用以治疗病位较“近”之病证。

缓方与急方,是以药力而言。气味薄而药力缓的方剂,称为缓方。气味厚而药力峻烈的方剂,称为急方。如;病在上焦者,欲其药力作用于上,则宜用缓方;病在下焦者,欲其药力能直达下焦病所,则宜用急方。此外,如病情较缓的,可用缓方;病势危急的,当用急方。

若病情复杂,单独使用奇方或偶方,大方或小方等不易奏效的,应综合使用各类方剂以治之,如此组成之方叫做重方,后世又惯称之为“复方”。张志聪《黄帝内经素问集注》解释说:“所谓重方者,谓之奇偶之并用也。”

\xiaojie

本章选择《内经》中比较集中阐发治则、治法的篇章,作为学习研究的示范。

治则,包括针对疾病寒热、虚实的治则;疾病病位表里上下内外的治则;病发先后的治则;疾病轻重、微甚的治则;邪正的治则;气机聚散的治则及动静的治则等。此外,标本先后也属于治则范畴。病发先后的标本治疗,原则上先治先发之病,后治后发之病,即“本而标之”,属治法之常。在特殊情况下,先治其后发病而后治其先发病,即“标而本之”,属于权宜之策。疾病的标本治疗尚包括合治和分治,疾病较轻者,则“间者并行”而合治;而对于无论是先病或是后病的重者,则行单治“甚者独行”。

治法,除五方不同治法外,在药物疗法中主要有解表法、消导法、攻下法以及益气、温阳、滋阴、补血法等,涵盖了后世汗吐下和消清温补治疗八法。阐发了正治、反治法,阳虚、阴虚生寒热的治法,病发先后的标本治法,以及治法的综合运用。

正治法、反治法是针对疾病微甚而提出的两种治法。正治法适用于病情较轻而单纯的疾病,治疗上所选方药的属性与病气相逆,是常规的治疗方法,即“微者逆之”。反治法是区别于正治法的变通治疗方法,适用于病情较重而复杂的疾病,治疗上所选方药与病象相从,即“甚者从之”。“必伏其所主而先其所因,其始则同,其终则异”。反治法中药性与“所主”(治病之本)相逆,这与正治法在治疗本质上是一致的,因而正治、反治两法殊途而同归。

阳虚阴虚寒热的治疗,临证易犯“虚虚”之误,原因在于只见寒热而不辨虚实,错把虚证当实证。热由阴虚当甘寒益阴,以“壮水之主”;寒由阳虚当甘温助阳,以“益火之源”。抓住了疾病的本质,才能应手获效。

治法综合运用,即“圣人杂合以治”,主张不同的治疗方法合理配合使用。《素问·汤液醪醴论》运用按摩、温衣、缪刺、发汗利小便、服五谷汤液醪醴治疗阳虚水肿,是不同治疗方法综合运用的范例。

\zuozhe{(王贵臣)}
\ifx\allfiles\undefined
\end{document}
\fi