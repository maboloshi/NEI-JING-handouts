% -*- coding: utf-8 -*-
%!TEX program = xelatex
\ifx \allfiles \undefined
\documentclass[draft,12pt]{ctexbook}
%\usepackage{xeCJK}
%\usepackage[14pt]{extsizes} %支持8,9,10,11,12,14,17,20pt

%===================文档页面设置====================
%---------------------印刷版尺寸--------------------
%\usepackage[a4paper,hmargin={2.3cm,1.7cm},vmargin=2.3cm,driver=xetex]{geometry}
%--------------------电子版------------------------
\usepackage[a4paper,margin=2cm,driver=xetex]{geometry}
%\usepackage[paperwidth=9.2cm, paperheight=12.4cm, width=9cm, height=12cm,top=0.2cm,
%            bottom=0.4cm,left=0.2cm,right=0.2cm,foot=0cm, nohead,nofoot,driver=xetex]{geometry}

%===================自定义颜色=====================
\usepackage{xcolor}
	\definecolor{mybackgroundcolor}{cmyk}{0.03,0.03,0.18,0}
	\definecolor{myblue}{rgb}{0,0.2,0.6}

%====================字体设置======================
%--------------------中文字体----------------------
%-----------------------xeCJK下设置中文字体------------------------------%
\setCJKfamilyfont{song}{SimSun}                             %宋体 song
\newcommand{\song}{\CJKfamily{song}}                        % 宋体   (Windows自带simsun.ttf)
\setCJKfamilyfont{xs}{NSimSun}                              %新宋体 xs
\newcommand{\xs}{\CJKfamily{xs}}
\setCJKfamilyfont{fs}{FangSong_GB2312}                      %仿宋2312 fs
\newcommand{\fs}{\CJKfamily{fs}}                            %仿宋体 (Windows自带simfs.ttf)
\setCJKfamilyfont{kai}{KaiTi_GB2312}                        %楷体2312  kai
\newcommand{\kai}{\CJKfamily{kai}}
\setCJKfamilyfont{yh}{Microsoft YaHei}                    %微软雅黑 yh
\newcommand{\yh}{\CJKfamily{yh}}
\setCJKfamilyfont{hei}{SimHei}                                    %黑体  hei
\newcommand{\hei}{\CJKfamily{hei}}                          % 黑体   (Windows自带simhei.ttf)
\setCJKfamilyfont{msunicode}{Arial Unicode MS}            %Arial Unicode MS: msunicode
\newcommand{\msunicode}{\CJKfamily{msunicode}}
\setCJKfamilyfont{li}{LiSu}                                            %隶书  li
\newcommand{\li}{\CJKfamily{li}}
\setCJKfamilyfont{yy}{YouYuan}                             %幼圆  yy
\newcommand{\yy}{\CJKfamily{yy}}
\setCJKfamilyfont{xm}{MingLiU}                                        %细明体  xm
\newcommand{\xm}{\CJKfamily{xm}}
\setCJKfamilyfont{xxm}{PMingLiU}                             %新细明体  xxm
\newcommand{\xxm}{\CJKfamily{xxm}}

\setCJKfamilyfont{hwsong}{STSong}                            %华文宋体  hwsong
\newcommand{\hwsong}{\CJKfamily{hwsong}}
\setCJKfamilyfont{hwzs}{STZhongsong}                        %华文中宋  hwzs
\newcommand{\hwzs}{\CJKfamily{hwzs}}
\setCJKfamilyfont{hwfs}{STFangsong}                            %华文仿宋  hwfs
\newcommand{\hwfs}{\CJKfamily{hwfs}}
\setCJKfamilyfont{hwxh}{STXihei}                                %华文细黑  hwxh
\newcommand{\hwxh}{\CJKfamily{hwxh}}
\setCJKfamilyfont{hwl}{STLiti}                                        %华文隶书  hwl
\newcommand{\hwl}{\CJKfamily{hwl}}
\setCJKfamilyfont{hwxw}{STXinwei}                                %华文新魏  hwxw
\newcommand{\hwxw}{\CJKfamily{hwxw}}
\setCJKfamilyfont{hwk}{STKaiti}                                    %华文楷体  hwk
\newcommand{\hwk}{\CJKfamily{hwk}}
\setCJKfamilyfont{hwxk}{STXingkai}                            %华文行楷  hwxk
\newcommand{\hwxk}{\CJKfamily{hwxk}}
\setCJKfamilyfont{hwcy}{STCaiyun}                                 %华文彩云 hwcy
\newcommand{\hwcy}{\CJKfamily{hwcy}}
\setCJKfamilyfont{hwhp}{STHupo}                                 %华文琥珀   hwhp
\newcommand{\hwhp}{\CJKfamily{hwhp}}

\setCJKfamilyfont{fzsong}{Simsun (Founder Extended)}     %方正宋体超大字符集   fzsong
\newcommand{\fzsong}{\CJKfamily{fzsong}}
\setCJKfamilyfont{fzyao}{FZYaoTi}                                    %方正姚体  fzy
\newcommand{\fzyao}{\CJKfamily{fzyao}}
\setCJKfamilyfont{fzshu}{FZShuTi}                                    %方正舒体 fzshu
\newcommand{\fzshu}{\CJKfamily{fzshu}}

\setCJKfamilyfont{asong}{Adobe Song Std}                        %Adobe 宋体  asong
\newcommand{\asong}{\CJKfamily{asong}}
\setCJKfamilyfont{ahei}{Adobe Heiti Std}                            %Adobe 黑体  ahei
\newcommand{\ahei}{\CJKfamily{ahei}}
\setCJKfamilyfont{akai}{Adobe Kaiti Std}                            %Adobe 楷体  akai
\newcommand{\akai}{\CJKfamily{akai}}

%------------------------------设置字体大小------------------------%
\newcommand{\chuhao}{\fontsize{42pt}{\baselineskip}\selectfont}     %初号
\newcommand{\xiaochuhao}{\fontsize{36pt}{\baselineskip}\selectfont} %小初号
\newcommand{\yihao}{\fontsize{28pt}{\baselineskip}\selectfont}      %一号
\newcommand{\xiaoyihao}{\fontsize{24pt}{\baselineskip}\selectfont}
\newcommand{\erhao}{\fontsize{21pt}{\baselineskip}\selectfont}      %二号
\newcommand{\xiaoerhao}{\fontsize{18pt}{\baselineskip}\selectfont}  %小二号
\newcommand{\sanhao}{\fontsize{15.75pt}{\baselineskip}\selectfont}  %三号
\newcommand{\sihao}{\fontsize{14pt}{\baselineskip}\selectfont}%     四号
\newcommand{\xiaosihao}{\fontsize{12pt}{\baselineskip}\selectfont}  %小四号
\newcommand{\wuhao}{\fontsize{10.5pt}{\baselineskip}\selectfont}    %五号
\newcommand{\xiaowuhao}{\fontsize{9pt}{\baselineskip}\selectfont}   %小五号
\newcommand{\liuhao}{\fontsize{7.875pt}{\baselineskip}\selectfont}  %六号
\newcommand{\qihao}{\fontsize{5.25pt}{\baselineskip}\selectfont}    %七号   %中文字体及字号设置
\xeCJKDeclareSubCJKBlock{SIP}{
	"20000 -> "2A6DF,   % CJK Unified Ideographs Extension B
	"2A700 -> "2B73F,   % CJK Unified Ideographs Extension C
	"2B740 -> "2B81F    % CJK Unified Ideographs Extension D
}
%\setCJKmainfont[SIP={[AutoFakeBold=1.8,Color=red]Sun-ExtB},BoldFont=黑体]{宋体}    % 衬线字体 缺省中文字体

\setCJKmainfont[Path=fonts/,
				SIP={[Path=fonts/,AutoFakeBold=1.8,Color=red]simsunb.ttf},
				BoldFont={simhei.ttf}]{simsun.ttc}
%SimSun-ExtB
%Sun-ExtB
%AutoFakeBold:自动伪粗,即正文使用\bfseries时生僻字使用伪粗体;
%FakeBold:强制伪粗,即正文中生僻字均使用伪粗体
%\setCJKmainfont[BoldFont=STHeiti,ItalicFont=STKaiti]{STSong}
%\setCJKsansfont{微软雅黑}黑体
%\setCJKsansfont[BoldFont=STHeiti]{STXihei} %serif是有衬线字体sans serif 无衬线字体
%\setCJKmonofont{STFangsong}    %中文等宽字体

%--------------------英文字体----------------------
\setmainfont[Path=fonts/,
			 BoldFont={simhei.ttf}]{simsun.ttc}
%\setmainfont[BoldFont=黑体]{宋体}  %缺省英文字体
%\setsansfont
%\setmonofont

%===================目录分栏设置====================
\usepackage[toc,lof,lot]{multitoc}      % 目录(含目录、表格目录、插图目录)分栏设置
	%\renewcommand*{\multicolumntoc}{3} % toc分栏数设置,默认为两栏(\multicolumnlof,\multicolumnlot)
	%\setlength{\columnsep}{1.5cm}      % 调整分栏间距
	\setlength{\columnseprule}{0.2pt}   % 调整分栏竖线的宽度

%==================章节格式设置====================
\setcounter{secnumdepth}{3} % 章节等编号深度 3:子子节\subsubsection
\setcounter{tocdepth}{2}    % 目录显示等度 2:子节

\xeCJKsetup{%
	CJKecglue=\hspace{0.15em},      % 调整中英(含数字)间的字间距
	%CJKmath=true,                  % 在数学环境中直接输出汉字(不需要\text{})
	AllowBreakBetweenPuncts=true,   % 允许标点中间断行,减少文字行溢出
}

\ctexset{%
	part={
		name={,篇},
		number=\SZX{part},
		format={\chuhao\bfseries\centering},
		nameformat={},titleformat={}
	},
	section={
		number={\chinese{section}},
		name={第,节}
	},
	subsection={
		number={\chinese{subsection}、},
		aftername={\hspace{-0.01em}}
	},
	subsubsection={
		number={(\chinese{subsubsection})},
		aftername={\hspace {-0.01em}},
		beforeskip={1.3ex minus .8ex},
		afterskip={1ex minus .6ex},
		indent={\parindent}
	},
	paragraph={
		beforeskip=.1\baselineskip,
		indent={\parindent}
	}
}

\newcommand*\SZX[1]{%
	\ifcase\value{#1}%
		\or 上%
		\or 中%
		\or 下%
	\fi
}

%====================页眉设置======================
\usepackage{titleps}%或者\usepackage{titlesec},titlesec包含titleps
\newpagestyle{special}[\small\sffamily]{
	%\setheadrule{.1pt}
	\headrule
	\sethead[\usepage][][\chaptertitle]
	{\chaptertitle}{}{\usepage}
}

\newpagestyle{main}[\small\sffamily]{
	\headrule
	%\sethead[\usepage][][第\thechapter 章\quad\chaptertitle]
%  {\thesection\quad\sectiontitle}{}{\usepage}}
	\sethead[\usepage][][第\chinese{chapter}章\quad\chaptertitle]
	{第\chinese{section}节\quad\sectiontitle}{}{\usepage}
}

\newpagestyle{main2}[\small\sffamily]{
	\headrule
	\sethead[\usepage][][第\chinese{chapter}章\quad\chaptertitle]
	{第\chinese{section}節\quad\sectiontitle}{}{\usepage}
}

%================ PDF 书签设置=====================
\usepackage[depth=2,        % 书签深度 2:子节
			open,           % 默认展开书签
			openlevel=2,    % 展开书签深度 2:子节
			numbered,       % 显示编号
			atend,
			]{bookmark}     % 相比hyperref,该宏包大多数时候只需要编译一次,
							% 而且书签的颜色和字体也可以定制。
							% 比hyperref 更专业 (自动加载hyperref)

%\bookmarksetup{italic,bold,color=blue} % 书签字体斜体/粗体/颜色设置

%------------重置每篇章计数器,必须在hyperref/bookmark之后------------
\makeatletter
	\@addtoreset{chapter}{part}
\makeatother

%------------hyperref 超链接设置------------------------
\hypersetup{%
	pdfencoding=auto,   % 解决新版ctex,引起hyperref UTF-16预警
	colorlinks=true,    % 注释掉此项则交叉引用为彩色边框true/false
	pdfborder=001,      % 注释掉此项则交叉引用为彩色边框
	citecolor=teal,
	linkcolor=myblue,
	urlcolor=black,
	%psdextra,          % 配合使用bookmark宏包,可以直接在pdf 书签中显示数学公式
}

%------------PDF 属性设置------------------------------
\hypersetup{%
	pdfkeywords={黄帝内经,内经,内经讲义,21世纪课程教材},    % 关键词
	%pdfsubject={latex},        % 主题
	pdfauthor={主编:王洪图}, % 作者
	pdftitle={内经讲义},        % 标题
	%pdfcreator={texlive2011}   % pdf创建器
}

%------------PDF 加密----------------------------------
%仅适用于xelatex引擎 基于xdvipdfmx
%\special{pdf:encrypt ownerpw (abc) userpw (xyz) length 128 perm 2052}

%仅适用于pdflatex引擎
%\usepackage[owner=Donald,user=Knuth,print=false]{pdfcrypt}

%其他可使用第三方工具 如:pdftk
%pdftk inputfile.pdf output outputfile.pdf encrypt_128bit owner_pw yourownerpw user_pw youruserpw

%=============自定义环境、列表及列表设置================
\def\biaoti#1{\vspace{1.7ex plus 3ex minus .2ex}{\bfseries #1}}%\noindent\hei
\def\xiaobt#1{{\bfseries #1}}
\def\xiaojie {\vspace{1.8ex plus .3ex minus .3ex}\centerline{\large\bfseries 小\ \ 结}\vspace{.1\baselineskip}}
\def\zuozhe#1{\rightline{\bfseries #1}}

\newcounter{yuanwen}    % 新计数器 yuanwen
\newcounter{jiaozhu}    % 新计数器 jiaozhu

\newenvironment{yuanwen}[2][【原文】]{%
	%\biaoti{#1}\par
	\stepcounter{yuanwen}   % 计数器 yuanwen+1
	\bfseries #2}
	{}

\usepackage{enumitem}
\newenvironment{jiaozhu}[1][【校注】]{%
	%\biaoti{#1}\par
	\stepcounter{jiaozhu}   % 计数器 jiaozhu+1
	\begin{enumerate}[%
					label=\mylabel{\arabic*}{\circledctr*},before=\small,fullwidth,%
					itemindent=\parindent,listparindent=\parindent,%labelsep=-1pt,%labelwidth=0em,
					itemsep=0pt,topsep=0pt,partopsep=0pt,parsep=0pt]}
	{\end{enumerate}}

%===================注解与原文相互跳转====================
%----------------第1部分 设置相互跳转锚点-----------------
\makeatletter
	\protected\def\mylabel#1#2{% 注解-->原文
		\hyperlink{back:\theyuanwen:#1}{\Hy@raisedlink{\hypertarget{\thejiaozhu:#1}{}}#2}}

	\protected\def\myref#1#2{% 原文-->注解
		\hyperlink{\theyuanwen:#1}{\Hy@raisedlink{\hypertarget{back:\theyuanwen:#1}{}}#2}}
	%此处\theyuanwen:#1实际指thejiaozhu:#1,只是\thejiaozhu计数器还没更新,故使用\theyuanwen计数器代替
\makeatother

\protected\def\myjzref#1{% 脚注中的引用(引用到原文)
	\hyperlink{\theyuanwen:#1}{\circlednum{#1}}}

\def\sb#1{\myref{#1}{\textsuperscript{\circlednum{#1}}}}    % 带圈数字上标

%----------------第2部分 调整锚点垂直距离-----------------
\def\HyperRaiseLinkDefault{.8\baselineskip} %调整锚点垂直距离
%\let\oldhypertarget\hypertarget
%\makeatletter
%   \def\hypertarget#1#2{\Hy@raisedlink{\oldhypertarget{#1}{#2}}}
%\makeatother

%====================带圈数字列表标头====================
%\newfontfamily\circledfont[Path = fonts/]{meiryo.ttc}  % 日文字体,明瞭体
\newfontfamily\circledfont{Meiryo}  % 日文字体,明瞭体

\protected\def\circlednum#1{{\makexeCJKinactive\circledfont\textcircled{#1}}}

\newcommand*\circledctr[1]{%
	\expandafter\circlednum\expandafter{\number\value{#1}}}
\AddEnumerateCounter*\circledctr\circlednum{1}

% 参考自:http://bbs.ctex.org/forum.php?mod=redirect&goto=findpost&ptid=78709&pid=460496&fromuid=40353

%======================插图/tikz图========================
\usepackage{graphicx,subcaption,wrapfig}    % 图,subcaption含子图功能代替subfig,图文混排
	\graphicspath{{img/}}                   % 设置图片文件路径

\def\pgfsysdriver{pgfsys-xetex.def}         % 设置tikz的驱动引擎
\usepackage{tikz}
	\usetikzlibrary{calc,decorations.text,arrows,positioning}

%---------设置tikz图片默认格式(字号、行间距、单元格高度)-------
\let\oldtikzpicture\tikzpicture
\renewcommand{\tikzpicture}{%
	\small
	\renewcommand{\baselinestretch}{0.2}
	\linespread{0.2}
	\oldtikzpicture
}

%=========================表格相关===============================
\usepackage{%
	multirow,                   % 单元格纵向合并
	array,makecell,longtable,   % 表格功能加强,tabu的依赖
	tabu-last-fix,              % "强大的表格工具" 本地修复版
	diagbox,                    % 表头斜线
	threeparttable,             % 表格内脚注(需打补丁支持tabu,longtabu)
}

%----------给threeparttable打补丁用于tabu,longtabu--------------
%解决方案来自:http://bbs.ctex.org/forum.php?mod=redirect&goto=findpost&ptid=80318&pid=467217&fromuid=40353
\usepackage{xpatch}

\makeatletter
	\chardef\TPT@@@asteriskcatcode=\catcode`*
	\catcode`*=11
	\xpatchcmd{\threeparttable}
		{\TPT@hookin{tabular}}
		{\TPT@hookin{tabular}\TPT@hookin{tabu}}
		{}{}
	\catcode`*=\TPT@@@asteriskcatcode
\makeatother

%------------设置表格默认格式(字号、行间距、单元格高度)------------
\let\oldtabular\tabular
\renewcommand{\tabular}{%
	\renewcommand\baselinestretch{0.9}\small    % 设置行间距和字号
	\renewcommand\arraystretch{1.5}             % 调整单元格高度
	%\renewcommand\multirowsetup{\centering}
	\oldtabular
}
%设置行间距,且必须放在字号设置前 否则无效
%或者使用\fontsize{<size>}{<baseline>}\selectfont 同时设置字号和行间距

\let\oldtabu\tabu
\renewcommand{\tabu}{%
	\renewcommand\baselinestretch{0.9}\small    % 设置行间距和字号
	\renewcommand\arraystretch{1.8}             % 调整单元格高度
	%\renewcommand\multirowsetup{\centering}
	\oldtabu
}

%------------模仿booktabs宏包的三线宽度设置---------------
\def\toprule   {\Xhline{.08em}}
\def\midrule   {\Xhline{.05em}}
\def\bottomrule{\Xhline{.08em}}
%-------------------------------------
%\setlength{\arrayrulewidth}{2pt} 设定表格中所有边框的线宽为同样的值
%\Xhline{} \Xcline{}分别设定表格中水平线的宽度 makecell包提供

%表格中垂直线的宽度可以通过在表格导言区(preamble),利用命令 !{\vrule width1.2pt} 替换 | 即可

%=================图表设置===============================
%---------------图表标号设置-----------------------------
\renewcommand\thefigure{\arabic{section}-\arabic{figure}}
\renewcommand\thetable {\arabic{section}-\arabic{table}}

\usepackage{caption}
	\captionsetup{font=small,}
	\captionsetup[table] {labelfont=bf,textfont=bf,belowskip=3pt,aboveskip=0pt} %仅表格 top
	\captionsetup[figure]{belowskip=0pt,aboveskip=3pt}  %仅图片 below

%\setlength{\abovecaptionskip}{3pt}
%\setlength{\belowcaptionskip}{3pt} %图、表题目上下的间距
\setlength{\intextsep}   {5pt}  %浮动体和正文间的距离
\setlength{\textfloatsep}{5pt}

%====================全文水印==========================
%解决方案来自:
%http://bbs.ctex.org/forum.php?mod=redirect&goto=findpost&ptid=79190&pid=462496&fromuid=40353
%https://zhuanlan.zhihu.com/p/19734756?columnSlug=LaTeX
\usepackage{eso-pic}

%eso-pic中\AtPageCenter有点水平偏右
\renewcommand\AtPageCenter[1]{\parbox[b][\paperheight]{\paperwidth}{\vfill\centering#1\vfill}}

\newcommand{\watermark}[3]{%
	\AddToShipoutPictureBG{%
		\AtPageCenter{%
			\tikz\node[%
				overlay,
				text=red!50,
				%font=\sffamily\bfseries,
				rotate=#1,
				scale=#2]
				{#3};
		}
	}
}

\newcommand{\watermarkoff}{\ClearShipoutPictureBG}

\watermark{45}{15}{草\ 稿}    %启用全文水印

%=============花括号分支结构图=========================
\usepackage{schemata}

\xpatchcmd{\schema}
	{1.44265ex}{-1ex}
	{}{}

\newcommand\SC[2] {\schema{\schemabox{#1}}{\schemabox{#2}}}
\newcommand\SCh[4]{\Schema{#1}{#2}{\schemabox{#3}}{\schemabox{#4}}}

%=======================================================

\begin{document}
\pagestyle{main2}
\fi
\chapter{养生}%第八章
养生,即保养生命,它是通过各种方法,颐养生命,增强体质,预防疾病,从而达到延年益寿目的的一种医事活动。

人类的养生活动起源很早。据《路史》记载,早在唐尧时代,人们就知道用舞蹈预防关节疾病。战国秦汉著作中,就有不少养生方法和学说的论述,但形成较为系统的理论,则始于《内经》。养生学说是《内经》学术体系的重要组成部分,它主要阐述生命衰亡的机理,养生的原理、原则与方法。

《内经》有关养生的系统内容见于《素问》的“上古天真论”、“四气调神大论”、“生气通天论”及《灵枢》的“本神”、“天年”等篇,但散于各篇的片断文论亦复不少,学习时宜互参。

\section{素问·上古天真論}%第一節

\biaoti{【原文】}

\begin{yuanwen}
昔在黃帝,生而神靈,弱而能言\sb{1} ,幼而徇齊\sb{2} ,長而敦敏\sb{3} ,成而登天\sb{4} ,廼\sb{5} 問於天師曰:余聞上古之人,春秋皆度百歲而動作不衰;今時之人,年半百而動作皆衰者,時世异耶?人将失之耶?

岐伯對曰:上古之人,其知道者,法于陰陽,和於術數\sb{6} ,食饮有節,起居有常,不妄作勞,故能形与神俱\sb{7} ,而盡终其天年,度百歲乃去。今時之人不然也,以酒爲漿,以妄爲常,醉以入房,以欲竭其精,以耗\sb{8} 散其真,不知持满\sb{9} ,不時御神\sb{10} ,務快其心,逆于生樂\sb{11} ,起居無節,故半百而衰也。

夫上古聖人之教下也,皆謂之\sb{12} 虛邪贼風\sb{13} ,避之有时,恬惔虛无\sb{14} ,真氣從之,精神內守\sb{15} ,病安从来?是以志閑而少欲\sb{16} ,心安而不惧,形勞而不倦\sb{17} ,氣從以顺,各從其欲,皆得所願。故美其食,任其服\sb{18} ,樂其俗,髙下不相慕\sb{19} ,其民故曰樸\sb{20} 。是以嗜欲不能勞其目\sb{21} ,淫邪不能惑其心,愚智賢不肖,不懼于物\sb{22} ,故合于道。所以能年皆度百岁而動作不衰者,以其德全不危\sb{23} 也。
\end{yuanwen}

\biaoti{【校注】}

\begin{jiaozhu}
  \item 弱而能言:《史记索隐》:“弱,谓幼弱时也。盖未合能言之时,而黄帝即言。”
  \item 徇齐:徇(xùn音迅),周遍。齐,迅速。徇齐,言博知而迅速。
  \item 敦敏:敦厚敏达。
  \item 登天:登天子之位。又,丹波元坚曰:“以上六句,疑王氏所补,非古经之文……其文取之于《史记》《大戴礼》及《孔子家语》”。
  \item 廼:通“乃”字。
  \item 术数:此指专门的养生方法和技术。
  \item 形与神俱:俱,偕也,有共存、协调之意。姚止庵注:“形者神所依,神者形所根,神形相离,行尸而已。故惟知道者,为能形与神俱。”
  \item 耗:通“好”,即嗜好。
  \item 不知持满:王冰注:“言爱精保神如持盈满之器,不慎而动,则倾竭天真。”
  \item 不时御神:时,善也。御,统摄、治理的意思。不时御神,谓不善于调养精神活动。
  \item 逆于生乐:与生命健康的快乐背道而弛。
  \item 上古圣人之教下也,皆谓之:《新校正》云:“金元起注云:上古圣人之教也,下皆为之。”《太素》、《千金》同。
  \item 虚邪贼风:四时不正之气。《灵枢·九宫八风》云:“从其冲后来为虚风,伤人者也,主杀主害者。”
  \item 恬惔虚无:恬淡,安静之义。虚无,心无杂念。恬惔虚无,即思想安闲清静,没有杂念。
  \item 精神内守:言精神守持于内而不妄耗于外。
  \item 志闲而少欲:闲,《说文》:“阑也,从门中有木。”《广韵》:“防也,御也。”引申为限制、控制。志闲而少欲,即节制情志,而使贪欲减少。
  \item 形劳而不倦:张介宾注:“形劳而神逸,何倦之有?”
  \item 任其服:任,随便。服,服装。马莳注:“有所服,则任用之而不求其华。”
  \item 高下不相慕:指社会地位尊卑贵贱不相倾慕,而安于本位。
  \item 樸:原指未经雕琢的木材,此引申为质朴敦厚。
  \item 嗜欲不能劳其目:言嗜好欲求之物,不能劳其视听。这里的目,相对于下文“心”,而泛指感官知觉。
  \item 不惧于物:不为外物所惊扰。
  \item 德全不危:德,指养生修道有得于心;全面符合养生之道称为德全。不危,不受到衰老死亡的危害。
\end{jiaozhu}

\biaoti{【理论阐释】}

1.关于养生原则和方法

本篇从适应外环境的变化和保持健康的生活方式两个方面确定基本的养生原则。认为生命之气通于天,人与自然是一个整体;人体脏腑、经络及精气神的活动相互协调,也是一个整体,从而构成有序的生命活动及其过程。因此,凡自然环境的异常变化和人类自身的身心活动均可影响其生理活动。当这两种影响超出人体自我调节的限度时,即可破坏有序的生命活动而致病。疾病耗伤人体的脏腑、经络及精气神,就会导致病理性衰老,故《素问·阴阳应象大论》说:“喜怒不节,寒暑过度,生乃不固。”

对于这个养生原则,《灵枢·天年》以先天禀赋不足,后天又“数中风寒”,反复罹患外感之症,推断其必“中寿而尽”,从外邪之害强调适应外界环境异常变化的重要性。故本篇在养生方法中提出“法于明阳”,顺四时寒暑;“虚邪贼风,避之有时”,护养真气。又述及“以酒不浆”、“醉以入房”、“不时御神”、“起居无节”等“以妄为常”、“逆于生乐”则半百而衰,从不良生活方式对人体的负面影响强调内养的重要性。因而在养生方法中提出“食饮有节,起居有常,不妄作劳”,“恬淡虚无”、“精神内守”,调养真气。由此看来,危害人体生命、导致病理性衰老的主要是外感与内伤两类疾病。包括传染病在内的外感病,过去曾经是导致中国人平均寿命低的主要威胁。而今随着社会的进步,生活及科技水平的提高,心脑血管疾病、恶性肿瘤、呼吸系统疾病等内科病已经上升为影响长寿的主要原因。这类疾病没有特效药物,原因复杂,与不良的生活方式有密切关系,通过提倡健康的生活方式预防这些疾病,是提高寿命的主要途径。总之,养生当从内外两个方面“治未病”,防病于先即能预防病理性衰老,顺生命自然盛衰之道,尽终天年。

2.关于精神调养

本篇在各种养生方法中,特别强调精神调养。精神活动是由五脏所产生的,又能反作用于五脏,影响生理活动,故《灵枢·本脏》说:“志意者,所以御精神,收魂魄,适寒温,和喜怒者也。”“志意和,则精神专直,魂魄不散,悔怒不起,五脏不受邪矣。”因而,《素问·灵兰秘典论》说:“心者君主之官,神明出焉……主明则下安,以此养生则寿。”“主不明则十二官危,使道闭塞不通,形乃大伤,以此养生则殃。”《内经》以心为精神之主宰,调心以使“主明”,说明调养精神在养生中的重要地位。本篇从避免过度的情志活动、保持心态的安闲清静、排除杂念和保持精神守持于内而不外耗两个方面,对调养精神法进行概括,为后世所遵循。

3.关于“形与神俱”

本段提出养生的目标是“形与神俱,尽终其天年”,其中“形与神俱”涉及健康标准,反映《内经》形神统一的学术思想。形指包括各种组织在内的有形可见的躯体,神则指无形的生命能力,表现为思维情志、感觉与运动本能及各种基本生理功能等。形因神而活,神能御形;神得形而存,形壮则神旺,形神互存互济,协调统一,故而健康应是形体无病痛之扰,情思无偏造之苦,身心和谐的生理状态。生长壮老死是生命的必然过程,衰老不可避免,自有天年之限,但可以通过养生活动保持身心和谐、躯体与机能和谐,这便是生理性衰老,古称“寿而康”。届时虽然形神皆不及少壮,但仍可以维持相应水平,生活自理,精神不败。欲达此目的,关键在于预防早衰。而环境不良、妄行不节而扰动脏腑、耗伤精气,致使淫邪内生、外邪侵袭,真邪相攻,精气衰败,形神相失,则衰老早至,即病理性衰老。这些理论,对于养生学说和老年病学科的建立有重要指导意义。

\biaoti{【临证指要】}

1.和于术数

和于术数,就是正确掌握各种养生技术。养生术种类繁多,门派各异,方法独特。有以自我修炼为主的,如气功、导引、自我按摩、各种拳术等。有借助外力外物的,如保健针灸、食饵药补等。不论采用哪种方法,都必须注意以下两点:一是各种养生术均有各自的宗旨、特点和针对性,要根据这些养生术的原理、特点和要求,结合本人情况,如身体素质、文化基础、环境与经济条件等,因人、因时、因地选择为宜。以药补为例,阳虚者宜补阳,素体阴虚火旺者非其所宜。二是各种养生术都有特定的方法、要求,掌握其技巧和要领至关重要,一般需要有指导,故有拜师之说。以气功修炼为例,意守丹田要求若有若无、若存若亡,太过则着相,极昜出偏,甚至走火入魔。

2.恬惔虚无、精神内守

恬惔虚无主要是调和情绪,保持心态的安闲清静,排除杂念,防止情绪的剧烈波动,干扰气机的正常运动,维护体内气化活动的良好环境。本篇“志闲而少欲,心安而不惧”,“美其食,任其服,乐其俗,高下不相慕”等,就是古人推崇的作法。本篇后文圣人养生“以恬愉为务,以自得为功”可视为具体实施。这里的“虚无”,其主要精神是排除各种不良情绪,如狂喜、暴怒、悲忧、恐惧等。精神内守与本篇后文“独立守神”及《素问·生气通天论》的“传(抟)精神”同义,多数医家认为是古代进行精神修炼的专门功夫。其要领是入静、意守,神不外驰,通过颐养意志,影响生理机能,充实元气,防病缓老。

\biaoti{【原文】}

\begin{yuanwen}
帝曰:人年老而無子\sb{1} 者,材力\sb{2} 盡耶?將天数\sb{3} 然也?岐伯曰:女子七岁,腎气盛,齒更\sb{4} 髮長;二七而天癸至\sb{5} ,任脈通,太衝脈\sb{6} 盛,月事以时下,故有子;三七,腎氣平均\sb{7} ,故真牙\sb{8} 生而長極;四七,筋骨堅,髮長極,身體盛壯;五七,陽明脈衰,面始焦,髮始堕;六七,三陽脈衰於上\sb{9} ,面皆焦,髮始白;七七,任脈虚,太衝脈衰少,天癸竭,地道不通\sb{10} ,故形壤而無子也。丈夫八岁,腎气實,髮長齒更;二八,腎气盛,天癸至,精氣溢寫\sb{11} ,陰陽和\sb{12} ,故能有子;三八,腎氣平均,筋骨勁强,故真生而長極;四八,筋骨隆盛,肌肉满壯;五八,腎氣衰,髮堕齒槁;六八,陽氣衰竭\sb{13} 于上,而焦,髮鬢颁白\sb{14} ;七八,肝氣衰,筋不能動;八八,天癸竭,精少,腎藏衰,形體皆極\sb{15} ,则齒髮去。腎者主水\sb{16} ,受五藏六府之精而藏之,故五藏盛乃能寫\sb{17} 。 今五藏皆衰,筋骨解堕,天癸盡矣,故髮鬢白,身體重,行步不正,而無子耳。

帝曰:有其年已老而有子者,何也?岐伯曰:此其天壽過度\sb{18} ,氣脈常\sb{19} 通。而腎氣有餘也。此雖有子,男不過盡八八,女不過盡七七,而天地之精氣\sb{20} 皆竭矣。帝曰:夫道者,年皆百數,能有子乎?岐伯曰:夫道者,能却老而全形,身年雖壽,能生子也。
\end{yuanwen}

\biaoti{【校注】}

\begin{jiaozhu}
  \item 无子:丧失生育能力。与下文“有子”相反,有子即具备了生育能力。
  \item 材力:此指精力。
  \item 天数:即天年,自然所赋之寿数。
  \item 齿更:更,更换。人到七八岁,乳牙陆续脱落,被恒齿代替,谓之齿更。
  \item 天癸至:天,先天。癸,十干之一,配五行属水,此指癸水。至,成熟之意。天癸是精水一类的物质,源于先天,藏于肾,乃男女生殖机能盛衰的基础。
  \item 太冲脉:即冲脉。
  \item 平均:充足的意思。
  \item 真牙:真,通“巔”。真牙,尽头齿,即智齿。
  \item 三阳脉衰于上:三阳脉,指太阳、阳明、少阳脉,皆循行于面部。女子六七,面部荣华颓落,故云衰于上。
  \item 地道不通:地道,即通行月经之道。地道不通,指月经停止来潮。
  \item 精气溢泻:溢,盈满。肾气充实,生殖之精盈满自泻。
  \item 阴阳和:指男女两性交合。一说指男子阴阳气血调和。
  \item 竭:《甲乙经》无此字,可从。
  \item 颁白:颁,通“斑”。颁白,即黑白相杂。
  \item 天癸竭,精少,肾脏衰,形体皆极:此十二字原在“七八肝气衰筋不能动”句下,今据丹波元坚《素问绍识》之说移于此。形体皆极,指身体各部分均已衰竭。
  \item 肾者主水:此指肾藏精的功能。
  \item 五脏盛乃能泻:五脏精气盛满,乃泻藏于肾。一说五脏盛,肾才能泻精。
  \item 天寿过度:天寿,先天寿数。过度,超过常人限度。
  \item 常:义同“尚”,仍然的意思。
  \item 天地之精气:男女之天癸。
\end{jiaozhu}

\biaoti{【理论阐释】}

1.肾主生殖与生理发育

本篇以人年老丧失生育能力为题,阐述人的生殖功能盛衰过程,其主导因素在于肾气自然盛衰的规律。先天之精由父母遗传而来,藏于肾,精化为气,是为先天精气,即本篇之肾气。先天之精生天癸,人之肾气发育充盛,则天癸成熟,男子精液溢泻,女子月经来潮,并具有生育能力;肾气发育至极,便由盛转衰,生育能力也渐减弱:及至肾气衰至一定限度,天癸便趋衰竭,于是女子月经闭止,男子精液稀少,而丧失生育能力。同时,随着肾气盛衰,形体变化也展现出同步盛衰过程,主要表现在齿、发、筋骨的生长衰老,面部的荣祜等方面。生育能力成熟,生理发育也见身体盛壮;生育能力衰退,身体衰老征象也明显呈现出来。这是因为人的生理盛衰发育亦本原于先天肾气,其机理有二:一是先天之精发育为人体脏腑经络组织器官,如《灵枢·经脉》:“人始生,先成精,精成而脑髓生。骨为干,脉为营,筋为刚,肉为墙,皮肤坚而毛发长。”二是作为人体精气之本源受后天培育充养形体,如《灵枢·刺节真邪论》:“真气者,所受于天,与谷气并而充身者也。”如此,人之生理发育与生殖机能盛衰均受制于先天肾气,故经文述及男女二七、二八至七七、八八在生殖、生理发育由盛转衰后,以“肾者主水”作结,姚止庵注云:“男女之壮也,并始于肾气之盛实;其后(当为“弱”字)也,亦由于肾气之衰微。人之盛衰,皆本原于肾,此故总以肾结之。”这就为后世关于肾主生殖、肾主生长衰老,并称肾为先天之本的理论奠定了基础,也为中医学从肾气衰竭探讨衰老原理,从生殖功能状况判断衰老进度以及节欲保精防衰缓老的方法提供了重要依据。

2.关于天癸

本篇提出天癸这一概念,作为生殖机能盛衰的决定因素,为中医学性生殖生理、病理及其疾病诊治奠定了理论基础。

(1)天癸的生理:一是制约人体生殖机能的成熟与衰竭,体现在月经的来潮与闭绝、精气的溢泻与稀少等,二是决定性机能的强弱。孩提能悲能喜,能怒能思而绝无欲念。其有情窦早开者,亦在肾气将盛,天癸将至之年。可见肾气未盛,癸水未足,则不生欲念。如肾气衰,癸水竭,则欲念自除。故男女二七、二八天癸至才能“阴阳和”,天癸是性欲形成的重要基础。三是促进第二性征的发育。天癸通过冲任,下至阴器,上荣口唇。女于泄血不荣口唇,故有月经而不生须;男子则有须而无月经。《灵枢·五音五味》举出宦官去其宗筋(阴器,包括睾丸),伤冲脉,竭天癸,故不生须。而天宦(天阉)则是“天之不足”,即天癸异常,冲任不盛,唇口不荣的缘故。

(2)天癸与冲任的关系:肾属水藏精,化生天癸;冲为血海,任主胞胎,任脉通,太冲脉盛,不仅月经以时下,亦关乎男子精气的疏泻,因而冲任通达、充盛的前提是天癸成熟,故而天癸、肾、冲任三者,肾主天癸的产生与成熟,冲任司天癸的通行,与生殖器相连,三者协同作用,维持人的性生殖机能。

\biaoti{【临证指要】}

\xiaobt{冲任与妇科月经、胎孕病证}

本篇“(女子)二七而天癸至,任脉通,太冲脉盛,月事以时下,故有子。”“七七任脉虚,太冲脉衰少,天癸竭,地道不通,故形坏而无子也。”提出冲脉、任脉与女子月经、胎孕关系之说,主要内容是:①冲任充盛、通畅则月经按时来潮,具备孕育胎儿的生殖能力;冲任虚衰则月经闭止,丧失孕育胎儿的生殖能力。②冲任盛衰、通闭,又受天癸成熟、衰竭的制约,而天癸之源在肾藏的先天精气,如此则形成“肾气——天癸——冲任——月经、胎孕”的性生殖轴。后世据此论女子月经、胎孕生理,如陈自明《妇人大全良方·调经门》说:“冲为血海,任主胞胎,肾气全盛,二脉流通,经月渐盈,应时而下。”张志聪《黄帝内经素问集注》说:“女子之天癸,溢于冲任,充肤热肉,为经水下行而妊子也。”皆论月经形成、受孕养胎是冲任的功能,因而冲任的盛衰、通闭、寒热是月经、胎孕病证的机理所在,临床上癥瘕经闭、痛经、崩漏、不孕、流产等病证多从冲任论治,如《医宗金鉴·妇科心法要诀》说:“女子不孕之故,由伤其冲任也……若为三因之邪伤其冲任之脉,则有月经不调、赤白带下、经漏等病生焉。或因宿血积子胞中,新血不能成孕;或因胞寒胞热,不能摄精成孕;或因体盛痰多,脂膜壅塞胞中而不孕。皆当细审其因,按证调治,自能有子也。”张锡纯《医学衷中参西录》制理冲汤、理冲丸,治妇女经闭不行,或产后恶露不尽、结为癥瘕;安冲汤治妇女经水行时多而且久,过期不止或不时漏下;固冲汤治妇女血崩;温冲汤治血海虚寒不育,均本于此。

\biaoti{【原文】}

\begin{yuanwen}
黃帝曰:余聞上古有真人\sb{1} 者,提挈天地,把握陰陽\sb{2} ,呼吸精氣\sb{3} ,獨立守神\sb{4} ,肌肉若一\sb{5} ,故能寿敝天地\sb{6} ,無有终時,此其道生\sb{7} 。

中古之時,有至人\sb{8} 者,淳德全道\sb{9} ,和於陰陽,調於四時,去世離俗\sb{10} ,積精全神,游行天地之間,視聽八達之外\sb{11} 。此盖益其壽命而强者也,亦歸於真人。

其次,有聖人者,處天地之和,從八風之理\sb{12} ,適嗜欲於世俗之間,无恚嗔之心\sb{13} ,行不欲離於世,被服章\sb{14} ,举不欲觀於俗\sb{15} ,外不勞形於事,內無思想之患,以恬愉為務,以自得為功\sb{16} ,形體不敝,精神不散,赤可以以百數。

其次,有賢人者,法則天地,象似日月\sb{17} ,辯列星辰\sb{18} ,逆從陰陽\sb{19} ,分别四時,將從上古合同於道\sb{20} ,亦可使益壽而有極時。
\end{yuanwen}

\biaoti{【校注】}

\begin{jiaozhu}
  \item 真人:《淮南子·精神训》:“真人者,性合于道也……此精神之所以能登假于道也。”言修其得道,精神返于至真之人。
  \item 提挈天地,把握阴阳:提挈、把握同义。言把握自然界阴阳变化的规侓。
  \item 呼吸精气:即气功中的吐纳调息。
  \item 独立守神:独立,自我主宰、控制。守神,即精神内守而不外驰。
  \item 肌肉若一:肌肤保持青春而不衰老。一说生气功中的“调身”之法,使全身筋骨肌肉保持高度协调统一。
  \item 寿敝天地:与天地同寿。王冰注:“敝,尽也。”
  \item 道生:吴昆注:“以其道成,故能长生。”
  \item 至人:《庄子·天下篇》:“不离于真,谓之至人。”故下文云;“亦归于真人。”
  \item 淳德全道:具有淳朴敦厚品德,全面掌握养生之道。
  \item 去世离俗:脱离世俗,专事修道。
  \item 游行天地之间,视听八达之外:达,其余刻本均作“远”字。言至人“去世离俗”,广涉天地山水,视听达于至远之处。
  \item 处天地之和,从八风之理:八风,即四正四隅八方之风,各有节气相应而至。王冰注:“所以处天地之淳和,顺八风之正理者,欲其养正,避彼虚邪。”
  \item 恚嗔之心:恼怒、怨恨的情绪。
  \item 被服章:丹波元简:“此三字,新校正为衍文,当然耳。”
  \item 举不欲观于俗:观,示、炫耀。言举止行为不在世俗炫耀。
  \item 以自得为功:马莳:“以悠然自得为己功。”
  \item 象似日月:仿效日月昼夜盈亏之道。
  \item 辩列星辰:辩,通“辨”。列,位次。吴崑注:“推步天象也。”古有据天象变化而行养生之法。
  \item 逆从阴阳:逆从,偏义复词,取“从”义。言顺从阴阳变化规律。
  \item 将从上古合同于道:将,随也。言追随上古之人,使自己的行为符合养生之道。
\end{jiaozhu}

\biaoti{【理论阐释】}

\xiaobt{《内经》养生思想与道家}

以老、庄为代表人物的道家,是先秦诸子中的重要学术派别。《史记》称之为“道德家”,《汉书》始称“道家”。本节“德”、“道”源于《老子》,“真人”“至人”首见于《庄子》,精气、守神、积精全神等均系道家习用术语,其中“美其食,任其服,乐其俗”亦极似《老子·八十章》之文,可见《内经》与道家的渊源关系。在养生思想方面,《内经》亦受道家影响极深,再加上道家人物王冰的编次注释,更显得十分突出。①在养生观上,本于道家的天道观,养生法道,道法自然,奉养天真,返本还原。②在养生原则上,道法清静,頤养天真之气。③在养生方法上,重视“术数”。《庄子》有“心斋”“坐忘”“吹昫呼吸,吐故纳新,熊经鸟伸”等养生方法的记载,本“呼吸精气,独立守神,肌肉若一,’以及“象似日月,辩别星辰”,亦似道家讲的特殊养生技术。当然,本篇所述“术数”限于养生之术,应是《内经》借用术语。《汉书·艺文志》列出天文、历谱、五行、蓍龟、杂占、形法六类术数,其内涵外延与《内经》所述自然有别。总之《内经》养生思想受道家影响是广泛而深刻的,对于今天强调自身保健、建立健康的生活方式,以协调机体内外环境,增强自身抗病能力与调节能力,发扬中华民族的养生传统是有现实意义的。

\section{素問·四氣調神大論(節選)}%第二節

\biaoti{【原文】}

\begin{yuanwen}
春三月,此謂發陳\sb{1} ,天地俱生,萬物以榮。夜臥早起,廣步於庭,被髮缓形\sb{2} ,以使志生;生而勿殺,予而勿奪,賞而勿罰\sb{3} 。此春气之應,養生之道也。逆之則傷肝,夏為寒變\sb{4} ,奉長者少。

夏三月,此謂蕃秀\sb{5} 。天地氣交,萬物華實。夜臥早起,無厭於日。使志無怒,使華英成秀\sb{6} 。使氣得泄,若所愛在外\sb{7} 。此夏气之應,養長之道也。逆之則傷心,秋為痎瘧\sb{8} ,奉收者少,冬至重病\sb{9} 。

秋三月,此謂容平\sb{10} 。天氣以急,地氣以明\sb{11} 。早臥早起,與雞俱輿。使志安寧,以緩秋刑\sb{12} ,收敛神气,使秋氣平,無外其志,使肺氣清\sb{13} 。此秋气之應,養收之道也。逆之則傷肺,冬為飧泄\sb{14} ,奉藏者少。

冬三月,此謂閉藏\sb{15} 。水冰地坼,無擾乎陽\sb{16} 。早臥晚起,必待日光。使志若伏若匿,若有私意,若已有得,去寒就溫,無泄皮膚,使氣亟奪\sb{17} 。此冬气之應,養藏之道也。逆之則傷肾,春為痿厥\sb{18} ,奉生者少。
\end{yuanwen}

\biaoti{【校注】}

\begin{jiaozhu}
  \item 发陈:推陈出新。张介宾注:“发,启也。阵,故也。春阳上升,发育庶物,启故从新,故曰发陈。”
  \item 被发缓形:被,通披。被发,披开束发。缓形,松缓衣带,让形体舒缓。马莳注:“被犮而无所束,缓形而无所拘,使志意于此而发生。”
  \item 生而勿杀,予而勿夺,赏而勿罚:予同“与”。马莳注:“其待物也,当生则生之而勿之杀,当与则与之而勿之夺,当赏则赏之而勿之罚。凡此者,盖以春时主生,皆以应夫春气而尽养生之道也。”
  \item 逆之则伤肝,夏为寒变:肝主春令应生发之气,逆春阳生发之气即伤肝。以下伤心、伤肺、伤肾,均通此理。寒变,喻昌《医门法律》:“夏月得病之总名。缘肝木弗荣,不能生其心火,至夏心火当旺反衰……得食则饱闷,遇事则狐疑,下利奔迫,惨然不乐,甚者战慄,如丧神守。”
  \item 蕃秀:蕃,茂也,盛也。秀,华也,美也。马莳注:“阳气已盛,物蕃且秀,故气象谓之蕃秀也。”
  \item 使华英成秀:张介宾注:“华英,神气也。”秀,秀丽,旺盛之意。言使人的神气旺盛饱满。
  \item 使气得泄,若所爱在外:使体内阳气宣发于外,如出汗;精神外向,意气舒展,无令抑郁,以顺应夏气长旺,阳盛于外。
  \item 秋为痎疟:痎疟,疟疾的总称。张介宾注:“心伤则暑气乘之,至秋而金气收敛,暑邪内郁,于是阴欲入而阳欲拒之,故为寒;火欲出而阴束之,故为热。金火相争,故寒热往来而为痎疟。”
  \item 冬至重病:丹波元简注:“据前后文例,四字恐剩文。”可从。
  \item 容平:容,万物之容貌。平,平定。马莳注:“阴气已上,万物之容至此平定,故气象谓之容平。”
  \item 天气以急,地气以明:张介宾注:“风气劲疾曰急,物色清肃曰明。”
  \item 使志安宁,以缓秋刑:秋气肃杀,故称“秋刑”。张介宾注:“阳和日退,阴寒日生,故欲神志安宁,以避肃杀之气。”
  \item 收斂神气,使秋气平,無外其志,使肺氣清;“收斂神气”与“無外其志”同义,神气欲其内敛,勿令外驰也;“秋氣平”与“肺氣清”同义,使肺气如秋之平定而清肃。
  \item 冬为飧泄:张介宾注:“肺属金,王于秋,秋失所养故伤肺,肺伤则肾水失其所生,故当冬令而为肾虚飧泄。”
  \item 闭藏:马莳注;“阳气已伏,万物潜藏,故气象谓之闭藏。”
  \item 水冰地坼,无扰乎阳:坼,裂也。冬季寒冽,水成冰而地冻裂,此地气闭藏,而其阳气固潜于内,不应受扰动。
  \item 无泄皮肤,使气亟夺:无泄皮肤,即勿出汗。亟,频数、屡次。言无令频数汗出,耗散阳气,逆冬藏之道。
  \item 春为痿厥:痿厥,肢体痿软无力、手足逆冷。张介宾注:“肾主水,王于冬,冬失所养,故伤肾。肾伤则肝木失其所生,肝主筋,故当春令而筋病为痿。阳欲藏,故冬不能藏则阳虚为厥。”
\end{jiaozhu}

\biaoti{【理论阐释】}

\xiaobt{四时气象与养生}

本篇先论四时气象,而后述摄养之法。四时气象本于天,摄养之法用于人,体现了天人合一、人法自然的养生思想。第一,春之发陈,夏之蕃秀,秋之容平,冬之闭藏,阐发四时生长收藏的气象特点。所谓象,即万物形态容貌征象,现于外,是有形的;气则藏于内,是象之所以如此表征的内在依据,是无形的。两者相辅相成,生动的表述了四时特点。发陈,表征春阳生发、推陈出新的特点,故说“天地俱生,万物以荣”;蕃秀,表征夏季阳气长旺,万物茂盛的特点,故说“天地气交,万物华实”;容平,表征秋季阳气开始收效,万物容貌清肃平定的特点,故说“天气以急,地气以明”;闭藏,表征冬季阳气沉潜,万物蛰伏自固的特点,故说“水冰地坼,无扰乎阳”。这些描述,对于确立顺应四时的系统养生方法具有重要指导作用。第二,以四时气象作类比指导养生,提出“四气调神”之法。本篇从形体起居和精神活动两个方面讨论四时养生的具体方法,均体现四时气象的基本特性,如春季起居应早起散步、舒缓形体,意念情志宜促生、多赏予,慎夺取、戒杀伐,以适应春气除陈布新、生发疏达、外向宣散的特点;冬季起居应晚起避寒、忌妄动汗出,意念情志宜潜伏忌张扬,以适应冬气潜藏不露、内向蛰伏的特点。以这种思想为指导,后世除形体起居、精神调养而外,还发明了多种顺应四时特点的养生方法。第三,本篇所述四时气象,除指导顺时养生之外,也是理解《内经》五脏概念中四时内涵的重要经文。这是由于《内经》的五脏“以四时之法成”(《素问·宝命全形论》),故王冰注《素问·五脏生成篇》“五脏之象可以类推”说,“象,谓气象也。言五脏虽隐而不见,然其气象性用,犹可以物类推之。”本篇亦提出,逆养生之道则伤肝,逆养长之道则伤心,逆养收之道则伤肺,逆养藏之道则伤肾。这里的肝心肺肾即法于四时之五脏,“四气调神”就是顺四时气象调养五脏之气。

\biaoti{【临证指要】}

1.顺应四时特点养生

本篇顺应四时特点养生方法,主要论及形体起居和精神情志,具有示范作用,后世医家和养生家据此加以发挥,扩大应用范围·提出以四时法则指导的饮食宜忌、针灸药饵、导引武术等养生方法,如元代邱处机《摄生消息论》、明代冷谦《修龄要旨》、明代高濂《遵生八笺》等,专有四季调摄法。举春季调摄为例,《摄生消息论·春季摄生消息》说:“当春之时,食味宜减酸益甘以养脾气。”“春日融和,当眺园林亭阁,虚敞之处,用摅滞怀,以畅生气。不可兀坐,以生抑郁。饭酒不可过多,米面团饼,不可多食,致伤脾胃,难以消化。老人切不可以饥腹多食,以快一时之口,致生不测。”《修龄要旨·四时调摄》说:“春三月,此谓发陈,夜卧早起,节情欲以葆生生之气,少饮酒以防逆上之火。”“用嘘字导引……此能去肝家积聚风邪毒气,不令病作。”

2.关于冬三月“早卧晚起必待日光”

冬季阳气沉潜,天气严寒,万物蛰伏,人之精气亦应内潜闭藏,因此起居作息要“早卧晚起”增加居室时间,减少冒寒机会,去寒就温,无扰乎阳,以适应之,这种精神尤适用于年老、幼弱、久病等生机薄弱之人,这是符合实际情況的。“必待日光”,言其晚也,而且也有一定道理。一是冬日昼短夜长,阳衰阴盛,必借助日光以养阳气,消散阴霾,对于气阳不足之人有一定意义;二是寒凝气敛,污浊空气难以消散,必待日光温热方能流通,对于呼吸道疾病的患者尤为重要。

\biaoti{【原文】}

\begin{yuanwen}
逆春气則少陽不生,肝氣內變\sb{1} ;逆夏气則太陽不長,心氣內洞\sb{2} ;逆秋气則太陰不收,肺氣焦滿\sb{3} ;逆冬气則少陰不藏,腎氣獨沉\sb{4} 。

夫四時陰陽者,萬物之根本也。所以聖人春夏養陽,秋冬養陰\sb{5} ,以從其根,故與萬物沉浮於生長之門\sb{6} 。 逆其根,則伐其本,壤其真矣。故陰陽四時者,萬物之終始也,死生之本也。逆之則災害生,從之則苛疾不起,是謂得道\sb{7} 。道者聖人行之,愚人佩\sb{8} 之。從陰陽則生,逆之則死,從之則治,逆之則亂,反順為逆,是謂內格\sb{9} 。

是故聖人不治已病治未病,不治已亂治未亂,此之謂也。夫病已成而後葯之,亂已成而後治之,譬猶渴而穿井,鬭而鑄錐,不亦晚乎?
\end{yuanwen}

\biaoti{【校注】}

\begin{jiaozhu}
  \item 肝气内变:变,变动,即病变。张介宾注:“逆春气则少阳之令不能生发,肝气被郁,内变为病。”
  \item 心气内洞:洞,空虚、不足。张介宾注:“逆夏气则太阳之令不长,而心虚内洞,诸阳之病生矣。”
  \item 肺气焦满:张介宾注:“逆秋气则太阴之令不收,而肺热叶焦为胀满也。”
  \item 肾气独沉:张介宾注:“逆冬气则少阴之令不藏,而肾气独沉。藏者藏于中,沉者沉于下,肾气不蓄藏,则注泄沉寒等病生矣。”
  \item 春夏养阳,秋冬养阴:春夏养人之生气、长气,秋冬养人之收气、藏气。高世栻注:“圣人春夏养阳,使少阳之气生,太阳之气长;秋冬养阴,使太阴之气收,少阴之气藏。”
  \item 与万物沉浮于生长之门:沉浮,随波逐浪之意。门,关键,在此指四时阴阳,如《素问集注》朱济公注:“阴阳出入,故谓之门。”与万物沉浮于生长之门,即与万物一样,生存于四时阴阳变化之中。
  \item 得道:得,此处作“合”解。得道,符合养生法则。
  \item 佩:通“背”字,即违背之意。
  \item 内格:体内脏腑气血的活动与自然界阴阳消长变化相格拒。王冰注:“内性格拒于天道也。”
\end{jiaozhu}

\biaoti{【理论阐释】}

1.关于“四时阴阳者,万物之根本也”

“四时阴阳者,万物之根本也”句,是贯穿全篇的中心思想,它既是《内经》“天人相应”整体观的理论基础,又是中医养生学说得以建立的学术支柱。

本篇正是在这一学术基础上,建立了“四气调神”的养生学说。其中的“神”应是与“四时之法”相应的五脏之神,亦即五脏之气。通过顺应四时特点调节形体活动、起居作息及神情意志,调养五脏之气,使之与自然界的阴阳有序消长、万物生长收藏相统一,达到“从其根”,补养真气,增强体质、预防疾病的目的,亦即本篇所说“此春气(夏气、秋气、冬气)之应,养生(养长、养收、养藏)之道”;若逆之就会“伐其本,坏其真”,削弱或损伤相应脏气,分别发生肝、心,肺、肾病变。

2.关于“春夏养阳,秋冬养阴”

“春夏养阳,秋冬养阴”是本篇提出的“四气调神”养生原则,指春夏顺应生长之气以养阳,秋冬顺应收藏之气以养阴,在注家中马莳、高世栻持这种见解,与经意相合。然而除此之外,历代注家尚有三种不同的认识:一是以王冰为代表的阴阳互制论,认为春夏阳盛,宜食寒凉抑制亢阳,“全阴则阳气不极”;秋冬阴盛,宜食温热抑制盛阴,“全阳则阴气不穷”。二是以张介宾为代表的阴阳互根论,认为春夏养阳,以为秋冬阴之基,故春夏每因风凉生冷,伤其阳气而患疟泄等病;秋冬养阴,以为春夏阳之基,故秋冬每因纵欲过热,伤其阴而患火证。三以张志聪为代表的内外阴阳虚盛论,认为春夏阳盛于外而虚于内,故有“夏月伏阴”之病,因而春夏宜养其内虚之阳;秋冬阴盛于外而虚于内,故有“冬月伏阳”之病,因而秋冬宜养其内虚之阴。

以上三种解释,结合本篇内容分析,显然难合经旨,但却也不无道理,且在生活及医疗实践活动中有所验证,并具有一定的理论价值和临床意义。①王冰从阴阳互制讲理,在自然界春夏之阳盛继之以秋冬之阴,秋冬之阴盛继之以春夏之阳,以防阴阳之极,人亦同此理,春夏宜食寒食凉、秋冬宜食温食热,以防体内阴阳盛衰过度而发病。此不仅阐发了养生学说中阴阳互制的理论与原则,而且符合生活实际,特别是对于阴阳虚实偏颇的体质有切实的指导意义。王冰此注扩大了“春夏养阳,秋冬养阴”养生原则的应用范围。②张介宾从阴阳互根的理论推之,春夏之阳为秋冬阴盛之基,故衣食行为不可过凉,以免伤其阳,而至秋冬之时,人体不能应时而阴盛;反之亦然。此说在理论上将养生原则引申,在应用上思路一转,便可意识到某些秋冬发作之病源于春夏阳气失养,若于春夏之时壮阳为治,可望取得好的疗效;反之某些春夏发作之病,秋冬养阴疗效甚佳。因而临床上有“冬病夏治”“夏病冬治”之法。③张志聪从内外阴阳虚盛的理论推之,为“夏月伏阴”用温热之治、“冬月伏阳”用寒凉之治提供理论依据。二张之注引导“春夏养阳,秋冬养阴”从养生原则演变为治疗原则,也是扩大了应用范围,丰富了中医治疗学内容。

3.关于“治未病”

本篇以“渴而穿井”“斗而铸锥”为喻,将未病先防的防病思想提高到战略高度,成为养生学说建立的理论基础。

中国古代早就在理性上重视预防的价值与意义,《周易·既济》象曰:“君于以思患而豫防之。”豫通“预”,凡事豫则立、不豫则废,预防是主动处理事物达到理想效果的最佳途径,故不治已乱治未乱。将此观念应用于医学,首先是未病先防,本篇称之为“治未病”,即通过各种方法和措施,增强体质,避邪抗邪,预防疾病,防衰缓老。本篇所论“四气调神”,就是在未病之前,顺四时调养五脏之气,使外不受邪气之侵,内能充实和畅真元,是养生大法之一。

“治未病”除未病先防含义之外,在《内经》还有既病防变的内涵。如《素问·刺热论》:“肝热病者,左颊先赤……病虽未发,见赤色者刺之,名曰治未病也。”要求医者在掌握疾病传变规律的基础上,密切注意病情,洞察其演变趋势,抓时机,早遏某路,化解病邪,争取病变的良好转机。如《难经·七十七难》:“所谓治未病者,见肝之病,则知肝当传之与脾,故先实其脾气,无令得受肝之邪,故曰治未病焉。”《内经》将此作为衡量医生医疗水平的尺度,即如《素问·阴阳应象大论》所说:“邪风之至,疾如风雨,故善治者治皮毛,其次治肌肤,其次治筋脉,其次治六腑,其次治五脏。治五脏者,半死半生也。”并规定“上工冶未病”。后世将已病防变“治未病”作为治疗学上的基本原则,普遍应用于临床,如《叶香岩外感温热篇》:“若斑出热不解者,胃津亡也,主以甘寒……或其人肾水素亏,虽未及下焦,先自彷徨矣,必验之于舌,如甘寒之中加入咸寒,务在先安未受邪之地,恐其陷入易易耳。”

\biaoti{【临证指要】}

\xiaobt{“春夏养阳,秋冬养阴”养生原则的临床应用}

1.指导养生防病

(1)顺应四时阴阳特点保健:除本篇形体活动、起居作息、精神调摄外,后世又补充饮食、劳逸、服饵、气功等方面的方法和内容。举夏季为例,夏日“天地气交,万物华实”,人体要顺应阳盛长养之气的特点,使心气长旺,提高抗病能力。形体锻炼,坚持室外活动,日光浴。情志调节,积极进取,勿生懈惰厌倦之心,以使阳气宜散于外。起居调节,晚睡早起,午睡不可太久,注意避暑尤忌过于趋凉。饮食调养,宜消淡忌油腻,多食营养丰富的蔬菜、瓜果之类,但不宜过凉。药饵,宜频服淡盐水、绿豆粥、鲜芝麻叶开水冲泡代茶之类防暑汤饮。

(2)结合体质特点施养防病:当体质阴阳之偏与季节阴阳之偏性质相同时,防止机体阴阳之偏极;当体质阴阳之偏与季节阴阳之偏性质相反时,乘季节阴阳之偏势,纠正机体阴阳之偏颇,使之趋于平秘。如阴盛或阳虚阴盛之人,秋冬养阴,目的是“全阳则阴气不穷”,故宜适当进行形体锻炼,情志宁静而欢愉,忌孤独凄凉;注意衣着、居室保暖;宜温热食物,如牛羊鸡等畜禽之肉,宜进温热助阳益气补药,少食瓜果,忌寒凉滋腻食物、药饵;亦可行秋冬导引之法。春夏属阳,在防止阳邪感伤前提下,宜食用温平益气和阳之品,如龙眼肉、大枣等物,并可适当进补温热补阳益气药,慎食生冷、油腻,忌苦寒药物攻伐,亦应禁当风凉爽;还可针灸保健穴,如气海、关元、命门、足三里等。

2.指导临床,审时施治

(1)顺时用药:人体五脏之气随四时递迁而升降沉浮,若时至而脏气不应,如春时阳气升发不足,是为肝虚,出现眩晕、体倦乏力等,其余仿此。李时珍《本草纲目·四时用药例》述薄荷、荆芥之类辛温,顺春升之气;香藿、生姜之类辛热,顺夏浮之气;人参、二术、黄柏之类甘苦辛温,顺长夏化成之气;芍药、乌梅之类酸温,顺秋降之气;黄芩、知母之类苦寒,顺冬沉之气;并说此即“所谓顺时气而养天和也”。临床可据证随时用药,或入治疗方剂中。

(2)参时调阴阳:一是春夏用寒用凉,全阴以养阳;秋冬用温用热,全阳以养阴。如春夏素体阳盛阴弱着,用药宜偏寒凉而远温热,以防热伤津液;阴精不足者,当滋阴清热,慎用桂附之类。秋冬素体阴盛阳弱者,用药宜偏温热而远寒凉,以防寒伤阳气;阳气已衰者,当益气助阳,而慎用石膏、三黄之类。故古有“冬不用白虎,夏不用青龙”之说。二是春夏用温用热,内养其阳;秋冬用凉用寒,内养其阴。如春夏食凉食冷腹痛泄泻、身凉肢厥即“夏月伏阴”者,当辛热散寒回阳;秋冬厚衣食热过度,阳热内盛而烦渴咳喘即“冬月伏阳”者,必不免苦寒泄火。

(3)冬病夏治,夏病冬治:对于阳气虚弱,人冬辄发的“冬病”如咳喘、痰饮等,在夏天给予养阳药物,如金匮肾气丸、参蛤散、补中益气汤、苓桂术甘汤等内服,或以温热药制成饼灸、制成硬膏贴敷,借助时令长旺之阳气,培植人体真阳,用以弥补冬时阳气之不足,抗御阴寒之邪,可以预防或减轻病症发作。对于阴精亏损、入夏辄发的“夏病”,如疰夏等,在冬天给予养阴药物,如六味地黄丸、生脉饮等,借助时令闭藏之机,培植人体真阴,用以弥补夏时阴精不足,而济盛夏之阳,可以预防或减轻病症发作。

\section{靈樞·天年}%第三節

\biaoti{【原文】}

\begin{yuanwen}
黃帝間於岐伯曰:願聞人之始生,何氣築為基\sb{1} ?何立而為楯\sb{2} ?何失而死?何得而生?岐伯曰:以母為基,以父為楯\sb{3} ;失神者死,得神者生也。

黃帝曰:何者為神\sb{4} ?岐伯曰:血氣已和,榮衛已通,五藏已成,神氣舍心\sb{5} ,魂魄畢具,乃成爲人。
\end{yuanwen}

\biaoti{【校注】}

\begin{jiaozhu}
  \item 基:基础,基质。
  \item 楯(shǔn):《说文》:“阑槛也。”即栏杆,在这里引申为遮蔽和捍卫之意。
  \item 以母为基,以父为楯:人体胚胎发生,是以母之阴血为基础,以父精所化阳气为护卫,阴阳交感,精气相结合而成。
  \item 何者为神:神,即神机,此指生命力。何者为神,问神何以形成。
  \item 神气舍心:此指心得以藏神,心神生成并展开活动。
\end{jiaozhu}

\biaoti{【理论阐释】}

1.《内经》关于人体胚胎生成的理论

《内经》以阴阳学说为指导,探索人类个体生成的机理与过程,提出人体胚胎由父母精气相结合,阴以为基,阳以为用,阴阳交感,生发出新生命的胚胎发生学说,是中医胎孕理论的基础之一。其意义有二:①生命之来源既是父母之精,则父母之精的强弱及和谐与否,是形成后代个体先天禀赋的基础,如张介宾《类经·疾病类六十二》,说:“夫禀赋为胎元之本,精气之受于父母者也。”“凡少年之子多有羸弱者,欲勤而精薄也;老年之子反多强壮者,欲少而精全也。多饮者子多不育,盖以酒乱精,则精半非真而湿热盛也。”强调父母精血健全强壮对于后代的重要性。又,《礼记》记载有“取妻不取同姓”(《曲礼上》)的规定,其原因是“同姓不婚,恶不殖也。”(《国语·晋语四》)、“男女同姓,其生不蕃。”(《左传·僖公二十三年》),从医学的角度说就是父母生殖精气阴阳和谐与失调之理,从而阐发了古代关于反对近亲结婚、提倡适龄婚育和寡欲优生的思想。②禀受于父母的先天之精与生殖之精皆藏于肾,因而肾在先天禀赋中占有重要地位。这就为后世从肾的保养与培补以强身防衰、治疗小儿先天发育不良,奠定了理论基础。

2.关于胎儿护养与胎教

从胚胎发生至分娩,是胎儿发育的过程,其脏腑肢体相继成长,神气依次具备,全靠母体气血滋养,母体情况如何,都必然会影响胎儿发育,也是后代先天禀赋形成的基础。诸凡饮食起居、劳逸房事、情志感发,有所失调;外邪、跌仆,以及针药失当,均能伤胎,故不能不慎养。此外,本篇所论胎儿神情气质的发育,是在气血营卫、脏腑经脉发育基础上,逐步成就的,因而古代有在胎孕期施教之说,认为外界良性的声色等形式的刺激,可以通过母体影响胎儿神情气质,并称之为“胎教”,对后代个体先天禀赋的形成也有一定影响,具有优生学意义,值得进一步研究。

\biaoti{【原文】}

\begin{yuanwen}
黃帝曰:人之壽夭各不同,或夭壽,或卒死,或病久,願聞其道。岐伯曰:五藏堅固\sb{1} ,血脈和調,肌肉解利\sb{2} ,皮膚緻密,營衛之行,不失其常\sb{3} ,呼吸微徐\sb{4} ,氣以度行\sb{5} ,六府化榖,津液布揚,各如其常,故能長久。

黃帝曰:人之壽百歲而死何以致之?岐伯曰:使道隧以長\sb{6} ,基牆髙以方\sb{7} ,通調營衛\sb{8} ,三部三里起\sb{9} ,骨髙肉滿,百歲乃得終。
\end{yuanwen}

\biaoti{【校注】}

\begin{jiaozhu}
  \item 五脏坚固:指五脏发育良好,功能健全,阴阳协调。
  \item 肌肉解利:指肌肉分理之间滑润,气道通行无碍。
  \item 營卫之行,不失其常:常,正常规律。杨上善注:“谓营卫气,一日一夜,各循其道,行五十周,营卫其身而无错失。”
  \item 呼吸微徐:指气息调匀,不粗不疾。
  \item 气以度行:指气血运行与呼吸之间的比例合乎常度。
  \item 使道隧以长:使道,杨上善:“鼻孔使气之道”,即鼻孔。一说指人中沟。隧以长,即深而长。
  \item 基墙高以方:基墙,张介宾注:“指面部而言。骨骼为基,蕃蔽为墙。”高以方,丰满方大。
  \item 通调营卫:指面色红润,光泽有神。
  \item 三部三里起:三部与三里同义,指面部上、中、下三部分,分别以额角、鼻准、下颌为标志。起,高起而不平陷。
\end{jiaozhu}

\biaoti{【理论阐释】}

\xiaobt{关于长寿的条件和特征}

寿命长短是所有人关心的问题,也是古今中外医学家们研究的永久课题。本篇命名“天年”,即说明要探讨这个课题。就《内经》全书而论,人的寿命决定于两个方面,一是物种遗传因素。高世栻《素问直解》注《六微旨大论》“化有小大,期有近远”说:“生化有小大,死期有远近,如朝菌晦朔,蝼蛄春秋,此化之小、期之近者也;蓂灵大椿,千百岁为春,千百岁为秋,此化之大者、期之远者也。”人类作为自然界的物种之一,自有其物种寿限,即自然寿命,《内经》认为是百岁,王冰引《尚书》为120岁。现代研究,有从人类成熟期、细胞分裂次数等不同方法计算,约为120$\sim$150岁。而个体遗传因素,与其祖辈寿数有关,具有家族倾向;同时与父母生殖精气的强弱、和谐与否以及胚胎生成后的孕养有关,共同构成个体的先天禀赋。这类因素使人的寿数在出生之时即定,张介宾谓之“天定”,徐大椿谓之“定分”。

先天禀赋的强弱,也就是对人的寿夭预测,可以通过两个方面观察。一是观察人体各种生理机能是否健全。五脏发育良好,则气血得以化生,精神魂魄旺盛;六腑发育良好,则水谷化为精微,津液润养全身;荣卫气血运行通利和调,循常不乱,则脏腑肢节得养;腠理致密,则不受邪侵扰;呼吸微徐,则是脏气安定、神气内守而不外泄之征,是肺主治节良好的表现。二是观察头面发育状态。头面部骨肉血脉及五官状态,又是禀赋强弱、厚薄,先天发育是否良好的标志。基墙高以方、三步三里起,说明头面部骨肉丰满;通调营卫,说明面部血脉充盛;使道隧以长,即鼻孔深长,说明清浊之气能和畅吐纳。禀赋强壮、先天发育良好,则后天生命活动有丰厚基础,长寿有基;禀赋薄弱、先天发育不良,则为夭折埋下祸根。这些认识,有古人长期观察之客观依据,与预言人事穷通祸福的“相面术”不能等同对待,其科学性和具体应用当进一步研究。

\biaoti{【原文】}

\begin{yuanwen}
黃帝曰:其氣之盛衰,以至其死,可得聞乎?岐伯曰:人生十歲,五藏始定,血氣已通,其氣在下\sb{1} ,故好走\sb{2} 。二十歲,血氣始盛,肌肉方長,故好趨。三十歲,五藏大定,肌肉堅固,血脈盛滿,故好步。四十歲,五藏六府,十二經脈,皆大盛以平定,腠理始疏,榮華頹落,髮頗斑白\sb{3} ,平盛不搖\sb{4} ,故好坐。五十歲,肝氣始衰,肝葉始薄,膽汁始滅\sb{5} ,目始不明。六十歲,心氣始衰,苦憂悲,血氣懈惰,故好臥。七十歲,脾氣虚,皮膚枯\sb{6} 。 八十歲,肺氣衰,魄離\sb{7} ,故言善悮\sb{8} 。九十歲,腎氣焦\sb{9} ,四藏經脈空虛\sb{10} 。 百歲,五藏皆虛,神氣皆去,形骸獨居而終矣。
\end{yuanwen}

\biaoti{【校注】}

\begin{jiaozhu}
  \item 其气在下:气,指主司人体生长的先天之气,与前文“何者为神”的神同义。此气藏于肾,自下而升,人生十岁,此气始盛,是生长发育的开端,故云“其气在下”。
  \item 好走:《说文》段注:“《释名》曰:徐行曰步,疾行曰趋,疾趋曰走。”好走,形容少儿活泼爱动。
  \item 发颇斑白:颇,《太素》作“鬢”,可从。斑白,黑白相间,俗称花白。
  \item 平盛不搖:平盛,盛到极限。搖,《辞海》:“上升貌。”不摇,不再发育,由盛渐衰也。
  \item 滅:《甲乙经》《太素》均作“减”,可从。
  \item 皮肤枯:《甲乙经》作“皮肤始枯”,下有“故四肢不举”,可参。
  \item 魄离:《甲乙经》作“魂魄离散”,可参。
  \item 悮:通“误”字。
  \item 肾气焦;焦,枯竭的意思。肾气焦,即肾所藏先天精气枯竭。
  \item 四脏经脉空虚:四脏经脉,指肝心肺肾及其经脉。空虚,衰竭。
\end{jiaozhu}

\biaoti{【理论阐释】}

1.关于个体出生后的生命历程及其阶段性

人之生命,本源于先天精气,此精气在人出生后即藏于肾,据《素问·上古天真论》“肾者主水”以及本篇“其气之盛衰,以至其死”的论述,此先天精气有物种遗传的自然盛衰规律,它制约着机体脏腑、经脉、气血的盛衰变化,从而使人的生命活动表现出由幼稚到成熟、由盛壮到衰竭的生长壮老已过程。

对于这个过程,本篇以百岁为期,以10岁为一阶段,分十个阶段论述其各段的表现及生理特点。从出生到十岁,是人体发育之始,生气由下而升,以“好走”概括其生机勃发,活泼爱动的生理、心理特点。20$\sim$30岁,生机旺盛,发育健全,以“好趋”“好步”概括其生理、心理盛壮、成熟而稳重的特点。40岁,脏腑经脉气血盛至极限,盛极转衰,开始出现生气衰退征兆,以“好坐”概括其由盛而衰的生理、心理特点。从50$\sim$90岁,生气衰退逐渐加重,肝心脾肺肾之精气相继由衰至竭,以“好卧”概括其生机颓废的生理、心理特点。及至百岁,五脏精气均告枯竭,生命力败亡而死。本篇对于人类个体生命全过程及其生理、心理特点的描述,不仅具有医学科学的研究价值,而且对于中医儿科、内科的建立以及老年疾病的临床诊治也有重要意义。

2.《内经》关于生命过程阶段性的论述

本段与《素问·上古天真论》“人老而无于者”一段,均论述生命过程及其阶段性,但彼以女士男八为阶段,重在阐发生殖机能盛衰规律,而且所述自一七、一八至七七、八八,只是生命的部分过程;此则以十为阶段,重在阐述人体生理机能变化规律,是生命的全过程。此外,《素问·阴阳应象大论》“七损八益”调阴阳一段,也论及生命过程的阶段性,然该篇仅述及40、50、60等三个阶段,且重在阐述衰老进程。三篇所论各有侧重,可以互相阐发。

\biaoti{【临证指要】}

\xiaobt{关于生命过程中阶段性特点及其临床意义}

本篇所述生命过程各阶段的生理、心理特点,为临床各科的形成及其诊治原则,奠定了理论基础。

1.婴幼儿:从出生至十余岁,是婴幼阶段,这一时期生机蓬勃,发育迅速而生理机能尚未成熟,故儿科病证,除先天发育不良外,多外感、伤食,易虚易实,传变迅速,必须及时诊治,当泻则泻,当补则补,贵在切当。

2.青壮年:二三十岁,是为青壮年阶段,人生最辉煌时期,脏腑成熟,气血盛壮,神气健全,抗邪能力最强,疾病虚少实多,治疗以祛邪泻实为主。

3.中年:人生四十岁,生长发育盛极而衰,乃生命过程中盛衰转折阶段,不仅生机开始衰退,而且以往所受病理损伤也由隐伏而显现出来,新疾旧患,虚实夹杂,因此,其疾病的诊治,需要详察病因,细致辨证,分清标本,多法循序处理。

4.老年:五六十岁以至于死,人体生机进一步衰退,不仅表现为明显的老态,而且因虚生实,浊物积聚,形成虚实夹杂、标本互制状态,慢性病多,病程长,并易感外邪,故老年病证,以虚为本,治疗则应攻邪不忘图本,补正不忘疏导,贵在调理,尤重治养结合。

\biaoti{【原文】}

\begin{yuanwen}
黃帝曰:其不能終壽而死者,何也?岐伯曰:其五藏皆不堅,使道不長,空外以張\sb{1} ,喘息暴疾\sb{2} ,又卑基牆,薄脈少血\sb{3} ,其肉不石\sb{4} ,數中風寒,血氣虛,脈不通,真邪相攻,亂而相引\sb{5} ,故中壽而盡也。
\end{yuanwen}

\biaoti{【校注】}

\begin{jiaozhu}
  \item 空外以张:空,同“孔”。空外以张,指鼻孔外翻。
  \item 喘息暴疾:指呼吸急促。
  \item 卑基墙,薄脉少血:卑,低下。卑基墙,与前文“基墙高以方”相反,指面部瘦薄,骨肉塌陷。薄脉少血,即脉小血少,面色枯萎无神。
  \item 石:《太素》作“实”,可参。
  \item 真邪相攻,乱而相引:指正邪相互斗争,气生紊乱,不能驱邪外出,反致引邪深入。
\end{jiaozhu}

\biaoti{【理论阐释】}

\xiaobt{个体寿命长短的先后天因素研究}

本篇先论先天禀赋强壮是长寿的坚实基础,后论先天禀赋薄弱,后天失于调养,真气虚馁,正难御邪,大病既成,损寿夭折,这就确立了先后天因素共同决定寿命长短的养生长寿观。

人的生命源于先天之精,精化气生神,是生命活动的基础,古人称之为“先天生后天”;而此精、气、神又必受后天滋养培育,才能源源不断地化生,维持生命活动,此又称“后天养先天”。因此,先天禀赋是天年寿数的依据和基础,后天调养则是天年寿数得以实现的条件,两者之间是辩证统一关系。所以,先天禀赋强壮,包括具有家族遗传优势者,如若后天调养良好,必得上寿;但若恃强妄为,逆于生乐,则竭精耗真,仅能取中下寿。先天禀赋薄弱者,如若后天调养得当,亦可中寿,甚或上寿;但若不能调养,甚或放纵嗜欲,反复伤邪,则无异对薄弱生命雪上加霜,必致短命夭折。

基于以上理论,本篇提示,为维持生命活动的充沛旺盛,尽终天年,养生活动应始自胚胎、终至老死。其原则是先后天并重,精气神兼养。重先天者,优生,责在父母;重后天者,调摄,责在自身。精气神,后世概括为人身“三宝”,是养生活动调摄的对象,各年龄阶段调养的重点又不同,其具体方法可参照有关篇章。

\xiaojie

本章论养生,选《素问》“上古天真论”“四气调神大论”和《灵枢》“天年”共计3篇,主要阐述养生的理论基础,养生的原则和养生的具体方法。

养生的理论基础是衰老学说,认为衰老是生命活动的自然过程,其主导因素是肾所藏先天精气的自然盛衰规律,而其它脏腑,特别是化生水谷精气的脾胃盛衰也是衰老的重要因素,因而衰老是不可避免的,但可以通过后天养护,强身防病,预防早衰,尽终天年。

养生的原则是:①先后天并重。养先天重在父母,强壮禀赋;养后天重在个人,强身防病。②内外兼养。外避邪气以防耗真气,内治精神等以调养真气。③动静结合。动以养形,静以养神。④天人合一。养生活动因时、因地、因人事制宜。⑤综合调理。

养生的基本方法:①法于阴阳,顺应四时昼夜阴阳消长特点调养心身活动。②和于术数,施行合宜的养生术。③食饮有节,讲究和五味忌偏嗜,适寒温、节饥饱等。④起居有常,生活、工作要有规律。⑤不妄作劳,身、心、房事劳作均应适度。⑥恬惔虚无、精神内守,调养神情意志。⑦及时避邪毒。

\zuozhe{(烟建华)}
\ifx \allfiles \undefined
\end{document}
\fi