% -*- coding: utf-8 -*-
%!TEX program = xelatex
\ifx \allfiles \undefined
\documentclass[draft,12pt]{ctexbook}
%\usepackage{xeCJK}
%\usepackage[14pt]{extsizes} %支持8,9,10,11,12,14,17,20pt

%===================文档页面设置====================
%---------------------印刷版尺寸--------------------
%\usepackage[a4paper,hmargin={2.3cm,1.7cm},vmargin=2.3cm,driver=xetex]{geometry}
%--------------------电子版------------------------
\usepackage[a4paper,margin=2cm,driver=xetex]{geometry}
%\usepackage[paperwidth=9.2cm, paperheight=12.4cm, width=9cm, height=12cm,top=0.2cm,
%            bottom=0.4cm,left=0.2cm,right=0.2cm,foot=0cm, nohead,nofoot,driver=xetex]{geometry}

%===================自定义颜色=====================
\usepackage{xcolor}
  \definecolor{mybackgroundcolor}{cmyk}{0.03,0.03,0.18,0}
  \definecolor{myblue}{rgb}{0,0.2,0.6}

%====================字体设置======================
%--------------------中文字体----------------------
%-----------------------xeCJK下设置中文字体------------------------------%
\setCJKfamilyfont{song}{SimSun}                             %宋体 song
\newcommand{\song}{\CJKfamily{song}}                        % 宋体   (Windows自带simsun.ttf)
\setCJKfamilyfont{xs}{NSimSun}                              %新宋体 xs
\newcommand{\xs}{\CJKfamily{xs}}
\setCJKfamilyfont{fs}{FangSong_GB2312}                      %仿宋2312 fs
\newcommand{\fs}{\CJKfamily{fs}}                            %仿宋体 (Windows自带simfs.ttf)
\setCJKfamilyfont{kai}{KaiTi_GB2312}                        %楷体2312  kai
\newcommand{\kai}{\CJKfamily{kai}}
\setCJKfamilyfont{yh}{Microsoft YaHei}                    %微软雅黑 yh
\newcommand{\yh}{\CJKfamily{yh}}
\setCJKfamilyfont{hei}{SimHei}                                    %黑体  hei
\newcommand{\hei}{\CJKfamily{hei}}                          % 黑体   (Windows自带simhei.ttf)
\setCJKfamilyfont{msunicode}{Arial Unicode MS}            %Arial Unicode MS: msunicode
\newcommand{\msunicode}{\CJKfamily{msunicode}}
\setCJKfamilyfont{li}{LiSu}                                            %隶书  li
\newcommand{\li}{\CJKfamily{li}}
\setCJKfamilyfont{yy}{YouYuan}                             %幼圆  yy
\newcommand{\yy}{\CJKfamily{yy}}
\setCJKfamilyfont{xm}{MingLiU}                                        %细明体  xm
\newcommand{\xm}{\CJKfamily{xm}}
\setCJKfamilyfont{xxm}{PMingLiU}                             %新细明体  xxm
\newcommand{\xxm}{\CJKfamily{xxm}}

\setCJKfamilyfont{hwsong}{STSong}                            %华文宋体  hwsong
\newcommand{\hwsong}{\CJKfamily{hwsong}}
\setCJKfamilyfont{hwzs}{STZhongsong}                        %华文中宋  hwzs
\newcommand{\hwzs}{\CJKfamily{hwzs}}
\setCJKfamilyfont{hwfs}{STFangsong}                            %华文仿宋  hwfs
\newcommand{\hwfs}{\CJKfamily{hwfs}}
\setCJKfamilyfont{hwxh}{STXihei}                                %华文细黑  hwxh
\newcommand{\hwxh}{\CJKfamily{hwxh}}
\setCJKfamilyfont{hwl}{STLiti}                                        %华文隶书  hwl
\newcommand{\hwl}{\CJKfamily{hwl}}
\setCJKfamilyfont{hwxw}{STXinwei}                                %华文新魏  hwxw
\newcommand{\hwxw}{\CJKfamily{hwxw}}
\setCJKfamilyfont{hwk}{STKaiti}                                    %华文楷体  hwk
\newcommand{\hwk}{\CJKfamily{hwk}}
\setCJKfamilyfont{hwxk}{STXingkai}                            %华文行楷  hwxk
\newcommand{\hwxk}{\CJKfamily{hwxk}}
\setCJKfamilyfont{hwcy}{STCaiyun}                                 %华文彩云 hwcy
\newcommand{\hwcy}{\CJKfamily{hwcy}}
\setCJKfamilyfont{hwhp}{STHupo}                                 %华文琥珀   hwhp
\newcommand{\hwhp}{\CJKfamily{hwhp}}

\setCJKfamilyfont{fzsong}{Simsun (Founder Extended)}     %方正宋体超大字符集   fzsong
\newcommand{\fzsong}{\CJKfamily{fzsong}}
\setCJKfamilyfont{fzyao}{FZYaoTi}                                    %方正姚体  fzy
\newcommand{\fzyao}{\CJKfamily{fzyao}}
\setCJKfamilyfont{fzshu}{FZShuTi}                                    %方正舒体 fzshu
\newcommand{\fzshu}{\CJKfamily{fzshu}}

\setCJKfamilyfont{asong}{Adobe Song Std}                        %Adobe 宋体  asong
\newcommand{\asong}{\CJKfamily{asong}}
\setCJKfamilyfont{ahei}{Adobe Heiti Std}                            %Adobe 黑体  ahei
\newcommand{\ahei}{\CJKfamily{ahei}}
\setCJKfamilyfont{akai}{Adobe Kaiti Std}                            %Adobe 楷体  akai
\newcommand{\akai}{\CJKfamily{akai}}

%------------------------------设置字体大小------------------------%
\newcommand{\chuhao}{\fontsize{42pt}{\baselineskip}\selectfont}     %初号
\newcommand{\xiaochuhao}{\fontsize{36pt}{\baselineskip}\selectfont} %小初号
\newcommand{\yihao}{\fontsize{28pt}{\baselineskip}\selectfont}      %一号
\newcommand{\xiaoyihao}{\fontsize{24pt}{\baselineskip}\selectfont}
\newcommand{\erhao}{\fontsize{21pt}{\baselineskip}\selectfont}      %二号
\newcommand{\xiaoerhao}{\fontsize{18pt}{\baselineskip}\selectfont}  %小二号
\newcommand{\sanhao}{\fontsize{15.75pt}{\baselineskip}\selectfont}  %三号
\newcommand{\sihao}{\fontsize{14pt}{\baselineskip}\selectfont}%     四号
\newcommand{\xiaosihao}{\fontsize{12pt}{\baselineskip}\selectfont}  %小四号
\newcommand{\wuhao}{\fontsize{10.5pt}{\baselineskip}\selectfont}    %五号
\newcommand{\xiaowuhao}{\fontsize{9pt}{\baselineskip}\selectfont}   %小五号
\newcommand{\liuhao}{\fontsize{7.875pt}{\baselineskip}\selectfont}  %六号
\newcommand{\qihao}{\fontsize{5.25pt}{\baselineskip}\selectfont}    %七号   %中文字体及字号设置
\xeCJKDeclareSubCJKBlock{SIP}{
  "20000 -> "2A6DF,   % CJK Unified Ideographs Extension B
  "2A700 -> "2B73F,   % CJK Unified Ideographs Extension C
  "2B740 -> "2B81F    % CJK Unified Ideographs Extension D
}
%\setCJKmainfont[SIP={[AutoFakeBold=1.8,Color=red]Sun-ExtB},BoldFont=黑体]{宋体}    % 衬线字体 缺省中文字体

\setCJKmainfont{simsun.ttc}[
  Path=fonts/,
  SIP={[Path=fonts/,AutoFakeBold=1.8,Color=red]simsunb.ttf},
  BoldFont=simhei.ttf
]

%SimSun-ExtB
%Sun-ExtB
%AutoFakeBold:自动伪粗,即正文使用\bfseries时生僻字使用伪粗体;
%FakeBold:强制伪粗,即正文中生僻字均使用伪粗体
%\setCJKmainfont[BoldFont=STHeiti,ItalicFont=STKaiti]{STSong}
%\setCJKsansfont{微软雅黑}黑体
%\setCJKsansfont[BoldFont=STHeiti]{STXihei} %serif是有衬线字体sans serif 无衬线字体
%\setCJKmonofont{STFangsong}    %中文等宽字体

%--------------------英文字体----------------------
\setmainfont{simsun.ttc}[
  Path=fonts/,
  BoldFont=simhei.ttf
]
%\setmainfont[BoldFont=黑体]{宋体}  %缺省英文字体
%\setsansfont
%\setmonofont

%===================目录分栏设置====================
\usepackage[toc,lof,lot]{multitoc}    % 目录(含目录、表格目录、插图目录)分栏设置
  %\renewcommand*{\multicolumntoc}{3} % toc分栏数设置,默认为两栏(\multicolumnlof,\multicolumnlot)
  %\setlength{\columnsep}{1.5cm}      % 调整分栏间距
  \setlength{\columnseprule}{0.2pt}   % 调整分栏竖线的宽度

%==================章节格式设置====================
\setcounter{secnumdepth}{3} % 章节等编号深度 3:子子节\subsubsection
\setcounter{tocdepth}{2}    % 目录显示等度 2:子节

\xeCJKsetup{%
  CJKecglue=\hspace{0.15em},      % 调整中英(含数字)间的字间距
  %CJKmath=true,                  % 在数学环境中直接输出汉字(不需要\text{})
  AllowBreakBetweenPuncts=true,   % 允许标点中间断行,减少文字行溢出
}

\ctexset{%
  part={
    name={,篇},
    number=\SZX{part},
    format={\chuhao\bfseries\centering},
    nameformat={},titleformat={}
  },
  section={
    number={\chinese{section}},
    name={第,节}
  },
  subsection={
    number={\chinese{subsection}、},
    aftername={\hspace{-0.01em}}
  },
  subsubsection={
    number={(\chinese{subsubsection})},
    aftername={\hspace {-0.01em}},
    beforeskip={1.3ex minus .8ex},
    afterskip={1ex minus .6ex},
    indent={\parindent}
  },
  paragraph={
    beforeskip=.1\baselineskip,
    indent={\parindent}
  }
}

\newcommand*\SZX[1]{%
  \ifcase\value{#1}%
    \or 上%
    \or 中%
    \or 下%
  \fi
}

%====================页眉设置======================
\usepackage{titleps}%或者\usepackage{titlesec},titlesec包含titleps
\newpagestyle{special}[\small\sffamily]{
  %\setheadrule{.1pt}
  \headrule
  \sethead[\usepage][][\chaptertitle]
  {\chaptertitle}{}{\usepage}
}

\newpagestyle{main}[\small\sffamily]{
  \headrule
  %\sethead[\usepage][][第\thechapter 章\quad\chaptertitle]
%  {\thesection\quad\sectiontitle}{}{\usepage}}
  \sethead[\usepage][][第\chinese{chapter}章\quad\chaptertitle]
  {第\chinese{section}节\quad\sectiontitle}{}{\usepage}
}

\newpagestyle{main2}[\small\sffamily]{
  \headrule
  \sethead[\usepage][][第\chinese{chapter}章\quad\chaptertitle]
  {第\chinese{section}節\quad\sectiontitle}{}{\usepage}
}

%================ PDF 书签设置=====================
\usepackage{bookmark}[
  depth=2,        % 书签深度 2:子节
  open,           % 默认展开书签
  openlevel=2,    % 展开书签深度 2:子节
  numbered,       % 显示编号
  atend,
]
  % 相比hyperref,bookmark宏包大多数时候只需要编译一次,
  % 而且书签的颜色和字体也可以定制。
  % 比hyperref 更专业 (自动加载hyperref)

%\bookmarksetup{italic,bold,color=blue} % 书签字体斜体/粗体/颜色设置

%------------重置每篇章计数器,必须在hyperref/bookmark之后------------
\makeatletter
  \@addtoreset{chapter}{part}
\makeatother

%------------hyperref 超链接设置------------------------
\hypersetup{%
  pdfencoding=auto,   % 解决新版ctex,引起hyperref UTF-16预警
  colorlinks=true,    % 注释掉此项则交叉引用为彩色边框true/false
  pdfborder=001,      % 注释掉此项则交叉引用为彩色边框
  citecolor=teal,
  linkcolor=myblue,
  urlcolor=black,
  %psdextra,          % 配合使用bookmark宏包,可以直接在pdf 书签中显示数学公式
}

%------------PDF 属性设置------------------------------
\hypersetup{%
  pdfkeywords={黄帝内经,内经,内经讲义,21世纪课程教材},    % 关键词
  %pdfsubject={latex},        % 主题
  pdfauthor={主编:王洪图},   % 作者
  pdftitle={内经讲义},        % 标题
  %pdfcreator={texlive2011}   % pdf创建器
}

%------------PDF 加密----------------------------------
%仅适用于xelatex引擎 基于xdvipdfmx
%\special{pdf:encrypt ownerpw (abc) userpw (xyz) length 128 perm 2052}

%仅适用于pdflatex引擎
%\usepackage[owner=Donald,user=Knuth,print=false]{pdfcrypt}

%其他可使用第三方工具 如:pdftk
%pdftk inputfile.pdf output outputfile.pdf encrypt_128bit owner_pw yourownerpw user_pw youruserpw

%=============自定义环境、列表及列表设置================
% 标题
\def\biaoti#1{\vspace{1.7ex plus 3ex minus .2ex}{\bfseries #1}}%\noindent\hei
% 小标题
\def\xiaobt#1{{\bfseries #1}}
% 小结
\def\xiaojie {\vspace{1.8ex plus .3ex minus .3ex}\centerline{\large\bfseries 小\ \ 结}\vspace{.1\baselineskip}}
% 作者
\def\zuozhe#1{\rightline{\bfseries #1}}

\newcounter{yuanwen}    % 新计数器 yuanwen
\newcounter{jiaozhu}    % 新计数器 jiaozhu

\newenvironment{yuanwen}[2][【原文】]{%
  %\biaoti{#1}\par
  \stepcounter{yuanwen}   % 计数器 yuanwen+1
  \bfseries #2}
  {}

\usepackage{enumitem}
\newenvironment{jiaozhu}[1][【校注】]{%
  %\biaoti{#1}\par
  \stepcounter{jiaozhu}   % 计数器 jiaozhu+1
  \begin{enumerate}[%
    label=\mylabel{\arabic*}{\circledctr*},before=\small,fullwidth,%
    itemindent=\parindent,listparindent=\parindent,%labelsep=-1pt,%labelwidth=0em,
    itemsep=0pt,topsep=0pt,partopsep=0pt,parsep=0pt
  ]}
  {\end{enumerate}}

%===================注解与原文相互跳转====================
%----------------第1部分 设置相互跳转锚点-----------------
\makeatletter
  \protected\def\mylabel#1#2{% 注解-->原文
    \hyperlink{back:\theyuanwen:#1}{\Hy@raisedlink{\hypertarget{\thejiaozhu:#1}{}}#2}}

  \protected\def\myref#1#2{% 原文-->注解
    \hyperlink{\theyuanwen:#1}{\Hy@raisedlink{\hypertarget{back:\theyuanwen:#1}{}}#2}}
  %此处\theyuanwen:#1实际指thejiaozhu:#1,只是\thejiaozhu计数器还没更新,故使用\theyuanwen计数器代替
\makeatother

\protected\def\myjzref#1{% 脚注中的引用(引用到原文)
  \hyperlink{\theyuanwen:#1}{\circlednum{#1}}}

\def\sb#1{\myref{#1}{\textsuperscript{\circlednum{#1}}}}    % 带圈数字上标

%----------------第2部分 调整锚点垂直距离-----------------
\def\HyperRaiseLinkDefault{.8\baselineskip} %调整锚点垂直距离
%\let\oldhypertarget\hypertarget
%\makeatletter
%  \def\hypertarget#1#2{\Hy@raisedlink{\oldhypertarget{#1}{#2}}}
%\makeatother

%====================带圈数字列表标头====================
\newfontfamily\circledfont[Path = fonts/]{meiryo.ttc}  % 日文字体,明瞭体
%\newfontfamily\circledfont{Meiryo}  % 日文字体,明瞭体

\protected\def\circlednum#1{{\makexeCJKinactive\circledfont\textcircled{#1}}}

\newcommand*\circledctr[1]{%
  \expandafter\circlednum\expandafter{\number\value{#1}}}
\AddEnumerateCounter*\circledctr\circlednum{1}

% 参考自:http://bbs.ctex.org/forum.php?mod=redirect&goto=findpost&ptid=78709&pid=460496&fromuid=40353

%======================插图/tikz图========================
\usepackage{graphicx,subcaption,wrapfig}    % 图,subcaption含子图功能代替subfig,图文混排
  \graphicspath{{img/}}                     % 设置图片文件路径

\def\pgfsysdriver{pgfsys-xetex.def}         % 设置tikz的驱动引擎
\usepackage{tikz}
  \usetikzlibrary{calc,decorations.text,arrows,positioning}

%---------设置tikz图片默认格式(字号、行间距、单元格高度)-------
\let\oldtikzpicture\tikzpicture
\renewcommand{\tikzpicture}{%
  \small
  \renewcommand{\baselinestretch}{0.2}
  \linespread{0.2}
  \oldtikzpicture
}

%=========================表格相关===============================
\usepackage{%
  multirow,                   % 单元格纵向合并
  array,makecell,longtable,   % 表格功能加强,tabu的依赖
  tabu-last-fix,              % "强大的表格工具" 本地修复版
  diagbox,                    % 表头斜线
  threeparttable,             % 表格内脚注(需打补丁支持tabu,longtabu)
}

%----------给threeparttable打补丁用于tabu,longtabu--------------
%解决方案来自:http://bbs.ctex.org/forum.php?mod=redirect&goto=findpost&ptid=80318&pid=467217&fromuid=40353
\usepackage{xpatch}

\makeatletter
  \chardef\TPT@@@asteriskcatcode=\catcode`*
  \catcode`*=11
  \xpatchcmd{\threeparttable}
    {\TPT@hookin{tabular}}
    {\TPT@hookin{tabular}\TPT@hookin{tabu}}
    {}{}
  \catcode`*=\TPT@@@asteriskcatcode
\makeatother

%------------设置表格默认格式(字号、行间距、单元格高度)------------
\let\oldtabular\tabular
\renewcommand{\tabular}{%
  \renewcommand\baselinestretch{0.9}\small    % 设置行间距和字号
  \renewcommand\arraystretch{1.5}             % 调整单元格高度
  %\renewcommand\multirowsetup{\centering}
  \oldtabular
}
%设置行间距,且必须放在字号设置前 否则无效
%或者使用\fontsize{<size>}{<baseline>}\selectfont 同时设置字号和行间距

\let\oldtabu\tabu
\renewcommand{\tabu}{%
  \renewcommand\baselinestretch{0.9}\small    % 设置行间距和字号
  \renewcommand\arraystretch{1.8}             % 调整单元格高度
  %\renewcommand\multirowsetup{\centering}
  \oldtabu
}

%------------模仿booktabs宏包的三线宽度设置---------------
\def\toprule   {\Xhline{.08em}}
\def\midrule   {\Xhline{.05em}}
\def\bottomrule{\Xhline{.08em}}
%-------------------------------------
%\setlength{\arrayrulewidth}{2pt} 设定表格中所有边框的线宽为同样的值
%\Xhline{} \Xcline{}分别设定表格中水平线的宽度 makecell包提供

%表格中垂直线的宽度可以通过在表格导言区(preamble),利用命令 !{\vrule width1.2pt} 替换 | 即可

%=================图表设置===============================
%---------------图表标号设置-----------------------------
\renewcommand\thefigure{\arabic{section}-\arabic{figure}}
\renewcommand\thetable {\arabic{section}-\arabic{table}}

\usepackage{caption}
  \captionsetup{font=small,}
  \captionsetup[table] {labelfont=bf,textfont=bf,belowskip=3pt,aboveskip=0pt} %仅表格 top
  \captionsetup[figure]{belowskip=0pt,aboveskip=3pt}  %仅图片 below

%\setlength{\abovecaptionskip}{3pt}
%\setlength{\belowcaptionskip}{3pt} %图、表题目上下的间距
\setlength{\intextsep}   {5pt}  %浮动体和正文间的距离
\setlength{\textfloatsep}{5pt}

%====================全文水印==========================
%解决方案来自:
%http://bbs.ctex.org/forum.php?mod=redirect&goto=findpost&ptid=79190&pid=462496&fromuid=40353
%https://zhuanlan.zhihu.com/p/19734756?columnSlug=LaTeX
\usepackage{eso-pic}

%eso-pic中\AtPageCenter有点水平偏右
\renewcommand\AtPageCenter[1]{\parbox[b][\paperheight]{\paperwidth}{\vfill\centering#1\vfill}}

\newcommand{\watermark}[3]{%
  \AddToShipoutPictureBG{%
    \AtPageCenter{%
      \tikz\node[%
        overlay,
        text=red!50,
        %font=\sffamily\bfseries,
        rotate=#1,
        scale=#2
      ]{#3};
    }
  }
}

\newcommand{\watermarkoff}{\ClearShipoutPictureBG}

\watermark{45}{15}{草\ 稿}    %启用全文水印

%=============花括号分支结构图=========================
\usepackage{schemata}

\xpatchcmd{\schema}
  {1.44265ex}{-1ex}
  {}{}

\newcommand\SC[2] {\schema{\schemabox{#1}}{\schemabox{#2}}}
\newcommand\SCh[4]{\Schema{#1}{#2}{\schemabox{#3}}{\schemabox{#4}}}

%=======================================================

\begin{document}
\pagestyle{main}
\fi
%下篇《皇帝内经》与医学相关专题研究

\chapter{《黄帝内经》的医学哲学思想}%第一章

\section{天人观}%第一节

天人问题是中国哲学的基本问题,“天”与“人”是中国哲学的一对重要范畴。

\subsection{天道观}%一、

《黄帝内经》有关“天”的论述集中反映了天道观、宇宙论思想,《黄帝内经》的天道观是其医学哲学的重要组成部分。

\subsubsection{“天”是独立于人的意志之外的客观自然存在}%(一)

“天”字在《内经》中,含义较为复杂。从语义学上讲,主要指天空、自然界、天气、天时。如《素问·阴阳应象大论》曰:“天不足西北,故西北方阴也。”又引申为自然的状态、本来的面貌。如“天真”、“天年”、“天寿”、“天数”中的“天”。

从哲学上看,《内经》的天道观与殷周时期的天道观是不同的。殷周时期的“天”主要是指意志之天、主宰之天、神灵之天。到了周末,天的权威性开始减弱,春秋战国时代的诸子百家,改变了殷周天人关系理论。《内经》的“天”主要是指独立于人的意志之外的、不以人的意志为转移的客观存在,是不断运动变化的物质世界。天不仅是无意志、无目的的,而且是无限的。如《素问·天元纪大论》曰:“故在天为气,在地成形,形气相感而化生万物矣。”《灵枢·经水》曰:“天至高,不可度;地至广,不可量。”虽然《内经》中有一些“天”字从表面上看是指有意志、有目的“天”,但实际上往往用在反诘问句中,是为了否定有意志、有目的的“天”的。如《灵枢·本神》对精神疾患的病因所发出的“天之罪与?人之过乎?”的诘问,其中的“天”不能简单地看成就是有意志的神、万物的主宰。

\subsubsection{天地的生成与结构}%(二)

《内经》认为天地是阴阳二气不断分化积累的结果,是一个生成的过程。《素问·阴阳应象大论》说:“积阳为天,积阴为地。”阴阳产生天地,无限的天地宇宙化生出无穷的事物。《素问·六节藏象论》说:“天至广,不可度;地至大,不可量。……草生五色,五色之变,不可胜视,草生五味,五味之美,不可胜极,嗜欲不同,各有所通。”现实世界是逐步产生出来的。《素问·天元纪大论》说:“太虚廖廓,肇基化元,万物资始,五运终天,布气真灵,揔统坤元,九星悬朗,七曜周旋,曰阴曰阳,曰柔曰刚,幽显既位,寒暑弛张,生生化化,品物咸章。”太虚就是广阔无限的“天”,太虚与真元之气是整个宇宙产生的基础,万物产生的本原。

\subsection{人道观}%二、

《黄帝内经》的人道观——人学思想相当丰富。作为以人为研究对象的医学著作,《内经》必须回答人道——人学的基本问题,从而构成了《内经》颇具特色的人道观和人学思想。

\subsubsection{人的本原和生成}%(一)

《内经》吸收了《周易》、《庄子》有关人的生成的思想,认为人是由于天地之气的相互作用而产生的。《素问·宝命全形论》说:“人以天地之气生,四时之法成。”“夫人生于地,悬命于天,天地合气,命之曰人。”《内经》的作者已涉及到生命起源的问题,认识到生命是天地阴阳两气相感的产物,是自然界物质变化的结果。《灵枢·本神》说:“天之在我者德也,地之在我者气也,德流气薄而生者也。”说明天德和地气的交互作用产生了人类。这里的“德”,也是一种气,是指万物成长的内在基础,《庄子·天地》说:“物得以生,谓之德。”可见,“德”是生成万物的一种内在能动力量、是一种有利于生命起源的本原物质。

\subsubsection{人的形神关系}%(二)

人的形神问题即形体与精神的关系问题,是先秦诸子哲学中论述较多的一个问题。《内经》继承了庄子“精神生于道,形体生于精”、后期墨家“刑(形)与知处”、荀子“形具而神生”的形神观,结合当时的医学科学成就,丰富和发展了先秦以来的形神说。形,在《内经》中主要有两种涵义,一是指人的形体,二是指万事万物的形体(物质形态)。神,在《内经》中主要有三种涵义:一是指人体的精神意识;二是指生物体的生理功能和综合生命力;三是指宇宙自然世界的运动变化及其规律性。《内经》有关人的“形”“神”关系主要表现为形体与精神的关系、形体与功能的关系。

《内经》认为人的精神包括思维、情志、感觉等精神意识活动。人的形体生成精神,精神是形体的产物;精神意识又反作用于形体,并对形体起一定的主导作用。这些精神意识活动都是在五脏、特别是心的功能基础上产生出来的。《素问·宣明五气》说:“五脏所藏,心藏神,肺藏魄,肝藏魂,脾藏意,背藏志,是谓五脏所藏。”《素问·阴阳应象大论》又说:“人有五脏化五气,以生喜、怒、悲、忧、恐。”喜、怒、悲、忧、恐五种情志是人对外界刺激反应出来的一种精神活动。五种情志分别由心、肝、肺、脾、肾五脏产生。说明五脏精气是情志活动的物质基础。《内经》运用阴阳的对立统一关系来说明形体和机能的关系,《素问·阴阳应象大论》说:“阴在内,阳之守也;阳在外,阴之使也。”《素问·生气通天论》说:“阴者,藏精而起亟也;阳者,卫外而为固也。”物质形体为阴,生命功能为阳。内在的形体物质是外在的生命功能的物质基础,外在的生命功能又是内在的生命物质的主导和护卫。健全的形体是机能旺盛的物质保证,机能旺盛又是形体强健的根本条件。为了说明“形”,《内经》提出了“精”的概念,认为“精”不仅是构成人的形体而且是构成人的生理功能的基本物质。《灵枢·本神》说:“故生之来谓之精,两精相搏谓之神。”

\subsubsection{人性}%(三)

先秦诸子百家对人性问题进行了热烈讨论,提出了各自的观点,如孟子主张性善论,荀子主张性恶论,告子主张性无善恶论,道家主张人性自然论,等等。《内经》作为一部天地人三位一体的综合性医学著作,对人性问题也有所涉及,提出了具有医学特色的人性论观点。

《内经》对于人性善恶问题的探讨是与人的气质、人格等内容交织在一起的,《内经》的人性学说深受诸子影响,内容较复杂,既有儒家的善恶论思想,又有道家的自然论思想。从善恶角度说,《内经》主要受有善有恶说与董仲舒“性三品”说的影响。《灵枢·通天》根据人的气质性格将人分为太阴、少阴、太阳、少阳、阴阳和平五类,这同时是一种人性的分类法,因为其中包含有对人性善恶的价值评价,太阴、少阴之人属于性恶之列,阴阳和平之人属于性善之列,从《内经》的描述来看,阴阳和平之人具有道家理想人格的色彩,而与儒家圣人形象有所不同。至于太阳、少阳之人则既不属于善者之列,也不属于恶者之列。可见《内经》人性说并不是简单的善恶二分,而是包含善恶的阴阳五分,如果再简单归纳一下就是阴、阳、阴阳和平三类。

《内经》不仅对人性作了分类描述,而且对不同人性的形成作了本体论的阐释。《灵枢·通天》认为:太阴之人“多阴而无阳”,少阴之人“多阴少阳”,太阳之人“多阳而少阴”,少阳之人“多阳少阴”,阴阳和平之人“阴阳之气和”,《灵枢·行针》认为重阳之人“颇有阴”。将先天阴阳之“气”作为人性的基础,这是先秦诸子人性论所未涉及的。作为医学著作,《内经》并不太关注人性的社会性以及人性是否可以改变问题,而是以气秉论人性,从先天生理因素寻找人性的根据,关注五态之人的发病及其治法。《内经》阴阳五分的人性论思想的目的不是解释道德现象以及提供治国方略的理论根据,而是为养生治疗提供理论指导。因此《内经》特别重视人性修养对于养生治疗的作用。

\subsection{天人观}%三、

天人关系论是中国哲学包括《内经》哲学天人学说的核心。先秦哲学家提出了“天人合一”、“天人相分”和“天人相胜”等观点。在天人关系问题上,《内经》主张“天人合一”论,具体表现为"天人相应”学说,可以说“天人相应”思想是《内经》的核心思想之一。《内经》反复强调人“与天地相应,与四时相副,人参天地”(《灵枢·刺节真邪》),“人与天地相参也”(《灵枢·岁露》),“人之所以参天地而应阴阳”(《灵枢·经水》,“与天地如一”(《素问·脉要精微论》)。《内经》“天人相应”学说主要体现在以下三个方面。

\subsubsection{天人相似}%(一)

天人相似指人体与天地万物的形态结构相类似。《内经》认为人的身体结构体现了天地的结构。例如《灵枢·邪客》把人体形态结构与天地万物一一对应起来,一一作了类比。人体的结构可以在自然界中找到相对应的东西,人体仿佛是天地的缩影。《灵枢·经水》在解释十二经脉与十二经水的对应关系时说:“凡此五藏六府十二经水者,外有源泉而内有所禀。”认为外在的十二经水和内在的十二经脉都有一个共同的来源,即天地之气。天地之气在外形成十二经水,在内形成十二经脉。人体的十二经脉与自然界的十二经水是相应的。十二经水是行水的,而十二经脉是行血的,如同经水有远近深浅的差别,十二经脉中的气血也有远近深浅的不同,二者是相对应的。这种思想的形成,与汉代盛行的“人副天数”有密切关系。

\subsubsection{天人相动}%(二)

天人相动是指人体生理功能节律随天地四时之气运动变化而改变。人与天之间存在着随应而动和制天而用的统一关系。《内经》认为人体生理功能变化的节律与天地自然四时变化的节律一致,人体生理功能随着自然界年、季、月、日、时的变化而发生相应的变化。就一年四时而言,“春生、夏长、秋收、冬藏,是气之常也。人亦应之。”(《灵枢·顺气一日分为四时》)人的生理功能活动随春夏秋冬四季的变更而发生生长收藏的相应变化。就一年十二月而言,“正月二月,天气始方,地气始发,人气在肝。三月四月,天气正方,地气定发,人气在脾。五月六月,天气盛,地气高,人气在头。七月八月,阴气始杀,人气在肺。九月十月,阴气始冰,地气始闭,人气在心。十一月十二月,冰复,地气合,人气在肾。”(《素问·诊要经终论》)随着月份的推移,人气在不同部位,发挥相应的作用。就一日而言,“阳气者,一日而主外,平旦人气生,日中而阳气隆,日西而阳气已虚,气门乃闭。”(《素问·生气通天论》)随着自然界阳气的消长变化,人体的阳气也发生相应的改变。人体卫气也随着昼夜出阳入阴的变化而变化,卫气白昼行于阳经二十五度,夜晚行于阴经二十五度。在一日之内也体现了一年四季的变化节律,这一点在病理上表现较明显,“以一日分为四时,朝则为春,日中为夏,日入为秋,夜半为冬。朝则人气始生,病气衰,故旦慧;日中人气长,长则胜邪,故安;夕则人气始衰,邪气始生,故加;夜半人气入脏,邪气独居于身,故甚也。”“夫百病者,多以旦慧昼安,夕加夜甚。”(《灵枢·顺气一日分为四时》)

\subsubsection{天人相通}%(三)

天人相通指人与天的规律相通。《内经》认为,人体不仅与自然界的共性运动规律相通,而且与自然界的具体运动规律相通。阴阳五行是宇宙事物的总规律,不管是对自然界,还是对人体生理变化,都具有普遍的指导意义。《灵枢·通天》说:“天地之间,六合之内,不离于五,人亦应之,非徒一阴一阳而已也。”由于人体和自然界有着共同的规律,因而可以归为同“类”。《内经》利用这个“类”从已知的自然界的事物去推知人体脏腑的生理功能,提出了比类的方法,“及于比类,通合道理。”(《素问·示从容论》)根据“天人相应”的原理,通过“外揣”即对外在自然现象的观察,以自然运动规律来类推人体生命运动规律。《素问·阴阳应象大论》通过“清阳为天,浊阴为地。地气上为云,天气下为雨,雨出地气,云出天气”的自然现象分布以及变化规律,推论出人体内存在着同样的生理变化规律:“故清阳出上窍,浊阴出下窍:清阳发腠理,浊阴走五藏;清阳实四肢,浊阴归六府。”说明人体内存在与天地之气同一形式的新陈代谢过程。《内经》进而认为人体五藏与自然界的四时五行遵从同一运动规律。《素问·刺禁论》曰:“肝生于左,肺藏于右,心部于表,肾治于里,脾为之使,胃为之市。”从今天的解剖学角度看,这段话所言的五脏方位是错误的,然而这里的“左”、“右”、“表”、“里”以及“生”、“藏”、“部”、“治”等并非解剖学的定位概念,而是气机运动的动态功能概念,是从阴阳、四时、五行总体规律上类比、推理出来的。

《内经》天人相应思想强调自然的运动变化对人的生理、病理机能的制约作用的观点,与董仲舒的天人感应论是不同的。董仲舒认为天不仅能影响人,人亦能影响天,天人感应的中介是气,这样气就具有了神秘的性质,将天人格化,最终陷入神学目的论中。《内经》的天人相应论,不承认人能影响天,不将天意志化,而是把天看成是客观存在的物质自然。其天人相应思想是建立在气论自然观基础上的。人类作为气所化生的万物中的一部分,其运动变化的规律节律与天地自然是一致的。因此天能够影响人,而人并不能影响天。

\subsection{天人相应观在建构中医学体系中的作用}%四、

《内经》的天人观是《内经》医学形成的哲学基础,其中“天人合一”、“天人相应”观体现了整体系统论思想,促使了《内经》整体系统医学体系的形成。具体表现在以下方面。

\subsubsection{藏象学说}%(一)

《内经》藏象学说,是在天人相应的思想指导下建构起来的。《内经》认为“有诸内必形诸外”,人体脏腑、气血、经络深藏于体内,但可显象于外,可以通过已知的自然现象去推知隐蔽的内脏功能。所谓“藏象”即指藏于内、象于外。根据外在的“象”可以推测内在的脏腑功能、气血活动、经脉长短。

根据五时推知五脏的生理功能特点,如《素问·六节藏象论》认为肝、心、脾、肺、肾分别与春、夏、长夏、秋、冬通应,分别主阳气始生、阳气旺长、阳气盛极、阳气渐消、阴气旺盛。根据天地四时寒温推知气血律液的活动规律,如《灵枢·卫气行》、《灵枢·营卫生会》说营卫的运行“与天地同纪”,并以太阳的视运动作为认识人体营卫运行规律的依据。太阳视运动的轨道分阴阳,太阳昼行于阳十四舍,夜行于阴十四舍,人身亦有阴阳,营卫运行“阴阳相贯,如环无端”,日行于阳二十五度,夜行于阴二十五度。根据月亮盈亏消长推论人体气血的盛衰变化,如《素问·八正神明论》说:“月始生,则血气始精,卫气始行;月廓满,则血气实,肌肉坚;月廓空,则肌肉减,经络虚,卫气去,形独居。”《灵枢·岁露》也论述了“月满”和“月空”对人体气血盛衰的影响。根据二十八宿天象推知人身二十八脉的长短,如《灵枢·五十营》根据日行二十八宿“一万三千五百息,气行五十营于身”、“呼吸定息,气行六寸”等数据计算出周身十六丈二尺。根据四季的变化推知脉象的节律变化,随着四季的变化,脉象有不同的表现。如《素问·玉机真藏论》说:“春脉如弦”、“夏脉如钩”、“秋脉如浮”、“冬脉如营”,四时脉象的节律变化、五藏机能的递相旺衰与四时的生长收藏一一相应。

\subsubsection{病机学说}%(二)

根据天时和自然现象推论发病及其病因,如《素问·咳论》说:“五脏各以其时受病,非其时各传以与之。人与天地相参,故五脏各以治时,感于寒则受病,微则为咳,甚者为泄为痛。乘秋则肺先受邪,乘春则肝先受之,乘夏则心先受之,乘至阴则脾先受之,乘冬则肾先受之。”从对自然现象的观察中推论疾病的发生。《内经》认为疾病的发生是从致病因素侵袭人体开始的,如《灵枢·顺气一日分为四时》说:“夫百病之所始生者,必起于燥湿寒暑风雨,阴阳喜怒,饮食居处。”其中燥湿寒暑风雨六种气候和自然现象,分别具有善行数变、温热、上炎、重独、干燥、凝滞收引的特性,无论是太过还是不及都会引起相应的病证。《内经》根据阴阳四时消长变化推论疾病传变,认为不同性质、不同季节的病因往往侵袭与之同类的部位,如《素问·金匮真言论》说:“东风生于春,病在肝,俞在颈项;南风生于夏,病在心,俞在胸胁;西风生于秋;病在肺,俞在肩背;北风生于冬,病在肾,俞在腰股;中央为土,病在脾,俞在脊。”人体正气盛衰决定疾病的进退,而人体正气的盛衰与昼夜四时阴阳消长同步,因此疾病的进退也随昼夜四时阴阳消长发生相应的变化,《灵枢·顺气一日分为四时》说:“夫百病者,多以旦慧昼安,夕加夜甚。”充分体现了天人相应的思想。人体疾病传变是有规律性的,《素问·玉机真藏论》论述了以风寒邪气为病因的外感病按照五行相胜的顺序传变的规律。

\subsubsection{诊法学说}%(三)

《内经》以天地四时相应观念指导诊断辨证,如《素问·三部九候论》受天地人三才思想的启示而建立三部九候全身遍诊法。又如《素问·脉要精微论》认为脉随四时阴阳的变动而上下浮沉,表现为春规、夏矩、秋衡、冬权的四时脉象。根据异常脉象与四时的关系,可以判断疾病所在和死亡时间。《内经》以阴阳五行观念指导诊断辨证,认为在诊断时,气色的清浊、音声的高低、脉象的浮沉、尺肤的滑涩,等等,皆可归属于阴阳,进而判断疾病的阴证、阳证本质。在面部色诊时,借鉴五行原理,确立所病部位,推论五官、五体、五色主病,如《素问·刺热论》说:“肝热病者,左颊先赤;心热病者,颜先赤;脾热病者,鼻先赤;肺热病者,右颊先赤;肾热病者,颐先赤。”从而建立了五色主病的诊断方法。

\subsubsection{治疗学说}%(四)

根据天地人“三才”思想提出因时因地因人的“三因”论治学说,如《素问·阴阳应象大论》说:“治不法天之纪,不用地之理,则灾害至矣。”强调三因论治的重要性。《素问·五常政大论》说:“必先岁气,无伐天和。”用药论治,必须顺应四时节令,不可违时妄治。《素问·异法方宜论》说:“黄帝问曰:医之治病也,一病而治各不同,皆愈何也?岐伯曰:地势使然也。”五方地势不同,体质、发病各异,治疗手段相应有别。根据天地生化原理提出正治反治的法则。《素问·至真要大论》说逆者正治,从者反治。”正治又称逆治,就是逆其证候性质而治疗;反治又称从治,就是顺从其病症假象而治疗。根据时间特征决定用药针灸,《素问·藏气法时论》提出“肝主春,足厥阴、少阳主治,其日甲乙,肝苦急,急食甘以缓之”等五藏之病应在其主时之日用药治疗的观点。

\subsubsection{养生学说}%(五)

《内经》认为人类生存在自然界当中,只有与自然息息相通、顺从自然规律来生活、养生,才能健康长寿。因此取法自然、顺应天时是《内经》养生学说的基本原则。《素问·四气调神大论》说:“夫四时阴阳者,万物之根本也,所以圣人春夏养阳,秋冬养阴,以从其根,故与万物沉浮于生长之门。逆其根则伐其本,坏其真矣。故阴阳四时者,万物之终始也,死生之本也,逆之则灾害生,从之则苛疾不起,是谓得道。”人在漫长的生物进化过程中,形成了与昼夜变化规律相应的生物节律,只有顺应这种节律变化才能保持人体健康。《内经》特别重视日常生活的养生,强调节制有度,只有做到饮食有节、起居有常、房室有度、寤寐适时、情志调畅,才能寿尽百岁。这一养生原则就是在天人相应的思想指导下形成的。

\section{气—阴阳—五行论}%第二节

“气”“阴阳”“五行”是中国古代哲学的重要范畴,被中国古代多数哲学家用来说明宇宙的本原、事物的构成及变化规律。《内经》采用“气”“阴阳”“五行”的范畴和“气—阴阳—五行”模型说明人体生命的本质动力、生理功能、病理变化及珍断治疗。可以说“气—阴阳—五行”是中医学的理论基础和哲学指导。

\subsection{“气—阴阳—五行”的来源}%一、

\subsubsection{气的来源}%(一)

“气”字甲骨文中已经出现,原指气体状态的存在物,如云气、蒸气、烟气以及风等。“气”的抽象概念在古籍文献中最早见于《国语·周语》。西周末期,周幽王二年(公元前780年),三川皆地震,伯阳父解释说:“夫天地之气,不失其序,若过其序,民乱之也。阳伏而不能出,阴迫而不能烝,于是有地震。”这里的“气”指天地之气、阴阳之气,已演变为一个抽象的具有哲学意味的概念。

春秋时代,老子、孔子都讲过“气”。《老子》说:“万物负阴而抱阳,冲气以为和。”(《四十二章》)这里的“气”是一个哲学概念,“冲气”就是阴气与阳气的调和、和合。战国时期,《孟子》、《管子》、《庄子》、《荀子》都讲“气”,而且大都是从哲学上讲的。集先秦诸子之大成的《易传》,提出了“气”化生万物:“精气为物,游魂为变。”“天地氤氲,万物化醇,男女媾精,万物化生。”(《系辞传》)“二气感应以相与……观其所感而天下万物之情可见矣。”(《咸·彖传》)认为天下万物皆由阴阳二气相感交合而生成。在汉代,“气”己是一个重要的哲学范畴。

\subsubsection{阴阳的来源}%(二)

“阴阳”观念起源很早,大约在上古农耕时代。上古时代人们观察日月之象,昼夜、阴晴、寒暑变化,发现大量相反相对现象,又在农业生产中发现向阳者丰收、背阴者减产等现象,殷、周时期,人们就总结出“相其阴阳”的生产经验。最早记载阴阳观念的是《易经》。《易经》大约成书于西周前期,由六十四卦卦爻象符号系统与六十四卦卦爻辞文字系统组成。其最基本的符号是“两爻”:“—”和“-{-}”反映了上古先哲的阴阳观念。在卦文辞文字中,也有大量的表示阴阳对立的词语,如乾坤、泰否、剥复、损益、既济未济等卦名,还有吉凶、上下、大小、往来等卦爻辞词语。可见至迟在殷、周之际,阴阳观念已相当成熟。从《尚书》、《诗经》等古籍看,也反映了阴阳的观念。

突破原始意义而开始具有哲学意义的“阴阳”概念出现在《国语》、《左传》中。据《囯语·周语》的记载,“阴阳”概念的出现至迟是在西周末年。周宣王即位(公元前827年),卿士虢文公劝诉宣王不可废除籍田仪式,其中以“阴阳”二气解释土地解冻、春雷震动的原因:“阴阳分布,震雷出滞。”(《国语·周语上》)周幽王二年(公元前780年),太史伯阳父以“阴阳”二气解释地震:“阳伏而不能出,阴迫而不能烝,于是有地震。”(《国语·周语上》)可见,西周末年的“阴阳”已抽象为具有普适意义的“二气”到了春秋战国时期,儒家、道家、墨家、法家、兵家、杂家都普通使用“阴阳”概念。道家的创始人老子是第一个真正将“阴阳”提升为哲学范畴的哲学家。战国时期更出现了专论“阴阳”的阴阳家,以邹衍为代表的阴阳家不仅融合了阴阳学说与五行学说,而且以阴阳五行解释季节变化和农作物生长,解释王朝的更替、政治的兴衰。

将“阴阳”思想更加系统化、理论化,并达到空前水平的是战国时期成书的《易传》。《易传》将“阴阳”提升到哲学本体论层面,并明确提出“一阴一阳之谓道”的命题。可以说《易传》是我国第一部系统论述阴阳哲学的专著。

\subsubsection{五行的来源}%(三)

“五行”说起源于殷商时代,当时出现了“四方”观念,甲骨文中有“四方”和“四方风”的记载,从中央看四方乃是殷人的方位观。殷商大墓和明堂中有大量的表示五方图案的构造。“五行”概念的真正出现是在周代。春秋时代出现五行相胜学说。战国时代出现五行相生学说、五行与阴阳配合学说,此时五行已成为一种宇宙模型被广泛运用;到了汉代,阴阳五行已共同成为神圣不可更改的世界规、方法论,并一直延续到清末。

从现存文献看,最早记载“五行”概念的是《尚书》,《尚书》有两篇文献中提到“五行”一词,一篇是《夏书·甘誓》,一篇是《周书·洪范》;另一篇文献《虞书·大禹谟》提到了“五行”的具体名目。先秦古籍《逸周书》也提到了“五行”,并有五行相胜的记载。《左传》、《国语》中记载了大量的有关“五行”的言论或事件。先秦诸子如《孙子》、《墨子》、《管子》等均有关于“五行”的记载。子思和孟子“案往旧造说,谓之五行”。(《荀子·非十二子》)邹衍第一次把阴阳说和五行说结合起来,用阴阳消长的道理来说明五行的运动变化,构成阴阳五行说。并提出“五德终始”(又称“五德转移”)说,用五行相胜的过程解释社会历史的发展。汉代是阴阳五行学说被泛化和神学化的时代,汉武帝时,董仲舒将阴阳五行由对自然现象的认识模型一跃而变成对社会政治的说理工具。可以说两汉时期重要的学术著作几乎都涉及到五行。

\subsection{“气—阴阳—五行”的内涵及其关系}%二、

\subsubsection{气的内涵}%(一)

从字义上看,“气”主要指为风、云、雾等自然界的气体存在物。《说文解字》说:“气,云气也,象形。”也指精良的粟米,引申为物之精华,即“精气”。

作为一个哲学概念,“气”主要有以下意义:

1.气是天地万物的本原,是生命的基本条件。《素问·阴阳应象大论》说:“清阳为天,浊阴为地。”“天有精,地有形,天有八纪,地有五里,故能为万物之父母。”清阳和浊阴是气的两种形式,阴阳二气不仅产生天地,而且产生万物,包括人,《素问·宝命全形论》说:“人以天地之气生。”

2.气是无形的客观存在。《素问·六微旨大论》认为气的升降出入,表现为“无形无患”。气无形但气聚为有形。《素问·六节藏象论》说:“气合而有形。”

3.气是天地万物感应的中介。物体与物体之间充满了气,每一个物体内部也充满了气,充斥于天地万物之间的气是联系天地万物的中介,也是联系每一物体内部各部分的中介。万物以气为中介,相互感应,相互融和。正因为有了气,所以天地万物才成为一个合一的整体,每一个事物才成为一个内部互有关连的整体。

\subsubsection{阴阳的内涵}%(二)

从字义上看,“阴阳”指阳光照射不到的地方与阳光照射到的地方。《说文解字》:“阴,闇也。水之南、山之北也。”“阳,高明也。”段注:“山南曰阳。”从《尚书》、《诗经》中“阴”“阳”的意义看,大部分取此义。

作为一个哲学概念,“阴阳”主要指事物相对、相反但又合和、统一的属性。老子将万物看成“负阴而抱阳”,万物具有阴阳合抱的属性。《管子》、《庄子》进一步将阴阳与动静相联系,发挥“阴阳”的属性涵义。而真正完成并普遍使用“阴阳”属性涵义的是《易传》。《易传》“阴阳”虽也指日月、天地、乾坤等有形实体,但更多的是指刚柔、进退、往来、动静、阖辟、寒暑、伸屈、尊卑、吉凶、贵贱、险易、大小、得失、远近、健顺等相对属性。“阴阳”往往与“气”连用,表明阴阳是两种无形的“气”。

阴与阳的关系主要有:阴阳互根、阴阳互动、阴阳互制、阴阳消息、阴阳交感、阴阳转化、阴阳争扰、阴阳胜复等。

\subsubsection{五行的内涵与关系}%(三)

“五行”的最初意义指“五材”,即木、火、土、金、水五种具体的、基本的物质材料。《尚书·洪范》首次将五行称为水、火、木、金、土,《左传》、《国语》常将“地之五行”与“天之三辰”、“天之六气”相并称。作为哲学概念,“五行”主要指“五性”,即润下、炎上、曲直、从革、稼穑五种基本功能属性,这是《尚书·洪范》首次规定的。后世对五行的解释基本上没有偏离《洪范》的这种属性规定。归纳五行的基本意义为:水,表示有润下、寒冷属性和功能的事物或现象;火,表示具有炎热、向上属性和功能的事物或现象;木,表示具有生发、条达、曲直属性和功能的事物或现象:金,表示具有清静、肃杀、从革属性和功能的事物或现象;土,表示具有生养、化育属性和功能的事物或现象。后又被用于表示“五伦”即仁、义、礼、智、信(圣)五种道德伦常,“五类”即木、火、土、金、水五种分类原则。

五行之间的关系主要有生克、乘侮、胜复、制化等。

\subsection{“气—阴阳—五行”的特性}%三、

“气—阴阳—五行”不仅是《内经》重要的概念范畴,而且是《内经》最基本的思维模式。

\subsubsection{“气—阴阳—五行”模型的特性}%(一)

1.功能性。“气—阴阳—五行”是中国古代认识宇宙生命现象的思维模型,表示的是关系实在、功能实在,而不是物质实体、形态实体。虽然“气”“阴阳”“五行?最早都表示特定的物质实体,但当它一旦成为一种思维模型,一旦成为一个哲学范畴,并被中医广泛运用时,它就不再是指有形态结构的物质、实体。如“气”已经从云、风、雾等有形可感的实物转变为无形的抽象概念。“气”原本有两种状态:一种是凝聚的、有形的状态,分散细小的气凝聚为看得见摸得着的实体;一种是弥散的、无形的状态,细小分散的气由于不停地运动弥散而看不见摸不着。有形的气习惯上称为“形”,无形的气习惯上称为“气”。“气”具有超形态性,气非形却是形之本。“阴阳”从单纯指背阴、向阳的实体转变为抽象的功能属性。“阴阳”上升为哲学概念以后,已不再单纯指背阴、向阳的实体,而是指两种相反、相对的功能属性:凡具有推动、温煦、兴奋、发散、上升的功能,则属于“阳”;凡具有静止、寒冷、抑制、凝聚、下降的功能,则属于“阴”。“五行”从五种实体的元素材料转变为五种基本功能属性。“气—阴阳—五行”作为一种模型,从物质实体转变为关系实在、功能实在。

2.互换性。“气—明阳—五行”是一个三级合一的思维模型,三者之间具有互换性。从气的角度看,阴阳是二气,五行是五气;从阴阳角度看,气是阴阳的未分状态,五行是阴阳的分化状态。气—阴阳—五行是一个逐渐生成和分化的过程,是三个不同的层次。气生阴阳,阴阳生五行。《周易·系辞传》说:“易有太极,是生两仪,两仪生四象,四象生八卦。”太极(气)生两仪(阴阳)为第一级划分,阴阳生四象(太阳、太阴、少阳、少阴)为第二级划分,四象生八卦为第三级划分。《内经》根据人体的实际情况对阴阳作了有限的划分,其中“三阴三阳”是中医的发明。从某种意义上说,五行也是阴阳所化生。

3.普遍性。“气—阴阳—五行”是一个具有普遍性的思维模型,万事万物都适用于这一模型,它们无处不在,无时不有。分而言之,“气”至大而无外、至小而无内,充满宇宙万物之中,《庄子·知北游》说:“通天下一气耳。”气不仅生成万物,而且充斥万物生长化收藏的整个过程当中,连贯而不间断。阴阳和五行作为两种与五种有关联的气同样也是普遍存在的,天地万物皆涵阴阳五行之气;同时阴阳五行作为特殊的分类方法,可以运用于世界万事万物。

\subsubsection{“气”“阴阳”“五行”概念的特性}%(二)

作为单个的医学哲学概念,“气”“阴阳”“五行”除了都具有功能性、超形态性、普遍性外,又有各自的特性。

1.“气”的运动性与渗透性

哲学意义上的“气”已与“形”分立,“形”是有形的、静态的,“气”则是无形的、动态的。“气”具有运动不息、变化不止、连续不断的特性。气的运动有升降出入等形式。气的运动称为气机,气机必然产生各种变化,从而化生天地万物,称为气化。气化学说经历了精气与元气两个发展阶段。气机与气化的关系:气机是气化的前提,气化是气机的结果。没有气的运动就没有气的化生,没有气的化生就没有世界万物的运动变化。气无形质而可以渗透、贯穿到一切有形质的事物之中,无处不入,无时不入;同时气又可以吸收其他事物的成分而组成各种各样的气,如阳气、阴气、天气、地气、风气、云气等。

2.“阴阳”的相对性

事物的阴阳属性是相对的,不是绝对的。具体表现为,一是阴阳要随着比较标准的改变而改变。阴阳是通过比较而确定的,单一方面无法定阴阳,没有比较标准也不能定阴阳,比较的标准不同,作出的阴阳判断也不同。如以0度水为标推,则-1度水是阴,1度水是阳;如以10度水为标准,则1度水为阴,11度水为阳。二是阴阳要随着关系的改变而改变。阴阳并不是实体,也不是事物所固有的本质,阴阳表示的是事物之间的关系。如在男与女这组关系中,男是阳,女是阴;而在父母与子女这组关系中,母(女)则为阳,子(男)则为阴。三是阴中有阳、阳中有阴。因为阴阳是层层可分的,阴阳中复有阴阳。如昼为阳、夜为阴;昼中上午为阳(阳中之阳)、下午为阴(阳中之阴),夜中前半夜为阴(阴中之阴)、后半夜为阳(阴中之阳)。

3.“五行”的时序性

五行常用来表示五类事物之间的排列次序和变化过程。《尚书·洪范》:“一曰水,二曰火,三曰木,四曰金,五曰土”,这种次序被后世用来说明事物发展的节律和周期。然而五行次序并不是固定的,在不同著作中往往有不同的次序,甚至同一部著作在不同篇章中也会出现不同的五行次序,如《管子》、《黄帝内经》等。不同的五行次序往往反映不同的宇宙发生观、事物运动周期观。五行的排列次序大多数书上并没有标明“一二三四五”的次序。从各种五行次序看,有的用的是相生次序,有的用的是相克次序,有的则混杂不一。五行的次序往往与社会历史、一年四季等配合,用来说明各自的循环周期、兴衰变化节律。

\subsection{“气—阴阳—五行”在建构中医学体系中的作用}%四、

“气—阴阳—五行”在《内经》中有的是哲学概念,有的是医学概念,更多的则是医学哲学相混合的概念。这些概念范畴在建构中医学体系中起到重要作用;“气—阴阳—五行”还是《内经》最重要、最基本的思维模型,这一模型被广泛运用于说明人体生命的生成与活动、人体生命的功能结构、病理变化、疾病的诊断与治疗。

\subsubsection{人体生命的生成与活动}%(一)

《内经》用“气”说明人体生命的本原和生成。如《素问·宝命全形论》说:“人生于地,悬命于天。天地合气,命之曰人。”从本体论层面说明“气”是人的总体来源。从个体生成层面上说,生命的直接来源是父母阴阳两精的结合,父母精血被称为先天之精气,如《灵枢·天年》“人之始生……以母为基,以父为楯。”人既生之后,其发育、成长、生存所需的物质能量,则要依靠水谷精微及大气,称之为后天之精气。《素问·六节藏象论》:“天食人以五气,地食人以五味。”《内经》认为正气、精气是生命活动的动力。人的五脏、六腑、形体、官窍、血、津液等生理功能活动,都必须在气的推动下进行,如肺司呼吸,脾主运化水谷精微,肝主疏泄气机等。

用阴阳思维模式说明人的生命活动,如《素问·六微旨大论》认为气化有升降出入四神形态,升降出入即是两对阴阳,“出入废则神机化灭,升降息则气立孤危。故非出入,则无以生长壮老已;非升降,则无以生长化收藏。是以升降出入,无器不有。”在正常情况下,气的“出”与“入”、“升”与“降”是相对的,相反而相成,是一种动态的有序的过程,从而保持了生命的正常、旺盛的活动。

用“气”说明人体生命的功能结构。《内经》将人体生命的功能结构看成是各种“气”的作用。人有各种各样的“气”,仅就人体部位功能而言就有:脏腑之气(包括五脏之气、六腑之气),经络之气(如十二经气或真气),俞穴之气(俞穴又称气穴、气门),形体之气(形体上中下之气、头身四肢之气、筋脉肌皮骨各部之气),特定聚散分布之气(元气、宗气、营气、卫气),等等。

用“阴阳”说明人体组织结构与生理功能。《内经》以“阴阳”分析概括人的组织结构、人体整体和局部的生理功能及其物质、属性。就功能与物质而言,则功能为阳,物质为阴。就精与气而言,則精为阴,气为阳。就营气与卫气而言,则营气为阴,卫气为阳。《内经》反复强调“生之本,本于阴阳”《素问·生气通天论》,并把机体的正常状态称为“阴平阳秘”、“阴阳匀平”,不是指阴阳的绝对平衡,而是强调人体生命运动过程是一个阴阳制约、阴阳消长的过程,阴阳双方要达到动态的平衡、动态的和谐。

用五行说明人体五脏的生理功能。五行与五脏的配属经过了一个从哲学到医学、从物质实体到功能实在的过程。《黄帝内经》采用五行—五脏的模式,将五行与五脏的功能属性作了规范和确定,以五行功能说明五脏的生理功能,从而打破了解剖学五脏的功能界线,上升为五大功能系统。五脏是一个中心,不仅将人体各种组织器官一一对应地联系在一起,而且将自然界的时间、空间、气味、色彩、味道等因素有机地联系起来,构成了一个天人相应、内外相通的功能网络。五脏的生理功能是依据五行的生克制化原理联系在一起的。五行—五脏之间的相生相克是双向的,正因为有这种双向联系,才使人体生理功能得以协调和正常。

\subsubsection{人体病理变化}%(二)

用气机失调说明人体病理变化。如精气不足,称为气虚;气的升降出入运动不能保持协调平衡,称为“气机失调”。升多降少,谓之气逆;升少降多,谓之气陷。气的运动受阻,运行不利,称作“气机不畅”;气的运动受阻严重并在某些局部淤滞不通,称为“气滞”;气的外出运动太过,称作“气脱”;气的出入运动不及而结聚于内,称作“气结”、“气郁”,严重者称为“气闭”。气机失调,表现在脏腑上可见:脾失宣降,胃气上逆,脾气下陷,肾不纳气,肝气郁结,等等。

用阴阳盛衰说明人体病理变化。疾病的发生发展与正气、邪气有关。正气分阴阳——阴气与阳气;邪气分阴阳——阴邪与阳邪。在六淫邪气中,寒、燥、湿为阴邪,风、暑、火(热)为阳邪。以阴阳偏盛(胜)、阴阳偏衰概括病理病机。

用五行生克乘侮说明五脏病理变化。疾病传变分为两类:一是相生关系的“母病及子”、“子病及母”,如“水不涵木”证、“心肝血虚”证。二是相克关系的“相乘”、“相侮”,如“木旺乘土”证、“土虚水侮”证。此外,五行理论还用来说明五脏的发病与季节的关系、五脏发病的规律与预后的规律等。

\subsubsection{疾病的诊断与治疗}%(三)

用气—阴阳说明疾病的诊断治疗。如以“阴阳”概括病变部位、性质及症状的属性,作为辨证的纲领。治疗疾病,就是调整失衡失调的阴阳,使之恢复到相对平衡的健康状态,故《素问·至真要大论》说:“谨察阴阳所在而调之,以平为期。”阴阳学说还可用于分析归纳药物的性味,指导养生健体、预防疾病等各个方面。

用五行说明疾病的诊断治疗。如根据五色之间以及色脉之间的生克关系,推断病情的轻重及疾病的预后。用于治疗,主要判断一脏受病涉及另一脏;依五行生克乘侮规律作出相应的调整,以控制其传变;根据五行相生原理确定虚则补其母、实则泻其子的治则和滋水涵木、益火补土等治法,根据五行相克原理确定抑强、扶弱的治则和抑木扶土、培土制水、佐金平木、泻南补北等治法。

\section{思维方法论}%第三节

方法论是关于认识世界和改造世界根本方法的理论,有哲学科学方法论、一般科学方法论、具体科学方法论之分。思维方法论属于哲学科学方法论,指在思维过程中复制和再现研究对象或现象的各种方式。具体地说就是思维主体按照自身的特定需要与目的,运用思维工具去接受、反映、理解、加工客观对象或客体信息的思维活动的方式或模式,本质上反映思维主体、思维对象、思维工具三者关系的一种稳定的、定型的思维结构。传统思维方式是传统文化的最高凝聚和内核。中医思维方法论即中医研究人的生命活动、形成医学理论体系以及应用于临床实践的根本方法。

在中国传统哲学思想的深刻影响下,中医学形成了不同于西医学的思维方式。这一独特的思维方式主要表现为整体思维、意象思维、变易思维。

\subsection{整体思维}%一、

\subsubsection{整体思维的涵义}%(一)

所谓整体思维,就是以普遍联系、相互制约的观点看待世界及一切事物的思维方式。这种思维方式不仅把整个世界视为一个大的有机整体,世界的一切事物都是连续的、不可割裂的,事物和事物之间具有相互联系、相互制约的关系,而且把每一个事物的各部分又各自视为一个小的有机整体,部分作为整体的构成要素,其本身也是一个连续、不可割裂的整体,部分与部分呈现出多种因素、多种部件的普遍联系。认为天与人之间、事物与事物之间同源、同构、同序、同律。

中国传统哲学,不论是儒家还是道家,都强调整体思维。《周易》一书提出了整体论的初步图式。例如,从全书结构形式上看,《周易》最基本的单位是阴爻(-{-})和阳爻(—)。阴爻和阳爻三次组合构成八卦,六次组合构成六十四卦。一切自然现象和人事吉凶都纳入八卦、六十四卦系统中,表现出一种整体观念。《易传》以普通联系、相互制约的观点解释《易经》。六十四卦六爻同时具有下中上、初中末、天地人之义,是一个天人时空统一的整体系统。道家代表人物庄子讲“天地与我并生,而万物与我为一。”(《庄子·齐物论》)天地是一个整体,人与世界是一个整体。任何一个局部都体现着全体,比如庄子认为“道”无所不在,甚至在“蝼蚁”、“秭稗”中。

“天人合一”是这种整体思维的根本特点。所谓天人合一,是指天道与人道、自然与人相通、相类和统一。在传统思维中,儒道两家都主张“天人合一”。道家倾向于把人自然化,儒家倾向于把自然人化,但他们都认为,人和自然界是一气相通、一理相通的。老子的道天地人“四大”,《易传》的天地人“三才”,正是这种整体思维的早期表现。而董仲舒以阴阳五行为框架的天人感应论,则提出了更加完备的整体模式。中国传统哲学的这一思维倾向,直接孕育了《内经》的整体性思维方式。当然,在长期的医学实践中,《内经》又将传统哲学的整体性思维具体化、科学化。

\subsubsection{《内经》中的整体思维}%(二)

在整体思维指导下,《内经》建构了一个三才合一的整体医学模式,如《素问·阴阳应象大论》说:“其在天为玄,在人为道,在地为化。化生五味,道生智,玄生神。”并以三才为经,五行为纬,论述天、地、人诸事物的类属及其相互关系。《内经》的整体思维主要体现在以下两方面。

1.人体本身是一个有机联系的整体。《内经》将人体本身看成一个有机联系的整体,人体内部部分与部分之间既是连续的、不可割裂的,又是互相制约、互为作用的。《内经》将人体生命活动整体系统各部分、各要素(子系统)的有机联系归结为阴阳对立统一、五行生克制化、气机升降出入三种模式。用阴阳模式说明人体生命活动由相互联系、相互对立、相互制约、相互转化的两大类生理机能结构组成;用五行模式说明人体五脏功能活动是多级多路反馈联系的有机系统;用气机升降出入模式说明人不但与自然界交换物质、能量、信息,而且人体内部物质、能量、信息也是运动转化的。

《内经》认为在人生命活动中,人的生理、心理、躯体三者是有机联系的,即生命能力与躯体形骸之间、精神心理与躯体生理之间有着密切关系,提出了“形神一体”和“心身一体”的观念。在形态结构上,中医学认为人以五脏为中心,通过经络系统把六腑、五体、五官、九窍、四肢百骸等全身组织器官组合成一有机的整体,并通过精、气、血、津液的作用,完成机体统一的机能活动。在生理功能上,中医学认为人体的各个脏腑器官都是互相协调活动的,任何一个脏腑、器官、组织的活动都是整体机能活动不可分割的一部分,每个器官、组织在这个整体中既分工不同,又密切配合。在人体这个系统中,脏腑经络、形体官窍、精气神等要素之间具有相互作用的整体调控规律,在每一脏腑经络、形体官窍的子系统中又有更小的子系统,又各有阴阳、气血。在病理变化上,中医着眼于分析局部病变所反映的整体病理状态,局部病变对其他部分、对整体的影响,注重对人天系统、人体内五脏经络系统、五脏经络内各子系统等各级系统进行调控,以抑制其病理变化。在疾病诊断上,由于各脏腑、组织、器官在生理、病理上的相互联系和相互影响,决定了中医在诊断疾病时,可以通过观察分析五官、形体、色脉等的外在病理表现,借以分析、揣测内在脏腑的病变情况,从而对患者作出正确的判断,并进行治疗。《内经》中有关脉诊、目诊、面诊等全息诊法记载,正是整体思维的反映。在疾病治疗上,既注意脏、腑、形、窍之间的联系,也注意五脏系统之间的联系。在养生保健上,也体现整体观念,如在养生动静关系上强调要动中寓静、动静结合、动而中节。

2.人与外界环境构成一个有机的整体。《内经》不仅认为人体本身是一个有机整体,而且认为人与天也是一个有机整体。《内经》有“生气通天”的论断,认为“人与天地相参也,与日月相应也。”(《灵枢·岁露》)强调人与外界环境的密切联系,从人与自然环境、社会环境的整体联系中考察人体生理、心理、病理过程,研究人体开放系统与周围环境交换物质、信息、能量以及随宇宙节律进行新陈代谢活动的规律,并提出相应的治疗养生方法。

人生活于自然环境之中,当自然环境发生变化时,人体也会发生与之相应的变化。《内经》根据五行学说,把一年分为五季,认为春温、夏热、长夏湿、秋燥、冬寒就是一年四时中气候变化的一般规律。在四时气候的规律性变化影响下,人也表现出春生、夏长、长夏化、秋收、冬藏等相应的生理变化过程。一日昼夜昏晨自然界阴阳的消长也对人产生一定的作用。《灵枢·顺气一日分为四时》说:“以一日分为四时,朝则为春,日中为夏,日入为秋,夜半为冬。”人体的机能活动产生与昼夜节律变化相似的变化以适应环境的改变。如《素问·生气通天论》说:“故阳气者,一日而主外,平旦人气生,日中而阳气隆,日西而阳气已虚,气门乃闭。”地理区域是自然环境中的一个重要因素。在不同地区,由于气候、土质和水质不同,也可在一定程度上影响人们的生理机能和心理活动。如江南地区地势低平,多湿热,故人体的腠理多疏松,体格多瘦削;西北地区地势高而多山,多燥寒,故人体的腠理多致密,体格偏壮实。生活在已经习惯的环境中,一旦易地而居,许多人初期都感到不太适应,但经过一定的时间,大多数人是能够逐渐适应的。

人是社会的人,社会环境同样会影响人的机能活动,关乎人体的健康与疾病。《内经》指出:“故贵脱势,虽不中邪,精神内伤,身必败亡。始富后贫,虽不伤那,皮焦筋屈,痿躄为挛。”(《素问·疏五过论》)说明社会环境的剧烈变动对人的心身机能的巨大影响。《内经》强调人因社会经济、政治地位不同,在体质方面也存在一定的差异,因此在疾病治疗时要因人而异。

总之,《内经》整体思维是一种有机论思维,它与西方的整体思维有所不同。《内经》强调人体的整体功能,把现实事物看成是一个自组织的有机系统,整体不可以还原为部分。西方的整体观是机械决定论的,它注重实体和元素,把现实事物看做是无数的细小部分组成的复合体,整体可以还原为部分。《内经》有机整体性思维具有西方精密的还原分析思维所不可及的视野,能够发现用分解方法所不能及的客体的一些属性和特点。但是,我们应当清醒地看到,中医学的整体思维虽然强调对人体、人与自然社会的整体性、统一性的认识,却缺乏对这一整体各个细节的精确性认识,因而对整体性和统一性的认识也是不完备的。这种思维虽然缺少片面性,但它的不片面是建立在模糊直观的基础之上,中医理论整体观带有原始的、朴素的、直觉的、想象的成分。这是我们在把握中医学整体思维时应当注意的。《内经》整体思维与现代系统思维有相同之处但不能等同。《内经》整体思维是系统思维的原始形态,具备了系统思维的基本特征,在一定意义上两者是一脉相承的。但我们应该看到两者之间存在较大的不同之处。现代系统论作为严格意义上的科学方法论是20世纪以来人类科学研究的成果,是在科学技术高度发展的基础上产生的。《内经》整体论与现代系统论并不在同一层次上,因此应积极吸取现代系统论的新思路、新方法,使中医学整体论跃上新的层次。

\subsection{意象思雏}%二、

\subsubsection{意象思维的涵义}%(一)

所谓意象思维,就是指运用带有感性、形象、直观的概念、符号表达对象世界的抽象意义,通过类比、象征方式把握对象世界联系的思维方式。意象思维作为中国传统思维方式的重要内容之一,与西方人重抽象思维的倾向形成反差。

中华民族的意象思维在古代得到特别的发展而早熟,《周易·系辞传》说:“易者,象也。象也者,像也。”“夫象,圣人有以见天下之赜,而拟诸其形容,象其物宜,是故谓之象。”“见乃谓之象。”“象”字有三重涵义:一指事物可以感知的现象,包括肉眼可以看见的物象和虽肉眼无法看见但可以感知的物象;二指摹拟的象征性符号,如卦象、爻象;三指取象、象征,为动词意。“意”是“象”所象征的事物蕴涵的特性和规律。《周易·系辞传》说:“立象以尽意,设卦以尽情伪。”所谓“意象”就是经过人为抽象、体悟而提炼出来的带有感性形象的概念或意义符号。就“象”与“意”的关系而言,意为象之本,象为意之用;象从意,意主象。意象思维的含义在于:一方面它通过形象性的概念与符号去理解对象世界的抽象意义,另一方面它又通过带有直观性的类比推理形式去把握和认识对象世界的联系。传统哲学的意象思维渗透到《内经》中,成为中医学思维方式的主要内容之一。

意象思维主要体现在取象比类的思维方法之中。取象思维就是在思维过程中以“象”为工具,以认识、领悟、模拟客体为目的的方法。取“象”是为了归类或比类,即根据被研究对象与已知对象在某些方面的相似或相同,推导其他方面也有可能相似或类同。取象的范围不是局限于具体事物的物象、事象,而是在功能关系、动态属性相同的前提下可以无限地类推、类比。

\subsubsection{《内经》中的意象思维}%(二)

《内经》的意象思维主要体现在以下方面:

1.运用取象比类法建构藏象理论。对于藏象理论的形成,《素问·五脏生成论》提出“五脏之象,可以类推”的原则,王冰注释:“象,谓气象也。言五脏虽隐而不见,然其气象性用,犹可以物类推之。”张介宾说:“象,形象也。藏居于内,形见于外,故曰藏象”。(《类经·藏象类》)裉据五行之象,《素问·金匮真言论》从直观经验入手,按照功能行为的相同或相似归为同类的原则,将自然界和人体分为五类,然后发掘出蕴涵于“象”中的深层的藏象理论。

2.运用取数比类法说明生理病理现象。取数比类是以易数表示抽象意义,并通过易数推演事物变化规律的方法。易数主要有卦爻数、千支数、五行生成数(河图数)和九宫数(洛书数)。《素问·金匮真言论》用五、六、七、八、九说明“五脏应四时,各有收受”的整体联系。《素问·六元正纪大论》以“太过者其数成,不及者其数生,土常以生也”及数的生克胜复之理阐释五运六气的常变规律。《素问》“运气七篇”用的是干支之数,通过取数比类推测六十年气候的变化规律及其与人体疾病的关系。《素问·上古天真论》人体发育与生殖基数的女七男八,即阴阳进退之数;《素问·金匮真言论》五脏四时应数,即五行生成之数。

值得注意的是,取象比类作为人类把握对象世界的一种方式,历来就具有很重要的认识论价值和科学价值。通过类比,可以启迪人的思维,帮助人们打开想象的翅膀,由此推彼,触类旁通,去认识和发现新的事物。医家们在医学实践中运用这一思维方法,发明了不少新的诊疗方法。但是,取象比类这一思维方法的缺陷也很明显,那就是过于注重事物或现象的共性、共同点和相似点,忽视了不同事物的特性和不同点。如果所推导出的属性恰好是它们的不同点,那么得出的结论就必然是错误的。

\subsection{变易思维}%三、

\subsubsection{变易思维的涵义}%(一)

所谓变易思维是以运动变化观点考察一切事物的思维方式。变易思维强调事物的运动变化,注重在两极对立中把握事物的辩证统一,因而具有辩证法的特征。变易思维从属于辩证思维。辩证思维包含整体思维、相成思维、变易思维。相成思维把任何事物都看成是相互对立、相互依存、相互转化、相互包含的两个方面的统一体。可以说相成思维是变易思维的基础,变易思维强调对立两面的相互作用推动事物的发生发展变化。

在中国哲学史上,变易思维的产生源远流长。首先,道家学说中含有丰富的辩证法思想。道家的创始者老子认为,“道生一,一生二,二生三,三生万物,万物负阴而抱阳,冲气以为和。”(《老子》四十二章)说明“道”作为宇宙的本源,其内部总是包含着阴阳对立的两种势力,正是这两种对立力量的推动,产生了万事万物,由“一”到“二”到“三”到“万物”的过程正是道化万物的过程。老子还提出“反者道之动”的著名命题,说明事物的发展是一个向其对立面转化的过程。庄子也强调事物的变化,将事物生杀盛衰之化视为一个具有连续性、整体性的变动不息的洪流。儒家也把宇宙看成是变动不居的过程。孔子曾说:“四时行焉,百物生焉。”(《论语·阳货》),把自然界的变化看成是一个如江河之水流动的连续过程:“子在川上曰:逝者如斯夫,不舍昼夜。”(《论语·子罕》)。提出“叩其两端而竭焉”(《论语·于罕》),强调要考察问题的两个方面。作为先秦哲学的集大成者,《易传》表现出更为明显的变易思维特征。如果说《易经》本身就是一部研究“变易”的著作,那么,作为《易经》解释之作的《易传》更是明确把宇宙规定为一个运动变化的大过程。《周易·系辞传》曰:“易之为书也不可远,为道也屡迁,变动不居,周流六虚,上下无常,刚柔相易,不可为典要,唯变所适。”

“一阴一阳之谓道。”“刚柔相推而生变化。”认为宇宙的本性就是变动不居的,天地万物均处于运动变化的状态,相反相成,相反的双方、对立的两面(阴与阳,刚与柔)是事物变化的根本原因。这一思想直接影响了《黄帝内经》医学体系的形成。

\subsubsection{《内经》中的变易思维}%(二)

l.运用变易思维说明人体生命运动变化过程。《内经》认为人体生命是一个生长壮老已的运动变化过程,脏腑经络气血具有升降出入运动机制与规律。《素问·玉版论要》说:“道之至数……神转不回,回则不转,乃失其机”,认为有序的运动变化是生命存在的基本形式。《内经》还认为,要维持人体生命的正常运转,关键是保持动态平衡,具体表现为阴阳对立统一,气血相辅相成,气机升降出入,以及营卫循环不止。

2.运用变易思维说明病理现象及其变化和发展的规律。《内经》从致病因素与抗病能力双方的对立斗争与胜负关系论述疾病发生的机理,认为导致疾病发生的双方又互为消长,具有相对的性质,提出“生病起于过用”(《素问·经脉别论》)的发病观。又如《内经》从人体各层次机能的紊乱失调以至于衰竭分离认识病理的变化,认为人体阴阳的和谐平衡被破坏即阴阳失调、气机升降出入逆乱,则会致病。从气的升降出入来看,《素问·六微旨大论》说:“出入废则神机化灭,升降息则气立孤危。故非出入,则无以生长壮老已;非升降,则无以生长化收藏。是以升降出入,无器不有。”说明没有升降出入,便没有生长壮老已的生命过程,也没有自然界生长化收藏的生化过程,升降出入的反常会导致疾病的发生。

3.运用变易思维指导中医临床诊断和治疗。《内经》认为诊断施治必须先审阴阳,要依据阴阳、寒热、虚实、表里、气血、水火、标本等对立统一关系的不同性质及其在一定条件下相互转化的规律确立不同病症的证治原则。就治疗而言,《内经》依据本质与现象的辩证关系,提出了治病求本、补泻调整、因势利导等治疗法则和病治从逆、病治异同等具体治法,强调法随病变。《内经》还提出病有标本、治有先后缓急的治则和因势利导、补虚泻实的治法。

4.运用变易思维指导疾病的预防与养生。《内经》认为,疾病应以预防为主,人体如能阴阳协调,并与天地间阴阳变化相协调,就可以做到防患病于未然,达到延年益寿的目的,故《素问·生气通天论》曰:“是以圣人陈阴阳,经脉和同,骨髓坚固,气血皆从。如是则内外调和,邪不能害,耳目聪明,气立如故。”

综上所述,《内经》在研究人体生理、病理和疾病诊治过程中,大量运用了变易思维的原则,使主观认识符合生命运动的客观变易过程,从变易思维本身的性质来看,它与现代系统科学的某些原则非常接近,都强调从组成事物整体的各个要素间的相互联系、相互作用上理解事物的本质及其发展规律,都强调事物的变化是一个“反复”的转化过程。但是,《内经》变易思维对组成事物整体的各要素间的相互联系和相互作用缺乏明确的科学根据,因而,只能说它是朴素的系统变易观,它对事物整体的认识也是笼统的、模糊的。往往将事物的变化看成是一个循环反复的封闭性过程,而不是不断向更高层次发展的开放性的变化过程。变易思维既有崇尚变易、穷则思变的积极的一面,也有因循保守、安于现状的消极的一面。

\zuozhe{(张其成)}
\ifx \allfiles \undefined
\end{document}
\fi