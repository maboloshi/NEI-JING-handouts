% -*- coding: utf-8 -*-
%!TEX program = xelatex
\ifx \allfiles \undefined
\documentclass[draft,12pt]{ctexbook}
%\usepackage{xeCJK}
%\usepackage[14pt]{extsizes} %支持8,9,10,11,12,14,17,20pt

%===================文档页面设置====================
%---------------------印刷版尺寸--------------------
%\usepackage[a4paper,hmargin={2.3cm,1.7cm},vmargin=2.3cm,driver=xetex]{geometry}
%--------------------电子版------------------------
\usepackage[a4paper,margin=2cm,driver=xetex]{geometry}
%\usepackage[paperwidth=9.2cm, paperheight=12.4cm, width=9cm, height=12cm,top=0.2cm,
%            bottom=0.4cm,left=0.2cm,right=0.2cm,foot=0cm, nohead,nofoot,driver=xetex]{geometry}

%===================自定义颜色=====================
\usepackage{xcolor}
  \definecolor{mybackgroundcolor}{cmyk}{0.03,0.03,0.18,0}
  \definecolor{myblue}{rgb}{0,0.2,0.6}

%====================字体设置======================
%--------------------中文字体----------------------
%-----------------------xeCJK下设置中文字体------------------------------%
\setCJKfamilyfont{song}{SimSun}                             %宋体 song
\newcommand{\song}{\CJKfamily{song}}                        % 宋体   (Windows自带simsun.ttf)
\setCJKfamilyfont{xs}{NSimSun}                              %新宋体 xs
\newcommand{\xs}{\CJKfamily{xs}}
\setCJKfamilyfont{fs}{FangSong_GB2312}                      %仿宋2312 fs
\newcommand{\fs}{\CJKfamily{fs}}                            %仿宋体 (Windows自带simfs.ttf)
\setCJKfamilyfont{kai}{KaiTi_GB2312}                        %楷体2312  kai
\newcommand{\kai}{\CJKfamily{kai}}
\setCJKfamilyfont{yh}{Microsoft YaHei}                    %微软雅黑 yh
\newcommand{\yh}{\CJKfamily{yh}}
\setCJKfamilyfont{hei}{SimHei}                                    %黑体  hei
\newcommand{\hei}{\CJKfamily{hei}}                          % 黑体   (Windows自带simhei.ttf)
\setCJKfamilyfont{msunicode}{Arial Unicode MS}            %Arial Unicode MS: msunicode
\newcommand{\msunicode}{\CJKfamily{msunicode}}
\setCJKfamilyfont{li}{LiSu}                                            %隶书  li
\newcommand{\li}{\CJKfamily{li}}
\setCJKfamilyfont{yy}{YouYuan}                             %幼圆  yy
\newcommand{\yy}{\CJKfamily{yy}}
\setCJKfamilyfont{xm}{MingLiU}                                        %细明体  xm
\newcommand{\xm}{\CJKfamily{xm}}
\setCJKfamilyfont{xxm}{PMingLiU}                             %新细明体  xxm
\newcommand{\xxm}{\CJKfamily{xxm}}

\setCJKfamilyfont{hwsong}{STSong}                            %华文宋体  hwsong
\newcommand{\hwsong}{\CJKfamily{hwsong}}
\setCJKfamilyfont{hwzs}{STZhongsong}                        %华文中宋  hwzs
\newcommand{\hwzs}{\CJKfamily{hwzs}}
\setCJKfamilyfont{hwfs}{STFangsong}                            %华文仿宋  hwfs
\newcommand{\hwfs}{\CJKfamily{hwfs}}
\setCJKfamilyfont{hwxh}{STXihei}                                %华文细黑  hwxh
\newcommand{\hwxh}{\CJKfamily{hwxh}}
\setCJKfamilyfont{hwl}{STLiti}                                        %华文隶书  hwl
\newcommand{\hwl}{\CJKfamily{hwl}}
\setCJKfamilyfont{hwxw}{STXinwei}                                %华文新魏  hwxw
\newcommand{\hwxw}{\CJKfamily{hwxw}}
\setCJKfamilyfont{hwk}{STKaiti}                                    %华文楷体  hwk
\newcommand{\hwk}{\CJKfamily{hwk}}
\setCJKfamilyfont{hwxk}{STXingkai}                            %华文行楷  hwxk
\newcommand{\hwxk}{\CJKfamily{hwxk}}
\setCJKfamilyfont{hwcy}{STCaiyun}                                 %华文彩云 hwcy
\newcommand{\hwcy}{\CJKfamily{hwcy}}
\setCJKfamilyfont{hwhp}{STHupo}                                 %华文琥珀   hwhp
\newcommand{\hwhp}{\CJKfamily{hwhp}}

\setCJKfamilyfont{fzsong}{Simsun (Founder Extended)}     %方正宋体超大字符集   fzsong
\newcommand{\fzsong}{\CJKfamily{fzsong}}
\setCJKfamilyfont{fzyao}{FZYaoTi}                                    %方正姚体  fzy
\newcommand{\fzyao}{\CJKfamily{fzyao}}
\setCJKfamilyfont{fzshu}{FZShuTi}                                    %方正舒体 fzshu
\newcommand{\fzshu}{\CJKfamily{fzshu}}

\setCJKfamilyfont{asong}{Adobe Song Std}                        %Adobe 宋体  asong
\newcommand{\asong}{\CJKfamily{asong}}
\setCJKfamilyfont{ahei}{Adobe Heiti Std}                            %Adobe 黑体  ahei
\newcommand{\ahei}{\CJKfamily{ahei}}
\setCJKfamilyfont{akai}{Adobe Kaiti Std}                            %Adobe 楷体  akai
\newcommand{\akai}{\CJKfamily{akai}}

%------------------------------设置字体大小------------------------%
\newcommand{\chuhao}{\fontsize{42pt}{\baselineskip}\selectfont}     %初号
\newcommand{\xiaochuhao}{\fontsize{36pt}{\baselineskip}\selectfont} %小初号
\newcommand{\yihao}{\fontsize{28pt}{\baselineskip}\selectfont}      %一号
\newcommand{\xiaoyihao}{\fontsize{24pt}{\baselineskip}\selectfont}
\newcommand{\erhao}{\fontsize{21pt}{\baselineskip}\selectfont}      %二号
\newcommand{\xiaoerhao}{\fontsize{18pt}{\baselineskip}\selectfont}  %小二号
\newcommand{\sanhao}{\fontsize{15.75pt}{\baselineskip}\selectfont}  %三号
\newcommand{\sihao}{\fontsize{14pt}{\baselineskip}\selectfont}%     四号
\newcommand{\xiaosihao}{\fontsize{12pt}{\baselineskip}\selectfont}  %小四号
\newcommand{\wuhao}{\fontsize{10.5pt}{\baselineskip}\selectfont}    %五号
\newcommand{\xiaowuhao}{\fontsize{9pt}{\baselineskip}\selectfont}   %小五号
\newcommand{\liuhao}{\fontsize{7.875pt}{\baselineskip}\selectfont}  %六号
\newcommand{\qihao}{\fontsize{5.25pt}{\baselineskip}\selectfont}    %七号   %中文字体及字号设置
\xeCJKDeclareSubCJKBlock{SIP}{
  "20000 -> "2A6DF,   % CJK Unified Ideographs Extension B
  "2A700 -> "2B73F,   % CJK Unified Ideographs Extension C
  "2B740 -> "2B81F    % CJK Unified Ideographs Extension D
}
%\setCJKmainfont[SIP={[AutoFakeBold=1.8,Color=red]Sun-ExtB},BoldFont=黑体]{宋体}    % 衬线字体 缺省中文字体

\setCJKmainfont{simsun.ttc}[
  Path=fonts/,
  SIP={[Path=fonts/,AutoFakeBold=1.8,Color=red]simsunb.ttf},
  BoldFont=simhei.ttf
]

%SimSun-ExtB
%Sun-ExtB
%AutoFakeBold:自动伪粗,即正文使用\bfseries时生僻字使用伪粗体;
%FakeBold:强制伪粗,即正文中生僻字均使用伪粗体
%\setCJKmainfont[BoldFont=STHeiti,ItalicFont=STKaiti]{STSong}
%\setCJKsansfont{微软雅黑}黑体
%\setCJKsansfont[BoldFont=STHeiti]{STXihei} %serif是有衬线字体sans serif 无衬线字体
%\setCJKmonofont{STFangsong}    %中文等宽字体

%--------------------英文字体----------------------
\setmainfont{simsun.ttc}[
  Path=fonts/,
  BoldFont=simhei.ttf
]
%\setmainfont[BoldFont=黑体]{宋体}  %缺省英文字体
%\setsansfont
%\setmonofont

%===================目录分栏设置====================
\usepackage[toc,lof,lot]{multitoc}    % 目录(含目录、表格目录、插图目录)分栏设置
  %\renewcommand*{\multicolumntoc}{3} % toc分栏数设置,默认为两栏(\multicolumnlof,\multicolumnlot)
  %\setlength{\columnsep}{1.5cm}      % 调整分栏间距
  \setlength{\columnseprule}{0.2pt}   % 调整分栏竖线的宽度

%==================章节格式设置====================
\setcounter{secnumdepth}{3} % 章节等编号深度 3:子子节\subsubsection
\setcounter{tocdepth}{2}    % 目录显示等度 2:子节

\xeCJKsetup{%
  CJKecglue=\hspace{0.15em},      % 调整中英(含数字)间的字间距
  %CJKmath=true,                  % 在数学环境中直接输出汉字(不需要\text{})
  AllowBreakBetweenPuncts=true,   % 允许标点中间断行,减少文字行溢出
}

\ctexset{%
  part={
    name={,篇},
    number=\SZX{part},
    format={\chuhao\bfseries\centering},
    nameformat={},titleformat={}
  },
  section={
    number={\chinese{section}},
    name={第,节}
  },
  subsection={
    number={\chinese{subsection}、},
    aftername={\hspace{-0.01em}}
  },
  subsubsection={
    number={(\chinese{subsubsection})},
    aftername={\hspace {-0.01em}},
    beforeskip={1.3ex minus .8ex},
    afterskip={1ex minus .6ex},
    indent={\parindent}
  },
  paragraph={
    beforeskip=.1\baselineskip,
    indent={\parindent}
  }
}

\newcommand*\SZX[1]{%
  \ifcase\value{#1}%
    \or 上%
    \or 中%
    \or 下%
  \fi
}

%====================页眉设置======================
\usepackage{titleps}%或者\usepackage{titlesec},titlesec包含titleps
\newpagestyle{special}[\small\sffamily]{
  %\setheadrule{.1pt}
  \headrule
  \sethead[\usepage][][\chaptertitle]
  {\chaptertitle}{}{\usepage}
}

\newpagestyle{main}[\small\sffamily]{
  \headrule
  %\sethead[\usepage][][第\thechapter 章\quad\chaptertitle]
%  {\thesection\quad\sectiontitle}{}{\usepage}}
  \sethead[\usepage][][第\chinese{chapter}章\quad\chaptertitle]
  {第\chinese{section}节\quad\sectiontitle}{}{\usepage}
}

\newpagestyle{main2}[\small\sffamily]{
  \headrule
  \sethead[\usepage][][第\chinese{chapter}章\quad\chaptertitle]
  {第\chinese{section}節\quad\sectiontitle}{}{\usepage}
}

%================ PDF 书签设置=====================
\usepackage{bookmark}[
  depth=2,        % 书签深度 2:子节
  open,           % 默认展开书签
  openlevel=2,    % 展开书签深度 2:子节
  numbered,       % 显示编号
  atend,
]
  % 相比hyperref,bookmark宏包大多数时候只需要编译一次,
  % 而且书签的颜色和字体也可以定制。
  % 比hyperref 更专业 (自动加载hyperref)

%\bookmarksetup{italic,bold,color=blue} % 书签字体斜体/粗体/颜色设置

%------------重置每篇章计数器,必须在hyperref/bookmark之后------------
\makeatletter
  \@addtoreset{chapter}{part}
\makeatother

%------------hyperref 超链接设置------------------------
\hypersetup{%
  pdfencoding=auto,   % 解决新版ctex,引起hyperref UTF-16预警
  colorlinks=true,    % 注释掉此项则交叉引用为彩色边框true/false
  pdfborder=001,      % 注释掉此项则交叉引用为彩色边框
  citecolor=teal,
  linkcolor=myblue,
  urlcolor=black,
  %psdextra,          % 配合使用bookmark宏包,可以直接在pdf 书签中显示数学公式
}

%------------PDF 属性设置------------------------------
\hypersetup{%
  pdfkeywords={黄帝内经,内经,内经讲义,21世纪课程教材},    % 关键词
  %pdfsubject={latex},        % 主题
  pdfauthor={主编:王洪图},   % 作者
  pdftitle={内经讲义},        % 标题
  %pdfcreator={texlive2011}   % pdf创建器
}

%------------PDF 加密----------------------------------
%仅适用于xelatex引擎 基于xdvipdfmx
%\special{pdf:encrypt ownerpw (abc) userpw (xyz) length 128 perm 2052}

%仅适用于pdflatex引擎
%\usepackage[owner=Donald,user=Knuth,print=false]{pdfcrypt}

%其他可使用第三方工具 如:pdftk
%pdftk inputfile.pdf output outputfile.pdf encrypt_128bit owner_pw yourownerpw user_pw youruserpw

%=============自定义环境、列表及列表设置================
% 标题
\def\biaoti#1{\vspace{1.7ex plus 3ex minus .2ex}{\bfseries #1}}%\noindent\hei
% 小标题
\def\xiaobt#1{{\bfseries #1}}
% 小结
\def\xiaojie {\vspace{1.8ex plus .3ex minus .3ex}\centerline{\large\bfseries 小\ \ 结}\vspace{.1\baselineskip}}
% 作者
\def\zuozhe#1{\rightline{\bfseries #1}}

\newcounter{yuanwen}    % 新计数器 yuanwen
\newcounter{jiaozhu}    % 新计数器 jiaozhu

\newenvironment{yuanwen}[2][【原文】]{%
  %\biaoti{#1}\par
  \stepcounter{yuanwen}   % 计数器 yuanwen+1
  \bfseries #2}
  {}

\usepackage{enumitem}
\newenvironment{jiaozhu}[1][【校注】]{%
  %\biaoti{#1}\par
  \stepcounter{jiaozhu}   % 计数器 jiaozhu+1
  \begin{enumerate}[%
    label=\mylabel{\arabic*}{\circledctr*},before=\small,fullwidth,%
    itemindent=\parindent,listparindent=\parindent,%labelsep=-1pt,%labelwidth=0em,
    itemsep=0pt,topsep=0pt,partopsep=0pt,parsep=0pt
  ]}
  {\end{enumerate}}

%===================注解与原文相互跳转====================
%----------------第1部分 设置相互跳转锚点-----------------
\makeatletter
  \protected\def\mylabel#1#2{% 注解-->原文
    \hyperlink{back:\theyuanwen:#1}{\Hy@raisedlink{\hypertarget{\thejiaozhu:#1}{}}#2}}

  \protected\def\myref#1#2{% 原文-->注解
    \hyperlink{\theyuanwen:#1}{\Hy@raisedlink{\hypertarget{back:\theyuanwen:#1}{}}#2}}
  %此处\theyuanwen:#1实际指thejiaozhu:#1,只是\thejiaozhu计数器还没更新,故使用\theyuanwen计数器代替
\makeatother

\protected\def\myjzref#1{% 脚注中的引用(引用到原文)
  \hyperlink{\theyuanwen:#1}{\circlednum{#1}}}

\def\sb#1{\myref{#1}{\textsuperscript{\circlednum{#1}}}}    % 带圈数字上标

%----------------第2部分 调整锚点垂直距离-----------------
\def\HyperRaiseLinkDefault{.8\baselineskip} %调整锚点垂直距离
%\let\oldhypertarget\hypertarget
%\makeatletter
%  \def\hypertarget#1#2{\Hy@raisedlink{\oldhypertarget{#1}{#2}}}
%\makeatother

%====================带圈数字列表标头====================
\newfontfamily\circledfont[Path = fonts/]{meiryo.ttc}  % 日文字体,明瞭体
%\newfontfamily\circledfont{Meiryo}  % 日文字体,明瞭体

\protected\def\circlednum#1{{\makexeCJKinactive\circledfont\textcircled{#1}}}

\newcommand*\circledctr[1]{%
  \expandafter\circlednum\expandafter{\number\value{#1}}}
\AddEnumerateCounter*\circledctr\circlednum{1}

% 参考自:http://bbs.ctex.org/forum.php?mod=redirect&goto=findpost&ptid=78709&pid=460496&fromuid=40353

%======================插图/tikz图========================
\usepackage{graphicx,subcaption,wrapfig}    % 图,subcaption含子图功能代替subfig,图文混排
  \graphicspath{{img/}}                     % 设置图片文件路径

\def\pgfsysdriver{pgfsys-xetex.def}         % 设置tikz的驱动引擎
\usepackage{tikz}
  \usetikzlibrary{calc,decorations.text,arrows,positioning}

%---------设置tikz图片默认格式(字号、行间距、单元格高度)-------
\let\oldtikzpicture\tikzpicture
\renewcommand{\tikzpicture}{%
  \small
  \renewcommand{\baselinestretch}{0.2}
  \linespread{0.2}
  \oldtikzpicture
}

%=========================表格相关===============================
\usepackage{%
  multirow,                   % 单元格纵向合并
  array,makecell,longtable,   % 表格功能加强,tabu的依赖
  tabu-last-fix,              % "强大的表格工具" 本地修复版
  diagbox,                    % 表头斜线
  threeparttable,             % 表格内脚注(需打补丁支持tabu,longtabu)
}

%----------给threeparttable打补丁用于tabu,longtabu--------------
%解决方案来自:http://bbs.ctex.org/forum.php?mod=redirect&goto=findpost&ptid=80318&pid=467217&fromuid=40353
\usepackage{xpatch}

\makeatletter
  \chardef\TPT@@@asteriskcatcode=\catcode`*
  \catcode`*=11
  \xpatchcmd{\threeparttable}
    {\TPT@hookin{tabular}}
    {\TPT@hookin{tabular}\TPT@hookin{tabu}}
    {}{}
  \catcode`*=\TPT@@@asteriskcatcode
\makeatother

%------------设置表格默认格式(字号、行间距、单元格高度)------------
\let\oldtabular\tabular
\renewcommand{\tabular}{%
  \renewcommand\baselinestretch{0.9}\small    % 设置行间距和字号
  \renewcommand\arraystretch{1.5}             % 调整单元格高度
  %\renewcommand\multirowsetup{\centering}
  \oldtabular
}
%设置行间距,且必须放在字号设置前 否则无效
%或者使用\fontsize{<size>}{<baseline>}\selectfont 同时设置字号和行间距

\let\oldtabu\tabu
\renewcommand{\tabu}{%
  \renewcommand\baselinestretch{0.9}\small    % 设置行间距和字号
  \renewcommand\arraystretch{1.8}             % 调整单元格高度
  %\renewcommand\multirowsetup{\centering}
  \oldtabu
}

%------------模仿booktabs宏包的三线宽度设置---------------
\def\toprule   {\Xhline{.08em}}
\def\midrule   {\Xhline{.05em}}
\def\bottomrule{\Xhline{.08em}}
%-------------------------------------
%\setlength{\arrayrulewidth}{2pt} 设定表格中所有边框的线宽为同样的值
%\Xhline{} \Xcline{}分别设定表格中水平线的宽度 makecell包提供

%表格中垂直线的宽度可以通过在表格导言区(preamble),利用命令 !{\vrule width1.2pt} 替换 | 即可

%=================图表设置===============================
%---------------图表标号设置-----------------------------
\renewcommand\thefigure{\arabic{section}-\arabic{figure}}
\renewcommand\thetable {\arabic{section}-\arabic{table}}

\usepackage{caption}
  \captionsetup{font=small,}
  \captionsetup[table] {labelfont=bf,textfont=bf,belowskip=3pt,aboveskip=0pt} %仅表格 top
  \captionsetup[figure]{belowskip=0pt,aboveskip=3pt}  %仅图片 below

%\setlength{\abovecaptionskip}{3pt}
%\setlength{\belowcaptionskip}{3pt} %图、表题目上下的间距
\setlength{\intextsep}   {5pt}  %浮动体和正文间的距离
\setlength{\textfloatsep}{5pt}

%====================全文水印==========================
%解决方案来自:
%http://bbs.ctex.org/forum.php?mod=redirect&goto=findpost&ptid=79190&pid=462496&fromuid=40353
%https://zhuanlan.zhihu.com/p/19734756?columnSlug=LaTeX
\usepackage{eso-pic}

%eso-pic中\AtPageCenter有点水平偏右
\renewcommand\AtPageCenter[1]{\parbox[b][\paperheight]{\paperwidth}{\vfill\centering#1\vfill}}

\newcommand{\watermark}[3]{%
  \AddToShipoutPictureBG{%
    \AtPageCenter{%
      \tikz\node[%
        overlay,
        text=red!50,
        %font=\sffamily\bfseries,
        rotate=#1,
        scale=#2
      ]{#3};
    }
  }
}

\newcommand{\watermarkoff}{\ClearShipoutPictureBG}

\watermark{45}{15}{草\ 稿}    %启用全文水印

%=============花括号分支结构图=========================
\usepackage{schemata}

\xpatchcmd{\schema}
  {1.44265ex}{-1ex}
  {}{}

\newcommand\SC[2] {\schema{\schemabox{#1}}{\schemabox{#2}}}
\newcommand\SCh[4]{\Schema{#1}{#2}{\schemabox{#3}}{\schemabox{#4}}}

%=======================================================

\begin{document}
\pagestyle{main}
\fi
\chapter{《黄帝内经》的天文历法医学思想}%第二章

\section{《内经》的天文医学思想}%第一节

《内经》蕴涵有较为丰富的古天文学内容,运用天文学知识说明医学原理,建构医学体系。

\subsection{《内经》的宇宙结构学说及其医学意义}%一、

\subsubsection{《内经》的宇宙结构学说}%(一)

我国古代的宇宙结构学说,主要有盖天说、浑天说和宣夜说三种。第一,盖天说,始于西周前期,主要记载于《周髀算经》该说认为宇宙天地的构形是天圆地方,天形如张盖,顶高八万里而向四周下垂,日、月、五星在天穹上随天旋转;天如同一磨盘,被推着左转(从东向南向西),日、月、五星在“天”这个左转的磨盘上右转(从西向南向东);天穹象一个斗笠,大地象一个倒扣着的盘子,北极是天的最髙点,四周下垂;天穹上有日月星辰交替出没,在大地上产生昼夜的变化,昼夜变化是因为太阳早上从阳中出,而夜晚入于阴中。第二,浑天说,始于战国时期,主要记载于东汉张衡的《浑天仪注》。该说认为:天是一个浑圆的球,象一个鸡蛋。其中一半贮有水,圆形的地球浮在水面上,天之包地,犹壳之裹黄。中空的圆球如车轱般旋转,日、月、星辰附着在圆球的内壳上运行,周旋无终,其形浑浑。第三,宣夜说,始于战国时代,主要记载于《晋书·天文志》,认为天既不是一个蛋壳,也不是一个苍穹或圆面,而是无边无涯的空间,空间充满了气,日月星辰飘浮在气中,它们的运动受到气的制约,气的作用和运动不是任意的,而是有一定规则的。

对于宇宙的结构,《内经》中有盖天说、浑天说和宣夜说的描述。《灵枢·邪客》说:“天圆地方,人头圆足方以应之。”含有盖天说思想。《素问·五运行大论》说:“帝曰:地之为下,否乎?岐伯曰:地为人之下,太虚之中者也。帝曰:冯乎?岐伯曰:大气举之也。”认为大地悬浮于宇宙之中,但不是凭借水的作用托浮,而是依靠大气的力量支撑。反映浑天说思想,又含有宣夜说的成分。《素问·宝命全形论》说:“天覆地载,万物悉备,莫贵于人。人以天地之气生,四时之法成。”有盖天说的成分,但主要是强调“气”的作用,因而含有宣夜说思想。可以说《内经》的宇宙结构观主要是浑天说与宣夜说。

\subsubsection{《内经》宇宙结构学说的医学意义}%(二)

《内经》认为太虚大气托举大地是由于太虚大气形成了天地,按不同性质将太虚大气分为两大类,即阴气和阳气,并由阴阳二气形成了天地。所谓“积阳为天,积阴为地”,“阳化气,阴成形”,“清阳上天,浊阳归地”(见《素问·阴阳应象大论》),说明天是清阳的聚积,由于阳气轻清,升散飞扬,不停地运动,因而没有形体;地是独阴的堆积,由于阴气重浊,沉降凝结,静而固守,因而累积的阴气成了具有形体的大地。

由于《内经》强调大气贯穿于宇宙各处,包括人体内之脏腑经络,因而在它推步气的周日运行即推步太阳周日运行时,自然地将人体与宇宙结构联系起来,将人体气血运行与日行二十八宿直接联系起来。其太虚大气的运行规则不仅用以描述昼夜进程、四季进程,而且用以描述对人的影响。《内经》认为:“人以天地之气生”,太虚大气形成了天地和人,太虚大气不仅作用于大地,而且作用于人。作用于大地的寒暑燥湿风火六种阴阳程度不同的气也作用于人。以此推测人体得病的情况。

《内经》对天文现象的描述,往往带有占星术色彩。如《灵枢·九宫八风》的九宫图与西汉太乙九宮占盘格局大体一致。古代占星术用于医学,它不是从原始的前兆迷信中产生的,而是由具有丰富天文、气象知识的医学家创造出来的。其中有一部分古天文、历法、气象知识,也有一部分具有必然因果联系的征兆观,因而反映了人与自然;切相应的观点,这些都是我们应当继承的。

\subsection{《内经》的天球思想及其医学意义}%二、

《内经》的天球思想与浑天说、宣夜说的宇宙观思想有密切关系。

中国天文学家假想天球上存在一些点和圈,把地球轴线无限延长的线与天球的交点称天极,其中在北方上空与天球的交点称北天极;地球赤道无限延长的平面与天球相交的大圆圈称天赤道;地球公转轨道平面无限延长与天球相交的大圆圈称黄道;地平面与天球相交的大圆圈称地平圈。天赤道从东向西划分为十二个方位,以十二地支标记,称十二辰。十二辰以正北为子,向东、向南、向西依次是丑、寅、卯、辰、巳、午、未、申、酉、戌、亥。正北为子,正东为卯,正南为午,正西为酉。《灵枢·卫气行》所说的“子午为经,卯酉为纬”即指此而言。天球上有了这些基本的点和圈,天体的视位置和视运动才能够得到精确的表述。

《内经》认为天球是一个以地球为中心的球形天空,这个天球不是宇宙的界限,但是它的“存在”对于观察天体的视位置和视运动客现上提供了行之有效的天文背景。由于地球自西向东自转和公转,故《内经》所涉及的天体在天球上呈现出两类运动的周年视运动,其中二十八宿在赤黄道带、北斗七星在恒显圈内自东向西左旋,日月五星在黄道自西向东右旋;全部天体的周日视运动,自东向西左旋。

\subsubsection{日月的运动及其医学意义}%(一)

1.日月的运动:对于日、月和五星的运动,《素问·天元纪大论》表述为“七曜周旋”的形式。七曜,即日、月和五星。七曜周旋,是指古人站在地球上所见到日、月、五星等天体在黄道上的视运动。

太阳的视运动有周日视运动和周年视运动两种。太阳的周日视运动自东向南向西左旋;太阳的周年视运动自西向南向东右旋。《内经》对太阳视运动的描述是和昼夜四时相联系的,例如《灵枢·卫气行》所说的“昼日行于阳二十五周,夜行于阴二十五周”,是说太阳的周日视运动;《素问·阴阳应象大论》所说的“天有八纪”,是指太阳的周年视运动中,太阳在黄道上的立春、春分、立夏、夏至、立秋、秋分、立冬、冬至八个不同的位置而言。

月亮在空中的周期运动有两种,一种是月相的朔弦望晦变化,称朔望月周期;另一种是月球在恒星背景中的位置变化,即月球绕地球公转一周的运动,称恒星月周期。对于朔望月,《素问·八正神明论》提到“月始生”、“月廓满”、“月廓空”的月相盈亏盛衰变化。《灵枢·岁露》说:“故月满则海水西盛”、“月廓空则海水东盛”,已经认识到月亮是引起潮汐的主要因素。对于朔望月周期,《内经》没有明确论及,但《素问·六节藏象论》有“大小月”的记载。对于恒星月周期,《素问·六节藏象论》仅仅提供了“日行一度,月行十三度有奇焉”的数据。“月行十三度有奇”,即月亮每日在周天运行的度数。《内经》以周天为365$\frac{1}{4}$度,每日行$\frac{137}{19}$度,则恒星月周期应该是365$\frac{1}{4}$+$\frac{137}{19}$=27.32天。

2.日月的医学意义:《灵枢·岁露》说:“人与天地相参也,与日月相应也。”说明日月与人有密切关系。日的医学意义,首先表现在太阳的能量对人体阳气的影响上。《素问·生气通天论》说:“阳气者,若天与日,失其所则折寿而不彰,故天运当以日光明。最故阳因而上,卫外者也。”人体的阳气,就象天空中的太阳一样,具有维持生命机能,保卫机体和抗御外邪的作用。其次是周日视运动促使人体形成相应的生理节律。该篇又说:“故阳气者,一日而主外,平旦人气生,日中而阳气隆,日西而阳气已虚,气门乃闭。是故暮而收拒,无扰筋骨,无见雾露,反此三时,形乃困薄。”平旦、日中、日西、日暮,是太阳周日视运动的不同位置所确立的昼夜时间。当人体处在太阳周日视运动确立的不同时间时,人体中的阳气也随太阳所布阳气的变化而变化,白天阳气活跃于外,晚上阳气收敛于内。当阳气拒守于内时,不要扰动筋骨,不要接近雾露,避免邪气的侵袭,这是养生所必须注意的基本法则。

月的医学意义,主要体现在月对人的相关影响上。首先,月相盈亏的变化对人体血气、肌肉、经络的生理活动产生周期性的影响。《素问·八正神明论》说:“月始生则血气始精,卫气始行。月廓满则血气实,肌肉坚。月廓空则肌肉减,经络虚,卫气去,形独居。”《灵枢·岁露》进一步提出:“月满则海水西盛,人血气积……至其月廓空则海水东盛,人气血虚”,从月相盈亏、月亮对地球的引潮现象考察了月对人的生理作用。其次,月相盈亏对人的发病有影响。《灵枢·岁露》的认识是:月满之时,“肌肉充,皮肤致,毛发坚,腠理郄,烟垢著。当是之时,虽遇贼风,其入浅不深”;至其月廓空之时,“其卫气去,形独居,肌肉减,皮肤纵,腠理开,毛发残,膲理薄,烟垢落。当是之时,遇贼风则其入深,其病人也卒暴。”临床诊治疾病或判断预后时,应该结合天时月相。为此,《灵枢·岁露》提出了“乘年之衰,逢月之空,失时之和”的“三虚”原则,逢三虚,则发病急暴,“其死暴疾”。《素问·至真要大论》也指出“遇月之空,亦邪甚也”。再次,月相盈亏影响治疗效果。《内经》对此论述颇多。《素问·八正神明论》指出:针刺的治疗原则是“月生无泻,月满无补,月郭空无治。”因为“月生而泻,是谓脏虚;月满而补,血气扬溢,络有留血,命曰重实;月郭空而治,是谓乱经。”针刺的具体手法中“以气方盛也,以月方满也,以日方温也”,故要“泻必用方”。对于针刺的用穴数也有明确的规定。《素问·刺腰痛篇》说:“以月生死为痏数。”王冰注曰:“月初向圆为月生,月半向空为月死,死月刺少,生月刺多。《素问·缪刺论》曰:月生一日一痏,二日二痏。渐多之,十五日十五痏。十六日十四痏,渐少之。”

\subsubsection{五星的运动及其医学意义}%(二)

1.五星的运动:五星指金、木、水、火、土五星,《内经》又称太白、岁星、辰星、荧惑、镇星。五星的视运动指观察者从地球上观察行星在天球上的位置移动。《素问·气交变大论》论述了五星的视运动,认识到行星的视运动有徐、疾、逆、顺、留、守的运动变化规律,有“以道留久,逆守而小”、“以道而去,去而速来,曲而过之”、“久留而环,或离或附”三种运动轨迹,还论述了五星的亮度与颜色的变化,认为五星在运动轨迹的各个位置上,亮度和大小有着不同的变化,尤其是地外行星在冲前后,也就是逆行时,往往显得最亮。

2.五星的医学意义:《内经》认为,天上的五大行星是金、木、水、火、土五行应天之气的表征,直接影响到人的五脏,《素问·金匮真言论》说:“东方青色……其应四时,上为岁星”;“南方赤色……其应四时,上为荧惑星”;“中央黄色……其应西时,上为镇星”;“西方白色……其应四时,上为太白星”,“北方黑色……其应四时,上为辰星”。意为五大行星是由五行之气化成的。《内经》还认为,岁运和五大行星视运动有关。《素问·气交变大论》说:“岁运太过,则运星北越;运气相得,则各行以道。”岁运太过,则主岁的运星向北偏行;如果没有太过与不及,就在正常轨道上顺行。不仅如此,岁运还与五大行星颜色的变化有关。该篇还说:“故岁运太过,畏星失色而兼其母;不及,则色兼其所不胜。”五大行星的颜色有正常、兼其母和兼其所不胜三种颜色。所谓兼其母的颜色,如岁星为木行的青色,兼有水行的青黑色;所谓兼其所不胜的颜色,则兼有金行的白色。显然,这三种颜色都与岁运有关系·体现了五星对医学的影响。

\subsubsection{北斗星及其医学意义}%(三)

1.北斗星:北斗星由北方天空恒显圈内天枢、天璇、天玑、天权、玉衡、开阳、摇光七颗较亮的恒星组成,古人用假想的线把它们连接起来,象酒斗的形状,所以称为北斗。其中天枢、天璇、天玑、天权四星组成斗身,叫斗魁,又称璇玑;玉衡、开阳、摇光三星组成斗柄,叫斗杓,又称玉衡。天枢、天璇两星之间划一条连线并延长五倍处,便是北极星,北极星又称“北辰”,是北方的标志。北极星居中,北斗星自东向西运转于外,旋指十二辰。北斗星主要用来指示方向,确定时节。

《内经》中多处提到北斗星和北极星的名称。《灵枢·九宫八风》有“太一”、“招揺”的记载,“太一”即指北极星,“指摇”指北斗星的斗柄。《素问·天元纪太论》还有“九星悬朗”的说法。公元前二千年以前,北斗星靠近北极,北斗七星连同斗柄延伸下去的玄戈(牧夫座$\lambda$)、招摇(天龙座$\lambda$)都在恒显圈内,故称“九星悬朗”。《内经》还有北斗星围绕北极星回转不息的描述,如《灵枢·九宫八风》叙述了“太一”依次移居九宫,实际上说明北斗星围绕北极星回转不息、旋指十二辰的运动。

2.北斗星的医学意义:首先,以北斗指向推知月建阴阳变化来解释六经证候的病理机转。例如《素问·脉解》说:“太阳所谓肿腰脽痛者,正月太阳寅,寅太阳也,正月阳气出在上而阴气盛,阳未得自次也,故肿腰脽痛也。”正月为一年之首,太阳为诸阳之首,故正月属于太阳,而月建在寅,是阳气升发的季节,但是阴寒之气尚盛,阳气当旺不旺,病及于经,所以腰肿、臀部疼痛。其次,以北斗指向推知四时气候变迁、八方气象变化对人体的影响。例如《灵枢·九宫八风》说:“太一移日,天必应之以风雨,以其日风雨则吉,岁美民安少病矣。先之则多雨,后之则多汗[旱]”。太一从上一宫转向下一宫的第一天,也就是交换节气的日子,如果风调雨顺,则年景必然谷物丰收,民众安居,很少疾病。假若交节之前有风雨,是气候有余,就会多雨;假若交节之后多风雨,是气候不足,就会多旱,雨、旱天气人就多病。

\subsubsection{二十八宿及其医学意义}%(四)

1.二十八宿:古天文学为了观测日、月、五星的运行确认了二十八群恒星标志,称为二十八宿。二十八宿不仅和四象结合,并且和五色、五方、五行相结合,东方苍龙,包括角、亢、氐、房、心、尾、箕七宿;南方朱雀,包括井、鬼、柳、星、张、翼、轸七宿;西方白虎,包括奎、娄、胃、昂、毕、觜、参七宿;北方玄武,包括斗、牛、女、虚、危、室、壁七宿。《内经》中已有记载,《灵枢·卫气行》说:“天周二十八宿而一面七星,四七二十八星,房昴为纬,虚张为经。”二十八宿的划分,主要是以土星的视运动作为依据的。《素问·八正神明论》说:星辰者,所以制日月之行也。”这个“制日月之行”的星辰就是分布在赤黄道上的恒星群。此外,又根据木星12年一周天,每年行经一次,在赤黄道上自西向东把二十八宿重新划归为十二次。十二次的名称是星纪、玄枵、娵訾、降娄、大梁、实沈、鹑首、鹑火、鹑尾、寿星、大火、折木。十二次是以牛宿所在的星纪作为首次。十二次与二十八宿具有对应的关系。此外,二十四节气与十二次的形成有着渊源的关系,二十四节气产生于十二次。

2.二十八宿的医学意义:首先,依据二十八宿确立人身经脉长度、营卫行度,《灵枢·五十营》说:“气行十六丈二尺,气行交通于中,一周于身,下水二刻,日行二十五分。”根据日行28宿,经过12时辰,水漏下100刻,卫气行身50周,呼吸13500息以及一息脉行0.6尺的基本数据,推算人身28脉的总长度为16丈2尺,日行一宿卫气行度为1.8周、水下一刻卫气行度为0.5周。其次,根据二十八宿确立十干统运原观。十干统运,又称中运、岁运,通主一年的气运,是推算客运的基础。十干统运的规律是:“甲己之岁,土运统之;乙庚之岁,金运统之;丙辛之岁,水运统之;丁壬之岁,木运统之;戊癸之岁,火运统之。”(《素问·天元纪大论》)古人仰观天象,发现丹天、黅天、苍天、素天、玄天五色之气横贯周天二十八宿,而二十八宿又与天干地支方位对应,根据五色之气所在的宿位便可以确定十干统运的原则。

《内经》是以虚宿为冬至,反映的是夏代的天象,《素问·脉解》说:“太阴子也,十一月万物皆藏于中。”张介宾注:“阴极于子,万物皆藏,故曰太阴子也。”“一阳下动,冬至候也。”(《类经·疾病类》)根据“子午为经”和“虚张为纬”的说法,《内经》的冬至点是在虚宿。根据《内经》的天象,二十八宿、十二次、二十四节气具有反旋的对应关系。

\section{《内经》的历法医学思想}%第二节

把年、月、日、时等计时单位按照一定的法则进行编排以便记录和计算较长的时间序列,这种法则叫历法。年、月、日等时间需要借助天体的运动测定,而天体的运动只有在恒星的背景上才能被显现出来。制定历法也必须以恒星背景作为时间标尺。为了提供太阳运行的准确标尺,古天文学又把十二次与二十八宿的具体星象分开,按照木星实际运行的度数将天球赤黄道带自西向东划分为十二次。从按具体星象区划天空上升到按无形的标志点均匀区划天空,从而使抽象的天度和十二次开始具有时间标尺的作用,并使年、月、日的计算进入量化的阶段。至此,观象授时退出历史舞台,历法的时代真正到来。

古人以昼夜交替的周期为一“日”,以月相变化的周期为一“月”(现代叫做朔望月),以寒来暑往的周期亦即地球绕太阳一周的时间为一“年”(现代叫做太阳年)。以朔望月为单位的历法是“阴历”,以太阳年为单位的历法是“阳历”。我国古代的历法不是阳历,也不是纯阴历,而是阴阳合历。我国夏代已产生天干十进制记日法,殷商已使用干支记日法、朔望记月法,战囯有古六历(古四分历),西汉有太初历、三统历,东汉有四分历(后汉四分历)。

\subsection{《内经》五运六气历}%一、

《内经》采用四分历,并发明了五运六气历。四分历以一回归年等于365$\frac{1}{4}$日,因岁余四分之一日而得名。四分历又用朔望月来定月,用闰月的办法使年的平均长度接近回归年,兼有阴历月和回归年双重性质,属于阴阳合历。以岁实(也叫岁周,相当于回归年)为365$\frac{1}{4}$日,朔策(也叫朔实,相当于朔望月)为29$\frac{499}{940}$日。岁余$\frac{1}{4}$日,通过置闰月调整岁实与朔策的长度,是一种既重视月相盈亏,又照顾二十四节气,年、月、日均依据天象的历法。《内经》实行的也是四分历,实际采用岁实为365$\frac{1}{4}$日的数据。其中的太阳历又有二十四节气与气候、物候变化相符,以表示一年之中生物的生化节律。

《内经》的历法不仅具有岁实$\frac{1}{4}$这个斗分,而且是以建寅为正,与《历术甲子篇》的西分法一脉相承。

值得重视的是,《内经》还独创了“五运六气历”,它也属于阴阳合历,以天干地支作为运算符号进行推演,阐明六十甲子年中天度、气数、气候、物候、疾病变化与防治规律,从时空角度反映天地人的统一。《内经》运气历采用十天干与十二地支相配以记年、月、日、时的方法,以十天干配合五运推算每年的岁运,以十二地支配合六气推算每年的岁气,并根据年干支推算六十年天时气候变化及其对人体生命活动的影响。

五运六气历划分的原则是“分则气分,至则气至”,表示气数与天度相对应。五运六气历将一年分为六步,也称六气。每一步气占二十四节气中的四个节气。每年的六步气是:第一步气始于大寒,历经立春、雨水、惊蛰;第二步气始于春分,历经清明、谷雨、立夏;第三步气始于小满,历经芒种、夏至、小暑;第四步气始于大暑,历经立秋、处暑、白露;第五步气始于秋分,历经寒露、霜降、立冬;第六步气始于小雪,历经大雪、冬至、小寒。然后又进入次年第一步气大寒。由上述六步气中二十四节气的分布可以看出,各步气的起始点均为中气,第二和第五步气正是春分和秋分。春分是第一步气与第二步气的分界,秋分是第四步气与第五步气的分界。如果将第一步气至第三步气看作上半年,第四步气至第六步气看作下半年,则第二步气和第五步气分别为上半年和下半年的中间,春分和秋分二分点就分别是上半年和下半年的分界线,这叫做“分则气分”。二十四节在六步气的分布中上半年阳气当令时,阳气鼎盛的极点是夏至;下半年阴气当令时,阴气鼎盛的极点是冬至。夏至和冬至分别为阴气生长和阳气生长的起点,说明“至”是阴阳气到了极点。这叫做“至则气至”。至点不在第三步气和第六步气的最后,而居于中间,这表示了这两步气是阴阳二气由小至极而又返还的标志点。

五运六气历的每一步气占四个节气的长度,大约是60天,其所以取大率六十天的理由是与六十干支有一种对应关系。《素问·六节藏象论》中说:“天以六六为节,地以九九制会。天有十日,日六竟而周甲,甲六复而终岁,三百六十日法也。”实际上是将太阳在天球上的视运行转化为气的运行,气的运行按《周易·系辞传》所说“变动不居,周流六虚”分为六步。

\subsection{《内经》历法的医学意义}%二、

《内经》五运六气历认为,作用于大地的寒暑燥湿风火六种气,不是完全“迟疾任情”的,而是分为有规则的六步。六步气与五行相配应:厥阴配风木,少阴配君火,太阴配湿土,少阳配相火,阳明配燥金,太阳配寒水。这样六气配上五行,就形成了一个五行相生的节令推移规则,这就完成了一年太虚大气对大地作用的运转,反映了太阳周年视运动的过程。五运和六气相配合按照其属性关系可分为相生、相克、同化等,就同化而言,又有太过、不及、同天化、同地化等差别。《内经》运气历的主要目的是根据气候变化规律推知对人体的影响。如:由客主加临可推测该年四时气候变化是否正常、人体是否得病。其奥秘在于观察客主加临的五行生克。如客主之气五行彼此相生或相同,称为“气相得”,则气候和平,人不病;如客主之气五行相克,称为“不相得”,则气候反常,人体致病。依据司天、在泉之气,可预测生物得胎孕或不孕、人体发病或不病。如岁厥阴司天之年,人们多病胃脘心部疼痛,上撑胀两胁,咽膈不通利,饮食不下,其病的根本在于脾藏,如果冲阳脉绝,则是死证,不能救治。又如《灵枢·九宫八风》的八方之风,其中“虚风”成为中医病因学说的内容之一。以黄道标度日月运行节律,将黄道划分为不同的节点系统,这些节点是太阳在黄道上的特征位置,用以司天地之气的分、至、启、闭,由此定出四时、八正、二十四节气历法,反映天地阴阳之气消长气数和生命活动的节律,推测人体脏腑气血盛衰变化规律。

《内经》历法包含着对日、月、年时间节律的认识,为人体生命节律的研究奠定了坚实的科学基础。人体的生命活动存在于时空之中,与时间节律有着密切的联系,表现出生命活动的日节律、月节律和年节律。对此,《内经》有精辟的论述,例如,人体生命活动的日节律,《素问·生气通天论》有“故阳气者,一日而主外,平旦人气生,日中而阳气隆,日西而阳气已虚,气门乃闭”的描述;人体生命活动的月节律,《素问·八正神明论》有“月始生,则血气始精,卫气始行;月廓满,则血气实;肌肉坚;月廓实,则肌肉减,经络虚,卫气去,形独居”的描述;人体生命的年节律,《素问·四气调神论》有“夫四时阴阳者,万物之根本也。所以圣人春夏养阳,秋冬养阴,以从其根。故与万物沉浮于生长之门”的描述。“以从其根”道出了历法对医学理论的重要意义。

\zuozhe{(张其成)}
\ifx \allfiles \undefined
\end{document}
\fi