% -*- coding: utf-8 -*-
%!TEX program = xelatex
\ifx \allfiles \undefined
\documentclass[12pt]{ctexbook}
%\usepackage{xeCJK}
%\usepackage[14pt]{extsizes} %支持8,9,10,11,12,14,17,20pt

%===================文档页面设置====================
%---------------------印刷版尺寸--------------------
%\usepackage[a4paper,hmargin={2.3cm,1.7cm},vmargin=2.3cm,driver=xetex]{geometry}
%--------------------电子版------------------------
\usepackage[a4paper,margin=2cm,driver=xetex]{geometry}
%\usepackage[paperwidth=9.2cm, paperheight=12.4cm, width=9cm, height=12cm,top=0.2cm,
%            bottom=0.4cm,left=0.2cm,right=0.2cm,foot=0cm, nohead,nofoot,driver=xetex]{geometry}

%===================自定义颜色=====================
\usepackage{xcolor}
  \definecolor{mybackgroundcolor}{cmyk}{0.03,0.03,0.18,0}
  \definecolor{myblue}{rgb}{0,0.2,0.6}

%====================字体设置======================
%--------------------中文字体----------------------
%-----------------------xeCJK下设置中文字体------------------------------%
\setCJKfamilyfont{song}{SimSun}                             %宋体 song
\newcommand{\song}{\CJKfamily{song}}                        % 宋体   (Windows自带simsun.ttf)
\setCJKfamilyfont{xs}{NSimSun}                              %新宋体 xs
\newcommand{\xs}{\CJKfamily{xs}}
\setCJKfamilyfont{fs}{FangSong_GB2312}                      %仿宋2312 fs
\newcommand{\fs}{\CJKfamily{fs}}                            %仿宋体 (Windows自带simfs.ttf)
\setCJKfamilyfont{kai}{KaiTi_GB2312}                        %楷体2312  kai
\newcommand{\kai}{\CJKfamily{kai}}
\setCJKfamilyfont{yh}{Microsoft YaHei}                    %微软雅黑 yh
\newcommand{\yh}{\CJKfamily{yh}}
\setCJKfamilyfont{hei}{SimHei}                                    %黑体  hei
\newcommand{\hei}{\CJKfamily{hei}}                          % 黑体   (Windows自带simhei.ttf)
\setCJKfamilyfont{msunicode}{Arial Unicode MS}            %Arial Unicode MS: msunicode
\newcommand{\msunicode}{\CJKfamily{msunicode}}
\setCJKfamilyfont{li}{LiSu}                                            %隶书  li
\newcommand{\li}{\CJKfamily{li}}
\setCJKfamilyfont{yy}{YouYuan}                             %幼圆  yy
\newcommand{\yy}{\CJKfamily{yy}}
\setCJKfamilyfont{xm}{MingLiU}                                        %细明体  xm
\newcommand{\xm}{\CJKfamily{xm}}
\setCJKfamilyfont{xxm}{PMingLiU}                             %新细明体  xxm
\newcommand{\xxm}{\CJKfamily{xxm}}

\setCJKfamilyfont{hwsong}{STSong}                            %华文宋体  hwsong
\newcommand{\hwsong}{\CJKfamily{hwsong}}
\setCJKfamilyfont{hwzs}{STZhongsong}                        %华文中宋  hwzs
\newcommand{\hwzs}{\CJKfamily{hwzs}}
\setCJKfamilyfont{hwfs}{STFangsong}                            %华文仿宋  hwfs
\newcommand{\hwfs}{\CJKfamily{hwfs}}
\setCJKfamilyfont{hwxh}{STXihei}                                %华文细黑  hwxh
\newcommand{\hwxh}{\CJKfamily{hwxh}}
\setCJKfamilyfont{hwl}{STLiti}                                        %华文隶书  hwl
\newcommand{\hwl}{\CJKfamily{hwl}}
\setCJKfamilyfont{hwxw}{STXinwei}                                %华文新魏  hwxw
\newcommand{\hwxw}{\CJKfamily{hwxw}}
\setCJKfamilyfont{hwk}{STKaiti}                                    %华文楷体  hwk
\newcommand{\hwk}{\CJKfamily{hwk}}
\setCJKfamilyfont{hwxk}{STXingkai}                            %华文行楷  hwxk
\newcommand{\hwxk}{\CJKfamily{hwxk}}
\setCJKfamilyfont{hwcy}{STCaiyun}                                 %华文彩云 hwcy
\newcommand{\hwcy}{\CJKfamily{hwcy}}
\setCJKfamilyfont{hwhp}{STHupo}                                 %华文琥珀   hwhp
\newcommand{\hwhp}{\CJKfamily{hwhp}}

\setCJKfamilyfont{fzsong}{Simsun (Founder Extended)}     %方正宋体超大字符集   fzsong
\newcommand{\fzsong}{\CJKfamily{fzsong}}
\setCJKfamilyfont{fzyao}{FZYaoTi}                                    %方正姚体  fzy
\newcommand{\fzyao}{\CJKfamily{fzyao}}
\setCJKfamilyfont{fzshu}{FZShuTi}                                    %方正舒体 fzshu
\newcommand{\fzshu}{\CJKfamily{fzshu}}

\setCJKfamilyfont{asong}{Adobe Song Std}                        %Adobe 宋体  asong
\newcommand{\asong}{\CJKfamily{asong}}
\setCJKfamilyfont{ahei}{Adobe Heiti Std}                            %Adobe 黑体  ahei
\newcommand{\ahei}{\CJKfamily{ahei}}
\setCJKfamilyfont{akai}{Adobe Kaiti Std}                            %Adobe 楷体  akai
\newcommand{\akai}{\CJKfamily{akai}}

%------------------------------设置字体大小------------------------%
\newcommand{\chuhao}{\fontsize{42pt}{\baselineskip}\selectfont}     %初号
\newcommand{\xiaochuhao}{\fontsize{36pt}{\baselineskip}\selectfont} %小初号
\newcommand{\yihao}{\fontsize{28pt}{\baselineskip}\selectfont}      %一号
\newcommand{\xiaoyihao}{\fontsize{24pt}{\baselineskip}\selectfont}
\newcommand{\erhao}{\fontsize{21pt}{\baselineskip}\selectfont}      %二号
\newcommand{\xiaoerhao}{\fontsize{18pt}{\baselineskip}\selectfont}  %小二号
\newcommand{\sanhao}{\fontsize{15.75pt}{\baselineskip}\selectfont}  %三号
\newcommand{\sihao}{\fontsize{14pt}{\baselineskip}\selectfont}%     四号
\newcommand{\xiaosihao}{\fontsize{12pt}{\baselineskip}\selectfont}  %小四号
\newcommand{\wuhao}{\fontsize{10.5pt}{\baselineskip}\selectfont}    %五号
\newcommand{\xiaowuhao}{\fontsize{9pt}{\baselineskip}\selectfont}   %小五号
\newcommand{\liuhao}{\fontsize{7.875pt}{\baselineskip}\selectfont}  %六号
\newcommand{\qihao}{\fontsize{5.25pt}{\baselineskip}\selectfont}    %七号   %中文字体及字号设置
\xeCJKDeclareSubCJKBlock{SIP}{
  "20000 -> "2A6DF,   % CJK Unified Ideographs Extension B
  "2A700 -> "2B73F,   % CJK Unified Ideographs Extension C
  "2B740 -> "2B81F    % CJK Unified Ideographs Extension D
}
%\setCJKmainfont[SIP={[AutoFakeBold=1.8,Color=red]Sun-ExtB},BoldFont=黑体]{宋体}    % 衬线字体 缺省中文字体

\setCJKmainfont{simsun.ttc}[
  Path=fonts/,
  SIP={[Path=fonts/,AutoFakeBold=1.8,Color=red]simsunb.ttf},
  BoldFont=simhei.ttf
]

%SimSun-ExtB
%Sun-ExtB
%AutoFakeBold:自动伪粗,即正文使用\bfseries时生僻字使用伪粗体;
%FakeBold:强制伪粗,即正文中生僻字均使用伪粗体
%\setCJKmainfont[BoldFont=STHeiti,ItalicFont=STKaiti]{STSong}
%\setCJKsansfont{微软雅黑}黑体
%\setCJKsansfont[BoldFont=STHeiti]{STXihei} %serif是有衬线字体sans serif 无衬线字体
%\setCJKmonofont{STFangsong}    %中文等宽字体

%--------------------英文字体----------------------
\setmainfont{simsun.ttc}[
  Path=fonts/,
  BoldFont=simhei.ttf
]
%\setmainfont[BoldFont=黑体]{宋体}  %缺省英文字体
%\setsansfont
%\setmonofont

%===================目录分栏设置====================
\usepackage[toc,lof,lot]{multitoc}    % 目录(含目录、表格目录、插图目录)分栏设置
  %\renewcommand*{\multicolumntoc}{3} % toc分栏数设置,默认为两栏(\multicolumnlof,\multicolumnlot)
  %\setlength{\columnsep}{1.5cm}      % 调整分栏间距
  \setlength{\columnseprule}{0.2pt}   % 调整分栏竖线的宽度

%==================章节格式设置====================
\setcounter{secnumdepth}{3} % 章节等编号深度 3:子子节\subsubsection
\setcounter{tocdepth}{2}    % 目录显示等度 2:子节

\xeCJKsetup{%
  CJKecglue=\hspace{0.15em},      % 调整中英(含数字)间的字间距
  %CJKmath=true,                  % 在数学环境中直接输出汉字(不需要\text{})
  AllowBreakBetweenPuncts=true,   % 允许标点中间断行,减少文字行溢出
}

\ctexset{%
  part={
    name={,篇},
    number=\SZX{part},
    format={\chuhao\bfseries\centering},
    nameformat={},titleformat={}
  },
  section={
    number={\chinese{section}},
    name={第,节}
  },
  subsection={
    number={\chinese{subsection}、},
    aftername={\hspace{-0.01em}}
  },
  subsubsection={
    number={(\chinese{subsubsection})},
    aftername={\hspace {-0.01em}},
    beforeskip={1.3ex minus .8ex},
    afterskip={1ex minus .6ex},
    indent={\parindent}
  },
  paragraph={
    beforeskip=.1\baselineskip,
    indent={\parindent}
  }
}

\newcommand*\SZX[1]{%
  \ifcase\value{#1}%
    \or 上%
    \or 中%
    \or 下%
  \fi
}

%====================页眉设置======================
\usepackage{titleps}%或者\usepackage{titlesec},titlesec包含titleps
\newpagestyle{special}[\small\sffamily]{
  %\setheadrule{.1pt}
  \headrule
  \sethead[\usepage][][\chaptertitle]
  {\chaptertitle}{}{\usepage}
}

\newpagestyle{main}[\small\sffamily]{
  \headrule
  %\sethead[\usepage][][第\thechapter 章\quad\chaptertitle]
%  {\thesection\quad\sectiontitle}{}{\usepage}}
  \sethead[\usepage][][第\chinese{chapter}章\quad\chaptertitle]
  {第\chinese{section}节\quad\sectiontitle}{}{\usepage}
}

\newpagestyle{main2}[\small\sffamily]{
  \headrule
  \sethead[\usepage][][第\chinese{chapter}章\quad\chaptertitle]
  {第\chinese{section}節\quad\sectiontitle}{}{\usepage}
}

%================ PDF 书签设置=====================
\usepackage{bookmark}[
  depth=2,        % 书签深度 2:子节
  open,           % 默认展开书签
  openlevel=2,    % 展开书签深度 2:子节
  numbered,       % 显示编号
  atend,
]
  % 相比hyperref,bookmark宏包大多数时候只需要编译一次,
  % 而且书签的颜色和字体也可以定制。
  % 比hyperref 更专业 (自动加载hyperref)

%\bookmarksetup{italic,bold,color=blue} % 书签字体斜体/粗体/颜色设置

%------------重置每篇章计数器,必须在hyperref/bookmark之后------------
\makeatletter
  \@addtoreset{chapter}{part}
\makeatother

%------------hyperref 超链接设置------------------------
\hypersetup{%
  pdfencoding=auto,   % 解决新版ctex,引起hyperref UTF-16预警
  colorlinks=true,    % 注释掉此项则交叉引用为彩色边框true/false
  pdfborder=001,      % 注释掉此项则交叉引用为彩色边框
  citecolor=teal,
  linkcolor=myblue,
  urlcolor=black,
  %psdextra,          % 配合使用bookmark宏包,可以直接在pdf 书签中显示数学公式
}

%------------PDF 属性设置------------------------------
\hypersetup{%
  pdfkeywords={黄帝内经,内经,内经讲义,21世纪课程教材},    % 关键词
  %pdfsubject={latex},        % 主题
  pdfauthor={主编:王洪图},   % 作者
  pdftitle={内经讲义},        % 标题
  %pdfcreator={texlive2011}   % pdf创建器
}

%------------PDF 加密----------------------------------
%仅适用于xelatex引擎 基于xdvipdfmx
%\special{pdf:encrypt ownerpw (abc) userpw (xyz) length 128 perm 2052}

%仅适用于pdflatex引擎
%\usepackage[owner=Donald,user=Knuth,print=false]{pdfcrypt}

%其他可使用第三方工具 如:pdftk
%pdftk inputfile.pdf output outputfile.pdf encrypt_128bit owner_pw yourownerpw user_pw youruserpw

%=============自定义环境、列表及列表设置================
% 标题
\def\biaoti#1{\vspace{1.7ex plus 3ex minus .2ex}{\bfseries #1}}%\noindent\hei
% 小标题
\def\xiaobt#1{{\bfseries #1}}
% 小结
\def\xiaojie {\vspace{1.8ex plus .3ex minus .3ex}\centerline{\large\bfseries 小\ \ 结}\vspace{.1\baselineskip}}
% 作者
\def\zuozhe#1{\rightline{\bfseries #1}}

\newcounter{yuanwen}    % 新计数器 yuanwen
\newcounter{jiaozhu}    % 新计数器 jiaozhu

\newenvironment{yuanwen}[2][【原文】]{%
  %\biaoti{#1}\par
  \stepcounter{yuanwen}   % 计数器 yuanwen+1
  \bfseries #2}
  {}

\usepackage{enumitem}
\newenvironment{jiaozhu}[1][【校注】]{%
  %\biaoti{#1}\par
  \stepcounter{jiaozhu}   % 计数器 jiaozhu+1
  \begin{enumerate}[%
    label=\mylabel{\arabic*}{\circledctr*},before=\small,fullwidth,%
    itemindent=\parindent,listparindent=\parindent,%labelsep=-1pt,%labelwidth=0em,
    itemsep=0pt,topsep=0pt,partopsep=0pt,parsep=0pt
  ]}
  {\end{enumerate}}

%===================注解与原文相互跳转====================
%----------------第1部分 设置相互跳转锚点-----------------
\makeatletter
  \protected\def\mylabel#1#2{% 注解-->原文
    \hyperlink{back:\theyuanwen:#1}{\Hy@raisedlink{\hypertarget{\thejiaozhu:#1}{}}#2}}

  \protected\def\myref#1#2{% 原文-->注解
    \hyperlink{\theyuanwen:#1}{\Hy@raisedlink{\hypertarget{back:\theyuanwen:#1}{}}#2}}
  %此处\theyuanwen:#1实际指thejiaozhu:#1,只是\thejiaozhu计数器还没更新,故使用\theyuanwen计数器代替
\makeatother

\protected\def\myjzref#1{% 脚注中的引用(引用到原文)
  \hyperlink{\theyuanwen:#1}{\circlednum{#1}}}

\def\sb#1{\myref{#1}{\textsuperscript{\circlednum{#1}}}}    % 带圈数字上标

%----------------第2部分 调整锚点垂直距离-----------------
\def\HyperRaiseLinkDefault{.8\baselineskip} %调整锚点垂直距离
%\let\oldhypertarget\hypertarget
%\makeatletter
%  \def\hypertarget#1#2{\Hy@raisedlink{\oldhypertarget{#1}{#2}}}
%\makeatother

%====================带圈数字列表标头====================
\newfontfamily\circledfont[Path = fonts/]{meiryo.ttc}  % 日文字体,明瞭体
%\newfontfamily\circledfont{Meiryo}  % 日文字体,明瞭体

\protected\def\circlednum#1{{\makexeCJKinactive\circledfont\textcircled{#1}}}

\newcommand*\circledctr[1]{%
  \expandafter\circlednum\expandafter{\number\value{#1}}}
\AddEnumerateCounter*\circledctr\circlednum{1}

% 参考自:http://bbs.ctex.org/forum.php?mod=redirect&goto=findpost&ptid=78709&pid=460496&fromuid=40353

%======================插图/tikz图========================
\usepackage{graphicx,subcaption,wrapfig}    % 图,subcaption含子图功能代替subfig,图文混排
  \graphicspath{{img/}}                     % 设置图片文件路径

\def\pgfsysdriver{pgfsys-xetex.def}         % 设置tikz的驱动引擎
\usepackage{tikz}
  \usetikzlibrary{calc,decorations.text,arrows,positioning}

%---------设置tikz图片默认格式(字号、行间距、单元格高度)-------
\let\oldtikzpicture\tikzpicture
\renewcommand{\tikzpicture}{%
  \small
  \renewcommand{\baselinestretch}{0.2}
  \linespread{0.2}
  \oldtikzpicture
}

%=========================表格相关===============================
\usepackage{%
  multirow,                   % 单元格纵向合并
  array,makecell,longtable,   % 表格功能加强,tabu的依赖
  tabu-last-fix,              % "强大的表格工具" 本地修复版
  diagbox,                    % 表头斜线
  threeparttable,             % 表格内脚注(需打补丁支持tabu,longtabu)
}

%----------给threeparttable打补丁用于tabu,longtabu--------------
%解决方案来自:http://bbs.ctex.org/forum.php?mod=redirect&goto=findpost&ptid=80318&pid=467217&fromuid=40353
\usepackage{xpatch}

\makeatletter
  \chardef\TPT@@@asteriskcatcode=\catcode`*
  \catcode`*=11
  \xpatchcmd{\threeparttable}
    {\TPT@hookin{tabular}}
    {\TPT@hookin{tabular}\TPT@hookin{tabu}}
    {}{}
  \catcode`*=\TPT@@@asteriskcatcode
\makeatother

%------------设置表格默认格式(字号、行间距、单元格高度)------------
\let\oldtabular\tabular
\renewcommand{\tabular}{%
  \renewcommand\baselinestretch{0.9}\small    % 设置行间距和字号
  \renewcommand\arraystretch{1.5}             % 调整单元格高度
  %\renewcommand\multirowsetup{\centering}
  \oldtabular
}
%设置行间距,且必须放在字号设置前 否则无效
%或者使用\fontsize{<size>}{<baseline>}\selectfont 同时设置字号和行间距

\let\oldtabu\tabu
\renewcommand{\tabu}{%
  \renewcommand\baselinestretch{0.9}\small    % 设置行间距和字号
  \renewcommand\arraystretch{1.8}             % 调整单元格高度
  %\renewcommand\multirowsetup{\centering}
  \oldtabu
}

%------------模仿booktabs宏包的三线宽度设置---------------
\def\toprule   {\Xhline{.08em}}
\def\midrule   {\Xhline{.05em}}
\def\bottomrule{\Xhline{.08em}}
%-------------------------------------
%\setlength{\arrayrulewidth}{2pt} 设定表格中所有边框的线宽为同样的值
%\Xhline{} \Xcline{}分别设定表格中水平线的宽度 makecell包提供

%表格中垂直线的宽度可以通过在表格导言区(preamble),利用命令 !{\vrule width1.2pt} 替换 | 即可

%=================图表设置===============================
%---------------图表标号设置-----------------------------
\renewcommand\thefigure{\arabic{section}-\arabic{figure}}
\renewcommand\thetable {\arabic{section}-\arabic{table}}

\usepackage{caption}
  \captionsetup{font=small,}
  \captionsetup[table] {labelfont=bf,textfont=bf,belowskip=3pt,aboveskip=0pt} %仅表格 top
  \captionsetup[figure]{belowskip=0pt,aboveskip=3pt}  %仅图片 below

%\setlength{\abovecaptionskip}{3pt}
%\setlength{\belowcaptionskip}{3pt} %图、表题目上下的间距
\setlength{\intextsep}   {5pt}  %浮动体和正文间的距离
\setlength{\textfloatsep}{5pt}

%====================全文水印==========================
%解决方案来自:
%http://bbs.ctex.org/forum.php?mod=redirect&goto=findpost&ptid=79190&pid=462496&fromuid=40353
%https://zhuanlan.zhihu.com/p/19734756?columnSlug=LaTeX
\usepackage{eso-pic}

%eso-pic中\AtPageCenter有点水平偏右
\renewcommand\AtPageCenter[1]{\parbox[b][\paperheight]{\paperwidth}{\vfill\centering#1\vfill}}

\newcommand{\watermark}[3]{%
  \AddToShipoutPictureBG{%
    \AtPageCenter{%
      \tikz\node[%
        overlay,
        text=red!50,
        %font=\sffamily\bfseries,
        rotate=#1,
        scale=#2
      ]{#3};
    }
  }
}

\newcommand{\watermarkoff}{\ClearShipoutPictureBG}

\watermark{45}{15}{草\ 稿}    %启用全文水印

%=============花括号分支结构图=========================
\usepackage{schemata}

\xpatchcmd{\schema}
  {1.44265ex}{-1ex}
  {}{}

\newcommand\SC[2] {\schema{\schemabox{#1}}{\schemabox{#2}}}
\newcommand\SCh[4]{\Schema{#1}{#2}{\schemabox{#3}}{\schemabox{#4}}}

%=======================================================

\begin{document}
\pagestyle{main}
\fi
\chapter{《黄帝内经》的心理医学思想}%第六章

心理医学是医学与心理学相结合的一门学科,虽然这一学科的建立只有几十年的时间,然而早在两千年前成书的《内经》中就已经蕴藏了丰富的心理医学思想,并对心理学中最基本的一些问题有了细致的观察与探讨,而这些内容已成为中医学理论的重要组成部分,始终指导着中医学的临床实践。因此,深入挖掘和研究《内经》的心理医学思想,不仅有利于中医学术的发展和临床疗效的提高,而且对现代心理医学的发展也具有重要意义。

心理学是研究人的心理活动规律的科学,它主要包括人的心理活动过程和个性心理特征两方面内容。心理活动过程指人的认知、情感和意志活动过程,个性心理特征则包括人的能力、气质、兴趣和倾向性等。对此,《内经》主要是用神这一概念来表述。据初步统计,《内经》中论述神的地方达150处之多,涉及到诸多的心理活动及概念,形成了自身独特的基本理论观点,并用以认识分析疾病、指导临床诊断和治疗等。这里主要分作“对心理活动的认识”、“基本理论观点”、“临床应用”三个部分加以阐述。

\section{《内经》对心理活动的认识}%第一节

\subsection{感知觉、记忆、思维与想象}%一、

感觉、知觉、记忆、思维、想象属人的认知过程,也是人的最基本的心理活动。

\subsubsection{感知觉}%(一)

感觉是人对客观事物的个别属性的感受与反映,而知觉则是感觉基础上对客观事物整体的反映。感知觉产生的基础首先是人体的感官,即与外界直接接触的器官,若此才能接受外界的刺激,故墨子称之为“五路”,荀子称之为“天官”、“五凿”,中医学称其为“窍”等,即强调与外界直接接触这一点。其次是内在脏腑的参与和主持,《灵枢·脉度》云:“五脏常内阅于上七窍也,故肺气通于鼻,肺和则鼻能知臭香矣;心气通于舌,心和则舌能知五味矣;肝气通于目,肝和则目能辨五色矣;脾气通于口,脾和则口能知五谷矣;肾气通于耳,肾和则耳能闻五音矣”。《灵枢·本神》曰:“所以任物者谓之心”。又云:“肺藏气,气舍魄”。其“魄”,张介宾《类经·藏象类》云:“魄之为用,能动能作,痛痒由之而觉也”。由此可知,脏腑在感知觉中的作用有二,其一是主持各感官功能,直接参与感觉过程;其二是在感觉基础上加工推理而为知觉,正如《墨子·经上》:“闻,耳之聪也,……循所闻而得其意,心之察也”。另外,《内经》也涉及了自我的内部感觉的内容,如《灵枢·大惑论》提及的迷惑眩晕的错觉与幻觉,《灵枢·海论》所说的“血海有余,则常想其身大,怫然不知其所病;血海不足,亦常想其身小,狭然不知其所病”之类,均属这方面内容。

\subsubsection{注意}%(二)

注意是心理活动对一定事物的指向和集中,它虽然不是独立的心理活动过程,但却是一切心理活动的开端,且伴随其始终。《灵枢·本神》云:“脾藏营,营舍意”,这里的意,有一层含义就是指注意,如张介宾《类经·藏象类》所云:“一念之生,心有所向,而未定者,曰意”。《内经》在诊治中十分强调对注意的运用,《素问·征四失论》批评了医生的四种过失,首先指出:“所以不十全者,精神不专,志意不理,外内相失,故时疑殆”。在针刺中,《素问·宝命全形论》提出:“如临深渊,手如握虎,神无营于众物”,《灵枢·终始》认为:“深居静处,占神往来,闭户塞牖,魂魄不散,专意一神,精气之(不)分,毋闻人声,以收其精,必一其神,令志在针,浅而留之,微而浮之,以移其神,气至乃休”。之所以如此,只因为有了注意才能清晰地反映周围世界中的某一特定事物,同时摆脱在当时不具有重要性的其余事物的干扰,而任何心理活动过程总是由于注意指向它所反映的事物才能产生。

\subsubsection{记忆}%(三)

记忆是人对过去经历中发生过的事物的反映,它是一种复杂的心理过程,主要包括记和忆两方面。记指对事物的识记和保持;忆是把记住的东西重新恢复出来,有再认和重现两种形式。《灵枢·本神》云:“所以任物者谓之心,心有所忆谓之意”。任物,即担任、反映客观事物。故属于感知过程,但亦归属记忆中的识记过程,忆,则具有记忆之义。可见,中医学把记忆主要归属于心。不过《灵枢·本神》又云:“脾藏营,营舍意”,说明脾也具有主记忆之功。意,除有意念萌动,属思维、想象外,还有记忆中的保持、再认、重现之义,正如段玉裁《说文解字注》所云:“意之训为测度、为记……训记者,如今人云记忆是也。其字俗作忆”。另外,记忆之病亦涉及多个脏腑,如《灵枢·大感论》云:“人之善忘者,何气使然?岐伯曰:上气不足,下气有余,肠胃实面心肺虚。虚则营卫留于下,久之不以时上,故善忘也”。后世更把脑、肾亦列其中,如汪昂《本草备要·本部》,他强调治健忘,必交通心肾,使心之神明下通于肾,肾之精华上升于脑。因此,记忆实是众多脏腑共同配合完成的,同时也受多个脏腑的调节。

\subsubsection{思维}%(四)

思维是人对客观事物间接与概括的反映,是力求反映事物本质属性和规律的认识活动。它具有概括性、间接性与问题性等特征。其思维的过程主要就是思考问题、解决问题的过程。从其包涵的内容而言,当有分析与综合、分类与比较、抽象与概括、具体化与系统化等几方面;从其表现形式的过程来分析,则正如《灵枢·本神》所言:“心有所忆谓之意,意之所存谓之志,因志而存变谓之思,因思而远慕谓之虑,因虑而处物谓之智”。意,属抽象思维,亦属思维过程的初期阶段。为确定、系统完善这些意念,并使之能符合具体事物,必然经历“所存”、“存变”、“远慕”、“因虑而处物”等过程,才能得以实现。而这些过程就是根据以往积累的经验、具体的实际事物、事物间的相互联系、事物的发展规律及其结果,来反复计度、深思远慕,进而掌握了事物的内在规律和特点,找到了解决问题的恰当方式与途径的过程。可见,其间正是以分析与综合、分类与比较、抽象与概括、具体化与系统化等几方面的运用作为基础的,这一点是毫无疑问的。《内经》认为,人的思维活动与脏腑密切相关,产生思维的基础主要责之于心。而思维的具体过程“所存”、“存变”、“远慕”、“因虑而处物”等却责之于脾与肾,所谓:“脾藏营,营舍意”,“肾藏精,精舍志”;《素问·阴阳应象大论》云:脾“在志为思”;《难经·三十四难》也说:“脾藏意与智,肾藏精与志也”。既云“因思而远慕谓之虑”,可知“虑”与“思”属同一类,亦当由脾所主。另外,《素问·灵兰秘典论》又说:“肝者,将军之官,谋虑出焉”。说明“虑”也涉及肝。又说:“胆者,中正之官,决断出焉。”而临床上一些犹豫不定、意志不坚之症,又常归属为胆虚,说明“虑”与“志”同胆也密切相关。

\subsubsection{想象}%(五)

想象是对以往的知觉经验己形成了的那些暂时联系进行新的加工整理过程,属抽象思维活动的继续,它可以使人认识无法直接感知到的事物形象,使人的认识大大超出时间、空间和具体条件的限制。想象分有意与无意两种,无意想象如人之做梦、触景生情等,为无目的产生;有意想象则为有目的的再造、创造与幻想事物形象。《素问·脉要精微论》、《灵枢·淫邪发梦》等所载之梦象,即属无意想象。《灵枢·本神》云:“随神往来谓之魂”,又云:“肝藏血,血舍魂”,张介宾《类经·藏象类》云:“魂之为言,如梦寐恍惚,变幻游行之境是也。神藏于心,故心静则神清;魂随乎神,故神昏则魂荡”。由此可知,魂不受心神支配则游行外出,即所谓“魂不守舍”,是产生无意想象的根源,当主要责之于心与肝。而有意想象,《内经》则有多处运用,如《素问·八正神明论》云:“请言神,神乎神,耳不闻,目明心开而志先,慧然独语,口弗能言,俱视独见,适若昏,昭然独明,若风吹云,故曰神”。这是对创造思维活动中所产生事物新形象即“灵感”过程的一种描述,它的要求是注意力高度集中于思考对象,思维处于极度敏捷状态,即“目明心开而志先”,当然它是以大量感知经验及所记忆的内容为基础的。至于说到想象的具体过程,实际上《灵枢·本神》:“心有所忆谓之意,意之所存谓之志,因志而存变谓之思,因思而远慕谓之虑,因虑而处物谓之智。”也可以看做是对它的概括,意念灵感的产生、完善及付诸实际仍需经历“所忆”、“所存”、“存变”、“远慕”、“因虑而处物”等几个阶段,其中关键之处仍在于“意”与“思”。而想象活动涉及脏腑也较多,如心、肝、脾、肾、胆等,由众多脏腑参与完成。

\subsection{意志}%二、

意志是自觉地确定目的并根据目的来支配和调节自己的行动,克服困难,从而实现预期目的的心理活动过程。其过程由采取决定和执行决定的阶段构成。以此定义而言,《灵枢·本神》所云:“意之所存谓之志,因志而存变谓之思,因思而远慕谓之虑,因虑而处物谓之智”,则又可以认为是阐述了意志的过程。“意之所存谓之志”,可以说是自觉地确定目的,属于采取决定的阶段,正如张介宾《类经·藏象类》所说:“谓意已决而卓有所立者曰志”。而“存变”、“远慕”、“因虑而处物”则是根据目的,克服困服、实现预期目的的心理活动,当属执行决定的阶段。“意”与“志”联在一起者,《内经》未见,而“志”与“意”连读者,却屡见不鲜。《灵枢·本脏》曰:“志意者,所以御精神,收魂魄,适寒温,和喜怒者也”。又云:“志意和,则精神专直,魂魄不散,悔怒不起,五脏不受邪矣”。从其能够支配“精神”、“魂魄”、“寒温”、“喜怒”等来看,则正是所谓“意志”。可见,人的意志对内有调节情感、驽驭精神活动之功,对外有促使人体适应外界环境变化之作用,属有目的的行为活动,是以随意动作为基础的,这些认识与现代心理学所述颇相吻合。

\subsection{情想}%三、

情感是人对客观事物态度的体验,是人的需要和客观事物之间关系的反映,其中往往是以是否满足人的需要为中介。而根据满足的程度,又可将之分为肯定(积极的)与否定(消极的)两方面。

早在《内经》成书之前,中国古贤就对情感有了深入的研究,尽管他们对情感具体内容的分类不尽统一,并将欲(好)、恶等抽象且概括性强的情感与具体情感内容(喜、怒等)平列,而且较之后世而言,情感内容不全(如少思),但是,对情感的认识也较深刻。首先,认为情感具有两极性特征,如《礼记·礼运》曰:“欲恶者,心之大端也”。孙希旦《礼记集解》则云:“情虽有七,而喜也、爱也,皆欲之别;怒也、哀也、惧也,皆恶之别也。故情七而欲恶可以该之,故曰欲恶者,心之大端也”。其二,认为情属“弗学而能”,即人之本能。其三,认为情乃受外界刺激而发,如《淮南子·原道训》云:“物至而神应,知之动也,知与物接,而好憎生焉”。这些为后世正确认识情感奠定了基础。

《内经》中运用阴阳五行将情感进行了系统划分,建立了情感与五脏之间关系学说。

首先,它较完整地提出了情感的具体内容,并认为情感的表现及其产生是以五脏精气活动为基础的。正如《素问·阴阳应象大论》所云:“人有五脏化五气,以生喜怒悲忧恐”,即明确指出情感的产生以五脏精气为基础,丰富了中医学形神一体现思想。至于情感的具体内容,该篇除提出了“喜怒悲优恐”外,又云:肝“在志为怒”、心“在志为喜”、脾“在志为思”、肺“在志为忧”、肾“在志为恐”,即怒、喜、思、忧、恐五志。需要说明的是,这里的“志”并无“意之所存谓之志”(《灵枢·本神》)之义,而是成为了“情”的代词,故中医学又把情感通常称为“情志”。《素问·举痛论》“九气”为病中,其属情感因素者则有怒、喜、悲、恐、惊、思六种。其后宋代陈言在总结《内经》的基础上,将情志确定为喜、怒、忧、思、悲、恐、惊而称为“七情”。至此,“七情”已较完整地概括与表达了人的情感内容,因而被中医学确定下来并运用至今。

其次,因情感具有两极性,故可以用阴阳分之,如《灵枢·行针》云:“多阳者多喜,多阴者多怒”。《素问·阴阳应象大论》、《素问·疏五过论》等篇还多次提到“暴怒伤阴,暴喜伤阳”。《内经》按五行与脏腑的特性建立五行五脏系统,而由于情感源自于五脏的精气活动,故《内经》也将情感按其特性而分属五行五脏,其中喜归心属火,怒归肝属木,悲、忧归肺属金,思归脾属土,恐归肾属水。惊,虽未明确归属,不过《素问·举痛论》说:“惊则心无所依,神无所归,虑无所定”,即认为可以影响诸多脏腑。实则,因其易导致气机紊乱使木之调畅异常,又具突然性而类风象,故属木而主归于肝。由于情感与五脏、五行的关系确立,故可以用此分析及治疗情志病证。

另外,情感虽分属五脏五行,但《内经》又认为情志过激伤人以伤心者居多,如《灵枢·邪气脏腑病形》的“愁忧恐惧则伤心”,《灵枢·口问》的“悲哀愁忧则心动,心动则五脏六腑皆摇”。据此,后世医家认为情感虽分属五脏,但又总统于心,如张介宾《类经·疾病类》云:“情志之伤,虽五脏各有所属,然求其所由,则无不从心而发……此所以五志惟心所使也”。

\subsection{睡眠}%四、

睡眠本属于人的生理过程,不在心理活动范畴之列,故现代心理学并不单独讨论睡眠问题。但中医学则十分重视睡眠,把失眠、嗜睡、多梦等列入神志疾患范畴,认为睡眠也是人的神志活动表现之一,故这里亦如以讨论。

《内经》对睡眠有独到的认识,首先认为人与自然界是统一的整体,而自然界之阴阳有昼夜晨昏的日节律变化,故人体的阳气,也随之有消长出入的日节律运动,于是产生人的寤寐活动,正如《灵枢·口问》所说:“阳气尽,阴气盛,则目瞑;阴气尽,而阳气盛,则寤矣”。其次,认为卫气运行于阳与阴,是导致寤寐的基本原因,故《灵枢·营卫生会》曰:“卫气行于阴二十五度,行于阳二十五度,分为昼夜,故气至阳而起,至阴而至”。而对嗜睡与失眠的产生亦用卫气运行失常来阐明,如《灵枢·大惑论》云:“卫气者,昼日常行于阳,夜行于明,故阳气尽则卧,阴气尽则寤。故肠胃大,则卫气行留久;皮肤湿,分肉不解,则行迟。留于阴也久,其气不清,则欲瞑,故多卧矣。其肠胃小,皮肤滑以缓,分肉解利,卫气之留于阳也久,故少瞑焉”。其三,认为睡眠由神主宰。《灵枢·本神》云:“随神往来者谓之魂”,神安则魂藏能寐,神不安则魂不安藏,则会出现不寐、多梦、梦游、梦语等多种睡眠障碍,故张介宾《景岳全书·杂证摸》云:“盖寐本乎阴,神其主也。神安则寐,神不安则不寐”。从脏腑与睡眠的关系而言,中医学更多地强调了心、肝、脾、胃、肠与胆。

\subsection{人格体质}%五、

人格,是现代心理学研究的重要内容,包括个性倾向性和个性心理特征两个方面,主要表现为个人在对人、对己、对事、对物等各方面适应时所形成的态度、趋向和所显示的独特个性,应该说它是由先天禀赋所决定的,同时又受到后天特别是幼年时期的影响,它具有相对持续性与稳定性。体质,属于生理和病理学范畴,主要指遗传禀赋、生理素质等多方面的个体差异,不过中医学中的“体质”含义,也有认为当包括心理素质者。《内经》常把人格与体质放在一起进行讨论,这正反映了中医学形神合一思想,故我们这里也将两者结合讨论。

《内经》主要运用阴阳五行提出了对人格体质类型的划分,《灵枢·通天》根据人阴阳气的多少将人区分为太阳、少阳、阴阳和平、少阴、太阴等五态。如云:“太阴之人,贪而不仁,下齐湛湛,好内而恶出,心和(《甲乙经》作“抑”)而不发,不务于时,动而后之,此太阴之人也……”。实际上是对人格的分类,其中包括了现代心理学所说的个人的能力、气质、性格、兴趣等。《灵枢·阴阳二十五人》则具体将人格与体质结合起来论述了二十五种类型,即把人按五行归类,分成木、火、土、金、水五种类型,然后再以五音类比,将五种类型的每一型分成一个具有典型特征的主型和四个各与主型不同且各自又互有区别的亚型,于是共得出二十五种类型,其具体内容,此不赘述。此外,《灵枢·论勇》、《灵枢·寿夭刚柔》还讨论了人之勇敢、怯懦,性情刚强、柔弱与人体质的关系。

《内经》还认为人格体质与五脏关系密切,人之五脏大小、高下、坚脆、端正、偏倾的不同,决定了其个人人格体质特点,如《灵枢·本脏》云:“五脏皆小者,少病,苦燋心,大愁忧;五脏皆大者,缓于事,难使以忧。五脏皆高者,好高举措;五脏皆下者,好出人下。五脏皆坚者,无病;五脏皆脆者,不离于病。五脏皆端正者,和利得人心;五脏皆偏倾者,邪心而善盗,不可以为人平,反复言语也”。

\section{《内经》心理医学思想的基本理论观点}%第二节

现代心理学认为,人的心理是人脑的机能,人的心理必须依附于脑,没有人脑就不能产生人的心理。对于这种心理与生理、心与身的关系,《内经》则主要从形神关系来认识,并且将人心理的产生与调节则主要责之于人体的核心——五脏。

\subsection{形神关系}%一、

心理和生理的关系从本质上讲就是形与神的关系。所谓形,主要指形体,它包括皮肉、筋脉、脏腑、气血津液精等;所谓神,在心理医学中,则主要指人的神志(心理)活动。《内经》关于形神关系的基本观点是:形为神之体,神为形之主,形与神俱则为健康之人。

\subsubsection{形为神之体}%(—)

首先,《内经》系统地论述了神的产生,即神由有形之精而化,受先天之精及后天水谷之精共同作用,且受外物之影响而成。《灵枢·本神》曰:“故生之来谓之精,两精相搏谓之神,”即言其受先天之精的影响;《素问·六节藏象论》云:“天食人以五气,地食人以五味。五气入鼻,藏于心肺,上使五色修明,音声能彰;五味入口,藏于肠胃,味有所藏,以养五气,气和而生,津液相成,神乃自生”,即言其受后天之精的作用;《灵枢·本神》所谓:“所以任物者谓之心”,随后才有意、思、虑、智,即言其必受外物刺激之影响而成。正因为有形之精对神的产生如此之重要,所以《内经》也经常把气、血、水谷精微、五脏之精气等径称为神;也正因为外物刺激对神的形成的关键作用,所以一方面强调感物而动的情志因素在人体发病中的重要地位,一方面把“怡惔虚无”、“不惧于物”作为养生的主要手段,以利于神的正常活动。但无论是先天之精、后天之精,还是外物环境,均为客观存在,属于“有形”之范畴。

其次,神寓于形中,形盛则神旺,形衰则神去。《内经》认为人只有具备了形体结构后,才能产生精神活动,而神产生之后则要依附于形而存在,虽然神遍布全身而存在,但以全身形体之中心——五脏为主,即《灵枢·本神》所提出的:心藏神、肝藏魂、脾藏意、肺藏魄、肾藏志。故当五脏发生病变时,神则会紊乱甚至去身而致死亡,即“肝气虚则恐,实则怒”、“心气虚则悲,实则笑不休”(《灵枢·本神》),“五脏皆虚,神气皆去,形骸独居而终矣”(《灵枢·天年》)。总之,形为神之体,不仅说明了神本于形而生,而且提出神随形之盛衰而盛衰,且不能离开形体而独立存在,正如《素问·上古天真论》所说:“形体不敝,精神不散”,而“形弊血尽”则“神不使也”(《素问·汤液醪醴论》)。

\subsubsection{神为形之主}%(二)

《淮南子·原道训》云:“以神为主者,形从而利;以形为制者,神从而害”。即言以神制形的重要性,其原因正如《淮南子·诠言训》所言:“神贵于形也。故神制则形从,形胜则神穷”。《内经》十分强调神对形的主宰作用,认为神虽由形所化,但反过来又作用于形,人体各脏腑组织器官的生理活动,均是在神的支配与调节之下协调有序地进行着的。《素问·灵兰秘典论》形象地用官职列举了各脏腑的功能及其重要性,提出这是一个既分工又合作的整体,但主宰这一整体的则是“神”,即藏神之心,故经文强调指出:“主明则下安……主不明则十二官危”。人生活在自然和社会中,则必然会受到外界的影响,而使人与自然、社会相适应者,也赖神的调节作用,正如《灵枢·本脏》所说:“志意者,所以御精抻,收魂魄,适寒温,和喜怒者也……志意和,则精神专直,魂魄不散,悔怒不起,五脏不受邪矣”。反之,在病理情況下,一旦失去神的主宰,则会危及生命,如《素问·五常政大论》说:“神去则机息”,《灵枢·天年》所言:“失神者死,得神者生”。这里的神,主要以人体生命力之“神机”以及感知觉、思维、意志之神为主。其实情感之神对形亦有重要作用,因而《内经》在致病因素中独重情志,《素问·举痛论》“九气”为病中,情志占有六条之多,在养生中又将调节情志列为重要措施,均说明了这一点。

\subsubsection{形与神俱则成为人}%(三)

《素问·上古天真论》说:“上古之人,其知道者,法于阴阳,和于术数,食饮有节,起居有常,故能形与神俱,而尽终其天年,度百岁乃去”。俱,协调、和谐之义,形神相俱,即言形神合一、相互协调而和偕,这既是古人衡量各种养生法度的标准,又是尽终天年的前提,可以说是生命现象存在的基本特征,是人体健康的重要标志。《灵枢·天年》在谈及胚胎形成时也说“血气已和,荣卫已通,五脏已成,神气舍心,魂魄毕具,乃成为人”。血气、营卫、五脏,皆形之类;神气、魂魄,皆神之类。其意为,人体生命的构成,不外形神两端,只有形神兼备,乃成为人;同时,唯有当血气“和”、营卫“通”、五脏“成”、神魂魄和谐时,才是一个具有生命活力的健康人。如果形神失谐或形神分离,则预示着病情危笃,甚或死亡。正如《素问·汤液醪醴论》云:“精气弛坏,荣泣卫除,故神去之”。“失神者死,得神者生”、“五脏皆虚,神气皆去,形骸独居而终矣”(《灵枢·天年》)。因此,《内经》主张养生应形神兼养,而以形神相俱为目的。

\subsection{脏腑藏神}%二、

关于脏腑与神志关系,《内经》有两个重要命题,一个是心主藏神明,一个是五脏藏神。由于《内经》本身是多种医学流派、各种学说结合的产物,所以可以把两者看作是不同的学说。

\subsubsection{心主藏神}%(一)

《内经》谈及心与神的关系主要有三种表达形式。其一,是涉及神的具体内容。如《灵枢·本神》:“所以任物者谓之心,心有所忆谓之意”,即言心有认知功能,并是人思维、意志活动的基础;将七情与五脏配属,其中心《素问·阴阳应象大论》称:“在声为笑”、“在志为喜”,言心主七情之喜,而喜的外在表现则为笑。另外,后世又根据《灵枢·本神》“随神往来者谓之魂”,神安魂藏则寐,认为心主睡眠;根据心开窍于舌,《灵枢·忧恚无言》:“舌者,音声之机也……横骨者,神气所使,主发舌者也”,认为心主言。而言语又是人思维、意志、情感等表现于外的一个窗口。第二种形式,是把心与其他四脏并列而各主神的一部分,但心所主却直言“神”,未言其具体,其他脏腑则分言魂、魄、意、志等。如《素问·宣明五气》及《灵枢·九针论》之“心藏神”、《灵枢·本神》“心藏脉,脉舍神”、《素问·六节藏象论》:“心者,生之本,神之变也”等。第三种形式,是单独言心具有主神明、精神之功,并为人体之主宰,如《素问·灵兰秘典论》:“心者,君主之官,神明出焉”、《灵枢·邪客》:“心者,五脏六腑之大主,精神之所舍也”等。

心主神的观念并非中医学独创,早在《内经》成书之前就已广泛存在于先秦诸家之论中,并已形成了较统一的认识。如《孟子·告子》云:“心之官则思,思则得之,不思则不能得也,此天之所与我者”。《孟子·尽心》曰:“君子所性,仁义礼智根于心”。《荀子》进一步指出心能主宰、控制情欲,改变人之性情,其《正名》云:“欲过之而动不及,心止之也。……欲不及而动过之,心使之也。”“性之好恶喜怒哀乐,谓之情。情然而心为之择”。中医学接受了这个认识,并结合中国社会制度传统的君臣制现念,形成了《内经》以君臣相傅论脏腑,其中心主神明为君主之官的思想。即《素问·灵兰秘典论》所云:“心者,君主之官,神明出焉”、“主明则下安,以此养生则寿,殁世不殆,以为天下则大昌;主不明则十二官危,使道闭塞不通,形乃大伤,以此养生则殃,以为天下者,其宗大危”。《灵枢·邪客》:“心者,五脏六腑之大主,精神之所舍也”等,因为心具有主神明、精神之功,为人体之主宰,故精神情志伤人首伤心,如《灵枢·邪气脏腑病形》云:“愁忧恐惧则伤心”、《灵枢·口问》曰:“悲哀愁忧则心动,心动则五脏六腑皆摇”、《灵枢·百病始生》云:“忧思伤心”等皆是。

\subsubsection{五脏藏神}%(二)

《内经》在继承中国哲学对心的认识、倡导心主神明、为君主之官的同时,还“心藏神”、“肝藏魂”、“肺藏魄”、“脾藏意”、“肾藏志”,从五脏整体角度阐发了脏腑与神志的关系。但由于五脏所对应的五神,即神、魂、魄、意、志其概念相互交叉包容、互为基础,而五神的产生与调节又是以五脏整体协调关系为基础的,故五脏藏神的含义在于把五脏看成一个整体,把神志活动(主要指认知、思维、意志过程)看成一个密不可分的整体,在五脏整体协调配合前提下完成认识过程。其神、魂、魄、意、志划分为五行、归属于五脏,仅是从认知、思维、意志过程中的某些心理活动具有不同的特性出发,给予类比而成,是用五行特性对这一过程的描述,而并非是对认知、思维、意志过程的实质内容与阶段进行严格的分类。因此,这一描述可以看作是为了说明人的认知、思维、意志过程也具有五行的某些特性,并且是以所有脏腑的参与作为基础的,其实也正是强调了五脏整体协调对其的主宰作用。

五行之间相生、相克,具有“亢则害,承乃制,制则生化”(《素问·六微旨大论》)的特性,是一整体观念,强调的是五行间的相互配合,其中无主次之分。以这种认识方法看待脏腑、神志,形成了五脏藏神的理论。可见,心主神明与五脏藏神的理论内涵及立论依据存在着很大差别,甚至可以说心主神明为君主之官的思想,反映了当时社会制度及哲学界“一元论”思想的影响;而以五行特性分析人体,立五脏为本,将人之神志活动分属五脏,则颇具“多元论”思想。二者当属《内经》时期不同的医学流派。

神以形为基础,同时又主宰形,形与神俱才是健康之人。无论是心主神明,还是五脏藏神,也均强调了这一点。

\subsection{以阴阳五行论七情}%三、

《内经》将喜、怒、忧、思、悲、恐、惊情绪变化与阴阳五行及五脏相配属。

喜,是因事遂心愿或自觉有趣而心情愉快的表现,因其活泼而表现于外,故有火之机动、活泼、炎上之象,属火而配属于心。

怒,是因遇到不符合情理或自己心境的事情而心中不快、甚至愤恨不平的情绪表现,缘其气机条达不畅而起,怒后又可引起气机上逆即升发太达,且怒象忽发忽止颇具木之象,故属木而配属于肝。

忧,是对某种未知结果而又不愿其发生的事情的担心,以至于形成一种焦虑、沉郁的情绪状态,因其内向而趋于气机之收敛,故属金而配属肺。

思,是思考问题时的精神状态,它是其他情志表现的基础,其他多种情志均是“思”而后发。这是由于脾为中央土,有灌溉其他各脏的作用。而思由脾所主,其性属土,故为多种情志活动的基础。

悲,是精神烦恼悲哀失望时产生的痛苦情绪,其象如秋风扫落叶之凄凉、毫无生机、气机内敛,故属金而主于肺。

恐,是机体面临并企图摆脱某种危险而又无能为力时产生的精神极度紧张的情绪体验,由于其发自于内且常引起气机下陷而属水主于肾。

惊,是在不自知的情况下突然遇到非常事件时,精神骤然紧张而骇惧的情绪表现,因其易导致气机紊乱使木之调畅异常,又具突然性而类风象,故属木而主于肝。

《内经》对七情不仅根据其各自的特性而进行了阴阳五行的划分,将之与五脏分别配属,而且提出所划分的七情之间具有五行相克关系,正如《素问·阴阳应象大论》所言:“悲胜怒”、“恐胜喜”、“怒胜思”、“喜胜忧”、“思胜恐”,又由于七情可引起人体气机的不同变化,怒则气上,恐则气下等,因此,为临床治疗因情志异常导致的疾病提供了依据。

此外,如第一节所述,《内经》还以阴阳论睡眠、以阴阳五行论人格体质,亦形成了自身独特的理论。

\section{《内经》心理医学思想的临床应用}%第三节

\subsection{心理因素与发病}%一、

\subsubsection{心理因素的致病作用}%(一)

神寓于形之中,但又主宰形体,故神志异常可以导致气血脏腑等形的改变,甚至引起疾病。关于情志因素致病,《内经》从以下几方面进行了阐述。其一,个体的欲望或需求不能得到满足时,则加剧心理活动,影响生理功能,进一步导致疾病的发生。《素问·瘦论》说:“有所失亡,所求不得,则发肺鸣,鸣则肺热叶焦发为痿躄。”《灵枢·贼风》说:“因而志有所恶,及有所慕,血气内乱。”其二,由于生活中的某些意外突发事件,使人体产生过强的心理应激反应,进而影响脏腑气血紊乱而致病。《素问·疏五过论》说:“离绝菀结,忧恐喜怒,五脏空虚,血气离守。”离者失其亲密,绝者断其所怀,菀谓思虑抑郁,结谓深情难解等生离死别造成悲苦抑郁,这些以及过度的忧恐喜怒,都能使五脏气血空虚受损致病。其三,各种原因使得个体原有的性情改变,而发生疾病。《素问·疏五过论》说:“故贵脱势,虽不中邪,精神内伤,身必败亡。始富后贫,虽不伤邪,皮焦筋屈,痿躄为挛。”即是指由于政治经济地位骤变,影响了情志,而导致了疾病。

《内经》重视心理因素致病的观点,是显而易见的,仅在《素问·举痛论》提出的“九气”为病中,属心理因素者就占了六种之多。

\subsubsection{心理因素的致病机理}%(二)

《素问·举痛论》言:“百病生于气,”认为许多疾病的发生,都是由于气机失调而引起,并指出心理因素致病的机理均是脏腑气机紊乱,如说:“怒则气逆,甚则呕血及飧泄,故气上矣。喜则气和志达,荣卫通利,故气缓矣。悲则心系急,肺布叶举,而上焦不通,荣收不散,热气在中,故气消矣。恐则精却,却则上焦闭,闭则气还,还则下焦胀,故气不(下)行矣”;“惊则心无所依,神无所归,虑无所定,故气乱矣”;“思则心有所存,神有所归,正气留而不行,故气结矣。”气机升降出入是脏腑的特性,也是其内在联系的基本形式,故脏腑间协调和谐的关系依靠脏腑的气机来维系。心理活动既然是以脏腑间和谐关系为前提条件,故必定与脏腑气机关系密切,当心理活动异常改变时,则引起气机紊乱,进而导致脏腑功能失调而发生疾病。

另外,人的心理活动是以形为基础,故若其过度则会耗形,而损伤五脏精气,《素问·疏五过论》说:“尝贵后贱,虽不中邪,病从内生,名曰脱营;尝富后贫,名曰失精。”这里的“脱营”、“失精”,皆是心志凄怆,情怀抑郁,或神气不伸而致营血不生,或心内忧煎、奉养日廉致精气内耗,最终所导致的慢性虚损性疾病,也是心理因素致病机理之一。

\subsubsection{心理因素致病的规律}%(三)

《内经》对心理因素致病规律的认识,主要有三种。其一,根据情志的阴阳五行五脏归属,某一情志过度可伤害其相对应的脏腑,如《素问·阴阳应象大论》提出的“怒伤肝”、“喜伤心”、“思伤脾”、“悲伤肺”、“恐伤肾”等。而由于五脏的五行相克关系,该脏又可以进一步伤害他脏,如怒伤肝,肝逆则克伐脾土,正如《素问·举痛论》所言:“怒则气逆,甚则呕血及飧泄,故气上矣。”张介宾《景岳全书·泄泻》也云:“凡遇怒气便作泄泻者,必先以怒时夹食,致伤脾胃,但有所犯,即随触而发,此肝脾二脏之病也。盖以肝木克脾土,脾受伤而然。”其二,各种心理因素首先伤心,然后再影响他脏。如《灵枢·邪气脏腑病形》云:“愁忧恐惧则伤心”,《灵枢·口问》曰:“悲哀愁忧则心动,心动则五脏六腑皆摇”,《灵枢·百病始生》云:“忧思伤心”等皆是。由于心主神、情志变化先由心发,故情志太过则先伤心,正如张介宾《类经·疾病类》所云:“情志之伤,虽五脏各有所属,然求其所由,则无不从心而发”。其三,《灵枢·本神》提出“心,怵惕思虑则伤神”、“脾,愁优而不解则伤意”、“肝,悲哀动中则伤魂”、“肺,喜乐无极则伤魄”、“肾,盛怒而不止则伤志”,即怵惕思虑伤心、愁忧伤脾、悲哀伤肝、喜乐伤肺、大怒伤肾,既未先集中于心再分散到五脏,也未按照五行配属的格局规律。说明情志伤人,错综复杂,有常有变,不可一概而论。

\subsection{心理与诊断}%二、

心理诊断指通过人体心理活动的外在征象,即人体外在的精神活动表现来诊断疾病。外在征象表现范围较广,如人的表情如何、思维是否敏捷、语言是否流利、逻辑性怎样等皆在此列,并可通过望、闻、问诊的方法来获得。由于人体内在的心理活动常常表露于外,表现在神、色、形、态之中,故可以将其作为诊断疾病、推断原委、推测预后的依据,如人们平常所说的“得意忘形”、“喜形于色”、“怒容满面”、“大惊失色”、“愁眉苦脸”等,皆属于此。今从望、闻、问诊三方面简述于此。

\subsubsection{望诊}%(一)

《内经》望诊包括望目、望舌、望色、望形、望姿态、望表情等诸方面,就心理与诊断而言,其一,注重从望诊中总结神气的得失,《素问·移精变气》云:“得神者昌,失神者死”。《内经》望神主要从精神、神志、神情、神气、神形、神色、神态几方面入手,见下表。

\begin{table}[h]%望神
  \centering
  %\caption{望神}\label{tab:望神}
  \begin{threeparttable}
    \tabulinesep=^5pt_4pt
    \begin{tabu}to.87\textwidth{X[-1,c,m]|X[5,m]|X[5,m]}
      \toprule
      \rowfont[c]{}
      分类 & 得\ \ 神                                                                           & 失\ \ 神 \\
      \midrule
      精神 & 精神抖撤,神识清朗,思维敏捷,形体健美,动作矫徤,语言洪亮,目光有神               & 精神萎靡,神识昏迷,动作迟缓,形体衰弱,语言细微,思维反应迟纯,目无神采 \\ \hline
      神志 & 意识清楚,欣然自得,语言自然、流畅,呼吸正常,神采奕奕                             & 意识不清,狂躁妄动,或抑郁沉默不语,表情淡漠,呼吸异常 \\ \hline
      神情 & 喜、怒、悲,思、恐,表情于面,情绪稳定,神情自若,和颜悦色,双目灵活,表情自然生动 & 喜、怒、悲、思、恐,败露于面,情绪异常,表情极端,双目暗淡 \\ \hline
      神气 & 神气十足,目光炯炯有神,行动敏捷,精力充沛,精气旺盛                               & 神气萎顿,目无光采,行动迟缓,形衰神疲,气短少言 \\ \hline
      神形 & 精盛力足,形体健美,生长发育正常,动作矫健灵活                                     & 精力衰败,形体衰惫,生长发育不良,动作迟钝 \\ \hline
      神色 & 神色调和,光泽活润,光亮含蓄                                                       & 神色衰败,面色晦暗,五脏邪气露于表面 \\ \hline
      神态 & 神态自若,姿态体位正常,动中有静,静中有动,肢体活动灵敏                           & 神态痛苦,姿态体位异常,动中烦乱,静中疲惫不堪,肢体活动强直 \\
      \bottomrule
    \end{tabu}
    %\hspace{2em}\footnotesize (摘自1985年全国首届中医心理学学术讨论会《中医心理学论丛》第18 页)
    \begin{tablenotes}
      \footnotesize
      \item[](摘自1985年全国首届中医心理学学术讨论会《中医心理学论丛》第18页)
    \end{tablenotes}
  \end{threeparttable}
\end{table}

其二,注重从表情入手,根据表情与五行五脏的配属关系,分析疾病的本质。《素问·调经论》说:“神有余则笑不休,神不足则悲”,《灵枢·本神》:“肝气虚则恐,实则怒”。即言用喜、悲、怒、恐等表情分析脏腑疾病。《素问·阴阳应象大论》认为心“在志为喜”,肝“在志为怒”,脾“在志为思”,肺“在志为悲”,肾“在志为恐”,故可以根据其表情,按五行生克规律去分析病情、诊断疾病。

其三,注重从行为表现及人体形态入手,归纳人体的人格体质。《灵枢·阴阳二十五人》、《灵枢·通天》详细记载了这方面的内容。

\subsubsection{闻诊}%(二)

闻诊中的心理诊断主要是从闻言语和听声音两方面进行分析。语言为人心理活动的主要外在表现,故也是心理诊断的主要依据。其一,闻言语。选词、语序、语言逻辑反映心理活动,而语言清朗与否、声调高低情况又常常与人的情感活动有关。《素问·脉要精微论》说:“言而微,终日乃复言者,此夺气也……言语善恶不避亲疏者,此神明之乱也”。指出声音低而内容重复,是人体气被耗夺之象,主要责之于肺;而语无伦次,妄言骂詈,则属心气散亡的表现。其二,听声音。情感变化可以引起声音改变,如喜则笑、悲则哭等,故《礼记·乐记》说:“音之起,由人心生也。人心之动,物使之然也。感于物而动,故形于声……是故其哀心感者,其声噍以杀;其乐心感者,其声啴以缓;其喜心感者,其声发以散;其怒心感者,其声粗以厉;其敬心感者,其声直以廉;其爱心感者,其声和以柔”。所以通过对某些声音的分析以诊断疾病,亦属心理诊断内容。《素问·阴阳应象大论》说:“肝在音为呼”、“心在音为笑”、“脾在音为歌”、“肺在音为哭”、“肾在音为呻”。明确了听声音诊疾病的部分内容。喻昌《医门法律》云:“凡闻声不能分呼笑歌哭呻,以求五脏善恶,五邪所干及神气所主之病者,医之过也”。呼笑哭歌呻明显属于情感活动的外在征象,故通过这些诊断疾病属心理诊断无疑。

\subsubsection{问诊}%(三)

问诊,是《内经》心理诊断中最重要、最直接的方法。医生通过与病人或家属进行有目的的交谈,可以了解病人的发病经过、自觉症状、思维意识、感觉记忆、情绪变动、生活习惯、人格气质等诸多有关心理活动的情况,用于诊断疾病。《灵枢·通天》将人格体质归纳为阴阳平和、太阴、少阴、太阳、少阳五态人,主要是通过询问了解人体过去和现在心理、行为的一贯性和规律性而得出的;《素问·脉要精微论》和《灵枢·淫邪发梦》有关梦象内容,也是通过问诊所得。《内经》十分重视问诊,如《素问·征四失论》说:“诊病不问其始,忧患饮食之失节,起居之过度,或伤于毒,不先言此,卒持寸口,何病能中。”并把“忧患”等心理活动作为必问内容。又如《素问·疏五过论》指出:“凡欲诊病者,必问饮食居处,暴乐暴苦,始乐后苦,”用以诊断疾病。同时《内经》还注重社会、生活环境对情志的影响,提出:“诊有三常,必问贵贱、封君败伤及欲侯王。故贵脱势,虽不中邪,精神内伤,身必败亡”。“离绝菀结,忧恐喜怒,五脏空虚,血气离守,工不能知,何术之语”。这些,为中医建立自然——社会——心理——生物先进的医学模式奠定了基础。现代心理医学同样把问诊作为主要诊断方法,医生通过与病人谈话,了解病人心理异常情况、性质、产生的原因,以达到诊断的目的,而其中的大部分内容,《内经》已经涉及。

\subsection{心理与治疗}%三、

心理治疗,又称精神疗法,即主要通过医者的言、行、情、态等影响患者的认知、情感和行为等,以达到治疗目的的一种方法。由于神寓于形之中,而神又为形之主宰,故调神能改变形,进而达到调形的目的。心理活动与人体气机关系密切,因而,它既可以引起疾病,也可以治疗疾病。利用心理活动的改变以影响气血的活动,进而治愈疾病,这是心理治疗的基本原理。由于诸多心理因素与五脏有归属关系,因此作为心理治疗的基本方法,可以直接利用这一关系,也可以利用其五脏五行的生克关系。对于一切疾病,心理治疗都有着或大或小的作用。在疾病的发生和发展过程中,心理因素所起的作用越强,则心理治疗的效果也就越大。病人的心理状态,病人的主观能动性,在治疗疾病的效果上有着重要的作用,故《灵枢·本神》提出了“凡刺之法,先必本于神”,《素问·五脏别论》有“病不许治者,病必不治,治之无功矣”的论断。

《内经》所载心理治疗的方法很多,今择其要简述于此。

\subsubsection{情志相胜}%(一)

《内经》将七情与五脏分别配属,并根据其各自的特性而进行了阴阳五行的划分,根据七情之间的五行相克关系,倡导情志相胜疗法,即怒伤肝,悲胜怒;喜伤心,恐胜喜;思伤睥,怒胜思;忧伤肺,喜胜优;恐伤肾,思胜恐(《素问·阴阳应象大论》)。这种方法巧妙地运用以偏纠偏的原理,根据五行互胜,用一种情志活动去纠正另一种情志刺激引起的疾病,从而达到愈病的目的,正如吴昆《医方考》中所说:“情志过极,非药可愈,须以情胜,《内经》一言,百代宗之,是无形之药也”。不过,后世在运用情志相胜之时也并未完全拘泥五行相胜这一固定模式,如《懦门事亲·内伤形》之因忧结块的喜胜悲案、病怒不食的喜胜怒案、《惊门》的惊者平之案、《九气感疾更相为治术》之恐惧胜喜案、《续名医类案·癫狂》之以喜治愈因忧致癫案、《哭笑》之悲胜喜案等,其中有按五行相胜者治,亦有不按五行相胜者治。

情志相胜治病的机理主要有三,其一,七情的五行相胜规律,如上述怒胜思等;其二,七情阴阳属性的对立制约,如:喜为阳,怒为阴,喜可胜怒;其三,情志调整人体气机活动的规律,如因惊吓而气机紊乱之病,可以使其情绪逐渐平和,而改变其气机使之痊愈,亦属“惊者平之”之列。另外,运用情志相胜之法,并不局限于情志疾病,对于一些以气机紊乱病变为主的疾病,亦可使用情志疗法,如《三国志·魏书·方技传》之华佗以怒愈病案、《医部全录·医术名流列传·文挚》之以怒愈病案筹,即属此列。

在运用情志相胜疗法时,要注意刺激的强度,不论采用突然地强大刺激,还是釆用持续不断地强化刺激,都应既要有超过、压倒致病情志因素的强度,又不能超出病人所能接受的限度。

\subsubsection{祝由}%(二)

祝,告也;由,生病原由也。所谓祝由,即通过祝说发病原由,转移患者的精神,而达到调整病人气机,治愈疾病的方法,所以又称为“移精变气法”。“祝由”一词,出自《素问·移精变气论》,其曰:“余闻古之治病,惟其移精变气,可祝由而已”。并云:“往古人居禽兽之间,动作以避寒,阴居以避暑,内无眷慕之累,外无伸宦之形。此恬惔之世,邪不能深入也。故毒药不能治其内,针石不能治其外,故可移精祝由而已。”另外《灵枢·贼风》云:“其祝而已者,其故何也?岐伯曰:先巫者,因知百病之胜,先知其病之所从生者,可祝而已也。”祝由不仅要求施术者有一定的医学知识,且术前必须了解患者发病的原因,然后才能采用胜以制之的恰当方法进行治疗。因此,运用祝由治病,必须具备医学知识和较强的分析推理能力,并善于运用语言技巧以取得患者信任,才能移易精神,改变其情性,调整其气机。

祝由的适应范围,主要有二,其一,某些精神情志方面的疾患,如由于疑神猜鬼、妄识幻想、惊恐迷惑、情志不遂、深情爱恶等所致病证;其二,某些轻微病证,即如《素问·移精变气论》所言,对“邪不能深入”的“可移精祝由而已”,而对于“贼风数至,虚邪朝夕,内至五脏骨髓,外伤空窍肌肤”者,“祝由不能已也”。

\subsubsection{劝说开导}%(三)

劝说开导,即针对病人不同的思想实际和个性特征,对患者采取启发诱导、劝慰开导,以解除病人的思想顾虑,提高其战胜疾病的信心,并使之主动积极地配合医生进行治疗,从而促进身心的康复。其具体方法,《灵枢·师传》进行了详细说明,其曰:“人之情,莫不恶死而乐生,告之以其败,语之以其善,导之以其所便,开之以其所苦,虽有无道之人,恶有不听者乎?”即其一,“告之以其败”,指出疾病的危害,引起病人的注意,使病人对疾病有正确的认识和态度;其二,“语之以其善”,指出只要与医生配合,治疗及时,措施得当,是可以恢复健康的,以增强病人战胜疾病的信心;其三,“导之以其所便”,告诉病人如何进行调养,指出治疗的具体措施;其四,“开之以其所苦”,即解除病人消极的心理状态,放下思想包袱,克服内心的苦闷、焦虑和紧张。这种劝慰开导式心理疗法的心理学依据就是“人之情,莫不恶死而乐生”。当然,在进行劝说开导时,要求医生深入了解患者的内心世界,准确地掌握病人的心理活动。总之,这种方法就是要通过说服、解释、鼓励、安慰、保证等法,动之以情、晓之以理、喻之以实例、明之以方法,从而达到宽慰病人情怀、改变病人精神,使之神情康复的目的。

\subsubsection{暗示}%(四)

暗示疗法指釆用含蓄、间接的方式,对病人的心理状态施加影响,诱导病人不经理智考虑和判断,即“无形中”直接地接受医生的治疗性意见,主要树立某种信念,或改变其情绪和行为,从而达到治疗目的。暗示的方法一般多采用语言,也可用手势、表情,或采用暗示性物体(包括药物)来进行。《素问·调经论》云:“按摩勿释,出针视之,曰我将深之,适人必革,精气自伏,邪气自乱”。这是《内经》运用暗示方法的典型例子。前述“祝由”法,愈病的原理在一定程度上也含有暗示。

\subsubsection{顺情从欲}%(五)

顺情从欲,亦属心理疗法之一,即指顺从病人的意志、情绪,满足病人心身需要。人的需求满足与否,会直接影响人的情绪行为,影响气血的活动。必要的生活欲望不能得到满足,不仅影响正常生理活动,甚至会导致病变。对此,仅用劝导安慰、移情易性,甚至采取强行压制的方式,均难以解除病人痛苦,只有当其欲望满足时,疾病才有可能向愈,正如《灵枢·师传》说:“未有逆而能治之也,夫惟顺而已矣……百姓人民,皆欲顺其志也”。并要求“入国问俗,入家问讳,上堂问礼,临病人问所便”。当然,对于心理上的欲望,应当分析对待。若是合理的欲望,客观条件又允许,则应尽力满足,或对其想法表示同情、理解和支持、保证等;而对于那些淫欲邪念、不切实际的欲望,则不能纵容和迁就,而应善意地、诚恳地采用说服教育等方法进行处理。

\subsubsection{导引吐纳}%(六)

导引吐纳,今称之为“气功”,从某种角度而言,也属于心理疗法之一,即通过静心调神,结合调整呼吸,进而达到调身之目的。《素问·上古天真论》所提出的“恬惔虚无,真气从之”、“精神内守”、“呼吸精气”、“独立守神”、“和于术数”等,多属于此例。

《内经》提出的心理疗法还有很多,如释梦、解惑、澄心静志等,此不详述。

\section{《内经》心理医学思想在中医学中的意义及评价}%第四节

《内经》心理医学思想是古代心理学知识和医学相结合的产物,它对促进中医学理论发展和指导临床实践都发挥了积极的作用。其一,促进医学模式的改变。《内经》强调社会因素对人心理的影响,重视心理在健康中的地位,认为形神俱全才是健康之人,而心理因素必然在导致疾病和治疗疾病中均有重要作用,因而促进了自然——社会——心理——生物医学模式的形成。其二,促进病因学的发展。《内经》十分重视心理因素在疾病发生中的作用,把情志变化作为引起疾病的重要原因之一。这一点,即使从今天看来,仍不失为先进的医学理论。近年来,人们发现疾病的发生率并未随着医疗保健费用的增加而下降,其中重要原因之一,就是忽略了人类有着极为丰富而复杂的心理活动,许多疾病的发生均与心理因素有关。其三,提高疗效。《内经》心理医学思想提示人们,不仅要重视药物、针灸等方法的治疗作用,更要重视社会和心理治疗的效能,重视“神”在疗效中的关键作用,从而提高了临床疗效。其四,预防疾病,延长寿命。《内经》心理医学思想强调精神调摄,既可以减少心理因素的致病作用,又可以加强心理保健,提高心身素质,以达到预防疾病,延长寿命之目的。

\zuozhe{(翟双庆)}
\ifx \allfiles \undefined
\end{document}
\fi 