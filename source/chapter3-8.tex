% -*- coding: utf-8 -*-
%!TEX program = xelatex
\ifx \allfiles \undefined
%\documentclass[draft,12pt]{ctexbook}
\documentclass[12pt]{ctexbook}
%\usepackage{xeCJK}
%\usepackage[14pt]{extsizes} %支持8,9,10,11,12,14,17,20pt

%===================文档页面设置====================
%---------------------印刷版尺寸--------------------
%\usepackage[a4paper,hmargin={2.3cm,1.7cm},vmargin=2.3cm,driver=xetex]{geometry}
%--------------------电子版------------------------
\usepackage[a4paper,margin=2cm,driver=xetex]{geometry}
%\usepackage[paperwidth=9.2cm, paperheight=12.4cm, width=9cm, height=12cm,top=0.2cm,
%            bottom=0.4cm,left=0.2cm,right=0.2cm,foot=0cm, nohead,nofoot,driver=xetex]{geometry}

%===================自定义颜色=====================
\usepackage{xcolor}
  \definecolor{mybackgroundcolor}{cmyk}{0.03,0.03,0.18,0}
  \definecolor{myblue}{rgb}{0,0.2,0.6}

%====================字体设置======================
%--------------------中文字体----------------------
%-----------------------xeCJK下设置中文字体------------------------------%
\setCJKfamilyfont{song}{SimSun}                             %宋体 song
\newcommand{\song}{\CJKfamily{song}}                        % 宋体   (Windows自带simsun.ttf)
\setCJKfamilyfont{xs}{NSimSun}                              %新宋体 xs
\newcommand{\xs}{\CJKfamily{xs}}
\setCJKfamilyfont{fs}{FangSong_GB2312}                      %仿宋2312 fs
\newcommand{\fs}{\CJKfamily{fs}}                            %仿宋体 (Windows自带simfs.ttf)
\setCJKfamilyfont{kai}{KaiTi_GB2312}                        %楷体2312  kai
\newcommand{\kai}{\CJKfamily{kai}}
\setCJKfamilyfont{yh}{Microsoft YaHei}                    %微软雅黑 yh
\newcommand{\yh}{\CJKfamily{yh}}
\setCJKfamilyfont{hei}{SimHei}                                    %黑体  hei
\newcommand{\hei}{\CJKfamily{hei}}                          % 黑体   (Windows自带simhei.ttf)
\setCJKfamilyfont{msunicode}{Arial Unicode MS}            %Arial Unicode MS: msunicode
\newcommand{\msunicode}{\CJKfamily{msunicode}}
\setCJKfamilyfont{li}{LiSu}                                            %隶书  li
\newcommand{\li}{\CJKfamily{li}}
\setCJKfamilyfont{yy}{YouYuan}                             %幼圆  yy
\newcommand{\yy}{\CJKfamily{yy}}
\setCJKfamilyfont{xm}{MingLiU}                                        %细明体  xm
\newcommand{\xm}{\CJKfamily{xm}}
\setCJKfamilyfont{xxm}{PMingLiU}                             %新细明体  xxm
\newcommand{\xxm}{\CJKfamily{xxm}}

\setCJKfamilyfont{hwsong}{STSong}                            %华文宋体  hwsong
\newcommand{\hwsong}{\CJKfamily{hwsong}}
\setCJKfamilyfont{hwzs}{STZhongsong}                        %华文中宋  hwzs
\newcommand{\hwzs}{\CJKfamily{hwzs}}
\setCJKfamilyfont{hwfs}{STFangsong}                            %华文仿宋  hwfs
\newcommand{\hwfs}{\CJKfamily{hwfs}}
\setCJKfamilyfont{hwxh}{STXihei}                                %华文细黑  hwxh
\newcommand{\hwxh}{\CJKfamily{hwxh}}
\setCJKfamilyfont{hwl}{STLiti}                                        %华文隶书  hwl
\newcommand{\hwl}{\CJKfamily{hwl}}
\setCJKfamilyfont{hwxw}{STXinwei}                                %华文新魏  hwxw
\newcommand{\hwxw}{\CJKfamily{hwxw}}
\setCJKfamilyfont{hwk}{STKaiti}                                    %华文楷体  hwk
\newcommand{\hwk}{\CJKfamily{hwk}}
\setCJKfamilyfont{hwxk}{STXingkai}                            %华文行楷  hwxk
\newcommand{\hwxk}{\CJKfamily{hwxk}}
\setCJKfamilyfont{hwcy}{STCaiyun}                                 %华文彩云 hwcy
\newcommand{\hwcy}{\CJKfamily{hwcy}}
\setCJKfamilyfont{hwhp}{STHupo}                                 %华文琥珀   hwhp
\newcommand{\hwhp}{\CJKfamily{hwhp}}

\setCJKfamilyfont{fzsong}{Simsun (Founder Extended)}     %方正宋体超大字符集   fzsong
\newcommand{\fzsong}{\CJKfamily{fzsong}}
\setCJKfamilyfont{fzyao}{FZYaoTi}                                    %方正姚体  fzy
\newcommand{\fzyao}{\CJKfamily{fzyao}}
\setCJKfamilyfont{fzshu}{FZShuTi}                                    %方正舒体 fzshu
\newcommand{\fzshu}{\CJKfamily{fzshu}}

\setCJKfamilyfont{asong}{Adobe Song Std}                        %Adobe 宋体  asong
\newcommand{\asong}{\CJKfamily{asong}}
\setCJKfamilyfont{ahei}{Adobe Heiti Std}                            %Adobe 黑体  ahei
\newcommand{\ahei}{\CJKfamily{ahei}}
\setCJKfamilyfont{akai}{Adobe Kaiti Std}                            %Adobe 楷体  akai
\newcommand{\akai}{\CJKfamily{akai}}

%------------------------------设置字体大小------------------------%
\newcommand{\chuhao}{\fontsize{42pt}{\baselineskip}\selectfont}     %初号
\newcommand{\xiaochuhao}{\fontsize{36pt}{\baselineskip}\selectfont} %小初号
\newcommand{\yihao}{\fontsize{28pt}{\baselineskip}\selectfont}      %一号
\newcommand{\xiaoyihao}{\fontsize{24pt}{\baselineskip}\selectfont}
\newcommand{\erhao}{\fontsize{21pt}{\baselineskip}\selectfont}      %二号
\newcommand{\xiaoerhao}{\fontsize{18pt}{\baselineskip}\selectfont}  %小二号
\newcommand{\sanhao}{\fontsize{15.75pt}{\baselineskip}\selectfont}  %三号
\newcommand{\sihao}{\fontsize{14pt}{\baselineskip}\selectfont}%     四号
\newcommand{\xiaosihao}{\fontsize{12pt}{\baselineskip}\selectfont}  %小四号
\newcommand{\wuhao}{\fontsize{10.5pt}{\baselineskip}\selectfont}    %五号
\newcommand{\xiaowuhao}{\fontsize{9pt}{\baselineskip}\selectfont}   %小五号
\newcommand{\liuhao}{\fontsize{7.875pt}{\baselineskip}\selectfont}  %六号
\newcommand{\qihao}{\fontsize{5.25pt}{\baselineskip}\selectfont}    %七号   %中文字体及字号设置
\xeCJKDeclareSubCJKBlock{SIP}{
  "20000 -> "2A6DF,   % CJK Unified Ideographs Extension B
  "2A700 -> "2B73F,   % CJK Unified Ideographs Extension C
  "2B740 -> "2B81F    % CJK Unified Ideographs Extension D
}
%\setCJKmainfont[SIP={[AutoFakeBold=1.8,Color=red]Sun-ExtB},BoldFont=黑体]{宋体}    % 衬线字体 缺省中文字体

\setCJKmainfont{simsun.ttc}[
  Path=fonts/,
  SIP={[Path=fonts/,AutoFakeBold=1.8,Color=red]simsunb.ttf},
  BoldFont=simhei.ttf
]

%SimSun-ExtB
%Sun-ExtB
%AutoFakeBold:自动伪粗,即正文使用\bfseries时生僻字使用伪粗体;
%FakeBold:强制伪粗,即正文中生僻字均使用伪粗体
%\setCJKmainfont[BoldFont=STHeiti,ItalicFont=STKaiti]{STSong}
%\setCJKsansfont{微软雅黑}黑体
%\setCJKsansfont[BoldFont=STHeiti]{STXihei} %serif是有衬线字体sans serif 无衬线字体
%\setCJKmonofont{STFangsong}    %中文等宽字体

%--------------------英文字体----------------------
\setmainfont{simsun.ttc}[
  Path=fonts/,
  BoldFont=simhei.ttf
]
%\setmainfont[BoldFont=黑体]{宋体}  %缺省英文字体
%\setsansfont
%\setmonofont

%===================目录分栏设置====================
\usepackage[toc,lof,lot]{multitoc}    % 目录(含目录、表格目录、插图目录)分栏设置
  %\renewcommand*{\multicolumntoc}{3} % toc分栏数设置,默认为两栏(\multicolumnlof,\multicolumnlot)
  %\setlength{\columnsep}{1.5cm}      % 调整分栏间距
  \setlength{\columnseprule}{0.2pt}   % 调整分栏竖线的宽度

%==================章节格式设置====================
\setcounter{secnumdepth}{3} % 章节等编号深度 3:子子节\subsubsection
\setcounter{tocdepth}{2}    % 目录显示等度 2:子节

\xeCJKsetup{%
  CJKecglue=\hspace{0.15em},      % 调整中英(含数字)间的字间距
  %CJKmath=true,                  % 在数学环境中直接输出汉字(不需要\text{})
  AllowBreakBetweenPuncts=true,   % 允许标点中间断行,减少文字行溢出
}

\ctexset{%
  part={
    name={,篇},
    number=\SZX{part},
    format={\chuhao\bfseries\centering},
    nameformat={},titleformat={}
  },
  section={
    number={\chinese{section}},
    name={第,节}
  },
  subsection={
    number={\chinese{subsection}、},
    aftername={\hspace{-0.01em}}
  },
  subsubsection={
    number={(\chinese{subsubsection})},
    aftername={\hspace {-0.01em}},
    beforeskip={1.3ex minus .8ex},
    afterskip={1ex minus .6ex},
    indent={\parindent}
  },
  paragraph={
    beforeskip=.1\baselineskip,
    indent={\parindent}
  }
}

\newcommand*\SZX[1]{%
  \ifcase\value{#1}%
    \or 上%
    \or 中%
    \or 下%
  \fi
}

%====================页眉设置======================
\usepackage{titleps}%或者\usepackage{titlesec},titlesec包含titleps
\newpagestyle{special}[\small\sffamily]{
  %\setheadrule{.1pt}
  \headrule
  \sethead[\usepage][][\chaptertitle]
  {\chaptertitle}{}{\usepage}
}

\newpagestyle{main}[\small\sffamily]{
  \headrule
  %\sethead[\usepage][][第\thechapter 章\quad\chaptertitle]
%  {\thesection\quad\sectiontitle}{}{\usepage}}
  \sethead[\usepage][][第\chinese{chapter}章\quad\chaptertitle]
  {第\chinese{section}节\quad\sectiontitle}{}{\usepage}
}

\newpagestyle{main2}[\small\sffamily]{
  \headrule
  \sethead[\usepage][][第\chinese{chapter}章\quad\chaptertitle]
  {第\chinese{section}節\quad\sectiontitle}{}{\usepage}
}

%================ PDF 书签设置=====================
\usepackage{bookmark}[
  depth=2,        % 书签深度 2:子节
  open,           % 默认展开书签
  openlevel=2,    % 展开书签深度 2:子节
  numbered,       % 显示编号
  atend,
]
  % 相比hyperref,bookmark宏包大多数时候只需要编译一次,
  % 而且书签的颜色和字体也可以定制。
  % 比hyperref 更专业 (自动加载hyperref)

%\bookmarksetup{italic,bold,color=blue} % 书签字体斜体/粗体/颜色设置

%------------重置每篇章计数器,必须在hyperref/bookmark之后------------
\makeatletter
  \@addtoreset{chapter}{part}
\makeatother

%------------hyperref 超链接设置------------------------
\hypersetup{%
  pdfencoding=auto,   % 解决新版ctex,引起hyperref UTF-16预警
  colorlinks=true,    % 注释掉此项则交叉引用为彩色边框true/false
  pdfborder=001,      % 注释掉此项则交叉引用为彩色边框
  citecolor=teal,
  linkcolor=myblue,
  urlcolor=black,
  %psdextra,          % 配合使用bookmark宏包,可以直接在pdf 书签中显示数学公式
}

%------------PDF 属性设置------------------------------
\hypersetup{%
  pdfkeywords={黄帝内经,内经,内经讲义,21世纪课程教材},    % 关键词
  %pdfsubject={latex},        % 主题
  pdfauthor={主编:王洪图},   % 作者
  pdftitle={内经讲义},        % 标题
  %pdfcreator={texlive2011}   % pdf创建器
}

%------------PDF 加密----------------------------------
%仅适用于xelatex引擎 基于xdvipdfmx
%\special{pdf:encrypt ownerpw (abc) userpw (xyz) length 128 perm 2052}

%仅适用于pdflatex引擎
%\usepackage[owner=Donald,user=Knuth,print=false]{pdfcrypt}

%其他可使用第三方工具 如:pdftk
%pdftk inputfile.pdf output outputfile.pdf encrypt_128bit owner_pw yourownerpw user_pw youruserpw

%=============自定义环境、列表及列表设置================
% 标题
\def\biaoti#1{\vspace{1.7ex plus 3ex minus .2ex}{\bfseries #1}}%\noindent\hei
% 小标题
\def\xiaobt#1{{\bfseries #1}}
% 小结
\def\xiaojie {\vspace{1.8ex plus .3ex minus .3ex}\centerline{\large\bfseries 小\ \ 结}\vspace{.1\baselineskip}}
% 作者
\def\zuozhe#1{\rightline{\bfseries #1}}

\newcounter{yuanwen}    % 新计数器 yuanwen
\newcounter{jiaozhu}    % 新计数器 jiaozhu

\newenvironment{yuanwen}[2][【原文】]{%
  %\biaoti{#1}\par
  \stepcounter{yuanwen}   % 计数器 yuanwen+1
  \bfseries #2}
  {}

\usepackage{enumitem}
\newenvironment{jiaozhu}[1][【校注】]{%
  %\biaoti{#1}\par
  \stepcounter{jiaozhu}   % 计数器 jiaozhu+1
  \begin{enumerate}[%
    label=\mylabel{\arabic*}{\circledctr*},before=\small,fullwidth,%
    itemindent=\parindent,listparindent=\parindent,%labelsep=-1pt,%labelwidth=0em,
    itemsep=0pt,topsep=0pt,partopsep=0pt,parsep=0pt
  ]}
  {\end{enumerate}}

%===================注解与原文相互跳转====================
%----------------第1部分 设置相互跳转锚点-----------------
\makeatletter
  \protected\def\mylabel#1#2{% 注解-->原文
    \hyperlink{back:\theyuanwen:#1}{\Hy@raisedlink{\hypertarget{\thejiaozhu:#1}{}}#2}}

  \protected\def\myref#1#2{% 原文-->注解
    \hyperlink{\theyuanwen:#1}{\Hy@raisedlink{\hypertarget{back:\theyuanwen:#1}{}}#2}}
  %此处\theyuanwen:#1实际指thejiaozhu:#1,只是\thejiaozhu计数器还没更新,故使用\theyuanwen计数器代替
\makeatother

\protected\def\myjzref#1{% 脚注中的引用(引用到原文)
  \hyperlink{\theyuanwen:#1}{\circlednum{#1}}}

\def\sb#1{\myref{#1}{\textsuperscript{\circlednum{#1}}}}    % 带圈数字上标

%----------------第2部分 调整锚点垂直距离-----------------
\def\HyperRaiseLinkDefault{.8\baselineskip} %调整锚点垂直距离
%\let\oldhypertarget\hypertarget
%\makeatletter
%  \def\hypertarget#1#2{\Hy@raisedlink{\oldhypertarget{#1}{#2}}}
%\makeatother

%====================带圈数字列表标头====================
\newfontfamily\circledfont[Path = fonts/]{meiryo.ttc}  % 日文字体,明瞭体
%\newfontfamily\circledfont{Meiryo}  % 日文字体,明瞭体

\protected\def\circlednum#1{{\makexeCJKinactive\circledfont\textcircled{#1}}}

\newcommand*\circledctr[1]{%
  \expandafter\circlednum\expandafter{\number\value{#1}}}
\AddEnumerateCounter*\circledctr\circlednum{1}

% 参考自:http://bbs.ctex.org/forum.php?mod=redirect&goto=findpost&ptid=78709&pid=460496&fromuid=40353

%======================插图/tikz图========================
\usepackage{graphicx,subcaption,wrapfig}    % 图,subcaption含子图功能代替subfig,图文混排
  \graphicspath{{img/}}                     % 设置图片文件路径

\def\pgfsysdriver{pgfsys-xetex.def}         % 设置tikz的驱动引擎
\usepackage{tikz}
  \usetikzlibrary{calc,decorations.text,arrows,positioning}

%---------设置tikz图片默认格式(字号、行间距、单元格高度)-------
\let\oldtikzpicture\tikzpicture
\renewcommand{\tikzpicture}{%
  \small
  \renewcommand{\baselinestretch}{0.2}
  \linespread{0.2}
  \oldtikzpicture
}

%=========================表格相关===============================
\usepackage{%
  multirow,                   % 单元格纵向合并
  array,makecell,longtable,   % 表格功能加强,tabu的依赖
  tabu-last-fix,              % "强大的表格工具" 本地修复版
  diagbox,                    % 表头斜线
  threeparttable,             % 表格内脚注(需打补丁支持tabu,longtabu)
}

%----------给threeparttable打补丁用于tabu,longtabu--------------
%解决方案来自:http://bbs.ctex.org/forum.php?mod=redirect&goto=findpost&ptid=80318&pid=467217&fromuid=40353
\usepackage{xpatch}

\makeatletter
  \chardef\TPT@@@asteriskcatcode=\catcode`*
  \catcode`*=11
  \xpatchcmd{\threeparttable}
    {\TPT@hookin{tabular}}
    {\TPT@hookin{tabular}\TPT@hookin{tabu}}
    {}{}
  \catcode`*=\TPT@@@asteriskcatcode
\makeatother

%------------设置表格默认格式(字号、行间距、单元格高度)------------
\let\oldtabular\tabular
\renewcommand{\tabular}{%
  \renewcommand\baselinestretch{0.9}\small    % 设置行间距和字号
  \renewcommand\arraystretch{1.5}             % 调整单元格高度
  %\renewcommand\multirowsetup{\centering}
  \oldtabular
}
%设置行间距,且必须放在字号设置前 否则无效
%或者使用\fontsize{<size>}{<baseline>}\selectfont 同时设置字号和行间距

\let\oldtabu\tabu
\renewcommand{\tabu}{%
  \renewcommand\baselinestretch{0.9}\small    % 设置行间距和字号
  \renewcommand\arraystretch{1.8}             % 调整单元格高度
  %\renewcommand\multirowsetup{\centering}
  \oldtabu
}

%------------模仿booktabs宏包的三线宽度设置---------------
\def\toprule   {\Xhline{.08em}}
\def\midrule   {\Xhline{.05em}}
\def\bottomrule{\Xhline{.08em}}
%-------------------------------------
%\setlength{\arrayrulewidth}{2pt} 设定表格中所有边框的线宽为同样的值
%\Xhline{} \Xcline{}分别设定表格中水平线的宽度 makecell包提供

%表格中垂直线的宽度可以通过在表格导言区(preamble),利用命令 !{\vrule width1.2pt} 替换 | 即可

%=================图表设置===============================
%---------------图表标号设置-----------------------------
\renewcommand\thefigure{\arabic{section}-\arabic{figure}}
\renewcommand\thetable {\arabic{section}-\arabic{table}}

\usepackage{caption}
  \captionsetup{font=small,}
  \captionsetup[table] {labelfont=bf,textfont=bf,belowskip=3pt,aboveskip=0pt} %仅表格 top
  \captionsetup[figure]{belowskip=0pt,aboveskip=3pt}  %仅图片 below

%\setlength{\abovecaptionskip}{3pt}
%\setlength{\belowcaptionskip}{3pt} %图、表题目上下的间距
\setlength{\intextsep}   {5pt}  %浮动体和正文间的距离
\setlength{\textfloatsep}{5pt}

%====================全文水印==========================
%解决方案来自:
%http://bbs.ctex.org/forum.php?mod=redirect&goto=findpost&ptid=79190&pid=462496&fromuid=40353
%https://zhuanlan.zhihu.com/p/19734756?columnSlug=LaTeX
\usepackage{eso-pic}

%eso-pic中\AtPageCenter有点水平偏右
\renewcommand\AtPageCenter[1]{\parbox[b][\paperheight]{\paperwidth}{\vfill\centering#1\vfill}}

\newcommand{\watermark}[3]{%
  \AddToShipoutPictureBG{%
    \AtPageCenter{%
      \tikz\node[%
        overlay,
        text=red!50,
        %font=\sffamily\bfseries,
        rotate=#1,
        scale=#2
      ]{#3};
    }
  }
}

\newcommand{\watermarkoff}{\ClearShipoutPictureBG}

\watermark{45}{15}{草\ 稿}    %启用全文水印

%=============花括号分支结构图=========================
\usepackage{schemata}

\xpatchcmd{\schema}
  {1.44265ex}{-1ex}
  {}{}

\newcommand\SC[2] {\schema{\schemabox{#1}}{\schemabox{#2}}}
\newcommand\SCh[4]{\Schema{#1}{#2}{\schemabox{#3}}{\schemabox{#4}}}

%=======================================================

\begin{document}
\pagestyle{main}
\fi
\chapter{五运六气学说}%第八章

五运六气,简称运气,是我国古代研究天时气候及其对生物和人体影响的一门学说。基于“天人相应”的观点,运气学说以天时气候变化,以及人体对其变化的反应为基础,从整体角度和宇宙节律方面探讨天时气候变化与人体健康和疾病的关系,因而对疾病的预防和治疗具有较大的指导意义。

\section{运气学说的内容}%第一节

运气学说主要由“五运”和“六气”组成,是运用阴阳、五运、六气等理论,并以天干、地支作为演绎工具符号,来推测气候变化、生物生化及其与疾病流行之间的关系。其基本内容包括干支甲子、五运、六气和运气同化等方面。《内经》主要见于《素问》的《六节藏象论》和《天元纪大论》、《五运行大论》、《六微旨大论》、《气交变大论》、《五常政大论》、《六元正纪大论》、《至真要大论》等七篇大论之中。

\subsection{干支甲子}%一、

干支,是天干地支的简称。甲子,是因天干始于甲,地支始于子,干支相合而得名。

天干,即甲、乙、丙、丁、戊、己、庚、辛、壬、癸,又称十天干或十干,最早是用来纪日的。地支,即子、丑、寅、卯、辰、巳、午、未、申、酉、戌、亥,又称十二地支或十二支,最早是用来纪月的。

\subsubsection{干支的阴阳属性}%(一)

干支各具有不同的阴阳属性。一般而言,天干属阳,地支属阴。然而在天干、地支中又可再分阴阳,其划分是以奇偶数为依据的,即:天干之中,甲、丙、戊、庚、壬属阳,又称阳干,乙、丁、己、辛、癸属阴,又称阴干;地支之中,子、寅、辰、午、申、戌属阳,又称阳支,丑、卯、巳、未、酉、亥属阴,又称阴支。(表\ref{tab:干支明阳属性表})

\begin{table}[htb]%干支明阳属性表
	\centering
	\caption{干支明阳属性表}\label{tab:干支明阳属性表}
	\begin{tabu}to.87\textwidth{X[2c]|X[2c]|*6{X[c]}}
		\toprule
		\multirow{2}{*}{天干} & 阳 & 甲 & 丙 & 戊 & 庚 & 壬 &    \\ \cline{2-8}
							  & 阴 & 乙 & 丁 & 己 & 辛 & 癸 &    \\
		\midrule
		\multirow{2}{*}{地支} & 阳 & 子 & 寅 & 辰 & 午 & 申 & 戌 \\ \cline{2-8}
							  & 阴 & 丑 & 卯 & 巳 & 未 & 酉 & 亥 \\
		\bottomrule
	\end{tabu}
\end{table}

\subsubsection{干支的五行属性}%(二)

干支的五行属性是,在天干中,甲乙属木,丙丁属火,戊己属土,庚辛属金,壬癸属水;在地支中,寅卯属木,巳未属火,申酉属金,亥子属水,辰戌丑未属土。(表\ref{tab:干支五行属性表})

\begin{table}[htb]%干支五行属性表
	\centering
	\caption{干支五行属性表}\label{tab:干支五行属性表}
	\begin{tabu}to.87\textwidth{X[-2,c]*5{|X[c]}}
		\toprule
		\diagbox[height=2.5em]
			{干支}{五行}& 木   & 火   & 土       & 金   & 水  \\
		\midrule
			天干        & 甲乙 & 丙丁 & 戊己     & 庚辛 & 壬癸 \\ \hline
			地干        & 寅卯 & 巳未 & 辰未戌丑 & 申酉 & 亥子 \\
		\bottomrule
	\end{tabu}
\end{table}

天干与五行的配属,是以五行之气的性质,结合五方五时生物生长收藏的规律为依据而确立的。如甲乙属东方,东方为木位,应于春季,春气主生,万物萌发;甲乙为万物初生,破甲乙屈之象,故属木。余可类推。

地支配属五行,主要是根据方位与月建(北斗星的斗纲所指)来确定的。依月建所指,每年农历的正月属寅,二月属卯,三月属辰,四月属巳,五月属午,六月属未,七月属申,八月属酉,九月属戌,十月属亥,冬月属子,腊月属丑。因木为东方之气,旺于春,寅卯月建是正、二月,位于东方,所以寅卯属木。火是南方之气,旺于夏,巳午的月建是四、五月,位于南方,所以巳午属火。金是西方之气,旺于秋,申酉的月建是七、八月,位于西方,所以申酉属金。水是北方之气,旺于冬,亥子的月建是十、十一月,位于北方,所以亥子属水。土为中央之气,寄旺于四季之末各十八日,辰未戌丑建于三、六、九、十二月,位于四方,所以辰未戌丑均属土。

此外,还可以根据常年气候运动规律来确定天干地支的五行属性。(表\ref{tab:天干纪运、地支纪气表})

\begin{table}[htb]%天干纪运、地支纪气表
	\centering
	\caption{天干纪运、地支纪气表}\label{tab:天干纪运、地支纪气表}
	\begin{tabu}to.87\textwidth{X[c]*5{|X[c]}}
		\toprule
					 五行所属 & 土 & 金 & 水 & 木 & 火     \\
		\midrule
		\multirow{2}{*}{天干} & 甲 & 乙 & 丙 & 丁 & 戊     \\
							  & 己 & 庚 & 辛 & 壬 & 癸     \\ \hline
		\multirow{2}{*}{地支} & 丑 & 卯 & 辰 & 巳 & 子\ 寅 \\
							  & 未 & 酉 & 戌 & 亥 & 午\ 申 \\
		\bottomrule
	\end{tabu}
\end{table}

表中,天干的五行配属是根据十干化运来确定的,而地支的五行配属则是根据地支化气来确定的。

\subsubsection{甲子}%(三)

甲子,是以天干和地支配合起来纪日纪月纪时纪年的一种方法。《素问·六微旨大论》说:“天气始于甲,地气始于子,子甲相合,命日岁立,谨候其时,气可与期。”具体的配合方法是:天干在上,地支在下,按着干支原有的次序,依次相加,五个阳干在上与六个阴支在下相配,阴阳相合,则成五、六之数,天干周转六次,地支周转五次,合成六十甲子,为甲子一周。(表\ref{tab:六十甲子表})

\begin{table}[htb]%六十甲子表
	\centering
	\caption{六十甲子表}\label{tab:六十甲子表}
	\begin{tabu}to.87\textwidth{X[1.3,c]*{10}{|X[c]}}
		\toprule
		天干 & 甲 & 乙 & 丙 & 丁 & 戊 & 己 & 庚 & 辛 & 壬 & 癸 \\
		地支 & 子 & 丑 & 寅 & 卯 & 辰 & 巳 & 午 & 未 & 申 & 酉 \\ \hline
		天干 & 甲 & 乙 & 丙 & 丁 & 戊 & 己 & 庚 & 辛 & 壬 & 癸 \\
		地支 & 戌 & 亥 & 子 & 丑 & 寅 & 卯 & 辰 & 巳 & 午 & 未 \\ \hline
		天干 & 甲 & 乙 & 丙 & 丁 & 戊 & 己 & 庚 & 辛 & 壬 & 癸 \\
		地支 & 申 & 酉 & 戌 & 亥 & 子 & 丑 & 寅 & 卯 & 辰 & 巳 \\ \hline
		天干 & 甲 & 乙 & 丙 & 丁 & 戊 & 己 & 庚 & 辛 & 壬 & 癸 \\
		地支 & 午 & 未 & 申 & 酉 & 戌 & 亥 & 子 & 丑 & 寅 & 卯 \\ \hline
		天干 & 甲 & 乙 & 丙 & 丁 & 戊 & 己 & 庚 & 辛 & 壬 & 癸 \\
		地支 & 辰 & 巳 & 午 & 未 & 申 & 酉 & 戌 & 亥 & 子 & 丑 \\ \hline
		天干 & 甲 & 乙 & 丙 & 丁 & 戊 & 己 & 庚 & 辛 & 壬 & 癸 \\
		地支 & 寅 & 卯 & 辰 & 巳 & 午 & 未 & 申 & 酉 & 戌 & 亥 \\
		\bottomrule
	\end{tabu}
\end{table}

在甲子纪年中,每年都由一个干支甲子为纪年符号。

六十甲子顺次而转,如环本端,无有止息。正如《素问·天元纪大论》所说:“天以六为节,地以五为制,周天气者,六期为一备,终地纪者,五岁为一周,……五六相合,而七百二十岁为一纪,凡三十岁;千四百四十气,凡六十岁而为一周,不及太过司皆见矣。”古人用甲子来纪日纪月纪时纪年,并推算四时节气,而运气学说则是以甲子作为演绎工具,来推算运气盛衰和测知气候变化。

\subsection{五运}%二、

五运,是木运、火运、土运、金运、水运的简称,具体指木、火、土、金、水五行之气在天地间的运行变化。由于五行与四季气候的关系是春温属木、夏热属火、长夏湿属土、秋燥属金、冬寒属水,所以五运实质上概括了一年四季的不同气候变化特征。同时,五运还可表示不同年份和不同节令的气候变化。五运具体有岁运、主运、客运的不同。

\subsubsection{岁运}%(一)

岁运又称中运、太运,统管全年岁气的变化,由于它能反映全年的气候特征、物化特点及发病规律,所以称为岁运。五行之气处于天地气机升降之中,因而又名之为中运。《素问·六元正纪大论》说:“天气不足,地气随之,地气不足,天气随之。运居其中而常先也。”岁运主年,与主运之分主一年之五时不同,故又称为大运。

岁运是根据十干统运来确定的。《素问·六元正纪大论》说:“先立其年,以明其气。”意谓先明确当年的干支甲子,再依据其年干确定岁运之气。具体的配属方法,正如《素问·天元纪大论》所说:“甲己之岁,土运统之;乙庚之岁,金运统之;丙辛之岁,水运统之;丁壬之岁,木运统之;戊癸之岁,火运统之。”这种据天干配五行以定岁运的方法,称为“十干统运”,亦称“十干纪运”。

十干统五运的道理,《内经》提出了五气经天之说。《素问·五运行大论》说:“臣览《太始天元册》文,丹天之气经于牛女戊分;黅天之气,经于心尾己分;苍天之气,经于危室柳鬼;素天之气,绘于亢氐昂毕;玄天之气,经于张翼娄胃。所谓戊己分者,奎壁角轸,则天地之门户也。夫候之所始,道之所生,不可不通也。”由此可见,十干化五运是由二十八宿在天体上的方位来确定的。所谓丹、黅、苍、素、玄,即红、黄、青、白、黑五色之气;牛、女、心、尾等乃古代二十八宿之名。(图\ref{fig:五气经天图})

%\begin{figure}[htb]
%  \centering
%  % Requires \usepackage{graphicx}
%  \includegraphics[width=0.50\textwidth]{五气经天图.png}\\
%  \caption{五气经天图}\label{fig:五气经天图}
%\end{figure}
\begin{figure}[htb]%五气经天图
	\centering
	\begin{tikzpicture}
	\def\nj{0.3}
	\def\bc{0.7}

	\foreach \r in {1,3,4,...,6} { \draw (0,0) circle(\nj+\r*\bc); } %画同心圆
	\draw[<->,dashed] (90-6.5*360/28:\nj+\bc*3) to node [very near start,fill=white,sloped,rotate=90]{素天}(90+5.5*360/28:\nj+\bc*3);
	\draw[<->,dashed,bend right=9] (90-1.5*360/28:\nj+\bc*3) to node [very near start,fill=white,sloped,rotate=90]{苍天}(-5.5*360/28:\nj+\bc*3);
	\draw[<->,dashed,bend right=11] (90+1.5*360/28:\nj+\bc*3) to node [near start,fill=white,sloped,rotate=-90]{玄天}(-1.5*360/28:\nj+\bc*3);
	\draw[<->,dashed,bend right] (-45:\nj+\bc*3) to node [near end,fill=white,sloped,rotate=-90]{丹天}(-45-5*360/28:\nj+\bc*3);
	\draw[<->,dashed,bend left] (90+45:\nj+\bc*3) to node [midway,fill=white,sloped,rotate=-90]{黅天}(90+45+5*360/28:\nj+\bc*3);
	\foreach \i in {1,...,4}{
		\pgfmathsetmacro{\angle}{45-360/4*(\i-1)}
		\draw (\angle:\nj+\bc*3) -- (\angle:\nj+\bc*4);
	}
	\foreach \secondlayer [count=\i] in {
		角,亢,氐,房,心,尾,箕,
		斗,牛,女,虚,危,室,壁,
		奎,娄,胃,昴,毕,觜,参,
		井,鬼,柳,星,张,翼,轸,
	}{
		\pgfmathsetmacro{\angle}{135+360/28/2+360/28*(\i-1)}
		\node[rotate=90+\angle] at (\angle:\nj+\bc*3.5-0.05) {\secondlayer};
	}
	\foreach \thirdlayer [count=\i] in {
		酉,辛,戊,乾,亥,壬,子,癸,
		丑,艮,寅,甲,卯,乙,辰,巽,
		巳,丙,午,丁,未,坤,申,庚,
	}{
		\pgfmathsetmacro{\angle}{-360/24*(\i-1)}
		\node[rotate=90+\angle] at (\angle:\nj+\bc*4.5-0.05) {\thirdlayer};
	}
	\draw (45:\nj+\bc*5) -- (45:\nj+\bc*6);%画等分线
	\draw (45-180:\nj+\bc*5) -- (45-180:\nj+\bc*6);%画等分线
	\node[rotate=-90-45] at (90+45:\nj+\bc*5.5-0.05) {己分};
	\node[rotate=45] at (-45:\nj+\bc*5.5-0.05) {戊分};
	\foreach \thirdlayer [count=\i] in {
		东,南,西,北,
	}{
		\pgfmathsetmacro{\angle}{-360/4*(\i+1)}
		\node[rotate=90+\angle] at (\angle:\nj+\bc*5.5-0.05) {\thirdlayer};
	}
	\node[rotate=-90-45] at (90+45:\nj+\bc*6.5-0.05) {地户};
	\node[rotate=45] at (-45:\nj+\bc*6.5-0.05) {天门};
	\end{tikzpicture}
	\caption{五气经天图}\label{fig:五气经天图}
\end{figure}

据五气经天图所见,牛、女二宿在北方偏东之癸位,奎、壁二宿当西北方戊位,“丹天之气经于牛女戊分”,所以戊癸主火运;心,尾二宿当东方偏北之甲位,角、轸二宿当东南方己位,“黅天之气经于心尾己分”,所以甲己主土运;危、室二宿当北方偏西之壬位,柳、鬼二宿当南方偏西之丁位,“苍天之气经于危室柳鬼”,所以丁壬主木运;亢、氐二宿当东方偏南之乙位,昴、毕二宿当西方偏南之庚位,“素天之气经于亢氐昴毕”,所以乙庚主金运;张、翼二宿位于南方偏东之丙位,娄、胃二宿位于西方偏北之辛位,“玄天之气经于张翼娄胃”,所以丙辛主水运。可见,五气经天说是以天文知识为基础提出的。

图中天门、地户的命名,是依据太阳在天体的位置及时令气候的变化而来的。当太阳的周年视运动位于奎、壁二宿时,正值春分,乃由春入夏之时,是一年中白昼变长的开始,此时,温气流行,万物生长,阳气开启,故曰天门。角、轸二宿为巽位己分,乃由秋入冬之时,见一年白昼变短的开始,此时凉气流行,万物收藏,阳气始敛,故称地户。

各年的岁运不同,以五行相生的次序轮转,每五年循环一周。

\subsubsection{主运}%(二)

主运,指分别主治一年五时的五运之气。由于它反映一年五时气候的正常变化,年年如此,固定不变,所以称为主运。

每年的主运分为木运、火运、土运、金运、水运五种,以五行相生的次序,始于木而终于水。五运中每运主一时,即各主73日零5刻,计365日零25刻,与周天之数相合。各主运的交运时间分别是:初运木运在大寒节当日,二运火运在春分节后13日,三运土运在芒种后10日,四运金运在处暑后7日,五运水运在立冬后4日。(图\ref{fig:五运主运图})

%\begin{figure}[htb]
%	\centering
%	% Requires \usepackage{graphicx}
%	\includegraphics[width=0.50\textwidth]{五运主运图.png}\\
%	\caption{五运主运图}\label{fig:五运主运图}
%\end{figure}
\begin{figure}[htb]%五运主运图
	\centering
	\begin{tikzpicture}
	\def\nj{0.25}
	\def\bc{0.65}

	\foreach \r in {1,...,6} { \draw (0,0) circle(\nj+\r*\bc); } %画同心圆
	\foreach \firstlayer/\thirdlayer/\fifthlayer [count=\i] in {
		土/三运/芒种后十日交,金/四运/处暑后七日交,水/终运/立冬后四日交,
		木/初运/大寒交日,火/二运/春分后十三日交
	}{
		\pgfmathsetmacro{\angle}{90-360/5*\i+360/5/2}
		\draw (\angle-360/5/2:\nj+\bc) -- (\angle-360/5/2:\nj+\bc*2);%第1层等分线
		\draw (\angle-360/5/2:\nj+\bc*3) -- (\angle-360/5/2:\nj+\bc*4);%第3 层等分线
		\draw (\angle-360/5/2:\nj+\bc*5) -- (\angle-360/5/2:\nj+\bc*6);%第5 层等分线
		\node[rotate=90+\angle] at (\angle:\nj+\bc*1.5-0.05) {\firstlayer};
		%\node[rotate=90+\angle] at (\angle:\nj+\bc*3.5-0.05) {\thirdlayer};
		\draw[decorate, decoration={text align=center, text along path,
			text={\fifthlayer}}] (\angle-360/5:\nj+\bc*5.5+0.1) arc (\angle-360/5:\angle+360/5:\nj+\bc*5.5+0.1);
		%\node[rotate=-90+\angle] at (\angle:\nj+\bc*5.5-0.05) {\fifthlayer};
	}
	\foreach \secondlayer/\thirdlayer/\fourthlayer [count=\i] in {
%		甲/太宫,己/少宫,庚/太商,乙/少商,丙/太羽,
%		辛/少羽,壬/太角,丁/少角,戊/太徵,癸/少徵,
		甲/运/太宫,己/三/少宫,庚/运/太商,乙/四/少商,丙/运/太羽,
		辛/终/少羽,壬/运/太角,丁/初/少角,戊/运/太徵,癸/二/少徵,
	}{
		\pgfmathsetmacro{\angle}{90-360/10*\i+360/10/2}
		\draw (\angle-360/10/2:\nj+\bc*2) -- (\angle-360/10/2:\nj+\bc*3);% 第2层等分线
		\draw (\angle-360/10/2:\nj+\bc*4) -- (\angle-360/10/2:\nj+\bc*5);% 第4层等分线
		\node[rotate=90+\angle] at (\angle:\nj+\bc*2.5-0.05) {\secondlayer};
		\node[rotate=90+\angle] at (\angle:\nj+\bc*3.5-0.05) {\thirdlayer};
		\node[rotate=90+\angle] at (\angle:\nj+\bc*4.5-0.05) {\fourthlayer};
	}
	\end{tikzpicture}
	\caption{五运主运图}\label{fig:五运主运图}
\end{figure}

主运分主五时,虽然常年不变,但主运五步却有太过不及的变化。在推算时,必须运用“五音建运”、“太少相生”和“五步推运”的方法。

1.五音建运:五音,即角、徵、宫、商、羽。为了推算方便,分别将五音建于五运之中,并用五音代表五运,具体是角为木音,徵为火音,宫为土音,商为金音,羽为木音,然后根据五音的太少来推求主运五步的太过和不及。五音建运不仅适用于主运,而且也同样适用于客运。

2.太少相生:太,即太过、有余;少,即不及、不足。五运的十干分阴阳,凡阳干属太,阴干属少。五音建五运亦有太、少之分,其分属是:甲己土运主宫音,甲属阳土为太宫,己属阴土为少宫;乙庚金运主商音,乙属阴金为少商,庚属阳金为太商;丙辛水运主羽音,丙属阳水为太羽,辛属阴水为少羽;丁壬木运主角音,丁属阴木为少角,壬属阳木为太角;戊癸火运主徵音,戌属阳火为太徵,癸属阴火为少徵。太少相生,是指建于五运的五音太少,按照五行相生的顺序发生变化,以成运气阴阳变化之理。正如张介宾所说:“盖太者属阳,少者属阴,阴以生阳,阳以生阴,一动一静,乃成易道。故甲以阳土,生乙之少商;乙以阴金,生丙之太羽;丙以阳水,生丁之少角;丁以阴木,生戊之太徵;戊以阳火,生己之少宫;己以阴土,生庚之太商;庚以阳金,生辛之少羽;辛以阴水,生壬之太角;壬以阳木,生癸之少徵;癸以阴火,复生甲之太宫。”(图\ref{fig:五音建运太少相生图})

%\begin{figure}[htb]
%	\centering
%	% Requires \usepackage{graphicx}
%	\includegraphics[width=0.50\textwidth]{五音建运太少相生图.png}\\
%	\caption{五音建运太少相生图}\label{fig:五音建运太少相生图}
%\end{figure}
\begin{figure}[htb]%五音建运太少相生图
	\centering
	\begin{tikzpicture}
	\def\nj{0.3}
	\def\bc{0.8}

	\foreach \r in {1,...,4} { \draw (0,0) circle(\nj+\r*\bc); } %画同心圆
	\foreach \firstlayer/\secondlayer/\thirdlayer [count=\i] in {
		甲/阳土/太宫,乙/阴金/少商,丙/阳水/太羽,丁/阴木/少角,戊/阳火/太徵,
		己/阴土/少宫,庚/阳金/太商,辛/阴水/少羽,壬/阳木/太角,癸/阴火/少徵,
	} {
		\pgfmathsetmacro{\angle}{180-360/10*\i+360/10}
		\draw (\angle-90:\nj+\bc) -- (\angle-90:\nj+\bc*4);%画等分线
		\node[rotate=90+\angle] at (\angle:\nj+\bc*1.5-0.05) {\firstlayer};
		\node[rotate=90+\angle] at (\angle:\nj+\bc*2.5-0.05) {\secondlayer};
		\node[rotate=90+\angle] at (\angle:\nj+\bc*3.5-0.05) {\thirdlayer};
	}
	\end{tikzpicture}
	\caption{五音建运太少相生图}\label{fig:五音建运太少相生图}
\end{figure}

3.五步推运:主运虽始于木角,终于水羽,年年不变,但由于主运的各运受岁运年干阴阳属性的影响而其阴阳太少有别,因此要明确各年份各步主时之运的太少,还必须用五步推运之法来推求。五步推运的方法,是根据当年年干的阴阳太少,并在“五音建运太少相生图”找出其主时之运,然后按逆时针方向,上推至角,以确定初运木角的太少,继而依太少相生确定二、三、四、终运的太少。(表\ref{tab:主运五步推运太少相生表})

\begin{table}[htb]%主运五步推运太少相生表
	\centering
	\caption{主运五步推运太少相生表}\label{tab:主运五步推运太少相生表}
	\setlength{\fboxsep}{1pt}

	\begin{threeparttable}
		\begin{tabu}to.87\textwidth{X[c]|@{\hspace{2.5em}}*4{X[-2]@{}}X}
			\toprule
			年干 & 初运                & 二运                & 三运                & 四运              & 终运         \\
			\midrule
			甲   & \fbox{木}→太生少→ & 火→少生太→        & \fbox{土}→太生少→ & 金→少生太        & →\fbox{水}  \\ \hline
			乙   & \fbox{木}→太生少→ & 火→少生太→        & \fbox{土}→太生少→ & 金→少生太        & →\fbox{水}  \\ \hline
			丙   & \fbox{木}→太生少→ & 火→少生太→        & \fbox{土}→太生少→ & 金→少生太        & →\fbox{水}  \\ \hline
			丁   & 木→少生太→        & \fbox{火}→太生少→ & 土→少生太→        & \fbox{金}→太生少 & →水         \\ \hline
			戊   & 木→少生太→        & \fbox{火}→太生少→ & 土→少生太→        & \fbox{金}→太生少 & →水         \\ \hline
			己   & 木→少生太→        & \fbox{火}→太生少→ & 土→少生太→        & \fbox{金}→太生少 & →水         \\ \hline
			庚   & 木→少生太→        & \fbox{火}→太生少→ & 土→少生太→        & \fbox{金}→太生少 & →水         \\ \hline
			辛   & 木→少生太→        & \fbox{火}→太生少→ & 土→少生太→        & \fbox{金}→太生少 & →水         \\ \hline
			壬   & \fbox{木}→太生少→ & 火→少生太→        & \fbox{土}→太生少→ & 金→少生太        & →\fbox{水}  \\ \hline
			癸   & \fbox{木}→太生少→ & 火→少生太→        & \fbox{土}→太生少→ & 金→少生太        & →\fbox{水}  \\
			\bottomrule
		\end{tabu}
		%\hspace{2em}\footnotesize 注:有□的为太,没有□的为少。
		\begin{tablenotes}
			\footnotesize
			\item[] \hspace{4em}注:有□的为太,没有□的为少。
		\end{tablenotes}
	\end{threeparttable}
\end{table}

例如,甲子之年,甲为阳土,岁运为太宫用事。再以太宫为起点,沿逆时针方向上推至角。即生太宫的是太徵,生少徵的是太角,所以当年主运的初运为太角。太少相生,可推知二运为少徵,三运为太宫,四运为少商,终运为太羽。余此类推。

从上表可见,主运的太过不及,是五年一循环,十年一周期。各年主运的太过不及,与该年岁运的太过不及是一致的。如戊年岁运为火运太过,则该年主运之二运火运也是太过。再如辛年岁运为水运不及,则该年主运之终运水运亦为不及。由此可知,推算主运的简便方法是:看该年的岁运是什么运,是太过还是不及,则该年的主运与岁运是一致的,再用太少相生前后一推便得。如甲年岁运为土运太过,则该年的主运三运土运也是太过,前推二运火运为不及,初运木运为太过,后推四运金运为不及,终运水运为太过。

\subsubsection{客运}%(三)

客运乃与主运相对而言,但因其十年之内年年不同,如客之来去,故名客运。

客运亦是主时之运,它与主运共同主持着每年五步的每一步。每年的客运也分为木运、火运、土运、金运、水运五种。客运与主运的相同点是:五运分主五时,每运各主七十三日零五刻;均按五行相生之序,太少相生,五步推运。二者的不同点在于客运随着岁运而变,年年不同,而主运则始于春角,终于冬羽,年年不变。

客运的推算方法,可分两步进行。第一步是先立年干,确定岁运的阴阳太少。第二步是以当年的岁运为初运,依五行太少相生的顺序,分作五步,行于五运之上,逐年变迁,十年一周期。如甲年岁运为阳土太宫,那么当年客运的初运亦为太宫,五行太少相生,其二运为少商,三运为太羽,四运为少角,终运为太徵。余可依此类推。(图\ref{fig:五运客运图})。

%\begin{figure}[htb]
%	\centering
%	% Requires \usepackage{graphicx}
%	\includegraphics[width=0.50\textwidth]{五运客运图.png}\\
%	\caption{五运客运图}\label{fig:五运客运图}
%\end{figure}
\begin{figure}[htb]%五运客运图
	\centering
	\begin{tikzpicture}
	\def\nj{0.2}
	\def\bc{0.7}

	\foreach \r in {1,...,4,5.5} { \draw (0,0) circle(\nj+\r*\bc); } %画同心圆
	\foreach \firstlayer/\secondlayer [count=\i] in {
		木/丁壬,火/戊癸,土/甲己,金/乙庚,水/丙辛,
	}{
		\pgfmathsetmacro{\angle}{90-360/5*\i+360/5/2}
		\draw (\angle-360/5/2:\nj+\bc) -- (\angle-360/5/2:\nj+\bc*3);%画等分线
		\node[rotate=90+\angle] at (\angle:\nj+\bc*1.5-0.05) {\firstlayer};
		\node[rotate=90+\angle] at (\angle:\nj+\bc*2.5-0.05) {\secondlayer};
	}
	\foreach \thirdlayer [count=\i] in {
		少角,太角,太徵,少徵,太宫,少宫,少商,太商,太羽,少羽,
	}{
		\pgfmathsetmacro{\angle}{90-360/10*\i+360/10/2}
		\draw (\angle-360/10/2:\nj+\bc*3) -- (\angle-360/10/2:\nj+\bc*4);% 画等分线
		\node[rotate=90+\angle] at (\angle:\nj+\bc*3.5-0.05) {\thirdlayer};
	}
	\foreach \fourthlayer [count=\i] in {
		初角,二徵,三宫,四商,终羽,
		初徵,二宫,三商,四羽,终角,
		初宫,二商,三羽,四角,终徵,
		初商,二羽,三角,四徵,终宫,
		初羽,二角,三徵,四宫,终商,
	}{
		\pgfmathsetmacro{\angle}{90-360/25*\i+360/25/2}
		\draw (\angle-360/25/2:\nj+\bc*4) -- (\angle-360/25/2:\nj+\bc*5.5);% 画等分线
		\node[rotate=90+\angle,text width =1em] at (\angle:\nj+\bc*4.8-0.05) {\fourthlayer};
	}
	\end{tikzpicture}
	\caption{五运客运图}\label{fig:五运客运图}
\end{figure}

\subsection{六气}%三、

六气,指风、热(暑)、火、湿、燥、寒等六种气候变化。六气分为主气、客气、客主加临三种,主气测常,客气测变,客主加临则是一种常变结合的综合分析方法。六气的推求,是以十二地支配合三阴三阳来进行推演分析的。

气候变化的来源是六气,六气产生的标象是三阴三阳。标本相合,即为风化厥阴,热化少阴,湿化太阴,火化少阳,燥化阳明,寒化太阳。所以,《素问·天元纪大论》说:“厥阴之上,风气主之;少阴之上,热气主之;太阴之上,湿气主之;少阳之上,相火主之;阳明之上,燥气主之;太阳之上,寒水主之。所谓本也,是谓六元。”

\subsubsection{地支纪气}%(一)

地支纪气,即十二支化气。十二支配六气是结合三阴三阳来完成的,正如《素问·五运行大论》所说:“子午之上,少阴主之;丑未之上,太阴主之;寅申之上,少阳主之;卯酉之上,阳明主之;辰戌之上,太阳主之;巳亥之上,厥阴主之。”上,指在上的天气,即司天之气所在的位置。其意是说逢子午之年,为少阴君火之气所主;丑未之年,为太阴湿土之气所主;寅申之年,为少阳相火之气所主;卯酉之年,为阳明燥金之气所主;辰戌之年,为太阳寒水之气所主;巳亥之年,为厥阴风木之气所主。(表\ref{tab:十二支配六气表})

\begin{table}[htb]%十二支配六气表
	\centering
	\caption{十二支配六气表}\label{tab:十二支配六气表}
	\begin{tabu}to.87\textwidth{X[1.3,c]*{6}{|X[c]}}
		\toprule
		十二支   & 子午 & 丑未 & 寅申 & 卯酉 & 辰戌 & 巳亥 \\
		\midrule
		三阴三阳 & 少阴 & 太阴 & 少阳 & 阳明 & 太阳 & 厥阴 \\ \hline
		六气     & 君火 & 湿土 & 相火 & 燥金 & 寒水 & 风木 \\
		\bottomrule
	\end{tabu}
\end{table}

\subsubsection{主气}%(二)

主气,即主时之气,主治一年四季的正常气候变化。主气包括风木、君火、相火、湿土、燥金、寒水六种,因其年年如此,恒居不变,静而守位,所以又称为地气。

主气分主一年的二十四个节气,即将一年24个节气分属于六步之中,每步主四个节气,计60天87刻半,始于厥阴风木,按五行相生次序,终于太阳寒水,年年不变。一年四季始于春,从大寒至春分,为初之气,属厥阴风木所主;从春分至小满,为二之气,属少阴君火所主;从小满至大暑,为三之气,因君火相火同气相随,故属少阳相火所主;从大暑至秋分,为四之气,属太阴湿土所主;从秋分至小雪,为五之气,属阳明燥金所主;从小雪至大寒,为六之气,属太阳寒水所主。正如《素问·六微旨大论》所说:“愿闻地理之应六节气位何如?岐伯曰:“显明之右,君火之位也。君火之右,退行一步,相火治之;复行一步,土气治之;复行一步,金气治之;复行一步,水气治之;复行一步,木气治之;复行一步君火治之。”显明,谓日出,其位应正东,偏北卯位。自东向南移,故曰右行。(图\ref{fig:六气主时节气图})

%\begin{figure}[htb]
%	\centering
%	% Requires \usepackage{graphicx}
%	\includegraphics[width=0.50\textwidth]{六气主时节气图.png}\\
%	\caption{六气主时节气图}\label{fig:六气主时节气图}
%\end{figure}
\begin{figure}[htb]%六气主时节气图
	\centering
	\begin{tikzpicture}
	\def\nj{0.2}
	\def\bc{0.65}

	\foreach \r in {1,...,4,5.3,6.3,7.1} { \draw (0,0) circle(\nj+\r*\bc); } %画同心圆
	\foreach \firstlayer/\fifthlayer [count=\i] in {
		子/十一月,丑/十二月,寅/一月,卯/二月,辰/三月,巳/四月,
		午/五月,未/六月,申/七月,酉/八月,戊/九月,亥/十月,
	}{
		\pgfmathsetmacro{\angle}{-90-360/12*(\i-1)}
		\draw (\angle-360/12/2:\nj+\bc) -- (\angle-360/12/2:\nj+\bc*2);%第1 层等分线
		\draw (\angle-360/12/2:\nj+\bc*5.3) -- (\angle-360/12/2:\nj+\bc*6.3);%第5层等分线
		\node[rotate=90+\angle] at (\angle:\nj+\bc*1.5-0.05) {\firstlayer};
		%\node[rotate=90+\angle] at (\angle:\nj+\bc*5.5-0.05) {\fifthlayer};
		\draw[decorate, decoration={text align=center, text along path,
			text={\fifthlayer}}] (\angle-360/12:\nj+\bc*5.8+0.1) arc (\angle-360/12:\angle+360/12:\nj+\bc*5.8+0.1);
	}
	\foreach \secondlayer/\thirdlayer [count=\i] in {
		气之终/太阳寒水,气之初/厥阴风木,气之二/少阴君火,
		气之三/少阳相火,气之四/太阴湿土,气之五/阳明燥金,
	}{
		\pgfmathsetmacro{\angle}{-90-360/6*(\i-1)}
		\draw (\angle+90:\nj+\bc*2) -- (\angle+90:\nj+\bc*4);%画等分线
		%\node[rotate=90+\angle] at (\angle:\nj+\bc*2.5-0.05) {\secondlayer};
		%\node[rotate=90+\angle] at (\angle:\nj+\bc*3.5-0.05) {\thirdlayer};
		\draw[decorate, decoration={text align=center, text along path,
			text={\secondlayer}}] (\angle-360/6:\nj+\bc*2.5+0.1) arc (\angle-360/6:\angle+360/6:\nj+\bc*2.5+0.1);
		\draw[decorate, decoration={text align=center, text along path,
			text={\thirdlayer}}] (\angle-360/6:\nj+\bc*3.5+0.1) arc (\angle-360/6:\angle+360/6:\nj+\bc*3.5+0.1);
	}
	\foreach \fourthlayer [count=\i] in {
		冬至,小寒,大寒,立春,
		雨水,惊蛰,春分,清明,
		谷雨,立夏,小满,芒种,
		夏至,小暑,大暑,立秋,
		处暑,白露,秋分,寒露,
		霜降,立冬,小雪,大雪,
	}{
		\pgfmathsetmacro{\angle}{-90-360/24*(\i-1)}
		\node[rotate=90+\angle,text width =1em] at (\angle:\nj+\bc*4.6) {\fourthlayer};
	}
	\foreach \sixthlayer [count=\i] in {
		东,南,西,北,
	}{
		\pgfmathsetmacro{\angle}{-360/4*(\i+1)}
		\node[rotate=90+\angle] at (\angle:\nj+\bc*6.8-0.05) {\sixthlayer};
	}
	\end{tikzpicture}
	\caption{六气主时节气图}\label{fig:六气主时节气图}
\end{figure}

《素问·六微旨大论》说:“亢则害,承乃制,制则生化,外列盛衰,害则败乱,生化大病。”上述六气之间即具有这种承制关系,正是由于这种承制约束,才能维持自然气候的正常变化。亦如《素问·六微旨大论》所说:“相火之下,水气承之;水位之下,土气承之;土位之下,风气承之;风位之下,金气承之;金位之下,火气承之;君位之下,阴精承之。”下,指下承之气,因其位居于本气之后,故称“下”。承,指承接着而来的制约之气。

\subsubsection{客气}%(三)

客气,即在天的三阴三阳之气,因其客居不定,与主气之固定不变有别,所以称为“客气”。客气和主气一样,也分为风木、相火、君火、湿土、燥金、寒水六种。客气运行六步的次序是先三阴,后三阳,具体次序是:一厥阴风木,二少阴教火,三太阴湿土,四少阳相火,五阳明燥金,六太阳寒水。正如《素问·六微旨大论》所说:“上下有位,左右有纪,故少阳之右,阳明治之;阳明之右,太阳治之;太阳之右,厥阴治之;厥阴之右,少阴治之;少阴之右,太阴治之;太阴之右,少阳治之。”

虽然客气、主气均分六步运行,但客气随着每年年支的变化而变化,它包括司天之气、在泉之气、左右四间气六步。三阴三阳六步之气,按照一定次序,分布于上下左右,互为司天,互为在泉,互为间气,以六年为一周期,演变不息。要推算客气,首先必须确定司天、在泉及左右四间气。

1.司天之气:司天,即轮值主司天气之意。司天之气位于正南主气的三之气上,主司上半年的气候变化,也称“岁气”,故《素问·六元正纪大论》说:“岁半之前,天气主之。”天气,即指司天之气。(图\ref{fig:司天在泉左右间气位置图})
%\begin{figure}[htb]
%	\centering
%	% Requires \usepackage{graphicx}
%	\includegraphics[width=0.50\textwidth]{司天在泉左右间气位置图.png}\\
%	\caption{司天在泉左右间气位置图}\label{fig:司天在泉左右间气位置图}
%\end{figure}
\begin{figure}[htb]%司天在泉左右间气位置图
	\centering
	\begin{tikzpicture}
	\def\nj{0.4}
	\def\bc{0.8}

	\foreach \r in {1,...,4} { \draw (0,0) circle(\nj+\r*\bc); } %画同心圆
%	\foreach \i in {1,...,6} {
%		\pgfmathsetmacro{\angle}{90-360/6*\i}
%		\draw (\angle-360/6/2:\nj+\bc) -- (\angle-360/6/2:\nj+\bc*4);%画等分线
%	}
	\foreach \firstlayer [count=\i] in {
		辰,戊,巳,亥,子,午,丑,未,寅,申,卯,酉,
	}{
		\pgfmathsetmacro{\angle}{90-360/12*\i-360/12/2}
		\node[rotate=90+\angle] at (\angle:\nj+\bc*1.5-0.05) {\firstlayer};
	}
	\foreach \secondlayer/\thirdlayer [count=\i] in {
		太阳寒水/左间,厥阴风木/右间,少阴君火/在泉,太阴湿土/左间,少阴相火/右间,金燥明阳/司天,
	}{
		\pgfmathsetmacro{\angle}{90-360/6*\i}
		\draw (\angle-360/6/2:\nj+\bc) -- (\angle-360/6/2:\nj+\bc*4);%画等分线
		%\node[rotate=-90+\angle] at (\angle:\nj+\bc*2.5-0.05) {\secondlayer};
		\draw[decorate, decoration={text align=center, text along path,
			text={\secondlayer}}] (\angle-360/6:\nj+\bc*2.5+0.1) arc (\angle-360/6:\angle+360/6:\nj+\bc*2.5+0.1);
		\node[rotate=90+\angle] at (\angle:\nj+\bc*3.5-0.05) {\thirdlayer};
	}
	\end{tikzpicture}
	\caption{司天在泉左右间气位置图}\label{fig:司天在泉左右间气位置图}
\end{figure}

司天之气的轮值是以纪年的地支来推演的。《素问·天元纪大论》说:“帝曰:其于三阴三阳合之奈何?鬼臾区曰:子午之岁,上见少阴;丑未之岁,上见太阴;寅申之岁,上见少阳;卯酉之岁,上见阳明;辰戌之岁,上见太阳;巳亥之岁,上见厥阴。”由此可见,其规律是:凡子午之岁,为少阴君火司天;丑未之岁,则为太阴湿土司天;寅申之岁,为少阳相火司天;卯酉之岁,为阳明燥金司天;辰戌之岁,为太阳寒水司天;巳亥之岁,为厥阴风木司天。

2.在泉之气:与司天相对之气为“在泉”,在泉之气亦属岁气,主管下半年的气候变化。故《素问·六元正纪大论》说:“岁半之后,地气主之。”地气,指在泉之气。在泉之气位于正北主气的终之气上。(图\ref{fig:司天在泉左右间气位置图})

在泉之气与司天之气的位置及阴阳之气的多少均是相对应的,因此可以根据司天之气来确定在泉之气。具体是:凡一阴司天,则一阳在泉;二阴司天,则二阳在泉;三阴司天,则三阳在泉。反之亦然。也就是说,子午少阴君火与卯西阳明燥金,丑未太阴湿土与辰戌太阳寒水,寅申少阳相火与巳亥厥阴风木,均两两相对,互为司天在泉。

3.间气:客气除司天和在泉外,其余的四气统称“间气”。即间气指客气中的初之气、二之气、四之气和五之气。《素问·至真要大论》说:“帝曰:间气何谓?岐伯曰:司左右者,是谓间气也。帝曰:何以异之?岐伯曰:主气者纪岁,间气者纪步也。”说明司天、在泉的左右之气均为间气,间气主要用来纪客气六步的。

间气位于司天、在泉的左右,而有司天左间右间和在泉左间右间的不同。司天的左间,位于主气的四之气上,右间位于主气的二之气上;在泉的左间位于主气的初之气上,右间位于主气的五之气上。司天左右间气的排列关系,如同《素问·五运行大论》所说:“诸上见厥阴,左少阴,右太阳。见少阴,左太阴,右厥阴。见太阴,左少阳,右少阴。见少阳,左阳明,右太阴。见阳明,左太阳,右少阳。见太阳,左厥阴,右阳明。所谓面北而命其位,言其见也。”此左右,是指面向北方在泉时的位置。

在泉的左右间气,则与司天之气相反。亦如《素问·五运行大论》所说:“何谓下?岐伯曰:厥阴在上,则少阳在下,左阳明,右太阴。少阴在上,则阳明在下,左太阳,右少阳。太阴在上,则太阳在下,左厥阴,右阳明。少阳在上,则厥阴在下,左少阴,右太阳。阳明在上,则少阴在下,左太阴,右厥明。太阳在上,则太阴在下,左少阳,右少阴。所谓面南而命其位,言其见也。”此左右,是指面向南方司天时所见的位置。

六气的运转,是按纪年岁支的顺序进行的,六年一周期,每年均各有司值。司天之气,自上右转,而下降于地;在泉之气,自下左转,而上升于天,左右旋转一周,复归于本位。所以《素问·五运行大论》说:“动静何如?岐伯曰:上者右行,下者左行,左右周天,余而复会也。”(图\ref{fig:司天在泉左右间气图})

%\begin{figure}[htb]
%	\centering
%	% Requires \usepackage{graphicx}
%	\includegraphics[width=0.50\textwidth]{司天在泉左右间气图.png}\\
%	\caption{司天在泉左右间气图}\label{fig:司天在泉左右间气图}
%\end{figure}
\begin{figure}[htb]%司天在泉左右间气图
	\centering%
	\begin{tikzpicture}
	\def\nj{0}
	\def\bc{0.55}

%=======================
	\def\zuohua#1#2#3{
		\begin{scope}[shift={#1}]
		\node[text width =1em] at (0,0) {#2};
		\foreach \r in {1,...,3} { \draw (0,0) circle(\nj+\r*\bc); } %画同心圆
		\foreach \firstlayer/\secondlayer [count=\i] in #3 {
			\pgfmathsetmacro{\angle}{90-360/6*(\i-1)}
			\draw (\angle-360/6/2:\nj+\bc) -- (\angle-360/6/2:\nj+\bc*3);% 画等分线
			\node[rotate=90+\angle,text width =1em] at (\angle:\nj+\bc*1.5-0.05) {\firstlayer};
			\draw[decorate, decoration={text align=center, text along path,
				text={\secondlayer}}] (\angle-360/6:\nj+\bc*2.5+0.1) arc (\angle-360/6:\angle+360/6:\nj+\bc*2.5+0.1);
		}
		\end{scope}
	}
%=======================
	\zuohua{(0,0)}{六步}{{三/司天,四/左间,五/右间,终/在泉,初/左间,二/右间,}}
	\foreach \neiquan/\list [count=\i] in {
%			六步/{三/司天,四/左间,五/右间,终/在泉,初/左间,二/右间,},
			卯酉/{天/阳明,左/太阳,右/厥阴,泉/少阴,左/太阴,右/少阳,},
			辰戊/{天/太阳,左/厥阴,右/少阴,泉/太阴,左/少阳,右/阳明,},
			巳亥/{天/厥阴,左/少阴,右/太阴,泉/少阳,左/阳明,右/太阳,},
			子午/{天/少阴,左/太阴,右/少阳,泉/阳明,左/太阳,右/厥阴,},
			丑未/{天/太阴,左/少阳,右/阳明,泉/太阳,左/厥阴,右/少阴,},
			寅申/{天/少阳,左/阳明,右/太阳,泉/厥阴,左/少阴,右/太阴,},
		}{
			\pgfmathsetmacro{\angle}{90-360/6*(\i-1)}
			\zuohua{(\angle:\bc*6+.25)}{\neiquan}{\list}
		}
	\end{tikzpicture}
	\caption{司天在泉左右间气图}\label{fig:司天在泉左右间气图}
\end{figure}

客气除上述的正常变化外,还可能出现以下两种异常情况。

一是客气的胜复变化。即司天的上半年若有超常的胜气发生,则下半年可发生相反的复气以克制之。如上半年热气偏胜,下半年即有寒气克制。有胜有复为常,有胜无复则亢而为害。

二是客气的不迁正、不退位。所谓“不迁正”,就是应值的司天之气不及,不能按时主值;“不退位”,则为已应值的司天之气太过,应退去面留而不去。如太阴湿土司天之年,湿气太过不去,至使第二年仍出现湿气太过的气候特征,便是太阴湿土不退位。

在上述不迁正、不退位的情况下,左右间气也当升不升,应降不降,从而会导致整个客气规律失序,变乱丛生。

\subsubsection{客主加临}%(四)

客主加临,是将每年轮值的客气六步,分别加于固定不变的主气六步之上。由于主气只能概括一年气候的常规变化,而气候的具体变化则取决于客气,因此只有将客主二气结合起来分析,才能把握当年气候的实际变化情况。

客主加临的方法,是将司天的客气加于主气的三之气上,在泉之气加于主气的终之气上,其余的四气则分别以次加临。加临之后,主气六步不动,客气六步则每年按三阴、三阳次序,依次转移,6年一转,运动不息。(图\ref{fig:客主加临图})

%\begin{figure}[htb]
%	\centering
%	% Requires \usepackage{graphicx}
%	\includegraphics[width=0.50\textwidth]{客主加临图.png}\\
%	\caption{客主加临图}\label{fig:客主加临图}
%\end{figure}
\begin{figure}[htb]%客主加临图
	\centering
	\resizebox{.65\textwidth}{!}{%
		\begin{tikzpicture}
		\def\nj{0}
		\def\bc{0.6}

%		\draw (-\bc*2,-\bc*9.8) -- (0,-\bc*9.8);
%		\draw (-\bc*2,-\bc*10) -- (90-4:\bc*0.2);
%		\draw (-\bc*2,-\bc*10) -- (0,-\bc*10);
%		\draw (-\bc*2,-\bc*10.2) -- (0,-\bc*10.2);
%		\node at (\bc*2.5,-\bc*10) {线框内是可以转动的};
		\foreach \r in {1,3,4,4.3,5.3,6.3,7.5,8.5,9.5} { \draw (0,0) circle(\nj+\r*\bc); } %画同心圆
		\foreach \i in {1,2,...,90} {
			\pgfmathsetmacro{\angle}{360/90*\i}
			\draw (\angle+4:\nj+\bc*4) -- (\angle:\nj+\bc*4.3);
		}
		\foreach \firstlayer/\secondlayer/\thirdlayer/\fourthlayer/\fifthlayer [count=\i] in {
			少阴/君火/子午/终之气/太阳寒水,太阴/湿土/丑未/初之气/厥阴风木,
			少阳/相火/寅申/二之气/少阴君火,阳明/燥金/卯酉/三之气/少阳相火,
			太阳/癸水/辰戊/四之气/太阴湿土,厥阴/风木/巳亥/五之气/阳明燥金,
		}{
			\pgfmathsetmacro{\angle}{-90-360/6*(\i-1)}
			\draw (\angle+90:\nj+\bc*1) -- (\angle+90:\nj+\bc*6.3);%画等分线
			\node[rotate=90+\angle,text width =1em] at (\angle:\nj+\bc*1.6){\firstlayer};
			\node[rotate=90+\angle] at (\angle:\nj+\bc*2.5) {\secondlayer};
			\node[rotate=90+\angle] at (\angle:\nj+\bc*3.5-0.05) {\thirdlayer};
			\draw[decorate, decoration={text align=center, text along path,
				text={\fourthlayer}}] (\angle-360/6:\nj+\bc*4.8+0.1) arc (\angle-360/6:\angle+360/6:\nj+\bc*4.8+0.1);
			\draw[decorate, decoration={text align=center, text along path,
				text={\fifthlayer}}] (\angle-360/6:\nj+\bc*5.8+0.1) arc (\angle-360/6:\angle+360/6:\nj+\bc*5.8+0.1);
		}
		\foreach \sixthlayer [count=\i] in {
			冬至,小寒,大寒,立春,
			雨水,惊蛰,春分,清明,
			谷雨,立夏,小满,芒种,
			夏至,小暑,大暑,立秋,
			处暑,白露,秋分,寒露,
			霜降,立冬,小雪,大雪,
		}{
			\pgfmathsetmacro{\angle}{-90-360/24*(\i-1)}
			\node[rotate=90+\angle,text width =1em] at (\angle:\nj+\bc*7-0.05) {\sixthlayer};
		}
		\foreach \seventhlayer [count=\i] in {
			十一月,十二月,正月,二月,三月,四月,
			五月,六月,七月,八月,九月,十月,
		}{
			\pgfmathsetmacro{\angle}{-90-360/12*\i+360/12}
			\draw (\angle-360/12/2:\nj+\bc*7.5) -- (\angle-360/12/2:\nj+\bc*8.5);%画等分线
			\draw[decorate, decoration={text align=center, text along path,
				text={\seventhlayer}}] (\angle-360/12:\nj+\bc*8+0.1) arc (\angle-360/12:\angle+360/12:\nj+\bc*8+0.1);
		}
		\foreach \eighthlayer [count=\i] in {
			东,南,西,北,
		}{
			\pgfmathsetmacro{\angle}{-360/4*(\i+1)}
			\node[rotate=90+\angle] at (\angle:\nj+\bc*9-0.05) {\eighthlayer};
		}
		\end{tikzpicture}
	}
	\caption{客主加临图}\label{fig:客主加临图}
\end{figure}

上图所示是卯酉年阳明燥金司天的客主加临情况,只要将图中客气圈逐年向左转动一格,便可获得各该年的客主加临图。

客主加临对气候和疾病的影响,可以从以下两个方面进行分析:

一是主客之气的得失。凡主客之气相生或主客同气,为相得;主客之气相克,为相失或不相得。相得则气候正常,不易生病;相失则气候反常,易于生病。正如《素问·五运行大论》说:“气相得则和,不相得则病。”

二是主客之气的顺逆。凡客气胜(克)主气,或客气君火加临于主气相火,为顺;凡主气胜(克)客气,或客气相火加临于主气君火,为逆。正如《素问·至真要大论》说:“主胜逆,客胜从。”《素问·六微旨大论》亦说:“君位臣则顺,臣位君则逆。”顺则气候平和,人体少病;逆则气候反常,人体多病。

\subsection{运气同化}%四、

运气同化,指岁运与岁气同类而化合的关系。所谓同化,是运与气彼此性质相同而相遇时,往往会产生同一性质的变化及气象反映。如木同风化,暑同火热化,土同湿化,金同燥化,水同寒化之类。在六十年中,运与气有二十六年的同化关系。

由于岁运有太过不及,岁气有司天在泉的不同,所以运气同化还有同天化、同地化的区别,而表现为天符、岁会、同天符、同岁会和太乙天符等五种类型。

\subsubsection{天符}%(一)

天符,指岁运之气与司天之气五行属性相符合的同化关系。如《素问·六微旨大论》说:“帝曰:土运之岁,上见太阴;火运之岁,上见少阳、少阴;金运之岁,上见阳明;木运之岁,上见厥阴;水运之岁,上见太阳;奈何?岐伯曰:天之与会也。故《天元册》曰天符。”

土运之岁,上见太阴,即己丑、己未年,土湿同化。火运之岁,上见少阳、少阴,即戊寅、戊申、戊子、戊午年,火与暑热同化。金运之岁,上见阳明,即乙卯、乙酉年,金燥同化。木运之岁,上见厥阴,即丁巳、丁亥年,木风同化。水运之岁,上见太阳,即丙辰、丙戌年,水寒同化。上述同化,均为岁运的五行属性与客气司天地支的五行属性相同,故称“天符”,即《素问·天元纪大论》所说:“应天者为天符”之意。(图\ref{fig:天符太乙图})
%\begin{figure}[htb]
%	\centering
%	% Requires \usepackage{graphicx}
%	\includegraphics[width=0.4\textwidth]{天符太乙图.png}\\
%	\caption{天符太乙图}\label{fig:天符太乙图}
%\end{figure}
\begin{figure}[htb]%天符太乙图
	\centering
	\begin{tikzpicture}
	\def\nj{0.1}
	\def\bc{1.1}

	\node[text width =1em] at (0,0) {司天};
	\foreach \r in {1,...,3} { \draw (0,0) circle(\nj+\r*\bc); } %画同心圆
	\foreach \i in {1,...,6} {
		\pgfmathsetmacro{\angle}{90-360/6*\i}
		\draw (\angle-360/6/2:\nj+\bc) -- (\angle-360/6/2:\nj+\bc*3);%画等分线
	}
	\foreach \firstlayer [count=\i] in {
		少阳,阳明,太阳,厥阴,少阴,太阴,
	}{
		\pgfmathsetmacro{\angle}{90-360/6*\i}
		\node[rotate=90+\angle,text width =1em] at (\angle:\nj+\bc*1.5-0.05) {\firstlayer};
	}
	\foreach \secondlayer [count=\i] in {
		戊寅,戊申,乙卯,乙酉,丙辰,丙戊,丁巳,丁亥,戊子,戊午,己丑,己未,
	}{
		\pgfmathsetmacro{\angle}{90-360/12*\i-360/12/2}
		\node[rotate=90+\angle,text width =1em] at (\angle:\nj+\bc*2.5-0.05) {\secondlayer};
	}
	\node[rotate=-90-45,text width =1em] at (90+45:\nj+\bc*3.5-0.05) {太乙};
	\node[rotate=45,text width =1em] at (-45:\nj+\bc*3.5-0.05) {太乙};
	\end{tikzpicture}
	\caption{天符太乙图}\label{fig:天符太乙图}
\end{figure}

\subsubsection{岁会}%(二)

岁会,指岁运之气与岁支之气五行属性相同的同化关系。《素问·六微旨大论》说:“木运临卯,火运临午,土运临四季,金运临酉,木运临子,所谓岁会,气之平也。”临,即本运加临本气。丁卯年,丁岁木运,卯位于东方属木,故称“木运临卯”。戊午年,戊为火运,午位于南方属火,故称“火运临午”。甲辰、甲戊、己丑、己未四年,甲己为土运,辰戌丑未分别位于东南、西南、东北、西北方属土,故称“土运临四季”。乙酉年,乙为金运,酉位于西方属金,故称“金运临酉”。丙子年,丙为水运,子位于北方属水,故称“水运临子”。凡上述八年为岁会。(图\ref{fig:岁会图})
%\begin{figure}[htb]
%	\centering
%	% Requires \usepackage{graphicx}
%	\includegraphics[width=0.4\textwidth]{岁会图.png}\\
%	\caption{岁会图}\label{fig:岁会图}
%\end{figure}
\begin{figure}[htb]%岁会图
	\centering
	\begin{tikzpicture}
	\def\nj{0.1}
	\def\bc{1.1}

	\node[text width =1em] at (0,0) {土运};
	\foreach \r in {1,...,3} { \draw (0,0) circle(\nj+\r*\bc); } %画同心圆
	\foreach \firstlayer [count=\i] in {
		金运,水运,木运,火运,
	}{
		\pgfmathsetmacro{\angle}{90-360/4*\i}
		\draw (\angle-360/4/2:\nj+\bc) -- (\angle-360/4/2:\nj+\bc*2);% 画等分线
		\node[rotate=90+\angle] at (\angle:\nj+\bc*1.5-0.05) {\firstlayer};
	}
	\foreach \secondlayer [count=\i] in {
		己未,乙酉,甲戊,丙子,己丑,丁卯,申辰,戊午
	}{
		\pgfmathsetmacro{\angle}{90-360/8*\i}
		\node[rotate=90+\angle,text width =1em] at (\angle:\nj+\bc*2.5-0.05) {\secondlayer};
	}
	\end{tikzpicture}
	\caption{岁会图}\label{fig:岁会图}
\end{figure}

\subsubsection{同天符}%(三)

同天符,是指岁运太过之气与客气在泉之气相合而同化的关系。《素问·六元正纪大论》说:“太过而同地化者三……甲辰、甲戌太宫,下加太阴;壬寅、壬申太角,下加厥阴;庚子庚午太商,下加阳明,如是者三。”还说:“加者何谓?……太过而加同天符。”甲辰、甲戌,岁土太宫,太阴湿土在泉,土湿同化;庚子、庚午,岁金太商,阳明燥金在泉,金燥同化;壬申、壬寅,岁木太角,厥阴风木在泉,风木同化。上述六年,同为太过的岁运之气与客气在泉之气属性相合而同化的关系,故均为同天符。其中,(图\ref{fig:同天符同岁会图})甲辰、甲戌两年既属同天符,又为岁会。

%\begin{figure}[htb]
%	\centering
%	% Requires \usepackage{graphicx}
%	\includegraphics[width=0.50\textwidth]{同天符同岁会图.png}\\
%	\caption{同天符同岁会图}\label{fig:同天符同岁会图}
%\end{figure}
\begin{figure}[htb]%同天符同岁会图
	\centering%
	\begin{tikzpicture}
	\def\nj{0.2}
	\def\bc{1}

	\node[text width =1em] at (0,0) {在泉};
	\foreach \r in {1,...,4} { \draw (0,0) circle(\nj+\r*\bc); } %画同心圆
	\foreach \firstlayer/\thirdlayer [count=\i] in {
		少阳/会岁同,阳明/符天同,太阳/会岁同,厥阴/符天同,少阴/会岁同,太阴/符天同,
	} {
		\pgfmathsetmacro{\angle}{90-360/6*\i}
		\draw (\angle-360/6/2:\nj+\bc) -- (\angle-360/6/2:\nj+\bc*4);%画等分线
		\node[rotate=90+\angle,text width =1em] at (\angle:\nj+\bc*1.5-0.05) {\firstlayer};
		\draw[decorate, decoration={text align=center, text along path,
			text={\thirdlayer}}] (\angle-360/6:\nj+\bc*3.5+0.1) arc (\angle-360/6:\angle+360/6:\nj+\bc*3.5+0.1);
	}
	\foreach \secondlayer [count=\i] in {
		癸巳,癸亥,庚子,庚午,辛丑,辛未,壬寅,壬申,癸卯,癸酉,申辰,申戊,
	} {
		\pgfmathsetmacro{\angle}{90-360/12*\i-360/12/2}
		\node[rotate=90+\angle,text width =1em] at (\angle:\nj+\bc*2.5-0.05) {\secondlayer};
	}
	\end{tikzpicture}
	\caption{同天符同岁会图}\label{fig:同天符同岁会图}
\end{figure}

\subsubsection{同岁会}%(四)

同岁会,是指岁运不及之气与客气在泉之气相合而同化的关系。《素问·六元正纪大论》说:“不及而同地化者亦三。……癸巳、癸亥少徵,下加少阳。辛丑、辛未少羽,下如太阳。癸卯、癸酉少徵,下加少阴,如是者三。”还说:“不及而加,同岁会也。”可见,在六十年中,“同岁会”共有六年,其中癸巳、癸亥、癸卯、癸酉为阴干火运不及之年,而客气在泉之气分别是少阴君火和少阳相火,属不及之火与在泉之君火、相火相合而同化。辛丑、辛未为阴干水运不及之年,其客气在泉之气是太阳寒水,属不及的水运与在泉的寒水相合而同化。(图\ref{fig:同天符同岁会图})

\subsubsection{太乙天符}%(五)

太乙天符,是指既是天符,又是岁会的年份。《素问·六微旨大论》说:“天符岁会何如?岐伯曰:太乙天符之会也。”可见,太乙天符是指岁运之气与司天之气、岁支之气三气相合而主令,即《素问·天元纪大论》“三合为治”之谓。六十年中,戊午、己酉、己丑、己未四年属于太乙天符(见图\ref{fig:天符太乙图})。如戊午年,既是“火运之岁,上见少阴”的天符年,又是“火运临午”的岁会年,故为“太乙天符”。

运气同化之年,往往气象单一,表现为一气独胜,容易给生物和人体造成较大的危害。正如《素问·六微旨大论》所说:“天符为执法,岁会为行令,太乙天符为贵人。帝曰:邪之中也奈何?岐伯曰:中执法者,其病速而危;中行令考,其病徐而持;中贵人者,其病暴而死。”一般情况下,岁运、司天、在泉在一年中各行其令,一旦三者相合而同化,就可产生单一剧烈的气候变化,所谓“执法”、“行令”、“贵人”便是对其力量和作用的形容。

\section{节运气学说在医学中的应用}%第二

运气学说在医学中主要是用以推测每年的气候变化,预测疾病的发生和流行,指导对该病的预防和治疗等方面。

\subsection{推测每年的气候变化}%一、

每年的气候变化,均有其一定的规律性,具体可用主运主气的运行变化来推测。

\subsubsection{推测每年气候的一般变化}%(一)

每年四季的气候变化,首先可用五运中的主运来推测。主运之五步,分别主治一年五时的正常气候。正如《素问·天元纪大论》所说:“天有五行御五位,以生寒暑燥湿风。”主运中,初运木运的时间是从每年的大寒至春分,相当于每年的春季,气候由寒转温,为风气主令,所以春季的气候特点以风(温)为主。二运火运的时间是从每年的清明至芒种,相当于每年的夏季,气候由温转热,为火气主令,所以夏季的气候特点以暑热(火)为主。三运土运的时间是从每年的夏至至处暑,相当于每年的夏秋之交,气候多余热未尽而多雨湿,为湿气主令,所以长夏的气候特点为湿为主。四运金运的时间是从每年的白露至立冬,相当于每年的秋季,气候由热转凉而干燥,为燥金主令,所以秋季的气候特点以燥为主。五运水运的时间是从每年的立冬至大寒,相当于每年的冬季,气供由凉转寒,为寒气主令,所以冬季的气候特点以寒为主。

其次,从六气之主气亦可推测每年四季气候的变化。用主气推测出的气候变化情况与主运相似。《素问·至真要大论》说:“厥阴司天,其化以风;少阴司天,其化以热;太阴司天,其化以湿;少阳司天,其化以火;阳明司天,其化以燥;太阳司天,其化以寒。”主气分为六步,每步主司四个节气。厥阴风木为初之气,从大寒至春分,相当于每年的初春,为风气主令,其气候变化为多风。少阴君火为二之气,从春分至小满,相当于每年的春末夏初,为火气主令,其气候变化为多热。少阳相火为三之气,从小满至大暑,相当于每年的夏季,为暑气主令,其气候变化为多火。太阴湿土为四之气,从大暑至秋分,相当于每年的暮夏初秋,为湿气主令,其气候变化为多湿。阳明燥金为五之气,从秋分至小雪,相当于每年的秋冬之间,为燥气主令,其气候变化的特点为多燥。太阳寒水为终之气,从小雪至大寒,相当于每年的严冬,为寒气主令,其气候变化为多寒。

由此可见,每年气候的一般变化规律是春季多风,夏季多暑热,长夏多湿,秋季多燥,冬季多寒。

\subsubsection{推测每年气候的特殊变化}%(二)

每年气候除了表现为春风,夏热、长夏湿、秋燥、冬寒的一般变化规律外,还可因各年值年大运及客气而反映出某些特殊性的变化来。

值年大运即岁运,主管着各年的岁气,其运的属性及太过不及,均可直接影响当年的气候变化。岁运的五行属性对当年气候有着直接的影响。如甲己年,岁运是土运,当年的气候变化便以湿为特点,其四季的气候除了风暑湿燥寒的变化外,还可表现出湿的特殊变化。乙庚年,岁运为金,当年的气候除了风暑湿燥寒的变化外,还可表现出燥的特殊变化。

各年岁运的太过、不及和平气的变化对当年气候的影响各有不同。

岁运太过之年,当年气候多为本气流行,同时本气太过还会克制所不胜之运而表现出所不胜之运的气候特点,如甲辰年,为土运太过之年,当年的气候便以湿为特点;同时因土能胜水,因此除了湿气偏盛外,还要考虑寒气的影响。

岁运不及之年,当年气候多表现为本气不及而所不胜气流行的特征。如乙亥年,为金运不及之年,当年气候除了以燥为特点外,还会出现所不胜火气流行的变化,因此乙亥年的气候主要表现为燥气不及、火气偏胜的特殊变化。

此外,由于在某气太过或不及时,往往随之会有复气产生,以制约其太过的偏胜之气,因此在具体分析岁运太过、不及的气候特点时,还应同时考虑相应复气的影响。

平气之年,是因运太过而被抑,或运不及而得助所形成,其气候特点表现为本气特征而平和。如《素问·五常政大论》说:“愿闻平气何如而名,何如而纪也?岐伯对曰:昭乎哉问也!木曰敷和,火曰升明,土曰备化,金曰审平,水曰静顺。”敷和、升明、备化、审平、静顺,均为五运平气,反映气候平和、生化正常。如辛亥年,岁运水运不及,但因亥子属水,使其不及的水运得到资助,因而成为平气之年,气候特征表现为寒冷平和而无太过、无不及。

客气主治的节气年年不同,各年值年的司天在泉之气可直接影响其所主半年的气候变化。由于司天之气主管上半年,在泉之气主管下半年,所以其气候变化则是上半年反映司天之气的特点,下半年反映在泉之气的特点。如庚子年的年支是子,逢子午年为少阴君火司天,阳明燥金在泉,因此当年上半年属火气主事而气候较往年为热,下半年属燥气主事而气候较往年为燥。但是,由于司天之气还可影响在泉之气及四间气而主管全年气候,因此下半年的气候除燥胜之外,往往还表现为热的特点。

\subsection{预测疾病的发生及流行}%二、

疾病的发生和流行与四时气候的变化密切相关,因此在运用运气学说推测气候变化的同时,还可预测当年疾病发生与流行情况。

\subsubsection{主运、主气与疾病的发生及流行}%(一)

从主运和主气的变化,可推测各年疾病发生和流行的一般情况。

主运有五步之分,木为初运,主于春,风为春季主气,风气通于肝,故春多肝病。表现为一是易患肝病,二是已患肝病者,易于复发或加重,如肝肾阴虚者,此时春阳升发,肝阳易亢,甚至化火生风,可致头痛、眩晕、中风等病证。从疾病流行来看,本季多风,以风邪致病为多见,外感急性热病之风温也常发生于本季。火为二运,主于夏,火为夏季主气,火气通于心,故夏多心病。表现为一是易患心病,二是已患心病者,易于复发或加重,如心火素旺之人,此时心火易亢,常可导致心烦、口舌生疮、小便短赤,甚至迫血妄行而发生吐血、衄血等病证。从疾病流行来看,本季多火,以火邪致病为多见,外感温热病中之疫病也常发生于本季。土为三运,主于长夏,温为长夏主气,湿气通于脾,故长夏多脾病。表现为一是易患脾病,二是已患脾病者,易于复发或加重,如素有脾虚失运之人,此时易被湿困,而引起头重、身困、大便溏泄、脘腹胀满等病证。从疾病流行来看,本季多湿,以湿邪致病为多见,外感温热病中的湿温也常发生于本季。燥为四运,主于秋,燥为秋季之气,燥气通于肺,故秋多肺病。表现为一是易患肺病;二是已患肺病者,易于复发或加重,如肺阴素亏之人,此时易被燥伤,而引起咽干鼻燥、咳血、便秘等病证。从疾病流行来看,本季多燥,以燥邪致病为多见,外感病中的秋燥也常发生于本季。水为五运,主于冬,寒为冬季主气,寒气通于肾,故冬多肾病。表现为一是易患肾病;二是已患肾病者,易于复发或加重,如肾阳素亏之人,此时易感寒邪,而引起恶寒、发热、头痛、咳喘等病证。从疾病流行来看,本季多寒,以寒邪致病为多见,外感热病中的风寒重证也常发生于本季。

六气变化与发病的关系,与五运基本相同。六气之主气分为六步,初之气为厥阴风木,其所主之节内多易患或流行肝病、风湿病、风病。二之气为少阴君火,其所主之节内多易患或流行心病、火热病。三之气为少阳相火,其所主之节内多易患或流行心病、暑热病。四之气为太阴湿土,多易患或流行脾胃病、湿病。五之气为阳明燥金,多易患或流行肺病、燥病。终之气为太阳寒水,多易患或流行肾病、寒病。

由此可见,主运、主气对疾病发生及流行的影响基本相同,均反映了各年气候变化的一般规律,相对易于把握。而每年发病的特殊规律则可用大运及司天在泉之气的变化来进行推测。

\subsubsection{大运、客气与疾病的发生及流行}%(二)

各年大运的五行属性不同,它对当年全年的疾病均发生着相应的影响,即当年全年的疾病均可表现出相应的五行属性特点。如甲己化土,逢甲逢己之年,其大运为土运,疾病以湿病、脾病为多见,且一年四季均可在所患病变的基础上,反映出湿的特点或与脾病相关。丙辛化水,逢丙逢辛之年,其大运为水运,疾病以寒病、肾病为多见,且一年四季均可在所患病变的基础上,反映出寒的特点或与肾病相关。

除了值年大运五行属性本身的影响外,在具体分析时还要考虑各年大运的太过、不及等因素。岁运大过之年,除了本气本脏之病外,还可病及所胜之气和脏;岁运不及之年,除了本气本脏之病外,还可病及所不胜之气和脏。如戊子年是火运太过之年,多患热病、心病,同时火胜克金,还易患肺病、燥病。辛丑年是水运不及之年,多患寒病、肾病,同时水不胜土,还易患脾病、湿病。

此外,无论是岁运太过或不及之年,均应考虑胜气、复气等对疾病的影响。

客气中其司天在泉之气对当年疾病的发生及流行亦有着相应的影响。司天之气主司上半年,故上半年的疾病多受司天之气的影响;在泉之气主司下半年,故下半年的疾病多受在泉之气的影响。如癸未年为太阴湿土司天,太阳寒水在泉,所以上半年为湿气主事,表现为湿病、脾病较多;下半年寒气主事,表现为寒病、肾病较多。如《素问·至真要大论》说:“太阴司天,湿淫所胜,则沉阴且布,雨变枯槁,胕肿,骨痛,阴痹。阴痹者,按之不得,腰脊头顶痛,时眩,大便难,阴气不用,饥不欲食,咳唾则有血,心如悬,病本于肾。”又说:“岁太阳在泉,寒淫所胜,则凝肃惨慄。民病少腹控睾,引腰脊,上冲心痛,血见,嗌痛,颔肿。”

但是,由于司天之气可通过影响在泉之气和左右间气而主管全年,因而对于下半年的疾病还必须同时考虑其司天之气的影响。此外,司天与在泉之气的五行生克关系也是影响疾病变化的因素之一,预测分析时也当一并考虑。

以上是运用运气学说推测疾病发生与流行的一般情况,在临床辨析中还必须因时因地因人制宜,灵活运用,综合分析,才能恰到好处。

\subsection{指导对疾病的预防和治疗}%三、

预防疾病的发生,是《内经》“治未病”的重要内容之一。要想防患于未然,就必须借助运气学说预测疾病发生和流行的季节、时间,从而事先采取有针对性的预防措施。如《素问·气交变大论》说:“岁木太过,风气流行,脾土受邪。民病飧泄食减,体重烦冤,肠鸣腹支满。……甚则忽忽善怒,眩冒巅疾。……反胁痛而吐甚。”可见,岁木太过之年,疾病发生和流行的规律是,风气流行,肝木太过,脾土受邪。肝木之气太过,则可见善怒、眩冒巅疾、两胁疼痛等病证;木胜克土,脾胃受病,则可见飧泄、食减、肠鸣腹满、呕吐等病证。预防上,逢岁木太过之年,宜采用抑肝扶脾之法来进行预防。如疏肝郁,畅七情,以防止肝气太过;和饮食,安肠胃,以补脾气之虚,兼防脾气受制。总之,根据当年气候变化及疾病流行的预测情况,采取各种相应的防范措施,以减少或控制疾病的发生和流行。

在疾病治疗方面,运气学说将季节气候变化与治疗用药有机联系起来,使得治疗用药有规律可循。如《素问·六元正纪大论》说:太阳司天之年,“岁宜苦以燥之温之”;阳明司天之年,“岁宜咸以苦以辛,汗之清之散之”;少阳司天之年,“岁宜咸辛宜酸,渗之泄之,渍之发之”;太阴司天之年,“岁宜以苦燥之温之,甚者发之泄之”;少阴司天之年,“岁宜咸以软之,而调其上,甚则以苦发之,以酸收之,而安其下,甚则以苦泄之”;厥阴司天之年,“岁宜以辛调上,以咸调下”。说明由于司天之气的不同,当年的疾病具有相应的特点,因而其治疗用药亦有相应的区别。

《素问·至真要大论》还对客气司天在泉及六气胜复所致病证的治疗用药规律进行了探讨,指出:“司天之气,风淫所胜,平以辛凉,佐以苦甘,以甘缓之,以酸泻之。……太阳之胜,治以甘热,佐以辛酸,以咸泻之。”“诸气在泉,风淫于内,治以辛凉,佐以苦,以甘缓之,以辛散之。……寒淫于内,治以甘热,佐以苦辛,以咸泻之,以辛润之,以苦坚之。”“厥阴之复,治以酸寒,佐以甘辛,以酸泄之,以甘缓之。……太阳之复,治以咸热,佐以甘辛,以苦坚之。”

由此可见,运气学说根据五运六气的变化,总结出了一整套治疗用病规律,对于临床辨证论治具有较大指导意义。

综上所述,运气学说是我国的一门古老科学,它是在阴阳五行学说指导下,通过对天人关系长期的生活观察及临床实践而形成的一种时间气象医学理论。尽管它与今天的气象学相比还相当朴素,然而其独特的宏观研究方法及几千年的反复验证,充分证明了其自身的科学价值。因此运气学说对生活、生产及医疗方面的指导作用是不容忽视的。

在学习继承运气学说的同时,也还必须看到,限于历史条件、科学水平的限制,运气学说对自然界复杂的气候变化及其与疾病相关性的推断也并非尽善尽美和完全符合客观实际,因而要求在运用时要灵活掌握,因时识宜,做到顺天以察运,因变以求气,特别在临床上切不可胶柱鼓瑟,刻板应用。

\zuozhe{(邱辛凡)}
\ifx \allfiles \undefined
\end{document}
\fi